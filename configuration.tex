% This is part of Mes notes de mathématique
% Copyright (c) 2011-2016
%   Laurent Claessens
% See the file fdl-1.3.txt for copying conditions.

% This file contains the ``configuration'' of some packages.

% No color for the links in the book because the aim is to
% be printed.
\notbool{isBook}
{
    \hypersetup{
    colorlinks=true,
    citecolor=blue,
    linkcolor=violet,
    urlcolor=blue,     % couleur des url
    filecolor=Violet   % couleur des textes qui sont des liens
    }
}

% For the code, I don't know. Removeing these lines
% will create spolynôme much complications in the
% lstset ... I wait someone to complain.

    \definecolor{dkgreen}{rgb}{0,0.4,0}
    \definecolor{gray}{rgb}{0.5,0.5,0.5}
    \definecolor{mauve}{rgb}{0.58,0,0.82}

% This 'lstset' is from Lilian Besson 
\lstset{ %
  inputencoding=utf8/latin1,
  backgroundcolor=\color{white},  % choose the background color; you must add \usepackage{color} or \usepackage{xcolor}
  basicstyle=\ttfamily, % \texttt\small,              % the size of the fonts that are used for the code, FIXME \ttfamily
  breakatwhitespace=false,        % sets if automatic breaks should only happen at whitespace
  breaklines=true,                % sets automatic line breaking
  captionpos=b,                   % sets the caption-position to bottom
  commentstyle=\small\color{dkgreen},   % comment style
%  deletekeywords={...},          % if you want to delete keywords from the given language
%  escapeinside={\%*}{*)},        % if you want to add LaTeX within your code
  frame=single,                   % adds a frame around the code
  keywordstyle=\small\color{blue},      % keyword style
  language=python,                % the language of the code
  fontadjust=false,
  % if you want to add more keywords to the set
%  morekeywords={define,domain,objects,init,goal,problem,action,parameters,precondition,effect,types,requirements,strips,typing},
  numbers=left,                   % where to put the line-numbers; possible values are (none, left, right)
  numbersep=5pt,                  % how far the line-numbers are from the code
  numberstyle=\tiny\color{gray},  % the style that is used for the line-numbers
  rulecolor=\color{black},        % if not set, the frame-color may be changed on line-breaks within not-black text (e.g. comments (green here))
  showspaces=false,               % show spaces everywhere adding particular underscores; it overrides 'showstringspaces'
  showstringspaces=false,         % underline spaces within strings only
  showtabs=false,                 % show tabs within strings adding particular underscores
  stepnumber=1,                   % the step between two line-numbers. If it's 1, each line will be numbered
  stringstyle=\small\color{mauve},      % string literal style
  tabsize=2,                      % sets default tabsize to 2 spaces
  prebreak = \raisebox{0ex}[0ex][0ex]{\ensuremath{\hookleftarrow}}, % pour la fin des lignes.
  aboveskip={1.5\baselineskip},
  title=\lstname                  % show the filename of files included with \lstinputlisting; also try caption instead of title
%  title=\tiny{File \textcolor{blue}{\url{\lstname}}}          % show the filename of files included with \lstinputlisting; also try caption instead of title
  %% FIXME title !
}


% Directories

% g@addto@macro\input@path gives the path in which LaTeX has to
%   search for its \input

% \exoDirectory provides the directory in which \Exo will search for the files
%    exo*.tex and corr*.tex

% The paths for 'phystricks' are given in 'src_pictures/Directories.py'


% We do not use 'exocorr' with Frido
\ifbool{isFrido}{}{
    \renewcommand{\exoDirectory}{tex/exocorr/}
    \corrPosition{1}
}


% For each part of the document, we begin with
% \emptyInputPath
% \addInputPath{ path-to-the-files-for-the-part  }
% In such a way, we are sure to cause an error if files are badly sorted.

% Be careful : the macro name 'emptyInputPath' is hard-coded in 'pytex'

\makeatletter
\providecommand{\input@path}{}
\newcommand{\addInputPath}[1]{ \g@addto@macro\input@path{ {#1/} } }
\newcommand{\emptyInputPath}{  \renewcommand{\input@path}{}  }
\makeatother
