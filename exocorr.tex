% This is a part of 'giuilietta'.
%
% Copyright (c) 2010-2012, 2019 
% Laurent Claessens <lauren@claessens-donadello.eu>.

\makeatletter

\newboolean{LangageFR}
\setboolean{LangageFR}{false}
\DeclareOption{fr}{\setboolean{LangageFR}{true}}
\ProcessOptions

% ENGLISH DEFAULT TEXTS
\newcommand{\@ExoTheoremName}{Exercise}
\newcommand{\@CorrText}{Correction of the exercise}
\newcommand{\@AlternativeText}{Other resolution}
\newcommand{\@LinkToCorrText}{Correction at page}

% FRENCH TRANSLATIONS
\ifthenelse{\boolean{LangageFR}}{%
\renewcommand{\@ExoTheoremName}{Exercice}
\renewcommand{\@CorrText}{Correction de l'exercice}
\renewcommand{\@AlternativeText}{Résolution alternative}
\renewcommand{\@LinkToCorrText}{Correction à la page}
}{}
 
\newcounter{CountExercice}
\renewcommand{\theCountExercice}{\arabic{CountExercice}}

\newcounter{@corrDraft}
\setcounter{@corrDraft}{0}
\newcommand{\corrDraft}{\setcounter{@corrDraft}{1}}
% When @corrDraft is 1, a tag is added in the document in order to indicate the file name


\newenvironment{exercice}[1][]{%
\refstepcounter{CountExercice}%
{\bf \@ExoTheoremName\ \theCountExercice} #1 \noindent\nopagebreak%
}{}

%\newenvironment{corrige}[1]{\par\bigskip\noindent {\bf \@CorrText\ \ref{exo#1}}\label{corr#1}\par\nopagebreak}{}
\newenvironment{corrige}[1]{\label{corr#1}\par\bigskip\noindent {\bf \@CorrText\ \ref{exo#1}}\par\nopagebreak \ifthenelse{\value{@corrDraft}=1}{--#1--}{} }{}
\newenvironment{alternative}{\par\bigskip\noindent {\bf \@AlternativeText\ }\par\nopagebreak}{}

\newcounter{@CorrPosition}
\newcommand{\corrPosition}[1]{\setcounter{@CorrPosition}{#1}}
% 0 : No corrections 
% 1 : Correction immediately under the questions
% 2 : Corrections in a specific chapter, use \corrChapitre


% \CorrectionRepertory is the name of the directory in which the correction file has to be searched. 
% By default it is the current repertory.
\newcommand{\CorrectionRepertory}{}



% The multiple-file system for corrections comes from the answer by mpg
% in the thread «\closeout d'un fichier dont je ne connais que le nom ou savoir le nom d'un \newwrite» 
% on the french speaking usent LaTeX forum.


% The function \stream@name produce the name of the stream in which we can write from
% the name of the file.
% For the file foo.tex, the stream will be named f@streal@foo
% Thus one write in foo.tex by \write\f@stream@foo
\newcommand \stream@name [1] {%
%f@stream@\detokenize{#1}% 
  f@stream@{#1}% 
}

\newcommand \NewCorrectionFile [1] {% #1 = filename
  \expandafter \New@CorrectionFile \csname\stream@name{#1}\endcsname {#1}% (2)
}

\newcommand \New@CorrectionFile [2] {% #1 = command, #2 = filename
  \newwrite #1%
  \immediate\openout #1 #2\relax
}

\newcommand \WriteInFile [2] {% #1 = nom de fichier, #2 = contenu
  \immediate\write \csname\stream@name{#1}\endcsname {#2}% (3)
}
\newcommand \CloseFile [1] {% #1 = nom de fichier
  \immediate\closeout \csname\stream@name{#1}\endcsname % (3)
}

% \@InterCorrFile is the name of the file that will contain the \input of the corrections files.
\newcommand{\@DefaultCorrFileName}{corr}

\NewCorrectionFile{\@DefaultCorrFileName}
\newcommand{\InterCorrFile}{\@DefaultCorrFileName}

\newcommand{\UseCorrectionFile}[1]{%
    \renewcommand{\InterCorrFile}{#1}
    \NewCorrectionFile{\InterCorrFile}
}

\newcommand{\exoDirectory}{./}

% The optional first argument says if you want to include the correction.
% \Exo{bla}  will include the correction (if exists)
% \Exo[0]{bla} will not include the correction, even if it exists

% The files are searched in the directory given by '\exoDirectory'. 
% Default : the current directory.
% In order to change : \renewcommand{\exoDiretory}{mydir/}    Do not forget the slash.


\newcommand{\Exo}[2][1]{%
\input{\exoDirectory exo#2}
\ifthenelse{\value{@corrDraft}=1}{--#2--}{}%
\ifthenelse{#1=1}{
\ifthenelse{\value{@CorrPosition}=1}{\IfFileExists{\exoDirectory corr#2.tex}{\input{\exoDirectory corr#2}}{}}{}
%\ifthenelse{\value{@CorrPosition}=2}{\IfFileExists{corr#2.tex}{\WriteInFile{\InterCorrFile}{\string\import{\CorrectionRepertory}{corr#2}}}{}}{}
\ifthenelse{\value{@CorrPosition}=2}{\IfFileExists{\exoDirectory corr#2.tex}{\WriteInFile{\InterCorrFile}{\string\input{\exoDirectory corr#2}}}{}}{}
}
{}
}

\newcommand{\corrChapitre}[2][\InterCorrFile]{%
\ifthenelse{\value{@CorrPosition}=2}{%
\chapter{#2}
\CloseFile{#1}
    \input{#1}
}{}		% Fin du ifthenelse
}		% Fin de la macro \corrChapitre

\newcommand{\corrref}[1]{%
\ifthenelse{\value{@CorrPosition}=2}{\par \@LinkToCorrText\ \pageref{corr#1}.}{}%
}

% Ces histoires de \newwrite pour écrire dans un fichier sont par exemple expliquées sur le fctt [1]
% Le \immediate vient de la discussion fctt [2]


%%%%%%%%%%
% References to Wikipedia
% I suppressed the command \wikipedia and \wikiversity on May 2, 2018.
% They are copied directly in mazhe.tex



\makeatother

% [1] http://groups.google.fr/group/fr.comp.text.tex/browse_thread/thread/76eb44e2e928ba8b/ab0f092c79a062cd?hl=fr&lnk=gst&q=newwrite#ab0f092c79a062cd 
% [2] http://groups.google.fr/group/fr.comp.text.tex/browse_thread/thread/80909a66028ef5a3/ab89634948317f36?hl=fr#ab89634948317f36
% [3] http://groups.google.fr/group/fr.comp.text.tex/browse_thread/thread/b2a0fa029bf9e0d9/fd3b47dfaac9cc96?hl=fr&lnk=gst&q=shorthand#fd3b47dfaac9cc96
% [4] http://groups.google.fr/group/fr.comp.text.tex/browse_thread/thread/68bf87c02c4515bc?hl=fr#
