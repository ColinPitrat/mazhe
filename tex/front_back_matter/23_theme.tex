\InternalLinks{mesure et intégration}       \label{THEMEooKLVRooEqecQk}
\begin{description}
    \item[Mesure] 
    À propos de mesure.
\begin{enumerate}
    \item
        Mesure positive, mesure finie et \( \sigma\)-finie, c'est la définition \ref{DefBTsgznn}.

    \item Le produit de tribus est donné par la définition~\ref{DefTribProfGfYTuR},     % Cette référence doit être vers le haut.
    \item
        Produit d'une mesure par une fonction, définition \ref{PropooVXPMooGSkyBo}.
    \item le produit d'espaces mesurés est donné par la définition~\ref{DefUMlBCAO}.     % Cette référence doit être vers le haut.
        \item
            Mesure de Lebesgue sur \( \eR\), définition~\ref{DefooYZSQooSOcyYN}.
        \item
            Une partie de \( \eR\) non mesurable au sens de Lebesgue : l'exemple \ref{EXooCZCFooRPgKjj}.
        \item
            Mesure de Lebesgue sur \( \eR^N\), définition~\ref{DEFooSWJNooCSFeTF}.
        \item
            Mesure à densité, définition~\ref{PropooVXPMooGSkyBo}.
\end{enumerate}
\item[Théorèmes d'approximation]
    Il est important de pouvoir approcher des fonctions continues ou \( L^p\) par des fonctions étagées, sinon on ne parvient pas à faire tourner la machine de l'intégration de Lebesgue.
    \begin{enumerate}
        \item
            Si \( f\colon S\to \mathopen[ 0 , +\infty \mathclose]\) est une fonction mesurable, le théorème fondamental d'approximation~\ref{THOooXHIVooKUddLi} donne une suite croissante de fonctions étagées qui converge vers \( f\).
        \item
            Les fonctions simples sont denses dans \( L^p\), proposition \ref{PROPooUQUBooAWgNhm}.
        \item
            Encadrament d'un borélien \( A\) par un fermé \( F\) et un ouvert \( V\) par le lemme \ref{LEMooCGKXooYWjRwk} : \( F\subset A\subset V\) avec \( \mu(V\setminus F)<\epsilon\).
        \item
            Approximation \( L^p\) de la fonction caractéristique d'un borélien par une fonction continue par le théorème \ref{ThoAFXXcVa}.
    \end{enumerate}
\item[Intégration]
    À propos d'intégration.
    \begin{enumerate}
        \item
            Intégrale associée à une mesure, définition~\ref{DefTVOooleEst}
    \item
        L'existence d'une primitive pour toute fonction continue est le théorème~\ref{ThoEXXyooCLwgQg}.
    \item
        La définition d'une primitive est la définition~\ref{DefXVMVooWhsfuI}.
    \item
        Primitive et intégrale, proposition~\ref{PropEZFRsMj}.
    \item
        Intégrale impropre, définition~\ref{DEFooINPOooWWObEz}.
\end{enumerate}
            
\end{description}

