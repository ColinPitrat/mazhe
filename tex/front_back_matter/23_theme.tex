
\InternalLinks{mesure et intégration}
\begin{enumerate}
    \item Le produit de tribus est donné par la définition~\ref{DefTribProfGfYTuR},     % Cette référence doit être vers le haut.
    \item le produit d'espaces mesurés est donné par la définition~\ref{DefUMlBCAO}.     % Cette référence doit être vers le haut.
        \item
            Mesure de Lebesgue sur \( \eR\), définition~\ref{DefooYZSQooSOcyYN}.
        \item
            Mesure de Lebesgue sur \( \eR^N\), définition~\ref{DEFooSWJNooCSFeTF}.
        \item
            Intégrale associée à une mesure, définition~\ref{DefTVOooleEst}
        \item
            Si \( f\colon S\to \mathopen[ 0 , +\infty \mathclose]\) est une fonction mesurable, le théorème fondamental d'approximation~\ref{THOooXHIVooKUddLi} donne une suite croissante de fonctions étagées qui converge vers \( f\).
        \item
            Mesure à densité, définition~\ref{PropooVXPMooGSkyBo}.
    \item
        L'existence d'une primitive pour toute fonction continue est le théorème~\ref{ThoEXXyooCLwgQg}.
    \item
        La définition d'une primitive est la définition~\ref{DefXVMVooWhsfuI}.
    \item
        Primitive et intégrale, proposition~\ref{PropEZFRsMj}.
    \item
        Intégrale impropre, définition~\ref{DEFooINPOooWWObEz}.
\end{enumerate}
