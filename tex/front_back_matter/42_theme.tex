\InternalLinks{densité}         \label{THEooPUIIooLDPUuq}
\begin{enumerate}
    \item
        Densité des polynômes dans \( C^0\big( \mathopen[ 0 , 1 \mathclose] \big)\), théorème de Bernstein~\ref{ThoDJIvrty}.
    \item
        Densité de \( \swD(\eR^d)\) dans \( L^p(\eR^d)\) pour \( 1\leq p<\infty\), théorème~\ref{ThoILGYXhX}.
    \item
        Densité de \( \swS(\eR^d)\) dans l'espace de Sobolev \( H^s(\eR^d)\), proposition~\ref{PROPooMKAFooKDNTbO}.

    \item
        Densité de \( \swD(\eR^d)\) dans l'espace de Sobolev \( H^s(\eR^d)\), proposition~\ref{PROPooLIQJooKpWtnV}.

        Cela est utilisé pour le théorème de trace~\ref{THOooXEJZooBKtXBW}.
    \item
        Les applications étagées dans les applications mesurables (qui plus est avec limite croissante), théorème fondamental d'approximation~\ref{LempTBaUw}.
    \item
        Les fonctions continues à support compact dans \( L^2(I)\), théorème~\ref{ThoJsBKir}.
    \item
        Les polynômes trigonométriques sont denses dans \( L^p(S^1)\) pour \( 1\leq p<\infty\). Deux démonstrations indépendantes par le théorème~\ref{ThoDPTwimI} et le théorème~\ref{ThoQGPSSJq}.
\end{enumerate}
Les densités sont bien entendu utilisées pour prouver des formules sur un espace en sachant qu'elles sont vraies sur une partie dense. Mais également pour étendre une application définie seulement sur une partie dense. C'est par exemple ce qui est fait pour définir la trace \( \gamma_0\) sur les espaces de Sobolev \( H^s(\eR^d)\) en utilisant le théorème d'extension~\ref{PropTTiRgAq}.

Comme presque tous les théorèmes importants, le théorème de Stone-Weierstrass possède de nombreuses formulations à divers degrés de généralité.
\begin{itemize}
    \item Le lemme~\ref{LemYdYLXb} le donne pour la racine carré.
    \item Le théorème~\ref{ThoGddfas} donne la densité des polynômes dans les fonctions continues sur un compact.
    \item Le théorème~\ref{ThoWmAzSMF} est une généralisation qui donne la densité uniforme d'une sous-algèbre de \( C(X,\eR)\) dès que \( X\) sépare les points.
    \item Le lemme~\ref{LemXGYaRlC} est une version pour les polynômes trigonométriques.
    \item
        Le lemme~\ref{LemYdYLXb} est une cas particulier du
        théorème~\ref{ThoGddfas}, mais nous en donnons une démonstration indépendante afin d'isoler la preuve
de la généralisation~\ref{ThoWmAzSMF}.
Une version pour les polynômes trigonométriques sera donnée dans le lemme~\ref{LemXGYaRlC}.
\end{itemize}
