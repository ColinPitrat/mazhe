\InternalLinks{convexité}

L'essentiel des résultats sur les fonctions convexes sont dans la section~\ref{SECooVZWWooUjxXYi}. On a surtout :
\begin{enumerate}
    \item
        Définition des fonctions convexes :~\ref{DefVQXRJQz} et~\ref{DEFooKCFPooLwKAsS} en dimension supérieure.
    \item
        En termes de différentielles,~\ref{PROPooYNNHooSHLvHp} pour la différentielle première et~\ref{CORooMBQMooWBAIIH} pour la hessienne.
    \item
        Une courbe paramétrée convexe est la définition~\ref{DEFooVQODooJSNYLw}.
    \item
        L'enveloppe convexe d'une courbe fermée simple et convexe :~\ref{PROPooWZITooTFiWsi}.
    \item
        Courbure et convexité d'une courbe paramétrée : section~\ref{SUBSECooNJOLooYuGRjA}.
    \item
        Une courbe paramétrée convexe est localement le graphe d'une fonction convexe, lemme~\ref{LEMooGEVEooHxPTMO}.
    \item
        La convexité est utilisée dans la méthode du gradient à pas optimal de la proposition~\ref{PropSOOooGoMOxG}.
\end{enumerate}

En terme de parties convexes, on : 
\begin{enumerate}
    \item
        Définition \ref{DEFooQQEOooAFKbcQ} d'une partie convexe d'un espace vectoriel.
    \item
        Une boule est convexe, exemple \ref{EXooKZAOooVZtAhX}.
\end{enumerate}

