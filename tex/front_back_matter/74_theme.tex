\InternalLinks{fonction puissance}      \label{THEMEooBSBLooWcaQnR}

Il y a beaucoup de choses à dire\ldots

\begin{description}
    \item[Définition] 
        Nous considérons, pour \( a>0\), la fonction \( g_a\colon \eR\to \eR\) donnée par \( g_a(x)=a^x\). La définition de cette fonction se fait en de nombreuses étapes.
\begin{enumerate}
    \item
        \( a^n\) pour \( n\in \eN\) en la définition \ref{DEFooGVSFooFVLtNo}.
    \item 
        \( a^n\) pour \( n\in \eZ\) en la définition \ref{DEFooKEBIooZtPkac}.
    \item
        \( a^{1/n}\) pour \( n\in \eZ\) en la définition \ref{DEFooJWQLooWkOBxQ}.
    \item
        \( a^q\) pour \( q\in \eQ\) en la définition \ref{DEFooJWQLooWkOBxQ}.
    \item
        \( \sqrt[n]{ x }\) en la définition \ref{DEFooPOELooPouwtD}.
    \item
        La fonction \( g_a\) est Cauchy-continue sur \( \eQ\), c'est la proposition \ref{PROPooQRFSooVzYdJM}.
    \item
        \( a^x\) pour \( a>0\) et \( x\in \eR\) en la définition \ref{DEFooOJMKooJgcCtq}.
    \item
        \( a^z\) pour \( a>0\) et \( z\in \eC\) en la définition \ref{DEFooRBTDooNLcWGj}.
\end{enumerate}

\item[Quelques propriétés]
\begin{enumerate}
    \item
        Pour tout \( q\in \eQ\), il y a un \( \sqrt{ q }\) dans \( \eR\), proposition \ref{PROPooUHKFooVKmpte}.
    \item
        Si \( a>0\) et \( x,y\in \eR\) nous avons \( (a^x)^y=(a^y)^x=a^{xy}\) par la proposition \ref{PROPooDWZKooNwXsdV}.
    \item
        La fonction puissance est strictement croissante (en ses deux arguments), proposition \ref{PROPooRXLNooWkPGsO}.
    \item
        La formule \( a^{-x}=1/a^x\) est la proposition \ref{PROPooVADRooLCLOzP}\ref{ITEMooSCJBooNVJZah}.
\end{enumerate}

\item[Dérivation]
    Comme toutes les choses sur la fonction puissance, les preuves sont assez différentes selon que l'on parle de \( a^x\) ou de \( x^a\).
\begin{enumerate}
    \item
        La fonction puissance est strictement croissante, proposition \ref{PROPooRXLNooWkPGsO}
    \item
        La fonction \( a^x\) est dérivable et sa dérivée vérifie \( g_a'(x)=g_a(x)g_a'(0)\), proposition \ref{PROPooMXCDooBffXbl}.
    \item
        La formule de dérivation pour \( x\mapsto x^q\) avec \( q\in \eQ\) est la proposition \ref{PROPooSGLGooIgzque}. 
    \item
        La dérivation de \( x\mapsto x^{\alpha}\) avec \( \alpha\in \eR\) est la proposition \ref{PROPooKIASooGngEDh}. Si elle est tellement loin, c'est parce qu'elle nécessite de permuter une limite de fonctions avec une dérivée.
    \item
        Pour la formule générale de dérivation de \( x\mapsto a^x\) demande de savoir les logarithmes (proposition \ref{PROPooKUULooKSEULJ}).
\end{enumerate}

\item[L'équation fonctionnelle]
    L'exponentielle et plus généralement la fonction puissance \( g_a(x)=a^x\) peut être introduite au moyen d'une équation fonctionnelle au lieu de l'équation différentielle usuelle. Cette fameuse équation fonctionnelle est
    \begin{equation}
        f(x+y)=f(x)f(y)
    \end{equation}
    en la définition \ref{DEFooPJKMooOfZzgy}.
\begin{enumerate}
    \item
        Équivanlence entre l'équation fonctionnelle et l'équation différentielle, proposition \ref{PROPooGBUPooWtWaFI}.
    \item
        La fonction \( g_a(x)=a^x\) vérifie l'équation fonctionnelle \( g_a(x+y)=g_a(x)g_a(y)\) et les conséquences. C'est la définition \ref{DEFooPJKMooOfZzgy} et les choses qui suivent.
    \item
        L'équation fonctionnelle pour une fonction continue \( f\colon \eR\to S^1\) est traitée dans la proposition \ref{PROPooVJLYooOzfWCd}.
\end{enumerate}
\end{description}

Une définition alternative de la fonction puissance serait de poser directement
\begin{equation*}
    a^x=e^{x\ln(a)}.
\end{equation*}
De là les propriétés se déduisent facilement. Dans cette approche, les choses se mettent dans l'ordre suivant :
\begin{itemize}
    \item Définir \( \exp(x)\) par sa série pour tout \( x\).
    \item Démontrer que \( \exp(q)=\exp(1)^q\) pour tout rationnel \( q\) (première partie de la proposition \ref{PropCELWooLBSYmS}).
    \item Définir \( e=\exp(1)\).
    \item Définir, pour \( x\) irrationnel, \( a^x=\exp(x\ln(a))\).
    \item Prouver que \( e^x=\exp(x)\) pour tout \( x\).
\end{itemize}
