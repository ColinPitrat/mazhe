
\InternalLinks{déterminant}     \label{THMooUXJMooOroxbI}
    \begin{enumerate}
    \item
        Les \( n\)-formes alternées forment un espace de dimension \( 1\), proposition~\ref{ProprbjihK}.
    \item
        Déterminant d'une famille de vecteurs~\ref{DEFooODDFooSNahPb}.
    \item
        Calcul d'un déterminant de taille \( 2\times 2\) : équation \eqref{EQooQRGVooChwRMd}.
    \item
        Déterminant d'un endomorphisme~\ref{DefCOZEooGhRfxA}.
    \item
        Interprétations géométriques
            \begin{enumerate}
        \item
            À propos d'orthogonalité, le déterminant est très lié au produit vectoriel en dimension \( 3\). Et il le généralise en dimension supérieure.
            \begin{enumerate}
                \item
            Liaison au produit vectoriel (orthogonalité) dans la proposition~\ref{PROPooTUVKooOQXKKl}.
        \item
            En particulier le lemme~\ref{LEMooFRWKooVloCSM} nous dit comment un déterminant donne un vecteur orthogonal à une famille donnée de vecteurs.
            \end{enumerate}
        \item
            Déterminant et aires, volumes
            \begin{enumerate}
                \item
            Déterminant et mesure de Lebesgue : théorème~\ref{ThoBVIJooMkifod}.
                \item
            Aire du parallélogramme : proposition~\ref{PropNormeProdVectoabsint}.
        \item
            Volume du parallélépipède avec le produit mixte et le déterminant \( 3\times 3\),~\ref{NORMooWWOKooWzScnZ}.
            \end{enumerate}
        \end{enumerate}
        Tant que nous en sommes dans les interprétations géométrique, il faut lier déterminant, produit vectoriel, orthogonalité et mesure en notant que l'élément de volume lors de l'intégration en dimension \( 3\) est donné par \eqref{EQooNYWSooZuvcPe} : \( dS=\| T_u\times T_v \|\) qui est la norme du produit vectoriel des vecteurs tangents à la paramétrisation.

        Nous voyons dans l'équation \eqref{EQooARMAooQPhQAL} que l'élément de volume pour une partie de dimension \( n\) dans \( \eR^m\) (à l'occasion d'y intégrer une fonction) est donné par un déterminant mettant en jeu les vecteurs tangents de la paramétrisation.
        \item
            Le déterminant de Vandermonde est à la proposition~\ref{PropnuUvtj}. Il est utilisé à divers endroits :
\begin{enumerate}
    \item
        Pour prouver que \( u\) est nilpotente si et seulement si \( \tr(u^p)=0\) pour tout \( p\) (lemme \ref{LemzgNOjY})
    \item
        Pour prouver qu'un endomorphisme possédant \( \dim(E)\) valeurs propres distinctes est cyclique (proposition~\ref{PropooQALUooTluDif}).
\end{enumerate}

   \end{enumerate}
