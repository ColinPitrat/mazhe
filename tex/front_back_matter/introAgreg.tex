% This is part of Le Frido
% Copyright (c) 2011-2013,2016-2017
%   Laurent Claessens
% See the file fdl-1.3.txt for copying conditions.

%+++++++++++++++++++++++++++++++++++++++++++++++++++++++++++++++++++++++++++++++++++++++++++++++++++++++++++++++++++++++++++
\section{Originalité}
%+++++++++++++++++++++++++++++++++++++++++++++++++++++++++++++++++++++++++++++++++++++++++++++++++++++++++++++++++++++++++++

Ces notes ne sont pas originales par leur contenu : ce sont toutes des choses qu'on trouve facilement sur internet; je crois que la bibliographie est éloquente à ce sujet. Ce cours se distingue des autres sur les points suivants.
\begin{description}
    \item[La longueur] J'ai décidé de ne pas me soucier de la taille du fichier. Il fera cinq mille pages s'il le faut, mais il restera en un bloc. Étant donné qu'il n'existe qu'une seule mathématique, il ne m'a pas semblé intéressant de produire une division artificielle entre l'analyse, la géométrie ou l'algèbre. Tous le résultats d'une branche peuvent (et sont) être utilisés dans toutes les autres branches.

        Dans cette optique, je me suis évertué à ne créer que des références «vers le haut». À moins d'oubli de ma part\footnote{Par exemple pour les théorèmes pour lesquels je n'ai pas lu ni a fortiori écrit de preuves.}, il n'y a aucun endroit du texte qui dépend d'un lemme démontré plus bas. Le fait qu'un théorème \( B\) soit plus bas qu'un théorème \( A\) signifie qu'on peut démontrer \( A\) sans savoir \( B\).

    \item[La licence] Ce document est publié sous une licence libre. Elle vous donne explicitement le droit de copier, modifier et redistribuer.

    \item[Les mises à jour] Ce document est régulièrement mis à jour. Des fautes d'orthographe sont corrigées (presque) chaque jour. Si vous me signalez une faute de mathématique, elle sera corrigée.
    \item[Transparence] Je ne fais pas semblant que ces notes soient parfaites. Les points sur lesquels je ne suis pas sûr, les preuves que j'ai inventées moi-même sont clairement indiqués pour inciter le lecteur à redoubler de prudence. Une liste de questions à résoudre est inclue en la section~\ref{SecooCKWWooBFgnea}. De plus de nombreuses notes en bas de page en fonte \info{texttt}\quext{Comme celle-ci} indiquent des points sur lesquels je doute ou des étapes intermédiaires de calculs que je ne parviens pas à reproduire en suivant mes sources. Lorsque vous voyez une telle note, redoublez de prudence, et allez voir la source.
        
\end{description}

%+++++++++++++++++++++++++++++++++++++++++++++++++++++++++++++++++++++++++++++++++++++++++++++++++++++++++++++++++++++++++++ 
\section{Quelques choix qui peuvent provoquer des quiproquos}
%+++++++++++++++++++++++++++++++++++++++++++++++++++++++++++++++++++++++++++++++++++++++++++++++++++++++++++++++++++++++++++

Comme tout cours de mathématique, ce cours fait des choix qui sont parfois discutables. Voici quelques points sur lesquels les choix faits ici ne sont peut-être pas ceux faits par tout le monde. Ce sont donc des points sur lesquels vous devez faire attention pour éviter les quiproquos lors par exemple d'un oral dans un concours.

\begin{enumerate}
    \item
        Nous utilisons la définition usuelle de limite d'une fonction en un point. Elle diffère de celle donnée par le ministère de l'enseignement en France. Si votre but est de passer un concours d'enseignement en France, vous devriez lire~\ref{SUBSECooVHKCooYRFgrb}; dans tous les autres cas, la définition prise ici est celle qu'il vous faut.
    \item
        Un compact est un partie dont tout recouvrement par des ouverts admet un sous-recouvrement fini. Le fait d'être séparable n'est pas inclus à la définition de compact. De nombreux textes français incluent la séparabilité dans la compacité.
    \item
        Le logarithme sur \( \eC\) est une application \( \ln\colon \eC^*\to \eC\) définie partout sauf en zéro. Elle n'est donc pas continue sur la fameuse demi-droite. À ne pas confondre avec une \emph{détermination} du logarithme qui est par définition continue et donc non définie sur la demi-droite.

        Cela est un choix très discutable. La raison de donner à la notation «\( \ln\)» cette signification est simplement de suivre l'usage de Sage. Pour Sage, \( \ln(-1)\) existe et vaut \( i\pi\).

        Voir les remarques~\ref{REMooFBLLooDnkmjR}.
    \item
        Le mot «corps» n'implique pas la commutativité, et nous n'utilisons pas la terminologie «anneau à division». Voir la remarque~\ref{REMooYRNUooYgBBKF} et la discussion~\ref{NORMooGPWRooIKJqqw}.

\end{enumerate}
