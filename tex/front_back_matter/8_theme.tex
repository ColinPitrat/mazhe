

\InternalLinks{racines de polynôme et factorisation de polynômes}
    \begin{enumerate}
        \item
            Si \( \eA\) est une anneau, la proposition \ref{PropHSQooASRbeA} factorise une racine.
        \item
            Si \( \eA\) est un anneau, la proposition \ref{PropahQQpA} factorise une racine avec sa multiplicité.
        \item
            Si \( \eA\) est un anneau, le théorème \ref{ThoSVZooMpNANi} factorise plusieurs racines avec leurs multiplicités.
        \item
            Si \( \eK\) est un corps et \( \alpha\) une racine dans une extension, le polynôme minimal de \( \alpha\) divise tout polynôme annulateur par la proposition \ref{PropXULooPCusvE}.
        \item
            Le théorème \ref{ThoLXTooNaUAKR} annule un polynôme de degré \( n\) ayant \( n+1\) racines distinctes.
        \item
            La proposition \ref{PropTETooGuBYQf} nous annule un polynôme à plusieurs variables lorsqu'il a trop de racines.
        \item
            En analyse complexe, le principe des zéros isolés \ref{ThoukDPBX} annule en gros toute série entière possédant un zéro non isolé.
        \item 
            Polynômes irréductibles sur \( \eF_q\).
        \end{enumerate}
