
\InternalLinks{équations diophantiennes}
    \begin{enumerate}
        \item
            Équation \( ax+by=c\) dans \( \eN\), équation \eqref{EqTOVSooJbxlIq}.
        \item Dans \ref{subsecZVKNooXNjPSf}, nous résolvons \( ax+by=c\) en utilisant Bézout (théorème \ref{ThoBuNjam}).
        \item L'exemple \ref{ExZPVFooPpdKJc} donne une application de la pure notion de modulo pour \( x^2=3y^2+8\). Pas de solutions.
        \item L'exemple \ref{ExmuQisZU} résout l'équation \( x^2+2=y^3\) en parlant de l'extension \( \eZ[i\sqrt{2}]\) et de stathme.
        \item Les propositions \ref{PropXHMLooRnJKRi} et \ref{propFKKKooFYQcxE} parlent de triplets pythagoriciens.
        \item Le dénombrement des solutions de l'équation \( \alpha_1 n_1+\ldots \alpha_pn_p=n\) utilise des séries entières et des décomposition de fractions en éléments simples, théorème \ref{ThoDIDNooUrFFei}.
        \item La proposition \ref{PROPooLPKUooAlsYJg} donne une bijection \( \eN\times \eN\to \eN\) en résolvant dans \( \eN\) (entre autres) l'équation \( k=y^2+x\) pour \( k\) fixé.
        \end{enumerate}

