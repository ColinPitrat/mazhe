\InternalLinks{polynômes}

\begin{description}
    \item[Coefficients dans un anneau commutatif]

        \begin{enumerate}
            \item
Les polynômes à coefficients dans un anneau commutatif  sont à la section \ref{SECooVMABooVdhbPo}.
\item Si \( A\) est un anneau intègre, l'anneau \( A[X]\) est euclidien et principal, corollaire \ref{CORooINGJooHdOKgn}.
        \end{enumerate}
        

    \item[Coefficients dans un corps]
        \begin{enumerate}
            \item
                
Les polynômes à coefficients dans un corps sont à la section \ref{SECooFYOGooQHitgE}.
\item
Nous parlons de l'idéal des polynômes annulateurs dans le théorème \ref{ThoCCHkoU}.
            \item
                Le théorème \ref{ThoCCHkoU} dit que \( \eK[X]\) est une anneau principal et que tous ses idéaux sont engendrés par un unique polynôme unitaire.
            \item
                Le polynôme minimal est irréductible, proposition \ref{PropRARooKavaIT}.
        \end{enumerate}
    \item[Polynôme primitif]
    
        \begin{enumerate}
            \item
                Un polynôme est irréductible sur \( A\) si et seulement si irréductible et primitif sur le corps des fractions, corollaire \ref{CORooZCSOooHQVAOV}.
        \end{enumerate}

\end{description}



