\InternalLinks{polynômes}

\begin{description}
    \item[Définitions]
        Soient un anneau \( A\), un corps \( \eK\), une extension \( \eL\) de \( \eK\) et un élément \( \alpha\in \eL\).
        \begin{enumerate}
            \item
                \( A[X]\), définition \ref{DefRGOooGIVzkx}; l'anneau \( \eK[X]\) a même définition parce que c'est un cas particulier. L'évaluation d'un polynôme en un élément de l'anneau, \( P(\alpha)\) est définie en \ref{DEFooOSWQooHYYwVE}.
            \item
                \( \eK(X)\), le corps des fractions de \( \eK[X]\), définition \ref{DEFooHUWWooHiuRBr}. Si \( R=P/Q\) dans \( \eK(X)\), l'évaluation est \( R(\alpha)=P(\alpha)Q(\alpha)^{-1}\), définition \ref{DEFooZHBZooKlNfGZ}.
            \item
                \( \eK(\alpha)_{\eL}\) est le plus petit corps de \( \eL\) contenant \( \eK\) et \( \alpha\), définition \ref{DEFooVSKGooMyeGel}.
        \end{enumerate}

    \item[Coefficients dans un anneau commutatif]

        \begin{enumerate}
            \item
Les polynômes à coefficients dans un anneau commutatif  sont à la section \ref{SECooVMABooVdhbPo}.
        \end{enumerate}
        

    \item[Coefficients dans un corps]
Les polynômes à coefficients dans un corps sont à la section \ref{SECooFYOGooQHitgE}.
        \begin{enumerate}
                
\item
Nous parlons de l'idéal des polynômes annulateurs dans le théorème \ref{ThoCCHkoU}.
            \item
                Le théorème \ref{ThoCCHkoU} dit que \( \eK[X]\) est une anneau principal et que tous ses idéaux sont engendrés par un unique polynôme unitaire.
            \item
                Le polynôme minimal est irréductible, proposition \ref{PropRARooKavaIT}.
            \item
                Quelque formules sur le \( \pgcd\), lemme \ref{LemUELTuwK}.
        \end{enumerate}
    \item[Polynôme primitif]
    
        \begin{enumerate}
            \item
                Un polynôme est irréductible sur \( A\) si et seulement si irréductible et primitif sur le corps des fractions, corollaire \ref{CORooZCSOooHQVAOV}.
        \end{enumerate}

    \item[Polynôme d'endomorphisme]
        C'est la section \ref{SECooUEQVooLBrRiE}.

    \item[Racines et factorisation]

    \begin{enumerate}
        \item
            Si \( \eA\) est une anneau, la proposition \ref{PropHSQooASRbeA} factorise une racine.
        \item
            Si \( \eA\) est un anneau, la proposition \ref{PropahQQpA} factorise une racine avec sa multiplicité.
        \item
            Si \( \eA\) est un anneau, le théorème \ref{ThoSVZooMpNANi} factorise plusieurs racines avec leurs multiplicités.
        \item
            Si \( \eK\) est un corps et \( \alpha\) une racine dans une extension, le polynôme minimal de \( \alpha\) divise tout polynôme annulateur par la proposition \ref{PropXULooPCusvE}.
        \item
            Le théorème \ref{ThoLXTooNaUAKR} annule un polynôme de degré \( n\) ayant \( n+1\) racines distinctes.
        \item
            La proposition \ref{PropTETooGuBYQf} nous annule un polynôme à plusieurs variables lorsqu'il a trop de racines.
        \item
            En analyse complexe, le principe des zéros isolés \ref{ThoukDPBX} annule en gros toute série entière possédant un zéro non isolé.
        \item 
            Polynômes irréductibles sur \( \eF_q\).
        \end{enumerate}

\end{description}



