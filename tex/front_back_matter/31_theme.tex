
\InternalLinks{isométries}      \label{THMooVUCLooCrdbxm}

Il y a \( (\eR^n,\| . \|)\) et \( \eR^n,d\).

Les isométries de \( \| . \|\) sont linéaires, tandit que les isométries de la distance contiennent aussi les translations et les rotation de centre différent de l'origine.

Ne pas confondre une isométrie d'un espace affine avec une isométrie d'un espace euclidien. Les isométrie d'un espace euclidien préservent le produit scalaire et fixent donc l'origine (lemme~\ref{LEMooYXJZooWKRFRu}). Les isométrie des espaces affines par contre conservent les distances (définition~\ref{DEFooZGKBooGgjkgs}) et peuvent donc déplacer l'origine de l'espace vectoriel sur lequel il est modelé; typiquement les translations sont des isométries de l'espace affine mais pas de l'espace euclidien.

Parfois, lorsqu'on coupe les cheveux en quatre, il faut faire attention en parlant de \( \eR^n\) : soit on en parle comme d'un espace métrique (muni de la distance), soit on en parle comme d'un espace normé (muni de la norme ou du produit scalaire).

\begin{description}
    \item[Général] 
        Quelque résultats généraux et en vrac à propos d'isométries.
\begin{enumerate}
    \item
        Définition d'une isométrie pour une forme bilinéaire,~\ref{DEFooIQURooMeQuqX}. Pour une forme quadratique : définition~\ref{DEFooECTUooRxBhHf}.
    \item
        Définition du groupe orthogonal~\ref{DEFooUHANooLVBVID}, et le spécial orthogonal \( \SO(n)\) en la définition~\ref{DEFooJLNQooBKTYUy}. Le groupe \( \SO(2)\) est le groupe des rotations, par corollaire~\ref{CORooVYUJooDbkIFY}.
    \item
        Le lemme~\ref{LEMooHRESooQTrpMz} donne à toute rotation une matrice de la forme connue. C'est autour de cela que nous définissons les angles, définition \ref{DEFooUPUUooKAPFrh}.
    \item
        Le groupe orthogonal est le groupe des isométries de \( \eR^n\), proposition~\ref{PropKBCXooOuEZcS}.
    \item
        Les isométries de l'espace euclidien sont affines,~\ref{ThoDsFErq}.
    \item
        Les isométries de l'espace euclidien comme produit semi-direct : $\Isom(\eR^n)\simeq T(n)\times_{\rho}\gO(n)$, théorème~\ref{THOooQJSRooMrqQct}.
    \item
        Isométries du cube, section~\ref{SecPVCmkxM}.
    \item
        Nous parlons des isométries affines du tétraèdre régulier dans la proposition~\ref{PROPooVNLKooOjQzCj}.
\end{enumerate}

    \item[Groupe diédral]
        Le groupe diédral est un peu central dans la théorie des isométries de \( (\eR^2,d)\) parce que beaucoup de sous-groups finis des isométries de \( (\eR^2,d)\) sont en fait isomorphes au groupe diédral.
        \begin{enumerate}
    \item
        Générateurs du groupe diédral, proposition~\ref{PropLDIPoZ}.
    \item
        Un sous-groupe fini des isométries de \( (\eR^2,d)\) contenant au moins une réflexion est isomorphe au groupe diédral par le théorème \ref{THOooKDMUooUxQqbB}.
    \item
        Le théorème \ref{THOooAYZVooPmCiWI} dit que le groupe des isométries propres d'une partie quelconque de \( (\eR^2,d)\) est soit cyclique soit isomorphe au groupe diédral.
        \end{enumerate}
        
    \item[Isométries et réflexions]
        Dans un espace euclidien, toute isométrie peut être décomposée en réflexions autour d'hyperplans. Voici quelque énoncés à ce propos.
\begin{enumerate}
    \item
        Définition d'un hyperplan \ref{DEFooEWDTooQbUQws}.
    \item
        En dimension \( 2\), une rotation est définie comme composée de deux réflexions en la définition \ref{DEFooFUBYooHGXphm}.
    \item
        En dimension \( 2\), les réflexion ont un déterminant \( -1\) par le lemme \ref{LEMooSYZYooWDFScw}.
    \item
        Les isométries du plan \( (\eR^2,d)\) sont données dans le théorème \ref{THOooVRNOooAgaVRN}, et sont au plus 3 réflexions par le théorème \ref{THOooRORQooTDWFdv}.
    \item
        Décomposition des isométries de \( \eR^n\) en réflexions par le lemme \ref{LEMooJCDRooGAmlwp}.
    \item
        En particulier, les éléments de \( \SO(3)\) sont des compositions de deux réflexions par le corollaure \ref{CORooJCURooSRzSFb}.
    \item
        Une isométrie de \( \eR^n\) préserve l'orientation si et seulement si est elle composition d'un nombre pair de réflexions. C'est le théorème \ref{THOooWBIYooCtWoSq}.
\end{enumerate}
\item[Sous-groupe fini]
    \begin{enumerate}
        \item
            Les sous-groupes finis des isométries de \( (\eR^2,d)\) sont cycliques, théorème \ref{THOooKDMUooUxQqbB}.
        \item
            Les sous-groupes finis de \( \SO(3)\) sont listés dans \ref{}.
        \item
            Les sous groupes finis de \( \SO(2)\) sont cycliques, lemme \ref{LEMooUKEVooAEWvlM}.
    \end{enumerate}
\end{description}
