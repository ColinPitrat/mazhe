
\InternalLinks{isométries}      \label{THMooVUCLooCrdbxm}

Ne pas confondre une isométrie d'un espace affine avec une isométrie d'un espace euclidien. Les isométrie d'un espace euclidien préesrvent le produit scalaire et fixent donc l'origine (lemme \ref{}<++>)<++>

\begin{enumerate}
    \item 
        Définition d'une isométrie pour une forme bilinéaire, \ref{DEFooGGTYooXsHIZj}. Pour une forme quadratique : définition \ref{DEFooECTUooRxBhHf}.
    \item
        Définition du groupe orthogonal \ref{DEFooUHANooLVBVID}, et le spécial orthogonal \( \SO(n)\) en la définition \ref{DEFooJLNQooBKTYUy}. Le groupe \( \SO(2)\) est le groupe des rotations, par corollaire \ref{CORooVYUJooDbkIFY}.
    \item
        Le lemme \ref{LEMooHRESooQTrpMz} donne à toute rotation une matrice de la forme connue. C'est autour de cela que nous définissons les angles.
    \item
        Le groupe orthogonal est le groupe des isométries de \( \eR^n\), proposition \ref{PropKBCXooOuEZcS}.
    \item
        Les isométries de l'espace euclidien sont affines, \ref{ThoDsFErq}.
    \item
        Les isométries de l'espace euclidien comme produit semi-direct : $\Isom(\eR^n)\simeq T(n)\times_{\rho}\gO(n)$, théorème \ref{THOooQJSRooMrqQct}.
    \item
        Isométries du cube, section \ref{SecPVCmkxM}.
    \item 
        Générateurs du groupe diédral, proposition \ref{PropLDIPoZ}.
\end{enumerate}

