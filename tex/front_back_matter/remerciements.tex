%+++++++++++++++++++++++++++++++++++++++++++++++++++++++++++++++++++++++++++++++++++++++++++++++++++++++++++++++++++++++++++
\section{Auteurs, contributeurs, sources et remerciements}
%+++++++++++++++++++++++++++++++++++++++++++++++++++++++++++++++++++++++++++++++++++++++++++++++++++++++++++++++++++++++++++

Les remerciements, dans chaque catégorie, sont mis dans l'ordre chronologique approximatif. Les noms en couvertures sont ceux qui ont fourni du code \LaTeX\ (typiquement : un patch via github), par ordre chronologique approximatif d'entrée dans le projet.

%---------------------------------------------------------------------------------------------------------------------------
\subsection{Ceux qui ont travaillé sur le Frido}
%---------------------------------------------------------------------------------------------------------------------------

\begin{enumerate}
    \item
        Carlotta Donadello pour l'ensemble du cours de CTU de géométrie analytique 2010-2011. Une grosse partie de «analyse réelle» vient de là.
    \item
        Les étudiants de géométrie analytique en CTU 2010-2011 ont détecté d'innombrables coquilles. Les étudiants du cours présentiel de géométrie analytique 2011-2012 ont signalé un certain nombre d'incorrections dans les exercices et les corrigés. Les agrégatifs de Besançon 2011-2012 pour leurs plans et leurs développements.
    \item
        Lilian Besson pour m'avoir signalé un paquet de fautes, et quelques points pas clairs en statistiques.
    \item
        Plouf qui m'a signalé une coquille dans le fil \href{http://passeurdesciences.blog.lemonde.fr/2014/01/24/la-selection-scientifique-de-la-semaine-numero-106}{la-selection-scientifique-de-la-semaine-numero-106}.
    \item
        Benjamin de Block pour des coquilles et une mise au point sur les conventions à propos de \( \eR^+\) et \( (\eR^+)^*\).
    \item
        Olivier Garet pour avoir répondu à plein de questions de probabilités.
    \item
        François Gannaz pour de la relecture et une version plus claire de la preuve (et de l'énoncé) de la proposition~\ref{PROPooIOQEooGMcCJm}.
    \item
        Danarmk pour des réponses à des questions dans les commentaires (allongement pour éviter un Overfull hbox) \url{http://linuxfr.org/nodes/110155/comments/1675589}. Et aussi pour \href{ https://github.com/LaurentClaessens/mazhe/issues/87 }{ une discussion } à propos de la topologie sur \( \swD(\Omega)\).
    \item
        Cédric Boutilier pour des réponses à des questions de probabilité statistique. \url{https://github.com/LaurentClaessens/mazhe/issues/16}
    \item
        Remsirems pour des réponses à des questions d'analyse. \url{http://linuxfr.org/nodes/110155/comments/1675813}
    \item
        Bertrand Desmons pour plusieurs patchs rendant plus clairs de nombreux passages sur les suites de Cauchy dans \( \eQ\).
    \item
        Anthony Ollivier pour m'avoir fait remarquer qu'il n'est pas vrai que \( A[X]\) est euclidien lorsque \( A\) est intègre (contre-exemple : \( A=\eZ\)). Ça fait une faute de moins dans le Frido.
    \item
        ybailly pour avoir détecté un bon nombre de coquilles dans la partie sur les ensembles de nombres.
    \item
        Éric Guirbal pour le remplacement de \info{frenchb} par \info{french}.
    \item
        cdrcprds pour une réponse à une question d'algèbre, démonstration à l'appui à propos de \href{https://github.com/LaurentClaessens/mazhe/issues/52#issuecomment-333251728}{pgcd}.
    \item
        Antoine Bensalah pour avoir répondu à une question sur Lax-Milgram tout en même temps que pointé une erreur dans la démonstration et fourni l'exemple~\ref{EXooTTBDooUNhBOc} sur l'optimalité de l'inégalité.
    \item
        Guillaume Deschamps pour ses remarques à propos du fait que le chapitre «constructions des ensembles» est très ardu.
    \item
        Guillaume Barriere pour sa relecture attentive jusqu'aux corps.
    \item
        Samy Clementz pour avoir découvert une faute dans la définition de mesure positive sur un espace mesurable.
    \item
        Sylvain Rousseau pour avoir clarifié une construction dans le théorème de Bower version \(  C^{\infty}\).
    \item
        Maxmax pour des typos dans l'index thématique.
    \item
        Laurent Choulette pour une typo dans les propriétés du neutre d'un groupe.
    \item
        Pierre Lairez pour la démonstration du théorème d'inversion de limite et de dérivée \ref{THOooXZQCooSRteSr} sans passer par les intégrales (et les lemmes correspondants à propos du module de continuité).
    \item
        Gregory Berhuy pour des réponses d'algèbres dans les catégories facile, moyen et difficile.
    \item
        Benoît Tran pour avoir signalé un paquet de typos dans la démonstration de l'ellipsoïde de John-Loewner et ses dépendances.
    \item
        Provatiscus pour avoir signalé un paquet de choses pas claires, et surtout pour avoir trouvé une faute dans la démonstration du fait qu'une fonction continue sur \( \eQ\) se prolonge en une fonction continue sur \( \eR\). Et pour cause : cet énoncé est faux. \url{https://github.com/LaurentClaessens/mazhe/issues/124}
    \item 
        William pour l'environnement \info{example} qui gère correctement le triangle.
\end{enumerate}

%---------------------------------------------------------------------------------------------------------------------------
\subsection{Aide directe, mais pas volontairement sur le Frido}
%---------------------------------------------------------------------------------------------------------------------------

\begin{enumerate}
    \item
        Plein de monde pour diverses contributions à des énoncés d'exercices. Pierre Bieliavsky pour les énoncés d'analyse numérique (MAT1151 à Louvain la Neuve 2009-2010). Jonathan Di Cosmo pour certaines corrections de MAT1151. François Lemeux, exercices sur les normes de matrices et correction de coquilles.  Martin Meyer et Mustapha Mokhtar-Kharroubi pour certains exercices du cours \emph{Outils mathématiques} (surtout ceux des DS et examens).
    \item
        Nicolas Richard et Ivik Swan pour les parties des exercices et rappels de calcul différentiel et intégral (Université libre de Bruxelles, 2003-2004) qui leurs reviennent.
    \item
	Carlotta Donadello pour la partie géométrie analytique : topologie dans \( \eR^n\), courbes, intégrales, limites. (Université de Franche-Comté 2010-2012)
    \item
        Le forum usenet de math, en particulier pour la construction des corps fini dans la fil « Vérifier qu'un polynôme est primitif » initié le 20 décembre 2011.
    \item
        Mihai Bostan nous a donné ses notes manuscrites de son cours présentiel de géométrie analytique 2009-2010. (Presque) Toute la structure du cours de géométrie analytique lui est due (qui est maintenant fondue un peu partout dans les chapitres d'analyse).
\end{enumerate}

%---------------------------------------------------------------------------------------------------------------------------
\subsection{Des gens qui ont fait un travail qui m'a bien servi}
%---------------------------------------------------------------------------------------------------------------------------

\begin{enumerate}
    \item
	Arnaud Girand pour avoir mis ses développements bien faits en ligne. Une bonne vingtaine de résultats un peu partout dans ces notes viennent de lui.
    \item
	Le site \url{http://www.les-mathematiques.net} m'a donné les preuves de nombreux résultats.
    \item
	Pierre Monmarché pour son document en ligne tout plein de développements, et des réponses à des questions.
    \item
        Tous les contributeurs du Wikipédia francophone (et aussi un peu l'anglophone) doivent être remerciés. J'en ai pompé des quantités astronomiques; des articles utilisés sont cités à divers endroits du texte, mais ce n'est absolument pas exhaustif. Il n'y a à peu près pas un résultat important de ces notes dont je n'aie pas lu la page Wikipédia, et souvent plusieurs pages connexes.
    \item
        Les intervenants du fil «\href{http://www.les-mathematiques.net/phorum/read.php?2,302266}{Antisymétrisation, alterné, déterminant et caractéristique}» sur \texttt{les-mathematiques.net} m'ont bien aidé pour la section sur les déterminants~\ref{SecGYzHWs} (bien qu'ils ne le savent pas).
    \item
        Xavier Mauquoy pour l'énoncé et la preuve du théorème~\ref{THOooYXJIooWcbnbm}.
    \item
        David Revoy pour les dessins de Pepper\&Carrot \href{https://www.peppercarrot.com/fr/article285/episode-8-pepper-s-birthday-party}{de la couverture}.
\end{enumerate}

J'ai souvent donné entre parenthèse à côté des mots « théorème », « lemme » ou « proposition » une ou plusieurs références vers les sources de la preuve que je donne. Ce sont parfois des liens vers la bibliographie; parfois aussi des liens hypertexte vers des sites, des blogs, etc. Tous ces gens ont fait du bon boulot. Sans toute cette « communauté », l'internet serait mort\footnote{Cette dernière phrase doit être comprise comme un appel à ne pas utiliser Moodle et autres iCampus pour diffuser vos cours de math, ou en tout cas pas comme moyen exclusif.}.

%+++++++++++++++++++++++++++++++++++++++++++++++++++++++++++++++++++++++++++++++++++++++++++++++++++++++++++++++++++++++++++
\section{Originalité}
%+++++++++++++++++++++++++++++++++++++++++++++++++++++++++++++++++++++++++++++++++++++++++++++++++++++++++++++++++++++++++++

Ces notes ne sont pas originales par leur contenu : ce sont toutes des choses qu'on trouve facilement sur internet; je crois que la bibliographie est éloquente à ce sujet. Ce cours se distingue des autres sur les points suivants.
\begin{description}
    \item[La longueur] J'ai décidé de ne pas me soucier de la taille du fichier. Il fera cinq mille pages s'il le faut, mais il restera en un bloc. Étant donné qu'il n'existe qu'une seule mathématique, il ne m'a pas semblé intéressant de produire une division artificielle entre l'analyse, la géométrie ou l'algèbre. Tous le résultats d'une branche peuvent (et sont) être utilisés dans toutes les autres branches.

        Dans cette optique, je me suis évertué à ne créer que des références «vers le haut». À moins d'oubli de ma part\footnote{Par exemple pour les théorèmes pour lesquels je n'ai pas lu ni a fortiori écrit de preuves.}, il n'y a aucun endroit du texte qui dépend d'un lemme démontré plus bas. Le fait qu'un théorème \( B\) soit plus bas qu'un théorème \( A\) signifie qu'on peut démontrer \( A\) sans savoir \( B\).

    \item[La licence] Ce document est publié sous une licence libre. Elle vous donne explicitement le droit de copier, modifier et redistribuer.

    \item[Les mises à jour] Ce document est régulièrement mis à jour. Des fautes d'orthographe sont corrigées (presque) chaque jour. Si vous me signalez une faute de mathématique, elle sera corrigée.
    \item[Transparence] Je ne fais pas semblant que ces notes soient parfaites. Les points sur lesquels je ne suis pas sûr, les preuves que j'ai inventées moi-même sont clairement indiqués pour inciter le lecteur à redoubler de prudence. Une liste de questions à résoudre est inclue en la section~\ref{SecooCKWWooBFgnea}. Voir \ref{SECooWVHBooCaYoXP} pour plus de détails.
\end{description}

%+++++++++++++++++++++++++++++++++++++++++++++++++++++++++++++++++++++++++++++++++++++++++++++++++++++++++++++++++++++++++++
\section{Les choses qui doivent vous faire tiquer}
%+++++++++++++++++++++++++++++++++++++++++++++++++++++++++++++++++++++++++++++++++++++++++++++++++++++++++++++++++++++++++++
\label{SECooWVHBooCaYoXP}

Un cours de math doit toujours être lu attentivement, surtout si vous avez l'intention de resservir à un jury le fruit de vos lectures. Dans ce livre, trois éléments doivent vous faire redoubler de prudence.

\begin{description}
    \item[La référence \cite{MonCerveau}] 
        D'abord les références à \cite{MonCerveau} indiquent qu'une bonne partie de ce qui suit est de l'invention personnelle de l'auteur. Cela ne veut évidemment pas dire que c'est moi qui ait découvert le résultat. Ça veut dire que je n'ai pas trouvé le résultat ou certaines parties de la preuve.
    \item[Les notes en bas de page]
        Certaine notes en bas de page sont écrite dans une fonte spéciale\quext{Les notes comme celle-ci signifient qu'il y a quelque chose dont je ne suis pas sûr.}. Elles indiquent des points sur lesquels je doute ou des étapes de calculs que je ne parviens pas à reproduire en suivant mes sources. Lorsque vous voyez une telle note, redoublez de prudence, allez voir la source, et écrivez-moi si vous pouvez résoudre le problème.
    \item[Les environnements dédiés]       
        Et enfin certains problèmes sont indiqués plus longuement dans un environnement dédié en petits caractères comme ceci :

        \begin{probleme}
            Les choses écrites comme ceci sont des questions ou des éléments sur lesquels j'ai un doute. Lisez-les attentivement. Ces notes mentionnent des points que personnellement je n'oserais pas affirmer plein d'aplomb à un jury d'agrégation.
        \end{probleme}
\end{description}

%+++++++++++++++++++++++++++++++++++++++++++++++++++++++++++++++++++++++++++++++++++++++++++++++++++++++++++++++++++++++++++
\section{Quelques choix qui peuvent provoquer des quiproquos}
%+++++++++++++++++++++++++++++++++++++++++++++++++++++++++++++++++++++++++++++++++++++++++++++++++++++++++++++++++++++++++++

Comme tout cours de mathématique, ce cours fait des choix qui sont parfois discutables. Voici quelques points sur lesquels les choix faits ici ne sont peut-être pas ceux fait par tout le monde. Ce sont donc des points sur lesquels vous devez faire attention pour éviter les quiproquos lors par exemple d'un oral dans un concours.

\begin{enumerate}
    \item
        Nous utilisons la définition usuelle de limite d'une fonction en un point. Elle diffère de celle donnée par le ministère de l'enseignement en France. Si votre but est de passer un concours d'enseignement en France, vous devriez lire~\ref{SUBSECooVHKCooYRFgrb}; dans tous les autres cas, la définition prise ici est celle qu'il vous faut.
    \item
        Un compact est un partie d'un espace topologique pour lequel tout recouvrement par des ouverts admet un sous-recouvrement fini. Le fait d'être séparable n'est pas inclus dans la définition de compact. De nombreux textes français incluent la séparabilité dans la compacité.
    \item
        Le logarithme sur \( \eC\) est une application \( \ln\colon \eC^*\to \eC\) définie partout sauf en zéro. Elle n'est donc pas continue sur la fameuse demi-droite. À ne pas confondre avec une \emph{détermination} du logarithme qui est par définition continue et donc non définie sur la demi-droite.

        Cela est un choix très discutable. La raison de donner à la notation «\( \ln\)» cette signification est simplement de suivre l'usage de Sage. Pour Sage, \( \ln(-1)\) existe et vaut \( i\pi\).

        Voir les remarques~\ref{REMooFBLLooDnkmjR}.
    \item
        Le mot «corps» n'implique pas la commutativité, et nous n'utilisons pas la terminologie «anneau à division». Voir la remarque~\ref{REMooYRNUooYgBBKF} et la discussion~\ref{NORMooGPWRooIKJqqw}.
\end{enumerate}

%+++++++++++++++++++++++++++++++++++++++++++++++++++++++++++++++++++++++++++++++++++++++++++++++++++++++++++++++++++++++++++ 
\section{Autre choix pas spécialement standards}
%+++++++++++++++++++++++++++++++++++++++++++++++++++++++++++++++++++++++++++++++++++++++++++++++++++++++++++++++++++++++++++

Nous listons ici quelque choix qui n'induisent pas de différences ou d'incompatibilité avec les autres cours, mais qui doivent être compris et justifiés.

\begin{enumerate}
    \item
        Nous n'utilisont pas les notations \( o(x)\) ou autres \( O(N^2)\). D'abord parce que je n'ai jamais très bien compris comment elles fonctionnent, et ensuite (surtout) parce que ces notations induisent en erreur. Ce sont des notations qui cachent sous des notations a peu près intuitive l'utilisation de théorèmes pas simples.

        Écrire
        \begin{equation}
            f(x)=P(x)+o(x^2),
        \end{equation}
        c'est un peu comme quand on écrit (horreur !)
        \begin{equation}
            F(x)=\int f(x)dx+C.
        \end{equation}
        Où est le \( x\) à droite ? Quel est le status de \( C \) ? 

        Même chose pour la notation \( f(x)=P(x)+o(x^2)\). Le \( x\) de \( o(x^2)\) est-il le \( x\) qu'on a à gauche ? Si \( g(x)=Q(x)+o(x^2)\), est-ce le même \( o\) que celui de \( f\) ?
    \item
        Nous allons être plus calme avec la notation \( A[X]\) pour les polynômes sur l'anneau \( A\), et encore moins \( A[X_1,\ldots, X_n]\) pour les polynômes de \( n\) variables. Au lieu de cela nous utilisons \( \Poly(A)\) et \( \Poly_n(A)\).

        Est-ce que vous diriez que \( A[X]=A[Y]\) ? Quelle est exactement la nature de \( X\) dans \( P=X^2+1\) ou dans \( P(X)=X^2+1\) ? Si \( P\in A[X]\) vaut \( P(X)=X^2\)  et si \( Q\in A[Y]\) vaut \( Q(Y)=Y^2\), est-ce que vous oseriez écrire \( P=Q\) ?
\end{enumerate}

%+++++++++++++++++++++++++++++++++++++++++++++++++++++++++++++++++++++++++++++++++++++++++++++++++++++++++++++++++++++++++++
\section{Sage est là pour vous aider}
%+++++++++++++++++++++++++++++++++++++++++++++++++++++++++++++++++++++++++++++++++++++++++++++++++++++++++++++++++++++++++++

Il existe de nombreux logiciels de mathématique. Notre préféré est \href{http://www.sagemath.org}{Sage} pour une raison très précise : en tant que langage de programmation, Sage est python qui est un langage généraliste. La syntaxe et la structure de Sage ne sont pas \emph{ad hoc} pour faire de math, et ce qu'on apprend en Sage peut être recyclé pour faire n'importe quoi : navigateur web, script de manipulation de texte, traitement d'image, réseau neuronaux, \ldots

%Par ailleurs, le vingt et unième siècle est déjà largement entamé; si vous vous lancez dans une carrière scientifique, il vous faudra maitriser l'informatique un peu plus solidement qu'être virtuose es trouver le trajet le plus court en bus sur votre téléphone.

Sage est un logiciel disponible pour l'épreuve de modélisation de l'agrégation de mathématique; il y a donc de bonnes chances que vous en ayez l'usage.

%---------------------------------------------------------------------------------------------------------------------------
\subsection{Lancez-vous dans Sage}
%---------------------------------------------------------------------------------------------------------------------------


\begin{enumerate}
	\item
        Aller sur \url{http://www.sagemath.org},
	\item
		créer un compte,
	\item
		créer des feuilles de calcul et s'amuser !!
\end{enumerate}

Il y a beaucoup de \href{http://lmgtfy.com/?q=sage+documentation}{documentation} sur le \href{http://www.sagemath.org}{site officiel}\footnote{\href{http://www.sagemath.org}{http://www.sagemath.org}}, et nous vous conseillons particulièrement le livre \cite{ooBLMMooWTPsQy}.

Si vous comptez utiliser régulièrement ce logiciel, je vous recommande \emph{chaudement} de \href{http://mirror.switch.ch/mirror/sagemath/index.html}{l'installer} sur votre ordinateur.

%---------------------------------------------------------------------------------------------------------------------------
\subsection{Exemples de ce que Sage peut faire pour vous}
%---------------------------------------------------------------------------------------------------------------------------

Ce livre est émaillé de petits bouts de code en Sage montrant ses différentes fonctionnalités là où nous en avons besoin\footnote{Soit un vrai besoin comme tracer un graphique en 3D, soit de la paresse comme calculer une grosse dérivée.}. Voici une liste (non exhaustive) de ce que Sage peut faire pour vous.

\begin{enumerate}

	\item
        Calculer des limites de fonctions, exemples~\ref{ExBCRXooDVUdcf} et~\ref{ExCWDRooKxnjGL}.
	\item
        Tracer des graphes de fonctions, exemple~\ref{ExCWDRooKxnjGL}.
	\item
        Tracer des courbes en trois dimensions, voir exemple~\ref{ExempleTroisDxxyy}. Notez que pour cela vous devez installer aussi le logiciel Jmol. Pour Ubuntu, c'est dans le paquet \info{icedtea6-plugin}.
	\item
		Calculer des dérivées partielles de fonctions à plusieurs variables, voir exemple~\ref{exJMGTooZcZYNy}.
	\item
        Résoudre des systèmes d'équations linéaires. Voir les exemples~\ref{exKGDIooVefujD} et~\ref{ExBGCEooPIQgGW}. Lire aussi \href{http://www.sagemath.org/doc/constructions/linear_algebra.html#solving-systems-of-linear-equations}{la documentation}.
	\item
        Tout savoir d'une forme quadratique, voir exemple~\ref{exBNGVooIvKfTT}.
	\item
        Calculer la matrice hessienne de fonctions de deux variables, déterminer les points critiques, déterminer le genre de la matrice hessienne aux points critiques et écrire extrema de la fonctions (sous réserve d'être capable de résoudre certaines équations), voir les exemples~\ref{exZHGRooTQpVpq} et~\ref{exHWIHooOAvaDQ}.
	\item
        Indiquer une infinité de solutions à une équation en utilisant des paramètres. L'exemple \ref{exEEHPooKDxLTJ} montre ça avec une équation algébrique. Un exemple concernant des fonctions trigonométriques :
        \begin{verbatim}
sage: solve(sin(x)/cos(x)==1,x,to_poly_solve=True)
[x == 1/4*pi + pi*z1]
sage: solve(sin(x)**2==cos(x)**2,x,to_poly_solve=True)
[sin(x) == cos(x), x == -1/4*pi + 2*pi*z86, x == 3/4*pi + 2*pi*z84]
        \end{verbatim}

        Notez l'option \info{to\_poly\_solve=true} dans \info{solve}.

	\item
        Calculer des dérivées symboliquement, voir exemple~\ref{exRNZKooUIOfPU}.
	\item
        Calculer des approximations numériques comme dans l'exemple~\ref{exLFYFooNCXCJz}.
    \item
        Calculer dans un corps de polynômes modulo comme \( \eF_p[X]/P\) où \( P\) est un polynôme à coefficients dans \( \eF_p\). Voir l'exemple~\ref{ExemWUdrcs}.
\end{enumerate}

Sage peut en général faire tout ce que vous êtes capable de faire à l'entrée en master et probablement bien plus, à la notable exception des limites à deux variables.

\begin{remark}
    Sage peut toutefois vous induire en erreur si vous n'y prenez pas garde parce qu'il sait des choses en mathématique que vous ne savez pas. Par conséquent il peut parfois vous donner des réponses (mathématiquement exactes) auxquelles vous ne vous attendez pas. Voir par exemple~\ref{ooOPWYooDDSZWx} pour le logarithme de nombres négatifs. Et aussi ceci :

\lstinputlisting{tex/sage/sageSnip017.sage}

Sage fait une différence entre \info{Infinity} et \info{+Infinity} et donne
\begin{equation}
    \lim_{x\to 0} \frac{1}{ x }=\infty
\end{equation}
ainsi que
\begin{equation}
    \lim_{x\to 0} \frac{1}{ x^2 }=+\infty.
\end{equation}
\end{remark}

Voir aussi la compactification en un point d'Alexandroff \ref{PROPooHNOZooPSzKIN}.

%+++++++++++++++++++++++++++++++++++++++++++++++++++++++++++++++++++++++++++++++++++++++++++++++++++++++++++++++++++++++++++
\section{Comment contribuer et aider ?}
%+++++++++++++++++++++++++++++++++++++++++++++++++++++++++++++++++++++++++++++++++++++++++++++++++++++++++++++++++++++++++++
\label{SecooCKWWooBFgnea}

Ces notes ne sont pas relues de façon systématique. Aucune garantie. Merci de me signaler toute faute ou remarque : le relecteur c'est toi. Voici une petite liste de questions que je me pose ou de choses écrites dont je ne suis pas certain. Si vous avez un avis ou une réponse à un des points, merci de vous faire connaitre.

Les questions ouvertes sont divisées en trois niveaux de difficulté :
\begin{enumerate}
    \item
        Niveau facile : un étudiant de licence devrait pouvoir le faire.
    \item
        Niveau moyen : un candidat à l'agrégation de mathématique devrait pouvoir le faire.
    \item
        Demande probablement de connaissances avancées en mathématique; au moins être tout à fait à l'aise avec le niveau d'agrégation.
\end{enumerate}

Quel que soit votre niveau, vous pouvez faire ceci :

\begin{enumerate}
    \item
        M'écrire pour me signaler toutes les fautes que vous voyez, même si vous n'êtes pas sûr.
    \item
        Si vous n'êtes pas expert, me signaler tous le endroits qui vous semblent obscurs. Vu que ces notes sont destinées à \emph{apprendre}, les avis des non experts sont très importants.
    \item
        Mettre une copie de (ou un lien vers) ces notes sur votre site.
\end{enumerate}

%--------------------------------------------------------------------------------------------------------------------------- 
\subsection{Des preuves qui manquent}
%---------------------------------------------------------------------------------------------------------------------------

Vous trouverez un peu partout des énoncés sans preuves. Certaines sont surement très faciles, et d'autres probablement assez compliquées. N'hésitez pas à rédiger une preuve et me l'envoyer.

Vous pouvez m'envoyer vos preuves sous forme de «c'est bien fait dans tel cours», avec une URL.

Ne me dites juste pas «c'est bien fait dans tel \emph{livre}». Je ne travaille pas à l'université, et je n'ai pas accès à une bibliothèque universitaire; je n'ai donc pas réellement accès à ces fameux «livres» dont tout le monde parle.

%--------------------------------------------------------------------------------------------------------------------------- 
\subsection{Du texte qui manque}
%---------------------------------------------------------------------------------------------------------------------------

Vous remarquerez que de nombreuses pages du Frido sont des enchainement de théorèmes et démonstrations sans articulations. Autrement dit, il manque ce qu'à l'agrégation on irait à l'oral quand on présente le plan.

%---------------------------------------------------------------------------------------------------------------------------
\subsection{Mes questions de géométrie}
%---------------------------------------------------------------------------------------------------------------------------

%///////////////////////////////////////////////////////////////////////////////////////////////////////////////////////////
\subsubsection{Facile}
%///////////////////////////////////////////////////////////////////////////////////////////////////////////////////////////

\begin{enumerate}
    \item
        Donner une interprétation en termes de plans, droites ou je ne sais quoi de géométrique au \( 4\)-cycle parmi les permutations des sommets du tétraèdre. D'où sort géométriquement la matrice \eqref{EQooONDUooYlduup} ?
\end{enumerate}

%///////////////////////////////////////////////////////////////////////////////////////////////////////////////////////////
\subsubsection{Moyen}
%///////////////////////////////////////////////////////////////////////////////////////////////////////////////////////////

\begin{enumerate}
    \item
        Démontrer la proposition~\ref{PROPooCOZCooCghwaR} qui donne les relations de Chasles pour un espace affine.
    \item
        En géométrie projective, dans la sphère de Riemann \( \hat\eC=P_1(\eC)=\eC\cup\{ \infty \}\) est-ce qu'il existe une notion de cercle dont le centre est \( \infty\) ? Voir le point~\ref{NORMooCXVJooMTMqEU}.
    \item
        Géométrie projective. Tout le monde semble définir le birapport en identifiant \( P(\eK^2)\) à \( \hat\eK=\eK\cup\{ \infty \}\). Bien entendu, personne ne semble s'être attribué la mission d'expliciter la dépendance du birapport en le choix de l'identification. Je le fais à la définition~\ref{DEFooBFSKooDwzwmO}.

        Mais cette définition dépend du choix d'identification \( \varphi\colon P(\eK^2)\to \hat\eK\), comme le montre l'exemple~\ref{EXooYCOYooWFSfUv}. J'ai donc défini des classes d'identifications possibles \( A(\varphi)\) en~\ref{DEFooMLQUooGwvQMh}. Et je démontre la proposition~\ref{PROPooTFMQooIOQGvs} que si \( \varphi_a\in A(\varphi)\) alors les birapports construits à partir de \( \varphi\) et \( \varphi_a\) sont identiques.

        Question : pourquoi personne ne semble faire ce travail ? En quoi l'identification \( \varphi_0\) que tout le monde utilise est plus canonique qu'une autre ? Est-ce que l'on peut décrire simplement les classes \( A(\varphi)\) ? Le groupe qui conserve le birapport associé à \( \varphi\) est-il isomorphe au groupe qui conserve le birapport associé à \( \varphi'\) ? Quels que soient \( \varphi\) et \( \varphi'\) ?

        Suis-je la seule personne au monde à m'être demandé si le birapport était un objet canonique ?
    \item
        En géométrie projective, dans \( P_1(\eC)=\hat\eC=\eC\cup\{ \infty \}\), si \( \ell\) est une droite dans \( \eC\), est-ce que la droite correspondante dans \( \hat\eC\) contient le point \( \infty\) ? Moi j'ai envie de dire que \( \infty\) est sur toutes les droites. Voir le problème~\ref{PROBooZHHTooIFNwxR} et la remarque~\ref{REMooBMAEooHDvNID}.

    \item
        Géométrie affine, barycentre. Les mauvaise langues diraient que tout cela est du snobisme autour de la paresse d'écrire \( \vect{ xy }\) au lieu de \( y-x\). Est-ce qu'il y a des cas où toute l'étude des espaces affines et des barycentres en particulier apportent \emph{réellement} plus qu'une facilité d'écriture par rapport à travailler dans le cadre vectoriel pur ?
\end{enumerate}


%---------------------------------------------------------------------------------------------------------------------------
\subsection{Mes questions d'algèbre}
%---------------------------------------------------------------------------------------------------------------------------

%///////////////////////////////////////////////////////////////////////////////////////////////////////////////////////////
\subsubsection{Facile}
%///////////////////////////////////////////////////////////////////////////////////////////////////////////////////////////

%///////////////////////////////////////////////////////////////////////////////////////////////////////////////////////////
\subsubsection{Moyen}
%///////////////////////////////////////////////////////////////////////////////////////////////////////////////////////////

\begin{enumerate}
    \item La définition \ref{DEFooLNEXooYMQjRo} de \( \sum_{i\in I}f(i)\)  (\( I\) est fini et \( f\colon I\to (G,+)\) est une application) demande une preuve. Comment prouver que
    \begin{equation}
        \sum_{i=1}^nf\big( \sigma_1(i) \big)=\sum_{i=1}^nf\big( \sigma_2(i) \big)
    \end{equation}
    lorsque \( \sigma_1\) et \( \sigma_2\) sont des bijections ? 

    De façon peut-être plus simple, il faut prouver la proposition \ref{PROPooJBQVooNqWErk} qui exprime l'invariance d'une somme commutative finie sous le groupe des bijections des termes.

    Je crains que cela demande de décomposer \( \sigma_1\sigma_2^{-1}\) en cycles et de pas mal chipoter pour tout ramener à des permutations de deux termes.
\end{enumerate}

%///////////////////////////////////////////////////////////////////////////////////////////////////////////////////////////
\subsubsection{Difficile}
%///////////////////////////////////////////////////////////////////////////////////////////////////////////////////////////

\begin{enumerate}
    \item   \label{ITEMooYWPLooMHuGzQ}
        Que penser de~\ref{EXooJRSUooYhAZkR} qui dit que l'extension de corps \( \eK(\alpha)\) dépend en réalité du corps ambiant dans lequel on calcule l'extension ?

        Est-ce qu'il existe des exemples moins triviaux et plus utiles que celui que je donne ?
    \item       \label{ITEMooUBZIooDDcfWg}
        Pour quelle classe d'anneaux et de polynômes le quotient \( A[X]/(P)\) est-il un corps ?
\end{enumerate}

%///////////////////////////////////////////////////////////////////////////////////////////////////////////////////////////
\subsubsection{Non classées}
%///////////////////////////////////////////////////////////////////////////////////////////////////////////////////////////

\begin{enumerate}
    \item
        Est-ce qu'il existe une structure raisonnable d'espace vectoriel sur \( \eZ\) ? Est-ce qu'il existe des corps discrets infinis ?

        Dans cette question, j'ai derrière la tête que dans un espace vectoriel topologique nous avons une notion de suite de Cauchy, définition~\ref{DefZSnlbPc}. Donc dans ce cas la notion d'espace complet est une notion topologique. Or il y a l'exemple~\ref{EXooNMNVooXyJSDm} qui donne deux distances sur \( \eN\), qui donnent la même topologique, mais l'un étant complet, l'autre non.

        Si il y avait une structure vectorielle sur \( \eN\), cela créerait une contradiction. Au moins au sens où la définition~\ref{DefZSnlbPc} de suite de Cauchy «topologique» ne redonne pas la même que la notion «métrique» de la définition usuelle~\ref{DEFooXOYSooSPTRTn}.
    \item
        La «décomposition en facteurs premiers» dans \( \eZ[i\sqrt{2}]\) que je donne dans l'exemple~\ref{ExluqIkE} est-elle correcte ? En particulier le lemme~\ref{LemTScCIv} ?
    \item
        Est-ce que la fin de la démonstration~\ref{ThojCJpFW} avec cette histoire d'ensemble \( \{ \xi_k^q\tq q\in \eN \}\) fini est compréhensible ?
    \item
        Les représentations \emph{irréductibles} sont les modules \emph{indécomposables}. Quid des modules irréductibles ? C'est pas un peu dingue de ne pas utiliser le mot «irréductible» pour désigner les mêmes choses dans le cas des modules et celui des représentations ?
    \item
        Rendre rigoureuse la remarque \eqref{RemmQjZOA} qui dit que les matrices dont le polynôme minimal est égal au polynôme caractéristique sont denses dans les matrices.
    \item
        La partie initiation de récurrence (\( r=2\)) de la preuve de la proposition~\ref{PropSVvAQzi} à propos de convexe et de barycentre est-elle correcte ? Ce passage de l'espace affine à l'espace vectoriel sous-jacent me paraît un peu facile.
    \item
        Est-ce que l'énoncé et la démonstration de la proposition~\ref{PropyMTEbH} sont corrects ? Si \( a\) et \( b\) sont des racines de \( P\), alors \( \mu_a\mu_b\) divise \( P\) (si \( \mu_a\neq \mu_b\)). Cette proposition est utilisée dans la démonstration de l'irréductibilité des polynômes cyclotomiques (proposition~\ref{PropoIeOVh}).
    \item
        À quoi sert l'hypothèse «autre que \( \eF_2\)» dans le lemme~\ref{LemcDOTzM} ? Peut-être dans la notion de déterminant parce qu'en caractéristique \( 2\), l'antisymétrie d'une forme linéaire n'implique le fait qu'elle soit alternée.
    \item
        L'inversibilité de la somme de Gauss (proposition~\ref{PropciRUov}) est-elle bien démontrée ?
    \item
        Des commentaires sur l'exemple~\ref{ExfUqQXQ} qui montre que \( X^p-X+1\) est irréductible sur \( \eF_p\).
    \item
        Les idéaux de \( A/I\) sont en bijection avec les idéaux de \( A\) contenant \( I\). Justification de l'équation \eqref{EqKbrizu}.
    \item
        À propos d'extensions algébriques, est-ce que la proposition~\ref{PropURZooVtwNXE} est correcte ? Est-ce qu'implicitement, il n'y a pas un sur-corps de \( \eK\) dans lequel il faut travailler ?
    \item
        À propos de construction à la règle et au compas. Pour l'addition d'angles, l'exemple~\ref{ExOVDooXnWPDl} explique comment on construit la somme de deux angles. Le problème est que cette construction se fait par intersection de deux cercles. Une des deux intersections donne \( \alpha+\beta\) et l'autre donne \( \alpha-\beta\). Comment par construction peut-on choisir le bon point ?
    \item
        À propos de chiffrement RSA, quelle est la probabilité que le message \( M\) ne soit pas premier avec \( p\) ? Est-ce que Alice (qui est celle qui chiffre avec la clef de Bob) peut le vérifier ? Que penser des points que j'énumère à la page \pageref{PageAKTBooMDeQxY} au dessus du problème~\ref{ProbGAYFooZATuYy} ?
    \item
        Isomorphisme du corps \( \eR\). Que penser de la remarque~\ref{REMooGHEDooOYYUPk} ?
\end{enumerate}

%---------------------------------------------------------------------------------------------------------------------------
\subsection{Mes questions d'analyse}
%---------------------------------------------------------------------------------------------------------------------------

%///////////////////////////////////////////////////////////////////////////////////////////////////////////////////////////
\subsubsection{Facile}
%///////////////////////////////////////////////////////////////////////////////////////////////////////////////////////////


%///////////////////////////////////////////////////////////////////////////////////////////////////////////////////////////
\subsubsection{Moyen}
%///////////////////////////////////////////////////////////////////////////////////////////////////////////////////////////

\begin{enumerate}
    \item
        Est-ce que l'énoncé du théorème de Müntz~\ref{ThoAEYDdHp} est correct ? Voir en particulier la remarque~\ref{REMooGPYYooCQJwFa} à propos de la présence du monôme  \( 1\) dans la liste.
    \item
        À propos de formule sommatoire de Poisson, est-ce que l'exemple~\ref{ExDLjesf} est bien fait ? En particulier la formule \eqref{EqjrNxLr} est-elle correcte et bien justifiée ?
    \item
        Que penser de la remarque~\ref{RemfdJcQF} qui dit qu'on doit avoir un théorème de complétion de partie orthonormale en une base orthonormale pour un espace de Hilbert ? C'est vrai ?
    \item
        Préciser l'énoncé et donner une démonstration de la proposition~\ref{PropMpBStL} qui traite de sommes dénombrables.
    \item
        Explosion en temps fini. C'est le corolaire~\ref{CorGDJQooNEIvpp}. Dans le cas où \( \lim_{t\to t_{max}} \| y(t) \|=\infty\) alors la dérivée de \( y\) n'est pas non plus bornée. Correct ?
    \item
        Est-ce qu'il y a moyen de définir la mesure produit (de deux espaces mesurés) sans passer par l'intégrale ? 

        Le théorème-définition \ref{ThoWWAjXzi} donne le produit de mesures par la formule
        \begin{equation}  
            (\mu_1\otimes \mu_2)(A)=\int_{\Omega_1}\mu_2\big( A_2(x) \big)d\mu_1(x)=\int_{\Omega_2}\mu_1\big( A_1(y) \big)d\mu_2(y).
        \end{equation}
        On peut faire plus abstrait ?

        Sans une telle définition, l'ordre imposé est :
        \begin{itemize}
            \item mesure
            \item intégrale
            \item mesure produit.
        \end{itemize}
        En particulier, quand on voit l'intégrale, la mesure de Lebesgue sur \( \eR^n\) n'est pas encore définissable.

        Il serait bien de pouvoir faire :
        \begin{itemize}
            \item mesure
            \item mesure de Lebesgue sur \( \eR\)
            \item mesure produit
            \item mesure de Lebesgue sur \( \eR^n\)
            \item intégrale.
        \end{itemize}

    \item
        La proposition \ref{PROPooMXCDooBffXbl} prouve que la fonction \( x\mapsto a^x\) (\( a>0\)) est dérivable. Pour ce faire, le concept de primitive est utilisé (pas jusqu'aux intégrales, cependant). Ça me semble incroyable qu'il faille ça pour prouver une dérivabilité.

        Prouver que la limite
        \begin{equation}
            \lim_{\epsilon\to 0}\frac{ a^{\epsilon}-1 }{ \epsilon }
        \end{equation}
        existe et n'est pas infinie sans recourir aux intégrales, aux primitives.
\end{enumerate}

%///////////////////////////////////////////////////////////////////////////////////////////////////////////////////////////
\subsubsection{Difficile}
%///////////////////////////////////////////////////////////////////////////////////////////////////////////////////////////

\begin{enumerate}
    \item
        Soit \( n\in \eN\), \( A\in \eR\) et \( x_0\in \eQ\). Nous considérons la suite\cite{BIBooMPXEooQLKhku}
        \begin{equation}
            x_{k+1}=\frac{1}{ n }\left( (n+1)x_k+\frac{ A }{ x_k^{n-1} } \right).
        \end{equation}
        Prouver que:
        \begin{itemize}
            \item C'est une suite de Cauchy
            \item \( x_k^n\to A\).
        \end{itemize}
        Il me faudrait une démonstration de cela sans passer par la méthode de Newton. Par exemple via le binôme de Newton.

        Mon but serait de définir \( a^{1/n}\) pour tout \( n\in \eN\) sans passer par de l'analyse (ou en tout cas pas en passant par le concept de fonction continue). Pour l'instant, c'est la définition \ref{DEFooJWQLooWkOBxQ} qui définit \( a^{1/n}\). Cela se base sur un argument de fonction continue strictement croissante pour obtenir une bijection.
    \item
        Est-ce que la proposition~\ref{PropooUEEOooLeIImr} qui donne le critère \( d(x_p,x_q)\leq \epsilon\) pour être une suite de Cauchy est valide dans un espace topologique métrique au lieu de normé ?  Dans quels cas a-t-on
        \begin{equation}
            d(a,b)=d(a+u)+d(b+u)
        \end{equation}
        lorsque \( d\) est une distance qui n'est pas spécialement induite d'une norme ?
    \item
        Soit \( V\), un espace vectoriel normé. Soit \( v\in V\). Est-ce qu'il existe un élément \( \varphi\in V'\) (application linéaire continue) telle que \( \varphi(v)=1\) et \( \| \varphi \|=1\) ?
    \item
        Que penser de~\ref{NORMooXTGBooKDnAhZ} qui tente d'expliquer pourquoi on ne définit pas l'intégrale d'une fonction non mesurable, malgré que le supremum qui la définirait existe forcément ?
    \item
        Changement de variables. À quel point la proposition \ref{PROPooILOEooBiumKD} est-elle équivalente au théorème usuel ?
\end{enumerate}

%///////////////////////////////////////////////////////////////////////////////////////////////////////////////////////////
\subsubsection{Non classées}
%///////////////////////////////////////////////////////////////////////////////////////////////////////////////////////////

\begin{enumerate}
    \item
        À propos de suites de Cauchy dans un espace vectoriel topologique et dans un espace métrique, est-ce que le théorème~\ref{THOooGQZSooAmQolf} est correct ?

        Soit \( V\) un espace vectoriel topologique métrisable\footnote{i.e. admet une base dénombrable de topologique, voir la proposition~\ref{PROPooPRLBooGtsRjr}}, alors il admet une métrique \( d\) compatible avec la topologie telle que une suite dans \( V\) est \( d\)-Cauchy si et seulement si elle est \( \tau\)-Cauchy.

        Dans cet ordre d'idée, il faut des exemples de :
        \begin{itemize}
            \item un espace vectoriel topologique métrisable et une métrique \( d\) compatible avec la topologie, mais dont les suites \( d\)-Cauchy ne sont pas celles \( \tau\)-Cauchy. Et en particulier dont la complétude est différente que celle de la «bonne» métrique donnée par la théorème~\ref{THOooAGBXooZnvQLK}.
            \item Et aussi un exemple pour la remarque~\ref{REMooUFQYooUVCCjs}.
        \end{itemize}
    \item
        L'exemple~\ref{ExfYXeQg} parle d'inverser une intégrale et une dérivée au sens des distributions pour prouver que la dérivée de \( \int_0^xg(t)dt\) par rapport à \( x\) est \( g\). Rendre cela rigoureux.
    \item
        À propos du théorème de récurrence de Poincaré~\ref{ThoYnLNEL}, l'application \( \phi\) doit être mesurable ? Répondre à la question posée sur la page de discussion de \wikipedia{fr}{Théorème_de_récurrence_de_Poincaré}{l'article sur wikipédia}.

        Toujours à propos du théorème de récurrence de Poincaré, il me semble qu'il y a un énoncé qui insiste sur la compacité de l'espace des phases et une démonstration utilisant la propriété de sous-recouvrement fini. Je serais content de retrouver cela. (ce serait sans doute mettable dans la leçon sur l'utilisation de la compacité)
        \item
            Dans \cite{OEVAuEz}, on parle de la proposition~\ref{PropZMKYMKI} à sa page \( 10\). Comment est-ce qu'on justifie le passage
            \begin{equation}
                \int_{\eR^d}T\big( y\mapsto \varphi(x)\psi(x-y) \big)dx=T\Big( y\mapsto\int_{\eR^d}\varphi(x)\psi(x-y)dx \Big).
            \end{equation}
            Sylvie Benzoni précise que «ceci demanderai quelques justifications». Où trouver lesdites justifications ? Il s'agit de permuter une distribution et une intégrale.

    \item
        Peut-on permuter une application linéaire et continue avec une somme pas spécialement dénombrable ? En supposant que \( \sum_{i\in I}f(v_i)\) existe, la proposition~\ref{PROPooWLEDooJogXpQ} semble dire que oui. Est-ce correct ?

        Peut-on avoir un exemple de partie sommable \( \{ v_i \}_{i\in I}\) et d'application linéaire continue \( f\) telle que la partie \( \{ f(v_i) \}\) ne soit pas sommable ?

        Peut-être ceci dans un espace de Hilbert.  \( v_i=\frac{1}{ i^2 }e_i\) puis \( f(e_i)=i^3e_i\).

    \item
        Soit une partie orthonormale \emph{pas spécialement dénombrable} \( \{ u_i \}_{i\in I}\) d'un espace de Hilbert (pas spécialement séparable). Si
        \begin{equation}
            x=\sum_{i\in I}x_iu_i,
        \end{equation}
        puis-je prendre le produit scalaire avec \( u_{k}\) et le permuter avec la somme pour déduire que \( x_k=\langle x, u_k\rangle \) ?

        C'est ce que je fais dans la proposition~\ref{PROPooWTOZooYZdlml}.
    \item
        Théorème de point fixe et équation différentielle. Que penser de l'exemple~\ref{EXooJXIGooQtotMc} qui itère la contraction de Cauchy-Lipschitz pour résoudre \( y'(t)=y(t)\), \( y(0)=1\) ? Est-ce que c'est générique comme comportement ? Est-ce que la convergence est efficace dans des cas moins triviaux ?
\end{enumerate}

%--------------------------------------------------------------------------------------------------------------------------- 
\subsection{Mes questions d'analyse numérique}
%---------------------------------------------------------------------------------------------------------------------------

\begin{enumerate}
    \item
        Différences finies. Il faut une analyse de consistance, stabilité et convergence du schéma à \( 9\) points pour le laplacien donné par \eqref{EQooKUMVooCVrzjt}. Je crois qu'il est d'ordre \( 6\), mais je n'en suis vraiment pas sûr.
\end{enumerate}

%---------------------------------------------------------------------------------------------------------------------------
\subsection{Mes questions de probabilité et statistiques.}
%---------------------------------------------------------------------------------------------------------------------------

%///////////////////////////////////////////////////////////////////////////////////////////////////////////////////////////
\subsubsection{Facile}
%///////////////////////////////////////////////////////////////////////////////////////////////////////////////////////////

%///////////////////////////////////////////////////////////////////////////////////////////////////////////////////////////
\subsubsection{Moyen}
%///////////////////////////////////////////////////////////////////////////////////////////////////////////////////////////

\begin{enumerate}
    \item
        Soit une variable aléatoire \( X\) à valeurs réelles. Est-ce que la tribu engendrée par \( X\) est d'une façon ou d'une autre engendrée par les «courbes de niveau» de \( X\) ? C'est-à-dire par les \( X^{-1}\big( \{ t \} \big)\) pour les \( t\in \eR\).

        C'est ce qui semble ressortir de l'exemple de~\ref{SUBSECooWOOGooVxflVZ}. Et intuitivement, je trouve que ça irait bien \ldots
\end{enumerate}


%///////////////////////////////////////////////////////////////////////////////////////////////////////////////////////////
\subsubsection{Difficile}
%///////////////////////////////////////////////////////////////////////////////////////////////////////////////////////////

%///////////////////////////////////////////////////////////////////////////////////////////////////////////////////////////
\subsubsection{Non classées}
%///////////////////////////////////////////////////////////////////////////////////////////////////////////////////////////

%---------------------------------------------------------------------------------------------------------------------------
\subsection{Mes questions de \LaTeX\ et programmation}
%---------------------------------------------------------------------------------------------------------------------------

%///////////////////////////////////////////////////////////////////////////////////////////////////////////////////////////
\subsubsection{Facile}
%///////////////////////////////////////////////////////////////////////////////////////////////////////////////////////////

\begin{enumerate}
    \item
        Comment faire en sorte que les mots commençant par «é» soient avec les «e» dans l'index, et non avant les «a» ? Il me faudrait quelque chose de plus automatique que faire \info{machin@truc}.
    \item 
        Il y a des problèmes dans la table des matières.  « Table des matières », « Index », et « Liste des notations » ne pointent pas vers la bonne page.
\end{enumerate}

%///////////////////////////////////////////////////////////////////////////////////////////////////////////////////////////
\subsubsection{Moyen}
%///////////////////////////////////////////////////////////////////////////////////////////////////////////////////////////

\begin{enumerate}
    \item
        Revoir le mécaniste de l'index thématique. Il faudrait pouvoir les trier avec des titres. Mais attention : il doit arriver avant la table des matières.
\end{enumerate}

%///////////////////////////////////////////////////////////////////////////////////////////////////////////////////////////
\subsubsection{Difficile}
%///////////////////////////////////////////////////////////////////////////////////////////////////////////////////////////

Je ne sais pas comment faire, et à mon avis il faudra innover.

\begin{enumerate}
    \item
        Si vous savez comment faire \info{pdf --> epub} pour créer un eBook, faites le moi savoir. Cahier des charges :
        \begin{itemize}
            \item libre, disponible sur Ubuntu
            \item en ligne de commande (en tout cas : exécutable depuis un script en python ou C++)
        \end{itemize}
        Attention : le Frido étant un truc assez compliqué, avant de répondre la première chose qui vous passe par la tête, assurez-vous que votre solution fait avancer les choses sur le Frido et non sur un petit document de test.

        Nous en avons déjà un peu discuté sur \url{https://github.com/LaurentClaessens/mazhe/issues/13}. Il faudra entre autres faire un script qui remplace tous les environnements tizk des fichiers \info{*.pstricks} (désolé pour la convention de nommage historique) par un simple \info{includegraphics} du fichier \info{pdf} correspondant que l'on trouvera dans le répertoire \info{auto/pictures\_tikz}.
    \item
        Écrire un script (en python ou autre) qui prend en argument deux numéros ou noms de chapitres et qui retourne l'ensemble des lignes du premier qui contient des \info{ref} ou \info{eqref} dont le label correspondant est dans le second.

        Attention : il faut tenir compte de \info{input} de façon récursive.

        Bonus : calculer le hash sha1 de chaque ligne du résultat et ne pas l'afficher si il se trouve dans la liste du fichier \info{commons.py}.
\end{enumerate}

%--------------------------------------------------------------------------------------------------------------------------- 
\subsection{Numérique}
%---------------------------------------------------------------------------------------------------------------------------

%///////////////////////////////////////////////////////////////////////////////////////////////////////////////////////////
\subsubsection{Moyen}
%///////////////////////////////////////////////////////////////////////////////////////////////////////////////////////////

\begin{enumerate}
    \item
        L'erreur de cancellation provoquée par la différence \( a-\tilde a\) lorsque \( a\) et \( \tilde a\) sont proches n'a pas de conséquences sur l'ordre de grandeur de la réponse. Seulement des conséquences sur la valeur des chiffres significatifs. Vrai ou faux ?

        Voir la remarque~\ref{REMooRQIJooNLdAZE}.
\end{enumerate}

%+++++++++++++++++++++++++++++++++++++++++++++++++++++++++++++++++++++++++++++++++++++++++++++++++++++++++++++++++++++++++++ 
\section{Taper du code pour le Frido}
%+++++++++++++++++++++++++++++++++++++++++++++++++++++++++++++++++++++++++++++++++++++++++++++++++++++++++++++++++++++++++++

Dans cette section nous donnons quelques indications sur la façon de taper du code \LaTeX\ pour le Frido.

Tout commence par télécharger les sources à l'adresse
\begin{center}
    \url{https://github.com/LaurentClaessens/mazhe}
\end{center}

%--------------------------------------------------------------------------------------------------------------------------- 
\subsection{Pour compiler le document vous même}
%---------------------------------------------------------------------------------------------------------------------------

Lisez le fichier \info{COMPILATION.md}.

%---------------------------------------------------------------------------------------------------------------------------
\subsection{Nommage des fichiers \info{tex}}
%---------------------------------------------------------------------------------------------------------------------------

J'ai pris l'habitude de préfixer les noms par un nombre. Par exemple \info{139\_EspacesVecto}. Le fait est qu'il est plus simple, pour ouvrir le ficher, de taper \info{139<TAB>} que de se souvenir si «EspacesVecto» est écrit avec une majuscule, en français, en anglais, \ldots

De plus un chapitre contenant plusieurs fichiers, nous nous retrouvons rapidement avec beaucoup de fichiers nommés \info{EspacesVecto1}, \info{EspacesVecto2}, etc.

Vous trouverez le prochain numéro disponible dans \info{réserve.tex}.

%---------------------------------------------------------------------------------------------------------------------------
\subsection{Inclure des exemples de code}
%---------------------------------------------------------------------------------------------------------------------------

Pour inclure du code Sage, nous utilisons la commande \info{\textbackslash lstinputlisting}. Ici encore, le fichier \info{réserve.tex} contient le prochain disponible.

%---------------------------------------------------------------------------------------------------------------------------
\subsection{Pour les exercices}
%---------------------------------------------------------------------------------------------------------------------------

ATTENTION : dans un futur proche, je vais supprimer tous les exercices et les mettre sous forme d'exemples. Une des raisons est de supprimer la dépendance en le paquet personnel \info{exocorr} qui rend compliqué la compilation du Frido par des tierces personnes.

\vspace{1cm}

Les exercices sont tapés dans les fichiers déjà pré-remplis \info{src\_exocorr/exo*.tex}. Les corrections sont dans le fichier \info{src\_exocorr/corr*.tex} correspondant. Ces fichiers ne sont pas inclus directement, mais via la macro \info{\textbackslash Exo}.

Le fichier \info{réserve.tex} contient le prochain disponible.

\paragraph{Exemple}

Vous voulez créer un exercice.
\begin{itemize}
    \item Allez voir dans \info{réserve.tex} la prochaine ligne \info{Exo} disponible.
    \item Mettons que ce soit   \info{\textbackslash Exo\{mazhe-0018\}}
    \item Supprimez cette ligne de \info{réserve.tex}, et mettez la où vous voulez voir paraitre votre exercice.
    \item Tapez votre exercice dans le fichier \info{src\_exocorr/exomazhe-0018.tex} et votre correction dans le fichier \info{src\_exocorr/corrmazhe-0018.tex}. Ces fichiers sont déjà créés et pré-remplis. Ne changez pas le code qui y est.
\end{itemize}

%+++++++++++++++++++++++++++++++++++++++++++++++++++++++++++++++++++++++++++++++++++++++++++++++++++++++++++++++++++++++++++
\section{Les politiques éditoriales}
%+++++++++++++++++++++++++++++++++++++++++++++++++++++++++++++++++++++++++++++++++++++++++++++++++++++++++++++++++++++++++++

Certaines parties de ce texte ne respectent pas les politiques éditoriales. Ce sont des erreurs de jeunesse, et j'en suis le premier triste.

%---------------------------------------------------------------------------------------------------------------------------
\subsection{Licence libre}
%---------------------------------------------------------------------------------------------------------------------------

Je crois que c'est clair.

%---------------------------------------------------------------------------------------------------------------------------
\subsection{pdflatex}
%---------------------------------------------------------------------------------------------------------------------------

Tout est compilable avec pdf\LaTeX. Pas de pstricks, de psfrag ou de ps<quoiquecesoit>.

%---------------------------------------------------------------------------------------------------------------------------
\subsection{utf8}
%---------------------------------------------------------------------------------------------------------------------------

Je crois que c'est clair.

%---------------------------------------------------------------------------------------------------------------------------
\subsection{Notations}
%---------------------------------------------------------------------------------------------------------------------------

On essaie d'être cohérent dans les notations et les conventions. Pour la transformée de Fourier par exemple, je crois que la définition du produit scalaire dans \( L^2\), des coefficients de Fourier, de la transformation et de la transformation inverse sont cohérents. Cela demande, lorsqu'on suit un livre qui ne suit pas les conventions utilisées ici, de convertir parfois massivement.

%---------------------------------------------------------------------------------------------------------------------------
\subsection{De la bibliographie}
%---------------------------------------------------------------------------------------------------------------------------

On évite d'écrire en haut de chapitre «les références pour ce chapitre sont \ldots». Il est mieux d'écrire au niveau des théorèmes, entre parenthèses, les références.

Lorsqu'on écrit l'énoncé d'un théorème sans retranscrire la démonstration, il faut mettre une référence vers un document \emph{en ligne} qui en contient la preuve. Il est vraiment fastidieux de chercher une preuve sur internet et de tomber sur des dizaines de documents qui donnent l'énoncé mais pas la preuve.

%---------------------------------------------------------------------------------------------------------------------------
\subsection{Faire des références à tout}
%---------------------------------------------------------------------------------------------------------------------------

Lorsqu'un utilise le théorème des accroissements finis, il ne faut pas écrire «d'après le théorème des accroissements finis, blablabla». Il faut écrire un \verb+\ref+ explicite vers le résultat. Cela alourdit un peu le texte, mais lorsqu'on joue avec un texte de plus de 2000 pages, il est parfois laborieux de trouver le résultat qu'on cherche (surtout s'il existe plusieurs versions d'un résultat et que l'on veut faire référence à une version particulière).

%---------------------------------------------------------------------------------------------------------------------------
\subsection{Des listes de liens internes}
%---------------------------------------------------------------------------------------------------------------------------

Le début du Frido contient une espèce d'index thématique. Il serait bon de l'étoffer.

%---------------------------------------------------------------------------------------------------------------------------
\subsection{Pas de références vers le futur}
%---------------------------------------------------------------------------------------------------------------------------

Dans le Frido, \emph{aucune} preuve ne peut faire une référence vers un résultat prouvé plus bas. On n'utilise pas le théorème 10 dans la démonstration du théorème 7. Cela est une contrainte forte sur le découpage en chapitres et sur l'ordre de présentation des matières.

Il est bien entendu accepté et même encouragé de mettre des notes du type «Nous verrons plus loin un théorème qui \ldots». Tant que ce théorème n'est pas \emph{utilisé}, ça va.

En faisant
\begin{quote}
    \begin{verbatim}
    pytex lst_frido.py --verif
    \end{verbatim}
\end{quote}
vous aurez une liste des références vers le bas. Cette liste doit être vide ! Ce programme cherche tous les \verb+\ref+ et \verb+\eqref+ ainsi que les \verb+\label+ correspondants et vous prévient lorsque le \verb+\label+ est après le \verb+\ref+.

Si vous pensez qu'une référence pointée doit être acceptée (par exemple parce c'est dans une des listes de liens internes), alors vous ajoutez son hash dans la liste du fichier \info{commons.py}. Si il s'agit vraiment d'une référence vers un résultat que vous utilisez, alors vous devez déplacer des choses. Soit votre résultat vers le bas, soit celui que vous utilisez vers le haut. Parfois cela demande de déplacer ou redécouper des chapitres entiers\ldots\ Si il n'y a vraiment pas moyen, bravo, vous venez de prouver que la mathématique est logiquement inconsistante.

%--------------------------------------------------------------------------------------------------------------------------- 
\subsection{Écriture inclusive}
%---------------------------------------------------------------------------------------------------------------------------

Je suis triste de devoir le préciser, mais le Frido est écrit en français. Nous n'utiliserons donc pas de féminisation abusives, et accepterons comme correcte des tournures comme, en parlant d'une fonction, «\emph{elle} est \emph{un} contre-exemple», ou en parlant d'un lemme que «\emph{il} est \emph{une} conséquence».

Parfois le genre d'un objet n'est pas bien défini. Par exemple \( 3/4\) est \emph{la} classe d'équivalence de \( (3,4)\) dans \( \eZ\times \eZ\setminus{0}\); mais en même temps c'est \emph{un} élément de \( \eQ\). Nous utiliserons alors, prudemment, un neutre en disant «\emph{il} est plus petit que \( 1\)».

%+++++++++++++++++++++++++++++++++++++++++++++++++++++++++++++++++++++++++++++++++++++++++++++++++++++++++++++++++++++++++++
\section{Vérifier si vous n'avez pas fait de bêtises}
%+++++++++++++++++++++++++++++++++++++++++++++++++++++++++++++++++++++++++++++++++++++++++++++++++++++++++++++++++++++++++++

Lorsqu'on fait de lourdes modifications (déplacement de parties, fusion de théorèmes, etc) il est toujours possible de faire des bêtises d'au moins deux types : créer des références vers le futur et supprimer des parties (genre couper-coller en oubliant le coller). Pour s'en prémunir, le script suivant lance quelques compilations et vérifications :

\begin{verbatim}
./testing.sh
\end{verbatim}
Aucune erreur ne devrait être signalée.

Attention : ce script fait quelques manipulations à base de \info{git stash} et crée une nouvelle branche (nom aléatoire assez long) pour tester votre dernière modification sans créer de commit.

%+++++++++++++++++++++++++++++++++++++++++++++++++++++++++++++++++++++++++++++++++++++++++++++++++++++++++++++++++++++++++++ 
\section{Acceptation des contribution}
%+++++++++++++++++++++++++++++++++++++++++++++++++++++++++++++++++++++++++++++++++++++++++++++++++++++++++++++++++++++++++++

TD;DR : pratiquement aucun patch n'est refusé.

Le premier critère d'acceptation d'une contribution est évidemment la correction mathématique.

%--------------------------------------------------------------------------------------------------------------------------- 
\subsection{Attention aux expressions rationnelles}
%---------------------------------------------------------------------------------------------------------------------------

Si vous trouvez une faute d'orthographe, rien ne vous empêche de faire une recherche de la même faute pour la corriger d'un seul coup dans \( 25\) fichiers. Faites toutefois attention à des remplacements automatiques sur base d'expressions rationnelles telles que
\begin{verbatim}
    des [a-z]*[^s]
\end{verbatim}
qui serait supposé détecter des erreurs de pluriel. Je vous laisse trouver au moins \( 5\) cas sans fautes qui satisfont cette expression.

Utilisez de telles expressions pour \emph{trouver} des fautes, pas pour les corriger.

%--------------------------------------------------------------------------------------------------------------------------- 
\subsection{Pas de modifications massives, automatiques pour des raisons cosmétiques}
%---------------------------------------------------------------------------------------------------------------------------

Il est un type de contributions que je ne vais plus accepter, ce sont les modifications massives et automatiques de \emph{tous} les fichiers pour des raisons de «propreté» du code. Exemples : 
\begin{itemize}
    \item Supprimer automatiquement tous les espaces en bout de lignes,
    \item Supprimer automatiquement toutes les lignes vides en fin de fichier
    \item Remplacer \info{\textbackslash ref} par \info{\textasciitilde\textbackslash ref}
    \item Remplacer \info{\textbackslash [} par \info{\textbackslash begin\{equation\}}
    \item Remplacer \info{\textbackslash og} par \info{«} (avec ou sans espaces devant ou derrière)
    \item \ldots
\end{itemize}

Ce type de substitutions automatiques créent des patch gigantesques qui prennent un temps astronomique à relire pour un bénéfice pas tellement évident.

Pire : ils ont des effets de bords pas toujours évident à détecter ou à prévoir.

N'oubliez pas que le Frido n'est pas que du \LaTeX. Il est aussi divers script de pré et post-compilation (y compris qui hackent des fichiers intermédiaires entre deux passes de \LaTeX). Ces scipts, écrits par votre très humble et très obéissant serviteur, ne sont pas parfaits et les parseurs reposent sur certaines hypothèses. Donc des choses qui ne devraient ne rien changer du point de vue de \LaTeX\ peuvent avoir des conséquences.

Bien entendu, si vous êtes en train de taper des math et que ce genre de «malpropreté» du code vous gêne, vous pouvez corriger dans les fichiers que vous modifiez.
