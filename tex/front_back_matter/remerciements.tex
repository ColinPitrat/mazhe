%+++++++++++++++++++++++++++++++++++++++++++++++++++++++++++++++++++++++++++++++++++++++++++++++++++++++++++++++++++++++++++
\section{Auteurs, contributeurs, sources et remerciements}
%+++++++++++++++++++++++++++++++++++++++++++++++++++++++++++++++++++++++++++++++++++++++++++++++++++++++++++++++++++++++++++

Les remerciements, dans chaque catégorie, sont mis dans l'ordre chronologique approximatif. Les noms en couvertures sont ceux qui ont fourni du code \LaTeX\ (typiquement : un patch via github), par ordre chronologique approximatif d'entrée dans le projet.

%--------------------------------------------------------------------------------------------------------------------------- 
\subsection{Ceux qui ont travaillé sur le Frido}
%---------------------------------------------------------------------------------------------------------------------------

\begin{enumerate}
    \item
        Carlotta Donadello pour l'ensemble du cours de CTU de géométrie analytique 2010-2011. Une grosse partie de «analyse réelle» vient de là.
    \item
        Les étudiants de géométrie analytique en CTU 2010-2011 ont détecté d'innombrables coquilles. Les étudiants du cours présentiel de géométrie analytique 2011-2012 ont signalé un certain nombre d'incorrections dans les exercices et les corrigés. Les agrégatifs de Besançon 2011-2012 pour leurs plans et leurs développements.
    \item
        Lilian Besson pour m'avoir signalé un paquet de fautes, et quelques points pas clairs en statistiques.
    \item
        Plouf qui m'a signalé une coquille dans le fil \href{http://passeurdesciences.blog.lemonde.fr/2014/01/24/la-selection-scientifique-de-la-semaine-numero-106}{la-selection-scientifique-de-la-semaine-numero-106}.
    \item
        Benjamin de block pour des coquilles et une mise au point sur les conventions à propos de \( \eR^+\) et \( (\eR^+)^*\).
    \item
        Olivier Garet pour avoir répondu à plein de questions de probabilités.
    \item
        François Gannaz pour de la relecture et une version plus claire de la preuve (et de l'énoncé) de la proposition \ref{PROPooIOQEooGMcCJm}.
    \item
        Danarmk pour des réponses à des questions \url{http://linuxfr.org/nodes/110155/comments/1675589}.
    \item
        Cédric Boutilier pour des réponses à des questions de probabilité statistique. \url{https://github.com/LaurentClaessens/mazhe/issues/16}
    \item
        Remsirems pour des réponses à des questions d'analyse. \url{http://linuxfr.org/nodes/110155/comments/1675813}
    \item
        Bertrand Desmons pour plusieurs patchs rendant plus clairs de nombreux passages sur les suites de Cauchy dans \( \eQ\).
    \item
        Anthony Ollivier pour m'avoir fait remarquer qu'il n'est pas vrai que \( A[X]\) est euclidien lorsque \( A\) est intègre (contre-exemple : \( A=\eZ\)). Ça fait une faute de moins dans le Frido.
    \item
        ybailly pour avoir détecté un bon nombre de coquilles dans la partie sur les ensembles de nombres.
    \item
        Éric Guirbal pour le remplacement de \info{frenchb} par \info{french}.
    \item
        cdrcprds pour une réponse à une question d'algèbre, démonstration à l'appui à propos de \href{https://github.com/LaurentClaessens/mazhe/issues/52#issuecomment-333251728}{pgcd}.
    \item
        Antoine Bensalah pour avoir répondu à une question sur Lax-Milgram tout en même temps que pointé une erreur dans la démonstration et fourni l'exemple \ref{EXooTTBDooUNhBOc} sur l'optimalité de l'inégalité.
    \item
        Guillaume Deschamps pour ses remarques à propos du fait que le chapitre «constructions des ensembles» est très ardu.
\end{enumerate}

%--------------------------------------------------------------------------------------------------------------------------- 
\subsection{Aide directe, mais pas volontairement sur le Frido}
%---------------------------------------------------------------------------------------------------------------------------

\begin{enumerate}
    \item
        Plein de monde pour diverses contributions à des énoncés d'exercices. Pierre Bieliavsky pour les énoncés d'analyse numérique (MAT1151 à Louvain la Neuve 2009-2010). Jonathan Di Cosmo pour certaines corrections de MAT1151. François Lemeux, exercices sur les normes de matrices et correction de coquilles.  Martin Meyer et Mustapha Mokhtar-Kharroubi pour certains exercices du cours \emph{Outils mathématiques} (surtout ceux des DS et examens).
    \item
               Nicolas Richard et Ivik Swan pour les parties des exercices et rappels de calcul différentiel et intégral (Université libre de Bruxelles, 2003-2004) qui leurs reviennent.
            \item
                Carlotta Donadello pour la partie géométrie analytique : topologie dans \( \eR^n\), courbes, intégrales, limites. (Université de Franche-Comté 2010-2012)
    \item
        Le forum usenet de math, en particulier pour la construction des corps fini dans la fil « Vérifier qu'un polynôme est primitif » initié le 20 décembre 2011.
    \item
        Mihai Bostan nous a donné ses notes manuscrites de son cours présentiel de géométrie analytique 2009-2010. (Presque) Toute la structure du cours de géométrie analytique lui est due (qui est maintenant fondue un peu partout dans les chapitres d'analyse).
\end{enumerate}

%--------------------------------------------------------------------------------------------------------------------------- 
\subsection{Des gens qui ont fait un travail qui m'a bien servi}
%---------------------------------------------------------------------------------------------------------------------------

\begin{enumerate}
            \item 
                Arnaud Girand pour avoir mis ses développements bien faits en ligne. Une bonne vingtaine de résultats un peu partout dans ces notes viennent de lui.
            \item
                Le site \url{http://www.les-mathematiques.net} m'a donné les preuves de nombreux résultats.
            \item
                Pierre Monmarché pour son document en ligne tout plein de développements, et des réponses à des questions.
    \item
        Tous les contributeurs du Wikipédia francophone (et aussi un peu l'anglophone) doivent être remerciés. J'en ai pompé des quantités astronomiques; des articles utilisés sont cités à divers endroits du texte, mais ce n'est absolument pas exhaustif. Il n'y a à peu près pas un résultat important de ces notes dont je n'aie pas lu la page Wikipédia, et souvent plusieurs pages connexes.
    \item
        Les intervenants du fil «\href{http://www.les-mathematiques.net/phorum/read.php?2,302266}{Antisymétrisation, alterné, déterminant et caractéristique}» sur \texttt{les-mathematiques.net} m'ont bien aidé pour la section sur les déterminants \ref{SecGYzHWs} (bien qu'ils ne le savent pas).
    \item
        Xavier Mauquoy pour l'énoncé et la preuve du théorème \ref{THOooYXJIooWcbnbm}.
    \item
        David Revoy pour les dessins de Pepper\&Carrot \href{https://www.peppercarrot.com/fr/article285/episode-8-pepper-s-birthday-party}{de la couverture}.
\end{enumerate}

J'ai souvent donné entre parenthèse à côté des mots « théorème », « lemme » ou « proposition » une ou plusieurs références vers les sources de la preuve que je donne. Ce sont parfois des liens vers la bibliographie; parfois aussi des liens hypertexte vers des sites, des blogs, etc. Tous ces gens ont fait du bon boulot. Sans toute cette « communauté », l'internet serait mort\footnote{Cette dernière phrase doit être comprise comme un appel à ne pas utiliser Moodle et autres iCampus pour diffuser vos cours de math, ou en tout cas pas comme moyen exclusif.}.
