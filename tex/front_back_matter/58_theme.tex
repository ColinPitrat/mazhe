

\InternalLinks{probabilités et espérances conditionnelles}

    Les deux définitions de base, sur lesquelles se basent toutes les choses conditionnelles sont :
    \begin{itemize}
        \item La probabilité conditionnelle d'un événement en sachant un autre : \( P(A|B)\) de la définition~\ref{DEFooGJVHooVbhVYv}.
        \item L'espérance conditionnelle d'une variable aléatoire sachant une tribu : \( E(X|\tribF)\) de la définition~\ref{ThoMWfDPQ}.
    \end{itemize}

    Les autres sont listées ci-dessous.
\begin{description}

    \item[La probabilité conditionnelle d'un événement par rapport à un autre] donnée dans la proposition~\ref{DEFooGJVHooVbhVYv} est le nombre
\begin{equation}
    P(A|B)=\frac{ P(A\cap B) }{ P(B) }
\end{equation}

\item[La probabilité conditionnelle d'un événement vis-à-vis d'une variable aléatoire discrète] est par la définition~\ref{DEFooFRLFooNvXuPK} la variable aléatoire donnée par
\begin{equation}
    P(A|X)(\omega)=P(A|X=X(\omega)).
\end{equation}
Dans le cas continu, c'est la définition~\ref{DEFooIUJMooBAVtMW} :
\begin{equation}
    P(A|X)=P(A|\sigma(X))=E(\mtu_A|\sigma(X)).
\end{equation}

\item[L'espérance conditionnelle d'une variable aléatoire par rapport à une tribu] \( E(X|\tribF)\) est la variable aléatoire \( \tribF\)-mesurable telle que
\begin{equation}
    \int_BE(X|\tribF)=\int_BX
\end{equation}
pour tout \( X\in \tribF\). Si \( X\in L^2(\Omega,\tribA,P)\) alors \( E(X|\tribF)=\pr_K(X)\) où \( K\) est le sous-ensemble de \( L^2(\Omega,\tribA,P)\) des fonctions \( \tribF\)-mesurables (théorème~\ref{ThoMWfDPQ}). Cela au sens des projections orthogonales.

\item[La probabilité conditionnelle d'un événement par rapport à une tribu] est la variable aléatoire
\begin{equation}
    P(A|\tribF)=E(\mtu_A|\tribF).
\end{equation}

\item[L'espérance conditionnelle d'une variable aléatoire par rapport à une autre] de la définition~\ref{DefooKIHPooMhvirn} est une variation sur le thème :
\begin{equation}
    E(X|Y)=E(X|\sigma(Y)),
\end{equation}

%TODO : mettre cette définition à côté de celle du conditionnement par rapport à la tribu.

\end{description}

Notons que partout, si \( X\) est une variable aléatoire, la notation «sachant \( X\)» est un raccourcis pour dire «sachant la tribu engendrée par \( X\)».

