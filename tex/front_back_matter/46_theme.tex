
\InternalLinks{permuter des limites}
\begin{description}
    \item[Fonctions définies par une intégrale]
        Les théorèmes sur les fonctions définies par une intégrale, section~\ref{SecCHwnBDj}. Nous avons entre autres
        \begin{enumerate}
            \item
                \( \partial_i\int_Bf=\int_B\partial_if\), avec \( B\) compact, proposition~\ref{PropDerrSSIntegraleDSD}.
            \item
                Si \( f\) est majorée par une fonction ne dépendant pas de \( x\), nous avons le théorème~\ref{ThoKnuSNd}.
            \item
                Si l'intégrale est uniformément convergente, nous avons le théorème~\ref{ThotexmgE} qui donne la continuité de $F(x)=\int_{\Omega}f(x,\omega)d\mu(\omega)$.
            \item
                Pour dériver \( \int_Bg(t,z)dt\) avec \( B\) compact dans \( \eR\) et \( g\colon \eR\times \eC\to \eC\), il faut aller voir la proposition~\ref{PROPooZCLYooUaSMWA}.
            \item
                En ce qui concerne le \( x\) dans la borne, le théorème \ref{PropEZFRsMj} lie primitive et intégrale en montrant que \( F(x)=\int_a^xf(t)dt\) est une primitive de \( f\) (sous certaines conditions). Le théorème fondamental de l'analyse \ref{ThoRWXooTqHGbC} en est une conséquence.
        \end{enumerate}
    \item[Conegence monotone]
        Théorème de la convergence monotone, théorème~\ref{ThoRRDooFUvEAN}.
    \item[Fubini]
        Le théorème de Fubini permet non seulement de permuter des intégrales, mais également des sommes parce que ces dernières peuvent être vues comme des intégrales sur \( \eN\) muni de la tribu des parties et de la mesure de comptage\footnote{Mesure de comptage, définition \ref{DEFooILJRooByDzhs}.}. Nous utilisons cette technique pour permute une somme et une intégrale dans l'équation \eqref{EQooWOLOooFHSrsx}.
\begin{itemize}
    \item
        le théorème de Fubini-Tonelli~\ref{ThoWTMSthY} demande que la fonction soit mesurable et positive;
    \item
        le théorème de Fubini~\ref{ThoFubinioYLtPI} demande que la fonction soit intégrable (mais pas spécialement positive);
    \item
        le corollaire~\ref{CorTKZKwP} demande l'intégrabilité de la valeur absolue des intégrales partielles pour déduire que la fonction elle-même est intégrable.
\end{itemize}

\item[Limite et dérivées, différentielle]
    \begin{enumerate}
        \item
            Permuter limite et dérivée, théorème \ref{THOooXZQCooSRteSr}.
        \item
 Permuter limite et dérivées partielles, théorème \ref{ThoSerUnifDerr}.
        \item
            Permuter limite et différentielle, théorème \ref{ThoLDpRmXQ}.
    \end{enumerate}
    Quelque remarques sur les techniques de démonstration.
    \begin{enumerate}
        \item
            Le résultat fondamental \ref{THOooXZQCooSRteSr} est démontré sans recourrir à des intégrales. Une preuve alternative, plus courte, avec des intégrales est donnée en \ref{NORMALooGYUEooKrYjyz}.
        \item
            Les résultats un peu plus élaborés \ref{ThoSerUnifDerr} et \ref{ThoLDpRmXQ} sont prouvés avec des intégrales, mais devraient pouvoir être adaptés.
    \end{enumerate}
\item[Somme et dérivée]
    Permuter somme et différentielle, théorème \ref{ThoLDpRmXQ}.
\item[Limite et mesure]
    Une mesure n'est pas toujours une limite, mais la définition d'une mesure positive sur un espace mesurable parle de permuter limite et mesure : définition \ref{DefBTsgznn}\ref{ItemQFjtOjXiii}.

\end{description}
