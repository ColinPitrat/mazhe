
\InternalLinks{permuter des limites}
\begin{enumerate}
    \item
        Les théorèmes sur les fonctions définies par des intégrales, section~\ref{SecCHwnBDj}. Nous avons entre autres
        \begin{enumerate}
            \item
                \( \partial_i\int_Bf=\int_B\partial_if\), avec \( B\) compact, proposition~\ref{PropDerrSSIntegraleDSD}.
            \item
                Si \( f\) est majorée par une fonction ne dépendant pas de \( x\), nous avons le théorème~\ref{ThoKnuSNd}.
            \item
                Si l'intégrale est uniformément convergente, nous avons le théorème~\ref{ThotexmgE}.
            \item
                Pour dériver \( \int_Bg(t,z)dt\) avec \( B\) compact dans \( \eR\) et \( g\colon \eR\times \eC\to \eC\), il faut aller voir la proposition~\ref{PROPooZCLYooUaSMWA}.
        \end{enumerate}
    \item
        Théorème de la convergence monotone, théorème~\ref{ThoRRDooFUvEAN}.
    \item
        Le théorème de Fubini permet non seulement de permuter des intégrales, mais également des sommes parce que ces dernières peuvent être vues comme des intégrales sur \( \eN\) muni de la tribu des parties et de la mesure de comptage. Nous utilisons cette technique pour permute une somme et une intégrale dans l'équation \eqref{EQooWOLOooFHSrsx}.
\begin{itemize}
    \item
        le théorème de Fubini-Tonelli~\ref{ThoWTMSthY} demande que la fonction soit mesurable et positive;
    \item
        le théorème de Fubini~\ref{ThoFubinioYLtPI} demande que la fonction soit intégrable (mais pas spécialement positive);
    \item
        le corollaire~\ref{CorTKZKwP} demande l'intégrabilité de la valeur absolue des intégrales partielles pour déduire que la fonction elle-même est intégrable.
\end{itemize}

%TODO : des démonstrations de ces trois théorèmes seraient les bienvenues.

\end{enumerate}
