\InternalLinks{suite de Cauchy, espace complet}     \label{THMooOCXTooWenIJE}

Nous parlons d'espaces topologiques complets. À ne pas confondre avec un espace mesuré complet, définition \ref{DefBWAoomQZcI}.

\begin{enumerate}
    \item
        La définition \ref{DEFooXOYSooSPTRTn} donne la notion de suite de Cauchy dans un espace métrique.
    \item
        La définition \ref{DefZSnlbPc} donne la notion de suite de \( \tau\)-Cauchy dans un espace vectoriel topologique.
    \item
        Deux espaces métriques (avec une distance) peuvent être isomorphes en tant qu'espaces topologiques, mais ne pas avoir les mêmes suites de Cauchy, exemple \ref{EXooNMNVooXyJSDm}.
    \item
        La proposition \ref{PropooUEEOooLeIImr} donne l'équivalence entre les suites de Cauchy et les suites \( \tau\)-Cauchy dans le cas des espaces vectoriels topologiques \emph{normés}.
    \item
        L'exemple \ref{EXooNMNVooXyJSDm} est un exemple pire que simplement une suite de Cauchy qui ne converge pas. Le problème de convergence de cette suite n'est pas simplement que la limite n'est pas dans l'espace; c'est que la suite de Cauchy donnée ne convergerait même pas dans \( \eR\).
    \item
        Le théorème \ref{ThoKHTQJXZ} est un théorème de complétion d'un espace métrique.
\end{enumerate}

Quelque espaces qui sont complets sont listés ci-dessous. Attention : la complétude est bien une propriété de la norme; le même ensemble peut être complet pout une norme et pas pour une autre. Si on vous demande «est-ce que tel espace est complet ?» vous devez répondre en demandant «pour quelle topologie ?» \ldots sauf si la topologie est évidente dans le contexte.
\begin{multicols}{2}
    \begin{enumerate}
        \item
            Les réels \( \eR\), théorème \ref{ThoTFGioqS}. 


        \item
            La proposition \ref{PropSYMEZGU} donne quelque espaces complets. Soit \( X\) un espace topologique métrique \( (Y,d)\) un espace espace métrique complet. Alors les espaces
    \begin{enumerate}
        \item
            \( \big( C^0_b(X,Y),\| . \|_{\infty} \big)\) 
        \item
            \( \big( C^0_0(X,Y),\| . \|_{\infty} \big)\)
        \item
            \( \big( C^k_0(X,Y),\| . \|_{\infty} \big)\)
    \end{enumerate} 
    sont complets.

\item
    Le lemme \ref{LemdLKKnd} dit que \( \big( C^0(A,B),\| . \|_{\infty}\big)\) est complet dès que \( A\) est compact et \( B\) est complet.

\item
    L'espace \( \swD(K)\) est complet tant pour la topologie des semi-normes que pour la topologie métrique (qui sont les mêmes). C'est la proposition \ref{PropQAEVcTi}.
\item
    L'espace \( \swS(\Omega)\) est complet et métrisable par la proposition \ref{PropIIAcyDq}.
    \end{enumerate}
\end{multicols}

    La limite uniforme d'une suite de fonctions dérivables n'est pas spécialement dérivable. Même si les fonctions sont de classe \(  C^{\infty}\), la limite n'est pas spécialement mieux que continue. En effet, le théorème de Stone-Weierstrass \ref{ThoGddfas} nous dit que les polynômes (qui sont \(  C^{\infty}\)) sont denses dans les fonctions continues sur un compact pour la norme uniforme. Vous ne pouvez donc pas espérer que \( \big( C^p(X,Y),\| . \|_{\infty} \big)\) soit complet en général.
