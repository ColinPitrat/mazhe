\InternalLinks{différentiabilité}
\begin{enumerate}
    \item
        Définition générale de la différentielle sur des espaces vectoriels normés : la proposition \ref{DefKZXtcIT}.
    \item
        Nous parlons de différentielle en dimension finie et donnons une interprétation géométrique en \ref{SEBSECooLPRQooJRQCFL}.
    \item
        La recherche d'extrema d'une fonction sur \( \eR^n\) passe par la seconde différentielle, proposition \ref{PropoExtreRn}.
    \item
        Lien entre différentielle seconde (hessienne) et convexité en la proposition \ref{PROPooBMIRooFkQSAb} et le corollaire \ref{CORooMBQMooWBAIIH}.
    \item
        La différentielle est liée aux dérivées partielles par les formules données au lemme \ref{LemdfaSurLesPartielles}
	\begin{equation}
        df_a(u)=\frac{ \partial f }{ \partial u }(a)=\Dsdd{ f(a+tu) }{t}{0}=\sum_{i=1}^mu_i\frac{ \partial f }{ \partial x_i }(a)=\nabla f(a)\cdot u.
	\end{equation}
\end{enumerate}
