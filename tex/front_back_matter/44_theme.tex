
\InternalLinks{dualité}

Ne pas confondre dual algébrique et dual topologique d'un espace vectoriel.

\begin{description}
    \item[Dual topologique et algébrique]
        Ils sont définis par \ref{DefJPGSHpn}. Le dual algébrique est l'ensemble des formes linéaires, et le dual topologique ne considère que les formes linéaires continues (en dimension infinie, les applications linéaires ne sont pas toutes continues).
    \item[Topologie]
        Une topologie possible sur le dual d'un espace vectoriel topologique est celle \( *\)-faible de la définition \ref{DefHUelCDD}.

    \item[Théorèmes de dualité]
        Quelque théorèmes établissent des dualités entre des espaces courants.
\begin{enumerate}
    \item
        Le théorème de représentation de Riesz \ref{ThoQgTovL} pour les espaces de Hilbert.
    \item
        La proposition \ref{PropOAVooYZSodR} pour les espaces \( L^p\big( \mathopen[ 0 , 1 \mathclose] \big)\) avec \( 1<p<2\).
    \item
        Le théorème de représentation de Riesz \ref{ThoSCiPRpq} pour les espaces \( L^p\) en général.
\end{enumerate}
Tous ces théorèmes donnent la dualité par l'application \( \Phi_x=\langle x, .\rangle \).

\end{description}
