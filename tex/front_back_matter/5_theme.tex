
\InternalLinks{méthode de Newton}
    \begin{enumerate}
        \item
            Nous parlons un petit peu de méthode de Newton en dimension \( 1\) dans \ref{SECooIKXNooACLljs}.
        \item
            La méthode de Newton fonctionne bien avec les fonctions convexes par la proposition \ref{PROPooVTSAooAtSLeI}.
        \item
            La méthode de Newton en dimension $n$ est le théorème \ref{ThoHGpGwXk}.
       \item
            Un intervalle de convergence autour de \( \alpha\) s'obtient par majoration de \( | g' |\), proposition \ref{PROPooRPHKooLnPCVJ}.
       \item
           Un intervalle de convergence quadratique s'obtient par majoration de \( | g'' |\), théorème \ref{THOooDOVSooWsAFkx}.
       \item
           En calcul numérique, section \ref{SECooIKXNooACLljs}.
       \end{enumerate}

