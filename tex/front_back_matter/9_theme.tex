\InternalLinks{arithmétique modulo, théorème de Bézout} \label{THEMEooNRZHooYuuHyt}
    \begin{enumerate}
        \item
            Pour \( \eZ^*\) c'est le théorème \ref{ThoBuNjam}.
        \item
            Théorème de Bézout dans un anneau principal : corollaire \ref{CorimHyXy}.
        \item
            Théorème de Bézout dans un anneau de polynômes : théorème \ref{ThoBezoutOuGmLB}.
        \item
            En parlant des racines de l'unité et des générateurs du groupe unitaire dans le lemme \ref{LemcFTNMa}. Au passage nous y parlerons de solfège.
        \item
            La proposition \ref{PropLAbRSE} qui donne tout entier assez grand comme combinaisons de \( a \) et \( b\) à coefficients positifs est utilisée en chaînes de Markov, voir la définition \ref{DefCxvOaT} et ce qui suit.
        \item
            Calcul effectif du PGCD puis des coefficients de Bézout : sous-sections \ref{SUBSECooAEBLooFGJRkg} et \ref{SUBSECooRHSQooEuBWbd}.
        \end{enumerate}

