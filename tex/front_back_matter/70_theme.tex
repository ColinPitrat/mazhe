\InternalLinks{groupe symétrique}       \label{THEMEooQEEWooXDhvhv}

\begin{enumerate}
    \item
        Définition~\ref{DEFooJNPIooMuzIXd}.
    \item
        La signature \( \epsilon\colon S_n\to \{ -1,1 \}\) est l'unique homomorphisme surjectif de \( S_n\) sur \( \{ -1,1 \}\), proposition \ref{ProphIuJrC}\ref{ITEMooBQKUooFTkvSu}.
    \item
        La table des caractères du groupe symétrique \( S_4\) est donné dans la section~\ref{SecUMIgTmO}.
    \item
        Le groupe symétrique \( S_4\) est le groupe des symétries affines du tétraèdre régulier, proposition~\ref{PROPooVNLKooOjQzCj}.
    \item
        Le groupe alterné \( A_5\) est l'unique groupe simple d'ordre \( 60\), proposition~\ref{PROPooUBIWooTrfCat}.
    \item
        La proposition~\ref{PROPooCPXOooVxPAij} donne la position du groupe alterné dans le groupe symétrique : \( A_n\) est un sous-groupe caractéristique de \( S_n\) et l'unique sous-groupe d'indice \( 2\).
\end{enumerate}
