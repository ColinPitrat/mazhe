% This is part of le Frido
% Copyright (c) 2016-2018
%   Laurent Claessens
% See the file fdl-1.3.txt for copying conditions.

%+++++++++++++++++++++++++++++++++++++++++++++++++++++++++++++++++++++++++++++++++++++++++++++++++++++++++++++++++++++++++++
\section*{Thèmes}
%+++++++++++++++++++++++++++++++++++++++++++++++++++++++++++++++++++++++++++++++++++++++++++++++++++++++++++++++++++++++++++


Ceci est une sorte d'index thématique.

% The macro '\InternalLink' writes in 'themestoc.tex' the lines like
% ~\ref {THTOC4} : méthode de Newton\\

% We open the file after the input (if before, we erase it) and
% we close at the end of this file.

\begin{multicols}{2}
\noindent
\ref {THTOC1} : intégration\\
\ref {THTOC2} : suites et séries\\
\ref {THTOC3} : polynôme de Taylor\\
\ref {THTOC4} : séries de Fourier\\
\ref {THTOC5} : normes\\
\ref {THTOC6} : topologie produit\\
\ref {THTOC7} : espaces métriques, normés\\
\ref {THTOC8} : norme opérateur\\
\ref {THTOC9} : gaussienne\\
\ref {THTOC10} : produit de compacts\\
\ref {THTOC11} : densité\\
\ref {THTOC12} : espaces de fonctions\\
\ref {THTOC13} : fonctions Lipschitz\\
\ref {THTOC14} : formule des accroissements finis\\
\ref {THTOC15} : différentiabilité\\
\ref {THTOC16} : points fixes\\
\ref {THTOC17} : théorèmes de Stokes, Green et compagnie\\
\ref {THTOC18} : permuter des limites\\
\ref {THTOC19} : applications continues et bornées\\
\ref {THTOC20} : inégalités\\
\ref {THTOC21} : connexité\\
\ref {THTOC22} : suite de Cauchy, espace complet\\
\ref {THTOC23} : application réciproque\\
\ref {THTOC24} : tribu, algèbre de parties, \( \lambda \)-systèmes et co.\\
\ref {THTOC25} : déduire la nullité d'une fonction depuis son intégrale\\
\ref {THTOC26} : mesure et intégrale\\
\ref {THTOC27} : équations différentielles\\
\ref {THTOC28} : injections\\
\ref {THTOC29} : logarithme\\
\ref {THTOC30} : inversion locale, fonction implicite\\
\ref {THTOC31} : convexité\\
\ref {THTOC32} : dualité\\
\ref {THTOC33} : opérations sur les distributions\\
\ref {THTOC34} : transformée de Fourier\\
\ref {THTOC35} : convolution\\
\ref {THTOC36} : méthode de Newton\\
\ref {THTOC37} : méthodes de calcul\\
\ref {THTOC38} : définie positive\\
\ref {THTOC39} : norme matricielle et rayon spectral\\
\ref {THTOC40} : série de matrices\\
\ref {THTOC41} : rang\\
\ref {THTOC42} : extension de corps et polynômes\\
\ref {THTOC43} : décomposition de matrices\\
\ref {THTOC44} : systèmes d'équations linéaires\\
\ref {THTOC45} : formes bilinéaires et quadratiques\\
\ref {THTOC46} : arithmétique modulo, théorème de Bézout\\
\ref {THTOC47} : polynômes\\
\ref {THTOC48} : invariants de similitude\\
\ref {THTOC49} : diagonalisation\\
\ref {THTOC50} : endomorphismes cycliques\\
\ref {THTOC51} : déterminant\\
\ref {THTOC52} : polynôme d'endomorphismes\\
\ref {THTOC53} : exponentielle de matrice\\
\ref {THTOC54} : types d'anneaux\\
\ref {THTOC55} : sous-groupes\\
\ref {THTOC56} : groupe symétrique\\
\ref {THTOC57} : action de groupe\\
\ref {THTOC58} : classification de groupes\\
\ref {THTOC59} : produit semi-direct de groupes\\
\ref {THTOC60} : théorie des représentations\\
\ref {THTOC61} : isométries\\
\ref {THTOC62} : caractérisation de distributions en probabilités\\
\ref {THTOC63} : théorème central limite\\
\ref {THTOC64} : lemme de transfert\\
\ref {THTOC65} : probabilités et espérances conditionnelles\\
\ref {THTOC66} : dénombrements\\
\ref {THTOC67} : enveloppes\\
\ref {THTOC68} : équations diophantiennes\\

\end{multicols}

\newwrite\themetoc
\immediate\openout\themetoc=themestoc.tex


% ATTENTION : il est très important que le titre «Thèmes» soit au haut d'une page et que le premier thème commence sur cette même page. Donc pas de texte trop long ici.
% La raison :
% Pour la division en volume, je prend le bloc 'thèmes' comme commençant à la page du premier. Cela est dû au fait que le titre n'est pas dans la TOC.

% Convention : les titres ne commencent pas par une majuscule
%  La raison : cela facilite les recherches dans le pdf (oui, je sais : on peut faire des recherches sans tenir compte des majuscules)


% ANALYSE

\InternalLinks{théorie de la mesure}       \label{THEMEooKLVRooEqecQk}
\begin{description}
    \item[Mesure] 
    À propos de mesure.
\begin{enumerate}
    \item
        Mesure positive, mesure finie et \( \sigma\)-finie, c'est la définition \ref{DefBTsgznn}.

    \item Le produit de tribus est donné par la définition~\ref{DefTribProfGfYTuR},     % Cette référence doit être vers le haut.
    \item
        Produit d'une mesure par une fonction, définition \ref{PropooVXPMooGSkyBo}.
    \item le produit d'espaces mesurés est donné par la définition~\ref{DefUMlBCAO}.     % Cette référence doit être vers le haut.
        \item
            Mesure de Lebesgue sur \( \eR\), définition~\ref{DefooYZSQooSOcyYN}.
        \item
            Une partie de \( \eR\) non mesurable au sens de Lebesgue : l'exemple \ref{EXooCZCFooRPgKjj}.
        \item
            Mesure de Lebesgue sur \( \eR^N\), définition~\ref{DEFooSWJNooCSFeTF}.
        \item
            Mesure à densité, définition~\ref{PropooVXPMooGSkyBo}.
\end{enumerate}
\item[Théorèmes d'approximation]
    Il est important de pouvoir approcher des fonctions continues ou \( L^p\) par des fonctions étagées, sinon on ne parvient pas à faire tourner la machine de l'intégration de Lebesgue.
    \begin{enumerate}
        \item
            Si \( f\colon S\to \mathopen[ 0 , +\infty \mathclose]\) est une fonction mesurable, le théorème fondamental d'approximation~\ref{THOooXHIVooKUddLi} donne une suite croissante de fonctions étagées qui converge vers \( f\).
        \item
            Les fonctions simples sont denses dans \( L^p\), proposition \ref{PROPooUQUBooAWgNhm}.
        \item
            Encadrement d'un borélien \( A\) par un fermé \( F\) et un ouvert \( V\) par le lemme \ref{LEMooCGKXooYWjRwk} : \( F\subset A\subset V\) avec \( \mu(V\setminus F)<\epsilon\).
        \item
            Approximation \( L^p\) de la fonction caractéristique d'un borélien par une fonction continue par le théorème \ref{ThoAFXXcVa}.
    \end{enumerate}
\end{description}

  % mesure et intégration
\InternalLinks{suites et séries}

\begin{description}
    \item[Suites] 
        Les suites réelles sont en général dans la proposition \ref{PropLimiteSuiteNum} et ce qui s'ensuit. Cette proposition est souvent prise comme définition lorsque seules les suites réelles ne sont considérées.
        \begin{enumerate}
    \item
        Les suites adjacentes, c'est la définition \ref{DEFooDMZLooDtNPmu}. Cela sert pour les séries alternées, théorème \ref{THOooOHANooHYfkII}, qui sert pour étudier la série de Taylor de \( \ln(x+1)\), voir le lemme \ref{LEMooWMGGooRpAxBa} et ce qui l'entoure.
    \item
        La définition de la convergence absolue est la définition~\ref{DefVFUIXwU}.
            \item
                Une suite réelle croissante et majorée converge, proposition \ref{LemSuiteCrBorncv}.
            \item
                Toute suite dans un compact admet une sous-suite convergente, théorème \ref{THOooRDYOooJHLfGq}.
            \item
                Pour tout réel, il existe une suite croissante de rationnels qui y converge, proposition \ref{PropSLCUooUFgiSR}.
        \end{enumerate}
    \item[Série] 
        Les séries sont en général dans la section \ref{SECooYCQBooSZNXhd}.
        \begin{enumerate}
    \item
        Quelques séries usuelles en \ref{SUBSECooDTYHooZjXXJf} : série harmonique, géométrique, de Riemann, et la mythique arithmético-géométrique.
    \item
        Critère des séries alternées, théorème \ref{THOooOHANooHYfkII}.
    \item
        Convergence d'une série implique convergence vers zéro du terme général, proposition~\ref{PROPooYDFUooTGnYQg}.
        \end{enumerate}
    \item[Sommes infinies]
        Nous pouvons dire plusieurs choses à propos d'une somme infinie.% cette phrase est là pour le mot-clef ``somme infinie''.
        \begin{enumerate}
            \item
Une somme indexée par un ensemble quelconque est la définition~\ref{DefHYgkkA}.
    \item
        La définition de la somme d'une infinité de termes est donnée par la définition~\ref{DefGFHAaOL}.
  \item
      si la série converge, on peut regrouper ses termes sans modifier la convergence ni la somme (associativité);
    Pour les sommes infinies l'associativité et la commutativité dans une série sont perdues. Néanmoins, il subsiste que
  \begin{enumerate}
  \item
      si la série converge absolument, on peut modifier l'ordre des termes sans modifier la convergence ni la somme (commutativité, proposition~\ref{PopriXWvIY}).
  \end{enumerate}
  \item Permuter une somme infinie avec une application linéaire : \( f(\sum_{i\in I}v_i)=\sum_{i\in I}f(v_i)\), c'est la proposition \ref{PROPooWLEDooJogXpQ}.
        \end{enumerate}
\end{description}
  % suites et séries

\InternalLinks{polynôme de Taylor}

Énoncés :

    \begin{enumerate}
    \item
        Énoncé : théorème \ref{ThoTaylor}.
    \item
        De classe \( C^2\) sur \( \eR^n\), proposition \ref{PROPooTOXIooMMlghF}.
    \item
    Avec un reste donné par un point dans \( \mathopen] x , a \mathclose[\), proposition \ref{PropResteTaylorc}.
        \item
            Avec reste intégral, proposition \ref{PropAXaSClx}.
        \item
            Le polynôme de Taylor généralise à l'utilisation de toutes les dérivées disponibles le résultat de développement limité donné par la proposition \ref{PropUTenzfQ}.
        \item
            Pour les fonctions holomorphes, il y a le théorème \ref{THOooSULFooHTLRPE} qui donne une série de Taylor sur un disque de convergence.
        \end{enumerate}

Utilisation :
\begin{enumerate}
        \item
            Il est utilisé pour justifier la méthode de Newton autour de l'équation \eqref{EQooOPUBooYaznay}.
    \item
        On utilise pas mal de Taylor dans les résultats liant extrema et différentielle/hessienne. Par exemple la proposition \ref{PropoExtreRn}.
\end{enumerate}
  % polynôme de Taylor
\InternalLinks{séries de Fourier}       \label{THMooHWEBooTMInve}
\begin{itemize}
    \item Formule sommatoire de Poisson, proposition \ref{ProprPbkoQ}.
    \item Inégalité isopérimétrique, théorème \ref{ThoIXyctPo}.
    \item Fonction continue et périodique dont la série de Fourier ne converge pas, proposition \ref{PropREkHdol}.

    \item
Nous allons montrer la convergence de \( \sum_{k\in \eZ}c_k(f) e^{inx}\) vers \( f(x)\) dans divers cas :
\begin{enumerate}
    \item
        Si \( f\) est continue et périodique, convergence au sens de Cesaro, théorème de Fejèr \ref{ThoJFqczow}.
    \item
        Convergence au sens \( L^2\Big( \mathopen[ 0 , 2\pi \mathclose] \Big)\) dans le théorème \ref{ThoYDKZLyv}.
    \item
        Si \( f\) est continue, périodique et sa série de Fourier converge uniformément, théorème \ref{PropmrLfGt}.
    \item
        Si \( f\) est périodique et la série des coefficients converge absolument pour tout \( x\), proposition \ref{PropSgvPab}.
    \item
        Si \( f\) est périodique et de classe \( C^1\), théorème \ref{ThozHXraQ}.
\end{enumerate}
Il est cependant faux de croire que la continuité et la périodicité suffisent à obtenir une convergence, comme le montrons dans la proposition \ref{PropREkHdol}.
\end{itemize}
  % séries de Fourier
\InternalLinks{normes}      \label{THEMEooUJVXooZdlmHj}

\begin{description}
    \item[Définition] Espace vectotiel normé : définition~\ref{DefNorme}.
    \item[Équivalence de norme]

        \begin{enumerate}
        \item
            Définition de l'équivalence de norme~\ref{DefEquivNorm}.
\item
    La proposition~\ref{PropLJEJooMOWPNi} sur l'équivalence des normes dans \( \eR^n\).
\item
    Toutes les normes sur un espace vectoriel de dimension finie sont équivalentes, théorème~\ref{ThoNormesEquiv}.
\item
    Montrer que le problème \( a-b\) est stable dans l'exemple~\ref{ExooXJONooTAYZVc}.
\item
    La proposition~\ref{PROPooWZJBooTPLSZp} donnant \( \rho(A)\leq \| A \|\) utilise l'équivalence de toutes les normes sur un espace vectoriel de dimension finie.

        \end{enumerate}

    \item[Norme opérateur et d'algèbre] voir le thème~\ref{THEMEooOJJFooWMSAtL}.

\end{description}
  %  normes

\InternalLinks{topologie produit}
    \begin{enumerate}
        \item
            La définition de la topologie produit est \ref{DefIINHooAAjTdY}.
        \item
            Pour les espaces vectoriels normés, le produit est donné par la définition \ref{DefFAJgTCE}.
        \item
            L'équivalence entre la topologie de la norme produit et la topologie produit est le lemme \ref{LEMooWVVCooIGgAdJ}.
        \end{enumerate}

  % topologie produit
\InternalLinks{espaces métriques, normés}
\begin{enumerate}
    \item
        Un espace métrique est un espace muni d'une distance, définition \ref{DefMVNVFsX}.
    \item
        La distance entre un point et un ensemble est la définition \ref{DEFooGNNUooFUZINs}.
    \item
        Le théorème-définition~\ref{ThoORdLYUu} donne la topologie sur un espace métrique en disant que les boules ouvertes sont une base de la topologie (définition \ref{DefQELfbBEyiB}).
    \item
        La définition de la convergence d'une suite est la définition~\ref{DefXSnbhZX}.
    \item
        Dans un espace vectoriel normé, une application est continue si et seulement si elle est bornée, proposition~\ref{PROPooQZYVooYJVlBd}.
    \item
        Un espace vectoriel topologique qui possède en tout point une base dénombrable de topologie accepte une distance, théorème \ref{THOooAGBXooZnvQLK}.
\end{enumerate}
  % espaces métriques, normés

    \InternalLinks{norme opérateur}     \label{THEMEooHSLLooBQpFAr}
    Pour la norme matricielle et le rayon spectral, voir le thème~\ref{THEMEooOJJFooWMSAtL}.
    \begin{enumerate}
        \item
            Définition~\ref{DefNFYUooBZCPTr}.
        \item
            Définition d'une algèbre :~\ref{DefAEbnJqI} et pour une norme d'algèbre :~\ref{DefJWRWQue}.
        \item
            Pour des espaces vectoriels normés, être borné est équivalent à être continu : proposition~\ref{PROPooQZYVooYJVlBd}.
        \item
            Le lemme à propos d'exponentielle de matrice~\ref{LemQEARooLRXEef} donne :
            \begin{equation}
                \|  e^{tA} \|\leq P\big( | t | \big)\sum_{i=1}^r e^{t\real(\lambda_i)}.
            \end{equation}
    \end{enumerate}
  % norme opérateur

\InternalLinks{gaussienne}
\begin{enumerate}
    \item
        Le calcul de l'intégrale
        \begin{equation}
            \int_{\eR} e^{-x^2}dx=\sqrt{\pi }
        \end{equation}
        est fait de deux façons dans l'exemple~\ref{EXooLUFAooGcxFUW}. Dans les deux cas, le théorème de Fubini~\ref{ThoFubinioYLtPI} est utilisé.
    \item
        Le lemme~\ref{LEMooPAAJooCsoyAJ} calcule la transformée de Fourier de $ g_{\epsilon}(x)=  e^{-\epsilon\| x \|^2}$ qui donne $\hat g_{\epsilon}(\xi)=\left( \frac{ \pi }{ \epsilon } \right)^{d/2} e^{-\| \xi \|^2/4\epsilon}$.
    \item
        Le lemme~\ref{LEMooTDWSooSBJXdv} donne une suite régularisante à base de gaussienne.
    \item
        Elle est utilisée pour régulariser une intégrale dans la preuve de la formule d'inversion de Fourier~\ref{PROPooLWTJooReGlaN}
\end{enumerate}
  % gaussienne

\InternalLinks{compacts}        \label{THEMEooQQBHooLcqoKB}
    \begin{description}

        \item[Propriétés générales]

            Quelques propriétés de compacts.

                \begin{enumerate}
    \item
        La définition d'un ensemble compact est la définition~\ref{DefJJVsEqs}.
    \item
        Les compacts sont les fermés bornés par le théorème~\ref{ThoXTEooxFmdI}.
    \item
        Le théorème de Borel-Lebesgue \ref{ThoBOrelLebesgue} dit qu'un intervalle de \( \eR\) est compact si et seulement si il est de la forme \( \mathopen[ a , b \mathclose]\).
    \item
        Théorème des fermés emboîtés dans le cas compact, corollaire \ref{CORooQABLooMPSUBf}. À ne pas confondre avec celui dans le cas des espaces métrique, théorème \ref{ThoCQAcZxX}.
    \item
        L'image d'un compact par une fonction continue est un compact, théorème~\ref{ThoImCompCotComp}.
    \item
        Suites dans un compact
        \begin{enumerate}
            \item
                Toute suite dans un compact admet une sous-suite convergente, théorème \ref{THOooRDYOooJHLfGq}.
            \item
                Dans \( \eR^n\), toute suite dans un compact admet une sous-suite convergente, théorème \ref{ThoBolzanoWeierstrassRn}. La démonstration de ce théorèma est non seulement plus compliquée que le cas général, mais utilise en plus le cas dans \( \eR\); lequel cas n'est pas démontré de façon directe dans le Frido.
            \item
                Un espace métrique est compact si et seulement si toute suite contient une sous-suite convergente. C'est le théorème de Bolzano-Weierstrass~\ref{ThoBWFTXAZNH}. La démonstration de ce théorème est indépendante.
        \end{enumerate}
    \item
        Une fonction continue sur un compact est bornée et attein ses bornes, théorème~\ref{ThoWeirstrassRn}.
    \item
        Une fonction continue sur un compact y est uniforément continue, théorème de Heine \ref{PROPooBWUFooYhMlDp}.
                \end{enumerate}

        \item[Produits de compacts]
            À propos de produits de compacts. C'est un compact dans tous les cas métriques\quext{Si vous connaissez des exemples non métriques de produits de compacts qui ne sont pas compacts, écrivez moi.}.
    \begin{enumerate}
    \item
        Les produits d'espaces métriques compacts sont compacts; c'est le théorème de Tykhonov. Nous verrons ce résultat dans les cas suivants.
        \begin{itemize}
    \item
         \( \eR\), lemme~\ref{LemCKBooXkwkte}.
    \item
        Produit fini d'espaces métriques compacts, théorème~\ref{THOIYmxXuu}.
    \item
        Produit dénombrable d'espaces métriques compacts, théorème~\ref{ThoKKBooNaZgoO}.
        \end{itemize}
    \end{enumerate}
    \end{description}
  % produit de compacts
\InternalLinks{densité}         \label{THEooPUIIooLDPUuq}
\begin{enumerate}
    \item 
        Densité des polynômes dans \( C^0\big( \mathopen[ 0 , 1 \mathclose] \big)\), théorème de Bernstein \ref{ThoDJIvrty}.
    \item
        Densité de \( \swD(\eR^d)\) dans \( L^p(\eR^d)\) pour \( 1\leq p<\infty\), théorème \ref{ThoILGYXhX}.
    \item
        Densité de \( \swS(\eR^d)\) dans l'espace de Sobolev \( H^s(\eR^d)\), proposition \ref{PROPooMKAFooKDNTbO}. 

    \item
        Densité de \( \swD(\eR^d)\) dans l'espace de Sobolev \( H^s(\eR^d)\), proposition \ref{PROPooLIQJooKpWtnV}. 

        Cela est utilisé pour le théorème de trace \ref{THOooXEJZooBKtXBW}.
    \item
        Les applications étagées dans les applications mesurables (qui plus est avec limite croissante), théorème fondamental d'approximation \ref{LempTBaUw}.
    \item
        Les fonctions continues à support compact dans \( L^2(I)\), théorème \ref{ThoJsBKir}.
\end{enumerate}
Les densités sont bien entendu utilisées pour prouver des formules sur un espace en sachant qu'elles sont vraies sur une partie dense. Mais également pour étendre une application définie seulement sur une partie dense. C'est par exemple ce qui est fait pour définir la trace \( \gamma_0\) sur les espaces de Sobolev \( H^s(\eR^d)\) en utilisant le théorème d'extension \ref{PropTTiRgAq}.

Comme presque tous les théorèmes importants, le théorème de Stone-Weierstrass possède de nombreuses formulations à divers degrés de généralité.
\begin{itemize}
    \item Le lemme \ref{LemYdYLXb} le donne pour la racine carré.
    \item Le théorème \ref{ThoGddfas} donne la densité des polynômes dans les fonctions continues sur un compact.
    \item Le théorème \ref{ThoWmAzSMF} est une généralisation qui donne la densité uniforme d'une sous-algèbre de \( C(X,\eR)\) dès que \( X\) sépare les points.
    \item Le lemme \ref{LemXGYaRlC} est une version pour les polynômes trigonométriques.
    \item
        Le lemme \ref{LemYdYLXb} est une cas particulier du
        théorème \ref{ThoGddfas}, mais nous en donnons une démonstration indépendante afin d'isoler la preuve
de la généralisation \ref{ThoWmAzSMF}. 
Une version pour les polynômes trigonométriques sera donnée dans le lemme \ref{LemXGYaRlC}.
\end{itemize}
  % densité
\InternalLinks{espaces de fonctions}                \label{THEMooNMYKooVVeGTU}

En ce qui concerne les densités, voir le thème~\ref{THEooPUIIooLDPUuq}.


\begin{description}
    \item[Topologie]

        Les espaces de fonctions sont souvent munis de topologies définies par des semi-normes.

        \begin{enumerate}
            \item
                La topologie des semi-normes est la définition~\ref{DefPNXlwmi}.
            \item
                La définition~\ref{DefFGGCooTYgmYf} donne les topologies sur \(  C^{\infty}(\Omega)\), \( \swD(K)\) et \( \swD(\Omega)\).
            \item
                La topologie \( *\)-faible sur \( \swD'(\Omega)\) est donnée par la définition~\ref{DefASmjVaT}.
        \end{enumerate}

    \item[L'espace \( { L^2\big( \mathopen[ 0 , 2\pi \mathclose] \big) } \)]

        C'est un espace très important, entre autres parce qu'il est de Hilbert et est bien adapté à la transformée de Fourier.

        \begin{enumerate}
        \item
            Un rappel de la construction en \ref{NORMooUEIEooYtlFse}.
            \item
                Le produit scalaire \( \langle f, g\rangle \) est donné en \eqref{EQooBFKDooMkCZOt} et la base trigonométrique est \eqref{EQooKMYOooLZCNap}.
            \item
                La densité des polynômes trigonométriques dans \( L^p(S^2)\) est le théorème~\ref{ThoQGPSSJq} ou le théorème~\ref{ThoDPTwimI}, au choix.
            \item
                Une conséquence de cette densité est que le système trigonométrique est une base hilbertienne de \( L^2\) par le lemme~\ref{LEMooBJDQooLVPczR}.
        \end{enumerate}

            L'espace \( L^2\) est discuté en analyse fonctionnelle, dans la section \ref{SECooEVZSooLtLhUm} et les suivantes parce que l'étude de \( L^2\) utilise entre autres l'inégalité de Hölder~\ref{ProptYqspT}.

        Le fait que \( L^2\) soit une espace de Hilbert est utilisé dans la preuve du théorème de représentation de Riesz~\ref{PropOAVooYZSodR}.

\end{description}
  % espaces de fonctions

\InternalLinks{fonctions Lipschitz}
    \begin{enumerate}
    \item
        Définition : \ref{DEFooQHVEooDbYKmz}.
    \item
        La notion de Lipschitz est utilisée pour définir la stabilité d'un problème, définition \ref{DEFooYIFAooSJbMkC}.
    \end{enumerate}

  % fonctions Lipschitz

\InternalLinks{formule des accroissements finis}
    Il en existe plusieurs formes :
    \begin{enumerate}
        \item
            Une version adaptée aux espaces de dimension finie est le théorème \ref{val_medio_2}.
        \item
            Pour les fonctions \( \eR\to \eR\) en le théorème \ref{ThoAccFinis}.
        \item
            Une généralisation pour les intervalles non bornés : théorème \ref{THOooRIIBooOjkzMa}.
        \item
            Espaces vectoriels normés, théorème \ref{ThoNAKKght}
    \end{enumerate}

  % formule des accroissements finis
\InternalLinks{limite et continuité}        \label{THEMEooGVCCooHBrNNd}


Les notions de limite et de continuité ainsi que leurs liens sont déjà entièrement faits dans la définition \ref{DefOLNtrxB} et le théorème \ref{ThoESCaraB}.

Voir l'exemple \ref{EXooKREUooLeuIlv} traité en détail.
  % limite et continuité
\InternalLinks{différentiabilité}
\begin{enumerate}
    \item
        Définition générale de la différentielle sur des espaces vectoriels normés : la proposition~\ref{DefKZXtcIT}.
    \item
        Nous parlons de différentielle en dimension finie et donnons une interprétation géométrique en~\ref{SEBSECooLPRQooJRQCFL}.
    \item
        La recherche d'extrema d'une fonction sur \( \eR^n\) passe par la seconde différentielle, proposition~\ref{PropoExtreRn}.
    \item
        Lien entre différentielle seconde (hessienne) et convexité en la proposition~\ref{PROPooBMIRooFkQSAb} et le corollaire~\ref{CORooMBQMooWBAIIH}.
    \item
        La différentielle est liée aux dérivées partielles par les formules données au lemme~\ref{LemdfaSurLesPartielles}
	\begin{equation}
        df_a(u)=\frac{ \partial f }{ \partial u }(a)=\Dsdd{ f(a+tu) }{t}{0}=\sum_{i=1}^mu_i\frac{ \partial f }{ \partial x_i }(a)=\nabla f(a)\cdot u.
	\end{equation}
\end{enumerate}
    % différentielle


\InternalLinks{points fixes}
    \begin{enumerate}
\item
    Il y a plusieurs théorèmes de points fixes.
    \begin{description}
        \item[Théorème de Picard]~\ref{ThoEPVkCL} donne un point fixe comme limite d'itérés d'une fonction Lipschitz. Il aura pour conséquence le théorème de Cauchy-Lipschitz~\ref{ThokUUlgU}, l'équation de Fredholm, théorème~\ref{ThoagJPZJ} et le théorème d'inversion locale dans le cas des espaces de Banach~\ref{ThoXWpzqCn}.
    \item[Théorème de Brouwer] qui donne un point fixe pour une application d'une boule vers elle-même. Nous allons donner plusieurs versions et preuves.
            \begin{enumerate}
                \item
                    Dans \( \eR^n\) en version \( C^{\infty}\) via le théorème de Stokes, proposition~\ref{PropDRpYwv}.
                \item
                    Dans \( \eR^n\) en version continue, en s'appuyant sur le cas \( C^{\infty}\) et en faisant un passage à la limite, théorème~\ref{ThoRGjGdO}.
                \item
                    Dans \( \eR^2\) via l'homotopie, théorème~\ref{ThoLVViheK}. Oui, c'est très loin. Et c'est normal parce que ça va utiliser la formule de l'indice qui est de l'analyse complexe\footnote{On aime bien parce que ça ne demande pas Stokes, mais quand même hein, c'est pas gratos non plus.}.
            \end{enumerate}
        \item[Théorème de Markov-Kakutani]\ref{ThoeJCdMP} qui donne un point fixe à une application continue d'un convexe fermé borné dans lui-même. Ce théorème donnera la mesure de Haar~\ref{ThoBZBooOTxqcI} sur les groupes compacts.
        \item[Théorème de Schauder]~\ref{ThovHJXIU} qui est une version valable en dimension infinie du théorème de Brouwer.
    \end{description}

\item Pour les équations différentielles
    \begin{enumerate}
        \item
            Le théorème de Schauder a pour conséquence le théorème de Cauchy-Arzela~\ref{ThoHNBooUipgPX} pour les équations différentielles.
        \item
            Le théorème de Schauder~\ref{ThovHJXIU} permet de démontrer une version du théorème de Cauchy-Lipschitz (théorème~\ref{ThokUUlgU}) sans la condition Lipschitz, mais alors sans unicité de la solution. Notons que de ce point de vue nous sommes dans la même situation que la différence entre le théorème de Brouwer et celui de Picard : hors hypothèse de type «contraction», point d'unicité.
    \end{enumerate}
\item
    En calcul numérique
    \begin{itemize}
        \item
            La convergence d'une méthode de point fixe est donnée par la proposition~\ref{PROPooRPHKooLnPCVJ}.
        \item
            La convergence quadratique de la méthode de Newton est donnée par le théorème~\ref{THOooDOVSooWsAFkx}.
        \item
            En calcul numérique, section~\ref{SECooWUVTooMhmvaW}
        \item
            Méthode de Newton comme méthode de point fixe, sous-section~\ref{SUBSECooIBLNooTujslO}.
    \end{itemize}

\item
    D'autres utilisation de points fixes.
\begin{itemize}
    \item
        Processus de Galton-Watson, théorème~\ref{ThoJZnAOA}.
    \item
        Dans le théorème de Max-Milgram~\ref{THOooLLUXooHyqmVL}, le théorème de Picard est utilisé.
\end{itemize}
\end{enumerate}
  % points fixes

\InternalLinks{théorèmes de Stokes, Green et compagnie}
    \begin{enumerate}
        \item
            Forme générale, théorème~\ref{ThoATsPuzF}.
        \item
            Rotationnel et circulation, théorème~\ref{THOooIRYTooFEyxif}.
        \end{enumerate}
        Le théorème de Stokes peut être utilisé pour montrer le théorème de Brower, proposition~\ref{PropDRpYwv}.
  % théorèmes de Stokes, Green et compagnie

\InternalLinks{permuter des limites}
\begin{description}
    \item[Fonctions définies par une intégrale]
        Les théorèmes sur les fonctions définies par une intégrale, section~\ref{SecCHwnBDj}. Nous avons entre autres
        \begin{enumerate}
            \item
                \( \partial_i\int_Bf=\int_B\partial_if\), avec \( B\) compact, proposition~\ref{PropDerrSSIntegraleDSD}.
            \item
                Si \( f\) est majorée par une fonction ne dépendant pas de \( x\), nous avons le théorème~\ref{ThoKnuSNd}.
            \item
                Si l'intégrale est uniformément convergente, nous avons le théorème~\ref{ThotexmgE}.
            \item
                Pour dériver \( \int_Bg(t,z)dt\) avec \( B\) compact dans \( \eR\) et \( g\colon \eR\times \eC\to \eC\), il faut aller voir la proposition~\ref{PROPooZCLYooUaSMWA}.
            \item
                En ce qui concerne le \( x\) dans la borne, le théorème \ref{PropEZFRsMj} lie primitive et intégrale en montrant que \( F(x)=\int_a^xf(t)dt\) est une primitive de \( f\) (sous certaines conditions). Le théorème fondamental de l'analyse \ref{ThoRWXooTqHGbC} en est une conséquence.
        \end{enumerate}
    \item[Conegence monotone]
        Théorème de la convergence monotone, théorème~\ref{ThoRRDooFUvEAN}.
    \item[Fubini]
        Le théorème de Fubini permet non seulement de permuter des intégrales, mais également des sommes parce que ces dernières peuvent être vues comme des intégrales sur \( \eN\) muni de la tribu des parties et de la mesure de comptage. Nous utilisons cette technique pour permute une somme et une intégrale dans l'équation \eqref{EQooWOLOooFHSrsx}.
\begin{itemize}
    \item
        le théorème de Fubini-Tonelli~\ref{ThoWTMSthY} demande que la fonction soit mesurable et positive;
    \item
        le théorème de Fubini~\ref{ThoFubinioYLtPI} demande que la fonction soit intégrable (mais pas spécialement positive);
    \item
        le corollaire~\ref{CorTKZKwP} demande l'intégrabilité de la valeur absolue des intégrales partielles pour déduire que la fonction elle-même est intégrable.
\end{itemize}

\item[Limite et dérivées, différentielle]
    \begin{enumerate}
        \item
            Permuter limite et dérivée, théorème \ref{THOooXZQCooSRteSr}.
        \item
 Permuter limite et dérivées partielles, théorème \ref{ThoSerUnifDerr}.
        \item
            Permuter limite et différentielle, théorème \ref{ThoLDpRmXQ}.
    \end{enumerate}
    Quelque remarques sur les techniques de démonstration.
    \begin{enumerate}
        \item
            Le résultat fondamental \ref{THOooXZQCooSRteSr} est démontré sans recourrir à des intégrales. Une preuve alternative, plus courte, avec des intégrales est donnée en \ref{NORMALooGYUEooKrYjyz}.
        \item
            Les résultats un peu plus élaborés \ref{ThoSerUnifDerr} et \ref{ThoLDpRmXQ} sont prouvés avec des intégrales, mais devraient pouvoir être adaptés.
    \end{enumerate}
\item[Somme et dérivée]
    Permuter somme et différentielle, théorème \ref{ThoLDpRmXQ}.

\end{description}
  % permuter des limites

\InternalLinks{applications continues et bornées}       \label{THEMEooYCBUooEnFdUg}
\begin{enumerate}
    \item
        Une application linéaire non continue : exemple~\ref{ExHKsIelG} de \( e_k\mapsto ke_k\). Les dérivées partielles sont calculées en \eqref{EQooWNLOooJNRUMQ}.
    \item
        Une autre application linéaire non continue en l'exemple~\ref{EXooDMVJooAJywMU} : la dérivation sur les polynômes.
    \item
        Une application linéaire est bornée si et seulement si elle est continue, proposition~\ref{PROPooQZYVooYJVlBd}.
\end{enumerate}
  % applications continues et bornées

\InternalLinks{inégalités}
\begin{description}
    \item[Inégalité de Jensen] 
        \begin{enumerate}
            \item
                Une version discrète pour \( f\big( \sum_i\lambda_ix_i \big)\), la proposition \ref{PropXIBooLxTkhU}.
            \item
                Une version intégrale pour \( f\big( \int \alpha d\mu \big)\), la proposition \ref{PropXISooBxdaLk}.
            \item
                Une version pour l'espérance conditionnelle, la proposition \ref{PropABtKbBo}.
        \end{enumerate}
    \item[Inégalité de Minkowsky]
        \begin{enumerate}
            \item
                Pour une forme quadratique \( q\) sur \( \eR^n\) nous avons $\sqrt{q(x+y)}\leq\sqrt{q(x)}+\sqrt{q(y)}$. Proposition \ref{PropACHooLtsMUL}.
            \item
                Si \( 1\leq p<\infty\) et si \( f,g\in L^p(\Omega,\tribA,\mu)\) alors \(  \| f+g \|_p\leq \| f \|_p+\| g \|_p\). Proposition \ref{PropInegMinkKUpRHg}.
            \item
                L'inégalité de Minkowsky sous forme intégrale s'écrit sous forme déballée
                \begin{equation}
                    \left[ \int_X\Big( \int_Y| f(x,y) |d\nu(y) \Big)^pd\mu(x) \right]^{1/p}\leq \int_Y\Big( \int_X| f(x,y) |^pd\mu(x) \Big)^{1/p}d\nu(y).
                \end{equation}
                ou sous forme compacte
                \begin{equation}
                    \left\|   x\mapsto\int_Y f(x,y)d\nu(y)   \right\|_p\leq \int_Y  \| f_y \|_pd\nu(y)
                \end{equation}
        \end{enumerate}
    \item[Transformée de Fourier]
                Pour tout \( f\in L^1(\eR^n)\) nous avons \( \| \hat f \|_{\infty}\leq \| f \|_1\), lemme \ref{LEMooCBPTooYlcbrR}.
    \item[Inégalité des normes]
        Inégalité de normes : si \( f\in L^p\) et \( g\in L^1\), alors \( \| f*g \|_p\leq \| f \|_p\| g \|_1\), proposition \ref{PROPooDMMCooPTuQuS}.

\end{description}

  % inégalités

\InternalLinks{connexité}
    \begin{enumerate}
        \item
            Définition \ref{DefIRKNooJJlmiD}
        \item
            Le groupe \( \SL(n,\eK)\) est connexe par arcs : proposition \ref{PROPooALQCooLZCKrH}.
        \item
            Le groupe \( \GL(n,\eC)\) est connexe par arcs : proposition \ref{PROPooVJNIooMByUJQ}.
        \item
            Le groupe \( \GL(n,\eC)\) est connexe par arcs, proposition \ref{PROPooVJNIooMByUJQ}.
        \item
            Le groupe \( \GL(n,\eR)\) a exactement deux composantes connexes par arcs, proposition \ref{PROPooBIYQooWLndSW}.
        \item
            Le groupe \( \gO(n,\eR)\) n'est pas connexe, lemme \ref{LEMooIPOVooZJyNoH}.
        \item
            Les groupe \( \gU(n)\) et \( \SU(n)\) sont connexes par arcs, lemme \ref{LEMooQMXHooZQozMK}.
        \item
            Le groupe \( \SO(n)\) est connexe mais ce n'est pas encore démontré, proposition \ref{PROPooYKMAooCuLtyh}.
        \item 
            Connexité des formes quadratiques de signature donnée, proposition \ref{PropNPbnsMd}.
        \end{enumerate}

  % connexité
\InternalLinks{suite de Cauchy, espace complet}     \label{THMooOCXTooWenIJE}

Nous parlons d'espaces topologiques complets. À ne pas confondre avec un espace mesuré complet, définition~\ref{DefBWAoomQZcI}.

\begin{enumerate}
    \item
        Corps complet : définition~\ref{DefKCGBooLRNdJf}\ref{ITEMooKZZYooDaidGU}, espace métrique complet : définition~\ref{DEFooHBAVooKmqerL}.
    \item
        La définition~\ref{DEFooXOYSooSPTRTn} donne la notion de suite de Cauchy dans un espace métrique.
    \item
        La définition~\ref{DefZSnlbPc} donne la notion de suite de \( \tau\)-Cauchy dans un espace vectoriel topologique.
    \item
        Deux espaces métriques (avec une distance) peuvent être isomorphes en tant qu'espaces topologiques, mais ne pas avoir les mêmes suites de Cauchy, exemple~\ref{EXooNMNVooXyJSDm}.
    \item
        La proposition~\ref{PropooUEEOooLeIImr} donne l'équivalence entre les suites de Cauchy et les suites \( \tau\)-Cauchy dans le cas des espaces vectoriels topologiques \emph{normés}.
    \item
        L'exemple~\ref{EXooNMNVooXyJSDm} est un exemple pire que simplement une suite de Cauchy qui ne converge pas. Le problème de convergence de cette suite n'est pas simplement que la limite n'est pas dans l'espace; c'est que la suite de Cauchy donnée ne convergerait même pas dans \( \eR\).
    \item
        Le théorème~\ref{ThoKHTQJXZ} est un théorème de complétion d'un espace métrique.
\end{enumerate}

Quelque espaces qui sont complets sont listés ci-dessous. Attention : la complétude est bien une propriété de la métrique; le même ensemble peut être complet pout une distance et pas pour une autre. Si on vous demande «est-ce que tel espace est complet ?» vous devez répondre en demandant «pour quelle distance ?» \ldots sauf si elle est évidente dans le contexte.
\begin{multicols}{2}
    \begin{enumerate}
        \item
            Les réels \( \eR\), théorème~\ref{THOooNULFooYUqQYo}.
        \item
            Un espace vectoriel normé sur un corps complet est complet, proposition~\ref{PROPooGJDTooXOoYfw}.
        \item
            La proposition~\ref{PropSYMEZGU} donne quelque espaces complets. Soit \( X\) un espace topologique métrique \( (Y,d)\) un espace espace métrique complet. Alors les espaces
    \begin{enumerate}
        \item
            \( \big( C^0_b(X,Y),\| . \|_{\infty} \big)\)
        \item
            \( \big( C^0_0(X,Y),\| . \|_{\infty} \big)\)
        \item
            \( \big( C^k_0(X,Y),\| . \|_{\infty} \big)\)
    \end{enumerate}
    sont complets.
\item
    Le lemme~\ref{LemdLKKnd} dit que \( \big( C^0(A,B),\| . \|_{\infty}\big)\) est complet dès que \( A\) est compact et \( B\) est complet.

\item
    L'espace \( \swD(K)\) est complet tant pour la topologie des semi-normes que pour la topologie métrique (qui sont les mêmes). C'est la proposition~\ref{PropQAEVcTi}.
\item
    L'espace \( \swS(\Omega)\) est complet et métrisable par la proposition~\ref{PropIIAcyDq}.
    \end{enumerate}
\end{multicols}

    La limite uniforme d'une suite de fonctions dérivables n'est pas spécialement dérivable. Même si les fonctions sont de classe \(  C^{\infty}\), la limite n'est pas spécialement mieux que continue. En effet, le théorème de Stone-Weierstrass~\ref{ThoGddfas} nous dit que les polynômes (qui sont \(  C^{\infty}\)) sont denses dans les fonctions continues sur un compact pour la norme uniforme. Vous ne pouvez donc pas espérer que \( \big( C^p(X,Y),\| . \|_{\infty} \big)\) soit complet en général.
    % Suite de Cauchy, espace complet

\InternalLinks{application réciproque}
\begin{enumerate}
    \item
        Définition~\ref{DEFooTRGYooRxORpY}.
    \item
        Dans le cas des réels, des exemples sont donnés en~\ref{EXooCWYHooLEciVj}.
    \item
        Continuité, proposition~\ref{PropIntContMOnIvCont}.
    \item
        Théorème de la bijection~\ref{ThoKBRooQKXThd} (qui contient aussi de la continuité).
    \item
        Dérivabilité, proposition~\ref{PropMRBooXnnDLq}.
\end{enumerate}
  % application réciproque

\InternalLinks{tribu, algèbre de parties, \( \lambda\)-systèmes et co.}  \label{INTooVDSCooHXLLKp}
    Il existe des centaines de notions de mesures et de classes de parties.
    \begin{enumerate}
        \item
            Le plus souvent lorsque nous parlons de mesure est que nous parlons de mesure positive, définition~\ref{DefBTsgznn} sur un espace mesuré avec une tribu, définition~\ref{DefjRsGSy}.
        \item
            Une mesure extérieure est la définition~\ref{DefUMWoolmMaf}
        \item
            Une algèbre de partie : définition~\ref{DefTCUoogGDud}. Une mesure sur une algèbre de parties : définition~\ref{DefWUPHooEklLmR}. L'intérêt est que si on connait une mesure sur une algèbre de parties, elle se prolonge en une mesure sur la tribu engendrée par le théorème de prolongement de Hahn~\ref{ThoLCQoojiFfZ}.
        \item
            Un \( \lambda\)-système : définition~\ref{DefRECXooWwYgej}.
        \item
            Une mesure complexe : définition~\ref{DefGKHLooYjocEt}.
    \end{enumerate}
  % tribu, algèbre de parties, \( \lambda\)-systèmes et co.

\InternalLinks{déduire la nullité d'une fonction depuis son intégrale}
Des résultats qui disent que si \( \int f=0\) c'est que \( f=0\) dans un sens ou dans un autre.
\begin{enumerate}
    \item
        Il y a le lemme~\ref{Lemfobnwt} qui dit ça.
    \item
        Un lemme du genre dans \( L^2\) existe aussi pour \( \int f\varphi=0\) pour tout \( \varphi\). C'est le lemme~\ref{LemDQEKNNf}.
    \item
        Et encore un pour \( L^p\) dans la proposition~\ref{PropUKLZZZh}.
    \item
        Si \( \int f\chi=0\) pour tout \( \chi\) à support compact alors \( f=0\) presque partout, proposition~\ref{PropAAjSURG}.
    \item
        La proposition~\ref{PropRERZooYcEchc} donne \( f=0\) dans \( L^p\) lorsque \( \int fg=0\) pour tout \( g\in L^q\) lorsque l'espace est \( \sigma\)-fini.
    \item
        Une fonction \( h\in C^{\infty}_c(I)\) admet une primitive dans \(  C^{\infty}_c(I)\) si et seulement si \( \int_Ih=0\). Théorème~\ref{PropHFWNpRb}.
\end{enumerate}
  % déduire la nullité d'une fonction depuis son intégrale

\InternalLinks{équations différentielles}
L'utilisation des théorèmes de point fixe pour l'existence de solutions à des équations différentielles est fait dans le chapitre sur les points fixes.
\begin{enumerate}
    \item
            Le théorème de Schauder a pour conséquence le théorème de Cauchy-Arzela \ref{ThoHNBooUipgPX} pour les équations différentielles.
        \item
            Le théorème de Schauder \ref{ThovHJXIU} permet de démontrer une version du théorème de Cauchy-Lipschitz (théorème \ref{ThokUUlgU}) sans la condition Lipschitz
        \item
            Le théorème de Cauchy-Lipschitz \ref{ThokUUlgU} est utilisé à plusieurs endroits :
            \begin{itemize}
                \item 
                    Pour calculer la transformée de Fourier de \(  e^{-x^2/2}\) dans le lemme \ref{LEMooPAAJooCsoyAJ}.
            \end{itemize}
    \item
        Théorème de stabilité de Lyapunov \ref{ThoBSEJooIcdHYp}.
    \item
        Le système proie prédateurs, Lokta-Voltera \ref{ThoJHCLooHjeCvT}
    \item 
        Équation de Schrödinger, théorème \ref{ThoLDmNnBR}.
    \item 
        L'équation \( (x-x_0)^{\alpha}u=0\) pour \( u\in\swD'(\eR)\), théorème \ref{ThoRDUXooQBlLNb}.
    \item 
        La proposition \ref{PropMYskGa} donne un résultat sur \( y''+qy=0\) à partir d'une hypothèse de croissance.
    \item
        Équation de Hill \( y''+qy=0\), proposition \ref{PropGJCZcjR}.
\end{enumerate}

  % équations différentielles

\InternalLinks{injections}
\begin{enumerate}
        \item
            L'espace de Sobolev \( H^1(I)\) s'injecte de façon compacte dans \( C^0(\bar I)\), proposition~\ref{ThoESIyxfU}.
        \item
            L'espace de Sobolev \( H^1(I)\) s'injecte de façon continue dans \( L^2(I)\), proposition~\ref{ThoESIyxfU}.
        \item
            L'espace \( L^2(\Omega)\) s'injecte continument dans \( \swD'(\Omega)\) (les distributions), proposition~\ref{PROPooYAJSooMSwVOm}.
\end{enumerate}
  % injections
\InternalLinks{logarithme}
\begin{enumerate}
    \item
    Le logarithme pour les réels strictement positifs \( \ln\colon \mathopen] 0 , \infty \mathclose[\to \eR\) est donné en la définition~\ref{DEFooELGOooGiZQjt}.
    \item
        Les principales propriétés sont dans la proposition \ref{PROPooLAOWooEYvXmI} : \( \ln(xy)=\ln(x)+\ln(y)\) etc.
    \item
        La proposition \ref{PROPooKPBIooJdNsqX} donne la série
        \begin{equation}
            \ln(1+x)=\sum_{k=1}^{\infty}\frac{ (-1)^{k+1} }{ k }x^k.
        \end{equation}
    \item
        L'exemple \ref{EXooYMEEooMGpUNM} donne l'encadrement \( 0.644\leq \ln(2)\leq 0.846\).
    \item
        La proposition~\ref{PropKKdmnkD} dit que toute matrice complexe admet un logarithme. En particulier une série explicite est donné pour le logarithme d'un bloc de Jordan.
    \item
        Sur les complexes, le logarithme \( \ln \colon \eC^*\to \eC\) est la définition~\ref{DEFooWDYNooYIXVMC}. Attention : ce n'est pas la seule définition possible.
    \item
        La série harmonique diverge à vitesse logarithmique, et la série des inverses des nombres premiers, c'est encore plus lent : théorème~\ref{ThonfVruT}.
\end{enumerate}
  % logarithmes
\InternalLinks{inversion locale, fonction implicite}
       \begin{description}
           \item[Des énoncés]
               \begin{enumerate}
    \item Inversion locale dans \( \eR^n\) : théorème~\ref{THOooQGGWooPBRNEX}. Pour un Banach c'est le théorème~\ref{ThoXWpzqCn}.
    \item
        Fonction implicite dans un Banach : théorème~\ref{ThoAcaWho}.
               \end{enumerate}
           \item[Des utilisations]
               \begin{enumerate}

    \item
        Utilisé pour montrer que le flot d'une équation différentielle est un \( C^p\)-difféomorphisme local, voir~\ref{NORMooWEWVooXbGmfE}. % position 1051229132
    \item
        Pour le théorème de Von Neumann~\ref{ThoOBriEoe}.
               \end{enumerate}
       \end{description}
 % inversion locale, fonction implicite
\InternalLinks{Convexité}

L'essentiel des résultats sur les fonctions convexes sont dans la sections \ref{SECooVZWWooUjxXYi}.

\begin{enumerate}
    \item
        Définition des fonctions convexes : \ref{DefVQXRJQz} et \ref{DEFooKCFPooLwKAsS} en dimension supérieure.
    \item 
        En termes de différentielles, \ref{PROPooYNNHooSHLvHp} pour la différentielle première et \ref{CORooMBQMooWBAIIH} pour la Hessienne.
    \item
        Une courbe paramétrée convexe est la définition \ref{DEFooVQODooJSNYLw}.
    \item
        L'enveloppe convexe d'une courbe fermée simple et convexe : \ref{PROPooWZITooTFiWsi}.
    \item 
        Courbure et convexité d'une courbe paramétrée : section \ref{SUBSECooNJOLooYuGRjA}.
    \item
        Une courbe paramétrée convexe est localement le graphe d'une fonction convexe, lemme \ref{LEMooGEVEooHxPTMO}.
\end{enumerate}

L'

 % convexité

% ANALYSES FONCTIONNELLE

\InternalLinks{dualité}     \label{THEMEooULGFooPscFJC}

Ne pas confondre dual algébrique et dual topologique d'un espace vectoriel.

\begin{description}
    \item[Dual topologique et algébrique]
        Ils sont définis par \ref{DefJPGSHpn}. Le dual algébrique est l'ensemble des formes linéaires, et le dual topologique ne considère que les formes linéaires continues (en dimension infinie, les applications linéaires ne sont pas toutes continues).
    \item[Topologie]
        Une topologie possible sur le dual d'un espace vectoriel topologique est celle \( *\)-faible de la définition \ref{DefHUelCDD}.

        Nous comparons les topologies faibles et de la norme en la section \ref{SECooKOJNooQVawFY}.
    \item[Théorèmes de dualité]
        Quelque théorèmes établissent des dualités entre des espaces courants.
\begin{enumerate}
    \item
        Le théorème de représentation de Riesz \ref{ThoQgTovL} pour les espaces de Hilbert.
    \item
        La proposition \ref{PropOAVooYZSodR} pour les espaces \( L^p\big( \mathopen[ 0 , 1 \mathclose] \big)\) avec \( 1<p<2\).
    \item
        Le théorème de représentation de Riesz \ref{ThoSCiPRpq} pour les espaces \( L^p\) en général.
\end{enumerate}
Tous ces théorèmes donnent la dualité par l'application \( \Phi_x=\langle x, .\rangle \).

\end{description}
  % dualité

\InternalLinks{opérations sur les distributions}
\begin{enumerate}
    \item
        Convolution d'une distribution par une fonction, définition par l'équation \eqref{EQooOUXKooGHDSzL}.
    \item
        Dérivation d'une distribution, proposition-définition~\ref{PropKJLrfSX}.
    \item
        Produit d'une distribution par une fonction, définition~\ref{DefZVRNooDXAoTU}.
\end{enumerate}
  % opérations sur les distributions

\InternalLinks{transformée de Fourier}
\begin{enumerate}
    \item
        Définition sur \( L^1\), définition \ref{DEFooRIXGooECoIbx}.
    \item
        La transformée de Fourier d'une fonction \( L^1(\eR^d)\) est continue, proposition \ref{PropJvNfj}.
    \item
    L'espace de Schwartz est stable par transformée de Fourier. L'application $\TF\colon \swS(\eR^d)\to \swS(\eR^d)$ est une bijection linéaire et continue. Proposition  \ref{PropKPsjyzT}
\end{enumerate}

  % transformée de Fourier

\InternalLinks{convolution}
\begin{enumerate}
    \item
        Définition pour \( f,g\in L^1\), théorème~\ref{THOooMLNMooQfksn}.
    \item
        Inégalité de normes : si \( f\in L^p\) et \( g\in L^1\), alors \( \| f*g \|_p\leq \| f \|_p\| g \|_1\), proposition~\ref{PROPooDMMCooPTuQuS}.
    \item
        \( \varphi\in L^1(\eR)\) et \( \psi\in\swS(\eR)\), alors \( \varphi * \psi\in \swS(\eR)\), proposition~\ref{PROPooUNFYooYdbSbJ}.
    \item
        Les suites régularisantes : \( \lim_{n\to \infty} \rho_n*f=f\) dans la proposition~\ref{PROPooYUVUooMiOktf}.
    \item
        Convolution d'une distribution par une fonction, définition par l'équation \eqref{EQooOUXKooGHDSzL}.
\end{enumerate}
  % convolution

% NUMÉRIQUE

\InternalLinks{méthode de Newton}
    \begin{enumerate}
        \item
            Nous parlons un petit peu de méthode de Newton en dimension \( 1\) dans~\ref{SECooIKXNooACLljs}.
        \item
            La méthode de Newton fonctionne bien avec les fonctions convexes par la proposition~\ref{PROPooVTSAooAtSLeI}.
        \item
            La méthode de Newton en dimension $n$ est le théorème~\ref{ThoHGpGwXk}.
       \item
            Un intervalle de convergence autour de \( \alpha\) s'obtient par majoration de \( | g' |\), proposition~\ref{PROPooRPHKooLnPCVJ}.
       \item
           Un intervalle de convergence quadratique s'obtient par majoration de \( | g'' |\), théorème~\ref{THOooDOVSooWsAFkx}.
       \item
           En calcul numérique, section~\ref{SECooIKXNooACLljs}.
       \end{enumerate}
  % méthode de Newton

\InternalLinks{méthodes de calcul}
\begin{enumerate}
    \item
        Théorème de Rothstein-Trager~\ref{ThoXJFatfu}.
    \item
        Algorithme des facteurs invariants~\ref{PropPDfCqee}.
    \item
        Méthode de Newton, théorème~\ref{ThoHGpGwXk}
    \item
        Calcul d'intégrale par suite équirépartie~\ref{PropDMvPDc}.
\end{enumerate}
  % méthodes de calcul

% ALGÈBRE
\InternalLinks{définie positive}        \label{THEMEooYEVLooWotqMY}
\begin{enumerate}
    \item
        Une application bilinéaire est définie positive lorsque \( g(u,u)\geq 0\) et \( g(u,u)=0\) si et seulement si \( u=0\) est la définition \ref{DEFooJIAQooZkBtTy}.
    \item
        Un opérateur ou une matrice est défini positif si toutes ses valeurs propres sont positives, c'est la définition \ref{DefAWAooCMPuVM}.
    \item
        Pour une matrice symétrique, définie positive implique \( \langle Ax, x\rangle >0\) pour tout \( x\). C'est le lemme \ref{LemWZFSooYvksjw}.
    \item
        Une application linéaire est définie positive si et seulement si sa matrice associée l'est. C'est la proposition \ref{PROPooUAAFooEGVDRC}.
\end{enumerate}
Remarque : nous ne définissons pas la notion de matrice définie positive dans le cas d'une matrice non symétrique.

  % définie positive

        \InternalLinks{norme matricielle et rayon spectral}     \label{THEMEooOJJFooWMSAtL}
    \begin{enumerate}
        \item
            Définition du rayon spectral \ref{DEFooEAUKooSsjqaL}.
        \item
            Lien entre norme matricielle et rayon spectral, le théorème \ref{THOooNDQSooOUWQrK} assure que $\|A\|_2=\sqrt{\rho(A{^t}A)}$.
        \item
            Pour toute norme algébrique nous avons \( \rho(A)\leq \| A \|\), proposition \ref{PropEDvSQsA}\ref{ITEMooVQQFooWrHWeO}.
        \item
            Dans le cadre du conditionnement de matrice. Voir en particulier la proposition \ref{PROPooNUAUooIbVgcN} qui utilise le théorème \ref{THOooNDQSooOUWQrK}.
    \end{enumerate}

  % norme matricielle et rayon spectral
\InternalLinks{Série de matrices}       \label{THEMEooPQKDooTAVKFH}

\begin{enumerate}
    \item
        Rayon spectral et norme opérateur : thème \ref{THEMEooOJJFooWMSAtL}.
    \item
        Exponentielle de matrices : thème \ref{THEMEooKXSGooCsQNoY}.
    \item
        Série entière de matrices : section \ref{secEVnZXgf}.
    \item
        Pour la série \( \sum_kA^k=(1-A)^{-1}\).
        \begin{itemize}
            \item Pour un espace de Banach : proposition \ref{PropQAjqUNp}.
            \item Pour les matrices nilpotentes : proposition \ref{PROPooWTFWooXHlmhp}.
            \item En lien avec le rayon spectral (si et seulement si \( \rho(A)<1\)) dans la proposition \ref{PROPooIUUWooQTRtkg}.
            \item Le lemme \ref{LemPQFDooGUPBvF} parle de la série entière \( \sum_{n\in \eN}z^{nk}=(1-z^k)^{-1}\).
        \end{itemize}
\end{enumerate}
  % séries de matrices

\InternalLinks{rang}
    \begin{enumerate}
        \item Définition~\ref{DefALUAooSPcmyK}.
        \item Le théorème du rang, théorème~\ref{ThoGkkffA}
        \item Prouver que des matrices sont équivalentes et les mettre sous des formes canoniques, lemme~\ref{LemZMxxnfM} et son corollaire~\ref{CorGOUYooErfOIe}.
        \item Tout hyperplan de \( \eM(n,\eK)\) coupe \( \GL(n,\eK)\), corollaire~\ref{CorGOUYooErfOIe}. Cela utilise la forme canonique sus-mentionnée.
        \item Le lien entre application duale et orthogonal de la proposition~\ref{PropWOPIooBHFDdP} utilise la notion de rang.
        \item Prouver les équivalences à être un endomorphisme cyclique du théorème~\ref{THOooGLMSooYewNxW} via le lemme~\ref{LEMooDFFDooJTQkRu}.
        \end{enumerate}
  % rang

\InternalLinks{extension de corps et polynômes} \label{THEMEooZYKFooQXhiPD}
    \begin{enumerate}
        \item
            Définition d'une extension de corps~\ref{DEFooFLJJooGJYDOe}.
        \item
            Pour l'extension du corps de base d'un espace vectoriel et les propriétés d'extension des applications linéaires, voir la section~\ref{SECooAUOWooNdYTZf}.
        \item
            Extension de corps de base et similitude d'application linéaire (ou de matrices, c'est la même chose), théorème~\ref{THOooHUFBooReKZWG}.
        \item
            Extension de corps de base et cyclicité des applications linéaires, corolaire~\ref{CORooAKQEooSliXPp}.
        \item
            À propos d'extensions de \( \eQ\), le lemme~\ref{LemSoXCQH}.
        \item
            Corps de rupture : définition~\ref{DEFooVALTooDJJmJv} existence par la proposition~\ref{PROPooUBIIooGZQyeE}. Il n'y a pas unicité.
        \item
            Corps de décomposition : définition~\ref{DEFooEKGZooSkvbum}. Attention : le plus souvent nous parlons de corps de décomposition d'un seul polynôme. Cette définition est un peu surfaite. Existence par la proposition~\ref{PROPooDPOYooFHcqkU} qui le donne même comme extension par toutes les racines, et unicité à isomorphisme près par le théorème~\ref{THOooQVKWooZAAYxK}, énoncé de façon plus simple (mais pas indépendante !) en la proposition~\ref{PropTMkfyM}.
    \end{enumerate}

Un trio de résultats d'enfer est :
\begin{enumerate}
    \item
        Dans un anneau principal qui n'est pas un corps, un idéal est maximal si et seulement si il est engendré par un irréductible (proposition~\ref{PropomqcGe}).
    \item
        Dans un anneau, un idéal \( I\) est maximam si et seulement si \( A/I\) est un corps (proposition~\ref{PROPooSHHWooCyZPPw})
    \item
        Si \( \eK\) est un corps, \( \eK[X]\) est principal (lemme~\ref{LEMooIDSKooQfkeKp}).
\end{enumerate}
  % extension de corps et polynômes

\InternalLinks{décomposition de matrices}   \label{DECooWTAIooNkZAFg}
\begin{enumerate}
    \item
        Décomposition de Bruhat, théorème~\ref{ThoizlYJO}.
    \item
        Décomposition de Dunford, théorème~\ref{ThoRURcpW}.
    \item
        Décomposition polaire~\ref{ThoLHebUAU} et la proposition~\ref{PropWCXAooDuFMjn} pour la régularité.
\end{enumerate}
  % décomposition de matrices

\InternalLinks{systèmes d'équations linéaires}
\begin{itemize}
    \item Algorithme des facteurs invariants \ref{PropPDfCqee}.
    \item Méthode du gradient à pas optimal \ref{PropSOOooGoMOxG}.
\end{itemize}

  % systèmes d'équations linéaires

\InternalLinks{formes bilinéaires et quadratiques}
    \begin{enumerate}
\item
    Les formes bilinéaires sont définies en \ref{DEFooEEQGooNiPjHz}.
\item
    Forme quadratique, définition \ref{DefBSIoouvuKR}
\end{enumerate}

  % formes bilinéaires et quadratiques
\InternalLinks{arithmétique modulo, théorème de Bézout} \label{THEMEooNRZHooYuuHyt}
    \begin{enumerate}
        \item
            Pour \( \eZ^*\) c'est le théorème~\ref{ThoBuNjam}.
        \item
            Théorème de Bézout dans un anneau principal : corollaire~\ref{CorimHyXy}.
        \item
            Théorème de Bézout dans un anneau de polynômes : théorème~\ref{ThoBezoutOuGmLB}.
        \item
            En parlant des racines de l'unité et des générateurs du groupe unitaire dans le lemme~\ref{LemcFTNMa}. Au passage nous y parlerons de solfège.
        \item
            La proposition~\ref{PropLAbRSE} qui donne tout entier assez grand comme combinaisons de \( a \) et \( b\) à coefficients positifs est utilisée en chaînes de Markov, voir la définition~\ref{DefCxvOaT} et ce qui suit.
        \item
            PGCD et PPCM sont dans la définition \ref{DefrYwbct}.
        \item
            Calcul effectif du PGCD puis des coefficients de Bézout : sous-sections~\ref{SUBSECooAEBLooFGJRkg} et~\ref{SUBSECooRHSQooEuBWbd}.
        \end{enumerate}
  % théorème de Bézout
\InternalLinks{polynômes}

\begin{description}
    \item[Définitions]
        Soient un anneau \( A\), un corps \( \eK\), une extension \( \eL\) de \( \eK\) et un élément \( \alpha\in \eL\).
        \begin{enumerate}
            \item
                \( A[X]\), définition~\ref{DefRGOooGIVzkx}; l'anneau \( \eK[X]\) a même définition parce que c'est un cas particulier. L'évaluation d'un polynôme en un élément de l'anneau, \( P(\alpha)\) est définie en~\ref{DEFooOSWQooHYYwVE}.
            \item
                Liens entre \( \eK[\alpha]\), \( \eK[X]\), \( \eK(\alpha)\) et \( \eK(X)\) lorsque \( \alpha\) est transcendant, proposition~\ref{PROPooSYQWooFbfQtm}.

                Et la proposition~\ref{PropURZooVtwNXE} pour le cas où \( \alpha\) est algébrique.
            \item
                Si \( A\) est un anneau et si \( \alpha\) est un élément d'une extension de \( A\) (comme anneau), nous écrivons \( A[\alpha]\) pour le plus petit sous-anneau de \( B\) contenant \( A\) et \( \alpha\). C'est la définition~\ref{DEFooRFBFooKCXQsv}.
            \item
                \( \eK(X)\), le corps des fractions de \( \eK[X]\), définition~\ref{DEFooQPZIooQYiNVh}. Si \( R=P/Q\) dans \( \eK(X)\), l'évaluation est \( R(\alpha)=P(\alpha)Q(\alpha)^{-1}\), définition~\ref{DEFooZHBZooKlNfGZ}.
            \item
                \( \eK(\alpha)_{\eL}\) est le plus petit corps de \( \eL\) contenant \( \eK\) et \( \alpha\), définition~\ref{DEFooVSKGooMyeGel}.
            \item
                À propos de polynômes à plusieurs variables.
                \begin{itemize}
                    \item Anneau de polynômes : \( A[X_1,\ldots, X_n]\) est la définition~\ref{DEFooZNHOooCruuwI}. 
                    \item Corps engendré : \( \eK(\alpha_1,\ldots, \alpha_n)\) est la définition~\ref{DEFooRHRKooPqLNOp}. 
                    \item Corps des fractions rationnelles : \( \eK(X_1,\ldots, X_n)\) est la définition~\ref{DEFooOCPHooXneutp}.
                \end{itemize}
        \end{enumerate}

    \item[Coefficients dans un anneau commutatif]

        \begin{enumerate}
            \item
Les polynômes à coefficients dans un anneau commutatif  sont à la section~\ref{SECooVMABooVdhbPo}.
        \end{enumerate}


    \item[Coefficients dans un corps]
Les polynômes à coefficients dans un corps sont à la section~\ref{SECooFYOGooQHitgE}.
        \begin{enumerate}

\item
Nous parlons de l'idéal des polynômes annulateurs dans le théorème~\ref{ThoCCHkoU}.
            \item
                Le théorème~\ref{ThoCCHkoU} dit que \( \eK[X]\) est un anneau principal et que tous ses idéaux sont engendrés par un unique polynôme unitaire.
            \item
                Le polynôme minimal est irréductible, proposition~\ref{PropRARooKavaIT}.
            \item
                Quelques formules sur le \( \pgcd\), lemme~\ref{LemUELTuwK}.
        \end{enumerate}
    \item[Polynôme primitif]

        \begin{enumerate}
            \item
                Un polynôme est irréductible sur \( A\) si et seulement si irréductible et primitif sur le corps des fractions, corolaire~\ref{CORooZCSOooHQVAOV}.
        \end{enumerate}

    \item[Polynôme d'endomorphisme]
        C'est la section~\ref{SECooUEQVooLBrRiE}.

    \item[Racines et factorisation]

    \begin{enumerate}
        \item
            Si \( \eA\) est un anneau, la proposition~\ref{PropHSQooASRbeA} factorise une racine.
        \item
            Si \( \eA\) est un anneau, la proposition~\ref{PropahQQpA} factorise une racine avec sa multiplicité.
        \item
            Si \( \eA\) est un anneau, le théorème~\ref{ThoSVZooMpNANi} factorise plusieurs racines avec leurs multiplicités.
        \item
            Le théorème~\ref{ThoSVZooMpNANi} nous indique que lorsqu'on a autant de racines (multiplicité comprise) que le degré, alors nous avons toutes les racines.
        \item
            Si \( \eK\) est un corps et \( \alpha\) une racine dans une extension, le polynôme minimal de \( \alpha\) divise tout polynôme annulateur par la proposition~\ref{PropXULooPCusvE}.
        \item
            Le théorème~\ref{ThoLXTooNaUAKR} annule un polynôme de degré \( n\) ayant \( n+1\) racines distinctes.
        \item
            La proposition~\ref{PropTETooGuBYQf} nous annule un polynôme à plusieurs variables lorsqu'il a trop de racines.
        \item
            En analyse complexe, le principe des zéros isolés~\ref{ThoukDPBX} annule en gros toute série entière possédant un zéro non isolé.
        \item
            Polynômes irréductibles sur \( \eF_q\).
        \end{enumerate}

\end{description}
 % polynômes

% ENDOMORPHISMES

\InternalLinks{invariants de similitude}
    \begin{enumerate}
        \item
            Théorème \ref{THOooDOWUooOzxzxm}.
        \item
            Pour prouver que la similitude d'applications linéaires résiste à l'extension du corps de base, théorème \ref{THOooHUFBooReKZWG}.
        \item
            Pour prouver que la dimension du commutant d'un endomorphisme de \( E\) est de dimension au moins \( \dim(E)\), lemme \ref{LEMooDFFDooJTQkRu}.
        \item
            Nous verrons dans la remarque \ref{REMooPVLEooYDRXQI} à propos des invariants de similitude que toute matrice est semblable à la matrice bloc-diagonale constituées des matrices compagnon (définition \ref{DEFooOSVAooGevsda}) de la suite des polynômes minimals.
        \end{enumerate}

  % invariants de similitude

\InternalLinks{diagonalisation}
    Des résultats qui parlent diagonalisation
    \begin{enumerate}
        \item
            Définition d'un endomorphisme diagonalisable :~\ref{DefCNJqsmo}.
        \item
            Conditions équivalentes au fait d'être diagonalisable en termes de polynôme minimal, y compris la décomposition en espaces propres : théorème~\ref{ThoDigLEQEXR}.
        \item
            Diagonalisation simultanée~\ref{PropGqhAMei}, pseudo-diagonalisation simultanée~\ref{CorNHKnLVA}.
        \item
            Diagonalisation d'exponentielle~\ref{PropCOMNooIErskN} utilisant Dunford.
        \item
            Décomposition polaire théorème~\ref{ThoLHebUAU}. \( M=SQ\), \( S\) est symétrique, réelle, définie positive, \( Q\) est orthogonale.
        \item
            Décomposition de Dunford~\ref{ThoRURcpW}. \( u=s+n\) où \( s\) est diagonalisable et \( n\) est nilpotent, \( [s,n]=0\).
        \item
            Réduction de Jordan (bloc-diagonale)~\ref{ThoGGMYooPzMVpe}.
        \item
            L'algorithme des facteurs invariants~\ref{PropPDfCqee} donne \( U=PDQ\) avec \( P\) et \( Q\) inversibles, \( D\) diagonale, sans hypothèse sur \( U\). De plus les éléments de \( D\) forment une chaîne d'éléments qui se divisent l'un l'autre.
        \end{enumerate}
        Le théorème spectral et ses variantes :
        \begin{enumerate}
            \item
                Théorème spectral, matrice symétrique, théorème~\ref{ThoeTMXla}. Via le lemme de Schur.
            \item
                Théorème spectral autoadjoint (c'est le même, mais vu sans matrices), théorème~\ref{ThoRSBahHH}
            \item
                Théorème spectral hermitien, lemme~\ref{LEMooVCEOooIXnTpp}.
            \item
                Théorème spectral, matrice normales, théorème~\ref{ThogammwA}.
            \end{enumerate}
        Pour les résultats de décomposition dont une partie est diagonale, voir le thème~\ref{DECooWTAIooNkZAFg} sur les décompositions.
  % diagonalisation

\InternalLinks{endomorphismes cycliques}
    \begin{enumerate}
        \item
            Définition \ref{DEFooFEIFooNSGhQE}.
        \item
            Son lien avec le commutant donné dans la proposition \ref{PropooQALUooTluDif} et le théorème \ref{THOooGLMSooYewNxW}.
        \item
            Utilisation dans le théorème de Frobenius (invariants de similitude), théorème \ref{THOooDOWUooOzxzxm}.
        \end{enumerate}

  % endomorphismes cycliques

\InternalLinks{déterminant}     \label{THMooUXJMooOroxbI}
    \begin{enumerate}
        \item
            Déterminant d'une matrice : définition \ref{DEFooYCKRooTrajdP}.
        \item
            Déterminant et manipulations de lignes et colonnes, section \ref{SUBSECooKMSVooBBHwkH} et les propositions qui précèdent à partir du lemme \ref{LEMooCEQYooYAbctZ} qui dit que \( \det(A)=\det(A^t)\).
    \item
        Les \( n\)-formes alternées forment un espace de dimension \( 1\), proposition~\ref{ProprbjihK}.
    \item
        Déterminant d'une famille de vecteurs~\ref{DEFooODDFooSNahPb}.
    \item
        Calcul d'un déterminant de taille \( 2\times 2\) : équation \eqref{EQooQRGVooChwRMd}.
    \item
        Déterminant d'un endomorphisme~\ref{DefCOZEooGhRfxA}.
    \item
        Interprétations géométriques
            \begin{enumerate}
        \item
            À propos d'orthogonalité, le déterminant est très lié au produit vectoriel en dimension \( 3\). Et il le généralise en dimension supérieure.
            \begin{enumerate}
                \item
            Liaison au produit vectoriel (orthogonalité) dans la proposition~\ref{PROPooTUVKooOQXKKl}.
        \item
            En particulier le lemme~\ref{LEMooFRWKooVloCSM} nous dit comment un déterminant donne un vecteur orthogonal à une famille donnée de vecteurs.
            \end{enumerate}
        \item
            Déterminant et aires, volumes
            \begin{enumerate}
                \item
            Déterminant et mesure de Lebesgue : théorème~\ref{ThoBVIJooMkifod}.
                \item
            Aire du parallélogramme : proposition~\ref{PropNormeProdVectoabsint}.
        \item
            Volume du parallélépipède avec le produit mixte et le déterminant \( 3\times 3\),~\ref{NORMooWWOKooWzScnZ}.
            \end{enumerate}
        \end{enumerate}
        Tant que nous en sommes dans les interprétations géométrique, il faut lier déterminant, produit vectoriel, orthogonalité et mesure en notant que l'élément de volume lors de l'intégration en dimension \( 3\) est donné par \eqref{EQooNYWSooZuvcPe} : \( dS=\| T_u\times T_v \|\) qui est la norme du produit vectoriel des vecteurs tangents à la paramétrisation.

        Nous voyons dans l'équation \eqref{EQooARMAooQPhQAL} que l'élément de volume pour une partie de dimension \( n\) dans \( \eR^m\) (à l'occasion d'y intégrer une fonction) est donné par un déterminant mettant en jeu les vecteurs tangents de la paramétrisation.
        \item
            Le déterminant de Vandermonde est à la proposition~\ref{PropnuUvtj}. Il est utilisé à divers endroits :
\begin{enumerate}
    \item
        Pour prouver que \( u\) est nilpotente si et seulement si \( \tr(u^p)=0\) pour tout \( p\) (lemme \ref{LemzgNOjY})
    \item
        Pour prouver qu'un endomorphisme possédant \( \dim(E)\) valeurs propres distinctes est cyclique (proposition~\ref{PropooQALUooTluDif}).
\end{enumerate}

   \end{enumerate}
  % déterminant

\InternalLinks{polynôme d'endomorphismes}
    \begin{enumerate}
    \item Endomorphismes cycliques et commutant dans le cas diagonalisable, proposition \ref{PropooQALUooTluDif}.
    \item Racine carré d'une matrice hermitienne positive, proposition \ref{PropVZvCWn}.
    \item Théorème de Burnside sur les sous-groupes d'exposant fini de \( \GL(n,\eC)\), théorème \ref{ThooJLTit}.
    \item Décomposition de Dunford, théorème \ref{ThoRURcpW}. 
    \item Algorithme des facteurs invariants \ref{PropPDfCqee}.
    \end{enumerate}

  % polynôme d'endomorphismes
\InternalLinks{exponentielle}        \label{THEMEooKXSGooCsQNoY}

Toutes les exponentielles sont définies par la série
\begin{equation}
    \exp(x)=\sum_{k=0}^{\infty}\frac{ x^k }{ k! },
\end{equation}
tant que la somme ait un sens.

\begin{description}
    \item[Réels]

    \item[Complexes]

        \begin{enumerate}
            \item
                Le fait que \(  e^{i\theta}\) donne tous les nombres complexes de norme \( 1\) est la proposition~\ref{PROPooZEFEooEKMOPT}.
            \item
                Le groupe des racines de l'unité est donné par l'équation \eqref{EqIEAXooIpvFPe}.
        \end{enumerate}

    \item[Algèbre normée commutative]

        Pour la définition c'est la proposition~\ref{DEFooSFDUooMNsgZY} et pour la régularité \(  C^{\infty}\) c'est la proposition~\ref{PROPooTBDAooQouzSk}.

    \item[Idem non commutatif]

        Il y a une tentative de théorème~\ref{THOooFGTQooZPiVLO}, mais c'est principalement pour les matrices qu'il y a des résultats.

    \item[Matrices]

        De nombreux résultats sont disponibles pour les exponentielles de matrices.

\begin{enumerate}
    \item
        La section~\ref{secAOnIwQM} parle d'exponentielle de matrices.
    \item
        L'exponentielle donne lieu à une fonction de classe \(  C^{\infty}\), proposition~\ref{PropXFfOiOb}.
    \item
            Le lemme à propos d'exponentielle de matrice~\ref{LemQEARooLRXEef} donne :
            \begin{equation}
                \|  e^{tA} \|\leq P\big( | t | \big)\sum_{i=1}^r e^{t\real(\lambda_i)}.
            \end{equation}
        \item
            La proposition~\ref{PropCOMNooIErskN} : si \( A\in \eM(n,\eR)\) a un polynôme caractéristique scindé, alors \( A\) est diagonalisable si et seulement si \( e^A\) est diagonalisable.
\item
    La section~\ref{subsecXNcaQfZ} parle des fonctions exponentielle et logarithme pour les matrices. Entre autres la dérivation et les séries.
\item
    Pour résoudre des équations différentielles linéaires : sous-section~\ref{SUBSECooMDKIooKaaKlZ}.
\item
    La proposition~\ref{PropKKdmnkD} dit que l'exponentielle est surjective sur \( \GL(n,\eC)\).
\item

La proposition~\ref{PropFMqsIE} : si \( u\) est un endomorphisme, alors \( \exp(u)\) est un polynôme en \( u\).
\item
    Calcul effectif : sous-section~\ref{SUBSECooGAHVooBRUFub}.
\item Proposition~\ref{PROPooZUHOooQBwfZq} : si \( A\in\eM(n,\eC)\) alors $ e^{\tr(A)}=\det( e^{A}).$
    \item
        Les séries entières de matrices sont traitées autour de la proposition~\ref{PropFIPooSSmJDQ}.
\end{enumerate}


\end{description}
  % exponentielle

% GÉOMÉTRIE
\InternalLinks{types d'anneaux}

\begin{enumerate}
    \item
        \( \eZ\) est intègre, exemple \ref{EXooLDXRooSxUAXs}, principal et euclidient (proposition \ref{PROPooPJGLooQSrJTU}).
    \item
        \( \eZ[X]\) n'est pas principal (voir \ref{ITEMooNQQMooSnuKvW}).
    \item   \label{ITEMooNQQMooSnuKvW}
        Si \( A\) est un anneau intègre qui n'est pas un corps, alors \( A[X]\) n'est pas principal, lemme \ref{LEMooDJSUooJWyxCL}.
    \item
        L'anneau des fonctions holomorphes sur un compact donné est principal, proposition \ref{PROPooVWRPooGQMenV}.
    \item
        L'anneau \( \eZ[i\sqrt{ 3 }]\) n'est pas factoriel, exemple \ref{EXooCWJUooCDJqkr}.
    \item 
        L'anneau \( \eZ[i\sqrt{ 5 }]\) n'est ni factoriel ni principal, exemple \ref{EXooYCTDooGXAjGC}.
\end{enumerate}
  % types d'anneaux

% GROUPES

\InternalLinks{sous-groupes}
\begin{enumerate}
    \item 
        Théorème de Burnside sur les sous-groupes d'exposant fini de \( \GL(n,\eC)\), théorème \ref{ThooJLTit}.
    \item 
        Sous-groupes compacts de \( \GL(n,\eR)\), lemme \ref{LemOCtdiaE} ou proposition \ref{PropQZkeHeG}.
\end{enumerate}

  % sous-groupes
\InternalLinks{groupe symétrique}       \label{THEMEooQEEWooXDhvhv}

\begin{enumerate}
    \item
        Définition \ref{DEFooJNPIooMuzIXd}.
    \item
        La table des caractères du groupe symétrique \( S_4\) est donné dans la section \ref{SecUMIgTmO}.
    \item
        Le groupe symétrique \( S_4\) est le groupe des syémtries affines du tétraèdre régulier, proposition \ref{PROPooVNLKooOjQzCj}.
    \item
        Le groupe alterné \( A_5\) est l'unique groupe simple d'ordre \( 60\), proposition \ref{PROPooUBIWooTrfCat}.
\end{enumerate}
 % groupe symétrique
\InternalLinks{action de groupe}    \label{THEMEooKZHBooRCULcr}
    \begin{enumerate}
        \item Définition d'une action de groupe sur un ensemble : \ref{DefActionGroupe}.
    \item Action du groupe modulaire sur le demi-plan de Poincaré, théorème \ref{ThoItqXCm}. 
    \end{enumerate}

  % action de groupe

\InternalLinks{classification de groupes}
\begin{enumerate}
    \item Structure des groupes d'ordre \( pq\), théorème \ref{ThoLnTMBy}.
    \item Le groupe alterné est simple, théorème \ref{ThoURfSUXP}.
    \item Théorème de Sylow \ref{ThoUkPDXf}. Tout le théorème, c'est un peu long. On peut se contenter de la partie qui dit que \( G\) contient un \( p\)-Sylow.
    \item Théorème de Burnside sur les sous groupes d'exposant fini de \( \GL(n,\eC)\), théorème \ref{ThooJLTit}.
    \item \( (\eZ/p\eZ)^*\simeq \eZ/(p-1)\eZ\), corollaire \ref{CorpRUndR}.
\end{enumerate}

  % classification de groupes


\InternalLinks{produit semi-direct de groupes}
    \begin{enumerate}
        \item
            Définition \ref{DEFooKWEHooISNQzi}.         % Ce commentaire sert à rendre cette ligne unique ooQLCCooQfibcp
        \item
            Le corollaire \ref{CoroGohOZ} donne un critère pour prouver qu'un produit \( NH\) est un produit semi-direct.
        \item
            L'exemple \ref{EXooHNYYooUDsKnm} donne le groupe des isométries du carré comme un produit semi-direct.
        \item
            Le théorème \ref{ThoLnTMBy} classifie les groupes d'ordre \( pq\) (\( p\), \( q\) premiers distincts) à grands coups de produit semi-directs.
        \item
            Le théorème \ref{THOooQJSRooMrqQct} donne les isométries de \( \eR^n\) par \( \Isom(\eR^n)=T(n)\times_{\rho} O(n)\) où \( T(n)\) est le groupe des translations.
        \item
            La proposition \ref{PROPooDHYWooXxEXvl} donne une décomposition du groupe orthogonal \( \gO(n)=\SO(n)\times_{\rho} C_2\) où \( C_2=\{ \id,R \}\) où \( R\) est de déterminant \( -1\).
        \item
            La proposition \ref{PROPooTPFZooKtFxhg} donne \( \Aff(\eR^n)=T(n)\times_{\rho}\GL(n,\eR)\) où \( \Aff(\eR^n)\) est le groupe des applications affines bijectives de \( \eR^n\).
        \end{enumerate}
  % produit semi-direct de groupes

\InternalLinks{théorie des représentations}
\begin{enumerate}
    \item Table des caractères du groupe diédral, section \ref{SecWMzheKf}.
    \item Table des caractères du groupe symétrique \( S_4\), section \ref{SecUMIgTmO}.
\end{enumerate}

  % théorie des représentations

\InternalLinks{isométries}      \label{THMooVUCLooCrdbxm}

Ne pas confondre une isométrie d'un espace affine avec une isométrie d'un espace euclidien. Les isométrie d'un espace euclidien préservent le produit scalaire et fixent donc l'origine (lemme~\ref{LEMooYXJZooWKRFRu}). Les isométrie des espaces affines par contre conservent les distances (définition~\ref{DEFooZGKBooGgjkgs}) et peuvent donc déplacer l'origine de l'espace vectoriel sur lequel il est modelé; typiquement les translations sont des isométries de l'espace affine mais pas de l'espace euclidien.

Parfois, lorsqu'on coupe les cheveux en quatre, il faut faire attention en parlant de \( \eR^n\) : soit on en parle comme d'un espace métrique (muni de la distance), soit on en parle comme d'un espace normé (muni de la norme ou du produit scalaire).

\begin{description}
    \item[Général] 
        Quelque résultats généraux et en vrac à propos d'isométries.
\begin{enumerate}
    \item
        Définition d'une isométrie pour une forme bilinéaire,~\ref{DEFooGGTYooXsHIZj}. Pour une forme quadratique : définition~\ref{DEFooECTUooRxBhHf}.
    \item
        Définition du groupe orthogonal~\ref{DEFooUHANooLVBVID}, et le spécial orthogonal \( \SO(n)\) en la définition~\ref{DEFooJLNQooBKTYUy}. Le groupe \( \SO(2)\) est le groupe des rotations, par corollaire~\ref{CORooVYUJooDbkIFY}.
    \item
        Le lemme~\ref{LEMooHRESooQTrpMz} donne à toute rotation une matrice de la forme connue. C'est autour de cela que nous définissons les angles, définition \ref{DEFooUPUUooKAPFrh}.
    \item
        Le groupe orthogonal est le groupe des isométries de \( \eR^n\), proposition~\ref{PropKBCXooOuEZcS}.
    \item
        Les isométries de l'espace euclidien sont affines,~\ref{ThoDsFErq}.
    \item
        Les isométries de l'espace euclidien comme produit semi-direct : $\Isom(\eR^n)\simeq T(n)\times_{\rho}\gO(n)$, théorème~\ref{THOooQJSRooMrqQct}.
    \item
        Isométries du cube, section~\ref{SecPVCmkxM}.
    \item
        Générateurs du groupe diédral, proposition~\ref{PropLDIPoZ}.
\end{enumerate}
    \item[Isométries et réflexions]
        Dans un espace euclidien, toute isométrie peut être décomposée en réflexions autour d'hyperplans. Voici quelque énoncés à ce propos.
\begin{enumerate}
    \item
        Définition d'un hyperplan \ref{DEFooEWDTooQbUQws}.
    \item
        En dimension \( 2\), une rotation est définie comme composée de deux réflexions en la définition \ref{DEFooFUBYooHGXphm}.
    \item
        En dimension \( 2\), les réflexion ont un déterminant \( -1\) par le lemme \ref{LEMooSYZYooWDFScw}.
    \item
        Les isométries du plan sont données dans le théorème \ref{THOooVRNOooAgaVRN}, et sont au plus 3 réflexions par le théorème \ref{THOooRORQooTDWFdv}.
    \item
        Décomposition des isométries de \( \eR^n\) en réflexions par le lemme \ref{LEMooJCDRooGAmlwp}.
    \item
        En particulier, les éléments de \( \SO(3)\) sont des compositions de deux réflexions par le corollaure \ref{CORooJCURooSRzSFb}.
    \item
        Une isométrie de \( \eR^n\) préserve l'orientation si et seulement si est elle composition d'un nombre pair de réflexions. C'est le théorème \ref{THOooWBIYooCtWoSq}.
\end{enumerate}
\end{description}
  % isométries


% PROBA-STAT

\InternalLinks{caractérisation de distributions en probabilités}
\begin{enumerate}
    \item
        La probabilité conjointe est la définition~\ref{DefFonrepConj}.
    \item
        La fonction de répartition est la définition~\ref{DefooYAZVooNdxDCx}.
    \item
        La fonction caractéristique est la définition~\ref{DefooEIVXooNtHLQQ}.
\end{enumerate}
  % caractérisation de distributions en probabilités

\InternalLinks{théorème central limite}
\begin{enumerate}
    \item
        Pour les processus de Poisson, théorème~\ref{ThoCSuLLo}.
\end{enumerate}
  % théorème central limite

\InternalLinks{lemme de transfert}      \label{THEMEooJREIooKEdMOl}
Il s'agit du résultat \( \hat{f'}=i\xi \hat{f}\).
\begin{enumerate}
    \item
        Lemme~\ref{LemQPVQjCx} sur \( \swS(\eR^d)\)
    \item
        Lemme~\ref{LEMooAGBZooWCbPDd} pour \( L^2\).
\end{enumerate}
  % lemme de transfert


\InternalLinks{probabilités et espérances conditionnelles}

    Les deux définitions de base, sur lesquelles se basent toutes les choses conditionnelles sont :
    \begin{itemize}
        \item L'espérance conditionnelle d'une variable aléatoire sachant une tribu : \( E(X|\tribF)\) de la définition~\ref{ThoMWfDPQ}.
    \end{itemize}

    Les autres sont listées ci-dessous.
\begin{description}

    \item[Probabilité conditionnelle]. 

        Plusieurs probabilités conditionnelles.
        \begin{itemize}
        \item D'un événement en sachant un autre : la définition \ref{DEFooGJVHooVbhVYv} donne
            \begin{equation}
                P(A|B)=\frac{ P(A\cap B) }{ P(B) }
            \end{equation}
            Cela est la définition de base. L'autre est une définition dérivée.
        \item D'un événement vis-à-vis d'une variable aléatoire discrète. C'est par la définition~\ref{DEFooFRLFooNvXuPK} qui défini la variable aléatoire
\begin{equation}
    P(A|X)(\omega)=P(A|X=X(\omega)).
\end{equation}
Dans le cas continu, c'est la définition~\ref{DEFooIUJMooBAVtMW} :
\begin{equation}
    P(A|X)=P(A|\sigma(X))=E(\mtu_A|\sigma(X)).
\end{equation}

    \item D'un événement par rapport à une tribu. C'est la variable aléatoire
\begin{equation}
    P(A|\tribF)=E(\mtu_A|\tribF).
\end{equation}

        \end{itemize}
    \item[Espérances conditionnelles] 
        Plusieurs espérances conditionnelles.
        \begin{itemize}
            \item d'une variable aléatoire par rapport à une tribu. La variable aléatoire \( E(X|\tribF)\) est la variable aléatoire \( \tribF\)-mesurable telle que
\begin{equation}
    \int_BE(X|\tribF)=\int_BX
\end{equation}
pour tout \( X\in \tribF\). Si \( X\in L^2(\Omega,\tribA,P)\) alors \( E(X|\tribF)=\pr_K(X)\) où \( K\) est le sous-ensemble de \( L^2(\Omega,\tribA,P)\) des fonctions \( \tribF\)-mesurables (théorème~\ref{ThoMWfDPQ}). Cela au sens des projections orthogonales.


    \item d'une variable aléatoire par rapport à une autre. La définition~\ref{DefooKIHPooMhvirn} est une variation sur le même thème :
\begin{equation}
    E(X|Y)=E(X|\sigma(Y)),
\end{equation}

%TODO : mettre cette définition à côté de celle du conditionnement par rapport à la tribu.
        \end{itemize}

\end{description}

Notons que partout, si \( X\) est une variable aléatoire, la notation «sachant \( X\)» est un raccourcis pour dire «sachant la tribu engendrée par \( X\)».

  % probabilités et espérances conditionnelles


\InternalLinks{dénombrements}
\begin{itemize}
    \item Coloriage de roulette (\ref{pTqJLY}) et composition de colliers (\ref{siOQlG}).
    \item Nombres de Bell, théorème~\ref{ThoYFAzwSg}.
    \item Le dénombrement des solutions de l'équation \( \alpha_1 n_1+\ldots \alpha_pn_p=n\) utilise des séries entières et des décomposition de fractions en éléments simples, théorème~\ref{ThoDIDNooUrFFei}.
\end{itemize}
  % dénombrements

\InternalLinks{enveloppes}
    \begin{enumerate}
        \item
            L'ellipse de John-Loewner donne un ellipsoïde de volume minimum autour d'un compact dans \( \eR^n\), théorème \ref{PropJYVooRMaPok}.
        \item
            Le cercle circonscrit à une courbe donne un cercle de rayon minimal contenant une courbe fermée simple, proposition \ref{PROPDEFooCWESooVbDven}.
    \item Enveloppe convexe du groupe orthogonal \ref{ThoVBzqUpy}.
        \end{enumerate}

  % enveloppes

\InternalLinks{équations diophantiennes}
    \begin{enumerate}
        \item
            Équation \( ax+by=c\) dans \( \eN\), équation \eqref{EqTOVSooJbxlIq}.
        \item Dans \ref{subsecZVKNooXNjPSf}, nous résolvons \( ax+by=c\) en utilisant Bézout (théorème \ref{ThoBuNjam}).
        \item L'exemple \ref{ExZPVFooPpdKJc} donne une application de la pure notion de modulo pour \( x^2=3y^2+8\). Pas de solutions.
        \item L'exemple \ref{ExmuQisZU} résout l'équation \( x^2+2=y^3\) en parlant de l'extension \( \eZ[i\sqrt{2}]\) et de stathme.
        \item Les propositions \ref{PropXHMLooRnJKRi} et \ref{propFKKKooFYQcxE} parlent de triplets pythagoriciens.
        \item Le dénombrement des solutions de l'équation \( \alpha_1 n_1+\ldots \alpha_pn_p=n\) utilise des séries entières et des décomposition de fractions en éléments simples, théorème \ref{ThoDIDNooUrFFei}.
        \end{enumerate}

  % équations diophantiennes
\InternalLinks{techniques de calcul}        \label{THEMEooLTCIooGDIPnF}

Il y en a pour tous les gouts.

\begin{description}
    \item[Primitives et intégrales]
        Toute la section~\ref{SECooKSOFooEVKDLh} donne des trucs et astuces pour trouver des primitives et des intégrales.
    \item[Limite à deux variables]

        Les exemples de limites à plusieurs variables font souvent intervenir des coordonnées polaires (du théorème \ref{THOooBETSooXSQhdX}) ou autres fonctions trignométriques. Ils sont donc placés beaucoup plus bas que la théorie.
        \begin{itemize}
            \item Méthode du développement asymptotique, sous-section~\ref{SUBSECooRAKKooAnpvkE}.
            \item Méthode des coordonnées polaires, sous-section~\ref{SUBSECooWCGMooPrXSpt}.
            \item Utilisation du théorème de la fonction implicite, dans l'exemple~\ref{EXooSDHDooJzDioW}.
        \end{itemize}

\end{description}
    % techniques de calcul


\immediate\closeout\themetoc
