
\InternalLinks{compacts}        \label{THEMEooQQBHooLcqoKB}
    \begin{description}

        \item[Propriétés générales]

            Quelque propriétés de compacts.

                \begin{enumerate}
    \item
        La définition d'un ensemble compact est la définition \ref{DefJJVsEqs}.
    \item
        Les compacts sont les fermés bornés par le théorème \ref{ThoXTEooxFmdI}.
    \item
        L'image d'un compact par une fonction continue est un compact, théorème \ref{ThoImCompCotComp}.
    \item 
        Un espace métrique est compact si et seulement si toute suite contient une sous-suite convergente. C'est le théorème de Bolzano-Weierstrass \ref{ThoBWFTXAZNH}.
    \item
        Une fonction continue sur un compact est bornée et attein ses bornes, théorème \ref{ThoWeirstrassRn}.
                \end{enumerate}

        \item[Produits de compacts] 
            À propos de produits de compacts. C'est un compact dans tous les cas métriques\quext{Si vous connaissez des exemples non métriques de produits de compacts qui ne sont pas compacts, écrivez moi.}.
    \begin{enumerate}
    \item
        Les produits d'espaces métriques compacts sont compacts; c'est le théorème de Tykhonov. Nous verrons ce résultat dans les cas suivants.
        \begin{itemize}
    \item
         \( \eR\), lemme \ref{LemCKBooXkwkte}.
    \item
        Produit fini d'espaces métriques compacts, théorème \ref{THOIYmxXuu}.
    \item
        Produit dénombrable d'espaces métriques compacts, théorème \ref{ThoKKBooNaZgoO}.
        \end{itemize}
    \end{enumerate}
    \end{description}
