
\InternalLinks{compacts}        \label{THEMEooQQBHooLcqoKB}
    \begin{description}

        \item[Propriétés générales]

            Quelques propriétés de compacts.

                \begin{enumerate}
    \item
        La définition d'un ensemble compact est la définition~\ref{DefJJVsEqs}.
    \item
        Les compacts sont les fermés bornés par le théorème~\ref{ThoXTEooxFmdI}.
    \item
        Le théorème de Borel-Lebesgue \ref{ThoBOrelLebesgue} dit qu'un intervalle de \( \eR\) est compact si et seulement si il est de la forme \( \mathopen[ a , b \mathclose]\).
    \item
        Théorème des fermés emboîtés dans le cas compact, corolaire \ref{CORooQABLooMPSUBf}. À ne pas confondre avec celui dans le cas des espaces métrique, théorème \ref{ThoCQAcZxX}.
    \item
        L'image d'un compact par une fonction continue est un compact, théorème~\ref{ThoImCompCotComp}.
    \item
        Suites dans un compact
        \begin{enumerate}
            \item
                Toute suite dans un compact admet une sous-suite convergente, théorème \ref{THOooRDYOooJHLfGq}.
            \item
                Dans \( \eR^n\), toute suite dans un compact admet une sous-suite convergente, théorème \ref{ThoBolzanoWeierstrassRn}. La démonstration de ce théorèma est non seulement plus compliquée que le cas général, mais utilise en plus le cas dans \( \eR\); lequel cas n'est pas démontré de façon directe dans le Frido.
            \item
                Un espace métrique est compact si et seulement si toute suite contient une sous-suite convergente. C'est le théorème de Bolzano-Weierstrass~\ref{ThoBWFTXAZNH}. La démonstration de ce théorème est indépendante.
        \end{enumerate}
    \item
        Une fonction continue sur un compact est bornée et atteint ses bornes, théorème~\ref{ThoWeirstrassRn}.
    \item
        Une fonction continue sur un compact y est uniformément continue, théorème de Heine \ref{PROPooBWUFooYhMlDp}.
                \end{enumerate}

        \item[Produits de compacts]
            À propos de produits de compacts. C'est un compact dans tous les cas métriques\quext{Si vous connaissez des exemples non métriques de produits de compacts qui ne sont pas compacts, écrivez moi.}.
    \begin{enumerate}
    \item
        Les produits d'espaces métriques compacts sont compacts; c'est le théorème de Tykhonov. Nous verrons ce résultat dans les cas suivants.
        \begin{itemize}
    \item
         \( \eR\), lemme~\ref{LemCKBooXkwkte}.
    \item
        Produit fini d'espaces métriques compacts, théorème~\ref{THOIYmxXuu}.
    \item
        Produit dénombrable d'espaces métriques compacts, théorème~\ref{ThoKKBooNaZgoO}.
        \end{itemize}
    \end{enumerate}
    \end{description}
