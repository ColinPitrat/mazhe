
\InternalLinks{polynôme de Taylor}

Énoncés :

    \begin{enumerate}
    \item
        Énoncé : théorème \ref{ThoTaylor}.
    \item
        De classe \( C^2\) sur \( \eR^n\), proposition \ref{PROPooTOXIooMMlghF}.
    \item
    Avec un reste donné par un point dans \( \mathopen] x , a \mathclose[\), proposition \ref{PropResteTaylorc}.

        \item
            Le polynôme de Taylor généralise à l'utilisation de toutes les dérivées disponibles le résultat de développement limité donné par la proposition \ref{PropUTenzfQ}.
        \item
            Pour les fonctions holomorphes, il y a le théorème \ref{THOooSULFooHTLRPE} qui donne une série de Taylor sur un disque de convergence.
        \end{enumerate}

Utilisation :
\begin{enumerate}
        \item
            Il est utilisé pour justifier la méthode de Newton autour de l'équation \eqref{EQooOPUBooYaznay}.
    \item
        On utilise pas mal de Taylor dans les résultats liant extrema et différentielle/hessienne. Par exemple la proposition \ref{PropoExtreRn}.
\end{enumerate}
