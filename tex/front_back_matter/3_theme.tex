
\InternalLinks{polynôme de Taylor}

\begin{description}
    \item[Énoncés] 

        Il existe de nombreux énoncés du théorème de Taylor, et en particulier beaucoup de formules pour le reste.

    \begin{enumerate}
    \item
        Énoncé : théorème~\ref{ThoTaylor}.
    \item
        Une majoration du reste est dans le théorème \ref{THOooEUVEooXZJTRL}
    \item
        De classe \( C^2\) sur \( \eR^n\), proposition~\ref{PROPooTOXIooMMlghF}.
    \item
    Avec un reste donné par un point dans \( \mathopen] x , a \mathclose[\), proposition~\ref{PropResteTaylorc}.
        \item
            Avec reste intégral, proposition~\ref{PropAXaSClx} et théorème \ref{THOooDGCJooXKmFTT} pour le cas simple \( \eR\to \eR\).
        \item
            Le polynôme de Taylor généralise à l'utilisation de toutes les dérivées disponibles le résultat de développement limité donné par la proposition~\ref{PropUTenzfQ}.
        \item
            Pour les fonctions holomorphes, il y a le théorème~\ref{THOooSULFooHTLRPE} qui donne une série de Taylor sur un disque de convergence.
        \end{enumerate}

    \item[Utilisation]

        Des polynômes de Taylor sont utilisés pour démontrer des théorèmes par-ci par-là.

\begin{enumerate}
        \item
            Il est utilisé pour justifier la méthode de Newton autour de l'équation \eqref{EQooOPUBooYaznay}.
    \item
        On utilise pas mal de Taylor dans les résultats liant extrema et différentielle/hessienne. Par exemple la proposition~\ref{PropoExtreRn}.
\end{enumerate}

\item[Quelques développements]

Voici quelques développements limités à savoir. Ils sont calculables en utilisant la formule de Taylor-Young (proposition~\ref{PropVDGooCexFwy}).
\begin{subequations}
    \begin{align*}
        e^x&=\sum_{k=0}^n\frac{ x^k }{ k! }+x^n\alpha(x)&\text{ordre } n, \text{proposition \ref{PROPooQBRGooAhGrvP}}\\
        \cos(x)&=\sum_{k=0}^p\frac{ (-1)^kx^{2k} }{ (2k)! }+x^{2p+1}\alpha(x)&\text{ordre } 2p+1,\text{proposition \ref{PROPooNPYXooTuwAHP}}\\
        \sin(x)&=\sum_{k=0}^p\frac{ (-1)^kx^{2k+1} }{ (2k+1)! }+x^{2p+2}\alpha(x)&\text{ordre } 2p+1,\text{proposition \ref{PROPooNPYXooTuwAHP}}\\
        \ln(1+x)&=\sum_{k=1}^n\frac{ (-1)^{k+1} }{ k }x^k+\alpha(x)x^n&\text{ordre }n,\text{proposition \ref{PROPooWCUEooJudkCV}}\\
        \ln(1+x)&=\sum_{k=1}^{\infty}\frac{ (-1)^{k+1} }{ k }x^k&\text{exact }\text{proposition \ref{PROPooKPBIooJdNsqX}}\\
        \ln(2)&=\sum_{k=1}^{\infty}\frac{ (-1)^{k+1} }{ k }&\text{exact }\text{proposition \ref{PROPooKPBIooJdNsqX}}\\
      (1+x)^l&=\sum_{k=0}^l\binom{ l }{ k }x^k&\text{exact si } l\text{ est entier.}\\
      (1+x)^{\alpha}&=1+\sum_{k=1}^n\frac{ \alpha(\alpha-1)\ldots(\alpha-k+1) }{ k! }x^k+x^n\alpha(x)&\text{ordre } n.
    \end{align*}
\end{subequations}
  Dans toutes ces formules, la fonction \( \alpha\colon \eR\to \eR\) vérifie \( \lim_{t\to 0} \alpha(t)=0\).

Le développement limité en $0$ d'une fonction paire ne contient que les puissances de $x$ d'exposant paire. Voir comme exemple le développement de la fonction cosinus.

\end{description}
