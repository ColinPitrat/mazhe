% This is part of Mes notes de mathématique
% Copyright (c) 2015-2017
%   Laurent Claessens
% See the file fdl-1.3.txt for copying conditions.

%+++++++++++++++++++++++++++++++++++++++++++++++++++++++++++++++++++++++++++++++++++++++++++++++++++++++++++++++++++++++++++ 
\section*{Ce cours à l'agrégation ?}
%+++++++++++++++++++++++++++++++++++++++++++++++++++++++++++++++++++++++++++++++++++++++++++++++++++++++++++++++++++++++++++

Peut-on utiliser ce cours pour \textbf{les oraux d'\href{http://agreg.org/}{agrégation}} (de mathématiques) ?  Cela est une question qui m'est arrivée quelques fois.  

Le règlement interdit d'apporter avec soi une version imprimée chez soi, et oblige de n'utiliser que des ressources commercialisées. Cela fait que le Frido \emph{tel que vous l'avez sous les yeux} n'est pas utilisable à l'agrégation. Pour utiliser le Frido, vous devrez payer; j'en suis le premier désolé.

Une version est commercialisée sur \href{http://www.thebookedition.com/fr/}{thebookedition.com}. Voir la page dédiée sur mon site :
\begin{center}
    \url{http://laurent.claessens-donadello.eu/frido.html}
\end{center}

Deux mots sur mon modèle économique. J'ai choisi de ne pas faire de bénéfice sur les ventes : à chaque copie vendue sur internet, je gagne zéro euro. Mon modèle économique est le don. Si vous réussissez l'agrégation et que vous pensez que le Frido y est pour quelque chose, n'hésitez pas à donner si votre situation vous le permet.

\vfill

J'accepte les donations.

% Moralement, vous devriez considérer ce fichier comme une section invariante
% au sens de la licence FDL.
% Autrement dit, je ne serais pas content que ce fichier soit modifié.

\begin{description}
\item[IBAN] FR76 3000 4004 0600 0035 2497 784
\item[BIC] BNPAFRPPBSC
\end{description}

