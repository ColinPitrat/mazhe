\InternalLinks{logarithme}
\begin{enumerate}
    \item
    Le logarithme pour les réels strictement positifs \( \ln\colon \mathopen] 0 , \infty \mathclose[\to \eR\) est donné en la définition~\ref{DEFooELGOooGiZQjt}.
    \item
        La proposition \ref{PROPooKPBIooJdNsqX} donne la série
        \begin{equation}
            \ln(1+x)=\sum_{k=1}^{\infty}\frac{ (-1)^{k+1} }{ k }x^k.
        \end{equation}
    \item
        La proposition~\ref{PropKKdmnkD} dit que toute matrice complexe admet un logarithme. En particulier une série explicite est donné pour le logarithme d'un bloc de Jordan.
    \item
        Sur les complexes, le logarithme \( \ln \colon \eC^*\to \eC\) est la définition~\ref{DEFooWDYNooYIXVMC}. Attention : ce n'est pas la seule définition possible.
    \item
        La série harmonique diverge à vitesse logarithmique, et la série des inverses des nombres premiers, c'est encore plus lent : théorème~\ref{ThonfVruT}.
\end{enumerate}
