\InternalLinks{séries de Fourier}       \label{THMooHWEBooTMInve}
\begin{itemize}
    \item Formule sommatoire de Poisson, proposition~\ref{ProprPbkoQ}.
    \item Inégalité isopérimétrique, théorème~\ref{ThoIXyctPo}.
    \item Fonction continue et périodique dont la série de Fourier ne converge pas, proposition~\ref{PropREkHdol}.

    \item
Nous allons montrer la convergence de \( \sum_{k\in \eZ}c_k(f) e^{inx}\) vers \( f(x)\) dans divers cas :
\begin{enumerate}
    \item
        Si \( f\) est continue et périodique, convergence au sens de Cesaro, théorème de Fejèr~\ref{ThoJFqczow}.
    \item
        Convergence au sens \( L^2\Big( \mathopen[ 0 , 2\pi \mathclose] \Big)\) dans le théorème~\ref{ThoYDKZLyv}.
    \item
        Si \( f\) est continue, périodique et sa série de Fourier converge uniformément, théorème~\ref{PropmrLfGt}.
    \item
        Si \( f\) est périodique et la série des coefficients converge absolument pour tout \( x\), proposition~\ref{PropSgvPab}.
    \item
        Si \( f\) est périodique et de classe \( C^1\), théorème~\ref{ThozHXraQ}.
\end{enumerate}
Il est cependant faux de croire que la continuité et la périodicité suffisent à obtenir une convergence, comme le montre dans la proposition~\ref{PropREkHdol}.
\end{itemize}
