% This is part of Le Frido
% Copyright (c) 2011-2013,2016-2017
%   Laurent Claessens
% See the file fdl-1.3.txt for copying conditions.


%+++++++++++++++++++++++++++++++++++++++++++++++++++++++++++++++++++++++++++++++++++++++++++++++++++++++++++++++++++++++++++
\section{Sage est là pour vous aider}
%+++++++++++++++++++++++++++++++++++++++++++++++++++++++++++++++++++++++++++++++++++++++++++++++++++++++++++++++++++++++++++

Il existe de nombreux logiciels de mathématique. Notre préféré est \href{http://www.sagemath.org}{Sage} pour une raison très précise : en tant que langage de programmation, Sage est python qui est un langage généraliste. La syntaxe et la structure de Sage ne sont pas \emph{ad hoc} pour faire de math, et ce qu'on apprend en Sage peut être recyclé pour faire n'importe quoi : navigateur web, script de manipulation de texte, traitement d'image, réseau neuronaux, \ldots

%Par ailleurs, le vingt et unième siècle est déjà largement entamé; si vous vous lancez dans une carrière scientifique, il vous faudra maitriser l'informatique un peu plus solidement qu'être virtuose es trouver le trajet le plus court en bus sur votre téléphone.

Sage est un logiciel disponible pour l'épreuve de modélisation de l'agrégation de mathématique; il y a donc de bonnes chances que vous en ayez l'usage.

%---------------------------------------------------------------------------------------------------------------------------
\subsection{Lancez-vous dans Sage}
%---------------------------------------------------------------------------------------------------------------------------


\begin{enumerate}
	\item
        Aller sur \url{http://www.sagenmath.org},
	\item
		créer un compte,
	\item
		créer des feuilles de calcul et s'amuser !!
\end{enumerate}

Il y a beaucoup de \href{http://lmgtfy.com/?q=sage+documentation}{documentation} sur le \href{http://www.sagemath.org}{site officiel}\footnote{\href{http://www.sagemath.org}{http://www.sagemath.org}}, et nous vous conseillons particulièrement le livre \cite{ooBLMMooWTPsQy}.

Si vous comptez utiliser régulièrement ce logiciel, je vous recommande \emph{chaudement} de \href{http://mirror.switch.ch/mirror/sagemath/index.html}{l'installer} sur votre ordinateur.

%---------------------------------------------------------------------------------------------------------------------------
\subsection{Exemples de ce que Sage peut faire pour vous}
%---------------------------------------------------------------------------------------------------------------------------

Ce livre est émaillé de petits bouts de code en Sage montrant ses différentes fonctionnalités là où nous en avons besoin\footnote{Soit un vrai besoin comme tracer un graphique en 3D, soit de la paresse comme calculer une grosse dérivée.}. Voici une liste (non exhaustive) de ce que Sage peut faire pour vous.

\begin{enumerate}

	\item
        Calculer des limites de fonctions, exemples \ref{ExBCRXooDVUdcf} et \ref{ExCWDRooKxnjGL}.
	\item
        Tracer des graphes de fonctions, exemple \ref{ExCWDRooKxnjGL}.
	\item
        Tracer des courbes en trois dimensions, voir exemple \ref{ExempleTroisDxxyy}. Notez que pour cela vous devez installer aussi le logiciel Jmol. Pour Ubuntu, c'est dans le paquet \info{icedtea6-plugin}.
	\item
		Calculer des dérivées partielles de fonctions à plusieurs variables, voir exemple \ref{exJMGTooZcZYNy}.
	\item
        Résoudre des systèmes d'équations linéaires. Voir les exemples \ref{exKGDIooVefujD} et \ref{ExBGCEooPIQgGW}. Lire aussi \href{http://www.sagemath.org/doc/constructions/linear_algebra.html#solving-systems-of-linear-equations}{la documentation}.
	\item
        Tout savoir d'une forme quadratique, voir exemple \ref{exBNGVooIvKfTT}.
	\item
        Calculer la matrice Hessienne de fonctions de deux variables, déterminer les points critiques, déterminer le genre de la matrice Hessienne aux points critiques et écrire extrema de la fonctions (sous réserve d'être capable de résoudre certaines équations), voir les exemples \ref{exZHGRooTQpVpq} et \ref{exHWIHooOAvaDQ}.
	\item
        Indiquer une infinité de solutions à une équation en utilisant des paramètres, voir l'exemple \ref{exEEHPooKDxLTJ}. Pour les fonctions trigonométriques, 
        \begin{verbatim}
sage: solve(sin(x)/cos(x)==1,x,to_poly_solve=True)                                                         
[x == 1/4*pi + pi*z1]
sage: solve(sin(x)**2==cos(x)**2,x,to_poly_solve=True)
[sin(x) == cos(x), x == -1/4*pi + 2*pi*z86, x == 3/4*pi + 2*pi*z84]
        \end{verbatim}

        Notez l'option \info{to\_poly\_solve=true} dans \info{solve}.

	\item
        Calculer des dérivées symboliquement, voir exemple \ref{exRNZKooUIOfPU}.
	\item
        Calculer des approximations numériques comme dans l'exemple \ref{exLFYFooNCXCJz}.
    \item
        Calculer dans un corps de polynômes modulo comme \( \eF_p[X]/P\) où \( P\) est un polynôme à coefficients dans \( \eF_p\). Voir l'exemple \ref{ExemWUdrcs}.
\end{enumerate}

Sage peut en général faire tout ce que vous êtes capable de faire à l'entrée en master et probablement bien plus, à la notable exception des limites à deux variables.

\begin{remark}
    Sage peut toutefois vous induire en erreur si vous n'y prenez pas garde parce qu'il sait des choses en mathématique que vous ne savez pas. Par conséquent il peut parfois vous donner des réponses (mathématiquement exactes) auxquelles vous ne vous attendez pas. Voir par exemple \ref{ooOPWYooDDSZWx} pour le logarithme de nombres négatifs. Et aussi ceci :
    
\lstinputlisting{tex/sage/sageSnip017.sage}

Sage fait une différence entre \info{Infinity} et \info{+Infinity} et donne
\begin{equation}
    \lim_{x\to 0} \frac{1}{ x }=\infty
\end{equation}
ainsi que
\begin{equation}
    \lim_{x\to 0} \frac{1}{ x^2 }=+\infty.
\end{equation}
\end{remark}
