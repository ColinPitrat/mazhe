\InternalLinks{exponentielle}        \label{THEMEooKXSGooCsQNoY}

Toutes les exponentielles sont définies par la série
\begin{equation}
    \exp(x)=\sum_{k=0}^{\infty}\frac{ x^k }{ k! },
\end{equation}
tant que la somme ait un sens.

\begin{description}
    \item[Réels]

    \item[Complexes]

        \begin{enumerate}
            \item
                Le fait que \(  e^{i\theta}\) donne tous les nombres complexes de norme \( 1\) est la proposition~\ref{PROPooZEFEooEKMOPT}.
            \item
                Le groupe des racines de l'unité est donné par l'équation \eqref{EqIEAXooIpvFPe}.
        \end{enumerate}

    \item[Algèbre normée commutative]

        Pour la définition c'est la proposition~\ref{DEFooSFDUooMNsgZY} et pour la régularité \(  C^{\infty}\) c'est la proposition~\ref{PROPooTBDAooQouzSk}.

    \item[Idem non commutatif]

        Il y a une tentative de théorème~\ref{THOooFGTQooZPiVLO}, mais c'est principalement pour les matrices qu'il y a des résultats.

    \item[Matrices]

        De nombreux résultats sont disponibles pour les exponentielles de matrices.

\begin{enumerate}
    \item
        La section~\ref{secAOnIwQM} parle d'exponentielle de matrices.
    \item
        L'exponentielle donne lieu à une fonction de classe \(  C^{\infty}\), proposition~\ref{PropXFfOiOb}.
    \item
            Le lemme à propos d'exponentielle de matrice~\ref{LemQEARooLRXEef} donne :
            \begin{equation}
                \|  e^{tA} \|\leq P\big( | t | \big)\sum_{i=1}^r e^{t\real(\lambda_i)}.
            \end{equation}
        \item
            La proposition~\ref{PropCOMNooIErskN} : si \( A\in \eM(n,\eR)\) a un polynôme caractéristique scindé, alors \( A\) est diagonalisable si et seulement si \( e^A\) est diagonalisable.
\item
    La section~\ref{subsecXNcaQfZ} parle des fonctions exponentielle et logarithme pour les matrices. Entre autres la dérivation et les séries.
\item
    Pour résoudre des équations différentielles linéaires : sous-section~\ref{SUBSECooMDKIooKaaKlZ}.
\item
    La proposition~\ref{PropKKdmnkD} dit que l'exponentielle est surjective sur \( \GL(n,\eC)\).
\item

La proposition~\ref{PropFMqsIE} : si \( u\) est un endomorphisme, alors \( \exp(u)\) est un polynôme en \( u\).
\item
    Calcul effectif : sous-section~\ref{SUBSECooGAHVooBRUFub}.
\item Proposition~\ref{PROPooZUHOooQBwfZq} : si \( A\in\eM(n,\eC)\) alors $ e^{\tr(A)}=\det( e^{A}).$
    \item
        Les séries entières de matrices sont traitées autour de la proposition~\ref{PropFIPooSSmJDQ}.
\end{enumerate}


\end{description}
