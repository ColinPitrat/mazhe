\InternalLinks{espaces de fonctions}                \label{THEMooNMYKooVVeGTU}

En ce qui concerne les densités, voir le thème \ref{THEooPUIIooLDPUuq}.


\begin{description}
    \item[Topologie]

        Les espaces de fonctions sont souvent munis de topologies définies par des semi-normes.

        \begin{enumerate}
            \item
                La topologie des semi-normes est la définition \ref{DefPNXlwmi}.
            \item
                La définition \ref{DefFGGCooTYgmYf} donne les topologies sur \(  C^{\infty}(\Omega)\), \( \swD(K)\) et \( \swD(\Omega)\).
            \item
                La topologie \( *\)-faible sur \( \swD'(\Omega)\) est donnée par la définition \ref{DefASmjVaT}.
        \end{enumerate}

    \item[L'espace \( { L^2\big( \mathopen[ 0 , 2\pi \mathclose] \big) } \)]

        C'est un espace très important, entre autres parce qu'il est de Hilbert et est bien adapté à la transformée de Fourier.

        \begin{enumerate}
        \item
            Définition de \( L^2(\Omega,\mu)\), \ref{DEFooSVCHooIwwuIx}.
            \item
                Le produit scalaire \( \langle f, g\rangle \) est donné en \eqref{EQooBFKDooMkCZOt} et la base trigonométrique est \eqref{EQooKMYOooLZCNap}.
            \item 
                La densité des polynômes trigonométriques dans \( L^p(S^2)\) est le théorème \ref{ThoQGPSSJq} ou le théorème \ref{ThoDPTwimI}, au choix.
            \item
                Une conséquence de cette densité est que le système trigonométrique est une base hilbertienne de \( L^2\) par le lemme \ref{LEMooBJDQooLVPczR}.
        \end{enumerate}

            L'espace \( L^2\)  est discuté en analyse fonctionnelle, en \ref{SECooEVZSooLtLhUm} parce que l'étude de \( L^2\) utilise entre autres l'inégalité de Hölder \ref{ProptYqspT}.

        Le fait que \( L^2\) soit une espace de Hilbert est utilisé dans la preuve du théorème de représentation de Riesz \ref{PropOAVooYZSodR}.

\end{description}
