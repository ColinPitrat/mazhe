
\InternalLinks{inégalités}
\begin{description}
    \item[Inégalité de Jensen]
        \begin{enumerate}
            \item
                Une version discrète pour \( f\big( \sum_i\lambda_ix_i \big)\), la proposition~\ref{PropXIBooLxTkhU}.
            \item
                Une version intégrale pour \( f\big( \int \alpha d\mu \big)\), la proposition~\ref{PropXISooBxdaLk}.
            \item
                Une version pour l'espérance conditionnelle, la proposition~\ref{PropABtKbBo}.
        \end{enumerate}
    \item[Inégalité pour les normes $ \ell^p$]
        \begin{enumerate}
            \item
                Hölder pour \( L^p\): \( \| fg \|_1\leq \| f \|_p\| g \|_q\), proposition \ref{ProptYqspT}.
            \item
                Hölder pour \( \ell^p\): \( \| x \|_q\leq n^{\frac{1}{ q }-\frac{1}{ p }}\| x \|_p\), proposition \ref{PROPooQZTNooGACMlQ}.
        \end{enumerate}
    \item[Inégalité de Minkowsky]
        \begin{enumerate}
            \item
                Pour une forme quadratique \( q\) sur \( \eR^n\) nous avons $\sqrt{q(x+y)}\leq\sqrt{q(x)}+\sqrt{q(y)}$. Proposition~\ref{PropACHooLtsMUL}.
            \item
                Si \( 1\leq p<\infty\) et si \( f,g\in L^p(\Omega,\tribA,\mu)\) alors \(  \| f+g \|_p\leq \| f \|_p+\| g \|_p\). Proposition~\ref{PropInegMinkKUpRHg}.
            \item
                L'inégalité de Minkowsky sous forme intégrale s'écrit sous forme déballée
                \begin{equation*}
                    \left[ \int_X\Big( \int_Y| f(x,y) |d\nu(y) \Big)^pd\mu(x) \right]^{1/p}\leq \int_Y\Big( \int_X| f(x,y) |^pd\mu(x) \Big)^{1/p}d\nu(y).
                \end{equation*}
                ou sous forme compacte
                \begin{equation*}
                    \left\|   x\mapsto\int_Y f(x,y)d\nu(y)   \right\|_p\leq \int_Y  \| f_y \|_pd\nu(y)
                \end{equation*}
        \end{enumerate}
    \item[Transformée de Fourier]
                Pour tout \( f\in L^1(\eR^n)\) nous avons \( \| \hat f \|_{\infty}\leq \| f \|_1\), lemme~\ref{LEMooCBPTooYlcbrR}.
    \item[Inégalité des normes]
        Inégalité de normes : si \( f\in L^p\) et \( g\in L^1\), alors \( \| f*g \|_p\leq \| f \|_p\| g \|_1\), proposition~\ref{PROPooDMMCooPTuQuS}.

\end{description}
