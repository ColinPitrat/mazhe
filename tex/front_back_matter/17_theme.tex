        \InternalLinks{norme matricielle et rayon spectral}     \label{THEMEooOJJFooWMSAtL}

    \begin{enumerate}
        \item
            Définition du rayon spectral~\ref{DEFooEAUKooSsjqaL}.
        \item
            Lien entre norme matricielle et rayon spectral, le théorème~\ref{THOooNDQSooOUWQrK} assure que $\|A\|_2=\sqrt{\rho(A{^t}A)}$.
        \item
            Lien entre valeurs propres et norme opérateur : le lemme~\ref{LEMooNESTooVvUEOv} pour les matrices symétriques strictement définies positives donne \( \| A \|_2=\lambda_{max}\).
        \item
            Pour toute norme algébrique nous avons \( \rho(A)\leq \| A \|\), proposition~\ref{PROPooWZJBooTPLSZp}.
        \item
            Dans le cadre du conditionnement de matrice. Voir en particulier la proposition~\ref{PROPooNUAUooIbVgcN} qui utilise le théorème~\ref{THOooNDQSooOUWQrK}.
        \item
            Rayon spectral et convergence de méthode itérative, proposition~\ref{PROPooAQSWooSTXDCO}.
    \end{enumerate}

    Une norme matricielle donne une topologie. Il y a donc également des liens entre rayon spectral et convergence de série. Dans cette optique, pour les séries de matrices, voir le thème~\ref{THEMEooPQKDooTAVKFH}.
