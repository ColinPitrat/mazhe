        \InternalLinks{norme matricielle, norme opérateur et rayon spectral}     \label{THEMEooOJJFooWMSAtL}

    La norme matricielle n'est rien d'autre que la norme opérateur de l'application linéaire donnée par la matrice.

    \begin{enumerate}
        \item
            Définition du rayon spectral~\ref{DEFooEAUKooSsjqaL}.
        \item
            Lien entre norme matricielle et rayon spectral, le théorème~\ref{THOooNDQSooOUWQrK} assure que $\|A\|_2=\sqrt{\rho(A{^t}A)}$.
        \item
            Lien entre valeurs propres et norme opérateur : le lemme~\ref{LEMooNESTooVvUEOv} pour les matrices symétriques strictement définies positives donne \( \| A \|_2=\lambda_{max}\).
        \item
            Pour toute norme algébrique nous avons \( \rho(A)\leq \| A \|\), proposition~\ref{PROPooWZJBooTPLSZp}.
        \item
            Dans le cadre du conditionnement de matrice. Voir en particulier la proposition~\ref{PROPooNUAUooIbVgcN} qui utilise le théorème~\ref{THOooNDQSooOUWQrK}.
        \item
            Rayon spectral et convergence de méthode itérative, proposition~\ref{PROPooAQSWooSTXDCO}.
    \end{enumerate}

    Pour la norme opérateur nous avons les résultats suivants.

    \begin{enumerate}
        \item
            Définition~\ref{DefNFYUooBZCPTr}.
        \item
            Définition d'une algèbre :~\ref{DefAEbnJqI} et pour une norme d'algèbre :~\ref{DefJWRWQue}.
        \item
            Pour des espaces vectoriels normés, être borné est équivalent à être continu : proposition~\ref{PROPooQZYVooYJVlBd}.
        \item
            Le lemme à propos d'exponentielle de matrice~\ref{LemQEARooLRXEef} donne :
            \begin{equation*}
                \|  e^{tA} \|\leq P\big( | t | \big)\sum_{i=1}^r e^{t\real(\lambda_i)}.
            \end{equation*}
    \end{enumerate}

    Une norme matricielle donne une topologie. Il y a donc également des liens entre rayon spectral et convergence de série. Dans cette optique, pour les séries de matrices, voir le thème~\ref{THEMEooPQKDooTAVKFH}.

