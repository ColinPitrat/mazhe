\InternalLinks{types d'anneaux}

\begin{enumerate}
    \item
        \( \eZ\) est intègre, exemple \ref{EXooLDXRooSxUAXs}, principal et euclidient (proposition \ref{PROPooPJGLooQSrJTU}).
    \item
        \( \eZ[X]\) n'est pas principal (voir \ref{ITEMooNQQMooSnuKvW}).
    \item   \label{ITEMooNQQMooSnuKvW}
        Si \( A\) est un anneau intègre qui n'est pas un corps, alors \( A[X]\) n'est pas principal, lemme \ref{LEMooDJSUooJWyxCL}.
    \item
        L'anneau des fonctions holomorphes sur un compact donné est principal, proposition \ref{PROPooVWRPooGQMenV}.
    \item
        L'anneau \( \eZ[i\sqrt{ 3 }]\) n'est pas factoriel, exemple \ref{EXooCWJUooCDJqkr}.
    \item 
        L'anneau \( \eZ[i\sqrt{ 5 }]\) n'est ni factoriel ni principal, exemple \ref{EXooYCTDooGXAjGC}.
    \item
        Tous les idéaux de \( \eZ/6\eZ\) sont principaux, mais \( \eZ/6\eZ\) n'est pas principal. Exemple \ref{EXooCJRPooYkWdyr}.
\end{enumerate}
