\InternalLinks{définie positive}        \label{THEMEooYEVLooWotqMY}
\begin{enumerate}
    \item
        Une application bilinéaire est définie positive lorsque \( g(u,u)\geq 0\) et \( g(u,u)=0\) si et seulement si \( u=0\) est la définition~\ref{DEFooJIAQooZkBtTy}.
    \item
        Un opérateur ou une matrice est défini positif si toutes ses valeurs propres sont positives, c'est la définition~\ref{DefAWAooCMPuVM}.
    \item
        Pour une matrice symétrique, définie positive implique \( \langle Ax, x\rangle >0\) pour tout \( x\). C'est le lemme~\ref{LemWZFSooYvksjw}.
    \item
        Une application linéaire est définie positive si et seulement si sa matrice associée l'est. C'est la proposition~\ref{PROPooUAAFooEGVDRC}.
\end{enumerate}
Remarque : nous ne définissons pas la notion de matrice définie positive dans le cas d'une matrice non symétrique.
