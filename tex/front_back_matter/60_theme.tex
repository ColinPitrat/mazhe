\InternalLinks{suites et séries}

\begin{description}
    \item[Suites] 
        Les suites réelles sont en général dans la section \ref{SECooLLUGooOwZRyI}.
        \begin{enumerate}
    \item
        Les suites adjacentes, c'est la définition \ref{DEFooDMZLooDtNPmu}. Cela sert pour les séries alternées, théorème \ref{THOooOHANooHYfkII}, qui sert pour étudier la série de Taylor de \( \ln(x+1)\), voir le lemme \ref{LEMooWMGGooRpAxBa} et ce qui l'entoure.
    \item
        La définition de la convergence absolue est la définition~\ref{DefVFUIXwU}.
            \item
                Une suite réelle croissante et majorée converge, proposition \ref{LemSuiteCrBorncv}.
            \item
                Toute suite dans un compact admet une sous-suite convergente, théorème \ref{THOooRDYOooJHLfGq} pour la version dans \( \eR\).
            \item
                Pour tout réel, il existe une suite croissante de rationnels qui y convergege, proposition \ref{PropSLCUooUFgiSR}.
        \end{enumerate}
    \item[Série] 
        Les séries sont en général dnas la section \ref{SECooYCQBooSZNXhd}.
        \begin{enumerate}
    \item
        Quelque séries usuelles en \ref{SUBSECooDTYHooZjXXJf} : série harmonique, géométrique, de Riemann, et la mythique arithmético-géométrique.
    \item
        Critère des séries alternées, théorème \ref{THOooOHANooHYfkII}.
    \item
        Convergence d'une série implique convergence vers zéro du terme général, proposition~\ref{PROPooYDFUooTGnYQg}.
        \end{enumerate}
    \item[Sommes infinies]
        \begin{enumerate}
            \item
Une somme indexée par un ensemble quelconque est la définition~\ref{DefHYgkkA}.
    \item
        La définition de la somme d'une infinité de termes est donnée par la définition~\ref{DefGFHAaOL}.
  \item
      si la série converge, on peut regrouper ses termes sans modifier la convergence ni la somme (associativité);
    Pour les sommes infinies l'associativité et la commutativité dans une série sont perdues. Néanmoins, il subsiste que
  \begin{enumerate}
  \item
      si la série converge absolument, on peut modifier l'ordre des termes sans modifier la convergence ni la somme (commutativité, proposition~\ref{PopriXWvIY}).
  \end{enumerate}
        \end{enumerate}
\end{description}
