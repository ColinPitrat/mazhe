
\InternalLinks{rang}
    \begin{enumerate}
        \item Définition~\ref{DefALUAooSPcmyK}.
        \item Le théorème du rang, théorème~\ref{ThoGkkffA}
        \item Prouver que des matrices sont équivalentes et les mettre sous des formes canoniques, lemme~\ref{LemZMxxnfM} et son corollaire~\ref{CorGOUYooErfOIe}.
        \item Tout hyperplan de \( \eM(n,\eK)\) coupe \( \GL(n,\eK)\), corollaire~\ref{CorGOUYooErfOIe}. Cela utilise la forme canonique sus-mentionnée.
        \item Le lien entre application duale et orthogonal de la proposition~\ref{PropWOPIooBHFDdP} utilise la notion de rang.
        \item Le lemme \ref{LEMooDFFDooJTQkRu} parle de commutant et utilise la notion de rang. Ce lemme sert à prouver diverses conditions équivalentes à être un endomorphisme cyclique dans le théorème \ref{THOooGLMSooYewNxW}.
        \end{enumerate}
