% This is part of the Exercices et corrigés de CdI-2.
% Copyright (C) 2008, 2009
%   Laurent Claessens
% See the file fdl-1.3.txt for copying conditions.

%+++++++++++++++++++++++++++++++++++++++++++++++++++++++++++++++++++++++++++++++++++++++++++++++++++++++++++++++++++++++++++
					\section{Suites de fonctions}
%+++++++++++++++++++++++++++++++++++++++++++++++++++++++++++++++++++++++++++++++++++++++++++++++++++++++++++++++++++++++++++

	\Exo{_I-1-1}
	\Exo{_I-1-2}
	\Exo{_I-1-3}
	\Exo{_I-1-4}
	\Exo{_I-1-5}
	\Exo{_I-1-6}
	\Exo{_I-1-7}
	\Exo{_I-1-8}
	\Exo{_I-1-8b}
	\Exo{_I-1-9}

%+++++++++++++++++++++++++++++++++++++++++++++++++++++++++++++++++++++++++++++++++++++++++++++++++++++++++++++++++++++++++++
					\section{Séries de fonctions}
%+++++++++++++++++++++++++++++++++++++++++++++++++++++++++++++++++++++++++++++++++++++++++++++++++++++++++++++++++++++++++++

	\Exo{_I-1-10}
	\Exo{_I-1-11}
	\Exo{_I-1-12}
	\Exo{_I-1-13}
	\Exo{_I-1-14}
	\Exo{_I-1-15}
	\Exo{_I-1-16}

Après avoir fait le théorème de Borel, essayons de voir ce qu'il dit. Le théorème de la page I.29 nous donne facilement, sous forme de séries de puissances, une fonction $ C^{\infty}$ sur un intervalle dont les dérivées en zéro sont données à l'avance. L'exercice que nous venons de faire nous permet de trouver une fonction $ C^{\infty}$ sur tout $\eR$ dont les dérivées sont données à l'avance.

% TODO : La ligne suivante m'a l'air un peu trop simple pour être vraie ... y réfléchir encore un peu avant de décommenter :
%De plus, la fonction $u$ ainsi définie est à support compact, contenu dans le support de $f$, parce que $t_n\leq 1$. Nous obtenons ainsi des fonctions $ C^{\infty}$ à support compact dont les dérivées en un point sont données à l'avance.

%+++++++++++++++++++++++++++++++++++++++++++++++++++++++++++++++++++++++++++++++++++++++++++++++++++++++++++++++++++++++++++
					\section{Existence d'intégrales}
%+++++++++++++++++++++++++++++++++++++++++++++++++++++++++++++++++++++++++++++++++++++++++++++++++++++++++++++++++++++++++++

En vertu des différents théorèmes, l'étude de la convergence et de l'existence d'une intégrale d'une fonction sur $[a,\infty[$ se fait dans l'ordre suivant :
\begin{itemize}
\item si l'intégrale de $f$ existe, alors elle converge,
\item si la fonction est positive et que son intégrale n'existe pas, alors elle ne converge pas,
\item si le signe de la fonction est variable et que l'intégrale n'existe pas, alors elle peut converger ou non, selon les cas.
\end{itemize}

	\Exo{_I-2-1}
	\Exo{_I-2-2}
	\Exo{_I-2-3}
	\Exo{_I-2-4}

%+++++++++++++++++++++++++++++++++++++++++++++++++++++++++++++++++++++++++++++++++++++++++++++++++++++++++++++++++++++++++++
					\section{Fonctions définies par des intégrales}
%+++++++++++++++++++++++++++++++++++++++++++++++++++++++++++++++++++++++++++++++++++++++++++++++++++++++++++++++++++++++++++


	\Exo{_I-3-1}
	\Exo{_I-3-2}

%+++++++++++++++++++++++++++++++++++++++++++++++++++++++++++++++++++++++++++++++++++++++++++++++++++++++++++++++++++++++++++
					\section{Convergence, continuité et dérivation sous le signe intégral}
%+++++++++++++++++++++++++++++++++++++++++++++++++++++++++++++++++++++++++++++++++++++++++++++++++++++++++++++++++++++++++++

	\Exo{_I-3-4}
	\Exo{_I-3-5}
	\Exo{_I-3-6}
	\Exo{_I-3-7}
	\Exo{_I-3-8}
	\Exo{_I-3-9}
	\Exo{_I-3-10}
	\Exo{_I-3-11}
%+++++++++++++++++++++++++++++++++++++++++++++++++++++++++++++++++++++++++++++++++++++++++++++++++++++++++++++++++++++++++++
					\section{Quelques propriétés des espaces fonctionnels}
%+++++++++++++++++++++++++++++++++++++++++++++++++++++++++++++++++++++++++++++++++++++++++++++++++++++++++++++++++++++++++++

	\Exo{_I-4-1}
	\Exo{_I-4-2}
	\Exo{_I-4-3}
