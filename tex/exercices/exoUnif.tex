% This is part of the Exercices et corrigés de mathématique générale.
% Copyright (C) 2009-2010,2015-2016
%   Laurent Claessens
% See the file fdl-1.3.txt for copying conditions.

%+++++++++++++++++++++++++++++++++++++++++++++++++++++++++++++++++++++++++++++++++++++++++++++++++++++++++++++++++++++++++++ 
\section{Prérequis}
%+++++++++++++++++++++++++++++++++++++++++++++++++++++++++++++++++++++++++++++++++++++++++++++++++++++++++++++++++++++++++++

Cette section contient quelques exercices du type de ce qui est plus ou moins censé être connu à l'entrée de l'université dans diverses sections scientifiques\footnote{Ils proviennent surtout d'un cours pour ingénieur de Louvain-la-Neuve.}.

\Exo{INGE1114-0006}
\Exo{INGE1114-0008}
\Exo{INGE1114-0009}

\Exo{INGE1114-0010}
\Exo{INGE1114-0011}
\Exo{INGE1114-0012}
%\Exo{INGE1114-0013}
%\Exo{INGE1114-0014}
%\Exo{INGE1114-0015}
\Exo{INGE1114-0016}
\Exo{INGE1114-0017}
\Exo{INGE1114-0018}
%\Exo{INGE1114-0019}
%\Exo{INGE1114-0020}


\Exo{INGE11140023}
\Exo{INGE11140024}
\Exo{INGE11140025}
\Exo{INGE11140027}

%+++++++++++++++++++++++++++++++++++++++++++++++++++++++++++++++++++++++++++++++++++++++++++++++++++++++++++++++++++++++++++
\section{Limites et continuité}
%+++++++++++++++++++++++++++++++++++++++++++++++++++++++++++++++++++++++++++++++++++++++++++++++++++++++++++++++++++++++++++

\Exo{INGE11140028}
\Exo{INGE11140029}
\Exo{INGE11140030}
\Exo{INGE11140031}
\Exo{INGE11140032}

%+++++++++++++++++++++++++++++++++++++++++++++++++++++++++++++++++++++++++++++++++++++++++++++++++++++++++++++++++++++++++++
\section{Suites numériques}
%+++++++++++++++++++++++++++++++++++++++++++++++++++++++++++++++++++++++++++++++++++++++++++++++++++++++++++++++++++++++++++

\Exo{INGE11140033}
\Exo{INGE11140034}
\Exo{INGE11140035}
\Exo{INGE11140036}
\Exo{INGE11140037}
% This is part of the Exercices et corrigés de mathématique générale.
% Copyright (C) 2009-2011
%   Laurent Claessens
% See the file fdl-1.3.txt for copying conditions.
%+++++++++++++++++++++++++++++++++++++++++++++++++++++++++++++++++++++++++++++++++++++++++++++++++++++++++++++++++++++++++++
					\section{Limites}
%+++++++++++++++++++++++++++++++++++++++++++++++++++++++++++++++++++++++++++++++++++++++++++++++++++++++++++++++++++++++++++

\Exo{General0010}
\Exo{General0011}
\Exo{0013}
\Exo{0017}
\Exo{0016}
\Exo{0024}

%+++++++++++++++++++++++++++++++++++++++++++++++++++++++++++++++++++++++++++++++++++++++++++++++++++++++++++++++++++++++++++
					\section{Dérivées et optimisation}
%+++++++++++++++++++++++++++++++++++++++++++++++++++++++++++++++++++++++++++++++++++++++++++++++++++++++++++++++++++++++++++

\Exo{General0012}
\Exo{General0013}
\Exo{General0014}
\Exo{General0015}
\Exo{General0016}

%+++++++++++++++++++++++++++++++++++++++++++++++++++++++++++++++++++++++++++++++++++++++++++++++++++++++++++++++++++++++++++
					\section{Primitives et intégration}
%+++++++++++++++++++++++++++++++++++++++++++++++++++++++++++++++++++++++++++++++++++++++++++++++++++++++++++++++++++++++++++

\Exo{General0017}
\Exo{General0018}
\Exo{General0019}
\Exo{General0020}
\Exo{General0021}
\Exo{General0022}
\Exo{General0023}
\Exo{General0024}
\Exo{General0025}
\Exo{General0026}
\Exo{General0027}

%---------------------------------------------------------------------------------------------------------------------------
					\subsection{Longueur d'un arc de courbe}
%---------------------------------------------------------------------------------------------------------------------------

\Exo{Inter0012}
\Exo{Inter0013}

%---------------------------------------------------------------------------------------------------------------------------
					\subsection{Aire d'une surface de révolution}
%---------------------------------------------------------------------------------------------------------------------------

\Exo{Inter0015}
\Exo{Inter0014}
\Exo{Inter0016}

% This is part of the Exercices et corrigés de mathématique générale.
% Copyright (C) 2009
%   Laurent Claessens
% See the file fdl-1.3.txt for copying conditions.
%+++++++++++++++++++++++++++++++++++++++++++++++++++++++++++++++++++++++++++++++++++++++++++++++++++++++++++++++++++++++++++
					\section{Équations différentielles}
%+++++++++++++++++++++++++++++++++++++++++++++++++++++++++++++++++++++++++++++++++++++++++++++++++++++++++++++++++++++++++++

%---------------------------------------------------------------------------------------------------------------------------
					\subsection{Équations à variables séparées}
%---------------------------------------------------------------------------------------------------------------------------

\Exo{EquaDiff0001}

%---------------------------------------------------------------------------------------------------------------------------
					\subsection{Équations homogènes}
%---------------------------------------------------------------------------------------------------------------------------

\Exo{EquaDiff0002}

%---------------------------------------------------------------------------------------------------------------------------
					\subsection{Équations linéaires}
%---------------------------------------------------------------------------------------------------------------------------

\Exo{EquaDiff0003}

%---------------------------------------------------------------------------------------------------------------------------
					\subsection{Problèmes divers}
%---------------------------------------------------------------------------------------------------------------------------

\Exo{EquaDiff0004}
\Exo{EquaDiff0005}
\Exo{EquaDiff0006}
\Exo{EquaDiff0007}
\Exo{EquaDiff0008}
\Exo{EquaDiff0009}

%---------------------------------------------------------------------------------------------------------------------------
					\subsection{Équations différentielles du second ordre}
%---------------------------------------------------------------------------------------------------------------------------

\Exo{EquaDiff0010}
\Exo{EquaDiff0011}
\Exo{EquaDiff0012}


\Exo{EquaDiff0013}
\Exo{EquaDiff0015}
\Exo{EquaDiff0014}
\Exo{EquaDiff0016}


% This is part of the Exercices et corrigés de mathématique générale.
% Copyright (C) 2009-2010
%   Laurent Claessens
% See the file fdl-1.3.txt for copying conditions.
%+++++++++++++++++++++++++++++++++++++++++++++++++++++++++++++++++++++++++++++++++++++++++++++++++++++++++++++++++++++++++++
					\section{Fonctions de deux variables réelles}
%+++++++++++++++++++++++++++++++++++++++++++++++++++++++++++++++++++++++++++++++++++++++++++++++++++++++++++++++++++++++++++

%---------------------------------------------------------------------------------------------------------------------------
					\subsection{Tracer}
%---------------------------------------------------------------------------------------------------------------------------

\Exo{FoncDeuxVar0001}

%---------------------------------------------------------------------------------------------------------------------------
\subsection{Limites à deux variables}
%---------------------------------------------------------------------------------------------------------------------------

\Exo{FoncDeuxVar0010}
\Exo{FoncDeuxVar0011}
\Exo{FoncDeuxVar0012}
\Exo{FoncDeuxVar0013}
\Exo{FoncDeuxVar0014}
\Exo{FoncDeuxVar0015}

\Exo{FoncDeuxVar0016}
\Exo{FoncDeuxVar0018}	

%---------------------------------------------------------------------------------------------------------------------------
\subsection{Dérivées partielles, différentielles totales}
%---------------------------------------------------------------------------------------------------------------------------
\Exo{FoncDeuxVar0002}
\Exo{FoncDeuxVar0003}

%---------------------------------------------------------------------------------------------------------------------------
\subsection{Différentiabilité, accroissements finis}
%---------------------------------------------------------------------------------------------------------------------------

\Exo{FoncDeuxVar0019}
\Exo{Maximisation-0001}
\Exo{FoncDeuxVar0026}
\Exo{FoncDeuxVar0021}
\Exo{FoncDeuxVar0022}
\Exo{FoncDeuxVar0023}
\Exo{DerrivePartielle-0000}
\Exo{DerrivePartielle-0001}
\Exo{FoncDeuxVar0025}

%---------------------------------------------------------------------------------------------------------------------------
\subsection{Plan tangent}
%---------------------------------------------------------------------------------------------------------------------------

\Exo{FoncDeuxVar0027}
\Exo{DerrivePartielle-0002}

%---------------------------------------------------------------------------------------------------------------------------
\subsection{Dérivées de fonctions composées}
%---------------------------------------------------------------------------------------------------------------------------

\Exo{DerrivePartielle-0003}
\Exo{FoncDeuxVar0017}
\Exo{DerrivePartielle-0004}
\Exo{DerrivePartielle-0005}
\Exo{FoncDeuxVar0030}
\Exo{FoncDeuxVar0024}
\Exo{FoncDeuxVar0020}
\Exo{DerrivePartielle-0006}

%---------------------------------------------------------------------------------------------------------------------------
\subsection{Dérivées de fonctions implicites}
%---------------------------------------------------------------------------------------------------------------------------
\Exo{FoncDeuxVar0004}
\Exo{FoncDeuxVar0005}
\Exo{FoncDeuxVar0006}
\Exo{FoncDeuxVar0007}

%---------------------------------------------------------------------------------------------------------------------------
\subsection{Extrema}
%---------------------------------------------------------------------------------------------------------------------------

\Exo{FoncDeuxVar0008}
\Exo{FoncDeuxVar0009}
\Exo{FoncDeuxVar0029}
\Exo{FoncDeuxVar0028}
\Exo{DerrivePartielle-0007}
\Exo{Maximisation-0002}
\Exo{DerrivePartielle-0008}
\Exo{DerrivePartielle-0009}
\Exo{DerrivePartielle-0010}
\Exo{Maximisation-0000}

% This is part of the Exercices et corrigés de mathématique générale.
% Copyright (C) 2009-2011,2014
%   Laurent Claessens
% See the file fdl-1.3.txt for copying conditions.
Lorsque nous demandons d'étudier une fonction, nous demandons les éléments suivants : domaine de définition, croissance, extrema, points d'inflexion, asymptote et dessiner le graphe.


\Exo{III-1}
\Exo{III-2}
\Exo{III-3}
\Exo{III-4}
\Exo{III-5}
\Exo{TP40001}
\Exo{TP40002}
\Exo{TP40003}
\Exo{TP40004}
\Exo{TP40005}
\Exo{TP50001}
\Exo{TP50002}
\Exo{TP50003}
\Exo{TP50004}

%---------------------------------------------------------------------------------------------------------------------------
\subsection{Quelques fautes usuelles}
%---------------------------------------------------------------------------------------------------------------------------

Pour l'exercice \ref{exoTP40001}, les fautes les plus souvent commises sont
\begin{enumerate}

	\item
		$f'= e^{2x}$ implique $f=\frac{1}{ 2 } e^{x}$. Cela n'est pas vrai. La dérivée de $ e^{2x}$ est $2 e^{2x}$. Le $2$ reste dans l'exponentielle.

	\item
		Lorsqu'on intègre par partie, il faut aussi mettre les bornes pour le morceau qui n'est pas dans la nouvelle intégrale :
		\begin{equation}
			\int_a^b fg'=[fg]_a^b-\int_a^bf'g.
		\end{equation}
\end{enumerate}

Pour l'exercice \ref{exoTP40002}, les fautes les plus souvent commises sont
\begin{enumerate}

	\item
		Lorsqu'on a trouvé la solution générale $y_k(x)$ qui dépend du paramètre $k$ (ou $C$), il faut encore trouver la valeur du paramètre $k$ telle que $y_k(\pi)=0$.

\end{enumerate}

Pour l'exercice \ref{exoTP40003}, les fautes les plus souvent commises sont
\begin{enumerate}

	\item
		Ne pas oublier que $e^0=1$.
\end{enumerate}


% This is part of Exercices et corrigés de CdI-1
% Copyright (c) 2011,2014
%   Laurent Claessens
% See the file fdl-1.3.txt for copying conditions.

\section{Intégrales de surface, Stokes et Green}


\setcounter{CountExercice}{0}


\noindent{\bf Exercice 6}\\

{\bf $(a)$ La suite $[k\rightarrow \f{1}{k}]$ est convergente.}\\

\noindent Nous allons montrer que cette suite converge vers $0$. Il faut donc prouver la chose suivante: 
   \begin{equation}\label{eqn1}\forall \epsilon >0 \hspace{0,3cm} \exists K_\epsilon \in \eN \hspace{0,3cm} {\rm tq}  \hspace{0,3cm}  \forall k\geq K_\epsilon, \hspace{0,3cm}  |x_k-x|<\epsilon\end{equation}
{Remarque}: On pourrait également montrer que cette suite est {\it de Cauchy} pour prouver qu'elle est convergente sans devoir déterminer sa limite.\\

\noindent Pour prouver que (\ref{eqn1}) s'applique bien à la suite des $\f{1}{k}$ il nous faut montrer que

   \begin{equation}\label{eqn2}\forall \epsilon >0 \hspace{0,3cm} \exists K_\epsilon \in \eN \hspace{0,3cm} {\rm tq}  \hspace{0,3cm}  \forall k\geq K_\epsilon,  \hspace{0,3cm} \f{1}{k}<\epsilon\end{equation}

   \noindent Ceci est une conséquence immédiate de l'exercice précédent. On peut également le montrer de la manière suivante: à $\epsilon$ positif donné, si nous arrivons à déterminer l'indice $K_\epsilon$ de \eqref{eqn2} tel que $\forall k\geq K_\epsilon,  \hspace{0,3cm} \f{1}{k}<\epsilon$, il est clair que la suite satisfait à la définition. Or, $\f{1}{k} < \epsilon \leftrightarrow \f{1}{\epsilon} <k$. Donc si nous prenons $K_\epsilon := \ulcorner  1/\epsilon \urcorner+1$, on a bien que $\forall k\geq K_\epsilon$, $\f{1}{k}<\epsilon$, ce qui est ce qu'il fallait démontrer.

\vspace{1cm}
{\bf $(b)$ La suite $(1, \f{1}{2}, -\f{1}{3},  \f{1}{4}, -\f{1}{5}, \ldots )$ est convergente.}\\

\noindent On remarque que cette suite tend vers zéro. (Il suffit de voir que le numérateur est borné et que le dénominateur  tend vers l'infini). Si on l'écrit  sous la forme standard, on obtient:                 
              \[x_1 = 1, x_k = \f{(-1)^k}{k} \hspace{0.3cm} \forall k\geq 2\] 
Donc, ce que nous voulons voir est que $x_k \longrightarrow_{k\rightarrow  \infty} 0$, i.e.: 
   \begin{equation}\label{eqn3}\forall \epsilon >0 \hspace{0,3cm} \exists K_\epsilon \in \eN \hspace{0,3cm} {\rm tq}                       
       \hspace{0,3cm}  \forall k\geq K_\epsilon,  \hspace{0,3cm} |\f{(-1)^k}{k}|<\epsilon\end{equation}
       
\noindent Étant donné que $|(-1)^k| = 1 \, \forall k$, il est clair que l'équation (\ref{eqn3}) est la même que l'équation (\ref{eqn2}), et donc que l'on peut affirmer que pour tout $\epsilon > 0$, il suffit de prendre $K\geq \f{1}{\epsilon}$ et la condition est satisfaite.

\noindent{\bf Exercice 7}\\

Ici il est demandé de prouver de nouvelles règles de calcul en repartant de la définition de la convergence vers l'infini:
\begin{equation}
 \label{eqnconvinfGene} x_k \longrightarrow \infty \hspace{0.3cm} {\rm si} \hspace{0.3cm}  \forall M > 0 \hspace{0.3cm} \exists K_M \in \eN \hspace{0.3cm} {\rm tq} \hspace{0.3cm} \forall k \geq K_M, x_k \geq M \end{equation}
{\bf (a) $ \lim(x_k+y_k) = +\infty$.}\\

\noindent On veut voir  la chose suivante:
\begin{equation}\label{eqnconvinfCasA}
   \forall M > 0 \hspace{0.3cm} \exists K_M \in \eN \hspace{0.3cm} {\rm tq} \hspace{0.3cm} \forall k \geq K_M, x_k + y_k \geq M 
  \end{equation}

\noindent Soit $M> 0$. Comme $x_k$ et $y_k$ convergent à l'infini, on sait que 
\[\left\{\begin{array}{c}   
         \exists K^x_M \in \eN \hspace{0.3cm} {\rm tq} \hspace{0.3cm} \forall k \geq K^x_M, x_k \geq \f{M}{2}\\																		 
        \exists K^y_M \in \eN \hspace{0.3cm} {\rm tq} \hspace{0.3cm} \forall k \geq K^y_M, y_k \geq \f{M}{2},																		
\end{array}\right.\]
et donc il suffit de prendre $K_M = \max(K_M^x, K_M^y)$ dans (\ref{eqnconvinfCasA}) pour s'assurer que la définition est satisfaite.


\vspace{0.5cm}
\noindent{\bf (b) $ \lim(x_ky_k) = +\infty$.}\\

\noindent On veut voir la chose suivante:
\begin{equation}
 \label{eqnconvinfprod}  \forall M > 0 \hspace{0.3cm} \exists K_M \in \eN \hspace{0.3cm} {\rm tq} \hspace{0.3cm} \forall k \geq K_M, x_k  y_k \geq M \end{equation}

\noindent Soit $M> 0$. Comme $x_k$ et $y_k$ convergent à l'infini, on sait que 
\[\left\{\begin{array}{c}   
         \exists K^x_M \in \eN \hspace{0.3cm} {\rm tq} \hspace{0.3cm} \forall k \geq K^x_M, x_k \geq \sqrt M\\																		 
        \exists K^y_M \in \eN \hspace{0.3cm} {\rm tq} \hspace{0.3cm} \forall k \geq K^y_M, y_k \geq \sqrt M,																		
\end{array}\right.\]

\noindent et donc il suffit  de prendre  $K_M = \max(K_M^x, K_M^y)$ dans (\ref{eqnconvinfprod}) pour s'assurer que la définition est satisfaite.

\vspace{0.5cm}
\noindent{\bf (d) Soit $z_k$ une suite tendant vers un réel $a$ strictement positif. Prouvez que $\lim(x_k  z_k) = +\infty$.}\\

Le but de l'exercice est toujours le même, c'est à dire de prouver que 
\begin{equation}		\label{eqnconvinfz}
  \forall M > 0 \hspace{0.3cm} \exists K_M \in \eN \hspace{0.3cm} {\rm tq} \hspace{0.3cm} \forall k \geq K_M, \;x_k  z_k \geq M 
\end{equation}

\noindent Soit $M>0$. On sait  que:

\begin{equation}
\label{eqn12}\left\{\begin{array}{l}   
        \forall \tilde{M} >0 \;\exists K^x_{\tilde{M}} \in \eN \hspace{0.3cm} {\rm tq} \hspace{0.3cm} \forall k \geq K^x_{\tilde{M}},\; x_k \geq  \tilde{M} \\																		 
       \forall \epsilon >0\;\exists K^z_\epsilon \in \eN \hspace{0.3cm} {\rm tq} \hspace{0.3cm} \forall k \geq K^z_\epsilon,\; |z_k-a| <\epsilon,																		
\end{array}\right.\end{equation}

\noindent Prenons un $\epsilon$ tel que $a-\epsilon>0$. Par la deuxième partie de (\ref{eqn12}) on voit qu'il existe un indice $ K^z_\epsilon$ tel que $ \forall k \geq K^z_\epsilon,\; z_k > a-\epsilon >0$.

\noindent Prenons un $\tilde{M}$ tel que $M= \tilde{M}(a-\epsilon)$. Par la première partie de (\ref{eqn12}) on voit qu'il existe un indice $ K^x_{\tilde{M}} $ tel que $\forall k \geq K^x_{\tilde{M}},\; x_k \geq  \tilde{M} $.

												
\noindent et donc il suffit  de prendre  $K_M = \max(K_{\tilde{M}}^x, K^z_\epsilon)$ dans (\ref{eqnconvinfz}) pour avoir que 
\[ \forall k \geq K_M, \;x_k  z_k \geq \tilde{M}(a-\epsilon)=M.\]


\noindent{\bf Exercice 8}\\

\noindent Une suite $x_k$ est bornée si $\exists N>0$ tel que $\forall k$, $|x_k| < N$.

\noindent On veut voir que $\f{x_k}{y_k}\longrightarrow 0$, i.e.

\begin{equation} 
\label{eqnconvborne}  \forall  \epsilon > 0 \hspace{0.3cm} \exists K_\epsilon \in \eN \hspace{0.3cm} {\rm tq} \hspace{0.3cm} \forall k \geq K_\epsilon, \; |\f{x_k}{y_k}| < \epsilon \end{equation}

\noindent Soit $\epsilon >0$. Comme la suite $x_k$ est bornée, on a que  $|\f{x_k}{y_k}|<\f{N}{|y_k|}\; \forall k$. On utilise maintenant le fait que $y_k \longrightarrow \infty$. Prenons $M=\f{N}{\epsilon}$. On peut écrire que $\exists K_M$ tel que $\forall k \geq K_M, \; y_k \geq M=\f{N}{\epsilon}$, et donc si dans (\ref{eqnconvborne}) on prend $K_\epsilon= K_M$ on a:\[\forall k \geq K_\epsilon,\; \; |\f{x_k}{y_k}|<\f{N}{|y_k|}<\f{N}{N/\epsilon}=\epsilon.\]



\noindent{\bf Exercice 9}\\

\noindent Pour cet exercice, on peut utiliser les règles de calcul. Il faut faire attention que ces règles ne s'appliquent que si toutes les limites existent!

\vspace{0.5cm}
\noindent{ (a)} $x_k = \f{k+2}{k}\cos(k\pi)$\\

\noindent On voit que cette suite va dans deux directions différentes, $+1$ et $-1$ à cause du facteur $\cos(k\pi)=(-1)^k$. Elle ne converge donc pas. Pour le prouver, on peut prendre deux suites partielles de la suite $x_k$ qui convergent vers des limites différentes. 

\noindent Choisissons \[\left\{ \begin{array}{rcl} y_k &= x_{2k}&= \f{(2k)+2}{2k}(-1)^{2k}\\
 							  z_k &= x_{2k+1} &= \f{(2k+1)+2}{2k+1} (-1)^{2k+1}\end{array}\right.\]

\noindent Comme $x_k =\f{k+1}{k}= 1+\f{1}{k}$	et que $\f{1}{k} \rightarrow  0$, nous pouvons appliquer les règles de calcul et en déduire que $x_k \rightarrow  1$. On fait la même chose pour $y_k$.				  


\vspace{0.5cm}
\noindent{ (c)} $x_k = \f{k^3+k+1}{5k^3+2}$\\

\noindent Nous avons que \[\forall k, \;\;\;\;x_k =\; (\f{k^3}{k^3})\f{1+\f{1}{k} +\f{1}{k^3}}{5+\f{2}{k^3}}=\;\f{1+\f{1}{k} +\f{1}{k^3}}{5+\f{2}{k^3}} \]
Comme \[\forall k \geq 1\;\; \f{1}{k^3} \; \leq \;\f{1}{k^2}\; \leq \; \f{1}{k}\] et comme $\f{1}{k}\rightarrow 0$, nous pouvons appliquer la règle de l'étau pour voir que \[\f{1}{k^3} \rightarrow 0 \; \; \; {\rm et } \;\; \;\f{1}{k^2} \rightarrow 0.\]
En appliquant les règles de calcul à la suite $x_k$ transformée, on voit donc que $x_k \rightarrow  \f{1}{5}$.

\vspace{0.5cm}
\noindent{ (d)} $x_k = \f{k+(-1)^k}{k-(-1)^k}$\\

\noindent On peut le voir par exemple par la règle de l'étau:
\[\forall k \geq 0, \;\;\; \f{k-1}{k+1} \leq \f{k+(-1)^k}{k-(-1)^k} \leq \f{k+1}{k-1}. \]
Or, comme les deux suites qui bornent la suite $x_k$ convergent toutes les deux vers $1$, il est clair que $x_k$ converge aussi vers $1$.


\vspace{0.5cm}
\noindent{ (d)} $x_k = x_{k-1}^2\;+\;1,\, x_1=1$\\

\noindent Suite définie par récurrence. Ses premiers éléments sont \[1, \; 2, \; 5, \;  26, \; 677, \; \ldots\]
Toute  limite admissible réelle finie $l$  de cette suite doit satisfaire à \[l=l^2+1\] ce qui implique qu'elle ne peut avoir de limite réelle finie. En regardant ses premiers éléments, on remarque immédiatement qu'elle semble converger à l'infini. Nous allons le prouver en utilisant la définition.

\noindent Soit $M> 0$. On a que \[x_k \geq k \, \forall k.\] En effet (par récurrence sur $k$): il est clair que $x_1 \geq 1$. Supposons que $x_k \geq k$. Ceci implique t-il que $x_{k+1}\geq k+1$? Par définition des $x_k$, $x_{k+1} = x_k^2+1$. Par l'hypothèse de récurrence, on a donc $x_{k+1}\geq (k)^2 +1\geq k+1$ ce qui prouve le résultat. Comme la suite $y_k=k$ converge à l'infini, il en est de même pour la suite $x_k$.



\section{Continuité de fonctions réelles}


\begin{center}
\LARGE \bf
Travaux Personnels 
\end{center}

\begin{bf}
\begin{center}
BAC2 en sciences mathématiques et physiques
\end{center}
\end{bf}

{\bf Exercice 1.} Calculer les limites suivantes

\b
a) $\displaystyle \lim_{n \to \infty} \left( 1+ \frac{2}{n-4} \right)^n$

\medskip
b) 
$\displaystyle \lim_{n \to \infty} 
         \left( 1+ \frac 1n \right)^{\sqrt{n}}$

\medskip
c) $\displaystyle \lim_{x \to \infty} 
    \left( 1+ \frac \alpha x \right)^x$

\medskip
d) 
$\displaystyle \lim_{x \to 0} \frac{\log \left( 1+ \alpha x \right)}{x}$


\medskip
e) 
$\displaystyle \lim_{x \to \infty} 
\frac{a_0+a_1x + \dots +a_nx^n}{b_0+b_1x + \dots +b_mx^m}$
\quad où\, $a_j, b_j \in \eC$ \,et\, $n,m \ge 0$

\medskip
f) 
$\displaystyle \lim_{x \to 0} \frac{\sqrt{1-\cos x}}{x}$  




{\bf Exercice 2.} Prouver que

\medskip
a)
$\displaystyle \lim_{x \to \infty} x^{\frac 1x} = \lim_{x \to 0^+} x^x = 1$

\medskip
b)
$\displaystyle \lim_{x \to \infty} \frac{x^{\ln x}}{{\mathrm e}^x} =0$
\quad
càd ${\mathrm e}^x$ croit plus vite que $x^{\ln x}$


{\bf Exercice 3.} Prouver que
$$
\cosh 2x \,=\, \cosh^2 x + \sinh^2 x,
\qquad
\sinh 2x \,=\, 2 \sinh x \cosh x
$$


{\bf Exercice 4.} Prouver que

a)
$1 + \cos z + \cos 2z + \dots + \cos nz = \displaystyle \cos \frac{nz}{2} \cdot \frac{\sin (n+1)z/2}{\sin z/2}$

b)
$1 + \sin z + \sin 2z + \dots + \sin nz = \displaystyle \sin \frac{nz}{2} \cdot \frac{\sin (n+1)z/2}{\sin z/2}$

{\it Aide:}\;
$\displaystyle \sum_{k=0}^n 
\euler^{\sii kz} 
= 
\frac{1-\euler^{\sii (n+1)z}}{1-\euler^{\sii z}}
= \euler^{\sii nz/2} \cdot \frac{\euler^{\sii (n+1)z/2} - \euler^{-\sii (n+1)z/2}}{
\euler^{\sii z/2}-\euler^{-\sii z/2}}
$

Rappelons qu'une fonction $f \colon \mathbb{C} \supset D \to \eC$ est {\bf uniformément continue} si pour tout $\epsilon >0$ il existe un \( \delta>0\) tel que 
$$
|x-y| < \delta \,\Longrightarrow\, |f(x)-f(y)| < \epsilon 
\quad \text{ pour tout }\, x,y \in D.
$$
Prouver que la fonction $f \colon \eR \to \eR$, $x \mapsto x^2$ est continue, mais n'est pas uniformément continue.


\section{Intégrales, longueur de courbes, EDO's linéaires}


\exerNico 
Soient $n,m \in \eN \cup \{0\}$.
Calculer
$$
\int_0^1 x^n (1-x)^m \,dx
\quad \text{ et } \quad
\int_{-1}^1 (1+x)^n (1-x)^m \,dx
$$

{\bf Solution:}
Posons $I_{n,m} := \int_0^1 x^n (1-x)^m \,dx$.
Intégration par partie donne
la formule récursive
$$
I_{n,m} \,=\, \frac {m}{n+1} I_{n+1,m-1}.
$$
Avec $I_{n+m,0} = \frac{1}{n+m+1}$ nous obtenons
$$
I_{n,m} \,=\, \frac{n!\,m!}{(n+m+1)!}
$$
La substitution $x := 2t-1$ fournit
$$
\int_{-1}^1 (1+x)^n (1-x)^m \,dx
\,=\, 2^{n+m+1} \int_0^1 t^n (1-t)^m \,dt \,=\,  2^{n+m+1} 
\cdot I_{n,m}. 
$$




\exerNico 
Soient $a,b >0$. 
Calculer
$$
\int_0^{\pi /2} \displaystyle \frac{d \varphi}{a^2 \sin^2 \varphi + b^2 \cos^2 \varphi}
$$

{\bf Solution:}
$$
\,=\, \int_0^{\pi /2} \frac{1 / \cos^2 \varphi}{a^2 \tan^2 \varphi+b^2} d\varphi \,=\, \int_0^\infty \frac {dt}{a^2t^2 + b^2} \,=\, \frac{\pi}{2ab}  
$$


\exerNico  
Calculer la longueur de l'arc de la parabole $y = x^2,\;x \in [0,b]$.

\medskip
{\bf Solution:}
$$
s \,=\, \int_0^b \sqrt{1+4x^2} \,dx \,=\, \frac b 2 \sqrt{1+4b^2}+ \frac 14 \ln \left(2b+ \sqrt{1+4b^2} \right)
$$


\exerNico  
La {\bf parabole de Neil} $\nu$ est la courbe définie par $\nu (t) = (t^2,t^3)$, pour  $t \in \eR$.

\medskip
a)
Esquisser la parabole de Neil.

\medskip
b)
Quelle est la signification du paramètre $t$?

\medskip
{\bf Solution:} $t = \tan \alpha$

\medskip
c)
Calculer la longueur de l'arc 
$\left\{ \nu (t) \mid t \in [0,\tau] \right\}$.


\medskip
{\bf Solution:}
$$
s \,=\, \int_0^\tau \sqrt{4 t^2+9t^4} \,d\tau \,=\, \frac{8}{27} \left( \left(1+ \frac 94 \tau^2\right)^{3/2}-1 \right)
$$



\exerNico  
La {\bf hélice} $\gamma$ de pas $2 \pi h$ est la courbe dans $\eR^3$ définie par
$$
\gamma(t) \,=\, \left( r \cos t , r \sin t , h t \right)  .
$$


\medskip
a)
Esquisser la hélice.

\medskip
b)
Expliquer le mot ``pas''.


\medskip
c)
Calculer la longueur de l'arc sur la hélice si on fait un tour.

\medskip
{\bf Solution:} 
$\int_0^{2\pi} \sqrt{r^2+h^2} \,dt \,=\, 2 \pi \sqrt{r^2+h^2}$


\bigskip
\exerNico 
Calculer un système fondamental réel pour

\medskip
a) $y^{(4)}-y = 0$,

\medskip
b) $y^{(4)} +4y'' +4y = 0$,

\medskip
c) $y^{(4)} -2y^{(3)} +5y'' = 0$.


\bigskip
{\bf Solution:}

\medskip
a) ${\rm e}^x, {\rm e}^{-x}, \cos x, \sin x$

\medskip
b) $\cos \sqrt{2} x, x \cos \sqrt{2}x, \sin \sqrt{2}x, x \sin \sqrt{2}x$

\medskip
c)
$1, x, {\rm e}^x \cos 2x, {\rm e}^x \sin 2x$



\bigskip
\exerNico 
Déterminer une solution particulière de l'équation
$y''+y=q$ pour

\medskip
a) $q = x^3$,

\medskip
b) $q = \sinh x$,

\medskip
c) $q = 1/\sin x$.
 

\bigskip
{\bf Solution:}

\medskip
a) $x^3 - 6 x$

\medskip
b) $\frac 12 \sinh x$

\medskip
c) $\sin x \cdot \ln |\sin x| - x \cos x$


\bigskip
\exerNico  
L'équation différentielle $m \ddot y = mg - k\dot y$ 
décrit la chute d'un corps soumit
à la gravitation si la friction est proportionnelle à la vitesse (``un homme tombant de l'avion'').

\medskip
Calculer la solution avec $y(0) =0, \dot y(0) = 0$.
Calculer la ``vitesse finale'' $v_\infty = \displaystyle \lim_{t \to \infty} \dot y (t)$.



\bigskip
{\bf Solution:}

\medskip
L'équation homogène $\ddot y + k/m \cdot y = 0$
possède les solutions $c_1+c_2 {\rm e}^{-k/m \cdot t}$,
où $c_1, c_2 \in \eR$.
 
L'équation inhomogène $\ddot y + k/m \cdot y = g$
possède comme solution particulière une fonction lineaire, càd 
$y_p = (mg/k)t)$.
En tenant compte des conditions initiales nous obtenons
$$
y(t) \,=\, \frac{mg}{k} \left( t-\frac mk (1-{\rm e}^{-k/m \cdot t})\right).
$$
En particulier, $v_\infty = mg/k$. 

 




\bigskip
\exerNico  
Regardons l'ensemble des solutions de l'équation différentielle $P({\rm D})y =0$.

Montrer l'équivalence entre les propositions suivantes :
\begin{enumerate}

\item
Pour toute solution $y$ on a $\displaystyle \lim_{t \to \infty} y(t) = 0$

\item
Pour toute racine $z$ du polynôme caractéristique on a ${\rm Re}\, z <0$.

\end{enumerate}
Dans ce cas, l'équation différentielle est appelé  \defe{asymptotiquement stable}{asymptotiquement stable}.

\bigskip
{\bf Solution:}
On a 
$\displaystyle \lim_{t \to \infty} y(t) = 0$ pour toute solution $y$ ssi c'est vrai pour tout élément d'un système fondamental.
On a $\displaystyle \lim_{t \to \infty} t^k {\rm e}^{\gl t}=0$ ssi ${\rm Re }\,\gl <0$,
d'où l'affirmation suit.






\section{Calcul de limites}

\exerNico Déterminez si les limites suivantes existent et dans
l'affirmative calculez les en utilisant, s'il y a lieu, la règle de
l'Hospital ou la règle de l'étau.
\begin{enumerate}
\item $  \lim_{x \rightarrow  +\infty} \frac{x+1}{x^2+2} $
\item $  \lim_{x \rightarrow  +\infty} \frac{\sin(x)}{x} $
\item $  \lim_{x \rightarrow  0} \frac{\sin(x)}{x} $
\item $  \lim_{x \rightarrow  +\infty}  \frac{x ^n}{e ^x} $
\item $  \lim_{x \rightarrow  +\infty} (1 + \frac{a}{x})^x $
\item $  \lim_{x \rightarrow  0} (\frac{1}{\sin(x)} - \frac{1}{x} )$
\item $  \lim_{x \rightarrow  +\infty} \cos( 2 \pi x) $
\item $  \lim_{x \rightarrow  +\infty} \frac{1}{\sin(x)+2}(x) +\ln(x)\cos(x) $
\item $  \lim_{x \rightarrow  +\infty} \frac{ \ln(x)(\sin(x) +2)}{x} $
\item $  \lim_{x \rightarrow  +\infty} x ^\frac{1}{x} $
\end{enumerate}

\exerNico Déterminez si les limites suivantes existent et dans
l'affirmative calculez-les.
\begin{enumerate}
\item $  \lim_{x \rightarrow  0} x \sin(\frac{1}{x}) $
\item $  \lim_{x \rightarrow  0} \frac{\sin(\sin(x))}{x} $
\item $  \lim_{x \rightarrow  +\infty} (\ln(x))^\frac{1}{1 - \ln(x)}$
\end{enumerate}

\exerNico Calculez les limites suivantes:
\begin{enumerate}
\item $  \lim_{x \rightarrow  +\infty} \frac{\ln(x)}{x ^a} $
\item $  \lim_{x \rightarrow  +\infty} \frac{\ln(x)^a}{x ^b} $
\item $  \lim_{x \rightarrow  +\infty} a ^x $
\item $  \lim_{x \rightarrow  +\infty} a ^\frac{1}{x} $
\end{enumerate}
où $a$ et $b$ sont des réels positifs.
%

%

\exerNico Déterminez, pour chacune des suites suivantes, si elle converge
et dans l'affirmative calculez sa limite.
\begin{enumerate}
\item $  k \rightarrow  \cos( 2 \pi k) $
\item $  k \rightarrow  \cos(\frac{\pi}{3} k) $
\item $  k \rightarrow  k(a ^\frac{1}{k} -1 ) $
\end{enumerate}
où $a$ est une réel.\\



\exerNico Calculez  les limites suivantes si elles existent.
\begin{enumerate}
\item $  \lim_{x \rightarrow  +\infty} \cos x $
\item $  \lim_{x \rightarrow  \pm \infty }\sqrt{2x^4+3}-x^2 $

\end{enumerate}

\exerNico Déterminez si la limite de chacune des suites suivantes
existe et dans l'affirmative calculez la.
\begin{enumerate}
\item $  \lim_{k \rightarrow  +\infty }(\frac{a k +1}{k})^k $
\item $  \lim_{k \rightarrow  +\infty}\frac{1}{\sin(\frac{\pi}{6}k)+1}(k) + \ln(k)\cos(\frac{\pi}{5}k)$
\item $  \lim_{k \rightarrow  +\infty} \frac{\ln(k)(\sin(\frac{\pi}{3}k) +1)}{k} $
\item $  \lim_{k \rightarrow  +\infty } \sqrt[3k]{k} (1 +
\frac{1}{3k})^{3k} $
\end{enumerate}
où $a$ est un réel. 

\section{Dérivabilité}



\exerNico Déterminez l'ensemble des points où les fonctions suivantes
sont continues et celui où elles sont dérivables. Prouvez soigneusement
vos résultats.
\begin{enumerate}
\item $ x \rightarrow x]$
\item $ x \rightarrow |x| $
\item $ x \rightarrow
	\left\{ \begin{array}{ll}
	\frac{1}{x} & \mbox{si } x \not= 0 \\
	0 & \mbox{sinon}
	\end{array} \right. $
\item $ x \rightarrow x^2  $
\end{enumerate}




\exerNico Étudiez la dérivabilité et la continuité
de la dérivée de chacune des fonctions suivantes:
\begin{enumerate}
\item $ x \rightarrow
\left\{ \begin{array}{ll}
0 & \mbox{si } x \not= 0 \\
1 & \mbox{sinon}
\end{array} \right.$
%
\item $ x \rightarrow
\left\{ \begin{array}{ll}
\sin(\frac{1}{x}) & \mbox{si } x \not= 0 \\
0 & \mbox{sinon}
\end{array} \right.$
%
\item $ x \rightarrow
\left\{ \begin{array}{ll}
x \sin(\frac{1}{x}) & \mbox{si } x \not= 0 \\
0 & \mbox{sinon}
\end{array} \right.$
%
\item $ x \rightarrow
\left\{ \begin{array}{ll}
x^2 \sin(\frac{1}{x}) & \mbox{si } x \not= 0 \\
0 & \mbox{sinon}
\end{array} \right.$
\end{enumerate}

Le but de cet exercice est aussi d'exhiber des exemples illustrant les
différents types de comportements possibles, relativement à la
continuité et la dérivabilité, d'une fonction en un point.

\exerNico Étudiez la dérivabilité et la continuité
de la dérivée de chacune des fonctions suivantes:
\begin{enumerate}
\item $ x \rightarrow
\left\{ \begin{array}{ll}
\frac{2x+a}{1+e^{\frac{1}{x}}} & \mbox{si } x \not= 0 \\
0 & \mbox{sinon}
\end{array} \right.$
%
\item $ x \rightarrow
\left\{ \begin{array}{ll}
\frac{\sin(x)}{x} & \mbox{si } x \not= 0 \\
1 & \mbox{sinon}
\end{array} \right.$
%
\item $ x \rightarrow
\left\{ \begin{array}{ll}
e^{\frac{-1}{x}} & \mbox{si } x > 0 \\
0 & \mbox{sinon}
\end{array} \right.$
%
\item $ [-\frac{1}{2}, \frac{1}{2}] \rightarrow \eR: x \rightarrow
\left\{ \begin{array}{ll}
(\frac{\sin(2x)}{x})^{x+1} & \mbox{si } x \not= 0 \\
1 & \mbox{sinon}
\end{array} \right.$
\end{enumerate}
où $a$ et $b$ sont des réels.


 \exerNico Considérons la fonction
$$f:\mathbb{R}\rightarrow\mathbb{R}:x\mapsto f(x)=\left\{
\begin{array}{ll}
x&\text{si }x\text{ est rationnel}\\
0&\text{si }x\text{ est irrationnel}
\end{array}
\right.$$

Vérifiez que $f$ est continue en $0$ mais n'est ni dérivable à  gauche ni dérivable à droite en
$0$.

\exerNico 
\begin{enumerate}
\item Soit $(X,d)$ un espace métrique et $f \colon (X,d) \to \eR$ une application continue.
Montrer que l'ensemble $$\left\{ x \mid f(x) = 0 \right\}$$ est fermé.

\item Soit $f \colon \eR \to \eR$ une application continue.
Montrer que l'ensemble 
$$
\{ x \in \eR \mid f(x) = x\}
$$
des points fixes de $f$ est fermé.

\end{enumerate}

\exerNico  Soit $A$ un sous-ensemble de l'espace métrique $(X,d)$.
Montrer que la fonction
$$
\dist_A \colon X \to \eR,
\quad x \mapsto \inf_{a \in A} d(a,x)
$$ 
est continue.


\exerNico  Soient $(X,d_X)$, $(Y,d_Y)$ deux espaces métriques.
Une application $f \colon X \to Y$ est {\bf Lipschitzienne}
s'il existe une constante $L \ge 0$ telle que
$$
d_Y \bigl( f(x), f(x') \bigr) \,\le\, L \,d_X (x,x') 
\quad \text{ pour tout } x,x' \in X.
$$
Dans ce cas, on dit que $f$ est {\bf $L$-Lipschitzienne}.


\begin{enumerate}
\item
Montrer qu'une application Lipschitzienne est continue.
\item Montrer qu'une application $f \colon \eR \to \eR$, $x \mapsto ax+b$
est Lipschitzienne.
Quelle est la plus petite constante $L$ qui convienne?

\item Montrer que les fonctions $z \mapsto |z|$, 
$z \mapsto \overline z$,
$z \mapsto {\rm Re\,} z$ et $z \mapsto {\rm Im\,} z$ 
de $\eC$ dans $\eR$ sont Lipschitziennes.
Quelle sont les plus petites constantes $L$ qui conviennent?
\item Montrer que la fonction $\dist_A \colon X \to \eR$ de l'Exercice~13 est Lipschitzienne.

\end{enumerate}

% This is part of Exercices et corrigés de CdI-1
% Copyright (c) 2011,2014
%   Laurent Claessens
% See the file fdl-1.3.txt for copying conditions.




\section{Intégration}

\exerNico 
Soient $n,m \in \eN \cup \{0\}$.
Calculer
$$
\int_0^1 x^n (1-x)^m \,dx
\quad \text{ et } \quad
\int_{-1}^1 (1+x)^n (1-x)^m \,dx
$$



\exerNico 
Soient $a,b >0$. 
Calculer
$$
\int_0^{\pi /2} \displaystyle \frac{d \varphi}{a^2 \sin^2 \varphi + b^2 \cos^2 \varphi}
$$


\exerNico  
Calculer la longueur de l'arc de la parabole $y = x^2,\;x \in [0,b]$.

\exerNico  
La {\bf parabole de Neil} $\nu$ est la courbe définie par
$\nu (t) = (t^2,t^3), \, t \in \eR^n$.
\begin{enumerate}
\item Esquisser la parabole de Neil.

\item Quelle est la signification du paramètre $t$?

\item Calculer la longueur de l'arc 
$\left\{ \nu (t) \mid t \in [0,\tau] \right\}$.
\end{enumerate}

\exerNico  
Une {\bf hélice} $\gamma$ de pas $2 \pi h$ est une courbe dans $(\eR^n)^3$ définie par
$$
\gamma (t) \,=\, \left( r \cos t , r \sin t , h t \right)  .
$$

\begin{enumerate}
\item Esquisser $\gamma$ et expliquer le mot ``pas''.

\item Calculer la longueur de l'arc sur la hélice si on fait un tour.
\end{enumerate}

\exerNico Calculez la longueur des arcs de courbe suivants:
\begin{enumerate}
\item $y= \ln(1-x^2)  \hspace{3.5cm} 0\leq x\leq \f{1}{2}$
\item  $y= x^{3/2}  \hspace{4.57cm} 0\leq x\leq 5$
\item $y = 1-\ln(\cos x) \hspace{3cm} 0\leq x \leq \f{\pi}{4}$
\item l'arc de cubique déterminé par $y=x^3+x^2+x+1$ avec $0\leq x \leq 1$.
\end{enumerate}

% This is part of Exercices et corrigés de CdI-1
% Copyright (c) 2011,2013-2014
%   Laurent Claessens
% See the file fdl-1.3.txt for copying conditions.

%+++++++++++++++++++++++++++++++++++++++++++++++++++++++++++++++++++++++++++++++++++++++++++++++++++++++++++++++++++++++++++
                    \section{Quelque corrections}
%+++++++++++++++++++++++++++++++++++++++++++++++++++++++++++++++++++++++++++++++++++++++++++++++++++++++++++++++++++++++++++

Ce qui suit sont des corrections d'exercices donnée sur les feuilles distribuées au début de l'année.

\noindent 31.
\begin{enumerate}
\item $df_{(1,1)}$ et $dg_{(\sqrt2,\frac{\pi}{4})}$\\
    \[\begin{array}{l}\frac{ \partial f }{ \partial x }(x,y) = \frac{1}{y}\ln(\frac{x}{y})e^{\frac{x}{y}}+\frac{1}{x}e^{\frac{x}{y}}\\
            \frac{ \partial f }{ \partial x }(1,1)=e\\
            \frac{ \partial f }{ \partial y }(x,y)=-\frac{x}{y^2}\ln(\frac{x}{y})e^{\frac{x}{y}}-xe^{\frac{x}{y}}\\
        \frac{ \partial f }{ \partial x }(1,1)=e\end{array}\]
            
 \noindent Par les règles de calcul, $f$ est différentiable en $(1,1)$. la différentielle $df_{(1,1)}$ est donc représentée dans les bases canoniques de $\eR^2$ et $\eR$ par la matrice jacobienne (ici gradient):\[df_{(1,1)}=(e \;\; -e)\]
 
 \[\begin{array}{lclllllcl}\frac{ \partial g_1 }{ \partial r }(r,\theta) &=&\cos(\theta)& & & & \frac{ \partial g_1 }{ \partial \theta }(r,\theta)   & =&-r\sin(\theta)\\
         \frac{ \partial g_1 }{ \partial r }(\sqrt2, \frac{\pi}{4})&=&\frac{\sqrt2}{2}& & &&\frac{ \partial g_1 }{ \partial \theta }(\sqrt2, \frac{\pi}{4})& =&-1 \\
         \frac{ \partial g_2 }{ \partial r }(r,\theta) &=&\sin(\theta)&  && &\frac{ \partial g_2 }{ \partial \theta }(r,\theta)  &=&r\cos(\theta) \\
     \frac{ \partial g_2 }{ \partial r }(\sqrt2, \frac{\pi}{4})&=&-\frac{\sqrt2}{2}&& & &\frac{ \partial g_2 }{ \partial \theta }(\sqrt2, \frac{\pi}{4})& = &1\end{array}\]

La fonction $g$ est également différentiable en $(\sqrt2, \frac{\pi}{4})$ et sa matrice Jacobienne est:
\[dg_{(\sqrt2, \frac{\pi}{4})}=\left(\begin{array}{cc} \frac{\sqrt2}{2} & -1\\
    \frac{\sqrt2}{2}&1\end{array}\right)\]  


\item $\tilde{f} \;=\;e^{\cos(\theta)}\ln(\cos(\theta))$.
\item On voit d'abord que $g(\sqrt2, \frac{\pi}{4})\;=\;(1,1)$. Donc
    \[\begin{array}{cccc} d\tilde{f}_{(\sqrt2, \frac{\pi}{4})} & = & df_{g(\sqrt2, \frac{\pi}{4})}\circ dg_{(\sqrt2, \frac{\pi}{4})}\\
        & =& df_{(1,1)}\circ dg_{(\sqrt2, \frac{\pi}{4})} \end{array}\]
                                et  la matrice jacobienne de la différentielle de la composée est donc:\[d\tilde{f}_{(\sqrt2, \frac{\pi}{4})}=(e\;\;-e)\left(\begin{array}{cc} \frac{\sqrt2}{2} & -1\\
                                    \frac{\sqrt2}{2}&1\end{array}\right)=(0\;\;-2e)\]

                                        

\end{enumerate}


\noindent 32.
\begin{enumerate}
    \item $\frac{ \partial g }{ \partial u } = e^v\frac{ \partial f }{ \partial x }(\star,\star)+2uv\frac{ \partial f }{ \partial y }(\star,\star)$
    \item $\frac{ \partial g }{ \partial v } = ue^v\frac{ \partial f }{ \partial x }(\star,\star)+(1+u^2)\frac{ \partial f }{ \partial y }(\star,\star)$
\end{enumerate}
où $(\star,\star) = (ue^v,v(1+u^2))$.

\vspace{1cm}

\noindent 33. \\

\noindent Soit $g:\eR^2\rightarrow \eR:(x,y)\rightarrow  f(x^2-y^2)$. Dérivées partielles de:\[(x,y)\rightarrow  y(\partial_xg)(x,y)+x(\partial_yg)(x,y)?\]
Nommons cette fonction $h$. 
\begin{enumerate}
\item $\partial_xg(x,y) = 2xf'(x^2-y^2)$
\item$\partial_yg(x,y) = -2yf'(x^2-y^2)$
\end{enumerate}
et donc $h(x,y) = 0 \, \forall (x,y)\in \eR^2$.

\vspace{1cm}


\noindent 34. \\

\noindent $h(t)=f(t,g(t^2))$.\\

\begin{enumerate}
    \item $h'(t)=\frac{ \partial f }{ \partial x }(\star,\star)+\frac{ \partial f }{ \partial y }(\star,\star)2tg'(t^2)$
    \item $ \begin{array}{rl} h''(t)=     &   \frac{ \partial^2f }{ \partial x }(\star,\star)+4tg'(t^2)\frac{ \partial^2f }{ \partial x\partial y }(\star,\star)+4t^2(g'(t^2))^2\frac{ \partial^2f }{ \partial y^2 }(\star,\star) \\           
        & +[2g'(t^2)+4t^2g''(t^2)]\frac{ \partial f }{ \partial y }(\star,\star)\end{array}$

\end{enumerate}
où $(\star,\star) = (t,g(t^2))$.

\vspace{1cm}

\noindent 35.
\[h:\eR^2\rightarrow \eR:(u,v)\rightarrow  f(g(ue^v),g(v)(1+u^2))^{g(u+v)}\]

\noindent Comme toujours il vaut mieux faire ce genre d'exercices prudemment. Renommons donc les diverses composantes de cette fonction.\\

\noindent Soit $l(u,v)=f(g(ue^v),g(v)(1+u^2))$. On a alors:
\begin{enumerate}
    \item $\frac{ \partial l }{ \partial u }(u,v) = \frac{ \partial f }{ \partial x }(\star,\star)g'(ue^v)e^v + \frac{ \partial f }{ \partial y }(\star,\star)g(v)2u$
    \item $\frac{ \partial l }{ \partial v }(u,v) = \frac{ \partial f }{ \partial x }(\star,\star)g'(ue^v)ue^v+\frac{ \partial f }{ \partial y }(\star,\star)g'(v)(1+u^2)$
\end{enumerate}
o\`{u} $(\star,\star)=(g(ue^v),g(v)(1+u^2))$.\\

\noindent Alors $h(u,v)=l(u,v)^{g(u+v)} = e^{g(u+v)\ln(l(u,v))}$ qui est facile à dériver:

\begin{enumerate}
    \item $\frac{ \partial h }{ \partial u } = [g'(u+v)\ln(l(u,v))+\frac{g(u+v)}{l(u,v)}\frac{ \partial l }{ \partial u }(u,v)] l(u,v)^{g(u+v)}$
    \item $\frac{ \partial h }{ \partial v } = [g'(u+v)\ln(l(u,v))+\frac{g(u+v)}{l(u,v)}\frac{ \partial l }{ \partial v }(u,v)] l(u,v)^{g(u+v)}$
\end{enumerate}



\noindent 26.
\begin{enumerate}
\item $(x,y)\rightarrow  3x^2+x^3y+x$.\\
Combinaison linéaire de fonctions continues et différentiables sur $\eR^2$ (Exercice: prouver rigoureusement que les polynômes sont bien des fonctions continues et différentiables sur $\eR^2$).


\item \(  (x,y)\rightarrow\begin{cases}
        e    &   \text{si } xy\neq 0\\
        e^{x+y}    &    \text{sinon}
    \end{cases}\)
N.B.: Il est toujours utile de se représenter le domaine de chacune des fonctions.

\noindent La première remarque est que cette fonction est clairement continue et différentiable en tout point hors de $\{xy=0\}$ (fonction constante). Sur $\{xy=0\}$?
\begin{enumerate}
\item Continuité:\\
Prenons un point dans $\{xy=0\}$, par exemple le point $(a,0)$ (Remarquez que le cas $(0,b)$ est réglé par symétrie). Pour voir si la fonction est continue en ce point il faut voir si \[\lim_{(x,y)\rightarrow (0,0)}f(x,y)=f(0,0)=e^a.\] Si on prend deux manières différentes d'aller vers $(a,0)$ ($y=0$ puis $x=a$) on voit que si $a \neq1$ la fonction ne peut pas être continue. Et en $(1,0)$? Si on $(x,y)\rightarrow (1,0)$ avec d'abord $y=0$ puis $y\neq0$ on aura regardé toutes les manières de tendre vers $(1,0)$. Or dans les deux cas les limites valent $e = f(1,0)$, ce qui prouve que la fonction est continue en $(1,0)$ (et $(0,1)$ par symétrie).

\item Différentiabilité:\\
Comme la fonction est discontinue en tout point $(a,0)$ et $(0,b)$ avec $a\neq1$ et $b\neq1$ elle est aussi non différentiable en chacun de ces points. Il reste donc les points $(1,0)$ et $(0,1)$. Comme toujours, nous regardons d'abord les dérivées directionnelles en $(1,0)$:
\[\frac{ \partial f }{ \partial u }(1,0) \;=\;\lim_{t\rightarrow 0}\frac{f(1+tu_1,tu_2)-e}{t}\]
Il y a deux possibilités: $u_2=0$ et donc $u=(\pm1.0)$ ou$u_2\neq0$ (pourquoi ne regarde-t-on que ces deux cas?).

\begin{enumerate}
\item si $u\neq(\pm1,0)$.\\
    $\frac{ \partial f }{ \partial u }(1,0) \;=\;\lim_{t\rightarrow 0}\frac{e-e}{t}\;=\;0$.
\item si $u=(\pm1,0)$, i.e. si $u=(1,0) = e_1$\\
    $\frac{ \partial f }{ \partial u }(1,0) = \frac{ \partial f }{ \partial x }(1,0)=\lim_{t\rightarrow 0}\frac{f(1+t,0)-e}{t}=\lim_{t\rightarrow 0}\frac{e^{1+t}-e}{t} =^H0$.
\end{enumerate}
\end{enumerate}
\underline{Conclusion}:\\
Si $f$ était différentiable en $(1,0)$, on aurait que sa différentielle prendrait la forme suivante:
\[\begin{array}{cc} df_{(1,0)}u& = \frac{ \partial f }{ \partial x }(1,0)u_1+\frac{ \partial f }{ \partial y }(1,0)u_2\\
    & = eu_1\;\;\forall u\in\eR^2 \end{array} \]
Sa différentielle satisferait également à:
\[  df_{(1,0)}u = \frac{ \partial f }{ \partial u }(1,0) = 0 \;\; \forall u \neq (\pm1,0) \in \eR^2\]
Les deux propriétés étant contradictoires, la fonction $f$ ne peut être différentiable en $(1,0)$ (ni en $(0,1)$ par symétrie).               

\item \( \rightarrow  \begin{cases}
        \frac{ x }{ y }    &   \text{si } y\neq 0\\
        0    &    \text{sinon}
    \end{cases}\)

Continue et différentiable sur $\eR-\{y=0\}$. Sur l'axe $y=0$ elle n'est pas continue.        
\item \( \rightarrow  \begin{cases}
        x+ay    &   \text{si } x>0\\
        x    &    \text{sinon}
    \end{cases}\)
Si $a=0$ fonction continue et différentiable sur $\eR^2$. Si $a\neq0$, fonction continue et différentiable partout en dehors de l'axe $x=0$. Sur cet axe, elle est discontinue en tout point sauf en $(0,0)$ où elle est continue. Mais elle n'est pas différentiable en $(0,0)$ car toutes ses dérivées directionnelles  n'y sont pas définies.
\item \(  \rightarrow\begin{cases}
        \frac{ xy^5 }{ x^6+y^6 }    &   \text{si } x\neq y\\
        0    &    \text{sinon}
    \end{cases}\)
Fonction continue et différentiable partout en dehors de la droite $x=y$.  La fonction est discontinue en chacun des points de cette droite.

\end{enumerate}

30.
\begin{enumerate}
\item $(u,v)\rightarrow  u^3+12u^2v-5v^3$\\
\begin{enumerate}
    \item $\frac{ \partial f }{ \partial u } = 3u^2+24uv$
    \item $\frac{ \partial f }{ \partial v } = 12u^2-15v^2$
\end{enumerate}
\item $(u,v)\rightarrow  f(u^2)\ln(v)$\\
\begin{enumerate}
    \item $\frac{ \partial f }{ \partial u } = 2uf'(u^2)\ln(v)$
    \item $\frac{ \partial f }{ \partial v } = \frac{f(u^2)}{v}$
\end{enumerate}
\item $(x,y)\rightarrow \tan(x+y^2)$\\
\begin{enumerate}
    \item $\frac{ \partial f }{ \partial x } =\frac{1}{cos^2(x+y^2)}$
    \item $\frac{ \partial f }{ \partial v } = \frac{2y}{cos^2(x+y^2)}$
\end{enumerate} 
\item $(r,\theta)\rightarrow  r^\theta$
\begin{enumerate}
    \item $\frac{ \partial f }{ \partial r } =\theta r^{\theta-1}$
    \item $\frac{ \partial f }{ \partial \theta }<++> =\ln(r)r^\theta$
\end{enumerate}
\item $(x,y)\rightarrow (x+3)e^x$
\begin{enumerate}
    \item $\frac{ \partial f }{ \partial x } =e^x(x+4)$
    \item $\frac{ \partial f }{ \partial y } =0$
\end{enumerate}
\item $(u,v)\rightarrow  \ln(f(uv)) $\\

\begin{enumerate}
    \item $\frac{ \partial f }{ \partial u } = \frac{vf'(uv)}{f(uv)}$
    \item $\frac{ \partial f }{ \partial v } = \frac{uf'(uv)}{f(uv)}$
\end{enumerate}\pagebreak
\end{enumerate}

\noindent 32.
\begin{enumerate}
    \item $\frac{ \partial g }{ \partial u } = e^v\frac{ \partial f }{ \partial x }(\star,\star)+2uv\frac{ \partial f }{ \partial y }(\star,\star)$
    \item $\frac{ \partial g }{ \partial v } = ue^v\frac{ \partial f }{ \partial x }(\star,\star)+(1+u^2)\frac{ \partial f }{ \partial y }(\star,\star)$
\end{enumerate}
o\`{u} $\frac{ \partial f }{ \partial x }(\star,\star) = \frac{ \partial f }{ \partial x }(ue^v,v(1+u^2))$ et $\frac{ \partial f }{ \partial y }(\star,\star) = \frac{ \partial f }{ \partial y }(ue^v,v(1+u^2))$.

\noindent 34. $h(t)=f(t,g(t^2))$.\\

\begin{enumerate}
    \item $h'(t)=\frac{ \partial f }{ \partial x }(\star,\star)+\frac{ \partial f }{ \partial y }(\star,\star)2tg'(t^2)$
    \item $ \begin{array}{rl} h''(t)=     &   \frac{ \partial^2f }{ \partial x^2 }(\star,\star)+4tg'(t^2)\frac{ \partial^2f }{ \partial x\partial y }(\star,\star)+4t^2(g'(t^2))^2\frac{ \partial^2f }{ \partial y^2 }(\star,\star) \\           
        & +[2g'(t^2)+4t^2g''(t^2)]\frac{ \partial f }{ \partial y }(\star,\star)\end{array}$

\end{enumerate}
où $(\star,\star) = (t,g(t^2))$.




 \section{Intégration}
 \subsection{Série A}
 Exercice 11
 \begin{enumerate}
   \exr $\int \frac{x^3+3x+1}{x} d x = \frac{x^3}3 + 3x + \ln(x)$%
   \exr $\int x^2d x = \frac{x^3}3$%
   \exr $\int 3(x^2+1)^2 d x = \int 3 x^4 + 6 x^2 + 3 d x = \frac 35
   x^5 + 2 x^3 + 3x$%
   \exr $\int (3x^2 - 6x)^3 (x-1) d x = \frac1{12} (3x^2 - 6x)^4$
 \end{enumerate}

 Exercice 12
 \begin{enumerate}
   \exr $\int \sin^2(x^2+1) \cos(x^2+1) x d x = \frac16
   \sin(x^2+1)^3$%
   \exr $\int \tan(x) d x = -\ln\abs{\cos(x)}$%
   \exr $\int \frac{1}{(2+\sqrt{x})\sqrt x} d x= 2 \ln(2+\sqrt{x})$%
   \exr $\int \frac{\ln(x)}{x(1- \ln^2(x)} d x = \frac12
   \ln\abs{1-\ln^2(x)}$%
 \end{enumerate}



   Travaux perso 2 ---------------

   1. Soit deux réels $x$ et $y$ vérifiant $0 < x < y$. On veut montrer
   que pour tout naturel $k \geq 2$, on a
   \[0 < \sqrt[k]{y} - \sqrt[k]{x} < \sqrt[k]{y-x}.\]

   La première inégalité vient de l'inégalité $x < y$ élevée à la
   puissance $\frac1k$.

   On peut ré-écrire la deuxième, sachant que $x > 0$, en divisant par
   $\sqrt[k]{x}$ pour obtenir
   \begin{equation}
    \sqrt[k]{\frac yx} - 1 - \sqrt[k]{\frac yx-1} < 0 \quad \text{ avec }\frac xy > 1
   \end{equation}
     ce qui s'écrit encore $f(t) < 0$ en posant
   $f(t) \pardef \sqrt[k]t - \sqrt[k]{t-1} - 1$. On peut alors étudier
   la fonction $f$. Étant donné que $f(1) = 0$, il suffirait que $f$
   soit strictement décroissante sur $]1;\infty[$ pour qu'on ait
   l'inégalité voulue, à savoir $f(t) < 0$ dès que $t > 1$.

   Pour le montrer, on voit que
   \[f^\prime(t) = \frac 1k \left(t^{\frac{1-k}k} -
     ({t-1})^{\frac{1-k}k}\right)\] d'où on tire les équivalences
   suivantes
   \begin{align}
     & & f^\prime(t) < 0\\
     &\ssi& t^{\frac{1-k}k} < ({t-1})^{\frac{1-k}k}\\
     &\ssi& t^{1-k} < ({t-1})^{1-k}\\
     &\ssi& t > t-1\\
     &\ssi& 0 > -1
   \end{align}
   où la dernière inégalité est manifestement vraie, ce qui prouve la
   première inégalité et achève l'exercice.

   2.


 \paragraph{Exercice 1}
 \begin{enumerate}
 \item Par exemple, $B(x,r)$ avec $x \in \eR^n$ et $r > 0$.

 \item On utilise la densité de $\eQ$ dans $\eR$ pour voir que $B(q,r)$
   ($q \in \eQ^n$ et $r > 0$) est également une base.

   On observe ensuite que seuls les $r$ \og petits\fg{} sont utiles,
   donc on se restreint aux boules de la forme $B(q,1/n)$ ($q \in
   \eQ^n$ et $n \in \eN_0$). Cet ensemble de boules est une base
   dénombrable\marginpar{Pourquoi ?} de la topologie usuelle sur
   $\eR^n$.
 \end{enumerate}

 \paragraph{Exercice 2}
 \emph{Principe.} L'idée est de considérer une propriété topologique
 (invariante par homéomorphisme) et de voir qu'elle est vérifiée par
 les ouverts de $\eR^2$ mais pas ceux du cône.

 \begin{lemma}Si $V$ est un voisinage de $0$ sur le cône $C$, alors
   $V\setminus\{0\}$ n'est pas connexe, donc n'est pas connexe par
   arc.\end{lemma}
 \begin{proof}Le cône $C$ est la réunion de $C^+ = C \cap
   \left(\eR^2\times \eR_0^+\right)$ et $C^- = C \cap \left(\eR^2\times
     \eR_0^-\right)$ car le seul point à cote nulle est la singularité
   $0$. Dès lors, $V$ s'écrit comme l'union disjointe de $V\cap C^+$
   et $V\cap C^-$, qui sont non-vides. Donc $V$ n'est pas
   connexe.\end{proof}

On procède en deux étapes, en montrant d'abord qu'il
   existe des points en \og dessous\fg{} et au \og dessus\fg{} de
   $0$, puis en essayant de les relier.
     Comme $V$ est un voisinage de $0$, il existe un ouvert $U$ du
     cône centré en $0$ inclut à $V$. Donc par définition de la
     topologie induite, et puisque les boules forment une base de la
     topologie de $\eR^3$, il existe une boule $B$ centrée en $0$ dont
     $U$ est la trace sur $C$, telle que $0 \in (B \cap C) \subset
     V$. On choisit $p = (p_x,p_y,p_z) \in (B \cap C)$, et en
     considérant $p^\prime = (p_x, p_y, -p_z)$ on a ainsi trouvé deux
     points qui vérifient $p_z > 0$ et $p^\prime_z < 0$ (au besoin,
     on les échange).

   \begin{enumerate}
   \item Supposons que $V\setminus\{0\}$ soit connexe par arc. Donc
     il existe un chemin
     \[\gamma : [0;1] \to V\setminus\{0\} : t \mapsto
     (\gamma_x(t),\gamma_y(t),\gamma_z(t))\] qui relie $p$ à
     $p^\prime$ et qui vérifie $\gamma_z(0) = p_z > 0$ et
     $\gamma_z(1) = -p_z < 0$. Or $\gamma_z(t)$ est une fonction
     continue (car $\gamma$ est continu), donc par le théorème des
     valeurs intermédiaires, il existe $\bar t$ qui vérifie
     $\gamma_z(\bar t) = 0$. Or le seul point de $C$ dont la cote
     (coordonnée en $z$) soit nulle est le sommet $0$ qui n'est pas
     dans $V\setminus\{0\}$, d'où la contradiction.
   \end{enumerate}

 \begin{rem}Soient deux espaces topologiques $E$ et $F$, et $f :
   E\to F$ un homéomorphisme. Pour toute partie $A$ de $E$,
   l'espace $E\setminus A$ est homéomorphe au sous-espace $F\setminus
   f(A)$ via la restriction $f_{\vert E\setminus A}$.\end{rem}

 \begin{lemma}Soient deux espaces topologiques $E$ et $F$, et $f :
   E\to F$ un homéomorphisme. $E$ est connexe par arc si et
   seulement si $F$ l'est.\end{lemma}
 \begin{proof}On montre en réalité que l'image d'un connexe par arc
   par une application continue est un connexe par arc, ce qui
   implique chaque sens de l'équivalence de l'énoncé.

   Soient $p$ et $q$ des points de $F$. Il existe un chemin reliant
   un antécédent de $p$ et un antécédent de $q$ (dans $E$). L'image
   de ce chemin est un chemin reliant $p$ et $q$ (dans $F$) puisque
   composé d'applications continues.
 \end{proof}

 \begin{lemma}Une sphère de $\eR^n$ est connexe par arc si $n >
   1$\end{lemma}
 \begin{proof}On voit qu'un cercle est connexe par arc car on a une
   paramétrisation en sinus et cosinus. Pour une sphère $S$ de centre
   $a$ en dimension $n > 2$, on se donne $p$ et $q$ sur $S$ et on
   définit $P$ le plan affine passant par $a$, $p$ et $q$. Alors $P
   \cap S$ est un cercle, donc on peut relier $p$ à $q$ par un chemin
   dans cette intersection.

   Pour voir sur une formule que $P \cap S$ est un cercle, on peut
   écrire $x - a = \lambda(a-p) + \mu(a-q)$ l'équation (en $x$) du
   plan $P$, et $\module{x-a}^2 = R^2$ l'équation (en $x$) de la
   sphère. En injectant, on obtient une équation du second degré en
   $\lambda,\mu$ qui se révèle être l'équation d'un cercle à une
   transformation affine près.
 \end{proof}

 \begin{lemma}Un ouvert connexe par arc dans $\eR^n$ ($n \geq 2$) reste
   connexe par arc même si on lui enlève un point.\end{lemma}
 \begin{proof}
   En effet, soit $U$ un tel ouvert connexe par arc, et $p$ un point
   de $U$. Soient $x$ et $y$ sur $U\setminus\{p\}$. Il existe un
   chemin $\gamma$ de $x$ à $y$. Si le chemin ne passe pas par $p$,
   c'est gagné. Si il passe par $p$, on choisit une boule $B$ fermée
   (de rayon non-nul) centrée en $p$ qui ne contient ni $x$ ni
   $y$. On note
   \[E = \gamma^{-1}(B) \subset [0;1]\] c'est un ensemble compact
   (fermé, par continuité de $\gamma$, et borné) dont on regarde le
   maximum $\bar t$ et le minimum $\underline t$.

   Il reste enfin à définir un chemin entre $p$ et $q$ par morceaux
   \begin{enumerate}
   \item Les points $p$ et $\gamma(\underline t)$ sont reliés par
     $\gamma$,
   \item Par connexité par arc, il existe un chemin sur la sphère qui
     relie $\gamma(\underline t)$ à $\gamma(\bar t)$,
   \item et enfin $\gamma(\bar t)$ et $q$ sont reliés via $\gamma$;
   \end{enumerate}
   ce qui achève la construction d'un chemin continu entre $p$ et
   $q$.
 \end{proof}
 Pour conclure l'exercice, par l'absurde, on prend un voisinage
 connexe et ouvert $V$ de $0$ dans le cône, homéomorphe à un ouvert
 connexe $U$ de $\eR^2$. Or $V\setminus\{0\}$ n'est pas connexe par
 arc, alors que l'ouvert dont on retire un point reste connexe par
 arc. C'est impossible, donc l'homéomorphisme n'existe pas, et le
 cône n'est pas une variété de dimension $2$.
