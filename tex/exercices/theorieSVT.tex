% This is part of Exercices de mathématique pour SVT
% Copyright (c) 2010,2014
%   Laurent Claessens et Carlotta Donadello
% See the file fdl-1.3.txt for copying conditions.

%---------------------------------------------------------------------------------------------------------------------------
\subsection{Suites}
%---------------------------------------------------------------------------------------------------------------------------

\begin{proposition}		\label{Propufulimite}
	Soit $(u_n)_{n\in\eN}$, une suite définie par récurrence par
	\begin{equation}
		u_{n+1}=f(u_n)
	\end{equation}
	où $f$ est une fonction suffisamment gentille\footnote{Nous ne rentrons pas dans les détails des hypothèses exactes. Sachez qu'il faut au moins que la fonction soit continue et bien définie.}. Si la suite $(u_n)_{n\in\eN}$ converge vers un nombre réel $u$, alors $u$ est une solution de l'équation  $u=f(u)$.
\end{proposition}
Lorsque $u$ est une solution de $u=f(u)$ on dit que $u$ est un point fixe de $f$. Attention : la proposition (\ref{Propufulimite}) ne garantit pas l'existence de la limite, ni que toute solution de $u=f(u)$ soit une limite. D'ailleurs l'équation $u=f(u)$ peut avoir plusieurs solutions sans que la suite n'ai de limites finis. Cela est le cas lorsque la suite diverge vers $\pm\infty$.

 \begin{example}
	 Soit la suite définie par
	 \begin{equation}
		 \begin{cases}
		 u_{n+1}=u_n^2\\
		u_0=2.
		 \end{cases}
	 \end{equation}
	 Les premiers termes sont $u_0=2$, $u_1=2^2=4$, $u_3=4^2=16$, etc. Cette suite diverge. Pourtant l'équation $u=u^2$ a des solutions : $u=0$ et $u=1$.

	 Si au lieu d'avoir $u_0=2$, on avait eu $u_0=\frac{ 1 }{2}$, alors nous aurions $u_2=\frac{ 1 }{ 4 }$, $u_3=\frac{1}{ 16 }$, etc. Cette suite converge vers $0$, qui est bien solution de $u=f(u)$.
\end{example}

\begin{proposition}		\label{Propsuiteborncv}
Toute suite monotone et bornée converge. En particulier, si une suite est décroissante et bornée vers le bas, alors elle converge. De même, si une suite est  croissante et bornée vers le haut, alors elle converge.
\end{proposition}

%---------------------------------------------------------------------------------------------------------------------------
\subsection{Techniques pour majorer et minorer}
%---------------------------------------------------------------------------------------------------------------------------

Dans de nombreux exercices sur les suites, une difficulté est de majorer ou minorer des expressions contenant $u_n$ sachant que $u_n$ est dans un certain intervalle. La technique la plus puissante pour ce faire demande une utilisation intensive des dérivées; nous n'allons pas parler de cela ici, mais sachez que ça existe.

\begin{example}
	Trouver	des bornes pour la quantité
	\begin{equation}		\label{EqExpuumulnuB}
		a_n=u_n\big( 1-\ln(u_n) \big)
	\end{equation}
	sachant que $1<u_n<e$.

	Trouvons une borne supérieure pour l'expression \eqref{EqExpuumulnuB}, c'est à dire, trouvons $M$ tel que nous soyons certain d'avoir $a_n<M$. Pour ce faire, nous remplaçons tous les $u_n$ par la valeur qui rend l'expression la plus grande possible. Le premier $u_n$ doit être remplacé par $e$. Le $u_n$ qui se trouve dans le logarithme doit par contre être remplacé par $1$ parce qu'il arrive dans terme qui se soustrait; pour minorer, il faut soustraire la quantité la plus petite possible. Nous pouvons donc dire que
	\begin{equation}
		a_n<e(1-\ln(1))=e.
	\end{equation}

	De la même façon, si nous voulons minorer $a_n$, c'est à dire trouver un $m$ tel que $m<a_n$, nous devons remplacer les $u_n$ par les valeurs qui rendent $a_n$ le plus petit possible. Le premier $u_n$ doit être remplacé par $1$, tandis que le second doit être remplacé par $e$ (pour soustraire le plus possible). Nous trouvons
	\begin{equation}
		a_n>1(1-\ln(e))=0.
	\end{equation}
	Nous avons donc
	\begin{equation}
		0<u_n\big( 1-\ln(u_n) \big)<e
	\end{equation}
	dès que $1<u_n<e$.

	Notez que cela ne sont pas les bornes optimales. Il est possible (en travaillant plus) de prouver que, sous les mêmes hypothèses, $u_n\leq 1$.
\end{example}

