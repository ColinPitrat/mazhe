% This is part of Exercices de mathématique pour SVT
% Copyright (c) 2011
%   Laurent Claessens et Carlotta Donadello
% See the file fdl-1.3.txt for copying conditions.

\begin{definition}
  Soit $a>0$ un nombre réel. La fonction exponentielle en base $a$ est la fonction définie par $x\mapsto a^x$.
\end{definition}

Le domaine de $x\mapsto a^x$ est $\eR$. La fonction exponentielle est croissante si $a>1$, décroissante si $0<a<1$, constante si $a=1$. Son image est
\begin{itemize}
\item $]0,+\infty[$ si $a\neq 1$,
    \item $\{1\}$ si $a=1$.
\end{itemize}

La fonction exponentielle satisfait les propriétés suivantes pour tous $x$ et $y$ dans $\eR$ :
\begin{itemize}
\item $a^0=1$ ;
  \item $a^{x+y}=a^xa^y$ ;
    \item $\displaystyle a^{x-y}=\frac{a^x}{a^y}$ ;
      \item $\displaystyle a^{xy}= (a^x)^y$.
\end{itemize}

Si $a\neq 0$ la fonction exponentielle est strictement monotone sur $\eR$ et par conséquence elle admet une fonction réciproque. D' où la définition suivante

\begin{definition}
  Soit $a>0$, $a\neq 1$. La fonction logarithme de base $a$, $\log_{a}$, est définie par la relation $x= a^{\log_{a}x}$.
\end{definition}

La fonction logarithme satisfait les propriétés suivantes pour tous $x$ et $y$ dans $\eR$ :
\begin{itemize}
\item $\log_a 1=0$ ;
  \item $\log_a x+\log_a y=\log_a (xy)$ ;
    \item $\displaystyle \log_a x-\log_a y=\log_a\left(\frac{x}{y}\right)$ ;
      \item $\displaystyle \log_a (x^y)= y\log_a x$.
\end{itemize}

En outre, la formule suivante permet de <<changer de base>> :
\begin{equation}
  \log_a(x)=\frac{\log_b x }{\log_b a}.
\end{equation}
Cela est particulièrement important parce que nous permet d'établir la relation entre le logarithme en base $a$ et le logarithme népérien (de base $e$).

