% This is part of the Exercices et corrigés de mathématique générale.
% Copyright (C) 2009,2015
%   Laurent Claessens
% See the file fdl-1.3.txt for copying conditions.
\paragraph{page 33, 2e quadri, séance 3}

\subparagraph{ex 23}
D'après la page 16bis, une \emph{fonction sinusoïdale} est une fonction de la forme $f(x) = a \sin(\omega x + \varphi)$. Pour avoir $f(0) = 0$, on peut par exemple choisir $\varphi = 0$. Par ailleurs, le maximum de la fonction sinus est $1$, donc la maximum de $f(x)$ est $a$ ; choisissons donc $a = 2$. Reste à déterminer $\omega$ pour qu'un maximum soit atteint en $x = 1$. Or on sait qu'un maximum de la fonction sinus est atteint en $\frac\pi2$ par exemple ; choisissons donc $\omega = \frac\pi2$, de sorte que si $x = 1$, on obtient $2 \sin(\frac\pi2) = 2$ qui est bien le maximum de $f$.

La fonction choisie est donc $f$ définie par $f(x) = 2 \sin(\frac \pi 2 x)$.

\subparagraph{ex 25}
Notons $f$ la fonction définie par
\begin{equation*}
	f(x) = \frac{x2+px+q2}{x}
\end{equation*}
et calculons
\begin{equation*}
	f^\prime(x) = \frac{x2-q2}{x2}
\end{equation*}
de sorte que $f^\prime(x) = 0$ si et seulement si $x = \pm q$ (avec $q \neq 0$, pour que la dérivée ait un sens). On remarque également, par le signe de la dérivée, que l'extrémum obtenu en $-| q |$ est un maximum, l'autre étant un minimum.

Dès lors nous aurons deux cas possibles : si on impose $f(-q) = 0$ et
$f(q) = 4$, on obtient $q = 1$ et $p = 2$, et si on impose $f(-q) = 4$
et $f(q) = 0$, on obtient $q = -1$ et $p = 2$.

Dans les deux cas, les extrémas sont $(-1,0)$ (maximum) et $(1,4)$
(minimum).

\subparagraph{ex 29}
Si le carton fait $a$ de largeur et $b$ de hauteur, alors la surface
imprimable --tenant compte des marges-- est $(a-3)(b-2)$ et est fixée
à $54$. On veut donc minimiser la surface de carton, donnée par $ab$,
sachant que $a = 3 + \frac{54}{b-2}$.

Définissons $f(b) = 3b + \frac{54b}{b-2}$, et trouvons-en le
minimum. Sa dérivée est
\begin{equation*}
  f^\prime(b) = 3 + \frac{54(b-2) - 54b}{(b-2)2} = \frac{3 (b-2)2 - 108}{(b-2)2}
\end{equation*}
et s'annule pour $b = -4$ (à rejeter, n'a pas de sens pour une
longueur) ou $b = 8$. D'après le signe de la dérivée, $b = 8$ fournit
bien un minimum.

En conclusion, $b = 8$ et $a = 12$.

\subparagraph{ex 30} Soient $a$ et $b$ ces nombres. On sait $a, b \geq
0$ et $a+b = 12$, donc $b = 12 - a$.
\begin{enumerate}
\item On veut minimiser $a2+ b2 = a2 + (12-a)2 = f(a)$. La dérivée
  $f^\prime(a) = 2a - 2 (12 - a)$ s'annule pour $a = 6$. La solution
  est donc $a = b = 6$.

\item On veut maximiser $a b2$ (ou $ba2$, mais il suffit d'échanger
  les nombres pour retomber sur le premier cas). On définit $f(a) = a
  (12-a)2$, et la dérivée
  \begin{equation*}
    f^\prime(a) = (12-a)2 - 2 a (12-a)
  \end{equation*}
  s'annule lorsque $a = 12$ (mais alors $b = 0$, à rejeter, ceci n'est
  pas un maximum) ou lorsque $a = 4$ ; les solutions sont donc $(4,8)$
  et le symétrique $(8,4)$.

\item On veut maximiser $ba3$ (même remarque que ci-dessus), donc on
  définit
  \begin{math}
    f(a) = (12-a) a3
  \end{math}
  dont la dérivée est
  \begin{equation*}
    f^\prime(a) = -a3 + 3 (12-a)a2
  \end{equation*}
  et s'annule pour $a = 0$ (pas un maximum) ou $9 = a$ ; donc les
  solutions sont $(9,3)$ et $(3,9)$.
\end{enumerate}

\subparagraph{ex 32}
On cherche $(x,y)$ tel que $y2 = 4ax$ et minimisant la distance
\begin{equation*}
d = \| (x,y)-(2a,a) \| = \sqrt{(x-2a)2 + (y-a)2} =
\sqrt{\left(\frac{y2}{4a}-2a\right)2 + (y-a)2}
\end{equation*}

Remarquons que minimiser $d$ revient à minimiser $d2$, donc posons
\begin{equation*}
f_a(y) = \left(\frac{y2}{4a}-2a\right)2 + (y-a)2
\end{equation*}
et calculons la dérivée
\begin{equation*}
  f_a^\prime(y)  = 2 \left(\frac{y2}{4a}-2a\right) \frac y a + 2
  (y-a) = \frac{y3}{4a2}-2a
\end{equation*}
qui s'annule lorsque $y = 2a$. Donc la solution est $(x,y) = (a,2a)$

\subparagraph{ex 33}
Soit $x$ la distance ``sur le rivage'' par rapport au premier bateau
où sera débarqué le passager ($x$ est entre $0$ et $5$). Alors il
s'agit de minimiser
\begin{equation*}
  d(x) = \sqrt{9+x2} + \sqrt{(5-x)2 + 81}
\end{equation*}
donc on calcule la dérivée
\begin{equation*}
  d^\prime(x) = \frac{x}{\sqrt{9+x2}} - \frac{5-x}{\sqrt{(5-x)2 + 81}}
 = \frac{x\sqrt{(5-x)2 + 81} + (x-5)\sqrt{9+x2}}{\sqrt{9+x2}\sqrt{(5-x)2 + 81}}
\end{equation*}
qui s'annule lorsque $3 |x| = |x-5|$ càd lorsque $3x = 5 - x$
(car $x \in [0,5]$), donc $x = \frac{5}{4}$.

Le trajet minimal du bateau est donc $d(\frac{5}{4}) = 13$.

Une autre de manière de voir le problème est de considérer le principe
de réflexion : le trajet minimal est alors donné par $\sqrt{52 +
  122} = 13$.
