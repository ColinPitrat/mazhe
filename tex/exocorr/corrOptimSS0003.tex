% This is part of Exercices et corrigés de CdI-1
% Copyright (c) 2011, 2019
%   Laurent Claessens
% See the file fdl-1.3.txt for copying conditions.

\begin{corrige}{OptimSS0003}

\begin{enumerate}

\item
Notons tout de suite qu'il n'y a pas d'extrémum globaux, par exemple parce que $\lim_{x\to \pm\infty} (x,0)=\pm\infty$.
La différentielle de $f$ est donnée par
\begin{equation}
	df(x,y)=(3x^2+6x-9,-3y^2+6y),
\end{equation}
qui s'annule en $(-3,0)$, $(-3,2)$, $(1,0)$, $(1,2)$. Ce sont donc ces seuls points qui sont susceptibles d'être des extrémums locaux. La matrice des dérivées secondes est
\begin{equation}
	d^2f(x,y)=\begin{pmatrix}
	6x+6	&	0	\\ 
	0	&	-6y+6	
\end{pmatrix}.
\end{equation}
Les valeurs propres sont données par l'équation
\begin{equation}
	\det\big(d^2f(x,y)-\lambda \mtu\big)=\begin{vmatrix}
	6x+6-\lambda	&	0	\\ 
	0	&	-6y+6-\lambda
\end{vmatrix},
\end{equation}
dont les solutions sont $\lambda_1(x,y)=6x+6$ et $\lambda_2(x,y)=-6y+6$.

Il suffit maintenant de calculer $\lambda_1$ et $\lambda_2$ pour les différents points critiques. Nous avons
\begin{subequations}
\begin{numcases}{}
\lambda_1(-3,0)=-12\\
\lambda_2(-3,0)=6,
\end{numcases}
\end{subequations}
\begin{subequations}
\begin{numcases}{}
\lambda_1(-3,2)=-12\\
\lambda_2(-3,2)=-6,
\end{numcases}
\end{subequations}
\begin{subequations}
\begin{numcases}{}
\lambda_1(1,0)=12\\
\lambda_2(1,0)=6,
\end{numcases}
\end{subequations}
\begin{subequations}
\begin{numcases}{}
\lambda_1(1,2)=12\\
\lambda_2(1,2)=-6,
\end{numcases}
\end{subequations}
Le point $(1,0)$ est donc minimum local parce que $d^2f$ y est définie positive, et le point $(-3,2)$ est maximum local parce que $d^2f$ y est définie négative.


\item
La différentielle est
\begin{equation}
	df=(3x^2+3y;3y^2+3x),
\end{equation}
donc les points critiques sont donnés par le système
\begin{subequations}
\begin{numcases}{}
	x^2+y=0\\
	y^2+x=0,
\end{numcases}
\end{subequations}
dont les solutions réelles sont $(0,0)$ et $(-1,-1)$. La matrice des dérivées secondes est donnée par
\begin{equation}
	d^2f(x,y)=\begin{pmatrix}
	6x	&	3	\\ 
	3	&	6y	
\end{pmatrix},
\end{equation}
dont nous devons chercher les valeurs propres. L'équation $\det(A-\lambda\mtu)=0$ est
\begin{equation}
	(6x-\lambda)(6y-\lambda)-9=0,
\end{equation}
et les solutions sont
\begin{equation}
	\lambda_{\pm}(x,y)=3(x+y)\pm 3\sqrt{ y^2-2xy+y^2+1 }.
\end{equation}
Étant donné que $\lambda_{\pm}(0,0)=\pm 3$, ce point n'est ni un maximum ni un minimum local. L'autre point critique est par contre un maximum local strict parce que $\lambda_+(-1,-1)=-3$ et $\lambda_-(-1,-1)=-9$, ce qui fait que $d^2f(-1,-1)$ est définie négative.


\item
La différentielle vaut
\begin{equation}
	df=(3x^2+1,-6y-2),
\end{equation}
qui ne s'annule jamais. Il n'y a donc aucun extrémum local. Notons aussi que pour tout $(x,y)\in\eR^2$, nous avons, pour tout $\epsilon>0$
\begin{equation}
	\begin{aligned}[]
		f(x+\epsilon,y)&>f(x,y)\\
		f(x-\epsilon,y)&<f(x,y),
	\end{aligned}
\end{equation}
le domaine étant ouvert, il n'y a donc pas d'extrémum global.

\item
Cette fois, le domaine est compact, et il existe certainement un maximum et un minimum global, qu'il faut aller chercher sur les bords.

\end{enumerate}

\end{corrige}
