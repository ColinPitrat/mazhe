% This is part of Exercices et corrigés de CdI-1
% Copyright (c) 2011
%   Laurent Claessens
% See the file fdl-1.3.txt for copying conditions.

\begin{corrige}{0022}

On obtient des candidats limite en appliquant les règles de calculs sur la relation de récurrence $x_k (1+x_{k-1}) = 1$. En effet, en supposant que la suite $(x_k)$ converge vers un réel $x$, alors ce réel doit satisfaire $x(1+x) = 1$ puisque les suites $k\mapsto x_k$ et $k\mapsto x_{k-1}$ convergent toutes les deux vers $x$. Ceci donne les deux \emph{candidats limite} solutions de cette équation :
\begin{equation}
  x_- = \frac{-1 - \sqrt5}{2} \quad\text{et}\quad x_+ = \frac{-1 + \sqrt5}2.
\end{equation}
Par récurrence, on prouve aisément (le faire !) que tous les termes de la suite sont strictement positifs. Cela exclu la possibilité $x_-$ qui est strictement négatif\footnote{Une suite de réels strictement positifs peut avoir une limite nulle, mais pas strictement négative\ldots méditez là dessus.}. Nous savons donc que \emph{si la limite de la suite existe}, alors cette limite est $x_+$.

Vérifions d'abord si la suite ne serait pas bornée vers le haut ou vers le bas par $x_+$. Supposons que $x_k<x_+$. Note que au moins $x_1$ satisfait cette condition, donc nous ne travaillons pas dans le vide. Nous avons alors $1+x_k\geq 1+x_+$, et donc
\begin{equation}
	x_{k+1}=\frac{1}{ 1+x_k }\leq\frac{1}{ 1+x_+ }.
\end{equation}
Mais $x_+$ est justement solution de l'équation $1/(1+x)=x$, donc $1/(1+x_+)=x_+$ et nous avons
\begin{equation}
	x_{k+1}\leq x_+
\end{equation}
dès que $x_k\geq x_+$. Inversement, nous trouvons que
\begin{equation}
	x_{k+1}\geq x_+
\end{equation}
dès que $x_k\geq x_+$. Donc la suite oscille en réalité autour de $x_+$. Les termes impairs seront tous plus petits que $x_+$ et les termes pairs seront tous plus grand que $x_+$. Les suites $(x_{2k})$ et $(x_{2k+1})$ sont donc à regarder séparément. Calculons pour voir si ces suites sont croissantes ou décroissantes :
\begin{equation}		\label{EqNombreOrxkPlusDeux}
	x_{k+2}=\frac{1}{  1+\frac{1}{ 1+x_k }  }=\frac{ x_k+1 }{ x_k+2 },
\end{equation}
et donc
\begin{equation}
	x_{k+2}-x_k=-\frac{ x_k^2+x_{k}-1 }{ x_k+2 }.
\end{equation}
Le dénominateur est toujours positif, tandis que le signe du numérateur dépend de $x_k$, et la valeur de changement de signe n'est autre que $x_+$. Pas mal hein !

Nous avons que 
\begin{equation}
	\begin{aligned}[]
		x_k>x_+\,&\Rightarrow\,x_{k+2}<x_k\\
		x_k<x_+\,&\Rightarrow\,x_{k+2}>x_k
	\end{aligned}
\end{equation}
Donc la suite des $(x_{2k})$ est plus grande que $x_+$ et décroissante, tandis que la suite des $x_{2k+1}$ est plus petite que $x_+$ et décroissante. Or nous savons qu'une suite bornée et monotone est convergente. Donc les deux suites sont convergentes.

D'après l'équation \eqref{EqNombreOrxkPlusDeux}, nous avons $x_{k+2}(x_k+2)=x_k+1$, et donc la seule limite possible vérifie $x(x+2)=x+1$, dont la seule solution acceptable est, encore une fois, $x_+$.

Nous sommes maintenant dans le cas d'une suite $x_k$ donc les deux sous-suites $x_{2k}$ et $x_{2k+1}$ sont convergentes et convergent vers la même limite, c'est à dire le cas de la question \ref{ItemAExo0013} de l'exercice \ref{exo0013} : la suite des $x_k$ converge vers $x_+$.

\end{corrige}
