\begin{corrige}{controlecontinu0008}
 
Une telle fonction existe et est unique. Pour la trouver il faut d'abord calculer les integrales partielles 
\begin{equation}
  \begin{aligned}
    &g(x,y,z)=\int \partial_xg(x,y,z)\, dx = xyz+\sin(xy)+f_1(y,z), \\
    &g(x,y,z)=\int \partial_yg(x,y,z)\, dy = xyz+\sin(xy)+f_2(x,z),\\
    &g(x,y,z)=\int \partial_zg(x,y,z)\, dz=xyz + f_3(x,y).\\
  \end{aligned}
\end{equation}
On obtient alors $g(x,y,z)= xyz+\sin(xy)+f_1(y,z)=xyz+\sin(xy)+f_2(x,z)=xyz + f_3(x,y) $. Cela nous dit que $g(x,y,z)=xyz+\sin(xy)+K $, où $K$ est une constante. Pour trouver la valeur de $K$ nous utilisons la dernière condition. Comme 
\[
g(0,1,2)= K,
\]
c'est évident que $K$ doit être $5$. 
\end{corrige}
