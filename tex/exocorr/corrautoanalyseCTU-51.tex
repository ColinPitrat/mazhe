% This is part of Analyse Starter CTU
% Copyright (c) 2014
%   Laurent Claessens,Carlotta Donadello
% See the file fdl-1.3.txt for copying conditions.

\begin{corrige}{autoanalyseCTU-51}

 
 
   \begin{enumerate}
 \item On calcule d'abord  
\[
\ln(x+1)-\sin(x) =x -\frac{x^2}{2} + \frac{x^3}{3} -\frac{x^4}{4} - x+\frac{x^3}{6} + x^4\alpha(x)= -\frac{x^2}{2}+  x^2\alpha(x),
\]
ce qui nous dit que au voisinage de zéro la fonction $\displaystyle \dfrac{\ln(1+x)-\sin x}{x}$ a le m\^eme comportement que $-x/2$. La valeur de la limite est donc zéro. 
   \item L'expression $(x+1)^{1/x}$ est {\bf par définition dans ce cours} équivalente à $e^{\frac{1}{x}\ln(x+1)}$, ce qui veut dire que son développement limité sera obtenu par la règle de développement d'une fonction composée. 

On calcule d'abord 
\[
\frac{1}{x}\ln(x+1) =\frac{1}{x} \left(x -\frac{x^2}{2} + \frac{x^3}{3} -\frac{x^4}{4} + \ldots\right) = 1 -\frac{x}{2} + \frac{x^2}{3} -\frac{x^3}{4} + \ldots
\]
On a alors que $e^{\frac{1}{x}\ln(x+1)}\approx e^{1 -\frac{x}{2} + \frac{x^2}{3} -\frac{x^3}{4} + \ldots}\approx e^{1 -\frac{x}{2}}$ lorsque $x$ est proche ce $0$. 
\begin{remark}
  Le développement de l'exponentielle autour de $1$ à été calculé dans l'exercice \ref{exoautoanalyseCTU-43}, il donc est possible d'utiliser le résultat pour terminer le calcul de cette limite, mais pour des raisons pédagogiques nous allons continuer le calcul.  
\end{remark}
\begin{equation*}
  e^{1 -\frac{x}{2}} = e +\frac{ex}{2} + \ldots
\end{equation*}
Nous avons alors que 
\[
\lim_{x\to 0}\left(\dfrac{(1+x)^{\frac{1}{x}}-e}{x}\right) = \lim_{x\to 0}\left(\dfrac{\frac{ex}{2}}{x}\right) = \frac{e}{2}.
\]  
   \item Le développement limité de $\sin(3x)$ lorsque $x$ est dans un voisinage de $\pi/3$ est le développement limité au voisinage de $0$ de la fonction $g(x) = \sin\left(3x-\pi\right)$ 
\[
g(x) = g(0) + g'(0) x + \ldots = \sin(-\pi) + 3\cos(-\pi) \left(x-\frac{\pi}{3}\right) + \ldots = -3\left(x-\frac{\pi}{3}\right) + \ldots.
\] 

Le développement de $\cos$ au voisinage de $\pi/3$ a été calculé dans l'exemple \ref{developcosenpisur3}, nous avons donc 
\[
\cos(x) = \frac{1}{2} -\frac{\sqrt{3}}{2}\left(x-\frac{\pi}{3}\right) + \ldots.
\]
La limite a calculer devient alors 
\[
\lim_{x\to \frac{\pi}{3}}\left(\dfrac{\sin (3x)}{1-2\cos (x)}\right) = \lim_{x\to \frac{\pi}{3}}\dfrac{-3\left(x-\frac{\pi}{3}\right)}{\sqrt{3}\left(x-\frac{\pi}{3}\right)} = -\sqrt{3}.
\] 
   \end{enumerate}



\end{corrige}   
