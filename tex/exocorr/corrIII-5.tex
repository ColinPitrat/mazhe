% This is part of the Exercices et corrigés de mathématique générale.
% Copyright (C) 2009
%   Laurent Claessens
% See the file fdl-1.3.txt for copying conditions.
\begin{corrige}{5}

La fonction est $g(x)= e^{-x^2}$, dont le domaine est $\eR$. La dérivée est
\begin{equation}
	g'(x)=-2x e^{-x^2}.
\end{equation}
Étant donné que l'exponentielle est toujours positive, cela a le signe opposé à $x$, donc
\begin{enumerate}

\item
Croissante pour $x<0$,
\item décroissante pour $x>0$,
\end{enumerate}
et maximum pour $x=0$.

La dérivée seconde est
\begin{equation}
	g''(x)=(4x^2-2) e^{-x^2},
\end{equation}
qui s'annule quand $x=\pm\sqrt{\frac{ 1 }{ 2 }}$, ce sont les points d'inflexion. Il y a asymptote horizontale  $y=0$ en $\pm\infty$.

%TODO : refaire le dessin
%Voir la figure \ref{LabelFigFonction}
%\newcommand{\CaptionFigFonction}{La fonction de l'exercice \ref{exo5}.}
%\input{auto/pictures_tex/Fig_Fonction.pstricks}


\end{corrige}
