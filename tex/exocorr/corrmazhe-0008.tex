% This is part of Analyse Starter CTU
% Copyright (c) 2014
%   Laurent Claessens,Carlotta Donadello
% See the file fdl-1.3.txt for copying conditions.

\begin{corrige}{mazhe-0008}

La fonction $f(x) = x^2 - x +1$ ne prends que de valeurs positives quand $x$ est dans l'intervalle $[0,2]$. En effet, la dérivée $f'(x) = 2x -1 $ ne vaut zéro que en $x = 1/2$, est négative sur $[0,1/2[$ et positive sur $]1/2, 2]$. Cela veut dire que $ f(1/2) = 1/4  -1/2 + 1  = 3/4$ est le minimum de $f$ sur $[0,2]$.

La valeur de l'aire à calculer est donc l'intégrale de $f$ sur l'intervalle $[0,2]$
\begin{equation*}
  \int_0^2 x^2 - x +1 \, dx = \left[\frac{1}{3} x ^3 -\frac{1}{2} x^2 + x\right]_0^2 = \frac{8}{3} -\frac{4}{2} + 2 = \frac{8}{3}.
\end{equation*}

\end{corrige}
	
