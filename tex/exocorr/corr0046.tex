% This is part of Exercices et corrigés de CdI-1
% Copyright (c) 2011
%   Laurent Claessens
% See the file fdl-1.3.txt for copying conditions.

\begin{corrige}{0046}

Pour prouver que cette fonction n'est pas continue en $(0,0)$, il suffit de prouver que sa limite n'est pas égale à sa valeur \ldots ou tout simplement que la limite n'existe pas, c'est bon aussi. Si je passe en polaire,
\begin{equation}
	f(r,\theta)=\frac{ r\cos(\theta)\sin(\theta)^2 }{ \cos(\theta)^2+r^2\sin(\theta)^4 }.
\end{equation}
Si nous prenons la limite $r\to 0$ avec $\theta$ constant, cette limite vaut zéro. Mais nous pouvons également prendre une limite sur un chemin plus subtil \ldots un chemin en polaire. Prenons n'importe que $r(t)\to 0$, et choisissons $\theta(t)$ tel que $\cos\big( \theta(t) \big)=r(t)$. Étant donné que $r(t)<1$, un tel $\theta$ existe toujours. Une conséquence est que $\sin\big( \theta(t) \big)\to 1$. Nous avons (nous ne recopions pas toujours toutes les dépendances en $t$) :
\begin{equation}
	\begin{aligned}[]
		\lim_{t\to 0}f\big( r(t),\theta(t) \big)	&=\lim_{t\to 0}\frac{ r(t)\cos\big( \theta(t) \big)\sin\big( \theta(t) \big)^2 }{ \cos\big( \theta(t) \big)^2+r(t)^2\sin\big( \theta(t) \big)^4 }\\
		&=\lim_{t\to 0}\frac{ \cos(\theta)^2\sin(\theta)^2 }{ \cos(\theta)^2+\cos(\theta)^2\sin(\theta)^4 }\\
		&=\lim_{t\to 0}\frac{ \sin(\theta)^2 }{ 1+\sin(\theta)^4 }\\
		&=\frac{ 1 }{2}.
	\end{aligned}
\end{equation}
Donc, la limite en $(0,0)$ de $f$ n'existe pas, ce qui entraîne que la fonction n'est pas continue et qu'elle n'est donc pas différentiable.

Une autre façon, plus simple si on y pense, de déduire la non continuité de $f$ en $(0,0)$ est de regarder la limite le long du chemin $\gamma(t)=(t^2,t)$. Pourquoi ce chemin ? Parce que le dénominateur est égal à $x^2+y^4$. En utilisant ce chemin, nous avons alors les même degrés dans les deux termes du dénominateur. Nous calculons
\begin{equation}
	\lim_{t\to 0}(f\circ\gamma)(t)=\lim_{t\to 0}\frac{ t^2\cdot t^2 }{ t^4+t^4 }=\frac{1}{ 2 }.
\end{equation}

Ce résultat avait déjà été déduit dans l'exemple \ref{Exemple0046Diff} du rappel théorique.



\begin{alternative}
	
Soit $(u,v)$ un vecteur fixé. Calculons la dérivées directionnelles de
la fonction, au point $(0,0)$, dans cette direction :
\begin{equation*}
  \begin{split}
    \frac{\partial f}{\partial(u,v)}(0,0) &= \limite[t\neq0] t 0
    \frac{f(tu, tv)   - f(0,0)}{t}\\
    &=\limite[t\neq0] t 0  \frac{t u t^2 v^2}{t(t^2 u^2 + t^4 v^4)}\\
    &=\limite[t\neq0] t 0  \frac{u v^2}{u^2 + t^2 v^4}\\
    &=
    \begin{cases}
      \frac{uv^2}{u^2} = \frac{v^2}u & \text{si }u \neq 0\\
      0 & \text{sinon}
    \end{cases}
  \end{split}
\end{equation*}
donc, quel que soit le vecteur $(u,v)$, la dérivée directionnelle
existe. Mais la fonction n'est pas continue car :
\begin{equation*}
  \limite[x=y^2\\y\neq0]{(x,y)}{(0,0)} f(x,y) =%
  \limite[x=y^2\\x\neq0]{(x,y)}{(0,0)} \frac{y^2 y^2}{y^4 + y^4} =%
  \limite[x=y^2\\x\neq0]{(x,y)}{(0,0)} \frac 12 = \frac 12
\end{equation*}
et $\frac12 \neq f(0,0) = 0$.


\end{alternative}

\end{corrige}
