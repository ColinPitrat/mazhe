% This is part of Exercices et corrigés de CdI-1
% Copyright (c) 2011,2014
%   Laurent Claessens
% See the file fdl-1.3.txt for copying conditions.

\begin{corrige}{0067}

\begin{enumerate}
\item La série
\begin{equation*}
	\sum_{k=0}^\infty z^k
\end{equation*}
est une série de puissance de coefficients $c_k = 1$ et de centre $z_0 = 0$. Pour trouver le rayon de convergence, nous utilisons la formule \eqref{EqRayCOnvSer}
\begin{equation*}
	\limsup\sqrt[k]{| c_k |}=\limsup\sqrt[k]{1}=1.
\end{equation*}
Le rayon de convergence est donc donné par $R = 1/1 = 1$.

Si $z$ vérifie $| z | = 1$, alors $\module{z^k} = 1$ et
\begin{equation*}
	\limite k \infty z^k \neq 0
\end{equation*}
ce qui prouve que la série ne converge pas sur le bord du disque de
convergence.

Remarquons que cette série est en réalité la série géométrique de raison $z$, dont on connaît la somme explicitement depuis l'exemple \ref{ExZMhWtJS}.

\item La série
\begin{equation*}
	\sum_{k=1}^\infty \frac{{(z+1)}^k}{k}
\end{equation*}
est une série de puissance de coefficients $c_k = \frac{1}{ k }$ et centrée en $z_0 = -1$. Nous calculons
\begin{equation}
	\alpha=\limsup\sqrt[k]{\frac{1}{ k }}=\limsup\frac{1}{ \sqrt[k]{k} }=1,
\end{equation}
donc le rayon de convergence est $R = 1/\alpha = 1$. Le théorème \ref{ThoSerPuissRap} dit tout ce qu'on veut savoir dans tout le plan complexe\footnote{N'hésitez pas à faire un petit dessin de la solution, afin de vous assurer que vous voyez bien qui sont $z$, $z_0$, $R$ et le disque de convergence.}, sauf \emph{sur} le cercle de rayon $1$ centré en $-1$.


Étudions la convergence absolue sur le bord du disque, c'est à dire pour les points $z$ tels que $| z+1 |=1$. Le terme général de la série devient
\begin{equation}
	\frac{ | z+1 |^k }{ k }=\frac{ 1 }{ k },
\end{equation}
qui n'est autre que le terme général de la série harmonique. Donc la série des modules (qui donne la convergence absolue) est la série harmonique. On en déduit que la série ne converge pas absolument sur le bord du disque de convergence.

Pour appliquer Abel, toujours pour $z$ tel que $\module{z+1} = 1$,
majorons les sommes partielles suivantes
\begin{equation*}
\begin{split}
\module{\sum_{k=1}^n {(z+1)}^k} &= \module{-1 + \sum_{k=0}^n
{(z+1)}^k}\\
&= \module{-1 + \frac{1 - {(z+1)}^{n+1}}{1 - (z + 1)}}\\
&\leq 1 + \module{\frac{1 - {(z+1)}^{n+1}}{-z}}\\
&\leq 1 + \frac{1 + \module{{(z+1)}^{n+1}}}{\module z}\\
&= 1 + \frac{1 + {\module{(z+1)}}^{n+1}}{\module z}\\
&= 1 + \frac{2}{\module z}
\end{split}
\end{equation*}
ce qui fournit une majoration (indépendante de $n$) pour tout $z$ vérifiant $z \neq 0$ et $\module{z+1}= 1$. On peut alors appliquer le critère de Abel, puisque la suite $\frac{1}{k}$ est clairement décroissante et tend vers $0$.

Le point $z = 0$ reste à analyser. Si on récrit la série de l'énoncé avec cette valeur de $z$, on retrouve la série harmonique (qui est divergente).

\conclusion
Si $\module{z+1} > 1$ ou si $z = 0$, la série diverge ; si $\module{z+1} = 1$ et $z \neq 0$, la série converge simplement ; si $\module{z+1} < 1$, la série converge absolument (et simplement).

\item 
La série
\begin{equation}\label{eqseriepuissances-exo-c}
	\sum_{k=1}^\infty \frac{k! {(z+i)}^k}{k^k}
\end{equation}
est une série de puissances de coefficients $c_k = \frac{ k! }{ k^k }$ et de centre $z_0 = -i$. La formule
\begin{equation}
	\alpha=\limsup\sqrt[k]{\frac{ k! }{ k^k }}
\end{equation}
ne nous ragoûte pas trop. Nous utilisons donc plutôt la formule alternative \eqref{EqAlphaSerPuissAtern} :
\begin{equation}
	\lim_{k\to \infty} \frac{(k+1)!}{{(k+1)}^{k+1}}\frac{k^k}{k!} = \limite k
	\infty \frac{k^k}{{(k+1)}^k} = \limite k \infty
	\frac{1}{{\left(1+\frac{1}{k}\right)}^k} = \frac 1 {e}
\end{equation}
qui montre que le rayon de convergence est $R = e$.

Pour $z$ tel que $| z+1 | = e$, on observe que le module du terme général est donné par
\begin{equation}
a_k = \frac{k! e^k}{k^k}
\end{equation}
Nous voulons donc vérifier si la série $\sum_k\frac{ k! }{ k^k }e^k$ converge. Pour ce faire, nous calculons
\begin{equation}\label{EqCalculPuiss0067ee}
	\frac{a_{k+1}}{a_k}=\frac{ (k+1)ek^k }{ (k+1)^{k+1} }=e\frac{ k^k }{ (k+1)^k }.
\end{equation}
La suite $\big( k/(k+1) \big)^k$ tend vers $1$, donc la suite des $a_{k+1}/a_k$ tend vers $1$. Nous pouvons cependant dire plus. En vertu de l'exercice \ref{exo0020}\ref{Item0020a}, la suite $x_k=\left( \frac{ k+1 }{ k } \right)^k$ qui définit $e$ est monotone croissante, donc la suite \eqref{EqCalculPuiss0067ee} est une suite monotone décroissante qui tend vers $1$. Chacun de ses termes est donc plus grand que $1$.


Le fait que ce rapport soit plus grand que $1$ montre que $a_{k+1} \geq a_k$, c'est-à-dire que la suite $a_k$ est croissante. En particulier, le terme général de la série (\ref{eqseriepuissances-exo-c}) ne peut pas tendre vers $0$ sur le bord du disque de convergence.

Notez que nous avons bien prouvé que la série ne converge pas sur le bord, et non seulement qu'elle ne converge pas absolument.

\item 
La série
\begin{equation*}
	\sum_{k=0}^\infty \frac{z^k}{k!}
\end{equation*}
est de la forme ci-dessus, avec $c_k = 1/k!$ et $z_0 = 0$. Calculons la limite
\begin{equation*}
\limite k \infty \frac{c_{k+1}}{c_k} = \limite k\infty
\frac{k!}{(k+1)!} = \limite k\infty \frac{1}{k+1} = 0
\end{equation*}
qui montre que la série possède un rayon de convergence infini,
c'est-à-dire qu'elle converge absolument quel que soit $z \in \eC$.

\item La série dépendant des paramètres $a \in \eR$ et $z \in \eC$
\begin{equation*}
\sum_{k=1}^\infty {(k+1)}^a {(z- 5 + i)}^k
\end{equation*}
est une série de puissance (en $z$) de coefficients $c_k =
{(k+1)}^a$ et centrée en $z_0 = 5 - i$. Le calcul de la limite
\begin{equation*}
\limite k\infty \frac{{(k+2)}^a}{{(k+1)}^a} = 1^a = 1
\end{equation*}
montre que le rayon de convergence vaut $R = 1$, indépendamment de
$a$.

Si $\module{z - 5 + i} = 1$, alors la série des modules, qui permet
d'établir la convergence absolue, devient
\begin{equation*}
\sum_{k=1}^\infty (k+1)^a
\end{equation*}
et est manifestement équivalente à la série de Riemann $\sum_{k=1}^\infty k^a$. La série de l'énoncé est donc absolument convergente si et seulement si $a < -1$.

Par ailleurs, le module du terme général étant ${(k+1)}^a$ (toujours pour $z$ sur le bord du disque), il tend vers $0$ si et seulement si $a < 0$. On en déduit que la série diverge dès que $a\geq 0$.

Pour $-1 \leq a < 0$, le « critère de Riemann » a échoué, et nous savons qu'elle ne converge pas absolument. Nous essayons donc d'utiliser le critère d'Abel pour savoir si la série convergerait simplement. Étant donnée la condition sur $a$, le facteur ${(k+1)}^a$ forme une suite décroissant vers $0$. On peut donc essayer de majorer les sommes partielles suivante :
\begin{equation*}
\begin{split}
\module{\sum_{k=1}^n {(z- 5 + i)}^k} &= \module{-1 + \frac{1 -
{(z- 5 + i)}^{n+1}}{1 - (z - 5 + i)}} \\
&\leq 1+ \module{\frac{1 -
{(z- 5 + i)}^{n+1}}{1 - (z - 5 + i)}} \\
&\leq 1 + \frac{1 + \module{{(z - 5 + i)}^{n+1}}}{\module{1 - (z
- 5 + i)}}\\
&= 1 + \frac2{\module{1 - (z - 5 + i)}}
\end{split}
\end{equation*}
ce qui est valable dès que $1 - (z - 5 + i) \neq 0$. On en déduit que la série converge simplement sur le disque $| z-5+i |=1$ dès que $-1\leq a < 0$ et $z \neq 6-i$.

Si $z = 6 - i$, la série initiale devient la série
\begin{equation*}
	\sum_{k=1}^\infty {(k+1)}^a
\end{equation*}
et elle ne converge que si $a < -1$ (qui est un cas déjà étudié) et
diverge sinon.

\conclusion 
	Indépendamment de $a$, le rayon de convergence vaut $1$, et la série converge absolument si $\module{z - 5 + i}<1$ ; diverge si $\module{z - 5 + i}>1$. En ce qui concerne les points $z$ vérifiant $\module{z - 5 + i}=1$, la série y converge absolument si et seulement si $a < -1$ ; converge simplement si $a < 0$ et $z \neq 6-i$ ; diverge si $a \geq 0$ ou si ($0 > a \geq -1$ et $z = 6-i$).

\item
Pour le rayon de convergence, nous calculons la limite
\begin{equation}
	\alpha=\lim_{k\to\infty}\left| \frac{ (-1)^{k+2}\sqrt{k+4} }{ (k+3)^2 }\cdot\frac{ (k+2)^2 }{ (-1)^{k+1}\sqrt{k+3} } \right| =\lim_{k\to \infty}\left( \frac{ k+2 }{ k+3 } \right)^2\sqrt{\frac{ k+4 }{ k+3 }}=1
\end{equation}
Le rayon de convergence est donc $1$ sans discussions.

Sur le bord, $| z+1-i |=1$, la convergence absolue est étudiée par la série
\begin{equation}
	\sum_k\frac{ \sqrt{k+1} }{ (k+2)^2 },
\end{equation}
qui se compare à la série $\sum_{k^{-3/2}}$, qui converge. Il y a donc convergence absolue sur le bord du disque de convergence.

\item
Ce qui est tout à fait \href{http://www.youtube.com/watch?v=5O2bOElnbOQ}{boulversifiant} dans la série
\begin{equation*}
	\sum_{k=1}^\infty \frac{{(-z + 2)}^k}{2k + \ln(k)},
\end{equation*}
c'est le signe devant le $z$. Qu'à cela ne tienne, nous faisons simplement la manipulation suivante pour mettre ce signe en évidence :
\begin{equation}
	(-z+2)^k=\big( -(z-2) \big)^k=(-1)^k(z-2).
\end{equation}
Nous sommes donc avec une série de puissances de coefficients $c_k = \frac{ (-1)^k }{ 2k+\ln(k) }$, et centrée en $z_0 = 2$. Le calcul de la limite
\begin{equation*}
	\limite k \infty \module{\frac{c_{k+1}}{c_k}}=\lim_{k\to\infty}\frac{ 2k+\ln(k) }{ 2(k+1)+\ln(k+1) }=1
\end{equation*}
montre que le rayon de convergence est $R = 1$.

Pour un point $z$ vérifiant $\module{z-2} = 1$, la série des modules (permettant d'analyser la convergence absolue) devient
\begin{equation*}
\sum_{k=1}^\infty \frac{1}{2k + \ln(k)}
\end{equation*}
qui est équivalente à la série harmonique, et donc diverge. Il n'y a donc pas convergence absolue sur le bord du disque. Rien n'est perdu cependant, il est toujours possible que nous ayons une convergence simple.

Pour analyser la convergence simple, toujours sur le bord du disque, utilisons le critère d'Abel : la suite $\frac{1}{2k+\ln(k)}$ est clairement décroissante et tend vers $0$. On écrit :
\begin{equation}
	\begin{aligned}[]
		\left| \sum_{k=1}^n(-z+2)^k \right| &=\left| -1+\sum_{k=0}^n(-z+2)^k \right| \\
			&\leq 1+\left| \sum_{k=1}^n(-z+2)^k \right| \\
			&=1+\frac{ | 1-(-z+2)^{n+1} | }{ |-1+z-2| }\\
			&\leq 1+\frac{ 1+| z-2 |^{n+1} }{ | z-1 | }\\
			&=1+\frac{1}{ | z-1 | }.
	\end{aligned}
\end{equation}
Cela nous fournit une borne pour les sommes partielles, tant que $z\neq 1$. Nous avons donc prouvé la convergence (non absolue) sur le disque de convergence moins le point $z=1$. Ce dernier point reste à étudier.

Pour analyser le point $z = 1$, reprenons l'énoncé initial : on retrouve la série
\begin{equation*}
\sum_{k=1}^\infty \frac{1}{2k + \ln(k)}    
\end{equation*}
qui est équivalente à la série harmonique et donc diverge.

\conclusion La série converge absolument si $z$ vérifie $\module{-z
+2}<1$ ; converge simplement si $\module{-z+2} = 1$ et $z \neq 1$
; diverge si $\module{-z+2}>1$ ou si $z = 1$.
\end{enumerate}


\end{corrige}
