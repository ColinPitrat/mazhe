% This is part of Exercices et corrigés de CdI-1
% Copyright (c) 2011
%   Laurent Claessens
% See the file fdl-1.3.txt for copying conditions.

\begin{exercice}\label{exoVariete0011}

Calculer les intégrales suivantes~:
\begin{enumerate}
\item
  \begin{equation}
    \int_\gamma y^2 d x + x^2 d y
  \end{equation}
  où $\gamma$ est le cercle de rayon $1$ centré en l'origine et
  parcouru dans le sens horlogique.

\item
  \begin{equation}
    \int_\gamma G
  \end{equation}
  où $G$ est le champs de vecteurs qui vaut $(y^2,x^2)$ en $(x,y)$ et
  où $\gamma$ est le cercle de rayon $1$ centré en l'origine et
  parcouru dans le sens trigonométrique.

\item
  \begin{equation}
    \int_\gamma (y-z)d x + (z-y)d y + (x-y) d z
  \end{equation}
  où $\gamma$ est l'arc d'hélice
  \begin{equation*}
    \{ (a \cos(t), a \sin(t), b t) \tq 0 \leq t \leq 2\pi\}
  \end{equation*}
  et où $a$ et $b$ sont des réels positifs.
\item
  \begin{equation}
    \int_\gamma (y+z) d x + (z+x) d y + (x+y) d z
  \end{equation}
  où $\gamma$ est le cercle à l'intersection de la sphère unité de
  $\eR^3$ et du plan d'équation $x+y+z = 0$, parcouru dans le sens
  indiqué par le vecteur $(1,1,-2)$ au point
  $(\frac{\sqrt2}2,-\frac{\sqrt2}2,0)$.
\end{enumerate}


\corrref{Variete0011}
\end{exercice}
