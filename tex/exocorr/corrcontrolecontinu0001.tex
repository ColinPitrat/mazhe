\begin{corrige}{controlecontinu0001}


\begin{enumerate}
    \item
        D'abord remarquons que la fonction \( \sqrt{1-x^2}\) n'est définie que pour \( | x |\leq 1\). Nous devons donc restreindre le domaine de \( x\in\mathopen[ -1 , \frac{ 1 }{2} [\). Afin de trouver les points d'intersection entre \( x^2\) et \( \sqrt{1-x^2}\), nous commençons par résoudre l'équation au carré:
        \begin{equation}
            x^4=1-x^2
        \end{equation}
        en posant \( u=x^2\), c'est à dire l'équation \( u^2+u-1=0\). Nous ne retenons que les solutions avec \( u\geq 0\) et ce que nous trouvons est \( x=\pm x_0\) avec
        \begin{equation}
            x_0=\frac{ \sqrt{\sqrt{5}-1} }{ 2 }\sqrt{2}.
        \end{equation}
        Étant donné que \( \sqrt{5}\) est un peu plus grand que \( 2\), ces solutions sont en valeur absolues plus petites que \( 1\). L'intersection \( x=-x_0\) est donc certainement importante : c'est à partir de là que \( x^2\) devient plus petit que \( \sqrt{1-x^2}\). Étant donné que \( x_0>1/2\), l'autre intersection n'est pas importante.
    
        Un dessin de la situation est à la figure \ref{LabelFigexamssepti}.
        \newcommand{\CaptionFigexamssepti}{Pour l'exercice \ref{exocontrolecontinu0001}}
        \input{auto/pictures_tex/Fig_examssepti.pstricks}

        Le domaine n'est pas ouvert parce que les points sur le bord du dessus (\( y=\sqrt{1-x^2}\)) sont dans le domaine; le domaine n'est pas non plus fermé parce que les points sur le bord droit \( x=1/2\) ne sont pas dans le domaine.

        Le domaine est borné.

        L'intérieur est donné par les équations \( x^2<y<\sqrt{1-x^2}\) avec \( x\in\mathopen] -1 , \frac{ 1 }{2} \mathclose[\). Notez que le point d'intersection à gauche n'est pas dans l'intérieur. L'adhérence est donnée par les équations \( x^2\leq y\leq\sqrt{1-x^2}\) avec \( x\in\mathopen[ -1 , \frac{ 1 }{2} \mathclose]\).

        La frontière est constituée du graphe des deux fonctions \( x^2\) et \( \sqrt{1-x^2}\) entre \( x=-1\) et \( x=\frac{ 1 }{2}\) ainsi que du segment \( x=\frac{ 1 }{2}\) avec \( y\) entre \( \frac{1}{ 4 }\) et \( \frac{ 3 }{ 4 }\).

    \item
        Ceci est un exercice très similaire à l'exercice \ref{exoEspVectoNorme0003}\ref{ItemexoEspVectoNorme0003iv}. Pour chaque \( x\in\eN\) nous avons toute la droite verticale correspondante, voir la figure \ref{LabelFigexamsseptii}.
        \newcommand{\CaptionFigexamsseptii}{Pour l'exerice \ref{exocontrolecontinu0001}.}

        L'ensemble est non borné, il est fermé (et non ouvert). Sa fermeture est lui-même, et son intérieur est vide. Sa frontière est lui-même.

        \input{auto/pictures_tex/Fig_examsseptii.pstricks}

\end{enumerate}


\end{corrige}
