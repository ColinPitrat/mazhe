% This is part of Exercices et corrigés de CdI-1
% Copyright (c) 2011
%   Laurent Claessens
% See the file fdl-1.3.txt for copying conditions.

\begin{corrige}{0014}

\begin{enumerate}
\item $x_n=n$ et $y_n=n^2$, et les inverses pour des suites qui tendent vers zéro.
\item Si $x_n\to\infty$, la suite $x_n^2$ tend plus vite.
\item La suite $x_n=n!$ va plus vite que l'exponentielle. En effet, $e^k$ n'est rien d'autre que $e\cdot e\cdot\ldots\cdot e$. Comparez
\begin{equation}
	e\cdot e\cdot e\cdot\ldots\cdot e
\end{equation}
avec
\begin{equation}
	1\cdot 2\cdot 3\cdot\ldots\cdot 10.
\end{equation}
Étant donné que $e<3$, nous avons
\begin{equation}
	\frac{ e^k }{ k! }<\frac{ e^2 }{ 2 }\cdot\left( \frac{ e }{ 3 } \right)^{k-2}\to 0.
\end{equation}

\end{enumerate}

\end{corrige}
