\begin{corrige}{devoir2-0002}

  \begin{enumerate}
  \item Les dérivées partielles de $f$ sont :
    \[ \partial_x f(x,y)= \cos(y), \qquad \partial_y f (x,y)= -x\sin(y).\]
  \item $\displaystyle \nabla f(x,y)\cdot v= \frac{\cos(y)}{\sqrt{2}}+\frac{x\sin(y)}{\sqrt{2}}.$ La valeur de cette fonction au point $(1,\pi)$  est $-\frac{1}{\sqrt{2}}$.
  \item Comme on a vu à la section 4.4 du poly le plan tangent au graphe de $f$ au point $(1,\pi)$ est le graphe de la fonction $T_{(1,\pi)}$ 
    \[ T_{(1,\pi)}(x,y)= f(1,\pi) + \nabla f(1,\pi)\cdot (x-1, y-\pi)= -1- (x-1)= -x.\]
    Cela veut dire que le plan tangent est l'ensemble $\mathcal{T}\subset\mathbb{R}^3$, 
    \[\mathcal{T}=\{(x,y,z)\in \mathbb{R}^3\, \vert \, z=-x \},\]
    L'équation $x+z=0$ est l'équation du plan $\mathcal{T}$. 
  \end{enumerate}
  Le vecteur normal au plan $\mathcal{T}$ un vecteur de norme 1 qui est orthogonal à tous les vecteurs contenus dans $\mathcal{T}$. Dans cet exemple le vecteur normal est  $\displaystyle \vect{ n }:=\frac{1}{\sqrt{2}}(1,0,1)$, parce que l'équation du plan nous dit que si le point $P:=(x,y,z)$ appartient à $\mathcal{T}$ alors $P\cdot\vect{ n }=0$. 
  En général, le vecteur normal est le vecteur qui a pour composantes les coefficients de $x$, $y$ et $z$ dans l'équation du plan. Pour comprendre comment ça arrive il faut un peu réfléchir sur comment on définit un plan. 

  Soit $P=(p_1, p_2, p_3)$ un point du plan et $\vect{ n }=(n_1, n_2, n_3)$ le vecteur normal au plan. Si $Q= (x,y,z)$ est un autre point du plan alors le vecteur $\bar{PQ}$ est dans le plan. On a alors $(Q-P)\cdot \vect{n}=0$, c'est-à-dire $n_1x+n_2y+n_3z=n_1p_1+n_2p_2+n_3p_3$. Le terme de droite est un nombre réel et il nous donne la distance entre le plan et l'origine. À gauche on voit que les coefficients de $x$, $y$ et $z$ sont bien les composantes de $\vect{n}$.  
  Voici un exercice supplémentaire (facultatif !!) sur les plans.  
    
  \textbf{Exercice :} C'est connu depuis l'antiquité que pour trouver un plan il nous suffit une des combinaisons suivantes :
  \begin{itemize}
  \item trois points du plan ;
  \item un point du plan et une droite contenue dans le plan  qui ne passe pas par le point ;
  \item deux droites parallèles dans le plan ;
  \item deux droites qui ont un point en commun dans le plan.
  \end{itemize}
  On vient de dire que pour trouver un plan il nous faut un point du plan $P$ et un vecteur normal au plan  $\vect{ n }$. Démontrer que chacune des combinaisons proposées est équivalente à la couple <<point du plan $+$ vecteur normal>>. Pour la preuve on peut utiliser la proposition B.2 du poly.    
  %TODO : retirer ce B.2 qui est codé en dur

\end{corrige}
