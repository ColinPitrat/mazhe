% This is part of Exercices et corrigés de CdI-1
% Copyright (c) 2011
%   Laurent Claessens
% See the file fdl-1.3.txt for copying conditions.

\begin{corrige}{0017}

% Les paramètres sont sup, inf, lim sup, lim inf, lim
\newcommand{\ResultSupInf}[5]{%
\begin{equation}
	\begin{aligned}[]
		\sup&=#1,	&	\inf&=#2,	&	\limsup&=#3,	&	\liminf&=#4,	&	\lim&=#5
	\end{aligned}
\end{equation}
				}

Pour les questions où la suite est définie en termes de $k$, nous supposons qu'elle commencent à $k=1$ (et non $k=0$).
\begin{enumerate}
\item Cette suite contient une sous-suite qui tend vers $1$ et une qui tend vers $-1$. Tous les termes sont, par ailleurs, plus petit que $1$ en norme.
	\ResultSupInf{1}{-1}{1}{-1}{NAN}
\item Étant donné que le nombre $1$ est présent dans toutes les queues de suites (et que c'est le plus petit), nous avons que la limite inférieure est $1$.
	\ResultSupInf{NAN}{0}{NAN}{1}{NAN}
\item Cette suite oscille autour de $1$, avec des sauts de plus en plus petits. Le plus grand terme de la suite est celui avec $k=2$, le plus petit est avec $k=1$.
	\ResultSupInf{1+\frac{1}{ 2 }}{0}{1}{1}{1}
\item Étant donné que nous n'avons pas considéré de relations d'ordre sur $\eC$, les concepts d'infimum, supremum et de limites supérieures et inférieures n'ont pas de sens. Le concept de limite, lui, par contre a encore un sens. Comme la norme de $i^k$ est constamment $1$, la limite est zéro.
	\ResultSupInf{NAN}{NAN}{NAN}{NAN}{0}
\item Cette suite n'est rien d'autre qu'une alternance de $\cos(\pi/4)$ et $-\cos(\pi/4)$.
	\ResultSupInf{\cos\frac{ \pi }{ 4 }}{-\cos\frac{ \pi }{ 4 }}{\cos\frac{ \pi }{ 4 }}{-\cos\frac{ \pi }{ 4 }}{NAN}
\item Cette suite étant strictement décroissante, il n'y a pas de limite supérieure et nous avons
	\ResultSupInf{1+25}{1}{NAN}{1}{1}
\item Pour calculer la limite supérieure, remarquons que nous avons toujours
\begin{equation}
	0<\sup\{ x_k\tq k\geq l \}\leq\frac{1}{ l }.
\end{equation}
Pour cette raison, la limite supérieure est nulle. La limite inférieure est également nulle, pour la même raison. Une autre manière de le voir est de voir que
\begin{equation}
	\left|\frac{ \sin(k) }{ k }\right|<\frac{ 1 }{ k },
\end{equation}
 et que donc la limite de la suite est zéro. Par la proposition de la page 44 du cours (point (ii)), la limite est, dans ce cas, égale à la limite supérieure et à la limite inférieure.

\item Il faut remarquer qu'à partir de $k=9$, la valeur de $\lfloor\frac{ 15 }{ 7+k }\rfloor$ est nulle. Pour les grands $k$, la suite n'est donc rien d'autre que $k\mapsto 1/k$ qui tend vers zéro. Cela donne que zéro est la limite inférieure, supérieure et la limite tout court.

L'infimum de la suite est donc zéro et le supremum est atteint en $k=8$ et vaut $64$. Ce terme est le dernier pour lequel la valeur entière vaut $1$. 

\end{enumerate}

\end{corrige}
