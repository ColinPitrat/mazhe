% This is part of Exercices et corrigés de CdI-1
% Copyright (c) 2011, 2019
%   Laurent Claessens
% See the file fdl-1.3.txt for copying conditions.

\begin{corrige}{Variete0002}

	Nous avons $f(x,y,z)=x^2y$ et $G(x,y,z)=x^2+y^2+z^2-1$. Le système à résoudre est
	\begin{equation}
		\left\{
		\begin{array}{ll}
			2xy+2\lambda x=0\\
			x^2+2\lambda y=0\\
			2\lambda z=0\\
			x^2+y^2+z^2-1=0.
		\end{array}
		\right.
	\end{equation}
	Il y a directement deux cas à séparer : $\lambda=0$ et $\lambda\neq 0$. 
	
	Commençons par $\lambda=0$, alors $x=0$ et $y^2+z^2=1$. Cela fait tout un cercle de points candidats. Sur ce cercle, la fonction vaut zéro, et en dehors du cercle, le signe de la fonction est le signe de $y$. Donc les points du cercle avec $y>0$ sont des minimums locaux (parce que dans un voisinage, la fonction est positive), et les points du cercle avec $y<0$ sont des minimums locaux. Les deux points avec $y=0$ ne sont ni l'un ni l'autre.

	Analysons maintenant ce qu'il se passe si $\lambda\neq 0$. Dans ce cas, $z=0$ et le système devient
	\begin{equation}
		\left\{
		\begin{array}{ll}
			2x(y+\lambda)=0\\
			x^2+2\lambda y=0\\
			x^2+y^2-1=0.
		\end{array}
		\right.
	\end{equation}
	Si $x=0$, alors $y=\pm 1$ et $\lambda=0$, ce qui fait rejeter la possibilité $x=0$. Donc $x\neq 0$ et $y=-\lambda$. Il reste les équations
	\begin{equation}
		\left\{
		\begin{array}{ll}
			x^2-2\lambda^2=0\\
			x^2+\lambda^2-1=0.
		\end{array}
		\right.
	\end{equation}
	Il y a quatre solutions : $x=\pm\sqrt{2/3}$ et $y=\pm\sqrt{1/3}$. Ces points sont maximums et minimums locaux.

\end{corrige}
