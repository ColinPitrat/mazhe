\begin{corrige}{devoir2-0003}
  Soient $g_1$ et $g_2$ deux fonctions de $\mathbb{R} \to \mathbb{R}^2$, $g_1(x)=(x,x) $ et $g_2(x)= (x, -x)$. 

On a alors que $f(x, x)= f\circ g_1 (x)$ et $f(x,-x)= f\circ g_2(x)$. On peut calculer les dérivées de ces deux fonctions scalaires  
\[\frac{d (f\circ g_1 )}{dx} (0)=2, \qquad \frac{d (f\circ g_2 )}{dx} (0)=0.\]
D'autre part la définition de dérivée d'une fonction composée nous dit que 
\[\frac{d (f\circ g_1 )}{dx} (0)= \nabla f (0,0)\cdot \frac{d g_1 }{dx} (0),\]
et 
\[\frac{d (f\circ g_2)}{dx} (0)= \nabla f (0,0)\cdot \frac{d g_2 }{dx} (0).\]
On obtient alors le système de deux équations 
\begin{equation}
  \begin{cases}
    \partial_x f (0,0) + \partial_y f(0,0)=2\\
    \partial_x f (0,0) - \partial_y f(0,0)=0.
  \end{cases}
\end{equation}
La seule solutions de ce système est $\partial_x f (0,0) =\partial_y f (0,0)=1 $.

Cet exercice est très difficile si on ne connaît pas la méthode de résolution. Cependant, la méthode n'est pas trop compliquée une fois qu'on a vu la solution et peut s'avérer utile en plusieurs situations. 
\end{corrige}
