% This is part of Exercices et corrections de MAT1151
% Copyright (C) 2010
%   Laurent Claessens
% See the file LICENCE.txt for copying conditions.

\begin{corrige}{SerieCinq0004}

	Le fait que l'application $\alpha$ vérifie $\langle \alpha u, v\rangle =\langle u, \alpha v\rangle $ implique que la matrice de $\alpha$ est une matrice symétrique (pourquoi ?). Elle possède par conséquent une base orthonormale de vecteurs propres. Notons $f_i$ cette base :
	\begin{equation}
		\alpha f_i=\lambda_if_i.
	\end{equation}
	Certains de $\lambda_i$ peuvent être nuls, ce n'est pas un problème.

	La norme de $\alpha$ est donnée par
	\begin{equation}
		\| \alpha \|=\sup_{\| x \|=1}\{ | \alpha(x) | \}.
	\end{equation}
	Si nous décomposons $x$ dans la base choisie des vecteurs propres de $\alpha_i$, nous avons $x=x_if_i$ (somme sous-entendue sur les $i$) avec la contrainte
	\begin{equation}
		\sum_ix_i^2=1.
	\end{equation}
	Nous avons 
	\begin{equation}
		\alpha x=x_i\alpha f_i=x_i\lambda_if_i,
	\end{equation}
	et comme la base des $f_i$ est orthonormale, nous avons
	\begin{equation}
		| \alpha(x) |^2=\sum_ix_i^2\lambda_i^2.
	\end{equation}

	Nous devons donc trouver pour quel $x$ sur la sphère unité la somme $\sum_ix_i^2\lambda_i^2$ est la plus grande. Les coefficients $\lambda_i$ étant fixés, il est intuitivement clair que le maximum est atteint lorsque les composantes de $x$ sont toutes nulles sauf celle qui correspond à la valeur propre la plus grande. Par conséquent, la norme de l'opérateur $\alpha$ serait égal à la valeur absolue de sa plus grande valeur propre.

	Prouvons cela.

	D'abord le supremum est bien un maximum parce que nous faisons un supremum sur une sphère ($\sum x_i^2=1$) et que la sphère est compacte\footnote{Cela n'est pas vrai en dimension infinie.}.

	Pour trouver un extremum sous contrainte, nous utilisons la méthode des multiplicateurs de Lagrange. Dans le cas qui nous occupe, le lagrangien est
	\begin{equation}
		L(x_i,\sigma)=\sum_ix_i^2\lambda_i^2+\sigma(x_1^2+\cdots +x_{n}^2-1).
	\end{equation}
	Le système d'équations à résoudre est donné par $\nabla L=0$ :
	\begin{subequations}
		\begin{numcases}{}
			\frac{ \partial L }{ \partial x_i }=2x_i(\lambda^2+\sigma)\\
			\frac{ \partial L }{ \partial \sigma }=x_1^2+\cdots +x_n^2-1.
		\end{numcases}
	\end{subequations}
	Si il existe au moins deux $\lambda_i$ différents, alors nous ne pouvons fixer $\sigma$ que pour annuler $2\lambda_i-\sigma$ pour un seul des $\lambda_i$. Les $x_i$ de toutes les autres valeurs propres doivent alors être zéro.

	Les extrema sont donc donnés par les vecteurs $x$ dont toutes les composantes sont nulles sauf celles qui correspondent à une valeur propre donnée. Par conséquent, la norme de $\alpha$ est donnée par
	\begin{equation}
		\| \alpha \|=\max\{ | \lambda_1 |,\ldots,| \lambda_n | \},
	\end{equation}
	et est donc bien égal à la plus grande valeur propre.

\end{corrige}
