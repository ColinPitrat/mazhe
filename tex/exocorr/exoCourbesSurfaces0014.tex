\begin{exercice}\label{exoCourbesSurfaces0014}

On considère la courbe paramétrée $(\eR, \vec{\gamma})$  dans $ \eR^3$ 
définie par 
\begin{equation}
\vec{\gamma} (t) = ( e^{2t} \cos t, e^{2t} \sin t, - e^{2t} +1).
\end{equation}

\begin{enumerate}
	\item
Calculer l'abscisse curviligne d'un point $ \vec{\gamma}(t)$ en prenant $ \vec{\gamma}(0)$ comme origine.

\item
 En un point $M$ d'abscisse curviligne $s$, déterminer les vecteurs $\vec{\tau}$, $\vec{n}$ et $\vec{\beta}$ du repère de Frenet, la courbure $c(s)$ et la torsion $ \theta (s)$. 
		
\end{enumerate}


\corrref{CourbesSurfaces0014}
\end{exercice}
