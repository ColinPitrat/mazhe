% This is part of the Exercices et corrigés de CdI-2.
% Copyright (C) 2008, 2009,2015, 2019
%   Laurent Claessens
% See the file fdl-1.3.txt for copying conditions.


\begin{corrige}{_I-1-11}

Afin de faire le coup du compact, nous étudions la convergence uniforme de la série sur tout compact de $]1,\infty[$. Soit $\epsilon>0$, et regardons ce qu'il se passe sur un compact dont le minimum\footnote{Pour rappel, un compact dans $\eR$ a \emph{toujours} un minimum.} est $1+\epsilon$. Dans ce cas, $n^x\geq n^{1+\epsilon}$, et donc $f_n(x)=\frac{ 1 }{ n^x }\leq \frac{1}{ n^{1+\epsilon} }$. Étant donné que la série numérique $\sum_{n=1}^{\infty}\frac{1}{ n^{1+\epsilon} }$ converge, la fonction de Riemann converge uniformément par le critère de Weierstrass (théorème \ref{ThoCritWeierstrass}). Nous avons donc convergence uniforme de la série sur tout compact de $]1,\infty[$, ce qui fait que $\zeta$ est une fonction continue pour $x>1$ par le théorème \ref{ThoSerUnifCont}.

Nous allons à présent utiliser le théorème \ref{ThoSerUnifDerr} pour prouver que la fonction de Riemann est $C^1$. Il faut donc prouver que la série des dérivées $(n^{-x})'=-\ln(n)n^{-x}$ converge uniformément sur tout compact de $]1,\infty[$.

Nous prenons encore une fois un compact $K$ dont le minimum est $1+\epsilon$. D'abord, nous majorons le logarithme par un $x^{\alpha}$ : lorsque $n$ est assez grand, nous avons
\begin{equation}
	\ln(n)n^{-x}\leq n^{\alpha}n^{-x};
\end{equation}
la proposition \ref{PROPooKVIFooGdKpfP}\ref{ITEMooCDSQooSIctbz} nous dit que pour tout $\alpha>0$, il existe un $n$ à partir duquel cette inégalité est valide. Étant donné que $1+\epsilon$ est le minimum du compact, nous pouvons encore majorer en remplaçant $x$ par $1+\epsilon$ :
\begin{equation}
	\ln(n)n^{-x}\leq \frac{1}{ n^{1+\epsilon-\alpha} }.
\end{equation}
Afin de pouvoir utiliser le critère de Weierstrass, nous devons nous assurer que la série $\sum_{n=1}^{\infty} \frac{ 1 }{ n^{1+\epsilon-\alpha} } $ converge. Cela n'est vrai que si $1+\epsilon-\alpha > 1$, mais le choix de $\alpha$ étant encore arbitraire, nous choisissons $0<\alpha<\epsilon$.

Ainsi, la série des dérivées converge uniformément sur tout compact et nous en déduisons que cette série est bien la dérivée de la fonction de Riemann qui est $C^1$.

Afin de traiter les dérivées d'ordre supérieur, il faut calculer 
\begin{equation}
	(n^{-x})^{p/}=(-1)^p\big( \ln(n) \big)^p n^{-x},
\end{equation}
et remarquer que $\lim_{x\to\infty} x^{\alpha}/(\ln(x))^p=\infty$. Par conséquent y a encore moyen de remplacer le logarithme par un $x^{\alpha}$. Le reste de la preuve est la même.

Ici se termine la correction de cet exercice. Nous restons cependant sur notre faim en ce qui concerne la convergence uniforme de la série sur l'ouvert $]1,\infty[$. En effet, nous avons prouvé la convergence uniforme sur tout compact (et cela nous a suffit pour résoudre l'exercice), mais nous n'avons pas prouvé que la série n'était pas uniformément convergente sur $]1,\infty[$ pour autant.

Nous allons montrer qu'il n'y a pas uniforme convergence en prouvant que si $x$ est assez proche de $1$, alors la suite des sommes partielles de $\zeta(x)$ est aussi proche que l'on veut de la suite des sommes partielles de $\sum_{n=1}^{\infty}\frac{1}{ n }$ qui, elle, diverge. 

\begin{lemma}
Nous avons
\begin{equation}
	\lim_{x\to 1}\zeta(x)=\infty
\end{equation}
où la limite est une limite à droite : la limite à gauche n'existe pas.
\end{lemma}

\begin{proof}
Soit $M>0$. Prouvons que $\exists\epsilon$ tel que $\zeta(1+\epsilon)\geq M$. D'abord, choisissons un $k$ tel que 
\begin{equation}
	\sum_{n=1}^k\frac{1}{ n }>M,
\end{equation}
et choisissons un $\epsilon$ tel que
\begin{equation}
	\max_{n\in\{ 1,\ldots,k \}}\left|  \frac{1}{ n }-\frac{1}{ n^{1+\epsilon} }\right|<\alpha.
\end{equation}
Un tel choix de $\epsilon$ est possible pour tout $\alpha$. Maintenant, nous choisissons $\alpha$ de façon à avoir $k\alpha<\sigma$. Avec ça, nous avons
\begin{equation}
	\sum_{n=1}^k\frac{1}{ n }-\sum_{n=1}^k\frac{1}{ n^{1+\epsilon} }=\sum_{n=1}^k\left( \frac{1}{ n }-\frac{1}{ n^{1+\epsilon} } \right)<k\alpha<\sigma.
\end{equation}
En prenant $\sigma$ tel que $M-\sum_{n=1}^k(1/n)<\sigma$, nous trouvons ainsi un $\epsilon$ tel que $\sum_{n=1}^k\frac{1}{ n^{1+\epsilon} }>M$. Cela prouve le lemme.
\end{proof}

Armé de ce lemme, il est maintenant aisé de prouver que la série définissant la fonction de Riemann n'est pas uniformément convergente sur $]1,\infty[$. Prenons la $k$ième somme partielle $s_k(x)=\sum_{n=1}^k(1/n^x)$. Pour chaque $k$, cela est une fonction bornée de $x$ (y compris en $x=1$), donc $\sup_{x\in]1,\infty[}s_k(x)=M_k$. Armé de cette majoration, nous faisons
\begin{equation}
	\| s_k-\zeta \|_{\infty}=\sup_{x\in]1,\infty[}\big( \zeta(x)-s_k(x) \big)>\sup\big( \zeta(x)-M_k \big)=\infty,
\end{equation}
il n'y a donc pas moyen que la limite de $\| \zeta-s_k \|_{\infty}$ quand $k\to\infty$ soit nulle. Il n'y a donc pas uniforme convergence de $\zeta$ sur l'intervalle $]1,\infty[$.


\end{corrige}
