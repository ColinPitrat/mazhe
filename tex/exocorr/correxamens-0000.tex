% This is part of Exercices et corrections de MAT1151
% Copyright (C) 2010
%   Laurent Claessens
% See the file LICENCE.txt for copying conditions.

\begin{corrige}{examens-0000}

Pour être un problème stable, il faut d'abord avoir une solution unique. Explicitons le problème composite :
\begin{equation}
	F(x,d)=x^2-x+\frac{1}{ 4 }\sin^2(d).
\end{equation}
Les solutions de $F(x,d)=0$ sont données par
\begin{equation}
	x(d)=\frac{ 1\pm\sqrt{1-\sin^2(d)} }{2}.
\end{equation}
Attention à ne pas oublier la valeur absolue :
\begin{equation}
	\sqrt{1-\sin^2(d)}=\sqrt{\cos^2(d)}=| \cos(d) |\neq\cos(d),
\end{equation}
donc
\begin{equation}
	x(d)=\frac{ 1\pm| \cos(d) | }{2}
\end{equation}
Étant donné que nous cherchons uniquement les solutions $x\geq\frac{ 1 }{2}$, nous ne gardons que la solution avec un plus. Au final l'unique solution au problème composite est
\begin{equation}
	x(d)=\frac{ 1+| \cos(d) | }{2}.
\end{equation}
À cause de la valeur absolue, cette fonction n'est pas $C^1$, mais sa dérivée reste bornée, donc elle vérifie la seconde condition de la stabilité. Vérifiez que l'exercice \ref{exoSerieUn0001} tient encore si nous remplaçons «$C^1$» par «dérivée bornée».

À cause de la formule de dérivation des fonctions composées, le conditionnement relatif du problème composite est le produit des conditionnements relatifs, voir exercice \ref{exoSerieUn0005}. Plus précisément :
\begin{equation}
	K_{f_1\circ f_2}(d)=K_1\big( f_2(d) \big)K_2(d),
\end{equation}
et non $K_1(d)K_2(d)$. 

Calculons les deux conditionnements relatifs.
\begin{equation}
	\begin{aligned}[]
		K_{\text{abs}}^{(1)}(d)&=\frac{ 1 }{2}| \sin(d)\cos(d) |\\
		K_{\text{rel}}^{(1)}(d)&=2\frac{ d\cos(d) }{ \sin(d) }.
	\end{aligned}
\end{equation}
et, étant donnée que
\begin{equation}		\label{EqComposeK0exam}
	x_2(d)=\frac{ 1+\sqrt{1-4d^2} }{2},
\end{equation}
nous avons
\begin{equation}
	\begin{aligned}[]
		K_{\text{abs}}^{(2)}(d)&=\frac{ 2d }{ \sqrt{1-4d^2} }\\
		K_{\text{rel}}^{(2)}(d)&=\frac{ 4d }{ \sqrt{1-4d^2}\big( 1+\sqrt{1-4d^2} \big) }.
	\end{aligned}
\end{equation}
En utilisant la formule \eqref{EqComposeK0exam}, nous avons
\begin{equation}
	K_{\text{rel}}(d)=\frac{ 2d\sin(d) }{ 1+| \cos(d) | },
\end{equation}
et le problème est bien conditionné lorsque $d$ est petit ou bien lorsque $\sin(d)$ est petit.

Pour trouver le conditionnement relatif du problème composite, nous pouvions aussi directement partir de la solution
\begin{equation}
	x(d)=\frac{ 1+| \cos(d) | }{2}.
\end{equation}
Pour la dérivée, attention que $| \cos(x) |'\neq| \sin(x) |$, mais grâce au fait que la formule du conditionnement absolue elle-même aie des valeurs absolues, nous avons quand même
\begin{equation}
	K_{\text{abs}}(d)=\frac{ | \sin(d) | }{2},
\end{equation}
 et donc
 \begin{equation}
	 K_{\text{rel}}(d)=\frac{ | \sin(d) | }{2}\cdot\frac{ d }{ \frac{ 1+| \cos(d) | }{2} }=\frac{ d\sin(d) }{ 1+| \cos(d) | }.
 \end{equation}

\end{corrige}
