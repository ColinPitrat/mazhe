% This is part of Exercices et corrigés de CdI-1
% Copyright (c) 2011,2014
%   Laurent Claessens
% See the file fdl-1.3.txt for copying conditions.

\begin{corrige}{0038}

Remarquons d'abord que si $x$ est irrationnel ou nul, et que $(q_i)$ est une
suite de rationnels qui tend vers $x$, alors $f_i \pardef f(q_i)$ tend vers
$0$.

En effet, par l'absurde, supposons qu'il existe $N \in \eN_0$ et une
sous-suite $(f_{i_k})$ telle que pour tout $k$, $f_{i_k} > \frac 1
N$. Mais alors $N! q_{i_k}$ est toujours un entier (car le
dénominateur de $q_{i_k}$ est inférieur à $N$, donc divise $N!$) et
\begin{equation*}
  \limite k \infty N! q_{i_k} = N! x \Rightarrow x = \frac 1{N!} \limite k
  \infty N! q_{i_k}
\end{equation*}
ce qui montre que $x$ est rationnel, car une suite d'entiers
convergente est toujours constante à partir d'un certain rang (et
possède une limite entière). Ceci est une contradiction avec « $x$
irrationnel ». Par ailleurs, si $x = 0$, alors $q_{i_k} = 0$ à
partir d'un certain $k$, mais alors $f_{i_k} = f(q_{i_k}) = 0$ est une
contradiction avec $f_{i_k} > \frac 1 N$. D'où la thèse.

Il est maintenant clair que si $a$ est irrationnel ou nul, la limite
\begin{equation*}
  \limite x a f(x) =
  \begin{arrowcases}
    \limite[x \in \eQ] x a f(x) = 0\\
    \limite[x \in \eR\setminus\eQ] x a f(x) = \limite[x \in
    \eR\setminus\eQ] x a 0 = 0\\
  \end{arrowcases}
\end{equation*}
est nulle, donc $f$ est continue en $a$. Par ailleurs, si $a$ est
rationnel
\begin{equation*}
  \limite x a f(x)\;\text{n'existe pas car} \quad \limite[x\in \eR\setminus\eQ] x a f(x) = 0 \neq f(a)
\end{equation*}
donc $f$ n'est pas continue en $a$.

\end{corrige}
