\begin{exercice}\label{exoCourbesSurfaces0011}

Soit $(I, \vec{\gamma})$ une courbe $ C^k$ dans $ \eR^N$, où $ k \geq 1$.  On suppose que $ \vec{\gamma}'(t) \neq 0$ pour tout $ t \in I$.  Soit $ t_0 \in I$ fixé et soit $ a_{t_0}(t) $ l'abscisse curviligne du point $\vec{\gamma}(t)$ par rapport à $\vec{\gamma}(t_0)$.

\begin{enumerate}
	\item
 Montrer que $a_{t_0} $ est une fonction $C^k$ sur $I$. 

\item

 Montrer que $\displaystyle \lim_{h \to 0} \frac{||\vec{\gamma}(t_0 + h) - \vec{\gamma}(t_0)||}{a_{t_0}(h) } = 1$
(en d'autres termes, pour $h$ assez petit la distance dans $\eR^N$ 
entre $ \vec{\gamma}(t_0 + h) $ et $ \vec{\gamma}(t_0)$ 
peut être approchée par la longueur de l'arc de courbe compris entre ces deux points).

\item
 Pour $ t_0, \; t_1 \in I$, on note $ k =a_{t_0}(t_1) $.  Montrer que pour tout $ t \in I$ on a 
 \begin{equation}
	a_{t_1}(t) = a_{t_0}(t) - k.
 \end{equation}
		
\end{enumerate}


\corrref{CourbesSurfaces0011}
\end{exercice}
