% This is part of Exercices et corrigés de CdI-1
% Copyright (c) 2011
%   Laurent Claessens
% See the file fdl-1.3.txt for copying conditions.

\begin{corrige}{0002}

Pour le \ref{itemexo1g}, la suite est décroissante, donc le maximum est atteint (et est le premier terme), et l'infimum sera la limite si elle existe. Cette limite sera le minimum si elle est atteinte. Ici, la limite est zéro, et n'est évidement pas atteinte.

Pour le \ref{itemexo1h}, la suite est alternée. Il faut donc bien se garder de prendre le plus petit en norme comme étant le minimum. La limite des termes positifs est $1$, et celle des termes négatifs est $-1$. Ce sont les candidats supremum et infimum. Étant donné que ces deux sous-suites sont croissantes en norme, ils sont effectivement supremum et infimum. Comme ils ne sont pas atteints, il n'y a pas de maximum, ni de minimum. En effet, quand il y a un supremum, un maximum ne peut que lui être égal.

La fonction présente dans l'exercice \ref{itemexo1k} est paire et toujours positive ou nulle. Le minimum (et donc infimum) est donc zéro : il est atteint et aucune valeurs négatives n'est atteinte. La fonction $x\mapsto x^2$ est une fonction croissante de $| x |$, donc il faut chercher le supremum sur le point du domaine le plus éloigné de zéro, c'est à dire les points arbitrairement proches de $-1$. Prouvons que $1$ est le supremum de l'ensemble considéré. En effet, pour tout $\epsilon$, le nombre $1+\epsilon$ n'est pas dans l'ensemble parce qu'il serait l'image de $\pm\sqrt{1+\epsilon}$ qui n'est pas dans $]-1,\frac{1}{ 2 }[$.

\end{corrige}
