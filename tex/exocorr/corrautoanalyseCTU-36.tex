% This is part of Analyse Starter CTU
% Copyright (c) 2014
%   Laurent Claessens,Carlotta Donadello
% See the file fdl-1.3.txt for copying conditions.

\begin{corrige}{autoanalyseCTU-36}

\begin{enumerate}
\item \begin{enumerate}
\item  Le polyn\^ome caractéristique de l'équation $(H_{1})$ est  $r^2-5r+6$, dont les racines sont $2$ et $3$. Nous sommes donc dans le cas où les deux racines sont réelles et distinctes. La solution générale de cette équation est donc, en appliquant la formule \eqref{sol_gen_reelle_ordre_deux_hom}
  \begin{equation*}
  \mathcal{Y}_h  = \left\{C_1 e^{2x} +C_2 e^{3x} \,:\, C_1,\, C_2 \in \eR\right\}.
  \end{equation*}
\item L'ensemble des deux conditions nous permet d'écrire un système de des équations pour déterminer les deux inconnues $C_1$ et $C_2$. 
  \begin{equation*}
    \begin{cases}
      C_1+ C_2 =-2 & \quad\text{ qui correspond à la condition }\varphi_{1}(0)=-2,\\
      2C_1 + 3C_2 = -2 & \quad\text{ qui correspond à la condition }\varphi_{1}'(0)=-2.
    \end{cases}
  \end{equation*}
  On a donc $C_1 = -4$ et $C_2= 2$.

 La solution particulière $\varphi_{1}$ est donc $\varphi_{1}(x)= -4e^{2x} +2 e^{3x}$.
\end{enumerate}
\item \begin{enumerate}
\item   Le polyn\^ome caractéristique de l'équation $(H_{2})$ est  $r^2+2r+2$, qui admet deux racines complexes conjuguées : $-1 +i $ et $-1 -i $. La solution générale réelle de l'équation est déterminée à partir de la formule \eqref{sol_gen_reelle_ordre_deux_hom_complconj}
  \begin{equation}\label{exoautoanal36part2}
    \mathcal{Y}_h  = \left\{ e^{-x}\left(C_1\cos(x) +C_2\sin(x)\right)  \,:\, C_1,\, C_2 \in \eR \right\}.
  \end{equation} 
\item Comme on a fait pour l'équation $(H_{1})$, il faut utiliser les deux conditions pour écrire un système de deux équations dans les inconnues $C_1$ et $C_2$.
  \begin{equation*}
    \begin{cases}
      C_1=-2 & \quad\text{ qui correspond à la condition }\varphi_{2}(0)=-2,\\
      -C_1 + C_2 = -2 & \quad\text{ qui correspond à la condition }\varphi_{2}'(0)=-2.
    \end{cases}
  \end{equation*}
  On a donc $C_1 = -2$ et $C_2= -4$.  

 La solution particulière $\varphi_{2}$ est donc $\varphi_{2}(x)= -2e^{-x}\left(\cos(x) +\sin(x)\right)$.
\end{enumerate}
\item \begin{enumerate}
\item  Le polyn\^ome caractéristique de l'équation $(H_{3})$ est  $r^2-4r+4$, qui a une racine double $r=2$. La solution générale réelle de l'équation est déterminée à partir de la formule \eqref{sol_gen_reelle_ordre_deux_hom_doublerac}
  \begin{equation*}
    \mathcal{Y}_h  = \left\{(C_1  +C_2x) e^{2 x} \,:\, C_1,\, C_2 \in \eR, \: x\in I\right\}.
  \end{equation*} 
\item En travaillant comme dans les deux cas précédents nous obtenons  $C_1 = -2$ et $C_2= 2$.

 La solution particulière $\varphi_{3}$ est donc $\varphi_{3}(x)=2(-1+ x) e^{2 x} $.
\end{enumerate}

\end{enumerate}

\end{corrige}   
