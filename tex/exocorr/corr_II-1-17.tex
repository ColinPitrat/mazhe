% This is part of the Exercices et corrigés de CdI-2.
% Copyright (C) 2008, 2009
%   Laurent Claessens
% See the file fdl-1.3.txt for copying conditions.


\begin{corrige}{_II-1-17}

\begin{enumerate}

\item
Si nous récrivons l'équation sous la forme
\begin{equation}
	y'=\frac{ l\ln(t)+y-t\ln(y) }{ t },
\end{equation}
nous voyons que l'équation est homogène. Si nous posons $v=yt$, nous trouvons
\begin{equation}
	\ln(t)=\int\frac{ dv }{ \ln(v) },
\end{equation}
qui est résolue à quadrature près.

\item
C'est une équation de Bernoulli, sous la forme
\begin{equation}
	3y'-ay=(1+t)y^{-2}.
\end{equation}
En suivant les notations de l'équation \eqref{EqBerNDiffalp}, 
\begin{equation}
	\begin{aligned}[]
		a(t)&=a\\
		b(t)&=\frac{ 1+t }{ 3 }\\
		\alpha&=-2.
	\end{aligned}
\end{equation}
Nous posons $z=y^3$, et nous avons l'équation différentielle
\begin{equation}
	z'-3au=\frac{ 1+t }{ 3 },
\end{equation}
qui est linéaire. La solution pour $y$ est donnée par
\begin{equation}
	y^3=C e^{at}-\frac{1}{ a^2 }(at+a+1).
\end{equation}

\item
Après avoir utilisé une règle de Leibnitz pour dériver $y\cos(t)$, nous mettons l'équation sous la forme
\begin{equation}
	y'=\frac{ ky^3 }{ \cos(t) }+\frac{ y\sin(t) }{ \cos(t) },
\end{equation}
qui est une équation de Bernoulli avec
\begin{equation}
	\begin{aligned}[]
		a(t)&=\tan(t)\\
		b(t)&=\frac{ k }{ \cos(t) }\\	
		\alpha&=3.
	\end{aligned}
\end{equation}
La solution est
\begin{equation}
	z=\frac{1}{ y^2 }=C\cos(t)^2-\frac{ k }{2}\cos(t)^2\ln\left| \frac{ 1+\sin(t) }{ 1-\sin(t) } \right| -k\sin(t).
\end{equation}

\item
$2tyy'-y^2-t^2=0$. En posant $z=y^2$, on trouve $z'=2yy'$, et donc
\begin{equation}
	tz'-z=t^2,
\end{equation}
qui est une équation linéaire. Cette équation peut soit se résoudre comme une équation linéaire, soit en écrivant
\begin{equation}
	\frac{ tz'-z }{ t^2 }=1,
\end{equation}
et en remarquant que le membre de gauche n'est autre que $(z/t)'$. Donc $\frac{ z }{ t }=t+C$, et
\begin{equation}
	z=y^2=t^2+Ct.
\end{equation}

\item
$y'=ty(t+y)-t(1+t)$. C'est une équation de Ricatti avec
\begin{equation}
	\begin{aligned}[]
		a(t)&=t\\
		b(t)&=t^2\\
		c(t)&=-t(1+t),
	\end{aligned}
\end{equation}
dont on devine la solution particulière $y(t)=1$. En suivant le schéma général, nous posons $y(t)=1+\frac{1}{ u(t) }$, et nous trouvons l'équation linéaire
\begin{equation}
	u'=-(2t+t^2)u-t
\end{equation}
pour $u$. L'équation homogène a pour solution
\begin{equation}
	u_H=K\exp\big( -t^2+\frac{ t^3 }{ 3 } \big).
\end{equation}
La méthode de variation des constantes donne $K$ sous la forme
\begin{equation}
	K=-\int \exp\big( t^2+\frac{ t^3 }{ 3 } \big).
\end{equation}

\item
$t^2y'+y^2-yt-t^2=0$.
Cette équation est Ricatta (avec $y=t$ comme solution particulière), mais elle est également une équation homogène. En posant $y=tv$, nous trouvons l'équation
\begin{equation}
	tv'+v^2=1,
\end{equation}
qui est une équation à variables séparées.

\item
$(7y^2-4t-3ty-3t^2)y'=y(2+3t+y)$. C'est une équation presque exacte, dont $M=y$ est un facteur intégrant. Solution :
\begin{equation}
	y^2\big( \frac{ 7 }{ 4 }y^2-2t-ty-\frac{ 3 }{ 2 }t^2 \big)=C.
\end{equation}

\item
$y''=y'(2+y'\tan(y))$. La variable $t$ n'entre n'intervient pas explicitement dans l'équation, donc nous posons $z\big( y(t) \big)=y'(t)$. Par calcul de dérivation,
\begin{equation}
	y''(t)=\frac{ dz }{ dy }\big( y(t) \big)\cdot \underbrace{y'(t)}_{=z},
\end{equation}
et l'équation de départ s'écrit
\begin{equation}
	\frac{ dz }{ dy }\big( y(t) \big)\cdot z\big( y(t) \big)=z\big( y(t) \big)\cdot\Big( 2+z\big( y(t) \big)+\tan\big( y(t) \big) \Big).
\end{equation}
L'équation pour $z(u)$ est donc
\begin{equation}
	z'z=z\big( 2+z\tan(u) \big).
\end{equation}
Une simplification par $z$ (ce qui revient à perdre les solutions constantes pour $y$) mène à l'équation $z'=2+z\tan(u)$, dont les solution sont données par
\begin{equation}
	z=2\tan(y)+\frac{ 2K }{ \cos(y) },
\end{equation}
qu'il faut maintenant identifier $y'$. L'équation que nous trouvons alors pour $y$ est
\begin{equation}
	y'=\frac{ 2\big( \sin(y)+K \big) }{ \cos(y) },
\end{equation}
que nous pouvons récrire sous la forme
\begin{equation}
	\big( \sin(y)+K \big)'=2\big( \sin(y)+K \big),
\end{equation}
et dont les solutions sont donc données par
\begin{equation}
	\sin(y)+K=L e^{2t}.
\end{equation}

\item
$y'=-ty/(t^2+y^2)$. C'est une équation homogène, donc on pose $v(t)=y(t)/t$ et on trouve l'équation
\begin{equation}
	tv'+v=-\frac{ v }{ 1+v^2 }
\end{equation}
que l'on peut remettre sous la forme
\begin{equation}
	\frac{1}{ t }\frac{ dt }{ dv }=\frac{ 1+v^2 }{ v(2+v^2) },
\end{equation}
ou encore
\begin{equation}
	\frac{ dt }{ t }=\frac{ 1+v^2 }{ v(2+v^2) }dv.
\end{equation}
Afin d'intégrer le second membre, nous décomposons mettons la fraction en fractions simples que nous cherchons sous la forme
\begin{equation}
	\frac{ 1+v^2 }{ v(2+v^2) }=\frac{ A }{ v }+\frac{ B }{ 2+v^2 }.
\end{equation}
Un calcul montre que
\begin{equation}
	\frac{ 1+v^2 }{ v(2+v^2) }=\frac{ 1 }{ 2v }+\frac{ v }{ 2(2+v^2) }.
\end{equation}
L'intégrale est maintenant aisée, et nous trouvons
\begin{equation}
	\ln\left| \frac{ 1 }{ t } \right| =\frac{ 1 }{2}\ln| v |+\frac{1}{ 4 }\ln| 2+v^2 |+C.
\end{equation}
En manipulant les logarithmes, nous trouvons
\begin{equation}
	\ln\left| \big( \frac{ y^2 }{ t^2 }+2\big)+\big( \frac{ y }{ t } \big)^2  \right|=\ln| Ct^{-4} |.
\end{equation}
Étant donné que le logarithme est injectif sur $\eR$, nous pouvons les simplifier. Le résultat est
\begin{equation}
	y^2(y^2+2t^2)=C.
\end{equation}

\end{enumerate}

\end{corrige}
