% This is part of the Exercices et corrigés de mathématique générale.
% Copyright (C) 2009-2010
%   Laurent Claessens
% See the file fdl-1.3.txt for copying conditions.


\begin{corrige}{INGE1121La0017}

	Pour trouver le rang, faisons un tout petit peu d'échelonnement :
	\begin{equation}
		\begin{aligned}[]
			L_3&\to L_2+L_1\\
			L_4&\to 5L_4+L_2
		\end{aligned}
	\end{equation}
	donne
	\begin{equation}
		\begin{pmatrix}
			 2	&	0	&	-2	&	0	\\
			 0	&	5	&	0	&	-1	\\
			 0	&	0	&	0	&	0	\\ 
			 0	&	0	&	0	&	24	 
		 \end{pmatrix}.
	\end{equation}
	Après avoir barré la troisième ligne, nous trouvons facilement une sous-matrice de taille $3\times 3$ qui a un déterminant non nul. Par conséquent, la rang de la matrice $A$ est $3$.
	
	Le polynôme caractéristique est de degré $4$, mais nous savons que $0$ est une valeur propre, donc nous n'aurons en réalité que du degré $3$ à résoudre. Mais comme nous savons un vecteur propre (donné dans l'énoncé), nous allons savoir une seconde valeur propre et le degré du polynôme caractéristique sera seulement deux.

	Pour savoir la valeur propre du vecteur donné, il suffit de lui appliquer la matrice, on trouve
	\begin{equation}
		\begin{pmatrix}
			 2	&	0	&	-2	&	0	\\
			 0	&	5	&	0	&	-1	\\
			 -2	&	0	&	2	&	0	\\ 
			 0	&	-1	&	0	&	5	 
		 \end{pmatrix}.
		 \begin{pmatrix}
			 1	\\ 
			 1	\\ 
			 -1	\\ 
			 1	
		 \end{pmatrix}=
		 \begin{pmatrix}
			 4	\\ 
			 4	\\ 
			 -4	\\ 
			 4	
		 \end{pmatrix},
	\end{equation}
	et donc $4$ est une valeur propre de vecteur propre $(1,1,-1,1)$.

	Après quelque laborieux calculs, le polynôme caractéristique se factorise en
	\begin{equation}
		P_A(\lambda)=(\lambda-4)(\lambda-6)\lambda(\lambda-4),
	\end{equation}
	donc les valeurs propres sont $\{ 4,6,0 \}$ et $4$ est double.

	Pour les vecteurs propres, nous résolvons les systèmes d'équations correspondants et les réponses sont
	\begin{equation}		\label{EqVectosDSProps}
		\begin{aligned}[]
			6 &\to (0, 1, 0, -1)\\
			0 &\to (1, 0, 1, 0)\\
			4 &\to (1, 0, -1, 0)\\
			4 &\to (0, 1, 0, 1)
		\end{aligned}
	\end{equation}
	Notez que l'espace propre pour la valeur $4$ étant de dimension $2$, il y a de nombreuses autres possibilités : toute les combinaisons linéaires des deux vecteurs proposés ici\footnote{Ce sont ceux que me donne mon ordinateur. J'insiste : soyez capable de résoudre les exercices à l'ordinateur; vous n'allez pas trainer toute votre vie à faire des calculs à la main.} sont corrects. Pour les autres valeurs, seuls les multiples sont corrects.

	La méthode de Gram-Schmidt (ne pas oublier de normaliser après) appliquée aux vecteurs \eqref{EqVectosDSProps} fournit la base orthonormée
	\begin{equation}
		\begin{aligned}[]
			(0, \sqrt{2}2, 0, -\sqrt{2}/2 )\\
			(\sqrt{2}/2, 0, \sqrt{2}/2, 0)\\
			(\sqrt{2}/2, 0, -\sqrt{2}/2, 0)\\
			(0, \sqrt{2}/2, 0, \sqrt{2}/2)
		\end{aligned}
	\end{equation}
	La matrice $B$ qui diagonalise $A$ est la matrice obtenue en mettant ces quatre vecteurs en colonne, et nous avons
	\begin{equation}
		B=\begin{pmatrix}
			0  		&	\sqrt{2}/2	&	 \sqrt{2}/2	&            0\\
			\sqrt{2}/2      &      0		&            0		& 	\sqrt{2}/2\\
			0		&	\sqrt{2}/2	&	-\sqrt{2}/2	&            0\\
			-\sqrt{2}/2	&            0		&            0		&	\sqrt{2}/2
		\end{pmatrix}
	\end{equation}
	et puis
	\begin{equation}
		B^TAB=
		\begin{pmatrix}
			 6	&	0	&	0	&	0	\\
			 0	&	0	&	0	&	0	\\
			 0	&	0	&	4	&	0	\\ 
			 0	&	0	&	0	&	4	 
		 \end{pmatrix}.
	\end{equation}
	Les coordonnées qui font que la forme quadratiques est belle sont données par $X=BY$, et donc par
	\begin{equation}
		\begin{aligned}[]
			x1 = \sqrt{2}y_2/2 + \sqrt{2}y_3/2\\
			x2 = \sqrt{2}y_1/2 + \sqrt{2}y_4/2\\
			x3 = \sqrt{2}y_2/2 - \sqrt{2}y_3/2\\
			x4 = -\sqrt{2}y_11/2 + \sqrt{2}y_4/2
		\end{aligned}
	\end{equation}
	Dans ces coordonnées, la forme quadratique devient simplement une somme des carrés des $y_i$ avec les valeurs propres comme coefficients :
	\begin{equation}
		p(X)=6y_1^2 + 4y_3^2 + 4y_4^2.
	\end{equation}
	Notez que cela ne dépend pas de $y_2$; cela correspond au fait qu'une des valeurs propres est nulle.

	Le genre de la forme quadratique est semi-définie positive parce que toutes ses valeurs propres sont positives ou nulles.	

\end{corrige}
