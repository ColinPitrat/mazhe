% This is part of Exercices et corrigés de CdI-1
% Copyright (c) 2011,2014
%   Laurent Claessens
% See the file fdl-1.3.txt for copying conditions.

\begin{corrige}{0034}

\begin{enumerate}
\item Si $\frac p q$ est un quotient d'entiers strictement positifs,
  alors
  % \begin{equation*}
  \begin{align*}
      q f(\frac pq) &= \underbrace{f(\frac pq) + \cdots + f(\frac pq)}_{q \text{ fois}}\\
    &= f(\underbrace{\frac pq + \ldots
    + \frac pq}_{q \text{ fois}})\\
    &= f(q \frac pq) = f(p) = f(\underbrace{1 +
    \ldots + 1}_{p\text{ fois}})\\
    &= \underbrace{f(1)+\cdots+f(1)}_{p\text{ fois}} = p \, f(1) = p
  \end{align*}
  % \end{equation*}
  donc $f(\frac pq) = \frac pq$.

\item On trouve $f(0) = f(0+0) = f(0)+f(0) = 2 f(0)$ donc
  $f(0)=0$. Par ailleurs, pour $x \in \eR$, on a
  \begin{equation*}
    0 = f(0) = f(x+(-x)) = f(x) + f(-x).
  \end{equation*}

\item On a montré que si $r < 0$ est un rationnel, alors $-r > 0$ donc $f(-r) = -r$ d'après le premier point, or $f(r) = -f(-r)$ d'après le point précédent, donc $f(r) = r$. Par ailleurs, si $x$ est un réel quelconque, il existe une suite $(r_1, r_2, \ldots)$ de rationnels qui tend vers $x$. Dès lors
  \begin{equation*}
    x = \limite i {+\infty} r_i =  \limite i {+\infty} f(r_i) = f\left(\limite i {+\infty} r_i\right) = f(x),
  \end{equation*}
parce que la continuité de $f$ permet d'inverser la limite et $f$. Ceci prouve le résultat demandé.
\end{enumerate}

\end{corrige}
