% This is part of Exercices et corrigés de CdI-1
% Copyright (c) 2011, 2019
%   Laurent Claessens
% See the file fdl-1.3.txt for copying conditions.

\begin{corrige}{OutilsMath-0007}

	Détaillons le premier exercice. Le changement de variable pour les coordonnées sphériques est donné par les équations \eqref{SubEqsCoordSphe}:
	\begin{subequations}
		\begin{align}
			x=\rho\sin(\theta)\cos(\varphi)=3\\
			y=\rho\sin(\theta)\sin(\varphi)=-3\\
			z=\rho\cos(\theta)=3\sqrt{2}.
		\end{align}
	\end{subequations}
	La première chose à calculer est la coordonnées radiale $\rho$. Elle est donnée par
	\begin{equation}
		\rho=\sqrt{x^2+y^2+z^2}=6.
	\end{equation}
	Maintenant nous pouvons trouver $\theta$ en écrivant l'équation pour $z$ :
	\begin{equation}
		6\cos(\theta)=3\sqrt{2}.
	\end{equation}
	Nous trouvons $\cos(\theta)=\sqrt{2}/2$. De là, $\theta$ vaut soit $\frac{ \pi }{ 4 }$ soit $-\frac{ \pi }{ 4 }$. Nous devons choisir $\theta=\frac{ \pi }{ 4 }$ parce que par définition des coordonnées polaires, nous avons toujours $\theta\in\mathopen[ 0 , \pi \mathclose]$.

	Écrivons maintenant l'équation pour $y$:
	\begin{equation}
		5\frac{ \sqrt{2} }{2}\sin(\varphi)=-3,
	\end{equation}
	donc $\sin(\varphi)=-\frac{1}{ \sqrt{2} }$. L'angle $\varphi$ vaut alors soir $-\frac{ \pi }{ 4 }$ soit $\frac{ 5\pi }{ 4 }$. Pour choisir, nous regardons l'équation pour $x$ qui nous indique que $\cos(\varphi)$ doit être positif. Par conséquent nous choisissons $\varphi=-\frac{ \pi }{ 4 }$. Par convention, nous prenons toutefois l'angle $\varphi$ entre $0$ et $2\pi$, donc nous ne prenons pas $\varphi=-\frac{ \pi }{ 4 }$ mais l'angle équivalent $\varphi=\frac{ 7\pi }{ 4 }$.

	Les coordonnées $(r,\theta,\phi)$ sont:
	\begin{enumerate}
		\item	
			$(\rho,\theta,\varphi)=(6,\frac{ \pi }{ 4 },\frac{ 7\pi }{ 4 })$;
		\item
			$\big( 2,\frac{ \pi }{2},-\frac{ \pi }{ 3 } \big)$;
		\item
			$\big( 1,\frac{ \pi }{ 6 },\frac{ \pi }{ 3 } \big)$;
		\item
			$(5,\pi,0)$.
	\end{enumerate}
	
\end{corrige}
