% This is part of Exercices et corrigés de CdI-1
% Copyright (c) 2011,2017
%   Laurent Claessens
% See the file fdl-1.3.txt for copying conditions.

\begin{corrige}{0015}

Étant donné qu'il n'y a qu'une quantité dénombrable de rationnels, une simple adaptation de la question \ref{ItemEnumiE13} de l'exercice \ref{exo0013} donne une suite qui contient une infinité de fois chaque rationnel. Toute suite de rationnels est une sous-suite de cette suite, et en particulier si $r$ est un réel quelconque, une suite de rationnels qui converge vers $r$ est une sous-suite de la suite considérée.

\begin{alternative}
En réalité, toute suite qui énumère tous les rationnels convient. Il ne faut pas spécialement que tous les rationnels arrivent chacun une infinité de fois. En effet, soit $r\in\eR$, et $B=B(r,\epsilon)$, une boule de rayon $\epsilon$ autour de $r$. Si $x_k$ est une suite qui énumère tous les rationnels, pour tout $N$, il existe une infinité de $x_n\in B$ ($b>N$), parce que cette boule contient une infinité de rationnels, qui ne peuvent donc pas être tous énumérés avant le $N$ième élément de la suite $x_k$.
\end{alternative}

\end{corrige}
