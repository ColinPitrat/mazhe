% This is part of the Exercices et corrigés de CdI-2.
% Copyright (C) 2008, 2009
%   Laurent Claessens
% See the file fdl-1.3.txt for copying conditions.


\begin{corrige}{_I-3-6}

L'intégrale peut être calculée explicitement pour chaque $x$ :
\begin{equation}
	F(x)=\int_0^{\infty}\frac{ x }{ x^2+t^2 }dt=\lim_{T\to\infty}\int_0^{T}\frac{ x }{ x^2+t^2 }dt=\lim_{T\to\infty}\left[ \arctg\frac{ t }{ x } \right]_0^T.
\end{equation}
D'où nous déduisons
\begin{equation}
	F(x)=\begin{cases}
	\frac{ \pi }{ 2 }	&	\text{si }x>0\\
	0			&	\text{si }x=0\\
	-\frac{ \pi }{2}	&	\text{si }x<0
\end{cases}
\end{equation}
Voyons maintenant comment se comporte la dérivée de cette fonction. Pour cela, nous regardons l'intégrale de la dérivée, et nous allons prouver sa convergence uniforme sur tout compact. Nous avons $\partial_x\left(\frac{ x }{ x^2+t^2 }\right)=(t^2-x^2)/(x^2+t^2)^2$, donc
\begin{equation}
	\left| \frac{ \partial  }{ \partial x }\left(\frac{ x }{ x^2+t^2 }\right) \right| \leq \frac{ t^2+x^2 }{ (x^2+t^2)^2 }\leq \frac{1}{ x^2+t^2 }
\end{equation}
et l'intégrale de cette dernière expression converge uniformément sur tout compacts de $\eR\setminus\{ 0 \}$. Mais sur chacun de ces compacts, $F(x)$ est constante, c'est à dire que pour tout $x\neq 0$,
\begin{equation}
	G(x)=\int_0^{\infty}\frac{ \partial  }{ \partial x }\left( \frac{ x }{ x^2+t^2 } \right)dt=\frac{ dF }{ dx }=0.
\end{equation}
Par ailleurs, $G(x)$ peut être écrite en termes de $F(x)$ et de l'intégrale que l'on cherche :
\begin{equation}
	0=G(x)=\int_0^{\infty}\frac{ x^2+t^2-2x^2 }{ (x^2+t^2)^2 }dt=\frac{1}{ x }F(x)-2x^2\int_0^{\infty}\frac{ dt }{ (x^2+t^2)^2 }.
\end{equation}
Donc pour $x\neq 0$,
\begin{equation}
	\int_0^{\infty}\frac{ dt }{ (x^2+t^2)^2 }=\frac{ F(x) }{ 2x^3 }=\begin{cases}
	\frac{ \pi }{ 4x^3 }	&	\text{si }x>0\\
	-\frac{ \pi }{ 4x^3 }	&	 \text{si }x<0.
\end{cases}
\end{equation}

\end{corrige}
