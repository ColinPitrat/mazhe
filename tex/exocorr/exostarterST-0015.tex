% This is part of Analyse Starter CTU
% Copyright (c) 2014,2017
%   Laurent Claessens,Carlotta Donadello
% See the file fdl-1.3.txt for copying conditions.

\begin{exercice}\label{exostarterST-0015}

  \begin{remark}
    Les points 4, 5 et 6 de cet exercice sont donnés en devoir.
  \end{remark}

On appelle \emph{sinus hyperbolique} $(\text{sinh})$ et \emph{cosinus hyperbolique} $(\text{cosh})$ les fonctions définies sur $\mathbb{R}$ par 
\begin{equation}\label{defcoshetsinh}
  \text{sinh}(x) = \frac{e^x-e^{-x}}{2} ; \qquad  \text{cosh}(x) = \frac{e^x+e^{-x}}{2}. 
\end{equation}
\begin{enumerate}
\item Trouver les domaines de définition de  $\sinh$ et $\cosh$, étudier leur parité. 
\item Ces fonctions sont continues et dérivables sur tout leur domaine. Trouver $\text{sinh}'(x)$ et  $\text{cosh}'(x)$.
\item Montrer que pour tout $x\in\mathbb{R}$ on a  $\text{cosh}^2 (x) - \text{sinh}^2 (x) = 1$. Que peut-on en déduire sur l'image de $\text{cosh}$ ?
%\item Étudier les variations des deux fonctions et en tracer une représentation graphique. 
\item Démontrer les formules suivantes :
\begin{enumerate}
\item $\text{sinh} (x+y)=\text{sinh}(x) \text{cosh}(y)+\text{cosh}(x)\text{sinh}(y)$ ;
\item $\text{cosh} (x+y)=\text{cosh}(x) \text{cosh}(y)+\text{sinh}(x)\text{sinh}(y)$.
\end{enumerate}
\item Donner des expressions de $\text{cosh}(2x)$ et $\text{sinh}(2x)$  en fonction de $\text{cosh}(x)$ et $\text{sinh}(x)$.
\item Simplifier l'expression $f(x)=\cosh\Big(\ln(x+\sqrt{x^2-1})\Big)$ en utilisant la définition \eqref{defcoshetsinh}.
\end{enumerate}


\corrref{starterST-0015}
\end{exercice}
