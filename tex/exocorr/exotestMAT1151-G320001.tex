\begin{exercice}\label{exotestMAT1151-G320001}

	Cauchy est un petit garçon plein de problèmes. L'un d'entre eux est de trouver la fonction $f\colon \mathopen[ 0 , 4 \mathclose]\to \eR$ telle que pour chaque $x\in\eR$ on ait
	\begin{equation}
		f'(x)=\frac{ 1 }{2}f(x)+2 e^{x/2}\cos(2x),
	\end{equation}
	et telle que
	\begin{equation}
		f(0)=0.
	\end{equation}
	Le petit Augustin Cauchy serait très content que tu lui traces une solution approchée (trouvée par Matlab) sur le même graphique que la solution exacte donnée par
	\begin{equation}
		y(x)=\sin(2x) e^{x/2}.
	\end{equation}

\corrref{testMAT1151-G320001}
\end{exercice}
