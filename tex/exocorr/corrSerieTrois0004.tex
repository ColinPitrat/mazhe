% This is part of Exercices et corrections de MAT1151
% Copyright (C) 2010
%   Laurent Claessens
% See the file LICENCE.txt for copying conditions.

\begin{corrige}{SerieTrois0004}

	En vertu de la remarque \ref{RemConvAlgoNewton}, nous devons simplement considérer la suite numérique $\{ x_n\}$ et en prouver la convergence vers la solution cherchée.

	Nous avons déjà montré à l'exercice \ref{exoSerieDeux0006} que la suite de ces $x_n$ était de toute façon convergente (parce que bornée et croissante ou bien bornée et décroissante d'après le côté). Cette limite est obligatoirement $x(a_0,b_0)$ parce que ce dernier est l'unique solution à l'équation
	\begin{equation}		\label{eqellsollimTrtr}
		\ell=\ell-\frac{ f(\ell) }{ f'(\ell) }.
	\end{equation}
	
	ATTENTION : remarquez que ce dernier point n'est pas tout à fait exact. Il y a deux solutions à l'équation \eqref{eqellsollimTrtr}. En regardant cependant attentivement la façon dont la suite des $x_n$ évolue dans l'exercice \ref{exoSerieDeux0006}, tu dois pouvoir te convaincre que celle dont nous parlons ici est la correcte, si nous commençons la récurrence du bon côté, c'est-à-dire du côté de la plus grande racine.

\end{corrige}
