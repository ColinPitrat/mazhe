\begin{corrige}{controlecontinu0004}


Le domaine d'intégration est donné par une moitié du cerce de centre $(1,0)$ et de rayon $1$, plus précisement la moitié contenue dans le premier quadrant. 

L'équation de ce cercle en  coordonnées polaires est $r=2\cos(\theta)$. Le fait qu'on ne considère que la partie contenue dans le premier quadrant se traduit par $\theta\in[0,\pi/2]$. 

Notre intégrale est alors 

\[
\int_0^{\pi/2}\int_0^{2\cos(\theta)} r^2\, dr\,d\theta= \frac{8}{3}\int_0^{\pi/2}\cos^3(\theta) \, d\theta. 
\]
La valeur de cette intégrale est $16/9$. Une méthode pour le voir est la suivante (on utilise plusieurs fois l'intégration par parties)   
\begin{equation}
  \begin{aligned}
    \int_0^{\pi/2}\cos^3(\theta) \, d\theta=& \left.\sin(\theta)\cos^2(\theta)\right\vert_0^{\pi/2} + 2\int_0^{\pi/2}\sin^2(\theta)\cos(\theta) \, d\theta\\
    &= \left.2\sin^3(\theta)\right\vert_0^{\pi/2} - 4\int_0^{\pi/2}\sin^2(\theta)\cos(\theta) \, d\theta.
  \end{aligned}
\end{equation}
Cela nous  dit que 
\begin{equation}
  \left.2\sin^3(\theta)\right\vert_0^{\pi/2} = 6\int_0^{\pi/2}\sin^2(\theta)\cos(\theta) \, d\theta=3\int_0^{\pi/2}\cos^3(\theta) \, d\theta
\end{equation}
c'est à dire 
\[
\int_0^{\pi/2}\cos^3(\theta) \, d\theta=\frac{2}{3}.
\]
\end{corrige}
