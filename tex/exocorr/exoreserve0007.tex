% This is part of Mes notes de mathématique
% Copyright (c) 2012
%   Laurent Claessens
% See the file fdl-1.3.txt for copying conditions.

\begin{exercice}\label{exoreserve0007}

    Soit \( f(x,y,z)= e^{2x}\sin(y)\).
    \begin{enumerate}
        \item
            Donner les dérivées partielles de \( f\) au point \( \big( \ln\sqrt{3},\pi/2,4 \big)\). Simplifier au mieux les réponses\footnote{Je ne veux plus voir de logarithmes dans une exponentielle.}.
        \item
            Donner \( \nabla f\).
        \item
            Soit le champ de vecteurs \( F(x,y,z)=\nabla f(x,y,z)\) et \( \gamma\) le chemin qui va en ligne droite de \( (0,0,0)\) jusqu'à \( \big( \ln\sqrt{3},\pi/2,4 \big)\). Calculer la circulation de \( F\) le long de \( \gamma\), c'est-à-dire l'intégrale \( \int_{\gamma}F\).

    \end{enumerate}

\corrref{reserve0007}
\end{exercice}
