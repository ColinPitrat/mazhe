% This is part of Exercices de mathématique pour SVT
% Copyright (c) 2010,2015,2017
%   Laurent Claessens et Carlotta Donadello
% See the file fdl-1.3.txt for copying conditions.

\begin{corrige}{TD5-0001}

			\newcommand{\CaptionFigUnSurxInt}{La surface achurée à gauche est la même que la surface achurée à droite. Seul le signe change lorsqu'on veut calculer l'intégrale.}
			\input{auto/pictures_tex/Fig_UnSurxInt.pstricks}

	\begin{enumerate}
		\item
			La technique usuelle pour intégrer une fonction dans laquelle se trouve une valeur absolue est de diviser l'intervalle d'intégration en sous-intervalles sur lesquels nous sommes sûr du signe. Dans le cas de la fonction cosinus entre $0$ et $2\pi$, nous divisons de la façon suivante :
			\begin{equation}
				\mathopen[ 0 , 2\pi \mathclose]=\underbrace{\mathopen[ 0 , \frac{ \pi }{2} \mathclose]}_{\cos(x)\geq 0}\cup\underbrace{\mathopen[ \frac{ \pi }{2} , \frac{ 3\pi }{2} \mathclose]}_{\cos(x)\leq 0}\cup\underbrace{\mathopen[ \frac{ 3\pi }{2} , 2\pi \mathclose]}_{\cos(x)\geq 0}.
			\end{equation}
			Sur les intervalles où $\cos(x)\geq 0$, nous avons $| \cos(x) |=\cos(x)$ et sur les intervalles où $\cos(x)\leq 0$, nous avons $| \cos(x) |=-\cos(x)$. L'intégrale se découpe donc de la façon suivante :
			\begin{equation}
				\begin{aligned}[]
					I&=\int_{0}^{\pi/2}\cos(x)dx+\int_{\pi/2}^{3\pi/2}(-)\cos(x)dx+\int_{3\pi/2}^{2\pi}\cos(x)dx\\
					&=[\sin(x)]_{x=0}^{x=\pi/2}-[\sin(x)]_{x=\pi/2}^{x=3\pi/2}+[\sin(x)]_{x=3\pi/2}^{x=2\pi}\\
					&=(1-0)-(-1-1)+(0-(-1))\\
					&=4.
				\end{aligned}
			\end{equation}
		\item
			Encore une fois, nous décomposons l'intervalle d'intégration en morceaux de telle manière que le signe de $x$ soit constant sur chacun des morceaux :
			\begin{equation}
				\begin{aligned}[]
					\int_{-1}^1x| x |dx&=\int_{-1}^0(-x^2)dx+\int_0^1x^2dx\\
					&=\left[ -\frac{ x^3 }{ 3 } \right]_{-1}^0+\left[ \frac{ x^3 }{ 3 } \right]_0^1\\
					&=0-\left( -\frac{ 1 }{ 3 } \right)+\frac{1}{ 3 }-0\\
					&=\frac{ 2 }{ 3 }.
				\end{aligned}
			\end{equation}
		\item
			En posant $y=2x$, nous trouvons $dy=2dx$, et par conséquent $dx=\frac{ dy }{2}$. En ce qui concerne les bornes, si $x=0$, alors $y=0$ et si $x=\pi$, alors $y=2\pi$. L'intégrale à calculer devient
			\begin{equation}
				\int_0^{\pi}\cos(2x)dx=\int_0^{2\pi}\cos(y)\frac{ dy }{ 2 }=0.
			\end{equation}
		\item


			L'erreur à ne pas faire est de se souvenir que $\ln(x)$ est une primitive de $\frac{1}{ x }$, et de dire
			\begin{equation}
				\int_{-2}^{-1}\frac{1}{ x }dx=\left[ \ln(x) \right]_{-2}^{-1}=\ln(-1)-\ln(-2).
			\end{equation}
            À moins que vous sachiez comment définir le logarithme d'une nombre négatif, ce calcul n'a pas de sens.

			En regardant la figure \ref{LabelFigUnSurxInt}, nous voyons que les aires à gauche sont les mêmes que les aires à droite, sauf que le signe change. Nous pouvons donc écrire
			\begin{equation}
				\int_{-2}^{-1}\frac{1}{ x }dx=-\int_1^2\frac{1}{ x }dx=-\Big( \ln(2)-\ln(1) \Big)=\ln(1)-\ln(2).
			\end{equation}
			Notez que le résultat est négatif comme le dessin l'indique (la fonction est négative entre $-2$ et $-1$).

			Une autre façon de faire est de retenir l'intégrale
			\begin{equation}
				\int \frac{1}{ x }dx=\ln| x |
			\end{equation}
			avec les valeurs absolues.
		\item
			L'intégrale la plus ressemblante que nous connaissons «par cœur» $\int\frac{1}{ 1+x^2 }dx=\arctan(x)$. Le truc est de poser $y=2x$ de façon que $4x^2$ se transforme en $y^2$. Nous avons $dx=dy/2$, et en ce qui concerne les bornes, si $x=0$, $y=0$ et $y=2$ lorsque $x1$. Par conséquent,
			\begin{equation}
				\int_0^1\frac{ dx }{ 1+4x^2 }=\frac{ 1 }{2}\int_0^2\frac{ dy }{ 1+y^2 }=\frac{ 1 }{2}\left[ \arctan(y) \right]_0^2=\frac{ 1 }{2}\arctan(2).
			\end{equation}
			Notez que $\arctan(0)=0$.
		\item
			Lorsque le numérateur et le dénominateur sont des polynômes du même degré, il faut faire apparaître le dénominateur au numérateur en ajoutant et en enlevant ce qu'il faut. Dans notre cas, nous avons 
			\begin{equation}
				\frac{ x^2-1 }{ x^2+1 }=\frac{ x^2+1-2 }{ x^2+1 }=\frac{ x^2+1 }{ x^2+1 }-\frac{ 2 }{ x^2+1 }=1-2\frac{ 1 }{ x^2+1 }.
			\end{equation}
			Par conséquent
			\begin{equation}
				\begin{aligned}[]
					\int_0^1\frac{ x^2-1 }{ x^2+1 }&=\int_0^1 1dx-2\int_0^1\frac{ dx }{ x^2+1 }\\
					&=[x]_0^1-2\left[ \arctan(x) \right]^1_0\\
					&=1-2\arctan(1)\\
					&=1-\frac{ \pi }{ 2 }.
				\end{aligned}
			\end{equation}
			Le fait que $\arctan(1)=\frac{ \pi }{ 4 }$ vient du fait qu'en $\pi/4$, nous avons $\sin(\pi/4)=\cos(\pi/4)$ et donc $\tan(\pi/4)=1$.

		\item
			Le changement de variable $x=2\cos(t)$ est donné dans le sens inverse de l'habitude, mais ce n'est pas grave. Nous avons $dx=-2\sin(t)dt$. Pour avoir $x=0$, il faut $t=\pi/2$ et pour avoir $x=1$, il faut $\cos(t)=\frac{ 1 }{2}$, et donc $t=\pi/3$. L'intégrale à calculer est donc
			\begin{equation}
				\int_0^1\sqrt{4-x^2}dx=\int_{\pi/2}^{\pi/3}\sqrt{4-4\cos^2(t)}(-2)\sin(t)dt.
			\end{equation}
			Pour pouvons simplifier l'expression à intégrer en sortant ce qu'on peut sortir de la racine carrée :
			\begin{equation}
				\sqrt{4-4\cos^2(t)}=\sqrt{4\big( 1-\cos^2(t) \big)}=2\sqrt{1-\cos^2(t)}=2\sqrt{\sin^2(t)}.
			\end{equation}
			Est-ce que nous pouvons dire $\sqrt{\sin^2(t)}=\sin(t)$ ? Étant donné que nous regardons l'intégrale avec $t$ entre $\pi/2$ et $\pi/3$, le sinus est toujours positif, et par conséquent nous pouvons écrire $\sqrt{\sin^2(t)}=\sin(t)$. Au final, l'intégrale que nous devons faire est
			\begin{equation}
				-4\int_{\pi/2}^{\pi/3}\sin^2(t)dt.
			\end{equation}
			Cette intégrale est «connue» parce qu'elle a déjà été calculée : elle est donnée à l'équation \eqref{subEqIntsincdxab}. En remplaçant $a$ par $\pi/2$ et $b$ par $\pi/3$ nous trouvons
			\begin{equation}
				\begin{aligned}[]
					-4\int_{\pi/2}^{\pi/3}\sin^2(x)dx&=-4\frac{ 1 }{2}\left( \frac{ \pi }{3}-\frac{ \pi }{ 2 } \right)-4\frac{ 1 }{2}\left[ -\sin(x)\cos(x) \right]_{\pi/2}^{\pi/3}\\
					&=-2\left( -\frac{ \pi }{ 6 } \right)-2\Big( -\underbrace{\sin(\pi/3)}_{=\sqrt{3}/2}\underbrace{\cos(\pi/3)}_{=1/2}+\sin(\pi/2)\underbrace{\cos(\pi/2)}_{=0} \Big)\\
					&=\frac{ \pi }{ 3 }-2\left( -\frac{ \sqrt{3} }{2}\frac{ 1 }{2} \right)\\
					&=\frac{ \pi }{ 3 }+\frac{ \sqrt{3} }{ 2 }.
				\end{aligned}
			\end{equation}
	\end{enumerate}

\end{corrige}
