% This is part of Exercices de mathématique pour SVT
% Copyright (C) 2010
%   Laurent Claessens et Carlotta Donadello
% See the file fdl-1.3.txt for copying conditions.

\begin{corrige}{TD3-0001}

	Il faut souvent penser à la suite $u_n=(-1)^n$ qui est un exemple de suite qui ne converge pas tout en restant bornée et qui prend «presque tout le temps» les mêmes valeurs.

	\begin{enumerate}
		\item
			Faux. Par exemple la suite $(-1)^nn$. Les termes pairs tendent vers l'infini et les termes impairs vers moins l'infini. Les premiers termes sont
			\begin{equation}
				0,-1,2,-3,4,-5,\ldots.
			\end{equation}
		\item\label{ItemTD31b}
			Faux. La suite $\frac{ (-1)^n }{ n }$ tend vers zéro, mais oscille entre les positifs et les négatifs. Les premiers termes sont
			\begin{equation}
				-1,\frac{ 1 }{ 2 },-\frac{ 1 }{ 3 },\frac{1}{ 4 },-\frac{1}{ 4 },\ldots.
			\end{equation}
		\item
			Vrai. À partir d'un certain moment, la suite doit se «stabiliser» autour de la valeur de la limite.
		\item
			Vrai. C'est la proposition \ref{Propsuiteborncv}.
		\item
			Faux. Même exemple que pour le point \ref{ItemTD31b}. Cette suite n'est ni monotone croissante ni monotone décroissante.
		\item
			Faux. Nous prenons presque le même exemple. Prenons la suite
			\begin{equation}
				0,1,0,\frac{ 1 }{2},0,\frac{ 1 }{3},0,\frac{1}{ 4 },0,\frac{1}{ 5 },\cdots
			\end{equation}
			C'est la suite des $\frac{1}{ n }$ dans laquelle nous avons inséré des $0$ à une place sur deux. Cette suite n'est décroissante à partir d'aucun moment parce qu'elle n'arrête pas de passer de $0$ à un nombre strictement positif.

		\item
			Faux. Si nous prenons la suite $u_n=(-1)^n$, nous avons $(| u_n |)=1$. Cette dernière suite converge vers $1$, mais la suite de départ ne converge pas.
		\item
			Vrai. Une sous-suite d'une suite convergente converge vers la même limite.
		\item
			Faux. Prenons les suites $x_n=n$ et $y_n=n$, c'est-à-dire deux fois la même suite. Évidemment, $(x_n-y_n)=0$ est une suite qui converge vers $0$, pourtant aucune des deux suites de départ ne converge.
		\item
			Faux. La suite $x_n=\frac{1}{ n }$ converge vers $0$. Mais si nous la multiplions par $y_n=4n$, le produit est $x_ny_n=4$ qui converge vers $4$.
		\item
			Faux. À ne pas confondre avec le résultat qui dit qu'une suite encadrée par deux suites qui convergent vers \emph{la même limite} est convergente. Prenons par exemple les suites $x_n=-2$ et $y_n=2$. Ce sont deux suites convergentes qui encadrent la suite $z_n=(-1)^n$. Cette dernière ne converge pas.
		\item
            Faux. Les deux suites peuvent être équivalentes et ne pas converger, penser à \( u_n=n^2+n\) et \( v_n=n^2\). 

            Par contre si les deux suites convergent, alors elles convergent vers la même limite. Supposons $(x_n\to\ell)$ et $(y_n\to\ell')$; dans ce cas les règles de calcul de limites s'appliquent et nous avons
			\begin{equation}
                \lim_{n\to \infty} \frac{ u_n }{ v_n }=\frac{ \ell }{ \ell' }.
			\end{equation}
            Mais si cette limite doit valoir \( 1\), alors nous devons avoir \( \ell=\ell'\).
		\item
			Faux. Si la suite que l'on met au dénominateur s'annule, alors la quotient n'a pas de sens. Attention à la différence entre «suite non nulle» et «suite qui ne s'annule jamais».
	\end{enumerate}

\end{corrige}
