% This is part of Exercices et corrigés de CdI-1
% Copyright (c) 2011
%   Laurent Claessens
% See the file fdl-1.3.txt for copying conditions.

\begin{corrige}{Variete0013}



\begin{enumerate}
\item Soit $C$ le cube, et $B$ le cube plein dont $C$ est le bord. Par
  le théorème de la divergence,
  \begin{equation*}
    \iint_C G\cdot d S = \iiint_B \nabla\cdot G
  \end{equation*}
  or la divergence de $G$ se calcule aisément puisque
  \begin{equation*}
    \nabla\cdot G = 2(x+y+z)
  \end{equation*}
  et donc l'intégrale recherchée est
  \begin{equation*}
    \int_0^a \int_0^a \int_0^a 2 (x+y+z) d xd y d z= 3 a^4.
  \end{equation*}
\item La surface totale $S$ borde un cylindre plein $C$. Le théorème
  de la divergence s'applique, et l'intégrale recherchée est
  \begin{equation*}
    \iint_S G \cdot d S = \iiint_C \nabla\cdot G = \iiint_C 3
  \end{equation*}
  ce qui revient à calculer le triple du volume du cylindre. Ce volume vaut
  $\pi$, et la réponse attendue est donc $3\pi$.

\item Ici la surface $S$ du paraboloïde ne délimite pas un volume
  fermé. Néanmoins, il semble plus facile de calculer le flux au
  travers du \og couvercle\fg{} $C$ d'équations $z = 1, x^2 + y^2 \leq 1$ que le flux au
  travers de $S$. On va donc appliquer le théorème de la divergence
  sous la forme~:
  \begin{equation*}
    \iint_S G \cdot d S + \iint_C G \cdot d S = \iint_{S \cup C} G \cdot d S =
    \iiint_V \nabla\cdot G
  \end{equation*}
  où $V$ est le volume délimité par la surface $S$ et son couvercle
  $C$.

  La divergence de $G$ vaut $2 z$, et donc en passant en coordonnées
  cylindriques on trouve
  \begin{equation*}
    \iiint_V \nabla\cdot G = \int_0^{2\pi}\int_0^1 \int_0^{\sqrt z} 2 z \rho d \rho
    d z d \theta = \frac{2 \pi}{3}.
  \end{equation*}

  Sur le couvercle, $G = (y^2, x^2, 1)$ et le couvercle est paramétrisé par
  \begin{equation*}
    \begin{cases}
      x = \rho \cos(t)\\
      y = \rho \sin(t)\\
      z = 1
    \end{cases}
  \end{equation*}
  avec pour vecteur normal unitaire extérieur $(0,0,1)$. Donc le flux au travers
  de $C$ vaut
  \begin{equation*}
    \iint_C \scalprod{(y^2,x^2,1)}{(0,0,1)} =  \iint_C 1 =
    \int_0^{2\pi} \int_0^1 \rho d\rho d \theta = \pi
  \end{equation*}
  ce qui n'est pas insensé puisqu'on a intégré la fonction $1$ sur un
  disque, donc on retrouve l'aire de ce disque.

  En conclusion, l'intégrale recherchée au départ vaut
  \begin{equation*}
    \iint_S G \cdot d S  = \iiint_V \nabla\cdot G - \iint_C G \cdot d S =
    \frac{2\pi}3 - \pi = -\frac\pi3
  \end{equation*}
\end{enumerate}


\end{corrige}
