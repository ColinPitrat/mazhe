% This is part of Exercices de mathématique pour SVT
% Copyright (c) 2011
%   Laurent Claessens and Carlotta Donadello
% See the file fdl-1.3.txt for copying conditions.

\begin{corrige}{SVT-0005}

    Histoire d'avoir des notations qui ressemblent à ce qu'on a d'habitude, nous écrivons \( y\) pour \( u_C\) et nous mettons l'équation différentielle sous la forme
    \begin{equation}        \label{EqHUhHWZ}
        y'=-\frac{ y }{ RC }+\frac{ V }{ RC }.
    \end{equation}
    Cela est une équation différentielle «presque» à variables séparée. Nous commençons par résoudre l'équation homogène associée
    \begin{equation}
        y'=-\frac{ y }{ RC }.
    \end{equation}
    Nous avons successivement
    \begin{subequations}
        \begin{align}
            y'&=-\frac{ y }{ RC }\\
            \frac{ dy }{ y }&=-\frac{1}{ RC }dt\\
            \ln(y)&=-\frac{ t }{ RC }+A\\
            y(t)&=K e^{-t/RC}.
        \end{align}
    \end{subequations}
    Affin de trouver la solution de l'équation complète nous utilisons la méthode de variation des constantes. Nous posons
    \begin{equation}
        y(t)=K(t) e^{-t/RC}
    \end{equation}
    et nous récrivons l'équation \eqref{EqHUhHWZ} :
    \begin{subequations}
        \begin{align}
            K' e^{-t/RC}-\frac{1}{ RC }K e^{-t/RC}&=-\frac{1}{ RC }K e^{-t/RC}+\frac{ V }{ RC }\\
            K'&=\frac{ V }{ RC } e^{t/RC}\\
            K&=V e^{t/RC}+B
        \end{align}
    \end{subequations}
    où nous avons noté \( B\) la constante d'intégration. La solution complète s'écrit donc
    \begin{equation}
        y(t)=\big( V e^{t/RC}+B \big) e^{-t/RC}=V+B e^{-t/RC}
    \end{equation}
    où \( B\) est une constante arbitraire. La fonction \( q(t)\) est obtenue en multipliant par \( C\) la fonction \( u_C(t)\) :
    \begin{equation}
        q(t)=VC+BC e^{-t/RC}.
    \end{equation}
    Afin d'avoir \( q(0)=0\) nous devons fixer \( B=V\), c'est à dire 
    \begin{equation}
        q(t)=VC(1+ e^{-t/RC}).
    \end{equation}
    Cela est la quantité de charges électriques sur le condensateur après un temps \( t\). Nous avons
    \begin{equation}
        \lim_{t\to \infty} q(t)=VC,
    \end{equation}
    qui correspond à la formule connue de la charge d'un condensateur lorsque le système est à l'équilibre.

\end{corrige}
