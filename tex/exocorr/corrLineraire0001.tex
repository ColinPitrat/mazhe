% This is part of the Exercices et corrigés de mathématique générale.
% Copyright (C) 2009
%   Laurent Claessens
% See the file fdl-1.3.txt for copying conditions.
\begin{corrige}{Lineraire0001}

	\begin{enumerate}

		\item
			$x=2$, $y=1$.
		\item
			Le système se met sous forme de matrice et puis se manipule de la façon suivante.
			\begin{equation}
				\left(
				\begin{array}{ccc|c}
					 1	&	2	&	3	&	0	\\ 
					  2	&	1	&	3	&	0	\\ 
					   3	&	2	&	1	&	0	 
				\end{array}
				\right)
				\sim
				\left(\begin{array}{ccc|c}
					 1	&	2	&	3&	0	\\ 
					  0	&	-3	&	-3	&	0	\\ 
					   0	&	-4	&	-8	&	0	 
				   \end{array}
				   \right)
				   \sim
				   \left(\begin{array}{ccc|c}
					    1	&	2	&	3	&	0	\\ 
					     0	&	1	&	1	&	0	\\ 
					      0	&	2	&	4	&	0 
				      \end{array}\right)
				      \sim
				      \left(\begin{array}{ccc|c}
					       1	&	2	&	3	&	0	\\ 
					        0	&	1	&	1	&	0	\\ 
						 0	&	0	&	2	&	0	 
					 \end{array}\right).
			\end{equation}
			En pratique, il n'est pas nécessaire de continuer, parce que maintenant la solution se lit facilement. En effet, la dernière ligne dit $z=0$. La seconde ligne dit $y+z=0$, (donc $y=0$) et la première ligne dit $x+2y+3z=0$, donc $x=0$.
		\item
			Ce système a $3$ équations pour $4$ inconnues. Nous ne nous attendons donc pas à pouvoir le résoudre complètement. La stratégie va donc être de tout exprimer en termes de la dernière variable. Nous n'allons donc pas essayer de mettre de zéros dans la dernière ligne.
			\begin{equation}
				\left(\begin{array}{cccc|c}
					 1	&	1	&	1	&	-3	&	2	\\ 
					  1	&	-2	&	2	&	15	&	-3	\\ 
					   1	&	1	&	-1	&	-9	&	0	
				 \end{array}\right)
				   \sim
				\left(\begin{array}{cccc|c}
					 1	&	1	&	1	&	-3	&	2	\\ 
					  1	&	-2	&	2	&	15	&	-3	\\ 
					   0	&	0	&	-2	&	-6	&	-2						 
				 \end{array}\right)\sim
				\left(\begin{array}{cccc|c}
					 1	&	1	&	1	&	-3	&	2	\\ 
					  0	&	-3	&	1	&	18	&	-5	\\ 
					   0	&	0	&	-2	&	-6	&	-2	
				 \end{array}\right)
			\end{equation}
			À partir d'ici, nous pouvons exprimer toutes les variables en fonction de $t$. La dernière ligne dit que $-2z-6t=-2$, ce qui donne
			\begin{equation}
				z=1-3t.
			\end{equation}
			La seconde ligne dit que $-3y+z+18t=-5$. En injectant dedans la valeur $z=1-3t$, nous trouvons 
			\begin{equation}
				y=2+5t,
			\end{equation}
			Et enfin, la première ligne dit que $x+y+z-3t=2$. En injectant les valeurs déjà trouvées de $y$ et $z$ en fonction de $t$, nous trouvons
			\begin{equation}
				x=t-1.
			\end{equation}
			Notons qu'il y a quand même moyen de se simplifier un peu la vie en ajoutant encore plus de zéros dans la matrice (tout en en cassant aucun !). En effet, on peut utiliser la troisième ligne pour mettre des zéros sur la colonne des $z$ des premières et deuxièmes lignes, et puis utiliser la deuxième ligne pour mettre un zéro sur la colonne des $y$ dans la première ligne :
			\begin{equation}
				\left(\begin{array}{cccc|c}
					 1	&	1	&	0	&	-6	&	1	\\ 
					  0	&	-3	&	0	&	15	&	-6	\\ 
					   0	&	0	&	1	&	3	&	1	
				   \end{array}\right)\sim
				   \left(\begin{array}{cccc|c}
					    1	&	0	&	0	&	-1	&	-1	\\ 
					     0	&	-1	&	0	&	5	&	-2	\\ 
					      0	&	0	&	1	&	3	&	1	
				      \end{array}\right).				      
			\end{equation}
			Sur cette dernière matrice, les solutions se lisent encore plus facilement.
		\item
			Cette fois, il y a plus d'équations que d'inconnues. Nous nous attendons donc à ce qu'il n'y ait soit pas de solutions, soit qu'il y ait une ligne « en trop ». Nous pouvons commencer par mettre des zéros sur la colonne du $z$ de la première et seconde ligne en utilisant la troisième.
			\begin{equation}
				\left(\begin{array}{ccc|c}
					1	&	4	&	1	&	12	\\	
					1	&	1	&	-1	&	0	\\	
						2	&	1	&	0	&	4	\\	
							1	&	0	&	1	&	4	
						\end{array}\right)
						\sim
				\left(\begin{array}{ccc|c}
					0	&	4	&	0	&	8	\\
					2	&	1	&	0	&	4\\
					2	&	1	&	0	&	4\\
					1	&	9	&	1	&	4
				\end{array}\right).
			\end{equation}
			La deuxième et la troisième ligne sont identique. Nous pouvons donc simplement barrer une des deux et continuer comme si nous avions que trois équations. Il est toujours bien de simplifier la première ligne par $4$. Cette première ligne ne contient qu'un seul coefficient non nul. Elle donne donc tout de suite $y=2$. À partir de là, le système est simple à résoudre. Nous avons
			\begin{equation}
				\left\{
				\begin{array}{ll}
					y=2\\
					2x+y=4\\
					x+z=4.
				\end{array}
				\right.
			\end{equation}
			Sachant que $y=2$, la seconde équation donne $x=1$, et sachant que $x=1$, la troisième donne $1+z=4$, c'est-à-dire $z=3$.

		\item
			Il y a deux équations pour $4$ inconnues, donc on va pouvoir laisser deux variables non résolues. Exprimons $x$ et $z$ en termes de $y$ et $t$ (tout autre choix est bon). La somme des deux équations donne tout de suite $3x+3y+3t=-3$, et donc
			\begin{equation}
				x=-1-y-t.
			\end{equation}
			En remettant cela dans la première équation, nous avons $-2-2y-2t+y-3z-t=1$, ce qui donne
			\begin{equation}
				z=\frac{ 3+y+3t }{ 3 }.
			\end{equation}
	\end{enumerate}

\end{corrige}
