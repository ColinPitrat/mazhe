% This is part of Exercices et corrigés de CdI-1
% Copyright (c) 2011,2017
%   Laurent Claessens
% See the file fdl-1.3.txt for copying conditions.

\begin{corrige}{OutilsMath-0082}

    Étant donné que $f$ ne dépend que de $\rho$, la formule \eqref{EqLaplaceSpheOM} du laplacien se réduit à
    \begin{equation}
        \Delta f=\frac{1}{ \rho^2\sin\theta }\frac{ \partial  }{ \partial \rho }\left( \rho^2\sin\theta\frac{ \partial f }{ \partial \rho } \right).
    \end{equation}
    Demander que cela soit nul revient à poser l'équation
    \begin{equation}
        2\frac{ \partial f }{ \partial \rho }+\rho\frac{ \partial^2f }{ \partial \rho^2 }=0.
    \end{equation}
    
    Histoire de retomber sur des notations habituelles, nous cherchons une fonction $f(x)$ telle que $2f'(x)+xf''(x)=0$. Nous posons d'abord $g(x)=f'(x)$. L'équation pour $g$ est
    \begin{equation}
        2g(x)+xg'(x)=0,
    \end{equation}
    c'est à dire
    \begin{equation}
        \frac{ g' }{ f }=-\frac{ 2 }{ x }.
    \end{equation}
    En intégrant des deux côtés,
    \begin{equation}
        \ln\big( g(x) \big)=-2\ln(x)+C.
    \end{equation}
    Les fonctions $g$ qui satisfont l'équation sont donc
    \begin{equation}
        g(x)=Kx^{-2}.
    \end{equation}
    Étant donné que $g=f'$, une intégration donne
    \begin{equation}
        f(x)=-\frac{ K }{ x }+C,
    \end{equation}
    et donc les fonctions harmoniques radiales sont les fonctions de la forme
    \begin{equation}
        f(\rho)=-\frac{ K }{ \rho }+C.
    \end{equation}

\end{corrige}
