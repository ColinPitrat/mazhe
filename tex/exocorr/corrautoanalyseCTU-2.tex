% This is part of Analyse Starter CTU
% Copyright (c) 2014
%   Laurent Claessens,Carlotta Donadello
% See the file fdl-1.3.txt for copying conditions.

\begin{corrige}{autoanalyseCTU-2}


\begin{enumerate}
\item La fonction dérivée de $f$ est  $f'(x) = 3x^2 + 1$. Il s'agit d'une fonction positive définie sur $\eR$. La fonction $f$ est donc strictement monotone croissante. 
\item Par le point précédent nous savons que $f$ est strictement monotone sur son ensemble de définition, qui est $\eR$ tout entier. La fonction $f$ est donc une bijection de $\eR$ dans son image. Pour trouver l'image de $f$ nous nous rappelons que $\lim_{x\to \pm\infty} f(x) = \pm\infty$ et que $f$ est continue (toute fonction polynomiale est continue). Le théorème \ref{ThoLEPooJxGXSN} nous dit alors que l'image de $f$ est $\eR$ tout entier. 
\item Il n'est pas aisé d'écrire $g$, la bijection réciproque de $f$, sous une forme analytique explicite. Par conséquent nous sommes obligés à répondre aux questions de ce point en utilisant les informations que nous avons sur $f$. D'abord, l'ensemble de définition de $g$ est l'image de $f$, qui est $\eR$. Ensuite, la dérivée de $g$ en $x=3$ peut \^etre calculée à partir de la formule \eqref{EqWWAooBRFNsv} 
  \begin{equation*}
    g'(3) = \frac{1}{f'(g(3))} = \frac{1}{3(g(3))^2 +1}.
  \end{equation*}
La valeur de $g(3)$ est la solution de $f(x) = 3$, c'est à dire $x^3+x+1 = 3$, on peut écrire $x(x^2+1) = 2$ et remarquer que $x=1$ est une solution de cette équation. C'est d'ailleurs la seule solution, car $f$ est une bijection. On a donc $g'(3) = 1/4$.
\end{enumerate}

\end{corrige}   
