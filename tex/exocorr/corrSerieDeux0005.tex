% This is part of (almost) Everything I know in mathematics
% Copyright (C) 2010,2016
%   Laurent Claessens
% See the file LICENCE.txt for copying conditions.

\begin{corrige}{SerieDeux0005}

	La plus grande racine est donnée, en fonction de $(a,b)\in\eR^2$, par
	\begin{equation}
		\xi(a,b)=\frac{ -a+\sqrt{a^2-4b} }{2}.
	\end{equation}
	Le gradient de cette fonction se \href{http://www.sagemath.org}{calcule} \href{http://www.sagemath.org/doc/reference/sage/plot/plot_field.html}{facilement} :
	\begin{equation}
		\begin{aligned}[]
			(\partial_a\xi)(a,b)&=-\frac{ 1 }{2}\frac{ \xi(a,b) }{ \sqrt{a^2-4b} }\\
			(\partial_b\xi)(a,b)&=-\frac{1}{ \sqrt{a^2-4b} }.
		\end{aligned}
	\end{equation}
	La norme du gradient vaut
	\begin{equation}
		\| \nabla\xi(a,b) \|=\frac{ \sqrt{\xi^2+4} }{ 2\sqrt{a^2-4b} }\simeq K^{\eta}_{\text{abs}}(a,b)
	\end{equation}
	si $\eta$ est assez petit. Nous trouvons ensuite le conditionnement relatif
	\begin{equation}
		K_{\text{rel}}\simeq \frac{    \sqrt{\xi^2+4}\sqrt{a^2+b^2}    }{ \xi\sqrt{a^2-4b} }.
	\end{equation}
	Pour que le conditionnement soit bon, il faut que $(a,b)\to(0,0)$. Il faut étudier cette limite.

\end{corrige}
