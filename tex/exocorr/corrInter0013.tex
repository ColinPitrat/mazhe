% This is part of the Exercices et corrigés de mathématique générale.
% Copyright (C) 2009, 2012
%   Laurent Claessens
% See the file fdl-1.3.txt for copying conditions.
\begin{corrige}{Inter0013}

La difficulté est de savoir entre quelle et quelle borne du paramètre $t$ se situe un « arc » de cycloïde. Manifestement, la fonction $x(t)$ n'est pas périodique, et même croissante. La fonction $y(t)$, par contre, est périodique de période $2\pi$. Notez que la vitesse de la courbe
\begin{equation}
	v=( \dot x(t),\dot y(t) )=a(1-\cos(t),\sin(t))
\end{equation}
a pour norme
\begin{equation}
	\| v \|=2a^2\big( 1-\cos(t) \big),
\end{equation}
s'annule précisément tous les $2\pi$. Nous prenons donc cela comme un bras de cycloïde.

%TODO : refaire la figure
%et d'ailleurs, le graphe de la figure \ref{LabelFigCycloide} nous montre bien comment ça se passe.
%\newcommand{\CaptionFigCycloide}{La cycloïde.}
%\input{auto/pictures_tex/Fig_Cycloide.pstricks}

En utilisant la formule de trigonométrie $1-\cos(t)=2\sin^2\frac{ x }{ 2 }$, et la formule de la longueur d'arc \eqref{EqLongArcParam}, nous avons la longueur donnée par
\begin{equation}
	l=| a |\sqrt{2}\int_0^{2\pi}\sqrt{1-\cos(t)}=2| a |\int_0^{\pi}|\sin(u)|2 du=4| a |\int_0^{\pi}\sin(u) du
\end{equation}
où nous avons utilisé le changement de variable $u=\frac{ x }{ 2 }$, ainsi que supprimé la valeur absolue parce que la fonction sinus est toujours positive entre $0$ et $\pi$. La longueur de l'arc est donc
\begin{equation}
	l=4a\left[ -\cos(u) \right]_0^{\pi}=8a,
\end{equation}
et donc $a=1$ répond à la question.

\end{corrige}
