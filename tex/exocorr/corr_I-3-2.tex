% This is part of the Exercices et corrigés de CdI-2.
% Copyright (C) 2008, 2009
%   Laurent Claessens
% See the file fdl-1.3.txt for copying conditions.


\begin{corrige}{_I-3-2}

Lorsque $n=1$, la solution proposée \eqref{EqSolPropoEqDiffEx} devient
\begin{equation}
	y(x)=\int_a^xf(t)dt,
\end{equation}
donc $y'(x)=f(x)$ et $y(a)=0$. Il faut maintenant étudier le cas où $n>1$. En utilisant la formule \eqref{EqFormDerrFnAvecBorneNInt}, nous trouvons
\begin{equation}
	\begin{aligned}[]
	\frac{ dy }{ dx }	&=	\int_a^x \frac{ \partial  }{ \partial x }\left( \frac{ (x-t)^{n-1} }{ (n-1)! }f(t)\right)dt+\underbrace{\left[   \frac{ (x-t)^{n-1} }{ (n-1)! }f(t)dt  \right]_{t=x}}_{=0}\cdot 1\\
				&=\int_a^x\frac{ (x-t)^{n-2} }{ (n-2)! }f(t)dt.
	\end{aligned}
\end{equation}
Si $n=2$, nous avons donc $\frac{ dy }{ dx }=\int_a^xf(t)dt$, ce qui fait $y''(x)=f(x)$ et $y(a)=y'(a)=0$. Par récurrence,
\begin{equation}
	\frac{ d^py }{ dy^p }=\int_a^x\frac{ (x-t)^{n-(p+1)} }{ \big( n-(p+1) \big)! }f(t)dt,
\end{equation}
et en particulier,
\begin{equation}
	\frac{ d^{n-1}y }{ dx^{n-1} }\int_a^xf(t)dt,
\end{equation}
ce qui donne immédiatement $y^{\underline{n}}(x)=f(x)$ et $y(a)=y'(a)=\ldots=y^{\underline{n-1}}(a)=0$.

\end{corrige}
