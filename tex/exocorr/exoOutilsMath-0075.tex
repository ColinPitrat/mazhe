% This is part of Exercices et corrigés de CdI-1
% Copyright (c) 2011,2013
%   Laurent Claessens
% See the file fdl-1.3.txt for copying conditions.

\begin{exercice}\label{exoOutilsMath-0075}

    En mécanique classique, le \defe{moment}{moment!d'une force} $M$ d'une force $F$ par rapport à un point $O$ est le produit de la forme $\| F \|$ par la distance $d$ entre $O$ et la ligne qui porte $F$, voir figure \ref{LabelFigMomentForce}.

    Le vecteur moment $\overline{ M }$ est le vecteur de taille $M$ et orthogonal au plan défini par $O$ et par $F$.

    Vérifier que $\overline{ M }=R\times F$ où $R$ est le vecteur qui lie $O$ à l'origine de $F$, voir figure 

    
\newcommand{\CaptionFigMomentForce}{Moment de force.}
\input{auto/pictures_tex/Fig_MomentForce.pstricks}


\corrref{OutilsMath-0075}
\end{exercice}
