% This is part of the Exercices et corrigés de mathématique générale.
% Copyright (C) 2009
%   Laurent Claessens
% See the file fdl-1.3.txt for copying conditions.
\begin{corrige}{EquaDiff0008}

L'équation donnée étant une équation linéaire nous résolvons d'abord l'équation homogène associée :
\begin{equation}
	y'_H+\frac{ y_H }{ 1+x }=0.
\end{equation}
En remettant les $y_H$ d'un côté et les $x$ de l'autre et en intégrant, nous trouvons la solution générale de l'homogène sous la forme
\begin{equation}
	y_H(x)=\frac{ C }{ 1+x }.
\end{equation}
La méthode de variation des constantes demande d'introduire
\begin{equation}
	\begin{aligned}[]
		y(x)&=\frac{ C(x) }{ 1+x }\\
		y'(x)&=\frac{ C'(x) }{ 1+x }-\frac{ C(x) }{ (1+x)^2 }
	\end{aligned}
\end{equation}
dans l'équation de départ. Les termes en $C$ non dérivés se simplifient, et nous restons avec
\begin{equation}
	C'(x)=x(1+x),
\end{equation}
que nous intégrons immédiatement : $C(x)=\frac{ x^2 }{ 2 }+\frac{ x^3 }{ 3 }+K$, et donc
\begin{equation}		\label{EqDol008SScAu}
	y(x)=\frac{ 3x^2+2x^3+K }{ 6(1+x) }.
\end{equation}

Nous pouvons résoudre les conditions initiales sans dériver \eqref{EqDol008SScAu}. En effet, l'équation de départ dit que $y'(0)+\frac{ y(0) }{ 1+0 }=0$. En demandant $y'(0)=1$, cette équation dit que $y(0)=K/6$. Nous posons maintenant $x=0$ dans \eqref{EqDol008SScAu} et nous demandons que le résultat soit $-1$. Ce que nous trouvons est $k=-6$, et donc la solution
\begin{equation}
	y(x)=\frac{ 3x^2+2x^3-6 }{ 6(1+x) }.
\end{equation}
\end{corrige}
