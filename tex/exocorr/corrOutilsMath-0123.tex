% This is part of Outils mathématiques
% Copyright (c) 2011
%   Laurent Claessens
% See the file fdl-1.3.txt for copying conditions.

\begin{corrige}{OutilsMath-0123}

    Au vu de la définition \ref{DefSurfReguliereOM}, il faut vérifier que les vecteurs \( T_u\) et \( T_v\) sont non nuls et non colinéaires. Les vecteurs tangents à le paramétrage sont vite calculés:
    \begin{equation}
        \begin{aligned}[]
            T_u=\begin{pmatrix}
                1    \\ 
                1    \\ 
                v    
            \end{pmatrix}&&T_v=\begin{pmatrix}
                -1    \\ 
                1    \\ 
                u    
            \end{pmatrix}.
        \end{aligned}
    \end{equation}
    Ils ne sont évidemment jamais nuls et ne sont jamais colinéaires parce qu'il n'existe pas de multiples de \( T_u\) qui soit égal à \( T_v\).

    La surface \( S\) sont nous devons calculer l'aire est l'image par \( \Phi\) du disque unité. Le bon paramétrage de la surface est donc \( \Phi\colon D\to S\). Nous devons donc calculer
    \begin{equation}
        Aire=\int_D\| T_u\times T_v \|\,dudv.
    \end{equation}
    Un petit calcul montre que
    \begin{equation}
        T_u\times T_v=\begin{vmatrix}
            e_x    &   e_y    &   e_z    \\
            1    &   1    &   v    \\
            -1    &   1    &   u
        \end{vmatrix}=(u-v)e_x-(u+v)e_y+2e_z.
    \end{equation}
    Nous devons donc calculer l'intégrale de la fonction
    \begin{equation}
        \| T_u\times T_v \|=\sqrt{2u^2+2v^2+4}
    \end{equation}
    sur le disque \( D\equiv u^2+v^2\leq 1\). Nous faisons cela en polaires : nous posons \( u=r\cos(\theta)\), \( v=r\sin(\theta)\), et nous n'oublions pas le jacobien des polaires :
    \begin{equation}
        Aire=\int_0^{2\pi}d\theta\int_0^1r\sqrt{2r^2+4}dr.
    \end{equation}
    En posant \( t=2r^2+4\), nous avons
    \begin{equation}
        \begin{aligned}[]
            Aire&=2\pi\int_4^6\sqrt{t}\frac{ dt }{ 4 }\\
            &=\frac{ \pi }{2}\int_4^6\sqrt{t}dt\\
            &=\frac{ \pi }{ 3 }(6\sqrt{6}-8).
        \end{aligned}
    \end{equation}

\end{corrige}
