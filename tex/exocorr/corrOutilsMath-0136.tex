% This is part of Outils mathématiques
% Copyright (c) 2012
%   Laurent Claessens
% See the file fdl-1.3.txt for copying conditions.

\begin{corrige}{OutilsMath-0136}

La matrice jacobienne se calcule facilement :
\begin{equation}
    \begin{pmatrix}
        y    &   x    &   0    \\
        1    &   1    &   0    \\
        0    &   0    &   \frac{1}{ 1+z^2 }
    \end{pmatrix}.
\end{equation}

L'application \( T\) n'est pas une bijection parce que ce n'est pas injectif. Par exemple \( T(1,2,0)=T(2,1,0)\). Plus généralement vu que \( x\) et \( y\) ont des rôles symétriques dans \( T\), nous avons toujours \( T(x,y,z)=T(y,x,z)\).


Beaucoup d'étudiants ont pris l'initiative de calculer le déterminant de la matrice jacobienne, bien que ce n'était pas demandé. Le voici :
\begin{equation}
    J_T=\det\begin{pmatrix}
        y    &   x    &   0    \\
        1    &   1    &   0    \\
        0    &   0    &   \frac{1}{ 1+z^2 }
    \end{pmatrix}=\frac{ y-x }{ 1+z^2 }.
\end{equation}
Bien entendu, ceux qui se sont vautrés dans le calcul ont perdu des points : si vous prenez l'initiative de me montrer que vous n'êtes pas capable de calculer un déterminant, j'apprécie l'initiative à sa juste valeur.

\end{corrige}
