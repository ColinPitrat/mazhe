% This is part of Exercices et corrigés de CdI-1
% Copyright (c) 2011-2012,2014
%   Laurent Claessens
% See the file fdl-1.3.txt for copying conditions.

\begin{corrige}{continueSup0004}

\begin{enumerate}
\item  Il suffit de prouver que ces fonctions sont continues en $0$.\\
\underline{en effet}: Supposons qu'on ait que $\lim_{x\rightarrow  0} \sin x = 0, \ \lim_{x\rightarrow  0} \cos x = 1$. Regardons la limite  en $a$ réel quelconque
\[
	\lim_{x\rightarrow  a} \sin x= \lim_{t\rightarrow  0}[ \sin (a+t)] = \sin a \lim_{t\rightarrow  0}[ \cos t] + \cos a \lim_{t\rightarrow  0} [\sin  t] = \sin a
\]
en utilisant les formules d'addition des arcs. Le même raisonnement s'applique à $\cos x$. 


\item $\lim_{t\rightarrow  0} \sin t  =0$?\\
\underline{Affirmation} $|\sin x| \leq |x|, \ \forall x \in \eR^n$.\\
\underline{en effet} Si $|x|\geq 1$, c'est évident.\\
Pour les réels $|x|\leq 1$, considérons juste les valeurs de $x$ positives (c'est suffisant), et regardons ce que ça donne dans le cercle trigonométrique.

\newcommand{\CaptionFigDessinLim}{Dessin qui permet de calculer quelques limites.}
\input{auto/pictures_tex/Fig_DessinLim.pstricks}

L'inégalité suivante est claire sur la figure \ref{LabelFigDessinLim} :
\[ {\rm Aire \ triangle \ }  OAP \leq {\rm Aire \  portion \  de \  disque \ } OAP\]
Or, l'aire du triangle $OAP$ est en fait $\f{1}{2}$base$\times$hauteur $= \f{1}{2}*1*\sin x$. Et celle de la portion de disque $OAP$ est en fait $\f{1}{2}r^2\theta = \f{1}{2}*1* x$. Ceci implique immédiatement que 
\[\f{1}{2}\sin x \leq \f{1}{2} x\] ce qui nous donne le résultat.

\item $\lim_{t\rightarrow  0} \cos t  =1$?\\

On sait que $1\geq \cos^2 x = 1-\sin^2 x \geq 1-x^2$ par le point (ii). Donc, 
\[
\sqrt{1-x^2}\leq \cos x \leq 1
\] ce qui par la règle de l'étau nous donne le résultat.

\item Les fonctions $\cos $ et $\sin$ sont dérivables partout. \\


Note : Il aurait été beaucoup plus efficace  de prouver directement qu'elles sont dérivables et d'en déduire qu'elles sont continues.

Regardons en $a\in \eR^n$: 

\[\lim_{x\rightarrow  a}\f{ \cos x  - \cos a}{x-a} = \lim_{x\rightarrow  a}\f{ -2\sin(\f{x+a}{2})\sin(\f{x-a}{2})}{x-a}\\
= -\sin a \left[  \lim_{x\rightarrow  a}\f{ \sin(\f{x-a}{2})}{\f{x-a}{2}}\right]\]


\[\lim_{x\rightarrow  a}\f{ \sin x  - \sin a}{x-a} = \lim_{x\rightarrow  a}\f{ 2\cos(\f{x+a}{2})\sin(\f{x-a}{2})}{x-a}\\
= \cos a \left[  \lim_{x\rightarrow  a}\f{ \sin(\f{x-a}{2})}{\f{x-a}{2}}\right]\]

Ceci nous montre que, à condition que nous arrivions à prouver que $\lim_{t\rightarrow  0} \f{\sin t}{t}=1$, nous aurons montré que 
\[
    \begin{array}{l} (\cos(x))'|_{x=a} = -\sin(a), \\ (\sin(x))'|_{x=a} = \cos(a),\end{array}
\]
 ce qui prouve que $\cos$ et $\sin$ sont dérivables en tout point, et même qu'elles sont dérivables une infinité de fois et que chacune de ces dérivées est elle-même continue. 

Il reste donc à prouver que $\lim_{t\rightarrow  0} \f{\sin t}{t}=1$. Prenons une fois de plus $x>0$. Si nous reprenons le graphe de la figure \ref{LabelFigDessinLim}, nous pouvons écrire les inégalités

\[ {\rm Aire \ }\Delta  OAP \leq {\rm Aire \  portion \  de \  disque \ } OAP \leq {\rm Aire \ }\Delta  OAT\] Sachant que la longueur du segment $AT$ est en fait égale à la tangente de l'angle qui le sous-tend, nous avons donc les inégalités:
\[\begin{array}{c} \f{1}{2}\sin x \leq \f{1}{2} x\leq\f{1}{2} \tan x\\ \\
	  \Rightarrow 1 < \dfrac{x}{\sin x} < \dfrac{1}{\cos x} \\\\
  \Rightarrow 1 > \dfrac{\sin x}{x} > \cos x\end{array}\]
	  et cette dernière égalité permet d'utiliser l'étau et d'obtenir notre résultat.	
	  
 \end{enumerate}

\end{corrige}
