\begin{corrige}{CalculDifferentiel0019}

	Le plus simple est de calculer à partir des définitions. Nous avons
	\begin{equation}
		\begin{aligned}[]
			\frac{ \partial g }{ \partial x }(x,y)&=\lim_{t\to 0} \frac{ g(x+t,y)-g(x,y) }{ t }\\
			&=\lim_{t\to 0} \frac{ f(y,x+t)-f(y,x) }{ t }\\
			&=\frac{ \partial f }{ \partial y }(y,x);
		\end{aligned}
	\end{equation}
	ensuite
	\begin{equation}
		\begin{aligned}[]
			\frac{ \partial g }{ \partial y }(x,y)&=\lim_{t\to 0} \frac{ g(x,y+t)-g(x,y) }{ t }\\
			&=\lim_{t\to 0} \frac{ f(y+t,x)-f(y,x) }{ t }\\
			&=\frac{ \partial f }{ \partial x }(y,x).
		\end{aligned}
	\end{equation}

	Pour $h$, le plus simple est de considérer $h$ comme une fonction composée :
	\begin{equation}
		h(x)=(f\circ\varphi)(x,y)
	\end{equation}
	où $\varphi\colon \eR^2\to \eR^2$ est donnée par $\varphi(x,y)=(x,x)$. En utilisant la formule de dérivation des fonctions composées,
	\begin{equation}
		\begin{aligned}[]
		h'(x)&=\partial_1f\big( \varphi(x,y) \big)\underbrace{\frac{ \partial \varphi_1 }{ \partial x }(x,y)}_{=1}+\partial_2f\big( \varphi(x,y) \big)\underbrace{\frac{ \partial \varphi_2 }{ \partial x }(x,y)}_{=1}\\
		&=\frac{ \partial f }{ \partial x }(x,x)+\frac{ \partial f }{ \partial y }(x,x).
		\end{aligned}
	\end{equation}

\end{corrige}
