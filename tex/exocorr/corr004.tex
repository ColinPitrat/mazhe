\begin{corrige}{004}
The natural mistake is to say ``$\theta$ takes values in $[0,2\pi[$ which is not an open interval. Then I cannot parametrize $E$ ''. One can use more than one local chart ! For instance it is possible to take
\[ 
\begin{aligned}
  \mU_1&=\,]0,2\pi[\,,\\
  \mU_2&=\,]\frac{ 3\pi }{ 2 }, \frac{ 5\pi }{ 2 }[
\end{aligned}
\]
and in both cases
\[ 
  \varphi_i(\theta)=
\begin{pmatrix}
\cos\theta&\sin\theta\\
-\sin\theta&\cos\theta
\end{pmatrix}.
\]
In order to write a tangent vector at $\theta=0$, we consider a general path in $E$ given by a path $\dpt{ \theta }{ \eR }{ \eR }$ such that $\theta(0)=0$. We have
\begin{equation}
\begin{split}
\Dsdd{ 
\begin{pmatrix}
\cos\theta(t)&\sin\theta(t)\\
-\sin\theta(t)&\cos\theta(t)
\end{pmatrix}
 }{t}{0}
&=
\begin{pmatrix}
-\theta'(0)\sin\theta(0)&\theta'(0)\cos\theta(0)\\
-\theta'(0)\cos\theta(0)&-\theta'(0)\sin\theta(0)
\end{pmatrix}\\
&=
\theta'(0)
\begin{pmatrix}
0&1\\-1&0
\end{pmatrix}
\end{split}
\end{equation}
which is a general anti-symmetric matrix. Formally, you have to interpret it as a vector in $\eR^4$, but matrix realization is fruitful to interpret the sequel.

Now the map $(df_v)_{\mtu}$ has to be applied to a tangent vector at point $\mtu$, so one computes (we set $v=(v_x,v_y)$)
\begin{equation}
\begin{split}
   (df_v)_{\mtu}
\begin{pmatrix}
0&a\\-a&0
\end{pmatrix}
&=
\Dsdd{ f_v
\begin{pmatrix}
\cos at&\sin at\\
-\sin at&\cos at
\end{pmatrix}}{t}{0}\\
&=\Dsdd{ 
\begin{pmatrix}
  v_x\cos at+v_y\sin at\\
-v_x\sin at+v_y\cos at
\end{pmatrix}
  }{t}{0}\\
&=a
\begin{pmatrix}
v_y\\-v_x
\end{pmatrix}.
\end{split}
\end{equation}
 
If you apply a ``very little'' rotation on the vector 
$
\begin{pmatrix}
v_x\\v_y
\end{pmatrix}$, the displacement is given by the vector $\begin{pmatrix}
v_y\\-v_x
\end{pmatrix}$.


\end{corrige}
