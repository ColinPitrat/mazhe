% This is part of the Exercices et corrigés de CdI-2.
% Copyright (C) 2008, 2009,2017-2018
%   Laurent Claessens
% See the file fdl-1.3.txt for copying conditions.


\begin{corrige}{_II-1-11}

Nous suivons les méthodes et notations du point \ref{SubSecEqDiffExacte}. 

\begin{enumerate}
\item 
$y'(t+y^2)+(y-t^2)=0$.

Ici, nous avons
\begin{equation}
	\begin{aligned}[]
		P(t,y)	&=y-t^2\\
		Q(t,y)	&=t+y^2.
	\end{aligned}
\end{equation}
Nous vérifions que l'équation est bien exacte : $\partial_yP=\partial_tQ=1$. Nous pouvons donc nous lancer à la recherche d'une fonction $f(t,y)$ telle que $df=Pdt+Qdy$. Nous pouvons la deviner : en intégrant $P$ par rapport à $t$, nous avons $f(t,y)=yt-\frac{ t^3 }{ 3 }+C$ où $C$ est une constante \emph{par rapport à $t$}. Donc $C$ peut dépendre de $y$. Il n'est pas très difficile de fixer $C(y)$ pour que $\partial_yf=Q$. Nous trouvons
\begin{equation}		\label{EqII111FonctionIntf}
	f(t,y)=\frac{1}{ 3 }(y^3-t^3)+ty.
\end{equation}
Un simple calcul montre que cette fonction fonctionne.

Ce résultat peut également être calculé en intégrant la forme $Pdt+Qdy$ le long d'un chemin qui relie $(0,0)$ à $(t,y)$. En tant que chemin, nous choisissons la composée de 
\begin{equation}
	\begin{aligned}[]
		\gamma_1(u)	&=(0,uy)\\
		\gamma_2(u)	&=(ut,y).
	\end{aligned}
\end{equation}
Nous trouvons, en appliquant la définition d'une intégrale de forme différentielle\footnote{Juste pour être sûr que vous ayez compris : vous savez la différence entre une  forme et une forme \emph{différentielle} ?} sur un chemin, nous trouvons
\begin{equation}
	\begin{aligned}[]
		f(t,y)	&=\int_{\gamma_1\circ\gamma_2}(Pdt+Qdy)\\
			&=\int_0^1(Pdt+Qdy)\big( \gamma_1(u) \big)\big( \gamma_1'(u) \big)du\\&\quad +\int_0^1(Pdt+Qdy)\big( \gamma_2(u) \big)\big( \gamma_2'(u) \big)du\\
			&=\int_0^1\big( (P\circ\gamma_1)(u)dt+(Q\circ\gamma_1)(u)dy \big)\begin{pmatrix}
	0	\\ 
	y	
\end{pmatrix}\\
			&\quad +\int_0^1\big( (P\circ\gamma_2)(u)dt+(Q\circ\gamma_2)(u)dy \big)\begin{pmatrix}
	t	\\ 
	0	
\end{pmatrix}\\
		&=\int_0^1yQ(0,uy)du+\int_0^1tP(ut,y)du\\
		&=\frac{ y^3 }{ 3 }+ty-\frac{ t^3 }{ 3 }.
	\end{aligned}
\end{equation}
Voilà qui confirme la formule \eqref{EqII111FonctionIntf}. Nous avons donc maintenant $Q=\frac{ \partial f }{ \partial y }$ et $P=\frac{ \partial f }{ \partial t }$. La solution de l'équation différentielle est donc donnée par l'équation implicite
\begin{equation}
	\frac{1}{ 3 }(y^3-t^3)+ty=C.
\end{equation}



\item
$\big(y+ty+\sin(y)\big)dt+\big(t+\cos(y)\big)dy=0$.

Nous avons
\begin{equation}
	\begin{aligned}[]
		P(t,y)	&=y+ty+\sin(y)\\
		Q(t,y)	&=t+\cos(y).
	\end{aligned}
\end{equation}
Un rapide calcul montre que cette équation n'est pas exacte. Nous faisons donc le coup du facteur intégrant \eqref{EqDuFacteurIntegrant} pour la rendre exacte. L'équation pour $M(t,y)$ est
\begin{equation}
	M\big( t+\cos(y) \big)=\big( t+\cos(y) \big)\frac{ \partial M }{ \partial t }-\big(y+ty+\sin(y)\big)\frac{ \partial M }{ \partial y }.
\end{equation}
Une très mauvaise idée serait d'essayer de la résoudre en général : tout ce dont nous avons besoin est d'une solution à cette équation. Étant donné que le coefficient devant $M$ est égal à celui devant $\partial_tM$, ce serait bien que le terme en $\partial_yM$ soit nul. Cherchons donc une solution qui ne soit pas fonction de $y$, mais seulement de $t$. Dans ce cas, nous trouvons à résoudre
\begin{equation}
	M\big( t+\cos(y) \big)=\big( t+\cos(y) \big)\frac{ \partial M }{ \partial t },
\end{equation}
c'est à dire $M=\partial_tM$. Une solution est $M(t,y)= e^{t}$.

Nous regardons donc maintenant l'équation différentielle
\begin{equation}
	e^t\big(y+ty+\sin(y)\big)dt+e^t\big(t+\cos(y)\big)dy=0,
\end{equation}
qui est exacte (vous pouvez le vérifier, si vous ne faites pas confiance en le calcul que nous venons de faire). La fonction qui intègre cette forme est
\begin{equation}
	f(t,y)=e^t\big( \sin(y)+yt \big).
\end{equation}
La solution implicite au problème est donc
\begin{equation}
	e^t\big( \sin(y)+yt \big)=C.
\end{equation}



\end{enumerate}


\end{corrige}
