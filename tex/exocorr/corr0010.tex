% This is part of Exercices et corrigés de CdI-1
% Copyright (c) 2011,2015
%   Laurent Claessens
% See the file fdl-1.3.txt for copying conditions.

\begin{corrige}{0010}

\begin{enumerate}
\item 
\item Soient $M>1$ et $K,L$ tels que $k>K$ et $l>L$ impliquent $x_k>M$ et $y_k>M$. Dans ce cas, $m>\max\{K,L\}$ implique $(xy)_m>M^2>M$.
\item Soient $M>0$, $K$ tel que $k>K$ implique $x_k>M+1$ et $L$ tel que $l>L$ implique $| z_l-a |<\frac{ 1 }{2}$. Alors $m>\max\{ K,L \}$ implique $(x+z)_k>M+1\pm\frac{ 1 }{2}>M+\frac{ 1 }{2}>M$.
\item Soient $M>0$ et $K$ tels que $k>K$ implique $x_k>\frac{ M }{ a }+1$, et $z_k>a-\epsilon$. Un tel $k$ peut être trouvé pour tout choix de $\epsilon$. Nous choisissons $\epsilon$ de façon à avoir $a-\epsilon>0$, et nous minorons
\begin{equation}
	x_kz_k>\left( \frac{ M }{ a }+1 \right)(a-\epsilon)=M+a-\frac{ \epsilon M }{ a }-\epsilon.
\end{equation}
Si nous prenons $\epsilon$ assez petit pour que $a-\frac{ \epsilon M }{ a }-\epsilon$, nous trouvons $x_kz_k>M$.

\item Soit $M<0$, il faut trouver un $K$ tel que $-x_k<M$ dès que $k>K$. Évidement, le $K$ tel que $x_k>-M$ fonctionne.

\end{enumerate}

\end{corrige}
