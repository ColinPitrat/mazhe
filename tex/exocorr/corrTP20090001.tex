% This is part of Exercices et corrigés de CdI-1
% Copyright (c) 2011-2012
%   Laurent Claessens
% See the file fdl-1.3.txt for copying conditions.

\begin{corrige}{TP20090001}

	\begin{enumerate}

		\item
			Un connexe dans $\eR$ est un intervalle. Prouvons d'abord que si $f$ est dérivable sur $\mathopen] a , b \mathclose[$ avec $f'(x)=0$ pour tout $x\in \mathopen] a , b \mathclose[$, alors $f$ est constante sur $\mathopen] a , b \mathclose[$. En effet, si $x,y\in\mathopen] a , b \mathclose[$, alors le théorème des accroissements finis (page 184) montre qu'il existe un $c\in\mathopen] x , y \mathclose[$ tel que
			\begin{equation}
				f(y)-f(x)=f'(c)(y-x).
			\end{equation}
			Étant donné que $f'(c)=0$, nous en déduisons que $f(x)=f(y)$.

			La propriété est donc prouvée dans le cas où $A$ est un intervalle ouvert. Si l'ensemble $A$ est un intervalle de la forme $\mathopen[ a , b \mathclose]$, alors on peut utiliser la continuité pour prouver que si $f$ est continues en $a$ et $b$ et constante sur $\mathopen] a , b \mathclose[$, alors elle est constante sur $\mathopen[ a , b \mathclose]$. Pour s'en convaincre, supposons que $f(x)=y_0$ sur $\mathopen] a , b \mathclose[$ et que $f(a)=y_1$. Dans ce cas, étant donné que $f$ est continue en $a$, il existe un voisinage de $a$ dans lequel $f$ prend ses valeurs uniquement dans $B(y_1,\epsilon)$. Si nous choisissons $\epsilon$ de telle manière à exclure $y_0$ de la boule (ce qui est possible si $y_1\neq y_0$), alors nous contredisons la continuité de $f$ en $a$ parce que tout voisinage de $a$ contient un point de $\mathopen] a , b \mathclose[$ où $f$ vaut $y_0$.

		\item
			Considérons $a_0$, un point par rapport auquel $\Omega$ est étoilé. Soit $a\in\Omega$. Considérons le segment de droite $\gamma\colon [0,1]1\to \Omega$ tel que $\gamma(0)=a_0$ et $\gamma(1)=a$. Nous considérons maintenant la fonction
			\begin{equation}
				\begin{aligned}
					g\colon [0,1]&\to \eR \\
					t&\mapsto (f\circ \gamma)(t). 
				\end{aligned}
			\end{equation}
			Cette fonction sert à ramener le problème sur $\Omega$ à un problème sur une droite. Nous avons que 
			\begin{equation}
				g'(t)=\frac{ dg }{ dt }(t)=\frac{ d }{ dt }\left[ f\big( \gamma(t) \big) \right]=\sum_i\frac{ \partial f }{ \partial x_i }\big( \gamma(t) \big)\cdot\frac{ \partial \gamma_i }{ \partial t }(t).
			\end{equation}
			Par hypothèse, $df=0$, ce qui veut dire que $\frac{ \partial f }{ \partial x_i }(p)=0$ pour tout $p\in\Omega$, et en particulier
			\begin{equation}
				\frac{ \partial f }{ \partial x_i }\big( \gamma(t) \big)=0
			\end{equation}
			pour tout $t\in\mathopen[ 0 , 1 \mathclose]$. Donc nous avons prouvé que $g'(t)=0$, et donc que $g$ est une constante sur $[0,1]$. Pour tout $a\in\Omega$ nous avons donc que $f(a)=f(a_0)$. Cela finit de prouver que $f$ est une constante.

	\end{enumerate}

\end{corrige}
