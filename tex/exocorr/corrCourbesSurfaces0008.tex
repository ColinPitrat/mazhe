\begin{corrige}{CourbesSurfaces0008}

	\begin{enumerate}
		\item
				
			Nous reprenons les indications des équations \eqref{EqFomVPcogammaN} et \eqref{EqFomVPcoordnorm} pour construire les coordonnées normales. Vu que $\gamma(0)=A$, nous pouvons faire commencer l'intégrale qui définit $\phi$ en zéro :
			\begin{equation}
				\phi(t)=\int_0^t\| f'(u) \|du=\int_0^t Rdu=tR.
			\end{equation}
			Le point $f(t)$ a donc pour abscisse curviligne le nombre $Rt$. Nous pouvons aller plus loin en écrivant complètement la paramétrisation. D'abord, $\phi^{-1}(s)=\frac{ s }{ R }$, par conséquent
			\begin{equation}
				\gamma_N(s)=(\gamma\circ\phi^{-1})(s)=\gamma(\frac{ s }{ R })=\left( R\cos(\frac{ s }{ R }),R(\sin\frac{ s }{ R }) \right).
			\end{equation}
			Ensuite il faut savoir la longueur de la courbe pour savoir sur quel intervalle varie le paramètre $s$. Cette longueur est donné par $l(f)$ qui se calcule par
			\begin{equation}
				l(f)=\int_0^{2\pi}\| f'(t) \|dt=R\int_0^{\pi}1dt=2\pi R.
			\end{equation}
	
			À partir de maintenant, nous passons de façon simple des coordonnées $f$ (c'est-à-dire les radians) aux coordonnées normales par l'égalité $f(t)=\gamma_N(Rt)$. Cela est une égalité dans $\eR^2$.

		\item	
			En ce qui concerne le vecteur tangent, c'est le corollaire \ref{CorTgSoCun} (et même en fait le \ref{CorUnitTgtaugpnorma}) qui nous indique la marche à suivre. Notez que lorsqu'on demande quelque chose au point $f(t)$, nous pouvons faire nos calculs avec les coordonnées normales en considérant le point $\gamma_N(Rt)$ (qui est le même). Quoi qu'il en soit, nous allons utiliser les coordonnées $f$ et calculer 
			\begin{equation}
				\tau(t)=\frac{ f'(t) }{ \| f'(t) \| }=\big( -\sin(t),\cos(t) \big).
			\end{equation}
			Pour le reste, c'est la définition \ref{DefCourbureNormleUnit} qui joue. Nous devons donc travailler en coordonnées normales. Nous posons donc $s=Rt$ et nous calculons tous les éléments au point $\gamma_N(s)$. Nous avons
			\begin{subequations}
				\begin{align}
					\gamma_N'(s)&=\left( -\sin(\frac{ s }{ R }),\cos(\frac{ s }{ R }) \right)\\
					\gamma_N''(s)&=\frac{1}{ R }\left( -\cos(\frac{ s }{ R }),-\sin(\frac{ s }{ R }) \right).
				\end{align}
			\end{subequations}
			Donc
			\begin{equation}
				\nu(s)=-\left( \cos(\frac{ s }{ R }),\sin(\frac{ s }{ R }) \right).
			\end{equation}
			En terme des coordonnées $f(t)$, nous avons $\nu(t)=-\left( \cos(t),\sin(t) \right)$.
			
			La courbure du cercle en $\gamma_N(s)$ vaut $c(s)=\| \gamma_N''(s) \|=\frac{1}{ R }$, et son rayon de courbure vaut alors $R$ \ldots ce n'est peut-être pas pour rien que cela se nomme «rayon de courbure».

		\item
			Le vecteur $\nu(s)$ pointe vers l'intérieur du cercle. Il est même exactement l'opposé du rayon partant du centre.

			Si nous prenons la paramétrisation $g$, il ne se passe rien pour la direction de la courbure. En effet, le vecteur tangent change de signe, mais lorsque nous prenons la dérivée seconde, le signe moins sort deux fois et par conséquent le vecteur normal $\nu$ continue à pointer vers l'intérieur.

	\end{enumerate}

\end{corrige}
