% This is part of Exercices et corrigés de CdI-1
% Copyright (c) 2011,2015
%   Laurent Claessens
% See the file fdl-1.3.txt for copying conditions.

\begin{corrige}{TP20090002}

	Étant donné que $T^p$ est contractante, elle possède un unique point fixe. Si $x_0$ est un point fixe de $T$, alors il est point fixe de $T^p$, en effet, pour tout $k>1$, nous avons
	\begin{equation}
		T^k(x_0)=T^{k-1}T(x_0)=T^{k-1}(x_0).
	\end{equation}
	Donc le point fixe de $T^p$ est l'unique candidat point fixe de $T$. Montrons que si $a$ est le point fixe de $T^p$, alors $a$ est aussi un point fixe de $T$. Par définition,
	\begin{equation}
		T^p(a)=a.
	\end{equation}
	En appliquant $T$ des deux côtés et en tenant compte du fait que $T\circ T^k=T^k\circ T$, nous avons
	\begin{equation}
		T^p\big( T(a) \big)=T(a),
	\end{equation}
	ce qui prouve que $T(a)$ est lui-même un point fixe de $T^p$. Mais le point fixe de $T^p$ étant unique, cela implique que $T(a)=a$, et donc que $a$ est un point fixe de $T$.

	En ce qui concerne les exemples, une multitude d'exemples sont possibles. Le plus facile est de prendre une application $T\colon \eR^2\to \eR^2$ qui envoie tous les points de $\eR^2$ sur $(0,0)$, sauf quelques uns qui vont vers un autre point, en choisissant cet autre point de telle manière que $T$ l'envoie sur $(0,0)$. De cette manière $T$ n'est pas une contraction parce qu'elle n'est pas continue, et $T^2$ est une contraction parce que $T^2=0$.

\end{corrige}
