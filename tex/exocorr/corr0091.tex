% This is part of Exercices et corrigés de CdI-1
% Copyright (c) 2011
%   Laurent Claessens
% See the file fdl-1.3.txt for copying conditions.

\begin{corrige}{0091}

Le théorème des accroissements finis nous dit qu'entre $x$ et $y$, il existe un $c$ tel que
\begin{equation}
	f(x)-f(y)=f'(c)(x-y)<M(x-y)
\end{equation}
si $M$ est une majoration de $f'$ sur $\mathopen]a,b\mathclose[$.

Si maintenant on me donne un $\epsilon$, il me suffit de prendre $\delta<\frac{ \epsilon }{ M }$, et j'ai
\begin{equation}
	| f(x)-f(y) |<\epsilon
\end{equation}
pour tout $x$ et $y$ tels que $| x-y |<\delta$.

\end{corrige}
