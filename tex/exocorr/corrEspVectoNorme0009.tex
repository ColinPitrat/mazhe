\begin{corrige}{EspVectoNorme0009}

	L'identification entre les vecteurs et les matrices consiste simplement à «déplier» la matrice pour en faire un vecteur. Par exemple, en dimension deux,
	\begin{equation}
		\begin{pmatrix}
			1	&	2	\\ 
			3	&	4	
		\end{pmatrix}\mapsto
		\begin{pmatrix}
			1	\\ 
			2	\\ 
			3	\\ 
			4	
		\end{pmatrix}\in\eR^4.
	\end{equation}
	En dimension $3$,
	\begin{equation}
		\begin{aligned}[]
			\begin{pmatrix}
				1	&	2	&	3	\\
				4	&	5	&	6	\\
				7	&	8	&	9
			\end{pmatrix}
			\mapsto
			\begin{pmatrix}
				1	\\ 
				2	\\ 
				3	\\ 
				4	\\ 
				5	\\ 
				6	\\ 
				7	\\ 
				8	\\ 
				9	
			\end{pmatrix}\in\eR^9.
		\end{aligned}
	\end{equation}
	
	Une matrice est inversible si et seulement si son déterminant est non nul. Or le déterminant est un polynôme en les composantes de la matrice. En dimension deux, nous avons
	\begin{equation}
		\det\begin{pmatrix}
			a	&	b	\\ 
			c	&	d	
		\end{pmatrix}=ad-bc,
	\end{equation}
	mais en écriture «dépliée», nous pouvons aussi bien écrire
	\begin{equation}
		\det\begin{pmatrix}
			a	\\ 
			b	\\ 
			c	\\ 
			d	
		\end{pmatrix}=ad-bc.
	\end{equation}
	En dimension $3$, le déterminant est donc un polynôme des $9$ variables qui apparaissent dans le vecteur «déplié». En général, dans $\eR^{n^2}$, nous considérons donc le polynôme $\det\colon \eR^{n^2}\to \eR$ qui à un vecteur $X\in\eR^{n^2}$ fait correspondre le déterminant de la matrice obtenue en «repliant» le vecteur $X$.

	Donc dans $\eR^{n^2}$, l'ensemble des matrices inversibles est donné par l'ensemble des vecteurs sur lesquels le polynôme $\det$ ne s'annule pas, c'est-à-dire
	\begin{equation}
		\{ X\in\eR^{n^2}\tqs \det(X)\neq 0 \}.
	\end{equation}
	Mais le déterminant est un polynôme, et donc une fonction continue. Cet ensemble est par conséquence ouvert par le corolaire \ref{CorfneqzOuvert}.

\end{corrige}
