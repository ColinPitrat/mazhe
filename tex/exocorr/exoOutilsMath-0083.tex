% This is part of Exercices et corrigés de CdI-1
% Copyright (c) 2011
%   Laurent Claessens
% See the file fdl-1.3.txt for copying conditions.

\begin{exercice}\label{exoOutilsMath-0083}

    Un vent se dirige vers le sud, mais la force de Coriolis le détourne un peu vers l'est. En termes de coordonnées sphériques sur la Terre (de rayon $R$), le vent est donné par
    \begin{equation}
        u(\rho,\theta,\varphi)=\sin(\varphi)e_{\theta}+e_{\varphi}.
    \end{equation}
    \begin{enumerate}
        \item
            Calculer $\nabla\cdot u$ et $\nabla\times u$.
        \item
            Est-ce qu'il existe une fonction $p(\rho,\theta,\varphi)$ telle que $u=\nabla p$ ?
    \end{enumerate}

\corrref{OutilsMath-0083}
\end{exercice}
