% This is part of Analyse Starter CTU
% Copyright (c) 2014
%   Laurent Claessens,Carlotta Donadello
% See the file fdl-1.3.txt for copying conditions.

\begin{corrige}{starterST-0018devoir}

  \begin{enumerate}
  %\item[(1)] Par parties : $F_{1}(x)=\displaystyle\int x(x^2+3) \,dx = x\left(\frac{x^3}{3} + 3x\right) - \int \frac{x^3}{3} + 3x \, dx = \frac{x^4}{4} + \frac{3x^2}{2} + C $ . 
   % À l'aide du changement de variable $ u = x^2+3$ : on a $du = 2x  dx$ et donc $F_{1}(x)=\displaystyle\int x(x^2+3) \,dx =\frac{1}{2} \int u \, du = \frac{u^2}{4} + C = \frac{1}{4}\left(x^4 + 6 x^2 + 9 \right) + C = \frac{x^4}{4} + \frac{3x^2}{2} + C $. Bien entendu, la constante $C$ a absorbé le $\frac{9}{4}$ dans le dernier passage.   
  %\item[(2)] À l'aide du changement de variable $ u = x^2+1$ : on a $du = 2x  dx$ et donc $F_{2}(x)=\displaystyle\int x\sqrt{1+x^2} \,dx = \frac{1}{2} \int \sqrt{u} \, du  = \frac{1}{3} u^{3/2} + C = \frac{1}{3} (x^2+1)^{3/2} + C  $.
 % \item[(3)] À l'aide du changement de variable $ u = x^3+1$ : on a $du = 3x^2  dx$ et donc $F_{3}(x)=\displaystyle\int \dfrac{x^2}{1+x^3} \,dx = \frac{1}{3} \int \frac{1}{u} \, du = \frac{1}{3}\ln(|u|) +C  = \frac{1}{3}\ln(|x^3 + 1|) +C  $. 
  %\item[(4)] Par parties : $F_{4}(x)=\displaystyle\int x e^x  \,dx = x e^x - \int e^x  \,dx = (x-1) e^x + C$. 
  %\item[(5)] Il faudra utiliser ici l'intégration par parties (comme dans \ref{primln}) et ensuite un changement de variable. On commence par calculer $F_{5}(x)=\displaystyle\int 1 \times \arcsin (x) \,dx  =  x\arcsin(x) - \int \frac{x}{\sqrt{1-x^2}} \, dx $. Ensuite on pose $u = 1-x^2$ et on obtient $F_{5}(x)=\displaystyle x\arcsin(x) +\frac{1}{2} \int \frac{1}{\sqrt{u}} \, du  = x\arcsin(x) +\sqrt{1-x^2} + C$. 
 % \item[(6)]  Il faudra utiliser ici l'intégration par parties deux fois de suite. $F_{6}(x)=\displaystyle\int x^2 e^x  \,dx = x^2 e^x -2\int x e^x  \,dx = x^2 e^x -2\left(x e^x - \int e^x  \,dx\right) = e^x\left(x^2-2x +2\right) +C$.  
 % \item[(7)] Cette primitive a été calculée dans l'exemple \ref{primln}. 
  \item[(8)]  Par parties : $F_{8}(x)=\displaystyle\int\arctan (x) \,\mathrm dx = x\arctan (x)-\int \frac{x}{x^2+1}\,dx = x\arctan (x)-\frac{1}{2}\ln(x^2+1) + C$. 
  \item[(9)]  Par parties : $F_{9}(x)=\displaystyle\int x\sin (x) \, dx = -x\cos(x) + \int\cos(x)\, dx = -x\cos(x) +\sin(x) + C$.  
  %\item[(10)] Le calcul se fait, comme dans l'exemple \ref{exemplepassagepolaires}, en s'appuyant sur des formules de trigonométrie. 

   % Première possibilité : $F_{10}(x)=\displaystyle\int\sin^2 (x)\,dx = \int \frac{1-\cos(2x)}{2} \, dx = \frac{1}{2}x - \frac{\sin(2x)}{4} + C$. 

    %Deuxième possibilité (avec une intégration par parties d'abord) : on écrit $F_{10}(x)=\displaystyle\int\sin^2 (x)\,dx = -\cos(x)\sin(x) + \int \cos^2(x) \, dx $ et ensuite on remarque que en intégrant une deuxième fois par parties on n'avance pas du tout. On utilise alors la formule $1 = \cos^2(x) + \sin^2(x)$ et on a  $F_{10}(x)=\displaystyle -\cos(x)\sin(x) + \int 1 - \sin^2(x) \, dx $.   On a alors que $\displaystyle\int\sin^2 (x)\,dx = \frac{1}{2} \left(-\cos(x)\sin(x)  + x\right) + C$. Le résultat est le m\^eme qu'on a obtenu ci-dessus, parce que $\sin(2x) = 2 \cos(x)\sin(x)$. 
  \item[(11)]  Par parties on a  
    \[
    F_{11}(x)=\displaystyle\int\dfrac{\ln(x)}{x}\, dx = \ln^2(x) -\int\frac{\ln(x)}{x}\, dx,
    \]
d'où on déduit que $F_{11}(x)=\frac{1}{2} \ln^2(x) +C$.
  %\item[(12)] À l'aide du changement de variable $ u = x^2$ : on a $du = 2x  dx$ et donc $F_{12}(x)=\displaystyle\int\dfrac{x}{1+x^4}\, dx = \frac{1}{2}\int \frac{1}{1 + u^2} \, du  = \frac{1}{2} \arctan(u) + C = \frac{1}{2} \arctan(x^2) + C  $. 
  \item[(13)] On observe que $F_{13}(x)=\displaystyle\int\dfrac{1}{x^2+4x+5}\, dx = \int\dfrac{1}{(x+2)^2+1}\, dx$. Le changement de variable $t = x+2$ nous donne  $\displaystyle F_{13}(x)=\int \frac{1}{t^2 + 1}\, dt = \arctan(t) + C = \arctan(x+2) + C$.
  %\item[(14)] Pour calculer cette primitive on voudrait écrire convenablement le polyn\^ome au dénominateur de $\displaystyle \dfrac{8x}{x^2+4x+5}$. On peut d'abord essayer d'en chercher les racines, mais une fois vu qu'elle ne sont pas réelles  nous devons abandonner l'espoir de décomposer la fonction rationnelle. Ici nous  pouvons observer que le numérateur est de degré 1 et le dénominateur de degré 2. Il faut donc chercher à trouver une fonction à intégrer (par changement de variable) de la forme $u'/u$. La dérivée de $x^2+4x+5 $ et $2x + 4$ donc $F_{14}(x)=4 \displaystyle\int\dfrac{2x + 4}{x^2+4x+5} - \dfrac{4}{x^2+4x+5}\,dx = 4\ln\left(|x^2+4x+5|\right) -16 \int\dfrac{1}{x^2+4x+5} \, dx $. L'intégrale qui reste n'est pas encore banale. L'astuce standard  dans ce cas consiste à écrire $x^2+4x+5$ dans la forme $1 +(\text{un carré})$ et ensuite utiliser un changement de variable pour se ramener à la dérivée de la fonction $\arctan$. En somme $\displaystyle  \int\dfrac{1}{x^2+4x+5} \, dx  =  \int\dfrac{1}{1 + (x+2)^2} \, dx $ et nous pouvons le calculer facilement à l'aide du changement de variable $ u = x+2$ :  $\displaystyle  \int\dfrac{1}{x^2+4x+5} \, dx = \int \frac{1}{1+u^2} \, du = \arctan(u) + C = \arctan(x+2) + C $. Finalement, $F_{14}(x)=4\ln\left(|x^2+4x+5|\right) -16\arctan(x+2) + C $.
  \item[(15)] Le changement de variable conseillé nous donne $F_{15}(x)=\displaystyle\int \dfrac{1}{e^{1-t}\sqrt{t}}\,(-e^{1-t}) dt = -\int \dfrac{1}{\sqrt{t}}\, dt = -2\sqrt{t} + C = -2\sqrt{1-\ln(x)} + C$.

  \end{enumerate}
\end{corrige}
