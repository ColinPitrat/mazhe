% This is part of Exercices et corrections de MAT1151
% Copyright (C) 2010
%   Laurent Claessens
% See the file LICENCE.txt for copying conditions.

\begin{exercice}\label{exoexamens-0000}

	On définit les problèmes suivants :
	\begin{equation}
		F_1(x_1,d_1)=x_1-\frac{1}{ 4 }\sin^2(d_1)=0,
	\end{equation}
	et
	\begin{equation}
		F_2(x_2,d_2)=x_2^2-x_2+d_2
	\end{equation}
	pour $x_2\geq \frac{1}{ 4 }$ et $d\in\mathopen[ 0,\frac{1}{ 4 } \mathclose]$. Nous considérons alors le problème composite
	\begin{equation}	\label{EqProbCompoExamz}
		F(x,d)=F_2(x,x_1(d))=0
	\end{equation}
	où $x_1(d)$ est solution du premier problème.

	\begin{enumerate}

		\item
			Le problème composite \eqref{EqProbCompoExamz} est-il stable ?
		\item
			Exprimer le conditionnement relatif du problème composite du problème composite en fonction de ceux des deux problèmes.
		\item
			Pour quelles valeurs de $d$ le problème \eqref{EqProbCompoExamz} est-il bien conditionné ?

	\end{enumerate}

\corrref{examens-0000}
\end{exercice}
