% This is part of Agregation : modélisation
% Copyright (c) 2011
%   Laurent Claessens
% See the file fdl-1.3.txt for copying conditions.

\begin{corrige}{Model-0002}

    Lorsque nous parlons de paramètres qui peuvent prendre un spectre continu de valeurs, il est inutile de calculer la \emph{probabilité} parce qu'elle est nulle. Le système de maximum de vraisemblance fonctionne avec les densités. Dans notre cas, la fonction de vraisemblance est le produit des densités :
    \begin{subequations}
        \begin{align}
            L(x_1,\ldots,x_n;\theta)&=\prod_{i=1}^np(x_i;\theta)\\
            &=\prod_i\frac{1}{ \theta }\mtu_{\mathopen[ 0, \theta \mathclose]}(x_i)\\
            &=\frac{1}{ \theta^n }\mtu_{\mathopen[ 0 , \theta \mathclose]^n}(x_1,\ldots,x_n).
        \end{align}
    \end{subequations}
    De cette expression nous voyons que \( \min\{ x_1,\ldots,x_n \}\) doit être positif en même temps que le maximum doit être plus petit que \( \theta\). Cette seconde condition peut s'écrire \( \mtu_{\mathopen[ \max\{ x_1,\ldots,x_n \} , \infty [}(\theta)\). Au final nous avons
    \begin{equation}
        L(x_1,\ldots,x_n;\theta)=\frac{1}{ \theta^n }\mtu_{\mathopen[ 0 , \infty [}\big( \min(x_1,\ldots,x_n) \big)\mtu_{\mathopen[ \max\{ x_1,\ldots,x_n \} , \infty [}(\theta).
    \end{equation}
    
    Il n'est évidemment pas possible de dériver explicitement cette expression. Par contre pour cette fonction soit non nulle, il faut obligatoirement \( \theta\geq\max\{ x_1,\ldots,x_n \}\). Par conséquent elle prend son maximum pour \( \theta=\max\{ x_1,\ldots,x_n \}\).

    La conclusion est que l'estimateur de maximum de vraisemblance de \( \theta\) est \( \hat\theta_n=\max_i\{ X_i \}\).

\end{corrige}
