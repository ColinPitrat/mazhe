% This is part of Exercices et corrigés de CdI-1
% Copyright (c) 2011,2014
%   Laurent Claessens
% See the file fdl-1.3.txt for copying conditions.

\begin{corrige}{Variete0015}

Le cornet de glace $S$ délimite un volume fermé $V$ constitué d'un demi-ellipsoïde $E$ et d'un bout de cône $C$. Notons que les deux bouts se recollent bien en $y=0$ parce que sur ce plan, l'ellipsoïde et le cône ont tous les deux pour équation  $x^2+y^2=16$.

Le théorème de la divergence s'applique, et l'intégrale recherchée vaut
\begin{equation*}
  \iint_S G \cdot d S = \iiint_E y + \iiint_C y
\end{equation*}
puisque $\nabla\cdot G = y$.

Pour calculer l'intégrale sur le demi-ellipsoïde, prenons des coordonnées sphériques un peu modifiées~:
\begin{equation*}
  \begin{cases}
    x = 4 r \cos \phi \sin \theta\\
    y = 3 r \cos \theta\\
    z = 4 r \sin \phi \sin \theta
  \end{cases}
\end{equation*}
dont le jacobien vaut $48 r^2 \sin \theta$. L'intégrale vaut\footnote{Pouvez vous justifier le fait que le rayon va de $0$ à $1$ ?}
\begin{equation*}
  \iiint_E y = %
  \int_0^{2\pi} \int_0^{\frac\pi2} \int_0^1 144 r^3 \sin\theta\cos\theta%
  d r d \theta d \phi =%
  144 \frac 14 \frac 12 2 \pi = 36 \pi.
\end{equation*}
Afin de calculer l'intégrale sur le cône, nous utilisons les coordonnées cylindriques~:
\begin{equation*}
  \begin{cases}
    x = \rho \cos t\\
    y = \tilde y - 4\\
    z = \rho \sin t
  \end{cases}
\end{equation*}
dont le jacobien vaut $\rho$. L'intégrale vaut
\begin{equation*}
  \iiint_C y = \int_0^{2\pi} \int_0^4 \int_0^{\tilde y} \tilde y - 4
  d\rho d\tilde yd t = \frac{- 64 \pi}{3}
\end{equation*}

L'intégrale recherchée au départ vaut donc \begin{math}
  \frac {44 \pi}3
\end{math}.

\end{corrige}
