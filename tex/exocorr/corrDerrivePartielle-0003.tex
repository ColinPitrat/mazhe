% This is part of the Exercices et corrigés de mathématique générale.
% Copyright (C) 2010
%   Laurent Claessens
% See the file fdl-1.3.txt for copying conditions.

\begin{corrige}{DerrivePartielle-0003}

	\begin{enumerate}

		\item
			La formule de dérivation des fonctions composées, dans ce cas, devient
			\begin{equation}
				\frac{ \partial w }{ \partial x }=\frac{ \partial w }{ \partial u }\big( u(x,y),v(x,y) \big)\frac{ \partial u }{ \partial x }(x,y)+\frac{ \partial w }{ \partial v }\big( u(x,y),v(x,y) \big)\frac{ \partial v }{ \partial x }(x,y).
			\end{equation}
			Les calculs donnent
			\begin{equation}
				2x\cos(v)-u\sin(v)=2x\cos(x,y)-(x^2+y^2)\sin(xy).
			\end{equation}

			En ce qui concerne la dérivée dans le sens de $y$ nous avons
			\begin{equation}
				\frac{ \partial w }{ \partial y }(x,y)=-(x^2 + y^2)x\sin(xy) + 2y\cos(xy)
			\end{equation}
		\item
			Ici c'est le même jeu, les réponses sont
			\begin{equation}
				\begin{aligned}[]
					\frac{ \partial w }{ \partial x }&=2xy\sin(y)\cos(x) + 6y^2\sin(x)\cos(x) + 2y\sin(x)\sin(y)\\
					\frac{ \partial w }{ \partial y }&=2xy\sin(x)\cos(y) + 2x\sin(x)\sin(y) + 6y\sin^2(x)
				\end{aligned}
			\end{equation}

	\end{enumerate}

\end{corrige}
