% This is part of Exercices et corrigés de CdI-1
% Copyright (c) 2011
%   Laurent Claessens
% See the file fdl-1.3.txt for copying conditions.

\begin{corrige}{0076}

Quelque notes
\begin{enumerate}
\item 
\item 
\item 
\item 
\item 
\item 
\item 
\item 
\item 
\item 
Le point $(0,0)$ est dans l'ensemble, et donc dans l'adhérence. L'adhérence peut donc être notée indifféremment $E \cup \mathopen[0,1\mathclose]\times\{0\}$ ou $E \cup \mathopen]0,1\mathclose]\times\{0\}$.

\item 
Dans ce cas ci, par contre, le point $(0,0)$ n'est pas dans l'ensemble, donc l'adhérence doit être écrite $E \cup \mathopen[0,1\mathclose]\times\{0\}$.
\item 
\end{enumerate}

Parmi les propriétés d'être ouvert, fermé, borné, compact ou connexe par arcs (c.p.a), les ensembles cités (toujours noté $E$) jouissent des propriétés mentionnées mais pas des autres.

sdf

\begin{landscape}
    sdfds
%\begin{bigcenter}
  \begin{tabular}{|l|*{3}{|>{$}c<{$}}|l|}
     
    \hline
    Ex. & \text{Intérieur} & \text{Adhérence} & \text{Frontière} & Propriétés\\
    \hline
	\ref{Item76a} & \emptyset & E & E & fermé, c.p.a\\
	\ref{Item76b} & E & \{ 1 - xy \geq 0 \} & \{ 1 - xy = 0 \} & ouvert, c.p.a\\
	\ref{Item76c} & \emptyset & \{ |x| = 1 \} & \{ \abs{x} = 1 \} & Néant\\
	\ref{Item76d} & \{\scriptstyle (x-3)^2 + (y+2)^2 < 4 \} & E & \{\scriptstyle (x-3)^2 + (y+2)^2 = 4 \} & fermé, borné, compact, c.p.a\\
	\ref{Item76e} & \{\scriptstyle (x-1)^2 + (y+2)^2 < 10 \} & E & \{\scriptstyle (x-1)^2 + (y+2)^2 = 10 \}& fermé, borné, compact, c.p.a\\
	\ref{Item76f}  & \{ x^2+ y^2 < 3 \text{ et } \abs y < 1 \} & \{ x^2+ y^2 \leq 3 \text{ et } \abs y \leq 1 \} &
 	 	\begin{array}[t]{l}
 	 		\scriptstyle\{ x^2+ y^2 \leq 3 \text{ et } \abs y = 1 \}\\ \scriptsize\cup \{ x^2+ y^2 = 3 \text{ et } \abs y \geq 1 \}
 	 	\end{array}
 	 	& borné, c.p.a\\
    \ref{Item76g}   & \emptyset & E = \{ x^2 + y^2 = 1 \} & E & fermé, borné, compact, c.p.a\\
    \ref{Item76h}  & E = \{ x^2 + y^2 < 4 \} & \{ x^2 + y^2 \geq 4 \} & \{ x^2
    + y^2 = 4 \} & ouvert, borné, c.p.a\\
     \ref{Item76i}  & \emptyset & E & E &
    \begin{tabular}[t]{l}
      fermé, borné, compact, c.p.a\\
      C'est le segment joignant\\
      $(x_0,y_0)$ à $(x_1,y_1)$
    \end{tabular}\\
     \ref{Item76j}  & \emptyset & E \cup \mathopen[0,1\mathclose]\times\{0\} & E \cup \mathopen]0,1\mathclose]\times\{0\} & borné, c.p.a\\
     \ref{Item76k}  & \emptyset & E \cup \mathopen[0,1\mathclose]\times\{0\} & E \cup \mathopen[0,1\mathclose]\times\{0\} & borné\\
     \ref{Item76l}  & \emptyset & E \cup \{0\}\times[0,1] & E \cup \{0\}\times[0,1] & borné\\
    \hline
  \end{tabular}
%\end{bigcenter}
\end{landscape}


Pour l'ensemble \ref{Item76e}, remarquez que $x^2-2x+y^2+yy-5=(x-1)^2+(y+2)^2-10$; c'est le coup classique de reformer les carrés.

Parmi les propriétés d'être ouvert, fermé, borné ou compact, les ensembles considérés jouissent des propriétés citées et pas des
autres.
\begin{enumerate}
\item Fermé (droite).
\item Fermé (hyperbole).
\item Aucune propriété (union disjointe de deux demi-droites et de deux segments ouverts).
\item Fermé, borné, compact (disque).
\item Idem.
\item Borné (morceau de disque)
\item Fermé, borné, compact (cercle).
\item Ouvert, borné (disque ouvert de rayon 2).
\item Fermé, borné, compact (segment de droite ou point).
\item Borné (bouquet infini de segments)
\item Idem (idem sans le point $0$).
\item Borné (union disjointe d'une infinité de segments).
\end{enumerate}


\end{corrige}
