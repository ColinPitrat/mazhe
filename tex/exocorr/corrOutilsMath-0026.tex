% This is part of Exercices et corrigés de CdI-1
% Copyright (c) 2011
%   Laurent Claessens
% See the file fdl-1.3.txt for copying conditions.

\begin{corrige}{OutilsMath-0026}

	Pour que nous soyons sur la sphère de rayon $1/\sqrt{3}$, il faut certainement imposer $\rho=\frac{1}{ \sqrt{3} }$. Afin d'être à une hauteur plus haute que $\frac{ 1 }{2}$, il faut imposer que $z>\frac{1}{ 2 }$, c'est-à-dire
	\begin{equation}
		\rho\cos(\theta)>\frac{ 1 }{2},
	\end{equation}
	ou encore : $\theta<\frac{ \pi }{ 6 }$. L'équation recherchée est
	\begin{subequations}
		\begin{numcases}{}
			r=\frac{1}{ \sqrt{3} }\\
			\theta\in\mathopen[ 0 , \frac{ \pi }{ 6 } \mathclose].
		\end{numcases}
	\end{subequations}

\end{corrige}
