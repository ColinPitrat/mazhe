% This is part of the Exercices et corrigés de mathématique générale.
% Copyright (C) 2009
%   Laurent Claessens
% See the file fdl-1.3.txt for copying conditions.
\begin{exercice}\label{exoLineraire0024}

	Exercice 13, page 81. Dans $\eR^2$, on donne les bases
	\begin{equation}
		\begin{aligned}[]
			E&=\left\{
			\begin{array}{ll}
				e_1=(1,0)\\
				e_2=(0,1)
			\end{array}
			\right.
			&F&=\left\{
			\begin{array}{ll}
				f_1=(1,1)\\
				f_2=(1,2)
			\end{array}
			\right.
			&G&=\left\{
			\begin{array}{ll}
				g_1=(1,2)\\
				g_2=(2,1).
			\end{array}
			\right.
		\end{aligned}
	\end{equation}
	\begin{enumerate}

		\item
			Quelles sont les composantes du vecteur $(a,b)$ dans les bases $E$, $F$ et $G$ ?
		\item
			Quelles sont les composantes du vecteur $7e_1-2e_2$ dans les bases $F$ et $G$ ?
		\item
			Quelles sont les composantes du vecteur $f_1+3f_2$ dans les bases $E$ et $F$ ?
		\item 
			Quelles sont les composantes du vecteur $g_1$ dans les bases $E$ et $F$ ?
	\end{enumerate}

\corrref{Lineraire0024}
\end{exercice}
