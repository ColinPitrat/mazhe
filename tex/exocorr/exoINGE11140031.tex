% This is part of Un soupçon de physique, sans être agressif pour autant
% Copyright (C) 2006-2009
%   Laurent Claessens
% See the file fdl-1.3.txt for copying conditions.


\begin{exercice}\label{exoINGE11140031}

	Calculer les limites suivantes en utilisant les règles de calcul.
	\begin{enumerate}

		\item	\label{ItemexoINGE_11140031xsinusx}
			$\lim_{x\to 0} x\sin(\frac{1}{ x })$.

		\item
			$\lim_{x\to \infty} \left( \frac{ x }{ x+1 } \right)^{x+2}$.

		\item	\label{ItemexoINGE_11140031}
			$\lim_{x\to 0} \frac{ \sin(\alpha x) }{ \sin(\beta x) }$.

			Pour effectuer cet exercice avec Sage, il faut taper les lignes suivantes~:
			\begin{enumerate}

				\item
					\texttt{var('x,a,b')}
				\item
					\texttt{f(x)=sin(a*x)/sin(b*x)}
				\item
					\texttt{limit(f(x),x=0)}

			\end{enumerate}
			Noter qu'il faut déclarer les variables \texttt{x}, \texttt{a} et \texttt{b}.

		\item
			$\lim_{x\to \infty} \left( 1+\frac{ a }{ x } \right)^x$.
			
		\item
			$\lim_{x\to \pm\infty} \frac{ \sqrt{x^2+1}-x }{ x-2 }$

			Les lignes pour résoudre par Sage sont~:
			\begin{enumerate}

				\item
					\texttt{f(x)=(sqrt(x**2+1))/(x-2)}
				\item
					\texttt{limit(f(x),x=oo)}
				\item
					\texttt{limit(f(x),x=-oo)}


			\end{enumerate}
			Noter la commande pour la racine carré~: \texttt{sqrt}. Étant donné que cette fonction diverge en $x=2$, si tu veux la tracer, il faut procéder en deux fois :
			\begin{enumerate}

				\item
					\texttt{plot(f,(-100,1.9))}
				\item
					\texttt{plot(f,(2.1,100))}

			\end{enumerate}
			La première ligne trace de $-100$ à $1.9$ et la seconde de $2.1$ à $100$. Ces graphiques vous permettent déjà de voir les limites. Attention : ils ne sont pas des \emph{preuves} ! Mais ils sont de sérieux indices qui peuvent vous inspirer dans vos calculs.

		\item
			$\lim_{x\to 2} \frac{ \sqrt{x+2}-2 }{ \sqrt{x+7}-3 }$.

		\item
			$\lim_{h\to 0} \frac{ (x+h)^3-x^2 }{ h }$.

	\end{enumerate}

\corrref{INGE11140031}
\end{exercice}
