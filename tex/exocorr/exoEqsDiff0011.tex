% This is part of Exercices et corrigés de CdI-1
% Copyright (c) 2011, 2019
%   Laurent Claessens
% See the file fdl-1.3.txt for copying conditions.

\begin{exercice}\label{exoEqsDiff0011}


Régulateur de Foucault.\\
Un point $P$ de masse $m$ est accroché à un fil sans masse enroulé autour d'un cylindre homogène de rayon $R$, de masse $M$ et d'axe horizontal fixe. La chute du point $P$  entraîne la rotation du cylindre. Ce cylindre, muni d'ailettes est soumis à la résistance de l'air que l'on représentera par un couple de frottement $-f\theta'$, où $\omega = \theta'$ représente la vitesse de rotation du cylindre. Le système étant abandonné sans vitesse initiale, on veut déterminer la fonction $\omega$. On note $z$ la longueur parcourue par le point $P$.

La mise en équation fait apparaitre les relations suivantes:
\begin{enumerate}
\item - en appliquant le théorème du moment cinétique
\[
	J\theta''=-f\theta'+RT,
\]
où $J=MR^2/2$ est le moment d'inertie du cylindre et $T$ est une force.
\item - en appliquant le principe fondamental de la dynamique pour le point $P$
\[
mz''=mg-T
\]
\end{enumerate}
\begin{enumerate}
\item Quelle est la longueur $\Delta z$ parcourue par $M$ lorsque le cylindre tourne  d'un angle $\Delta \theta$? En déduire une relation entre $z''$ et $\theta''$.
\item Trouver une équation différentielle vérifiée par $\omega$.
\item Résoudre cette équation. Que peut-on conclure pour le mouvement de la poulie lorsque $t$ devient grand?
\end{enumerate}

\end{exercice}
