% This is part of Exercices et corrigés de CdI-1   Supprimé le 10 juin 2013
% Copyright (c) 2011,2014,2017
%   Laurent Claessens
% See the file fdl-1.3.txt for copying conditions.

\begin{corrige}{Variete0017}


	\begin{enumerate}

		\item

			Le chemin $\gamma$ est dans le cadre d'application du théorème de Stokes puisqu'il est le bord d'un triangle $T$, dès lors
			\begin{equation*}
				\int_{\gamma} y^2 d x + z^2 d y + x^2 d z = \iint_T \nabla\times G \cdot dS
			\end{equation*}
			où $G = (y^2, z^2, x^2)$. Le rotationnel vaut
			\begin{equation*}
				\nabla\times G =
				\begin{vmatrix}
                    \vec{e_x} & \vect{e_y} & \vect{e_z}\\
					\partial_x & \partial_y & \partial_z \\
					y^2 & z^2 & x^2
				\end{vmatrix}
				= (-2 z, -2 x, - 2 y)
			\end{equation*}

			Paramétrons le triangle via
			\begin{equation}
				F(u,v)=\begin{pmatrix}
					1-u-v	\\ 
					u	\\ 
					v	
				\end{pmatrix}
			\end{equation}
			avec $u\in\mathopen] 0 , 1 \mathclose[$ et $v\in\mathopen] 0 , 1-u \mathclose[$. Le dessin de la figure \ref{LabelFigTriangleUV} montre que la paramétrisation choisie a la bonne orientation vis-à-vis de la convention prise par le théorème de Green.
			\newcommand{\CaptionFigTriangleUV}{Question d'orientation. La base $(\nu,T)$ a la même orientation que la base $(1_u,1_v)$. Cela fait que la carte choisie est de bonne orientation.}
			\input{auto/pictures_tex/Fig_TriangleUV.pstricks}

			Le vecteur normal à la surface, de cette paramétrisation, vaut
			\begin{equation*}
				\pder F u \wedge \pder F v = (1,1,1)
			\end{equation*}
			et donc l'intégrale devient
			\begin{equation*}
				\int_0^1 \int_0^{1-u} -2 d v d u = -1
			\end{equation*}

		\item
			Le rotationnel de $G$ vaut
			\begin{equation}
				\nabla\times G=2\begin{pmatrix}
					0	\\ 
					0	\\ 
					x-y	
				\end{pmatrix}.
			\end{equation}
			La paramétrisation du cercle, en coordonnées cylindriques, est
			\begin{equation}
				F(r,\theta)=\begin{pmatrix}
					r\cos\theta	\\ 
					r\sin\theta	\\ 
					0	
				\end{pmatrix}.
			\end{equation}
			Nous devons donc intégrer
			\begin{equation}		\label{EqEtapeUn153}
				\int_0^1\int_0^{2\pi} \langle \nabla\times G, N(r,\theta)\rangle   d\theta dr.
			\end{equation}
			Il est vite vu que
			\begin{equation}
				N(r,\theta)=\frac{ \partial F }{ \partial r }\times\frac{ \partial F }{ \partial \theta }=\begin{pmatrix}
					0	\\ 
					0	\\ 
					r	
				\end{pmatrix},
			\end{equation}
			donc le produit scalaire dans l'intégrale \eqref{EqEtapeUn153} donne à intégrer $2r^2(\cos\theta-\sin\theta)$. La symétrie de la fonction $\cos\theta-\sin\theta$ fait que son intégrale sur $\mathopen[ 0 , 2\pi \mathclose]$ est nulle.

			Si nous voulons utiliser Stokes en considérant la demi-sphère au lieu du cercle, c'est plus compliqué. Ce que nous devons calculer est le flux du champ de vecteur $\nabla\times G$ au travers de la demi-sphère. Nous venons de voir que ce flux au travers du « couvercle » est nulle. Donc nous pouvons appliquer le théorème de la divergence et dire
			\begin{equation}
				\int_{\text{demi-sphere}}\nabla\times G+\int_{\text{couvercle}}\nabla\times G=\int_{\text{volume}}\nabla\cdot(\nabla\times G)=0
			\end{equation}
			parce que la divergence d'un rotationnel est toujours nulle.

		\item
			Nous devons calculer 
			\begin{equation}
				\int_{\gamma}\langle G, \dot\gamma\rangle 
			\end{equation}
			où
			\begin{equation}
				\begin{aligned}[]
					\gamma(t)&=\begin{pmatrix}
						\cos(t)	\\ 
						\sin(t)	\\ 
						0	
					\end{pmatrix}&\gamma'(t)&=\begin{pmatrix}
						-\sin(t)	\\ 
						\cos(t)	\\ 
						0	
					\end{pmatrix}
				\end{aligned}
			\end{equation}
			et $t\colon 0\to 2\pi$. Le produit scalaire entre $\gamma'$ et $G$ vaut $x^3\sin(t)+y^3\cos(t)$, donc nous devons calculer
			\begin{equation}
				\int_0^{2\pi} \big[ x^3\sin(t)+y^3\cos(t)\big]\,dt=0.
			\end{equation}
			Si nous voulons faire cela en utilisant Stokes, nous remarquons que
			\begin{equation}
				\nabla\times G=-2\begin{pmatrix}
					0	\\ 
					1	\\ 
					0	
				\end{pmatrix},
			\end{equation}
			tandis que le vecteur normal est toujours $\begin{pmatrix}
				0	\\ 
				0	\\ 
				1	
			\end{pmatrix}$. Le produit scalaire entre les deux étant toujours nul, nous retrouvons le résultat comme quoi l'intégrale demandée est nulle.

		\item
			Le rotationnel de ce champ de vecteurs est toujours nul.

		\item
			L'équation $x^2+z^2=a^2$, $y=0$ est un cercle dans le plan $(x,z)$. Nous utilisons Stokes :
			\begin{equation}
				\int_{\text{cercle}}\langle G, T\rangle =\int_{\text{disque}}\langle \nabla\times G, N\rangle d\sigma_F
			\end{equation}
			avec la paramétrisation
			\begin{equation}
				F(x,z)=\begin{pmatrix}
					x	\\ 
					0	\\ 
					z	
				\end{pmatrix}
			\end{equation}
			Le vecteur normal est donné par
			\begin{equation}
				N=\frac{ \partial F }{ \partial x }\times\frac{ \partial F }{ \partial z }=-\begin{pmatrix}
					0	\\ 
					1	\\ 
					0	
				\end{pmatrix}.
			\end{equation}
			N'oubliez pas de faire des petits dessins en 3D pour montrer que le bon ordre des variables est bien $(x,z)$. Le rotationnel du champ de vecteur est donné par
			\begin{equation}
				\nabla\times G=\nabla\times\begin{pmatrix}
					y	\\ 
					z	\\ 
					x	
				\end{pmatrix}=-\begin{pmatrix}
					1	\\ 
					1	\\ 
					1	
				\end{pmatrix}.
			\end{equation}
			Le produit scalaire de $\nabla\times G$ avec $N$ est la constante $1$. La réponse est donc la surface du cercle de rayon $a$ : $\pi a^2$.

	\end{enumerate}
\end{corrige}
