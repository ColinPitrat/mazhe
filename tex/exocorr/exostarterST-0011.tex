% This is part of Analyse Starter CTU
% Copyright (c) 2014
%   Laurent Claessens,Carlotta Donadello
% See the file fdl-1.3.txt for copying conditions.

\begin{exercice}\label{exostarterST-0011}


\begin{enumerate}
\item Montrer que la fonction $\cos$ est bijective de $I=[0,\pi]$ sur $J=[-1,1]$. La bijection réciproque s'appelle $\arccos$ (on lit arc cosinus), préciser son ensemble de définition.
\item Préciser les valeurs de $\arccos(0)$,\quad $\arccos(0,5)$,\quad $\arccos\dfrac{\sqrt2}2$,\quad $\arccos\dfrac{\sqrt3}2$,\quad $\arccos(-0,5)$,\quad $\arccos(-\dfrac{\sqrt{3}}2)$.
\item Résoudre  les équations : {\bfseries  a/} $\arccos (x) = \dfrac{\pi}{4}$   \quad {\bfseries  b/} $\arccos (x) = \dfrac{3\pi}{4}$
\item $\cos \cfrac{5\pi}4=-\dfrac{\sqrt2}2$. Que vaut $\arccos(-\dfrac{\sqrt{2}}2)$ ? Peut-on comparer $\arccos(\cos x)$ et $x$ ?
\item Montrer que $\arccos$ est dérivable sur $\mathopen] -1 , 1 \mathclose[$ avec
			\begin{equation}
				\big( \arccos(x) \big)'=-\frac{ 1 }{ \sqrt{1-x^2} }.
			\end{equation}
\item Représenter la fonction $\arccos$. 
\end{enumerate}


\corrref{starterST-0011}
\end{exercice}
