% This is part of Outils mathématiques
% Copyright (c) 2011, 2019
%   Laurent Claessens
% See the file fdl-1.3.txt for copying conditions.

\begin{corrige}{OutilsMath-0113}

    Dans le paramétrage recommandée,
    \begin{equation}
        \phi(\rho,\theta,\varphi)=\begin{pmatrix}
            2\rho\sin(\theta)\cos(\varphi)    \\ 
            4\rho\sin(\theta)\sin(\varphi)    \\ 
            6\rho\cos(\theta)    
        \end{pmatrix},
    \end{equation}
    les paramètres ont comme domaines :
    \begin{equation}
        \begin{aligned}[]
            \rho\colon 0\to 3\\
            \theta\colon 0\to \pi\\
            \varphi\colon 0\to 2\pi.
        \end{aligned}
    \end{equation}
    Les vecteurs tangents sont :
    \begin{equation}
        \begin{aligned}[]
            T_{\rho}&=\begin{pmatrix}
                2\sin\theta\cos\varphi    \\ 
                4\sin\theta\sin\varphi    \\ 
                6\cos\theta    
            \end{pmatrix},\\
            T_{\theta}&=\begin{pmatrix}
                2\rho\cos\theta\cos\varphi    \\ 
                4\rho\cos\theta\sin\varphi    \\ 
                -6\rho\sin\theta    
            \end{pmatrix},\\
            T_{\varphi}&=\begin{pmatrix}
                -2\rho\sin\theta\sin\varphi    \\ 
                4\rho\sin\theta\cos\varphi    \\ 
                0    
            \end{pmatrix}.
        \end{aligned}
    \end{equation}
    Ici il est question d'intégrer sur un volume. L'élément de volume se crée avec le produit mixte. Un calcul (le faire\footnote{Oui, vraiment, fais le.} !!) utilisant la formule \eqref{EqProduitMixteDet} donne
    \begin{equation}
        T_{\rho}\cdot(T_{\theta}\times T_{\varphi})=2\cdot 4\cdot 6\cdot\rho^2\sin(\theta)=48\rho^2\sin(\theta).
    \end{equation}
    Ça, c'est pour le paramétrage de notre volume.

    La fonction à intégrer est 
    \begin{equation}
        f\big( \phi(\rho,\theta,\varphi) \big)=|6\rho\cos(\theta)|.
    \end{equation}
    Étant donné que $\rho$ est toujours positif, nous pouvons retirer la valeur absolue de $\rho$. Par contre sur l'intervalle d'intégration, $\cos(\theta)$ change de signe. Il faudra donc y penser. Quoi qu'il en soit, l'intégrale à effectuer est
    \begin{equation}
        \begin{aligned}[]
            \int_Vf&=\int_0^3d\rho\int_0^{\pi}d\theta\int_0^{2\pi}d\varphi\,48\cdot 6\cdot  \rho| \cos(\theta) |\cdot \rho^2\sin(\theta)\\
            &=6\cdot 48\cdot 2\pi\int_0^3\rho^3d\rho\int_0^{\pi}\sin(\theta)| \cos(\theta) |.
        \end{aligned}
    \end{equation}
    L'intégrale sur $\rho$ vaut $\frac{ 81 }{ 4 }$. Celle sur $\theta$ se coupe en deux :
    \begin{equation}
        \int_0^{\pi/2}\sin(\theta)\cos(\theta)-\int_{\pi/2}^{\pi}\sin(\theta)\cos(\theta)=1.
    \end{equation}
    Au final, l'intégrale vaut
    \begin{equation}
        \frac{ 6\cdot 2\cdot 48\cdot 81\pi }{ 4 }=11664\pi.
    \end{equation}
        
\end{corrige}
