% This is part of Outils mathématiques
% Copyright (c) 2011
%   Laurent Claessens
% See the file fdl-1.3.txt for copying conditions.

\begin{corrige}{OutilsMath-0108}

    Le chemin sur lequel on intègre est $\sigma(t)=(R\cos(t),R\sin(t))$. L'astuce à comprendre pour cet exercice est que
    \begin{equation}
        f\big( \sigma(t) \big)=t,
    \end{equation}
    étant donné que $t$ est l'angle correspondant au point $\big( R\cos(t),R\sin(t) \big)$. L'intégrale à calculer est donc
    \begin{equation}
        \int_{\sigma}fd\sigma=\int_0^{2\pi}t\| \sigma'(t) \|dt=R\int_0^{2\pi}t\,dt=2R\pi^2.
    \end{equation}

\end{corrige}
