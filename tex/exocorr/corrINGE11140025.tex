% This is part of Un soupçon de physique, sans être agressif pour autant
% Copyright (C) 2006-2009
%   Laurent Claessens
% See the file fdl-1.3.txt for copying conditions.

\begin{corrige}{INGE11140025}

	\begin{enumerate}

		\item
			Dans le cercle trigonométrique, nous avons $\cos(x)=1$ lorsque $x=0$. Les solutions sont donc 
			\begin{equation}
				x\in\{ \ldots,-4\pi,-2\pi, 0,2\pi,4\pi,\ldots \}=\{ 2k\pi \}_{k\in\eZ}.
			\end{equation}
		\item
		\item
		\item
		\item
			Comme à chaque fois que l'inconnue se trouve dans une fonction spéciale (cosinus, logarithme, exponentielle), un petit changement de variable est tout indiqué. Dans ce cas ci nous posons $y=\cos(3x)$, et nous trouvons l'équation
			\begin{equation}
				-y^2+\frac{ 5 }{2}y-1=0,
			\end{equation}
			dont les solutions sont $y=2$ et $y=1/2$.

			La solution $y=2$ correspond à $\cos(3x)=2$, qui n'a aucune solution. La solution $y=1/2$ par contre corresponds à $\cos(3x)=\frac{ 1 }{2}$. Cela donne les solutions suivantes pour $x$~:
			\begin{subequations}
				\begin{align}
					x&=\frac{ \pi }{ 6 }+\frac{ 2k\pi }{ 3 }\\
					x&=\frac{ \pi }{ 2 }+\frac{ 2k\pi }{ 3 }
				\end{align}
			\end{subequations}
		\item
		\item
		\item
			Nous savons que $\sin(2x)=0$ lorsque $2x=k\pi$, c'est à dire
			\begin{equation}
				x=\frac{ k\pi }{2}.
			\end{equation}
			Ici, comme partout nous sous-entendons $k\in\eZ$.
		\item
			La fonction cosinus ne passe \emph{jamais} par la valeur 4.
		\item
			Sur le cercle trigonométrique, nous avons $\sin(x)=\cos(x)$ lorsque $x=\frac{ \pi }{ 4 }$, mais aussi lorsque $x=\frac{ 5\pi }{ 4 }$. Les solutions sont donc
			\begin{equation}
				x=\frac{ \pi }{ 4 }+k\frac{ \pi }{ 2 }.
			\end{equation}
		\item
			La somme $\cos^2(x)+\sin^2(x)$ fait toujours $1$, il n'y a donc pas de solutions.
		\item
			Nous savons que $\cos(y)=\frac{ 1 }{2}$ lorsque $y=\pi/3$ et $-\pi/3$. Pour toutes les valeurs intermédiaires, $\cos(y)<\frac{ 1 }{2}$. Étant donné que nous regardons la valeur absolue, les valeurs de $y$ entre $2\pi/3$ et $4\pi/3$ sont également à prendre. Au final,
			\begin{equation}
				\begin{aligned}[]
				2x&\in\mathopen] -\frac{ \pi }{ 3 } , \frac{ \pi }{ 3 } \mathclose[+2k\pi \cup\mathopen] \frac{ 2\pi }{ 3 } , \frac{ 4\pi }{ 3 } \mathclose[+2k\pi\\
				&=\mathopen] -\frac{ \pi }{ 3 } , \frac{ \pi }{ 3 } \mathclose[+k\pi.
				\end{aligned}
			\end{equation}
		\item
			Prenons le sinus des deux membres de $\arcsin(x)=-\frac{ \pi }{ 4 }$. Le premier membre devient $x$, tandis que le second membre devient $\sin(-\pi/4)=-1/\sqrt{2}$. La solution est donc
			\begin{equation}
				x=-\frac{1}{ \sqrt{2} }.
			\end{equation}
	\end{enumerate}
\end{corrige}
