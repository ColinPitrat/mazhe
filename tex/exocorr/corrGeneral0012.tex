% This is part of the Exercices et corrigés de mathématique générale.
% Copyright (C) 2009-2011, 2019
%   Laurent Claessens
% See the file fdl-1.3.txt for copying conditions.
\begin{corrige}{General0012}

La stratégie est toujours de dériver la fonction, et puis de chercher les zéros de la dérivée.

\begin{enumerate}

\item
$y(x)=2\sin(x)+\cos(2x)$. Nous avons, en utilisant une petite formule de trigonométrie 
\begin{equation}
	y'(x)=2\cos(x)-2\sin(2x)=2\cos(x)-2\sin(x)\cos(x)
\end{equation}
Cette dérivée s'annule si $\cos(x)=0$, ou bien si $\sin(x)=\frac{ 1 }{2}$. Les valeurs, entre $\pi$ et $2\pi$, qui le font sont
\begin{equation}
	\frac{ \pi }{ 2 },\frac{ 3\pi }{2},\frac{ \pi }{ 6 },\frac{ 2\pi }{ 3 }.
\end{equation}
Les extrémums globaux sont $x=3\pi/2$ et $x=\pi/6$ où la fonction vaut respectivement $-3$ et $3/2$.

\item
$y(x)=x^2+\frac{ 250 }{ x }$, la dérivée est
\begin{equation}
	y'(x)=2x-\frac{ 250 }{ x }.
\end{equation}
Pour résoudre $y'=0$, nous commençons par multiplier l'équation par $x^2$ pour trouver $2x^3-250=0$, dont la seule solution réelle est $x=5$.

\item
$y(x)=e^x/x$, la dérivée est
\begin{equation}
	y'(x)=\frac{ (x-1) e^{x} }{ x^2 },
\end{equation}
qui s'annule pour $x=1$. Rappelez vous que $e^x$ ne s'annule jamais.
\end{enumerate}


\end{corrige}
