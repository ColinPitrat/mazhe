\begin{corrige}{EquaDiff0009}

L'équation différentielle à résoudre est $y'(x)^2=xy$. Cela se résous en prenant la racine carrée des deux côtés, et en ramenant les $y$ d'un côté et les $x$ de l'autre :
\begin{equation}
	\frac{ dy }{ \sqrt{y} }=\pm\sqrt{x}dx.
\end{equation}
Cette équation peut être intégrée des deux côtés. Cela donne
\begin{equation}
	y(x)=\left( \frac{ x^{3/2} }{ 3 }+C \right)^2.
\end{equation}
En particulier, $y(1)=\left( \frac{1}{ 3 }+C \right)^2$, donc pour que $y(1)=1$, il faut $C=\frac{ 2 }{ 3 }$ ou bien $C=-\frac{ 4 }{ 3 }$.

\end{corrige}
% This is part of the Exercices et corrigés de mathématique générale.
% Copyright (C) 2009
%   Laurent Claessens
% See the file fdl-1.3.txt for copying conditions.
