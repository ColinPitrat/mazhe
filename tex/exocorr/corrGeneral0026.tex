% This is part of the Exercices et corrigés de mathématique générale.
% Copyright (C) 2009-2011
%   Laurent Claessens
% See the file fdl-1.3.txt for copying conditions.
\begin{corrige}{General0026}

Comme d'habitude, nous écrivons $2^{-x}= e^{-x\ln(2)}$ avant même de commencer à réfléchir. L'intégrale est assez facile:
\begin{equation}
	V=\pi\int_0^{\infty} e^{-2\ln(2)x}=\left[ -\pi\left(  \frac{  e^{-2\ln(2)x} }{ 2\ln(2) } \right) \right]_0^{\infty}=\frac{ \pi }{ a }.
\end{equation}
Par contre, si nous n'intégrons que de $0$ à $d$, nous trouvons
\begin{equation}
	-\pi\left[ \frac{  e^{-2\ln(2)x} }{ 2\ln(2) } \right]_0^d=-\pi\frac{  e^{-2\ln(2)d} }{ a }+\frac{ \pi }{ 2\ln(2) }.
\end{equation}
Pour que l'un soit la moitié de l'autre, il faut
\begin{equation}
	1- e^{-2\ln(2)}=\frac{ 1 }{2},
\end{equation}
c'est-à-dire $d=\frac{1}{ 2 }$.

\end{corrige}
