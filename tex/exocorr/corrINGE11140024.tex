% This is part of Un soupçon de physique, sans être agressif pour autant
% Copyright (C) 2006-2009
%   Laurent Claessens
% See the file fdl-1.3.txt for copying conditions.


\begin{corrige}{INGE11140024}

	\begin{enumerate}

		\item
			$x=3$.
		\item
			On peut poser $y=2^{x-1}$ et alors $2^{x-3}=y/4$, $2^{3-x}=4/y$ et $2^{1-x}=1/y$. Ainsi, on a l'équation suivante pour $y$~:
			\begin{equation}
				y-\frac{ y }{ 4 }=\frac{ 4 }{ y }-\frac{1}{ y }.
			\end{equation}
			Étant donné que $y=0$ n'est pas possible, on peut multiplier tout par $y$ et obtenir une équation du second degré en $y$~:
			\begin{equation}
				\frac{ 3y^2 }{ 4 }-3=0,
			\end{equation}
			ou encore $y^2-4=0$, ce qui donne les solutions $y=2$ et $y=-2$. La solution $y=2$ donne $2^{x-1}=2$, et donc $x=2$. La solution $y=-2$ donne $2^{x-1}=-2$ qui est impossible.  La solution de l'exercice est donc $x=2$.
				
			Noter qu'on peut également poser $y=2^x$ ou bien $y=2^{x-3}$. Ces substitutions mènent à des calculs un peu différents, mais c'est tout aussi bien.
		\item
			En posant $y=e^x$, nous tombons sur l'équation
			\begin{equation}
				y^2-6y+5=0,
			\end{equation}
			dont les solutions sont $y=1$ et $y=5$. Les solutions en $x$ sont donc $x=\ln(1)=0$ et $x=\ln(5)$.
		\item
		\item
		\item
			Le problème de cet exercice est le logarithme en base $x$. Notez déjà les conditions d'existence : $x$ ne peut pas être négatif, ni nul, ni égal à un. Ensuite, nous utilisons la formule de changement de base des logarithmes pour transformer $\log_x$ en $\log_2$~:
			\begin{equation}
				\log_x(2)=\frac{ \log_2(2) }{ \log_2(x) }.
			\end{equation}
			La fraction se simplifie du fait que $\log_2(2)=1$. L'équation que nous devons regarder devient donc~:
			\begin{equation}
				2\log_2(x)+\frac{1}{ \log_2(x) }=3.
			\end{equation}
			La bonne idée est de poser $y=\log_2(x)$ et de trouver l'équation
			\begin{equation}
				2y^2-3y+1=0,
			\end{equation}
			dont les solutions sont $y=1$ et $y=1/2$. La solution $\log_2(x)=1$ donne $x=2$ et la solution $\log_2(x)=\frac{1}{ 2 }$ donne $x=\sqrt{2}$.
		\item
			Avant de commencer, notez les conditions d'existence $1-x>0$ qui demande $x<1$.	Nous savons que $\log_4(A)<0$ lorsque $A<1$. Cela est valable pour toute expression $A$, tant que la base du logarithme est plus grande que $1$. Donc les solutions de l'équation sont données par $x>0$. En tenant compte des conditions d'existence
			\begin{equation}
				x\in\mathopen] 0 , 1 \mathclose[.
			\end{equation}
			
		\item
			La fonction
			\begin{equation}
				x\mapsto\left( \frac{ 1 }{ 3 } \right)^x
			\end{equation}
			étant décroissante, l'équation est vraie quand 
			\begin{equation}
				3x>2x-9,
			\end{equation}
			c'est-à-dire pour $x>-9$.
			

	\end{enumerate}

\end{corrige}
