% This is part of the Exercices et corrigés de mathématique générale.
% Copyright (C) 2009-2011
%   Laurent Claessens
% See the file fdl-1.3.txt for copying conditions.
\begin{corrige}{General0018}

Le secret pour intégrer une fraction avec un second degré au dénominateur est de faire apparaître une combinaison de la forme $(\alpha x^2+A)+B$ au dénominateur, à la place du binôme donné.

\begin{enumerate}

\item
 Nous cherchons $A$ et $B$ pour que $3x^2-2x+4=(\sqrt{3}x+A)^2+B$. Pour ce faire, nous développons le carré (produit remarquable) et nous égalisons les termes de degré égaux :
\begin{equation}
	(\sqrt{3}x+A)^2=3x^2+2A\sqrt{3}x+A^2,
\end{equation}
doit être égalé à $3x^2-2x+4$. Nous en déduisons que $A=-1/\sqrt{3}$ et $B=11/3$. L'intégrale à calculer se récrit donc
\begin{equation}
	\int\frac{ dx }{ 3x^2-2x+4 }=\int\frac{ dx }{ \big( \sqrt{3-\frac{1}{ \sqrt{3} }} \big)^2+\frac{ 11 }{ 3 } }.
\end{equation}
Nous posons maintenant $u=\sqrt{3}x-1/\sqrt{3}$ et nous avons
\begin{equation}
	\frac{1}{ \sqrt{3} }\int\frac{ du }{ u^2+ \left( \sqrt{\frac{ 11 }{ 3 }} \right)^2  },
\end{equation}
qui se résous par la formule usuelle.

\item
\item
Cette intégrale se coupe en deux :
\begin{equation}
	I=\int\frac{ 3x }{ 1-4x^2 }+\int\frac{ 1 }{ 1-4x^2 }.
\end{equation}
Le deuxième morceau est usuel : il se traite avec le changement de variable $v=2x$. Pour le premier morceau, on a un $x$ au numérateur, et un dénominateur dont la dérivée contient juste un terme en $x$. Nous posons donc $u=1-4x^2$, $du=-8xdx$. Nous avons alors
\begin{equation}
	I=\int\frac{ 3x\left( -\frac{ du }{ 8x } \right) }{ u }+\int\frac{ dv/2 }{ 1-v^2 }=-\frac{ 3 }{ 8 }\ln(u)+\frac{1}{ 4 }\ln\left| \frac{ 1+v }{ 1-v } \right| .
\end{equation}

\item
Encore une fois, nous écrivons le dénominateur sous la forme $(\alpha x+A)^2+B$. Nous trouvons que
\begin{equation}
	6x^2+x-1=\big( \sqrt{6}x+\frac{1}{ 2\sqrt{6} } \big)^2-\frac{ 25 }{ 24 }.
\end{equation}
Le changement de variable $u=\sqrt{6}x+\frac{1}{ 2\sqrt{6} }$ amène $du=\sqrt{6}dx$ et 
\begin{equation}
	x=\frac{ u-\frac{1}{ 2\sqrt{6} } }{ \sqrt{6}} 
\end{equation}
donc 
\begin{equation}
	I=\int \frac{ \frac{ 7u-\frac{ 7 }{ 2\sqrt{6} } }{\sqrt{6}}\frac{1}{ \sqrt{6}du } } {  u^2-\left( \sqrt{\frac{ 25 }{ 24 }} \right)^2  }
\end{equation}
Cette intégrale se coupe en deux. Une de la forme $\int u/(u^2-A^2)$, et une de la forme $\int 1/(u^2-A^2)$. Ces deux ont déjà été traitées.


\end{enumerate}


\end{corrige}
