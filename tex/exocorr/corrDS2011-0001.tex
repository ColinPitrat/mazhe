\begin{corrige}{DS2011-0001}
  
    \begin{enumerate}
        \item
            

 % Supprimé le 27 juillet 2014
 %   Nous avons un dessin sur la figure \ref{LabelFigDSdmdexou}.
 %   \newcommand{\CaptionFigDSdmdexou}{Le dessin de l'ensemble à étudier.}
 %   \input{auto/pictures_tex/Fig_DSdmdexou.pstricks}

    \begin{center}
   \input{auto/pictures_tex/Fig_UZGooBzlYxr.pstricks}
    \end{center}


    Les deux points d'intersection \( A\) et \( B\) s'obtiennent en posant \( y=1\) dans les équations \( x=3-y\) et \( x=\sqrt{9-y^2}\). Les résultats sont \( A=(2,1)\) et \( B=(2\sqrt{2},1)\)

    L'intérieur et l'adhérence s'obtiennent en mettant respectivement les inégalités strictes et non strictes dans la définition. L'ensemble n'est ni ouvert, ni fermé. Il est borné. Sa frontière s'écrit comme l'union de trois pièces :
    \begin{equation}
        \partial A_1=\{ (x,1)\tq x\in\mathopen[ 2,2\sqrt{2}  \mathclose] \}\cup\{ x=3-y\tq x\in\mathopen[ 0 , 2 \mathclose] \}\cup\{ x=\sqrt{9-y^2}\tq x\in\mathopen[ 0 , 2\sqrt{2} \mathclose] \}.
    \end{equation}

\item

    L'ensemble \( A_2\) est une suite de points dans \( \eR^2\). Son intérieur est vide, son adhérence est lui-même, il n'est pas ouvert, il est fermé, et il est non borné.
    \end{enumerate}
    <++>

\end{corrige}
