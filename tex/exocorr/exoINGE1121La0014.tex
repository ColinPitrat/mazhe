% This is part of the Exercices et corrigés de mathématique générale.
% Copyright (C) 2009-2010
%   Laurent Claessens
% See the file fdl-1.3.txt for copying conditions.

\begin{exercice}\label{exoINGE1121La0014}

	(INGE1121, 6.4) Soit la forme quadratique
	\begin{equation}
		p(X)=2x_1x_2-x_1x_3+x_1x_4-x_2x_3+x_2x_4-"x_3x_4.
	\end{equation}
	\begin{enumerate}

		\item
			Calculer le rang et le déterminant de la matrice symétrique associée $A$:
		\item
			Qu'en déduire concernant les valeurs propres de $A$ ?
		\item
			$-1$ est une valeur propre de $A$. Calculer sa multiplicité dans le polynôme caractéristique de $A$.
		\item
			Construire une matrice orthogonale $B$ telle que $B^tAB$ soit diagonale.

	\end{enumerate}
	

\corrref{INGE1121La0014}
\end{exercice}
