\begin{corrige}{CourbesSurfaces0001}

	\begin{enumerate}
		\item
			Nous devons prouver que $\Graph(f)=\Graph(g)$. Si $(t,t)\in \Graph(f)$, alors $(t,t)=g(\sqrt{t})$ où $\sqrt{t}$ existe parce que $t\geq 0$. Nous avons donc $\Graph(f)\subset\Graph(g)$. Inversement, si $(t^2,t^2)\in\Graph(g)$, alors $(t^2,t^2)=f(t^2)$ où $t^2\in\mathopen[ 0 , 1 \mathclose]$ lorsque $t\in\mathopen[ -1 , -1 \mathclose]$.

			Les graphes sont donc identiques. Si ils étaient équivalents (définition \ref{DefAcrEquiva}), alors il y aurait une application inversible $\theta\colon [-1,1]\to [0,1]$ telle que $g(t)=f\big( \theta(t) \big)$, c'est-à-dire
			\begin{equation}
				(t^2,t^2)=\big( \theta(t),\theta(t) \big).
			\end{equation}
			Cela impose $\theta(t)=t^2$, mais cette fonction n'est pas inversible sur $[-1,1]$.

		\item
			Les deux graphes sont évidement les mêmes parce que les valeurs de $t$ entre $2\pi$ et $6\pi$ ne créent aucun nouveau points (périodicité de sinus et cosinus). En ce qui concerne l'équivalence des deux courbes, nous devons avoir une application $\theta\colon [0,6\pi]\to [0,2\pi]$ telle que
			\begin{equation}
				\big( \cos(t),\sin(t) \big)=\big( \cos\theta(t),\sin\theta(t) \big),
			\end{equation}
			ce qui impose $\theta(t)=t+2k\pi$ pour un certain $k$ entier. Aucune de ces fonctions n'est une bijection entre $[0,2\pi]$ et $[0,6\pi]$.
		\item
			Il nous faut une fonction $\theta\colon \mathopen] -\pi , \pi \mathclose[\to \eR$ telle que $g\big( \theta(t) \big)=f(t)$, c'est-à-dire
			\begin{equation}
				\left( \frac{ 1-\theta^2(t) }{ 1+\theta^2(t) },\frac{ 2\theta(t) }{ 1+\theta^2(t) } \right)=\big( \cos(t),\sin(t) \big).
			\end{equation}
			En résolvant pour la première composante, nous trouvons
            \begin{equation}        \label{Eqthetafromunzzzucs}
				\theta(t)=\pm\sqrt{\frac{ 1-\cos(t) }{ 1+\cos(t) }}
			\end{equation}
            où le signe est à discuter en fonction du signe de \( \sin(t)\). En utilisant l'indice, nous savons que cela se récrit avantageusement sous la forme
            \begin{equation}
                \theta(t)=\tan\left( \frac{ t }{2} \right).
            \end{equation}
            Sous cette forme, il est immédiatement apparent que c'est un difféomorphisme. Il reste à vérifier que
            \begin{equation}
                \sin(t)=\frac{ 2\theta(t) }{ 1+\theta^2(t) }.
            \end{equation}
            En utilisant \( \theta\) sous la forme \eqref{Eqthetafromunzzzucs}, cela se vérifie très facilement.

	\end{enumerate}

\end{corrige}
