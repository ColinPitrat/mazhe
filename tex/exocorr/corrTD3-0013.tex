% This is part of Exercices de mathématique pour SVT
% Copyright (c) 2010
%   Laurent Claessens et Carlotta Donadello
% See the file fdl-1.3.txt for copying conditions.

\begin{corrige}{TD3-0013}

    \begin{description}
        \item[Candidats limite] 

    Comme d'habitude, nous commençons par trouver les limites possibles en résolvant l'équation
    \begin{equation}        \label{Equumulnu}
        u=u-u\ln(u).
    \end{equation}
    Notez que $u=0$ pourrait être une limite parce que la fonction $u-u\ln(u)$ n'est pas définie en $u=0$, donc la proposition \ref{Propufulimite} ne peut pas conclure. Quoi qu'il en soit, les autres limites possible se trouvent en simplifiant l'équation \eqref{Equumulnu} par $u$. Nous tombons sur l'équation $1=1-\ln(u)$, c'est-à-dire $\ln(u)=0$ et donc $u=1$.

    Deux limites possibles : $0$ et $1$.

        \item[Croissance]

    En ce qui concerne la croissance de la suite, nous avons
    \begin{equation}
        \frac{ u_{n+1} }{ u_n }=1-\ln(u_n).
    \end{equation}
    Ici nous voyons une différence entre $u_n\in\mathopen] 0 , 1 \mathclose[$ et $u_n\in\mathopen] 1 , e \mathclose[$. 
    
    \begin{enumerate}
        \item
            Si $u_n\in\mathopen] 0 , 1 \mathclose[$, alors $1-\ln(u_n)>1$ et la suite est croissante. En d'autres termes, \( u_n<1\) implique \( u_{n+1}>u_n\).

        \item
            Si $u_n\in\mathopen] 1 , e \mathclose[$, nous avons $\frac{ u_{n+1} }{ u_n }<1$ et la suite est donc décroissante. En d'autres termes, si \( u_n>1\), alors \( u_{n+1}<u_n\).
    
            
    \end{enumerate}

        \item[Bornes]

            La fonction qui définit la récurrence est
            \begin{equation}
                f(x)=x-x\ln(x).
            \end{equation}
            La dérivée vaut \( f'(x)=-\ln(x)\).

            \begin{enumerate}
                \item
                    
                    Nous commençons par considérer le cas où la suite prend au moins une valeur en dessous de \( 1\).

                    Si \( x<1\), nous avons \( f'(x)>0\). Cela signifie que la fonction est croissante et que parmi les \( x\in\mathopen] 0 , 1 \mathclose[\), celui qui rend \( f(x)\) la plus grande possible est \( x=1\). Donc si \( u_n<1\), nous avons
    \begin{equation}
        u_{n+1}=f(u_n)<f(1)=1
    \end{equation}
    La conclusion est que si \( u_n<1\), alors \( u_{n+1}<1\). Dès que la suite passe par une valeur plus basse que \( 1\), elle y reste. Nous avons par ailleurs montré que pour les \( u_n<1\), la suite était croissante. Dès que la suite prend une valeur plus petite que \( 1\), elle est donc bornée et croissante et par conséquent convergente. Les seuls candidats étant \( 0\) et \( 1\), la suite converge vers \( 1\).

    
            \item

                Traitons à présent du cas où la suite reste toujours au dessus de \( 1\). Nous avons déjà vu que si \( u_n>1\), alors \( u_{n+1}<u_n\).

                Si \( x>1\), nous avons \( f'(x)<0\) et \( f\) est donc une fonction décroissante. Son minimum \emph{parmi les \( x\) plus grands que \( 1\)} est donc atteint pour \( x=1\). C'est-à-dire que
                \begin{equation}
                    f(x)<f(1)=1
                \end{equation}
                pour tout \( x>1\). En ce qui concerne notre suite, si \( u_n>1\), nous avons
                \begin{equation}
                    u_{n+1}=f(u_n)<f(1)=1.
                \end{equation}
                Si la suite passe par une valeur supérieure à \( 1\), elle retourne instantanément à une valeur inférieure à \( 1\).
    
            \end{enumerate}

    \end{description}
    
    En conclusion, si \( x\in\mathopen] 0 , 1 \mathclose[\) nous avons \( u_0<1\) et la suite reste constamment inférieure à \( 1\). Dans ce cas elle converge vers \( 1\).

    Si par contre \( x\in\mathopen] 1 , e \mathclose[\), nous avons \( u_0\in\mathopen] 1 , 0 \mathclose[\) et par conséquent \( u_1<1\). La suite converge donc également vers \( 1\).

\end{corrige}
