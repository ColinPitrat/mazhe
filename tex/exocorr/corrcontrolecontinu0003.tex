\begin{corrige}{controlecontinu0003}

    \begin{enumerate}
        \item
            Les points où la continuité n'est pas assurée sont les points de la forme \( (0,a)\) avec \( a\in\eR\). Nous devons donc calculer, pour tout \( a\) la limite
            \begin{equation}
                \lim_{(x,y)\to(0,a)}\frac{ y\sin^2(2x) }{ x }
            \end{equation}
            Nous pouvons calculer séparément les deux limites
            \begin{subequations}
                \begin{align}
                    \lim_{(x,y)\to(0,a)}y&=0\\
                    \lim_{(x,y)\to(0,a)}\frac{ \sin^2(2x) }{ x }&=0.
                \end{align}
            \end{subequations}
            Étant donné que les deux limites existent, la limite du produit est égale au produit des limites (proposition \ref{PropOpsSimplesLimites}). Notre limite vaut donc zéro et la fonction est continue (parce que la limite vers \( (0,a)\)) est égale à la valeur en \( (0,a)\).

            \item
                Le calcul montre que 
                \begin{subequations}        \label{SubeqsDerpartGAzztc}
                    \begin{align}
                        \frac{ \partial f }{ \partial x }&=\frac{ 4y\sin(2x)\cos(2x) }{ x }-y\frac{ \sin^2(2x) }{ x^2 }\\
                        \frac{ \partial f }{ \partial y }&=\frac{ \sin^2(2x) }{ x }.
                    \end{align}
                \end{subequations}
                En posant \( x=\frac{ \pi }{ 6 }\), \( y=4\) nous trouvons
                \begin{subequations}
                    \begin{align}
                        \frac{ \partial f }{ \partial x }(\frac{ \pi }{ 6 },4)&=\frac{ 24\sqrt{3} }{ \pi }-\frac{ 108 }{ \pi^2 }\\
                        \frac{ \partial f }{ \partial y }(\frac{ \pi }{ 6 },4)&=\frac{ 9 }{ 2\pi }.
                    \end{align}
                \end{subequations}
                
            \item
                Le point \( (0,2)\) n'étant pas dans le domaine sur lequel la fonction est «facile», nous devons utiliser la définition de la dérivée directionnelle :
                \begin{subequations}
                    \begin{align}
                        \frac{ \partial f }{ \partial x }(0,2)&=\lim_{t\to 0} \frac{ f(t,2)-f(0,2) }{ t }\\
                        &=\lim_{t\to 0} \frac{ 2\sin^2(2t) }{ t^2 }\\
                        &=2\lim_{t\to 0} \frac{ 2\sin(2t) }{ 2t }\frac{ 2\sin(2t) }{ 2t }\\
                        &=8.
                    \end{align}
                \end{subequations}
                Dans ce calcul, nous avons multiplié le numérateur et le dénominateur par \( 4\). La dérivée selon \( y\) est plus simple parce que \( f(0,2+t)=f(0,2)=0\), ce qui donne
                \begin{equation}        \label{EqAGzztczdttozfrac}
                    \frac{ \partial f }{ \partial y }(0,2)=\lim_{t\to 0} \frac{ f(0,2+t)-f(0,2) }{ t }=0.
                \end{equation}

            \item
                Nous devons vérifier si en prenant la limite \( (x,y)\to(0,2)\) des fonctions \eqref{SubeqsDerpartGAzztc} sont bien les nombres \( \partial_xf(0,2)\) et \( \partial_yf(0,2)\). La première limite à calculer est
                \begin{equation}        \label{EqlimxyAGzztcqymyfrac}
                    \lim_{(x,y)\to(0,2)}\frac{ 4y\sin(2x)\cos(2x) }{ x }-y\frac{ \sin^2(2x) }{ x^2 }.
                \end{equation}
                En décomposant et en sachant que \( \lim_{x\to 0} \sin(2x)/x=2\), nous trouvons que la limite \eqref{EqlimxyAGzztcqymyfrac} vaut zéro, ce qui ne correspond pas à la valeur de \( \partial_xf(0,2)\) donnée par \eqref{EqAGzztczdttozfrac}. La dérivée partielle par rapport à \( x\) n'est donc pas continue en \( (0,2)\). En ce qui concerne la dérivée par rapport à \( y\), nous avons
                \begin{equation}
                    \lim_{(x,y)\to(0,2)}\frac{ \sin^2(2x) }{ x }=0,
                \end{equation}
                cette dérivée directionnelle est donc continue.

    \end{enumerate}

\end{corrige}
