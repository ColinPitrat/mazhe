% This is part of the Exercices et corrigés de mathématique générale.
% Copyright (C) 2009
%   Laurent Claessens
% See the file fdl-1.3.txt for copying conditions.
\begin{corrige}{EquaDiff0013}

	\begin{enumerate}

		\item
			La solution générale de l'équation différentielle sans second membre est
			\begin{equation}
				y_H(x)=A+B e^{2x}.
			\end{equation}
			Pour trouver une solution particulière de l'équation avec second membre, nous essayons $y_P(x)=ax^2+bx$. En injectant cet essai dans l'équation différentielle, nous trouvons une équation pour $a$ et $b$ dont la solution donne
			\begin{equation}
				y_P(x)=-\frac{ x^2 }{2}-\frac{ x }{2}.
			\end{equation}
			La solution générale de l'équation différentielle avec second membre est donc
			\begin{equation}
				y(x)=A+B e^{2x}-\frac{ x^2 }{ 2 }-\frac{ x }{2}.
			\end{equation}
			
			La première condition à poser sur $A$ et $B$ est $y(0)=B+A=2$. Pour la seconde, nous commençons par calculer $y'(x)$ :
			\begin{equation}
				y'(x)=2 e^{2x}B-t-\frac{ 1 }{2}.
			\end{equation}
			La second condition à imposer est donc
			\begin{equation}
				y'(0)=2B-\frac{1}{ 2 }=0.
			\end{equation}
			En résolvant ce petit système algébrique, nous trouvons les valeurs de $A$ et $B$ qui conviennent~: $A=3/4$ et $B=1/4$.

		\item
			La solution de l'équation différentielle sans second membre est $y_H(x)=A e^{3x}+Bx e^{3x}$. Pour trouver une solution particulière, il est certain que $(ax+b) e^{3x}$ ne va pas fonctionner parce que c'est solution de l'équation sans second membre. Il faut donc prendre un degré plus haut. On essaye donc $y_P(x)=ax^4+bx^3$. En remettant dans l'équation de départ, nous trouvons les valeurs de $a$ et $b$~:
			\begin{equation}
				y_P(x)=\frac{ x^4 }{ 12 }.
			\end{equation}
			La solution générale à l'équation donnée est donc
			\begin{equation}
				y(x)=A e^{3x}+Bx e^{3x}+\frac{ x^4 }{ 12 } e^{3x}.
			\end{equation}
			Nous devons maintenant trouver les bonnes constantes $A$ et $B$ dans les situations demandées.
			\begin{enumerate}

				\item
					Nous voulons une tangente au point $(0,1)$, donc en particulier il faut que la courbe passe par là. Donc nous posons la première contrainte : $y(0)=1$. Ensuite, nous demandons que le coefficient directeur de la tangente en ce point soit nul, c'est à dire que $y'(0)=0$. Cela pose la seconde contrainte : 
					\begin{equation}
						y'(0)=0=\frac{ 1 }{ 12 }(12B), 
					\end{equation}
					donc $B=1$. La solution demandée est donc
					\begin{equation}
						y(x)= e^{3x}+x e^{3x}+\frac{ x^4 }{ 12 } e^{3x}.
					\end{equation}

				\item
					Cette fois, il faut résoudre les contraintes
					\begin{equation}
						\begin{aligned}[]
							y(1)&=0\\
							y'(1)&=\frac{ e^3 }{ 4 }.
						\end{aligned}
					\end{equation}
					La réponse est que $A=0$ et $B=-1/12$, et la solution demandée à l'équation différentielle est
					\begin{equation}
						\frac{ x e^{3x} }{ 12 }(x^3-1).
					\end{equation}
			\end{enumerate}
	\end{enumerate}
\end{corrige}
