% This is part of Exercices et corrections de MAT1151
% Copyright (C) 2010,2014
%   Laurent Claessens
% See the file LICENCE.txt for copying conditions.

\begin{corrige}{SerieDeux0003}

	\begin{enumerate}

		\item
			En dimension $1$, la différentielle est l'application linéaire de coefficient directeur donné par la dérivée. Dans le cas de $f(x)=x^3$, nous avons que 
			\begin{equation}
				df_2(x)=f'(2)x=12x.
			\end{equation}
		\item
			L'application proposée est une application linéaire. Elle est donc égale à sa différentielle. Nous allons le voir explicitement dans l'exercice \ref{ItemExoDeuxTroisc}.
		\item\label{ItemExoDeuxTroisc}
			Nous supposons que l'application $\alpha$ s'écrive $\alpha(x)=\sum_{ij}\alpha_{ij}x_j e_i$, et nous supposons que $w_0=\sum_kw_ke_k$. En omettant les sommes sous-entendues, nous pouvons écrire la fonction $f$ sous la forme
			\begin{equation}
				\begin{aligned}[]
					f(x)&=\langle \alpha(x),w_0\rangle\\
					&=\langle \alpha_{ij}x_j e_i,w_ke_k\rangle\\
					&=\sum_{ijk}\alpha_{ij}x_jw_k\underbrace{\langle e_i,e_k\rangle}_{=\delta_{ik}}\\
					&=\sum_{ij}\alpha_{ij}x_jw_i
				\end{aligned}
			\end{equation}
			Les dérivées partielles s'écrivent
			\begin{equation}
				\begin{aligned}[]
					\frac{ \partial f }{ \partial x_k }&=\sum_{ij}w_i\alpha_{ij}\underbrace{\frac{ \partial x_j }{ \partial x_k }}_{=\delta_{jk}}\\
					&=\sum_iw_i\alpha_{ik}.
				\end{aligned}
			\end{equation}
			Notez que ces coefficients sont constants. Cela est le fait que la différentielle d'une application linéaire est linéaire. Vérifions maintenant qu'elle est même égale à elle même. Appliquons $df$ au vecteur $v=v_ke_k$ :
			\begin{equation}
				\begin{aligned}[]
					df(v)&=\sum_k\frac{ \partial f }{ \partial x_k }v_k\\
					&=\sum_{ki}w_i\underbrace{\alpha_{ik}v_k}_{=\alpha(v)_i}\\
					&=\sum_i w_i\alpha(v)_i\\
					&=\langle w,\alpha(v)\rangle.
				\end{aligned}
			\end{equation}
			Nous retombons donc bien sur la fonction $f$ de départ.

		\item
			La matrice de la différentielle est donnée par
			\begin{equation}
				\begin{pmatrix}
					\frac{ \partial f_1 }{ \partial x }(1,1)	&	\frac{ \partial f_1 }{ \partial y }	(1,1)\\
					\frac{ \partial f_2 }{ \partial x }(1,1)	&	\frac{ \partial f_2 }{ \partial y }(1,1)	\\
					\frac{ \partial f_3 }{ \partial x }(1,1)	&	\frac{ \partial f_3 }{ \partial y }(1,1)	
				\end{pmatrix}
			\end{equation}
			où $f_1(x,y)=x$, $f_2(x,y)=y$ et $f_3(x,y)=x^2+y^2$. En calculant les dérivées partielles nous trouvons la matrice
			\begin{equation}
				\begin{pmatrix}
					1	&	0	\\
					0	&	1	\\
					2	&	2	
				\end{pmatrix},
			\end{equation}
			et donc son action sur un vecteur est donnée par
			\begin{equation}			\label{EqActionDiffddeuxtrois}
				df_{(1,1)}\begin{pmatrix}
					v_x	\\ 
					v_y	
				\end{pmatrix}
				=
				\begin{pmatrix}
					v_x	\\ 
					v_y	\\ 
					2v_x+2v_y	
				\end{pmatrix}.
			\end{equation}
			

	\end{enumerate}

\end{corrige}
