% This is part of Analyse Starter CTU
% Copyright (c) 2014,2017
%   Laurent Claessens,Carlotta Donadello
% See the file fdl-1.3.txt for copying conditions.


\begin{corrige}{mazhe-0006}

  \begin{enumerate}
  \item L'ensemble de définition de $f_2$ a été donné dans le cours, il s'agit de l'intervalle $[-1,1]$. Pour la fonction $f_1$ on commence par dire que la racine carrée est définie si $1-x^2 \geq 0$ et que à son tour la fraction rationnelle est définie uniquement si $\sqrt{1-x^2}\neq 0$. La fonction $\arctan$ par contre est définie sur $\eR$ tout entier. En somme nous avons que $\Dom_{f_2} = ]-1,1[$.

La dérivée de  $f_2$ a été donné dans le cours, il s'agit de $\displaystyle x\mapsto \frac{1}{\sqrt{1-x^2}}$. Pour calculer la dérivée de $f_1$ il faut appliquer à plusieurs reprises la formule de dérivation des fonctions composées
\begin{equation*}
  \begin{aligned}
    f_1' (x) = &\frac{1}{1+\left(\frac{x}{\sqrt{1-x^2}}\right)^2} \cdot \frac{\sqrt{1-x^2} + \frac{x^2}{\sqrt{1-x^2}}}{1-x^2} = \\
    &= \frac{1-x^2}{(1-x^2) +x^2}\cdot\frac{\frac{(1-x^2) +x^2}{\sqrt{1-x^2}}}{1-x^2} =\frac{1}{\sqrt{1-x^2}}.
  \end{aligned}
\end{equation*}

Les deux fonctions ont donc la m\^eme dérivée. Il faut aussi remarquer que les ensembles de définitions de dérivées sont égaux. 
  \item On peut en conclure que la fonction différence entre $f_1$ et $f_2$, définie sur la partie commune du domaine $]-1,1[$,  est une constante (c'est à dire, une fonction qui a dérivée nulle). Laquelle ? Il suffit de calculer $f_1(x)-f_2(x)$ pour une valeur de $x$ : si on prend par exemple $x =0$ on a $f_1(0) -f_2(0) =0$. Cela nous dit que $f_1$ et $f_2$ sont égales sur l'intervalle $]-1,1[$. On ne peut pas les comparer aux points $x = \pm 1$, car $f_1$ n'est pas définie à ces points.
  \end{enumerate}
\end{corrige}
