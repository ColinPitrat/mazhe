\begin{corrige}{SC_serie4-0003}

Nous suivons le plan suivant
\begin{itemize}
	\item 
		Nous commençons par définir la fonction qui donne le problème de Cauchy, c'est-à-dire $f(x,y)=1+y^2$. Notez qu'il faut bien définir une fonction de deux variables $x$ et $y$, même si on n'en utilise une seule.

	\item
		Ensuite, nous résolvons le système en mettant la solution dans les vecteurs \verb+x+ et \verb+y+. 
		
		Notez que j'ai l'impression que pour \href{http://octave.sourceforge.net/doc/f/ode45.html}{ode45}, la syntaxe est notablement différente entre Octave et Matlab.

	\item
		La commande \verb+print -dps exo43.ps+ sert à enregistrer le graphique dans le fichier \verb+exo43.ps+.
\end{itemize}

\lstinputlisting{tex/matlab/SC_exo_4-3.m}

\end{corrige}
