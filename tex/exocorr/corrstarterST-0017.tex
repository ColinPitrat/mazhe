% This is part of Analyse Starter CTU
% Copyright (c) 2014
%   Laurent Claessens,Carlotta Donadello
% See the file fdl-1.3.txt for copying conditions.

\begin{corrige}{starterST-0017}

\begin{enumerate}
  \item L'ensemble des primitives de $f$ est 
    \begin{equation*}
      \int x+2 \, dx = \frac{1}{2} x^2 + 2x + C, \qquad C \in \eR. 
    \end{equation*}
    Pour répondre à la question il faut trouver une valeur de $C$ telle que si $F_{1}(x)=\frac{1}{2} x^2 + 2x + C $ alors $F_{1}(2) = 0$, c'est à dire que $C$ est la solution de $6 + C = 0$. On a alors $C = -6$ et  $F_{1}(x)=\frac{1}{2} x^2 + 2x -6 $. 
    \item 
      \begin{Aretenir}
        Dans cette deuxième partie de l'exercice nous sommes obligés à préciser dans quel intervalle sera définie la primitive que nous intéresse. Cela est du au fait que l'ensemble de définition de la fonction $\tan$ consiste en une réunion d'intervalles disjoints. {\bf Cela ne veut pas dire qu'on calcul l'intégrale de $\tan$ sur l'intervalle $\left]-\dfrac{\pi}{2}\,;\,\dfrac{\pi}{2}\right[$ !!}
      \end{Aretenir}

      L'ensemble des primitives que nous intéresse est donc  
      \begin{equation*}
        \int \tan(x) \, dx = \int \frac{\sin(x)}{\cos(x)} \, dx = -\ln\left(|\cos(x)|\right)+C  = \ln\left(\frac{1}{\cos(x)}\right)+C,
      \end{equation*}
      pour $C \in \eR$, et $x$ dans  l'intervalle $\left]-\dfrac{\pi}{2}\,;\,\dfrac{\pi}{2}\right[$.
  
      Le choix de l'intervalle nous à permis d'omettre la valeur absolue, car la fonction $\cos$ est toujours positive sur $\left]-\dfrac{\pi}{2}\,;\,\dfrac{\pi}{2}\right[$.
    
            Pour déterminer la primitive qui s'annule en $\dfrac{\pi}{3}$ il faut résoudre l'équation $\ln\left(\frac{1}{\cos(\pi/3)}\right)+C = 0$. On trouve $C = \ln(2)$ et donc la primitive cherchée  est $F(x) = \ln\left(\frac{1}{\cos(x)}\right)+\ln(2) = \ln\left(\frac{2}{\cos(x)}\right)$.
  \end{enumerate}

\end{corrige}
