% This is part of Un soupçon de physique, sans être agressif pour autant
% Copyright (C) 2006-2009
%   Laurent Claessens
% See the file fdl-1.3.txt for copying conditions.


\begin{corrige}{INGE11140034}

	Afin de prouver la croissance, nous calculons $s_{n+1}-s_n$, et nous prouvons que cette différence est toujours positive. En mettant au même dénominateur et en simplifiant, nous trouvons
	\begin{equation}
		s_{n+1}-s_n=\frac{ 2(n+1)-7 }{ 3(n+1)+2 }-\frac{ 2n-7 }{ 3n+2 }=\frac{ 25 }{ 9n^2+21n+1 },
	\end{equation}
	qui est toujours positif quand $n$ est positif. Note qu'ici, on ne considère que les $n>0$. Nous travaillons avec des suites, pas avec des fonctions !
	Nous pouvons prouver que cette suite est majorée de la façon suivante~:
	\begin{equation}
		s_n=\frac{ 2n-7 }{ 3n+2 }<\frac{ 2n }{ 3n+2 }<\frac{ 2n }{ 3n }=\frac{ 2 }{ 3 }.
	\end{equation}
	Les inégalités sont vraies parce que on ajoute $7$ au numérateur et on enlève $2$ au dénominateur. La suite est donc majorée par $\frac{ 2 }{ 3 }$.

	Maintenant que nous savons que la suite est majorée et croissante, nous savons qu'elle est convergente. Le calcul de la limite se fait de façon usuelle, et le résultat est $\frac{ 2 }{ 3 }$.

\end{corrige}
