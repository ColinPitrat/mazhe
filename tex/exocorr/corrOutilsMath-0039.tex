% This is part of Exercices et corrigés de CdI-1
% Copyright (c) 2011,2014
%   Laurent Claessens
% See the file fdl-1.3.txt for copying conditions.

\begin{corrige}{OutilsMath-0039}

    Le coefficient directeur de la tangente est donné par la valeur de la dérivée en $\pi/2$.

    \begin{verbatim}
sage: f(x)=x*cos(2*x)
sage: (f.diff(x))(pi/2)
-1
    \end{verbatim}
    Si vous voulez voir le détail :
    \begin{verbatim}
sage: f.diff(x)
x |--> -2*x*sin(2*x) + cos(2*x)
    \end{verbatim}
    Notre tangente sera donc de la forme
    \begin{equation}
        y=-x+b
    \end{equation}
    et il faut encore trouver la constante $b$. Pour cela nous savons que la tangente doit passer par le point $(\frac{ \pi }{ 2 },f(\frac{ \pi }{ 2 }))$. Donc
    \begin{equation}
        f\left( \frac{ \pi }{ 2 } \right)=-\frac{ \pi }{ 2 }+b.
    \end{equation}
    Étant donné que $f(\pi/2)=-\pi/2$, nous avons $b=0$ et au final
    \begin{equation}
        y=-x.
    \end{equation}

\end{corrige}
