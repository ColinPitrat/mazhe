% This is part of Outils mathématiques
% Copyright (c) 2011
%   Laurent Claessens
% See the file fdl-1.3.txt for copying conditions.

\begin{corrige}{OutilsMath-0120}

    La surface est représentée à la figure \ref{LabelFigCouroneExam}. C'est la surface contenue entre les cercles de rayon \( 1\) et \( 2\), dans le cadrant où \( x\) et \( y\) sont positifs.
    \newcommand{\CaptionFigCouroneExam}{La surface de l'exercice \ref{exoOutilsMath-0120}.}
    \input{auto/pictures_tex/Fig_CouroneExam.pstricks}

    En coordonnées polaires cette surface est paramétrée par
    \begin{equation}
        \begin{aligned}[]
            r\colon 1\to 2\\
            \theta\colon 0\to \frac{ \pi }{2}.
        \end{aligned}
    \end{equation}
    En comptant le jacobien \( r\) des coordonnées polaires, l'intégrale à calculer est
    \begin{equation}
        I=\int_1^2dr\int_0^{\pi/2}d\theta\frac{ r^3\cos(\theta)\sin^2(\theta)+1 }{ r }r.
    \end{equation}
    La partie de l'intégrale qui peut poser problème est celle sur \( \theta\) :
    \begin{equation}
        \int_0^{\pi/2}\cos(\theta)\sin^2(\theta)d\theta.
    \end{equation}
    On pose \( u=\sin(\theta)\), et cela devient
    \begin{equation}
        \int_0^1u^2du=\frac{1}{ 3 }.
    \end{equation}
    Au final, le résultat est
    \begin{equation}
        I=\frac{ \pi }{2}+\frac{ 5 }{ 4 }.
    \end{equation}

\end{corrige}
