\begin{corrige}{CalculDifferentiel0001}

	\begin{enumerate}
		\item
			L'apparition de la combinaison $x^2+y^2$ est un signal impérieux pour passer aux coordonnées polaires :
			\begin{equation}
				f(\rho\cos\theta,\rho\sin\theta)=\rho^2\sin\frac{1}{ \rho^2 }.
			\end{equation}
			Cela est une fonction qui ne dépend pas de $\theta$ et donc le fait que $\lim_{\rho\to 0}f=0$ implique que $\lim_{(x,y)\to(0,0)}f(x,y)=0$. La fonction est donc continue parce que $f(0,0)=0$.

			En ce qui concerne les dérivées partielles,
			\begin{equation}		\label{EqCDundsdfx}
				\frac{ \partial f }{ \partial x }(x,y)=2x\sin\left( \frac{1}{ x^2+y^2 } \right)-2x\cos\left( \frac{1}{ x^2+y^2 } \right)\frac{1}{ x^2+y^2 }.
			\end{equation}
			Nous voyons que $\lim_{(x,y)\to(0,0)}\partial_xf(x,y)$ n'existe pas (regarder par exemple la fonction \eqref{EqCDundsdfx} le long du chemin $\gamma(t)=(t,t)$). La dérivée partielle est donc non continue.
			
			
		\item
			En ce qui concerne la continuité, le passage en polaire est conseillé parce que nous avons une fraction de polynômes :
			\begin{equation}
				f(r\cos\theta,r\sin\theta)=r(\cos^3\theta-\sin^3\theta).
			\end{equation}
			Avec les notations de la proposition \ref{PropMethodePolaire}, nous avons donc
			\begin{equation}
				E_r=\{ \rho(\cos^3\theta-\sin^3\theta)\tqs \rho\in\mathopen[ 0 , r \mathclose],\theta\in\mathopen[ 0 , 2\pi \mathclose] \}.
			\end{equation}
			Vu que la fonction $\cos^3\theta-\sin^3\theta$ est une fonction bornée de $\theta$ par exemple par $2$, nous avons
			\begin{equation}
				s_r<2r.
			\end{equation}
			Par conséquent $\lim_{r\to 0} s_r=0$, ce qui prouve que $\lim_{(x,y)\to(0,0)}f(x,y)=0$. La fonction est donc continue.

			En ce qui concerne la dérivabilité,
			\begin{equation}
				\partial_xf(x,y)=\frac{ x^4+3x^2y^2+2xy^3 }{ (x^2+y^2)^2 }.
			\end{equation}
			Cette fonction n'a pas de limite lorsque $(x,y)\to(0,0)$ parce que par exemple pour les chemins $\gamma(t)=(t,kt)$ nous avons
			\begin{equation}
				\partial_xf(t,kt)=\frac{ 2k^3+3k^2+1 }{ k^4+2k^2+1 },
			\end{equation}
			qui dépend clairement de $k$ et donc du chemin choisit. La dérivée par rapport à $x$ n'est donc pas continue en $(0,0)$. Cependant, cette dérivée partielle existe :
			\begin{equation}	\label{EqcddzuiiDifcy}
				\partial_xf(0,0)=\lim_{x\to 0} \frac{ \frac{ x^3 }{ x^2 }-0 }{ x }=1.
			\end{equation}
			De la même façon, $\partial_yf(0,0)=-1$.

		\item
			La fonction $f$ est continue en $(0,0)$ parce que en polaires elle s'écrit
			\begin{equation}
				f(\rho,\theta)=\rho\cos\theta| \sin\theta |.
			\end{equation}
			Cette fonction est toujours en norme plus petite que $\rho$, et par conséquent nous avons $\lim_{(x,y)\to(0,0)}f(x,y)=0$. En ce qui concerne la dérivée, nous avons
			\begin{equation}
				\partial_xf(x,y)=\frac{ | y | }{ \sqrt{x^2+y^2} }\left( 1-\frac{ x^2 }{ x^2+y^2 } \right).
			\end{equation}
			Sur la droite $(t,kt)$ nous avons
			\begin{equation}
				\begin{aligned}[]
					\frac{ | t | |k | }{ | t |\sqrt{1+k^2} }\left( 1-\frac{ t^2 }{ t^2(1+k^2) } \right)&=\frac{ | k | }{ \sqrt{1+k^2} }\left( 1-\frac{1}{ 1+k^2 } \right)\\
					&=\frac{ | k | }{ \sqrt{1+k^2} }\left( \frac{ 1+k^2-1 }{ 1+k^2 } \right)\\
					&=\frac{ | k |k^2 }{ (1+k^2)^{3/2} }.
				\end{aligned}
			\end{equation}
			Cela ne dépend pas de $t$. La limite n'existe donc pas. La dérivée selon $x$ n'est donc pas continue en $(0,0)$.

			La dérivée dans la direction de $x$ au point $(0,0)$ existe et vaut
			\begin{equation}		\label{EqCCDZUpxf}
				\partial_xf(0,0)=\lim_{x\to 0} \frac{ \frac{ 0 }{ \sqrt{x} }-0 }{ x }=0.
			\end{equation}
			En ce qui concerne la dérivée par rapport à $y$ aux point autres que $(0,0)$, nous avons
			\begin{equation}
				\partial_yf(x,y)=\frac{ | y | }{ \sqrt{x^2+y^2} }-\frac{ | y |x^2 }{ (x^2+y^2)^{3/2} }.
			\end{equation}
			La dérivée $\partial_yf(0,0)$ se calcule en utilisant le même type de calcul que celui donné dans l'équation \eqref{EqCCDZUpxf}.

		\item
			Cette fonction qui mélange des fonctions trigonométriques et des polynômes est la candidate parfaite pour utiliser un développement. La théorie du polynôme de Taylor nous enseigne qu'il existe une fonction $a\in o(x^5)$ (c'est à dire $\lim_{x\to 0} \frac{ a(x) }{ x^5 }=0$) telle que
			\begin{equation}
				\begin{aligned}[]
					\sin(x)&=x-\frac{ x^3 }{ 3! }+a(x)\\
					\sin(y)&=y-\frac{ y^3 }{ 3! }+a(y).
				\end{aligned}
			\end{equation}
			En substituant dans la définition de $f$, nous avons
			\begin{equation}
				\begin{aligned}[]
					f(x,y)&=\frac{ xy-\frac{ xy^3 }{3} +xa(y)-yx+\frac{ yx^3 }{ 3 }-ya(x) } { x^2+y^2 }\\
					&=\frac{ xy }{ 6 }\frac{ x^2-y^2 }{ x^2+y^2 }+\frac{ xa(y)-ya(x) }{ x^2+y^2 }.
				\end{aligned}
			\end{equation}
			La limite de la première partie est zéro, comme expliqué dans l'exemple \ref{Exemplexyxsqysq}.

			Pour la seconde partie, nous avons
			\begin{equation}
				0\leq\left| \frac{ xa(y) }{ x^2+y^2 } \right| \leq\left| \frac{ xa(y) }{ y^2 } \right| .
			\end{equation}
			Cela peut être vu comme le produit de la fonction $x$ par la fonction $a(y)/y^2$, dont les limites valent zéro. Nous avons donc montré que
			\begin{equation}
				\lim_{(x,y)\to(0,0)}f(x,y)=0,
			\end{equation}
			et par conséquent que la fonction est continue en $(0,0)$.

			Passons aux dérivées partielles. En ce qui concerne la dérivée $\partial_xf$, nous avons
			\begin{equation}
				\frac{ \partial f }{ \partial x }(x,y)=\frac{ \sin(y)-y\cos(x) }{ x^2+y^2 }+\frac{ 2x\big( x\sin(y)-y\sin(x) \big) }{ (x^2+y^2)^2 }.
			\end{equation}
			Encore une fois nous remplaçons les fonctions trigonométriques par leur développements. Pour rappel, en ce qui concerne le cosinus, il existe une fonction $b$ telle que
			\begin{equation}
				\cos(x)=1-\frac{ x^2 }{2}+b(x)
			\end{equation}
			avec $\lim_{x\to 0} \frac{ b(x) }{ x^4 }=0$. En remplaçant les différentes expressions pour $\cos(x)$, $\sin(x)$ et $\sin(y)$, nous trouvons
			\begin{equation}
				\begin{aligned}[]
					\frac{ \partial f }{ \partial x }(x,y)&=\frac{ y }{ x^2+y^2 }\left( \frac{ x^2 }{ 2 }-\frac{ y^2 }{ 6 } \right)+\frac{ a(y)-yb(x) }{ x^2+y^2 }\\
					&\quad+2x\frac{ xy }{ 6 }\frac{ y^2-x^2 }{ (x^2+y^2)^2 }+2x\frac{ xa(y)-ya(x) }{ (x^2+y^2) }.
				\end{aligned}
			\end{equation}
			Nous pouvons montrer terme à terme que le tout tend vers $0$ lorsque $(x,y)\to (0,0)$.

			En passant aux polaires, le premier terme devient $\frac{1}{ 6 }r\sin(\theta)\big( 3\cos^2(\theta)-\sin^2(\theta) \big)$. Cela est $r$ multiplié par une fonction bornée de $\theta$. La limite est donc $0$. L'autre terme polynomial se traite de la même manière. En ce qui concerne le terme
			\begin{equation}
				\frac{ a(y)-yb(x) }{ x^2+y^2 },
			\end{equation}
			nous avons
			\begin{equation}
				0\leq\left| \frac{ yb(x) }{ x^2+y^2 } \right| <\left| \frac{ yb(x) }{ x^2 } \right| .
			\end{equation}
			La limite de ce qui se trouve à droite est zéro par la propriété de $b(x)$. Au final nous avons prouvé que
			\begin{equation}
				\lim_{(x,y)\to(0,0)}\partial_xf(x,y)=0.
			\end{equation}
			Attention : cela ne prouve pas que $\partial_xf$ soit une fonction continue. Il faut encore vérifier si $\partial_xf(0,0)$ est nul ou non. Cela se fait en utilisant directement la définition :
			\begin{equation}
				\frac{ \partial f }{ \partial x }(0,0)=\lim_{t\to 0} \frac{ f(t,0)-f(0,0) }{ t }.
			\end{equation}
			Étant donné que $f(t,0)-f(0,0)=0$ pour tout $t$, la limite est nulle, ce qui prouve que $\partial_xf(0,0)=0$ et par conséquent que $\partial_xf$ est continue en $(0,0)$.

			Vu que dans la fonction $x$ et $y$ arrivent de façon symétrique, la même chose sera vrai pour $\partial_yf$. Les deux dérivées partielles étant continues en $(0,0)$, la fonction est donc différentiable en $(0,0)$ par la proposition \ref{Diff_totale}.


		\item
			Nous avons
			\begin{equation}
				\left| \frac{ \sin(xy) }{ | x |+| y | } \right| \leq\left| \frac{ \sin(xy) }{ | x | } \right| =| y |\left| \frac{ \sin(xy) }{ xy } \right| 
			\end{equation}
			où nous avons multiplié et divisé par $| y |$ pour faire apparaître la combinaison $\sin(xy)/xy$ dont nous savons que la limite est $1$. Nous avons donc $\lim_{(x,y)\to(0,0)}f(x,y)=0$.

			En ce qui concerne les dérivées partielles, nous rappelons que la dérivée de $| x |$ vaut $\signe(x)$, c'est à dire $1$ si $x$ est positif et $-1$ si $x$ est négatif. Nous avons
			\begin{equation}		\label{EqCD15diffc}
				\partial_xf(x,y)=\frac{ y\cos(xy) }{ | x |+| y | }+\frac{ \signe(x)\sin(xy) }{ (| x |+| y |)^2 }.
			\end{equation}
			Nous utilisons encore une fois les développements. La majorité des termes tendent vers zéro; par exemple,
			\begin{equation}
				\frac{ y\cos(xy) }{ | x |+| y | }=\frac{1}{ | x |+| y | }\left( y-\frac{ y^3x^2 }{2}+b(xy) \right).
			\end{equation}
			En passant aux polaires, le terme central disparaît, et le terme en $b(xy)$ disparaît parce que, proche de $(0,0)$, nous avons évidement $|xy|<|x|$ et par conséquent $| b(xy) |<b(x)$. En faisant le même genre de jeux avec l'autre terme de \eqref{EqCD15diffc}, nous devons calculer
			\begin{equation}
				\lim_{(x,y)\to(0,0)}\frac{ y }{ | x |+| y | }+\signe(x)\frac{ xy }{ (| x |+| y |)^2 }.
			\end{equation}
			Si on passe en coordonnées polaires, on voit que cette limite dépend de $\theta$, et donc n'existe pas. La dérivée partielle $\partial_xf$ n'est en particulier pas continue en $(0,0)$. Le nombre $\partial_xf(0,0)$ existe pourtant :
			\begin{equation}
				\partial_xf(0,0)=\lim_{x\to 0} \frac{ f(x,0)-f(0,0) }{ x }=\lim_{x\to 0} \frac{ \frac{ \sin(0) }{ | x |+0 }-0 }{ 0 }=0.
			\end{equation}
			

		\item
			Pour cette fonction la zone à problème n'est pas le point $(0,0)$. Les points à étudier sont les points sur le cercle $x^2+y^2=1$.

			Heureusement, l'exercice est assez vite fait parce que la fonction donnée est à symétrie radiale. En coordonnées polaires, un point du cercle est de la forme $(1,\theta)$. 
            
            Un bon exemple de chemin, en polaires, est $\big( (1-t),\theta \big)$. Note : le chemin choisit doit aller vers le point à tester lorsque $t\to 0$. Ici, le chemin $\big( (1-t),\theta \big)$ tend vers le point $(1,\theta)$ qui est bien un point sur le cercle, c'est à dire un point à tester. Dans les coordonnées polaires, nous avons $x^2+y^2=r$, et le long de notre chemin nous avons $r=1-t$. Nous avons donc
			\begin{equation}
                f(r,\theta)= \begin{cases}
                    e^{1/(r^2-1)}    &   \text{si } r<1\\
                    0    &    \text{si } r\geq 1.
                \end{cases}
			\end{equation}
            Nous avons
            \begin{equation}        \label{eqSYoHJK}
                \lim_{r\to 1^-} e^{1/(r^2-1)}=0,
            \end{equation}
            par conséquent la fonction est continue. Insistons sur le fait que la limite \eqref{eqSYoHJK} est calculée avec \( r<1\). Si nous avions calculé \( r\to 1\), la limite n'aurait pas existé.

            En ce qui concerne la dérivée par rapport à \( x\), nous devons considérer trois zones bien distinctes du plan. La première est constituée des points intérieurs au cercle \( r=1\), la seconde est le cercle lui-même et la troisième est l'extérieur du cercle.
            
            Commençons par considérer un point \( (x,y)\) du cercle \( r=1\) et calculons
            \begin{equation}
                \frac{ \partial f }{ \partial x }(x,y)=\lim_{\epsilon\to 0}\frac{ f(x+\epsilon,y)-f(x,y) }{ \epsilon }.
            \end{equation}
            Nous supposons \( x>0\) (le même calcul sera vrai avec \( x<0\)). Alors \( f(x+\epsilon,y)=0\) tant que \( \epsilon>0\); nous devons donc calculer
            \begin{equation}
                \frac{ \partial f }{ \partial x }(x,y)=\lim_{\epsilon\to 0^-} e^{1/((x+\epsilon)^2+y^2-1 )}=0.
            \end{equation}
            La dérivée partielle par rapport à \( x\) vaut donc zéro sur le pourtour du cercle. 
            
            Si \( (x,y)\) est à l'extérieur du cercle, nous avons bien évidemment \( \partial_xf(x,y)=0\).

            Pour \( (x,y)\) à l'intérieur du cercle nous utilisons les règles de dérivation pour obtenir
            \begin{equation}
                \frac{ \partial f }{ \partial x }(x,y)=2x\frac{  e^{1/(x^2+y^2-1)} }{ (x^2+y^2-1) }.
            \end{equation}
            La continuité de \( \partial_xf\) consiste à voir si cette expression tend vers zéro lorsque \( (x,y)\) tend vers un point du cercle. 

            En passant aux polaires, la limite en suivant la droite du rayon du cercle revient à étudier la limite
            \begin{equation}
                \lim_{r\to 1^-} \frac{  e^{1/(r^2-1)} }{ r^2-1 }.
            \end{equation}
            En passant au logarithme,
            \begin{equation}
                \ln\left( \frac{  e^{1/(r^2-1)} }{ r^2-1 } \right)=\frac{1}{ r^2-1 }-\ln(r^2-1).
            \end{equation}
            Une façon de voir cette limite est de considérer le chemin \( r(\epsilon)=\sqrt{1-\epsilon}\) (avec \( \epsilon>0\)) et de regarder la limite
            \begin{equation}
                \lim_{\epsilon\to0}\frac{1}{ \epsilon }-\ln(\epsilon).
            \end{equation}
            En mettant au même dénominateur et en appliquant la règle de l'Hospital,
            \begin{equation}
                \lim_{\epsilon\to0}\frac{1}{ \epsilon }-\ln(\epsilon)=\lim_{\epsilon\to 0^+}\frac{ 1-\epsilon\ln\epsilon }{ \epsilon }=\lim_{\epsilon\to 0}-\ln(\epsilon)-1=+\infty.
            \end{equation}
            Nous avons donc vu que en calculant la limite \( (x,y)\to (1,\theta_0)\)\footnote{Ici \( (x,y)\) est le point qui bouge, donné en cartésiennes et \( (1,\theta_0)\) avec \( \theta_0\) fixé est un point du cercle donné en polaires.} le long du le chemin
            \begin{equation}
                \gamma(t)=\big( r(t),\theta(t) \big)=( \sqrt{1-t},\theta_0 ),
            \end{equation}
            nous avons
            \begin{equation}
                \lim_{t\to 0} \frac{ \partial f }{ \partial x }\big( \gamma(t) \big)=+\infty.
            \end{equation}
            La dérivée partielle dans la direction de \( x\) n'est donc pas continue.

		\item
			La fonction $y^2\sin\left( \frac{ x }{ y } \right)$ a des problèmes sur tous les points du type $(a,0)$. Pour voir la continuité, nous devons donc calculer
			\begin{equation}
				\lim_{(x,y)\to(a,0)}y^2\sin\left( \frac{ x }{ y } \right).
			\end{equation}
			Nous traitons cela de façon classique :
			\begin{equation}
				0\leq| f(x,y) |\leq | y^2 |\to 0.
			\end{equation}
			Par la règle de l'étau nous avons la continuité de la fonction.

			Il s'agit maintenant d'étudier la continuité des dérivées partielles. La dérivée partielle par rapport à $x$ vaut
			\begin{equation}
				\frac{ \partial f }{ \partial x }(x,y)=\begin{cases}
					y\cos\frac{ x }{ y }	&	\text{si }y\neq 0\\
					0	&	 \text{sinon.}
				\end{cases}
			\end{equation}
			Pour la seconde ligne, nous avons utilisé le calcul direct
			\begin{equation}
				\frac{ \partial f }{ \partial x }(a,0)=\lim_{t\to 0} \frac{ f(a+t,0)-f(a,0) }{ t }=0.
			\end{equation}
			Par le même raisonnement que précédemment, $\lim_{(x,y)\to(a,0)}y\cos\frac{ x }{ y }=0$, donc la dérivée $\partial_xf$ est continue aux points $(a,0)$.

			Pour la dérivée partielle par rapport à $y$ nous avons
			\begin{equation}
				\frac{ \partial f }{ \partial y }(x,y)=\begin{cases}
					2y\sin\left( \frac{ x }{ y } \right)-x\cos\left( \frac{ x }{ y } \right)	&	\text{si }y\neq 0\\
					0	&	 \text{sinon.}
				\end{cases}
			\end{equation}
			Encore une fois, la seconde ligne provient d'un calcul utilisant directement la définition :
			\begin{equation}
				\frac{ \partial f }{ \partial y }(a,0)=\lim_{t\to 0} \frac{ f(a,t)-f(a,0) }{ t }=\lim_{t\to 0} t\sin\left( \frac{ a }{ t } \right)=0.
			\end{equation}
			Afin de voir la continuité de $\partial_yf$, il faut simplement calculer la limite $\lim_{(x,y)\to(a,0)}\frac{ \partial f }{ \partial y }(x,y)$ et voir si c'est égal à $0$ :
			\begin{equation}		\label{EqcCDusellcy}
				\lim_{(x,y)\to(a,0)}2y\sin\frac{ x }{ y }-x\cos\frac{ x }{ y }.
			\end{equation}
			Le premier terme tend vers $0$, tandis que le second est un produit d'une fonction qui tend vers $a$ (la fonction $x$) par la fonction $\cos\frac{ x }{ y }$ qui n'a pas de limite. Si $a\neq 0$, la limite \eqref{EqcCDusellcy} n'existe pas. La fonction $\partial_yf$ est donc continue en $(0,0)$.

			En résumé, la domaine de continuité de $\partial_yf$ est $\eR^2\setminus\{ y=0 \}\cup\{ (0,0) \}$.
			
			

		\item
			Lorsque nous devons calculer une limite d'une fonction exposant une autre fonction, le passage au logarithme est toujours une bonne idée : 
			\begin{equation}
				(x^2+y^2)^x= e^{x\ln(x^2+y^2)}.
			\end{equation}
			En passant aux polaires, il est vite vu que
			\begin{equation}
				\lim_{(x,y)\to(0,0)}x\ln(x^2+y^2)=0,
			\end{equation}
			et donc $\lim_{(x,y)\to(0,0)}f(x,y)=e^0=1$. La fonction est donc continue en $(0,0)$.

			En ce qui concerne la dérivée par rapport à $x$ par contre, elle n'existe même pas en $(0,0)$ parce que
			\begin{equation}
				\frac{ \partial f }{ \partial x }(0,0)=\lim_{t\to 0} \frac{ f(t,0)-f(0,0) }{ t }=\lim_{t\to 0} \frac{ (t^2)^t-1 }{ t }=\lim_{t\to 0} \frac{  e^{t\ln(t^2)}-1 }{ t }.
			\end{equation}
			Cette limite se fait en utilisant la règle de l'Hospital. La dérivée du numérateur vaut
			\begin{equation}
				\big( \ln(t^2)+2 \big) e^{t\ln(t^2)},
			\end{equation}
			dont la limite n'existe pas lorsque $t\to 0$.

		\item
			En tant que composée de fonctions continues, la fonction $f(x,y)=\sin(| xy |)$ est continue. Pour la dérivée, nous avons
			\begin{equation}
				\partial_xf(x,y)=
				\begin{cases}
					y\cos(xy)	&	\text{si }xy>0\\
					-y\cos(xy)	&	\text{si }xy<0\\
					?		&	\text{dans les autres cas}.
				\end{cases}
			\end{equation}
			Nous n'avons pas besoin de savoir ce qu'il se passe dans les autres cas (ce sont les points sur les axes). En effet, si nous essayons de voir la limite vers le point $(0,1)$ en suivant le chemin $(t,1)$, la limite n'existe pas parce que nous avons $1$ lorsqu'on considère $t>0$ et $-1$ avec $t<0$.

			Remarque : la fonction $f$ peut être vue comme la composée de $(x,y)\mapsto xy$, de $t\mapsto| t |$ et de $u\mapsto\sin(u)$. La fonction valeur absolue n'étant pas dérivable en $0$, on pouvait s'attendre à ce que la fonction $f$ ne soit pas dérivable aux points où $xy=0$.

		\item
			Dans cet exercice, les points à étudier sont ceux du cercle $x^2+y^2=1$. La continuité se règle en coordonnées polaires parce que la fonction vaut simplement $f(\rho,\theta)=1-\rho^2$ dont la limite vaut bien $0$ lorsque $\rho\to 1$.

			En ce qui concerne la dérivée par rapport à $x$ nous avons
			\begin{equation}
				\partial_xf(x,y)=\begin{cases}
					-2x	&	\text{si }x^2+y^2<1\\
					0	&	 \text{si }x^2+y^2>1\\
					?	&	\text{dans les autres cas}.
				\end{cases}
			\end{equation}
			Cette fonction n'est évidement pas continue sur les points du cercle tels que $x\neq 0$.

		\item
			Les points de raccordement entre les deux zones sont les points $(-a,a)$ et $(a,a)$ avec $a\geq 0$. Évidement, $x^2=y^2$ en tous ces points, donc la fonction est continue. Pour les dérivées,
			\begin{equation}
				\partial_xf(x,y)=\begin{cases}
					2x	&	\text{si }| x |>y\\
					0	&	 \text{sinon.}
				\end{cases}
			\end{equation}
			Parmi les points du type $(-a,a)$ et $(a,a)$, seul au point $(0,0)$ cette fonction est continue.
			
	\end{enumerate}

\end{corrige}
