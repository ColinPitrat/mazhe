% This is part of Analyse Starter CTU
% Copyright (c) 2014
%   Laurent Claessens,Carlotta Donadello
% See the file fdl-1.3.txt for copying conditions.

\begin{corrige}{autoanalyseCTU-37}

  \begin{enumerate}
  \item On connaît la solution générale de l'équation homogène associé à $y''+2y'+2y=6e^{-x}$ pour l'avoir trouvée dans l'exercice \ref{exoautoanalyseCTU-36}. Ici notre préoccupation principale sera donc de trouver une solution particulière de l'équation non homogène. Comme le terme de droite de l'équation est une exponentielle nous allons chercher une solution de la forme $y_p(x) =Ce^{-x} $. On obtient 
    \begin{equation*}
      Ce^{-x} -2Ce^{-x}+2Ce^{-x} = 6e^{-x}, 
    \end{equation*}
    ce qui implique que $C = 6$. La solution particulière $y_p$ est donc $y_p = 6 e^{-x}$. La solution générale de l'équation en examen est alors la somme entre la solution générale de l'équation homogène associée, \eqref{exoautoanal36part2}, et $y_p$ : 
    \begin{equation*}
      \mathcal{Y}  = \left\{ e^{-x}\left(C_1\cos(2x) +C_2\sin(2x)+ 6\right)  \,:\, C_1,\, C_2 \in \eR \right\}.
    \end{equation*} 

les contions initiales fixées nous permettent d'écrire le système de deux équation en deux inconnues $C_1$ et $C_2$ 
\begin{equation*}
  \begin{cases}
    C_1+6= 0,  & \quad\text{ qui correspond à la condition }y(0)=0,\\
    -(C_1+6) + 2C_2 = 0 & \quad\text{ qui correspond à la condition }y'(0)=0.
  \end{cases}
\end{equation*}
On trouve alors $C_1 = -6$ et $C_2 = 0$, et $y(x) = 6e^{-x}\left(1-\cos(2x)\right)$.
   \item \begin{enumerate}
  \item Le polyn\^ome caractéristique de l'équation homogène $y''-2y'+5y=0$ est $r^2-2r+5$, dont les racines sont les nombres complexes conjugués $1+i2$ et $1-i2$. La solution générale de l'équation homogène est donc $\displaystyle \mathcal{Y}_h  = \left\{ e^{x}\left(C_1\cos(2x) +C_2\sin(2x)\right)  \,:\, C_1,\, C_2 \in \eR \right\}.$

On cherche une solution particulière $y_p$ de l'équation non homogène $y''-2y'+5y=\cos(x)$. Comme le membre de droite est la fonction cosinus, $y_p(x)$ sera  de la forme $A\cos(x) + B\sin(x)$. En injectant $y_p$ dans l'équation nous obtenons 
\begin{equation*}
  -\left(A\cos(x) + B\sin(x)\right) -2 \left(B\cos(x) -A\sin(x)\right) + 5\left(A\cos(x) + B\sin(x)\right) = \cos(x),
\end{equation*}
qui correspond au système de deux équations pour les deux inconnues $A$ et $B$
\begin{equation*}
  \begin{cases}
    -A-2B +5A =1 ,& \quad\text{ ces sont les coefficients de cosinus, }\\
    -B+2A +5B = 0 ,& \quad\text{ ces sont les coefficients de sinus. }
  \end{cases}
\end{equation*}
On a alors $A=1/5$ et $B=-1/10$, et $y_p = \displaystyle \frac{1}{5} \cos(x) -\frac{1}{10}\sin(x)$.

La solution générale de l'équation  $y''-2y'+5y=\cos(x)$ est la somme de $\mathcal{Y}_h$ et $y_p$, c'est à dire 
\begin{equation*}
  \mathcal{Y} = \left\{ e^{x}\left(C_1\cos(2x) +C_2\sin(2x)\right)+ \frac{1}{5} \cos(x) -\frac{1}{10}\sin(x)  \,:\, C_1,\, C_2 \in \eR \right\}.
\end{equation*}

Les conditions initiales fixées nous donnent le système suivant pour trouver les valeurs de $C_1$ et $C_2$ correspondants à la solution particulière demandée :
\begin{equation*}
  \begin{cases}
    C_1+\frac{1}{5}= 0,  & \quad\text{ qui correspond à la condition }y(0)=0,\\
    C_1+2C_2-\frac{1}{10} = 0 & \quad\text{ qui correspond à la condition }y'(0)=0.
  \end{cases}
\end{equation*}
Donc $C_1= -1/5$ et $C_2 = 3/20$ et $y(x) = e^{x}\left(-\frac{1}{5}\cos(2x) +\frac{3}{20}\sin(2x)\right)+ \frac{1}{5} \cos(x) -\frac{1}{10}\sin(x)$.
 \item[(b)] On a établi plus t\^ot dans l'exercice que la solution générale de l'équation homogène est donc $\displaystyle \mathcal{Y}_h  = \left\{ e^{x}\left(C_1\cos(2x) +C_2\sin(2x)\right)  \,:\, C_1,\, C_2 \in \eR \right\}.$

On cherche une solution particulière $y_p$ de l'équation non homogène $y''-2y'+5y=x$. Comme le membre de droite est un polyn\^one de degré un, $y_p(x)$ sera  de la forme $Ax + B$. En injectant $y_p$ dans l'équation nous obtenons 
\begin{equation*}
  -2A + 5\left(Ax+B\right) = x,
\end{equation*}
qui correspond au système de deux équations pour les deux inconnues $A$ et $B$
\begin{equation*}
    5A =1 ,\text{ et, }-2A +5B = 0,
\end{equation*}
On a alors $A=1/5$ et $B=2/25$, et $y_p = \displaystyle \frac{1}{5}x +\frac{2}{25}$.

La solution générale de l'équation  $y''-2y'+5y=x$ est la somme de $\mathcal{Y}_h$ et $y_p$, c'est à dire 
\begin{equation*}
  \mathcal{Y} = \left\{ e^{x}\left(C_1\cos(2x) +C_2\sin(2x)\right)+ \frac{1}{5}x +\frac{2}{25} \ \,:\, C_1,\, C_2 \in \eR \right\}.
\end{equation*}

Les conditions initiales fixées nous donnent le système suivant pour trouver les valeurs de $C_1$ et $C_2$ correspondants à la solution particulière demandée :
\begin{equation*}
  \begin{cases}
    C_1+\frac{2}{25}= 0,  & \quad\text{ qui correspond à la condition }y(0)=0,\\
    C_1+2C_2+\frac{1}{5} = 0 & \quad\text{ qui correspond à la condition }y'(0)=0.
  \end{cases}
\end{equation*}
Donc $C_1= -2/25$ et $C_2 = -3/50$ et $y(x) = e^{x}\left(-\frac{2}{25}\cos(2x) -\frac{3}{50}\sin(2x)\right)+ \frac{1}{5}x+\frac{2}{25}$.
  \item[(c)] Pour obtenir une solution particulière de cette équation il suffit de sommer les solutions particulières trouvées pour les équations des points (Z.a) et (2.b) de cet exercice (on exploite ici le fait que l'équation soit linéaire). La solution générale de cette équation est donc 
 \begin{equation*}
  \mathcal{Y} = \left\{ e^{x}\left(C_1\cos(2x) +C_2\sin(2x)\right)+ \frac{1}{5} \cos(x) -\frac{1}{10}\sin(x)+\frac{1}{5}x +\frac{2}{25} \ \,:\, C_1,\, C_2 \in \eR \right\}.
\end{equation*}
Les conditions initiales fixées nous donnent le système suivant pour trouver les valeurs de $C_1$ et $C_2$ correspondants à la solution particulière demandée :
\begin{equation*}
  \begin{cases}
    C_1+\frac{7}{25}= 0,  & \quad\text{ qui correspond à la condition }y(0)=0,\\
    C_1+2C_2+\frac{1}{10} = 0 & \quad\text{ qui correspond à la condition }y'(0)=0.
  \end{cases}
\end{equation*}
Donc $C_1= -7/25$ et $C_2 = 9/100$ et 
\[y(x) = e^{x}\left(-\frac{7}{25}\cos(2x) +\frac{9}{100}\sin(2x)\right)+ \frac{1}{5} \cos(x) -\frac{1}{10}\sin(x)+ \frac{1}{5}x+\frac{2}{25}.
\]
  \end{enumerate}
  \end{enumerate}

\end{corrige}   
