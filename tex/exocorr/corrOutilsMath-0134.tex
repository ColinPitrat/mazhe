% This is part of Outils mathématiques
% Copyright (c) 2012
%   Laurent Claessens
% See the file fdl-1.3.txt for copying conditions.

\begin{corrige}{OutilsMath-0134}

    Le calcul du rotationnel donne immédiatement \( \nabla \times F=0\). En ce qui concerne l'intégrale, le plus simple est de chercher un potentiel (qui existe parce que le rotationnel est nul). Nous trouvons le potentiel
    \begin{equation}
        f(x,y,z)=xy+x\cos(y).
    \end{equation}
    Nous avons alors
    \begin{equation}
        \int_{\sigma}F=f(\sqrt{\pi},0,1)-f(0,0,1)=\sqrt{\pi}.
    \end{equation}

\end{corrige}
