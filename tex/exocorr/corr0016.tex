% This is part of Exercices et corrigés de CdI-1
% Copyright (c) 2011, 2019
%   Laurent Claessens
% See the file fdl-1.3.txt for copying conditions.

\begin{corrige}{0016}

La première chose à remarquer est que la suite $x_n=\sigma(n)$ tend vers l'infini. Maintenant, si nous désignons par $a$ la limite de la suite $(a_n)$, et si nous prenons $\epsilon>0$, nous avons un $N$ tel que $n>N$ implique $| a_n-a |\leq\epsilon$. Prenons $K$ tel que $k>K$ implique $\sigma(k)>N$. Dans ce cas, $| b_k-a |=| a_{\sigma(k)}-a |\leq\epsilon$, ce qui prouve que $(b_n)$ converge vers la même limite que $(a_n)$.


\begin{alternative}

\begin{remark}
  Pour résoudre cet exercice, il faut bien comprendre la définition de la limite~: celle-ci dit qu'à $\epsilon$ fixé, il n'y a qu'un nombre fini de termes de la suite qui sont « $\epsilon$-loin » de la limite. Toute la question de l'exercice est de savoir combien de ces termes « problématiques » on met dans $b_n$. L'injectivité de $\sigma$ assure que chaque terme « problématique » ne peut apparaitre qu'une fois au maximum, et donc qu'ils sont en nombre fini.
\end{remark}

Notons $a$ la limite de la suite $(a_n)$, et montrons que
$(b_n \pardef a_{\sigma(n)})$ converge vers $a$. Par définition de la
convergence de $(a_n)$, on sait que
\begin{equation*}
  \forall \epsilon > 0\, \exists K^\prime \tq (k^\prime \geq K^\prime \Rightarrow \abs{a_{k^\prime} - a} < \epsilon)
\end{equation*}
Pour $\epsilon > 0$ fixé, notons
\begin{equation*}
  E = \left\{ i \,\middle\vert\, \sigma(i) < K^\prime \right\}
  \quad\text{et}\quad K = 1 + \max E
\end{equation*}
qui existe bel et bien, car l'ensemble $E$ est fini d'après
l'injectivité de $\sigma$. Si $k \geq K$, alors $k \not\in E$ et donc
$\sigma(k) \geq K^\prime$ et
\begin{equation*}
  \abs{b_k - a} = \abs{a_{\sigma(k)} - a} < \epsilon
\end{equation*}
ce qui prouve la convergence de $(b_k)$ vers $a$.
\end{alternative}

\end{corrige}
