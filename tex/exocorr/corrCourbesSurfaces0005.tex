\begin{corrige}{CourbesSurfaces0005}

	\begin{enumerate}
		\item
			\newcommand{\CaptionFigCardioideexo}{La cardioïde de l'exercice \ref{exoCourbesSurfaces0005}.\ref{CSCi}.}
			\input{auto/pictures_tex/Fig_Cardioideexo.pstricks}
                        La courbe est périodique de période $2\pi$. La fonction $r(\theta)$ est symétrique au sens que $r(\theta)=r(2\pi-\theta)$. Cela est une conséquence du fait que la fonction $\cos(t)$ est paire. Donc nous trouvons une courbe symétrique par rapport à l'axe $Ox$. Nous avons ensuite $r(\pi)=0$, ce qui nous pousse à étudier la courbe seulement sur l'intervalle $[0,\pi]$.
			Nous avons $r(0)=2$ et $r(\pi/2)=1$; cela nous fait déjà deux points. Comme la dérivée $r'(\theta)=-\sin(\theta)$ est négative sur $[0,\pi]$, la distance entre la courbe et l'origine est monotone décroissante. Il est intéressant de remarquer que $r(\frac{ \pi }{ 4 })=1+\frac{1}{ \sqrt{2} }>\sqrt{2}$, ce qui fait que la courbe passe \emph{au dessus} du carré qui passerait par $(1,1)$, voir figure \ref{LabelFigCardioideexo}.

		\item
			Une première chose à faire est de trouver quelles sont les valeurs acceptables de $\theta$. Vu que $r$ doit être positif, nous devons avoir $\theta\in[0,\pi/2]$ or $\theta\in[3\pi/2,2\pi]$. La fonction $r(\theta)$ est périodique de période $\pi$, donc la courbe correspondant au deuxième intervalle n'est que une rotation de $\pi$ de la première courbe. Pour cette raison, nous considérons ici seulement le cas $\theta\in[0,\pi/2]$. La fonction $r$ est monotone croissante sur $\theta\in[0,\pi/4]$ et décroissante sinon. Nous avons $r(0)=0$, $r(\frac{ \pi }{ 4 })=1$ et $r(\frac{ \pi }{2})=0$. La courbe est dessinée sur la figure \ref{LabelFigCSCii}

			\newcommand{\CaptionFigCSCii}{La courbe de l'exercice \ref{exoCourbesSurfaces0005}.\ref{CSCii}.}
			\input{auto/pictures_tex/Fig_CSCii.pstricks}

			Notez qu'en $\theta=\pi/4$ nous avons $r=1$, et que par conséquent la courbe passe au-dessous du point $(1,1)$.
		\item
		  La courbe ici n'est pas périodique. Pour décrire de façon efficace une courbe en coordonnées polaires il faut permettre au paramètre $\theta$ de prendre ses valeurs dans tout $\eR$ et pas seulement dans l'intervalle $[0,2\pi]$, par contre le rayon $r$ ne pourra jamais être négatif. Le domaine de définition de la fonction $r$ est donc l'union de tous les intervalles sur lesquels la fonction $\sin$ est positive moins le point $\theta=0$
                  \begin{equation}
                    \textrm{Domaine}_r=\{[2k\pi, (2k+1)\pi]\,:\, k\in\eZ\}\setminus\{0\}.
                  \end{equation}
			Étant donné que $r(-\theta)=r(\theta)$, la courbe est symétrique par rapport à l'axe $Ox$. Nous pouvons donc nous concentrer sur les valeurs positives de $\theta$. Notre courbe se divisera donc en une infinité de branches. La première sera \( \mathopen] 0 , \pi \mathclose]\), la seconde sera \( \mathopen[ 2\pi , 3\pi \mathclose]\), etc.
            
            Le «point de départ» de la la première branche est \( \theta=0\). Certes, \( r(0)\) n'est pas défini, mais la limite existe :
            \begin{equation}
                \lim_{\theta\to 0} r(\theta)=\lim_{\theta\to 0} \frac{ \sin\theta }{ \theta }=1.
            \end{equation}
            La courbe part donc du point \( (1,0)\).

            
            
            
            En ce qui concerne la dérivée de $r$, nous avons
			\begin{equation}
				r'(\theta)=\frac{ \cos(\theta) }{ \theta }-\frac{ \sin(\theta) }{ \theta^2 }.
			\end{equation}
                        Cette fonction n'est jamais nulle dans l'intervalle $]0,\pi]$, par conséquent on sait que $r$ sera une fonction monotone (décroissante) sur la première branche de la courbe.  Calculons quelques valeurs. D'abord, $\lim_{\theta\to 0}r(\theta)=1$, $r(\frac{ \pi }{2})=\frac{ 2 }{ \pi }<1$ et $r(\pi)=0$. Nous avons aussi $r(\frac{ \pi }{ 4 })=\frac{ 2\sqrt{2} }{ \pi }<1$.
		        La fonction $r'(\theta)$ est a un zéro dans chaque intervalle de la forme $[2k\pi, (2k+1)\pi]$ pour $k>0$, en fait $r(n\pi)=0$ pour tout $n>0$ dans $\eN$. Trouver ces points critiques est un peu plus difficile, mais on comprend sans difficulté est que la valeur du maximum locale de $r$ est un nombre de plus en plus petit. Ces branches sont dont des courbes de longueur décroissante.  

			\newcommand{\CaptionFigCSCiii}{La courbe de l'exercice \ref{exoCourbesSurfaces0005}.\ref{CSCiii}. Les deux premières branches sont dessinées. Notez que la seconde est beaucoup plus petite que la première, et qu'elle part de \( (0,0)\).}

            Le résultat est sur la figure \ref{LabelFigCSCiii}.
            \input{auto/pictures_tex/Fig_CSCiii.pstricks}
            See also the subfigure \ref{LabelFigCSCiiissLabelSubFigCSCiii0}
            See also the subfigure \ref{LabelFigCSCiiissLabelSubFigCSCiii1}
            See also the subfigure \ref{LabelFigCSCiiissLabelSubFigCSCiii2}

		\item

			La première chose à faire est d'étudier le signe de la fonction $r(\theta)$ et de considérer le domaine où elle est positive. Un tableau de signe montre que $r$ est positive lorsque $\theta\in\mathopen] -\infty , -1 \mathclose[\cup\mathopen[ 1 , \infty [$. Commençons par le second morceau. 

			Pour $\theta=1$ nous avons $r(\theta)=0$. La courbe commence donc à l'origine et part directement selon un angle d'un peu moins de $90$ degrés ($1$ radian est moins de l'angle droit). Ensuite, le rayon n'arrête pas d'augmenter, et la limite $\lim_{r\to \infty} r(\theta)=1$. Nous avons donc une spirale qui devient de plus en plus dense lorsqu'elle s'approche du rayon $1$.

			En ce qui concerne l'autre partie, nous avons $\lim_{r\to -1^{-}} r(\theta)=\infty$, donc la courbe part vers l'infini en tendant vers la droite d'angle $-1$. D'autre par la limite $\lim_{r\to -\infty} r(\theta)=1$. La courbe est donc une spirale qui vient s'accumuler autour du cercle de rayon $1$ (mais depuis l'extérieur, cette fois).
			
			Cette double spirale est dessinée à la figure \ref{LabelFigCSCiv}.
			\newcommand{\CaptionFigCSCiv}{La courbe de l'exercice \ref{exoCourbesSurfaces0005}.\ref{CSCiv}.}
			\input{auto/pictures_tex/Fig_CSCiv.pstricks}

		\item

            %The result is on figure \ref{LabelFigCSCv}. % From file CSCv
\newcommand{\CaptionFigCSCv}{La courbe de l'exercice \ref{exoCourbesSurfaces0005}.\ref{CSCv}.}
\input{auto/pictures_tex/Fig_CSCv.pstricks}

			La chose la plus compliquée dans cet exercice est en réalité de tracer la courbe $r(\theta)$ dans le plan $(r,\theta)$. Elle est tracée à la figure \ref{LabelFigCSCvssLabelSubFigCSCv0}. Nous trouvons les points d'annulation de $r(\theta)$ en remplaçant $\cos(2\theta)$ par $\sin^2(\theta)-\cos^2(\theta)=1-2\cos^2(\theta)$. Ainsi nous avons l'équation
			\begin{equation}
				r(\theta)=-2\cos^2(\theta)+\cos(\theta)+1.
			\end{equation}
			En posant $y=\cos(\theta)$, cette fonction s'annule pour $y=1$ ou $y=-\frac{ 1 }{2}$, ce qui correspond aux valeurs $\theta=0$, $\theta=2\pi/3$ et $\theta=4\pi/3$.

			La courbe demandée est sur la figure \ref{LabelFigCSCvssLabelSubFigCSCv1}.

		\item

%The result is on figure \ref{LabelFigCSCvi}. % From file CSCvi
\newcommand{\CaptionFigCSCvi}{La courbe de l'exercice \ref{exoCourbesSurfaces0005}.\ref{CSCv}.}
\input{auto/pictures_tex/Fig_CSCvi.pstricks}

			%\newcommand{\CaptionFigSubfiguresCSCvi}{}
			%\input{auto/pictures_tex/Fig_SubfiguresCSCvi.pstricks}

			Étant donné que le dénominateur est toujours positif (ou nul), le domaine de variation acceptable de $\theta$ est la domaine sur lequel $\cos(\theta)$ est positif. Nous allons donc étudier la courbe pour $\theta\colon -\frac{ \pi }{2}\to \frac{ \pi }{2}$. Lorsque $\theta$ tend vers $-\frac{ \pi }{2}$ par des valeurs supérieures, $r(\theta)$ tend vers l'infini tandis que lorsque $\theta$ vaut zéro, le rayon $r(0)=1$.

		La fonction $r(\theta)$ est tracée sur la figure \ref{LabelFigCSCvissLabelSubFigCSCvi0}, et la courbe paramétrique correspondante est sur la figure \ref{LabelFigCSCvissLabelSubFigCSCvi1}.

	\end{enumerate}
	
\end{corrige}
