\begin{corrige}{_I-2-4}

Nous pouvons supposer que $z(s)\neq h$ en dehors de $s=s_1$, et nous faisons le changement de variable $\epsilon=s_1-s$. Nous obtenons donc l'intégrale
\begin{equation}
	\int_{s_1-s_0}^0\frac{ -d\epsilon }{ \sqrt{ 2g(h-z(s_1-\epsilon)) } }
\end{equation}
dans laquelle nous remplaçons (en vertu du théorème 1, page 206 du cours de première)
\begin{equation}
	z(s_1-\epsilon)=z(s_1)+a\epsilon+b\epsilon^2+o(\epsilon^2)
\end{equation}
où $a=z'(s_1)$ et $b=z''(s_1)^2/2$. Lorsque $a\neq 0$ et que $\epsilon$ est petit,
\begin{equation}
	\frac{1}{ \sqrt{ 2g( a\epsilon+b\epsilon^2+o(\epsilon^2)  ) } }<\frac{ M }{ \sqrt{\epsilon} }
\end{equation}
pour une certaine constante $M$. Nous déduisons donc que l'intégrale existe lorsque la dérivée première de $z(s)$ en $s_1$ n'est pas nulle. Pour le même genre de raisons, l'intégrale n'existe pas quand $f'(s_1)=0$, ce qui correspond à une situation où la courbe $z(s)$ « s'arrête » sur la valeur $h$.


\end{corrige}
% This is part of the Exercices et corrigés de CdI-2.
% Copyright (C) 2008, 2009
%   Laurent Claessens
% See the file fdl-1.3.txt for copying conditions.


