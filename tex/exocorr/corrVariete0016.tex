% This is part of Exercices et corrigés de CdI-1
% Copyright (c) 2011-2011
%   Laurent Claessens
% See the file fdl-1.3.txt for copying conditions.

\begin{corrige}{Variete0016}

	\begin{enumerate}

		\item
			Lorsqu'on ne précise rien, c'est qu'on demande que le chemin est dans le « bon » sens. Il n'y a donc pas de problèmes de sens, on peut appliquer le théorème de Green les yeux fermés. Tout ce que nous avons à faire est d'intégrer la fonction
			\begin{equation}
				\frac{ \partial Q }{ \partial x }-\frac{ \partial P }{ \partial y }
			\end{equation}
			où $P(x,y)=2(x^2+y^2)$ et $Q(x,y)=(x+y)^2$ sur le triangle. Cela se fait ainsi :
			\begin{equation}
				2\int_1^2dy\int_1^ydx(x-y)=-\frac{ 1 }{ 3 }.
			\end{equation}
			
		\item
			Juste pour rappel, le bon sens est le sens trigonométrique. Si $D$ est le disque, l'intégrale à faire est
			\begin{equation}
				-4\int_Dxy\,dx\,dy=\int_0^{2\pi}\int_0^R r^2\cos\theta\sin\theta drd\theta=0.
			\end{equation}
			
		\item
			La forme est $\omega=dx+xdy$, donc $P=1$ et $Q=x$, ce qui donne que la fonction à intégrer sur la surface est $1$. 
            %Le domaine est dessiné à la figure 
			%\ref{LabelFigSqrtCarre}
			%\newcommand{\CaptionFigSqrtCarre}{Le domaine d'intégration pour l'exercice \ref{exoVariete0016}.}
			%\input{auto/pictures_tex/Fig_SqrtCarre.pstricks}
			L'intégrale est donc
			\begin{equation}
				\int_0^2\int_{x^2}^{\sqrt{x}}dydx=\frac{1}{ 3 }.
			\end{equation}

		\item
			Nous avons repéré au point précédent que la forme $\omega =dx+xdy$ donnait $1$ comme fonction en prenant le passage à Green. Ici, nous devons intégrer la fonction $1$ sur l'ellipse. Nous allons donc intégrer la forme $\omega=dx+xdy$ sur le contour. Le chemin (dans le bon sens !) qui paramètre l'ellipse est donné par
			\begin{equation}
				\gamma(t)=\begin{pmatrix}
					a\cos(t)	\\ 
					b\sin(t)	
				\end{pmatrix}
			\end{equation}
			où $t\colon 0\to 2\pi$. La dérivée du chemin (son vecteur tangent) est donné par
			\begin{equation}
				\gamma'(t)=\begin{pmatrix}
					-a\sin(t)	\\ 
					b\cos(t)	
				\end{pmatrix}.
			\end{equation}
			Nous devons donc intégrer
			\begin{equation}
				\begin{aligned}[]
					\int_0^{2\pi}(dx+a\cos(t)dy)\begin{pmatrix}
						-a\sin(t)	\\ 
						b\cos(t)	
					\end{pmatrix}
					&=\int_0^{2\pi}\big( -a\sin(t)+ab\cos(t) \big)dt\\
					&=ab\int_0^{2\pi}\cos^2(t)dt\\
					&=\pi ab.
				\end{aligned}
			\end{equation}

		\item
			Cette fois, le sens de parcours est prescrit, et il est le sens inverse de celui où Green fonctionne. Nous devons donc ajouter un signe :
			\begin{equation}
				\int_{\gamma}-x^2ydx+xy^2dy=-\int_D(x^2+y^2)dxdy=-\int_0^1\int_0^{2\pi}r^3d\theta dr=-\frac{ \pi }{ 2 }.
			\end{equation}
			

	\end{enumerate}
\end{corrige}
