% This is part of Exercices de mathématique pour SVT
% Copyright (C) 2010
%   Laurent Claessens et Carlotta Donadello
% See the file fdl-1.3.txt for copying conditions.

\begin{corrige}{DS2010-1-0001}

	\begin{enumerate}
		\item
			La racine $\sqrt{x-1}$ demande $x-1\geq 0$, donc pour la fonction $f$, le domaine est $x\in\mathopen[ 1, +\infty  \mathclose[$.

			En ce qui concerne la fonction $g$, il faut penser à deux choses. La première est le dénominateur : $\ln(x)\neq 0$. La seconde est le logarithme : $x>0$.

			La première condition demande $x\neq 1$. Donc en combinant les deux, nous avons tous les nombres strictement plus grands que zéro, sauf $1$ : $x\in\mathopen] 0 , \infty \mathclose[\setminus\{ 1 \}$.

		\item
			\begin{enumerate}
				\item
					D'abord nous avons
					\begin{equation}
						(f\circ g)(x)=f\big( g(x) \big)=\sqrt{g(x)-1}=\sqrt{\frac{1}{ \ln(x) }-1}.
					\end{equation}
					Ensuite nous regardons les conditions pour le domaine. Il y en a trois parce qu'il y a une racine carrée, un dénominateur et un logarithme.
					\begin{enumerate}
						\item
							Pour la racine carrée, $\frac{1}{ \ln(x) }-1\geq 0$, ce qui donne l'inéquation
							\begin{equation}
								\frac{1}{ \ln(x) }\geq 1.
							\end{equation}
							Attention : le logarithme peut être négatif. Il n'est donc pas exact de faire directement passer le $\ln(x)$ de l'autre côté :
							\begin{equation}
								1\geq\ln(x).
							\end{equation}
							Ce passage n'est vrai que si $\ln(x)>0$. Mais si $\ln(x)<0$, nous n'avons certainement pas $\frac{1}{ \ln(x) }\geq 1$. Par conséquent, les conditions sont $1\geq\ln(x)$ {\bf et} $\ln(x)>0$. Cela donne $x>1$ et $x\leq e$.
						\item
							Pour le dénominateur, $\ln(x)\neq 0$ implique $x\neq 1$.
						\item
							Pour le logarithme, nous devons avoir $x>0$.

					\end{enumerate}
					Maintenant nous devons regarder l'intersection de ces trois conditions. Nous devons avoir $x>1$, $x>0$, $x\leq e$ et $x\neq 1$. Cela fait donc
					\begin{equation}
						x\in\mathopen] 1 , e \mathclose].
					\end{equation}
				\item
					Nous avons
					\begin{equation}
						(g\circ f)(x)=g\big( f(x) \big)=\frac{1}{ \ln\big( f(x) \big) }=\frac{1}{ \ln\big( \sqrt{x-1} \big) }.
					\end{equation}
					Encore une fois, nous avons trois conditions : un dénominateur, un logarithme et une racine carrée.
					\begin{enumerate}
						\item
							Pour le dénominateur, il faut $\ln\big( \sqrt{x-1} \big)\neq 0$, ce qui donne $\sqrt{x-1}\neq 1$, et donc $x-1\neq 1$, et finalement $x\neq 2$.
						\item
							Pour la racine, nous demandons $x-1\geq 0$, et par conséquent $x\geq 1$.
						\item
							Pour le logarithme, nous avons besoin de $\sqrt{x-1}>0$, c'est-à-dire $x> 1$. Pour rappel, la valeur d'une racine carrée n'est jamais négative, mais elle est nulle sur l'argument est zéro.
					\end{enumerate}
					En remettant tout ensemble, nous devons avoir simultanément $x> 1$, $x\neq 2$ et $x\geq 1$, c'est-à-dire
					\begin{equation}
						x\in\mathopen] 1 , \infty \mathclose[\setminus\{ 2 \}.
					\end{equation}
					
			\end{enumerate}
	\end{enumerate}

\end{corrige}
