% This is part of Exercices de mathématique pour SVT
% Copyright (c) 2010-2011
%   Laurent Claessens et Carlotta Donadello
% See the file fdl-1.3.txt for copying conditions.

\begin{corrige}{TD6b-0002}

    La technique de base est toujours la même : écrire \( y'\) sous la forme \( dy/dt\) et puis rassembler d'un côté les termes en \( y\) et de l'autre ceux en \( t\).

    \begin{enumerate}
        \item
            

            Nous avons successivement
            \begin{subequations}
                \begin{align}
                    \frac{ dy }{ dt }=t^3+t,\\
                    e^ydy=(t^3+t)dt\\
                    e^y=\frac{ t^4 }{ 4 }+\frac{ t^2 }{2}+C.
                \end{align}
            \end{subequations}
            La dernière ligne est simplement le fait d'avoir intégré des deux côtés. Nous isolons \( y\) en passant au logarithme :
            \begin{equation}
                y(t)=\ln\left( \frac{ t^4 }{ 4 }+\frac{ t^2 }{2}+C \right).
            \end{equation}


            Une méthode alternative à résoudre soi-même est de demander à un ordinateur de le faire :
            \begin{verbatim}
----------------------------------------------------------------------
| Sage Version 4.7.1, Release Date: 2011-08-11                       |
| Type notebook() for the GUI, and license() for information.        |
----------------------------------------------------------------------
sage: t=var('t')
sage: y=function('y',t)
sage: DE=diff(y,t)-(t**3+t)*exp(-y)
sage: desolve(DE,[y,t])
e^y(t) == 1/4*t^4 + 1/2*t^2 + c + 1/4            
            \end{verbatim}

            Notez qu'il ne termine pas complètement le travail parce qu'il ne fait pas le logarithme pour isoler \( y\). Par contre, il faut avouer que le gros est fait. Notez aussi qu'un peu naïvement, il écrit \( c+\frac{1}{ 4 }\) alors que \( c\) est une constante arbitraire; on peut simplement écrire \( c\).

        \item

            Nous avons à intégrer les deux membres de
            \begin{equation}
                \frac{ dy }{ 1+y^2 }=dt,
            \end{equation}
            ce qui donne \( \arctan(y)=t+C\). La solution est
            \begin{equation}
                y(t)=\tan(t+C).
            \end{equation}
            Tant qu'aucun contexte n'est donné, nous n'avons pas grand chose à dire sur le domaine (et pourtant il y aurait des choses à dire !).

        \item

            Nous devons intégrer
            \begin{equation}
                (1+e^y)dy=\cos(t)dt.
            \end{equation}
            La solution est
            \begin{equation}
                y+e^y=\sin(t)+C.
            \end{equation}
            Il n'y a pas de moyen simples de résoudre cette équation pour isoler \( y\). Voici donc un exemple d'équation différentielle que nous ne pouvons pas résoudre explicitement.

        \item

            À intégrer : 
            \begin{equation}
                \frac{ dy }{ y^2 }=dt,
            \end{equation}
            solution :
            \begin{equation}
                y=-\frac{1}{ t+C }.
            \end{equation}
            
        \item

            Dans le même ordre d'idée que le précédent :
            \begin{verbatim}
----------------------------------------------------------------------
| Sage Version 4.7.1, Release Date: 2011-08-11                       |
| Type notebook() for the GUI, and license() for information.        |
----------------------------------------------------------------------
sage: t=var('t')
sage: y=function('y',t)
sage: DE=diff(y,t)-y**(1/3)
sage: desolve(DE,[y,t])
3/2*y(t)^(2/3) == c + t
            \end{verbatim}
            
            Sage nous dit que
            \begin{equation}
                \frac{ 3 }{2}y^{2/3}=C+t
            \end{equation}
            La solution recherchée est donc
            \begin{equation}
                y(t)=\left( \frac{ 2 }{ 3 }(C+t) \right)^{-3/2}.
            \end{equation}
            
    \end{enumerate}

\end{corrige}
