% This is part of Outils mathématiques
% Copyright (c) 2011
%   Laurent Claessens
% See the file fdl-1.3.txt for copying conditions.

\begin{corrige}{OutilsMath-0125}

        Nous avons 
        \begin{equation}
            f(x,y)=x^2-2ixy+(iy)^2=x^2-y^2-2ixy.
        \end{equation}
        La partie réelle est donc bien \( x^2-y^2\) et la partie imaginaire est \( 2xy\). Le champ de vecteurs que nous regardons est donc
        \begin{equation}
            F(x,y)=\begin{pmatrix}
                x^2-y^2    \\ 
                -2xy    
            \end{pmatrix}.
        \end{equation}
        La divergence est
        \begin{equation}
            \nabla\cdot F=\frac{ \partial F_1 }{ \partial x }+\frac{ \partial F_2 }{ \partial y }=2x-2x=0.
        \end{equation}
        Le rotationnel est 
        \begin{equation}
            \nabla\times F=\begin{vmatrix}
                e_x    &   e_y    &   e_z    \\
                \partial_x    &   \partial_y    &   \partial_z    \\
                x^2-y^2    &   -2xy    &   0
            \end{vmatrix}=(-2y+2y)e_z=0.
        \end{equation}

\end{corrige}
