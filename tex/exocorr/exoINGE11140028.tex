% This is part of Exercices et corrigés de mathématique générale.
% Copyright (C) 2009-2010
%   Laurent Claessens
% See the file fdl-1.3.txt for copying conditions.


\begin{exercice}\label{exoINGE11140028}

	Calculer les limites suivantes ou prouver leur non-existence.
	\begin{enumerate}

		\item
			\begin{equation}
				\lim_{x\to\infty}\frac{ \sin(x)\cos(x) }{ x }
			\end{equation}
		\item
			\begin{equation}
				\lim_{x\to\infty}\frac{ x+\sin(x) }{ x-\sin(x) }
			\end{equation}
		\item
			\begin{equation}
				\lim_{x\to \infty} \sin(x)\frac{ x^2+2x+3 }{ x }.
			\end{equation}

	\end{enumerate}

	Comment résoudre cet exercice avec Sage ? Les lignes de code à rentrer pour résoudre la première limite sont dans l'ordre
	\begin{enumerate}

		\item	\label{ItemSageGenda}
			\texttt{var('x')}
		\item\label{ItemSageGendb}
			\texttt{f(x)=sin(x)*cos(x)/x}
		\item\label{ItemSageGendc}
			\texttt{limit(f(x),x=oo)}

	\end{enumerate}
	La ligne \ref{ItemSageGenda} déclare que la lettre \texttt{x} désignera une variable. La ligne \ref{ItemSageGendb} définit la fonction $f$. Notez que la multiplication est notée par la petite étoile \texttt{*}. Pour la ligne \ref{ItemSageGendc}, notez que l'infini est écrit par deux petits « o ».

	La fonction $f(x)=x^2+2x+3$ s'encode avec \texttt{f(x)=x**2+2*x+3}. Le symbole pour faire l'exposant est la double étoile \texttt{**}.

\corrref{INGE11140028}
\end{exercice}
