% This is part of Exercices de mathématique pour SVT
% Copyright (c) 2011,2015
%   Laurent Claessens and Carlotta Donadello
% See the file fdl-1.3.txt for copying conditions.

\begin{corrige}{TD6A-0003}

    \begin{enumerate}
        \item

            Soit \( g\) une fonction qui satisfait \( g''(t)=-g(t)\). Si nous posons
            \begin{subequations}
                \begin{align}
                    x(t)&=g(t)\\
                    y(t)&=g'(t)
                \end{align}
            \end{subequations}
            alors le couple \( (x,y)\) est une solution du système
            \begin{subequations}        \label{SubesqoSZWaq}
                \begin{numcases}{}
                    x'(t)=y\\
                    y'(t)=-x.
                \end{numcases}
            \end{subequations}
            En effet d'une part \( x'(t)=g'(t)=y(t)\), et d'autre part \( y'(t)=dg'/dt=g''(t)=-g(t)=x(t)\).

            Dans le sens inverse si le couple de \emph{fonctions}\footnote{Nous n'insisterons jamais assez sur le fait que \( x\) et \( y\) sont ici des fonctions.} \( (x,y)\) satisfait au système \eqref{SubesqoSZWaq}, alors en posant \( g(t)=x(t)\) nous obtenons une fonction \( g\) qui satisfait \( g''=-g\).

        \item

            Si \( x\) est une fonction constante, \( x'=0\) et la première équation du système donne \( y=0\). À ce moment, la seconde équation du système donne \( x(t)=0\). La seule solution constante est \( (0,0)\).

        \item

            En dérivant nous obtenons successivement :
            \begin{subequations}
                \begin{align}
                    x(t)&=A\cos(t)+B\sin(t)\\
                    x'(t)&=-A\sin(t)+B\cos(t)\\
                    x''(t)&=-A\cos(t)-B\sin(t).
                \end{align}
            \end{subequations}
            Nous avons donc \( x''=-x\) et la fonction proposée est donc une solution de l'équation.

        \item

            La trajectoire à dessiner est la courbe d'équation paramétrique
            \begin{subequations}
                \begin{numcases}{}
                    x(t)=A\cos(t)+B\sin(t)\\
                    y(t)=-A\sin(t)+B\cos(t)
                \end{numcases}
            \end{subequations}
            
            Le calcul montre que pour tout \( t\) nous avons
            \begin{equation}
                x(t)^2+y(t)^2=A^2+B^2,
            \end{equation}
            ce qui montre que nous avons un cercle de rayon \( \sqrt{A^2+B^2}\).

            Les trajectoires possibles sont donc de deux natures : d'abord il y a la trajectoire constante \( (x,y)=(0,0)\) et ensuite il y a des cercles de tous les rayons.

            Encore une fois remarquons que les constantes \( A\) et \( B\) ne jouent pas un rôle mineur. En effet la solution avec \( A=2\), \( B=3\) est :
            \begin{subequations}
                \begin{numcases}{}
                    x_1(t)=2\cos(t)+3\sin(t)\\
                    y_1(t)=-2\sin(t)+3\cos(t).
                \end{numcases}
            \end{subequations}
            C'est un cercle de rayon \( \sqrt{13}\). La solution avec \( A=3\), \( B=2\) est
            \begin{subequations}
                \begin{numcases}{}
                    x_2(t)=3\cos(t)+2\sin(t)\\
                    y_2(t)=-3\sin(t)+2\cos(t).
                \end{numcases}
            \end{subequations}
            C'est le même cercle ! Il y a par contre une différence de point de départ :
            \begin{equation}
                \begin{pmatrix}
                    x_1(0)    \\ 
                    y_1(0)    
                \end{pmatrix}=\begin{pmatrix}
                    2    \\ 
                    3    
                \end{pmatrix}
            \end{equation}
            tandis que
            \begin{equation}
                \begin{pmatrix}
                    x_2(0)    \\ 
                    y_2(0)    
                \end{pmatrix}=\begin{pmatrix}
                    3    \\ 
                    2    
                \end{pmatrix}.
            \end{equation}
            Ces deux solutions parcourent la même orbite, mais ne partent pas du même points !

            Fixer les constante \( A\) et \( B\) permet de fixer à la fois le rayon de l'orbite et son point de départ.

        \item

            Demander \( x(0)=1\) impose \( A=1\), mais \( B\) reste libre. Nous avons mis quelques trajectoires sur la figure \ref{LabelFigTrajs}.
            \newcommand{\CaptionFigTrajs}{Les trajectoires sont tracées pour \( t\) entre \( 0\) et \( 1\). Les points correspondent à \( t=0\). La courbe rouge correspond à \( B=-1\). Nous voyons qu'elle est sur la même trajectoire que \( B=1\), mais elle ne part pas du même point.}
            \input{auto/pictures_tex/Fig_Trajs.pstricks}

    \end{enumerate}
    

\end{corrige}
