\begin{corrige}{CourbesSurfaces0002}

Pour esquisser  le graphe d'une courbe paramétrique il faut d'abord étudier 
\let\OldTheenumi\theenumi
\renewcommand{\theenumi}{\roman{enumi}}
\begin{enumerate}
	\item les ensembles de définition des fonctions $x(t)$ et $y(t)$, 

        \item la périodicité, la parité de ces fonctions, 

        \item le sens de variation de $x$ et de $y$, 

        \item les points particuliers correspondant à des valeurs remarquables de $t$, les tangentes en ces points (s'il y en a), 

        \item les éventuels points d'inflexion et/ou de rebroussement, 

        \item les branches infinies (s'il y en a) et l'existence des asymptotes.
		
\end{enumerate}
\let\theenumi\OldTheenumi

Parfois, si la courbe est compliquée,  il faut aussi calculer en quelques points la tangente et la normale. La normale est la perpendiculaire à la tangente, et comme expliqué dans la section \ref{SecTracerParmCourbe}, la concavité est tournée dans le sens de la normale dirigée dans le même sens que la dérivée seconde.

  \begin{enumerate}
      \item \label{Itemzzdexoi}
      Les fonctions $x(t)$ et $y(t)$ sont définies sur le même domaine $\eR\setminus\{0\}$. Cela veut dire que les équations décrivent l'union de deux courbes. La fonction $x(t)$ est une fonction impaire et $x(t)=x(1/t)$. La fonction $y(t)$ n'a pas de symétries évidentes. L'étude des dérivées 
    \begin{equation}
      \begin{aligned}
        x'(t)= 1-\frac{1}{t^2},\\
        y'(t)= 1-\frac{1}{t^3},
      \end{aligned}
    \end{equation}
nous donne plusieurs indices. La fonction $x$ est croissante pour $|t|>1$ et décroissante si $|t|<1$. La fonction $y$ est croissante pour $t<0$ et pour $t>1$, sinon elle est décroissante. Le point $t=1$ est le minimum global pour les deux fonctions. Cela veut dire que la courbe de support $]0, +\infty[$ n'est jamais ni à gauche de  $x=2$ ni au dessus de $y=3/2$. Le point $t=-1$ est un point de maximum global pour la fonction $x$ (la courbe passe par le point $(-2, -1/2)$). La courbe de support $]-\infty, 0[$ n'est jamais à droite de $x=-2$.  

Les limites aux extrêmes du domaine sont les suivantes
\begin{equation}
  \begin{array}{ll}
    \lim_{t\to 0^{\pm}}x(t)=\pm \infty, & \lim_{t\to \pm\infty}x(t)=\pm \infty,\\
    \lim_{t\to 0^{\pm}}y(t)=+ \infty, & \lim_{t\to \pm\infty}y(t)=\pm \infty.\\
  \end{array}
\end{equation}
Les courbes ne sont donc pas bornées. On peut étudier leur comportement asymptotique. Si $t$ est très grand les termes $1/t$ et $1/(2t^2)$ sont très faibles. La courbe ressemble fort à la droite $x=y$ lorsque $t$ est grand. La même chose arrive quand $t$ est négatif et $|t|$ est très grand. 

Au contraire, si $t$ et positif et proche de zéro, les termes $1/t$ et $1/(2t^2)$ deviennent dominants. Le comportement de la courbe est alors très proche du comportement de la fonction $y=x^2/2$ pour $x>0$. Si $t$ est négatif et de valeur absolue petite alors la courbe ressemble encore beaucoup à $y=x^2/2$, pour $x<0$. Bien que $y(t)$ ne soit  pas une fonction paire elle devient «presque» paire là où son terme dominant est $1/(2t^2)$, qui est pair.

Le résultat est sur la figure \ref{LabelFigCbCartTui}.
\newcommand{\CaptionFigCbCartTui}{La courbe de l'exercice \ref{exoCourbesSurfaces0002}\ref{Itemzzdexoi}}
\input{auto/pictures_tex/Fig_CbCartTui.pstricks}

Une difficulté de l'exercice est de déterminer, pour les \( t\) entre zéro et un, quelle branche est au dessus de l'autre ? Comment justifier le sens des flèches dans la figure \ref{LabelFigCbCartTuii} ?

Si \( \ell>2\), il y a deux valeurs de \( t\in\mathopen] 0 , \infty \mathclose[\) pour lesquelles \( x(t)=\ell\); elles sont données par
\begin{equation}
    \begin{aligned}[]
        t_1=\frac{ \ell+\sqrt{\ell^2-4} }{ 2 }>1,&&t_2=\frac{ \ell-\sqrt{\ell^2-4} }{ 2 }<1.
    \end{aligned}
\end{equation}
Nous pouvons calculer:
\begin{equation}
    y(t_1)-y(t_2)=\sqrt{\ell^2-4}+\frac{ -2\ell\sqrt{\ell^2-4} }{ (\ell^2-2)^2-\ell^2(\ell^2-4) }=\sqrt{\ell^2-4}-\frac{ \ell\sqrt{\ell^2-4} }{ 2 }.
\end{equation}
Étant donné que \( \ell>2\), le second terme est plus grand que le premier et la différence est négative. Par conséquent,
\begin{equation}
    y(t_1)<y(t_2),
\end{equation}
et donc la branche des \( t>1\) est en dessous de la branche des \( y<1\), ce qui justifie le sens des flèches.


\item \label{Itemzzdexoii}
    Le domaine des fonctions $x(t)$ et $y(t)$ est $\eR$. Les deux fonctions sont  périodique de période $\pi$.  Pour le voir il suffit de remarquer que $2y(t)= x(t)\sin(2t)$. Étant donnée la périodicité,  nous considérons seulement la portion du domaine $[0,\pi]$. On peut déjà calculer $x(0)=1$ et $y(0)=0$. La fonction $x$ est toujours positive. Cela veut dire que la courbe est toute à droite de l'axe des $y$. La fonction $y$ est positive pour $t$ dans $[0,\pi/2]$ et négative pour $t$ dans $[\pi/2, \pi]$.

  Les dérivées sont 
  \begin{equation}
      \begin{aligned}
        x'(t)&= -2\cos(t)\sin(t),\\
        y'(t)&= -3\cos^2(t)\sin^2(t)+ \cos^4(t).
      \end{aligned}
    \end{equation}
On remarque que la droite tangente à la courbe est verticale pour $t=0$ et $t=\pi$.
 
La valeur $t=\pi/2$ est critique pour les deux fonctions, $x(\pi/2)=y(\pi/2)=0$. La fonction $x$ est décroissante pour $t\in ]0,\pi/2[$ et croissante pour $t\in]\pi/2, \pi [$. La valeur $t=\pi/2$ n'est pas une valeur de maximum ou de minimum pour $y$.

Trouvons les autres points critiques de la fonction \( y\), c'est à dire résolvons \( y'(t)=0\). Les points correspondants à \( \cos(t)=0\) ayant déjà été discutés, nous pouvons simplifier par \( \cos^2(t)\) et résoudre
\begin{equation}
    -3\sin^2(t)+\cos^2(t)=0.
\end{equation}
Il y a au moins deux façons de procéder. La première consiste à remplacer \( \cos^2(t)\) par \( 1-\sin^2(t)\) et puis résoudre par rapport à \( \sin(t)\). Dans ce cas nous tombons sur\footnote{Nous nous sommes restreints à \( t\in\mathopen[ 0 , \pi \mathclose]\).} \( \sin(t)=\pm\frac{1}{ 2 }\) et par conséquent \( t_1=\pi/6\) et \( t_2=5\pi/6\). La seconde méthode consiste à encore diviser par \( \cos^2(t)\) et tomber sur \( \tan^2(t)=\frac{1}{ 3 }\). Nous trouvons évidement les mêmes solutions.

Un calcul montre que $y$ est croissante sur  $[0, t_1]$ et $[t_2,\pi/2]$, sinon elle est décroissante.

    Lorsque $t$ s'approche de $\pi/2$ la valeur $\sin(t)$ s'approche de $1$. Cela veut dire que $y(t)$ devient presque $\cos^3(t)$ et la courbe ressemble à $y=x^{3/2}$ si $\cos(t)$ est positif et à  $y=-x^{3/2}$ si $\cos(t)$ est négatif.  

Le résultat est à la figure \ref{LabelFigCbCartTuii}.                                                                   
\newcommand{\CaptionFigCbCartTuii}{La courbe de l'exercice \ref{Itemzzdexoii}.}
\input{auto/pictures_tex/Fig_CbCartTuii.pstricks}

  %\item
  %\item
  %\item
  %\item
  \item\label{Itemzzdexoiii}
      La période de $x(t)$ est $\pi$ et la période de $y(t)$ est $2\pi/3$. La période de la courbe qu'on va tracer sera alors $2\pi$ qui est le plus petit multiple commun et non nul de $\pi$ et de $2\pi/3$.

      Afin de simplifier les calculs, nous pouvons remarquer que
      \begin{subequations}
          \begin{align}
              x(t+\pi)&=x(t)\\
              y(t+\pi)&=-y(t).
          \end{align}
      \end{subequations}
      La courbe a donc une symétrie par rapport à l'axe horizontal, et il est suffisant d'étudier la courbe entre \( 0\) et \( \pi\). Le reste se déduisant par symétrie. Voir la figure \ref{LabelFigExoParamCD}.
      \newcommand{\CaptionFigExoParamCD}{Le graphique de la courbe de l'exercice \ref{exoCourbesSurfaces0002}\ref{CourbSDvii}. La courbe en bleu représente les valeurs du paramètre entre \( 0\) et \( \pi\) tandis qu'en rouge nous avons tracé le paramètre entre \( \pi\) et \( 2\pi\).}
    \input{auto/pictures_tex/Fig_ExoParamCD.pstricks}

    Dans cette correction nous allons cependant faire «comme si» nous n'avions pas remarqué la symétrie.
      
      Une simple, mais longe, étude des signes de $x$ et $y$ nous dit que la courbe $\gamma(t)=(x(t), y(t))$ sera dans le premier quadrant pour $t$ contenu dans l'union des intervalles 
\[
I_1=\left[0,\frac{\pi}{3}\right],\qquad I_6=\left[\frac{4\pi}{3}, \frac{3\pi}{2}\right]. 
\]
De m\^me $\gamma(t)$ est dans le deuxième quadrant si $t$ est dans
\[
I_4=\left[\frac{2\pi}{3},\pi\right],\qquad I_8=\left[\frac{5\pi}{3}, 2\pi\right],
\]
$\gamma(t)$ est dans le troisième quadrant si $t$ est dans
\[
I_3=\left[\frac{\pi}{2},\frac{2\pi}{3}\right],\qquad I_7=\left[\frac{3\pi}{2}, \frac{5\pi}{3}\right],
\]
et enfin $\gamma(t)$ est dans le quatrième quadrant si $t$ est dans
\[
I_2=\left[\frac{\pi}{3},\frac{\pi}{2}\right],\qquad I_5=\left[\pi, \frac{4\pi}{3}\right].
\]

Il faut donc trouver tous les $9$ points d'intersection de $\gamma$ avec les axes 
\begin{equation}
  \begin{array}{lll}
    \gamma(0)=
    \begin{pmatrix}
      0\\
      0
    \end{pmatrix} & \gamma\left(\frac{\pi}{3}\right)= 
    \begin{pmatrix}
      \sin\left(\frac{2\pi}{3}\right)\\
      0
    \end{pmatrix} & \gamma\left(\frac{\pi}{2}\right)= 
    \begin{pmatrix}
      0\\
      -1
    \end{pmatrix}\\
    \gamma\left(\frac{2\pi}{3}\right)= 
    \begin{pmatrix}
      \sin\left(\frac{4\pi}{3}\right)\\
      0
    \end{pmatrix} & \gamma(\pi)=
    \begin{pmatrix}
      0\\
      0
    \end{pmatrix} & \gamma\left(\frac{4\pi}{3}\right)= 
    \begin{pmatrix}
      \sin\left(\frac{8\pi}{3}\right)\\
      0
    \end{pmatrix} \\
     \gamma\left(\frac{3\pi}{2}\right)= 
    \begin{pmatrix}
      0\\
      1
    \end{pmatrix} & \gamma\left(\frac{5\pi}{3}\right)= 
    \begin{pmatrix}
      \sin\left(\frac{10\pi}{3}\right)\\
      0
    \end{pmatrix} & \gamma(2\pi)=
    \begin{pmatrix}
      0\\
      0
    \end{pmatrix} 
  \end{array}
\end{equation}
  L'étude des dérivées premières 
  \begin{equation}
    \left\{\begin{array}{l}
      x'(t)= 2\cos(2t),\\
      y'(t)= 3\cos(3t),
    \end{array}\right.
  \end{equation}
  nous dit que la fonction $x$ a quatre points critiques dans $[0,2\pi]$ 
  \begin{equation}
    \begin{array}{llll}
      \frac{\pi}{4} (\textrm{max}) & \frac{3\pi}{4} (\textrm{min}) & \frac{5\pi}{4} (\textrm{max}) & \frac{7\pi}{4} (\textrm{min}),
    \end{array}
  \end{equation}
  et que la fonction $y$ en a six
  \begin{equation}
    \begin{array}{lll}
      \frac{\pi}{6} (\textrm{max}) & \frac{\pi}{2} (\textrm{min}) & \frac{5\pi}{6} (\textrm{max}) \\ 
      \frac{7\pi}{6} (\textrm{min}) & \frac{3\pi}{2} (\textrm{max}) & \frac{11\pi}{6} (\textrm{min}).
    \end{array}
  \end{equation}
  Les point critiques de $x$ correspondent aux points de $\gamma$ où la droite tangente est verticale. De même  les point critiques de $y$ correspondent aux points de $\gamma$ où la droite tangente est horizontale. Pour bien tracer la courbe il faut calculer le points de $\gamma$ pour tous ces dix valeurs de $t$ (omis). 

Il faut aussi trouver les droites tangentes à $\gamma$ aux points d'intersection avec les axes. Ces information sont suffisantes pour esquisser $\gamma$. 


Le résultat est sur la figure \ref{LabelFigCbCartTuiii}.
\newcommand{\CaptionFigCbCartTuiii}{Le dessin de l'exercice \ref{Itemzzdexoiii}.}
\input{auto/pictures_tex/Fig_CbCartTuiii.pstricks}

  \end{enumerate}



%	Pour tracer, 
%Nous avons tracé sur les figures
%\ref{LabelFigSubfiguresCDUTraceCDun},
%la tangente, la normale et la dérivée seconde en quelques points des différents exercices.

%\newcommand{\CaptionFigSubfiguresCDUTraceCDun}{Les éléments-clefs pour la courbe de l'exercice \ref{exoCourbesSurfaces0002}.\ref{CourbSDi} avec $t<0$.}
%\input{auto/pictures_tex/Fig_SubfiguresCDUTraceCDun.pstricks}
	
%The result is on figure \ref{LabelFigSubfiguresCDUTraceCDdeux}
%\newcommand{\CaptionFigSubfiguresCDUTraceCDdeux}{Les éléments-clefs pour la courbe de l'exercice \ref{exoCourbesSurfaces0002}.\ref{CourbSDii}. Ne pas oublier la symétrie verticale de la fonction.}
%\input{auto/pictures_tex/Fig_SubfiguresCDUTraceCDdeux.pstricks}

%See also the subfigure \ref{LabelFigSubfiguresCDUTraceCDdeuxssSS1TraceCDdeux}
%See also the subfigure \ref{LabelFigSubfiguresCDUTraceCDdeuxssSS2TraceCDdeux}

\end{corrige}
