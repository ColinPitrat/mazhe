% This is part of Exercices et corrigés de CdI-1
% Copyright (c) 2011,2015
%   Laurent Claessens
% See the file fdl-1.3.txt for copying conditions.

\begin{corrige}{Variete0012}

	\begin{enumerate}

		\item
			La sphère est une variété de dimension $2$ dans $\eR^3$. Nous avons donc besoin de cartes pour intégrer dessus. Les coordonnées sphériques sont une carte qui recouvrent toute la sphère sauf un bout de mesure nulle (voir page $486$). Nous allons donc considérer
			\begin{equation}
				\begin{aligned}
					F\colon \eR^+\setminus\{ 0 \}\times\mathopen] 0 , 2\pi \mathclose[\times\mathopen] 0 , \pi \mathclose[&\to \eR^3 \\
					(r,\theta,\varphi)&\mapsto \begin{pmatrix}
						R\cos(\theta)\sin(\varphi)	\\ 
						R\sin(\theta)\sin(\varphi)	\\ 
						R\cos\varphi	
					\end{pmatrix}.
				\end{aligned}
			\end{equation}
			Les vecteurs tangents à cette paramétrisation sont
			\begin{equation}
				\begin{aligned}[]
					\frac{ \partial F }{ \partial \theta }&=\begin{pmatrix}
						-R\sin\theta\sin\varphi	\\ 
						R\cos\theta\sin\varphi	\\ 
						0	
					\end{pmatrix},
					&\frac{ \partial F }{ \partial \varphi }&=\begin{pmatrix}
						R\cos\theta\cos\varphi	\\ 
						R\sin\theta\cos\varphi	\\ 
						-R\sin\varphi	
					\end{pmatrix},
				\end{aligned}
			\end{equation}
			et l'élément de surface est la norme de 
			\begin{equation}
				\begin{aligned}[]
					\partial_\theta F\times\partial_{\varphi}F
					&=\begin{vmatrix}
						e_1				&	e_2				&	e_3	\\
						-R\sin\theta\sin\varphi		&	R\cos\theta\sin\varphi		&	0	\\
						R\cos\theta\cos\varphi		&	R\sin\theta\cos\varphi		&	-R\sin\varphi
					\end{vmatrix}\\
					&=(-R^2\cos\theta\sin^2\varphi)e_1-(R^2\sin\theta\sin^2\varphi)e_2+(-R^2\sin\varphi\cos\varphi)e_3.
				\end{aligned}
			\end{equation}
			La norme du tout vaut
			\begin{equation}
				d\sigma(\theta,\varphi)=\| \partial_{\theta}F\times\partial_{\varphi}F \|=R^2\sin\varphi.
			\end{equation}
			Donc l'intégrale à calculer est
			\begin{equation}
				\int_0^{2\pi}\int_0^{\pi}R^2\sin\varphi\,d\varphi d\theta=2\pi R^2[-\cos\varphi]_0^{\pi}=4\pi R^2.
			\end{equation}

		\item
			Un point $(x,y)$ est sur le cercle lorsque $(x-\frac{ R }{ 2 })^2+y^2=\frac{ R^2 }{ 4 }$. En passant aux coordonnées polaires, l'intérieur du cercle est
			\begin{equation}
				r^2-Rr\cos\theta<0,
			\end{equation}
			que l'on peut simplifier par $r$ parce que $r$ est toujours strictement positif. Le cylindre (plein) dont on parle est donc donné par
			\begin{equation}
				r<R\cos\theta
			\end{equation}
			en coordonnées cylindriques. En coordonnées cylindriques, la sphère s'écrit
			\begin{equation}
				r^2+z^2=R^2.
			\end{equation}
			Ce sur quoi nous intégrons est le morceau de sphère au dessus (et en dessous) du cercle. Il est donc naturel d'utiliser ce dernier pour paramétriser la surface sur laquelle on veut intégrer. La carte est donc
			\begin{equation}		\label{EqParmsCylSphere}
				F(r,\theta)=\begin{pmatrix}
					r\cos\theta	\\ 
					r\sin\theta	\\ 
					\sqrt{R^2-r^2}	
				\end{pmatrix}
			\end{equation}
			à prendre sur l'ouvert $r<R\cos\theta$. Un peu de calcul montre que
			\begin{equation}
				\| \partial_rF\times\partial_{\theta}F \|=\frac{ rR }{ \sqrt{R^2-r^2} }.
			\end{equation}
			Il faut intégrer cela avec $\theta\in\mathopen] -\pi/2 , \pi/2 \mathclose[$ et $r\in\mathopen] 0 , R\cos\theta \mathclose[$ :
			\begin{equation}
				I=\int_{-\pi/2}^{\pi/2}\int_0^{R\cos\theta}\frac{ rR }{ \sqrt{R^2-r^2} }drd\theta=R^2(\pi-2).
			\end{equation}
			Ce résultat doit encore être multiplié par deux pour tenir compte de la partie de surface en dessous. Le résultat est donc
			\begin{equation}
				2R^2(\pi-2).
			\end{equation}
			
		\item
			Une carte pour le cône (à part le sommet) est donnée par le cercle sur lequel il se projette :
			\begin{equation}
				F(r,\theta)=\begin{pmatrix}
					r\cos\theta	\\ 
					r\sin\theta	\\ 
					r	
				\end{pmatrix},
			\end{equation}
			avec $r\colon 0\to b$ et $\theta\colon 0\to 2\pi$. Nous avons
			\begin{equation}
				\begin{aligned}[]
					\frac{ \partial F }{ \partial r }&=\begin{pmatrix}
						\cos\theta	\\ 
						\sin\theta	\\ 
						0	
					\end{pmatrix},
					&\frac{ \partial F }{ \partial \theta }&=\begin{pmatrix}
						-r\sin\theta	\\ 
						r\cos\theta	\\ 
						0	
					\end{pmatrix},
				\end{aligned}
			\end{equation}
			et 
			\begin{equation}
				d\sigma(r,\theta)=r\sqrt{2}.
			\end{equation}
			Nous devons donc calculer l'intégrale
			\begin{equation}
				\int_0^b\int_0^{2\pi}rd\sigma(r,\theta)=\int_0^b\int_0^{2\pi}r^2\sqrt{2}=\pi\sqrt{2}\frac{ b^3 }{ 2 }.
			\end{equation}
			
		\item
			Encore une fois, le cône va vers le haut et vers le bas. Il y aura donc lieu de multiplier le résultat par deux. Cherchons à quelle hauteur le cône coupe la sphère. Lorsque $y=0$, nous avons les équations $z=x$ et $x^2+z^2=1$, ce qui donne la hauteur $z=1/\sqrt{2}$. Nous devons donc intégrer sur le morceau de sphère qui flotte au dessus du cercle de rayon $1/\sqrt{2}$. Nous reprenons la paramétrisation en coordonnées cylindrique donnée par \eqref{EqParmsCylSphere}, dont nous connaissons déjà l'élément de surface, et  nous devons simplement calculer
			\begin{equation}
				\int_0^{1/\sqrt{2}}\int_0^{2\pi}\frac{ r }{ \sqrt{1-r^2} }d\theta dr=2\pi(1-\frac{1}{ \sqrt{2} }).
			\end{equation}
			Après multiplication par deux, nous avons la réponse
			\begin{equation}
				S=2\pi(2-\sqrt{2}).
			\end{equation}
			
		\item
			Une bonne carte pour le cylindre est donnée par
			\begin{equation}
				F(\theta,z)=\begin{pmatrix}
					\cos\theta	\\ 
					\sin\theta	\\ 
					z	
				\end{pmatrix}
			\end{equation}
			Nous voyons avec un tout petit peu de calcul que $d\sigma(r,\theta)=1$, de façon que la surface demandée soit
			\begin{equation}
				\int_0^{2\pi}\int_0^{\theta/2\pi}1 dzd\theta=\pi.
			\end{equation}
			Cela est exactement la surface du triangle de hauteur $1$ et de base donnée par la circonférence de la base du cylindre, un peu comme si le cylindre n'était pas vraiment courbé, ce qui est confirmé par le fait que l'élément de surface est la constante $1$.

	\end{enumerate}
	
\end{corrige}
