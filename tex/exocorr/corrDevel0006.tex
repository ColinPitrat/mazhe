% This is part of Exercices et corrigés de CdI-1
% Copyright (c) 2011
%   Laurent Claessens
% See the file fdl-1.3.txt for copying conditions.

\begin{corrige}{Devel0006}

Ici, le point crucial qui joue est l'unicité du polynôme dont il est question dans la proposition \ref{PropDevTaylorPol}. Nous utiliserons aussi toutes les propriétés de combinaisons des $o$.
\begin{enumerate}

\item
Le principe d'unicité dit qu'il existe un seul polynôme $P(t)$ d'ordre $3$ tel que
\begin{equation}	\label{EqCondPOlupolyTay}
	3t^4+t^3+t^2+t=P(t)+o(t^3),
\end{equation}
	mais $3t^4\in o(t^3)$, donc en fait le polynôme $P(t)=t^3+t^2+t=$ vérifie la condition \eqref{EqCondPOlupolyTay}. C'est donc lui, le développement de Taylor d'ordre $3$ de $3t^4+t^3+t^2+t$.


\item
Nous reprenons le développement de $\cos(x)-1$ de l'exercice \ref{exoDevel0005}. Nous avons
\begin{equation}
	\cos(x)-1=-\frac{ x^2 }{ 2 }+\frac{ x^4 }{ 4! }+o(x^4),
\end{equation}
donc
\begin{equation}
	\frac{ \cos(x)-1}{x}=-\frac{ x^3 }{ 2 }+\frac{ x^3 }{ 4! }+\frac{ o(x^4)}{x}
\end{equation}
Étant donné que $o(x^4)/x=o(x^3)$, nous avons que
\begin{equation}
	\frac{ \cos(x)-1}{x}=-\frac{ x^3 }{ 2 }+\frac{ x^3 }{ 4! }+o(x^3).
\end{equation}
Par unicité,
\begin{equation}
	-\frac{ x^3 }{ 2 }+\frac{ x^3 }{ 4! }
\end{equation}
est le développement de Taylor de la fonction proposée.

\item
\item
\item
Étant donné les solutions de l'exercice \ref{exoDevel0006}, nous savons que
\begin{equation}
	\begin{aligned}[]
		\sqrt{1+\big( \cos(x)-1 \big)}
			&=\sqrt{1+\big( -2x^2+\frac{ 2x^4 }{ 3 }+o(x^4) \big)}\\
			&=1+\frac{ A }{2}-\frac{ A^2 }{ 8 }+\frac{ A^3 }{ 16 }+o(A^3),
	\end{aligned}
\end{equation}
si nous appelons $A$ la quantité $-2x^2+\frac{ 2x^4 }{ 3 }+o(x^4)$. Nous ne cherchons que l'ordre $3$, donc nous voulons débusquer tous les termes en $o(x^3)$, et les rassembler sous le seul symbole « $+o(x^3)$ ».

En réalité, $A=-2x^2+o(x^3)$, donc $A^2\in o(x^3)$, et $o(A^3)\in o(x^3)$, il ne reste que
\begin{equation}
	\sqrt{1+\big( \cos(x)-1 \big)}=1-x^2+o(x^3).
\end{equation}
Note : si nous avions cherché les termes jusqu'à l'ordre $4$, alors il aurait fallu par exemple garder le terme en $x^4$ dans $A^2$, et au final,
\begin{equation}
		\sqrt{1+\big( \cos(x)-1 \big)}=1-x^2-\frac{ x^4 }{ 4 }+o(x^4).
\end{equation}

\end{enumerate}


\end{corrige}
