% This is part of Exercices et corrections de MAT1151
% Copyright (C) 2010, 2019
%   Laurent Claessens
% See the file LICENCE.txt for copying conditions.

\begin{corrige}{SerieUn0001}

    Il faut démontrer qu'une fonction $C^1$ sur $\eR$ vérifie automatiquement la condition \ref{DEFooYIFAooSJbMkC}\ref{ItemProbStableB} de la définition de la stabilité. Pour cela, remarquons qu'une fonction $C^1$ possède une dérivée continue, et donc bornée sur tout compact\footnote{Un compact est un ensemble fermé et borné, typiquement un intervalle du type $[a,b]$.}

	Prenons $\eta>0$ et $d_0\in\eR$ et puis un $d$ tel que $| d-d_0 |<\eta$. Par le théorème des bornes atteintes, la fonction $x'$ est bornée sur l'intervalle $[d_0-\eta,d_0+\eta]$. Appelons $K$ un majorant de $x'$ sur cet intervalle. La fonction
	\begin{equation}
		f(d)=x(d_0)+K| d-d_0 |
	\end{equation}
	majore $x(d)$, et donc on a 
	\begin{equation}
		\big| x(d)-x(d_0) \big|\leq K| d-d_0 |.
	\end{equation}

	Attention : vérifier si ce raisonnement est correct avec $d_0>d$, et adapter au besoin.
\end{corrige}
