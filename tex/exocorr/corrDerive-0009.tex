% This is part of Outils mathématiques
% Copyright (c) 2012,2015
%   Laurent Claessens
% See the file fdl-1.3.txt for copying conditions.

\begin{corrige}{Derive-0009}

    Pour les calculs, pas de complications :
    \begin{verbatim}
sage: f(x,y)=x**2*exp(3*y*sin(x))
sage: f.diff(x)            
(x, y) |--> 3*x^2*y*e^(3*y*sin(x))*cos(x) + 2*x*e^(3*y*sin(x))
sage: f.diff(y)
(x, y) |--> 3*x^2*e^(3*y*sin(x))*sin(x)
sage: f.diff(x)(x=pi/4,y=0)
1/2*pi
sage: f.diff(y)(x=pi/4,y=0)
3/32*pi^2*sqrt(2)
    \end{verbatim}
    C'est-à-dire que
    \begin{subequations}
        \begin{align}
            \frac{ \partial f }{ \partial x }(\frac{ \pi }{ 4 },0)=\frac{ \pi }{ 2 }\\
            \frac{ \partial f }{ \partial y } (\frac{ \pi }{ 4 },0)=\frac{ 3 }{ 32 }\pi^2\sqrt{2}.
        \end{align}
    \end{subequations}
    La différentielle est donc
    \begin{equation}
        df_{(\pi/4,0)}=\frac{ \pi }{ 2 }dx+\frac{ 3 }{ 32 }\pi^2\sqrt{2}dy.
    \end{equation}
    En ce qui concerne la dérivée directionnelle, nous faisons
    \begin{equation}
        \frac{ \partial f }{ \partial u }(\frac{ \pi }{ 4 },0)=df_{(\pi/4,0)}(u)=\frac{ \pi }{2}\frac{1}{ \sqrt{2} }-\frac{ 3 }{ 32 }\pi^2\sqrt{2}\frac{1}{ \sqrt{2} }=\frac{ \pi }{2}\frac{1}{ \sqrt{2} }-\frac{ 3 }{ 32 }\pi^2.
    \end{equation}

    Dans cet exercice, nous avons partout exploité la formule \eqref{EqsuitedfnfdsdfuOM} sous différentes formes.
\end{corrige}
