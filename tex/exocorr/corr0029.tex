% This is part of Exercices et corrigés de CdI-1
% Copyright (c) 2011
%   Laurent Claessens
% See the file fdl-1.3.txt for copying conditions.

\begin{corrige}{0029}

\begin{enumerate}
\item Nous avons $f(0,y)=0$, tandis que $f(x,x)=1/2$, donc il n'y a pas convergence.
\item La tangente de $\pi/2$ n'est pas définie, mais on sait que la limite à gauche vaut $+\infty$, tandis que la limite à droite vaut $-\infty$. Nous allons jouer là-dessus pour prouver que la limite proposée n'existe pas. Ce qu'il faut faire, c'est trouver deux chemins $\gamma_i\colon [0,1]\to \eR^2$ tels que $\gamma_1(1)=\gamma_2(t)=(1,1)$, mais tels que $\lim_{t\to 1}(f\circ \gamma_1)(t)\neq\lim_{t\to 1}(f\circ\gamma_2)(t)$.

Les chemins
\begin{equation}
	\begin{aligned}[]
		\gamma_1(t)	&=(1,t)\\
		\gamma_2(t)	&=(1,2-t)
	\end{aligned}
\end{equation}
font l'affaire. Toute la subtilité était de mettre $y\to 1$ avec dans un cas $y>1$ et dans l'autre cas, $t<1$.

\item Ici encore, on peut trouver un chemin qui fait tendre vers $+\infty$ et un autre qui fait tendre vers $-\infty$. Cela est d'ailleurs souvent le cas lorsqu'on est en présence d'un cas $1/0$ : on peut souvent trouver un chemin qui fait $1/+0$ et un chemin qui fait $1/-0$.

Ici, nous prenons $\gamma_1(t)=(t,0)$, donc
\begin{equation}
	f\big( \gamma_1(t) \big)=\frac{ 1 }{ 2t-2 }\to-\infty,
\end{equation}
tandis qu'avec $\gamma_2(t)=(2-t,0)$, nous trouvons
\begin{equation}
	(f\circ\gamma_2)(t)=\frac{1}{ 2(2-t)-2 }\to\infty.
\end{equation}

\item Cette fois, le $x$ ne peut rien pour faire basculer le signe du dénominateur. Nous prenons donc un chemin où $y\to 0$ par les négatifs, et un chemin avec $y\to 0$ par les positifs, par exemple $\gamma_1(t)=(t,1-t)$ et $\gamma_2(t)=(t,-1+t)$.
\end{enumerate}

\end{corrige}
