% This is part of Exercices et corrigés de CdI-1
% Copyright (c) 2011
%   Laurent Claessens
% See the file fdl-1.3.txt for copying conditions.

\begin{corrige}{0043}

Il suffit de dessiner les « zones à problèmes », qui sont les frontières entre les zones (c'est-à-dire les points qui ne sont pas dans l'intérieur d'une des zones). 
Cela est dessiné à la figure \ref{LabelFigQQa}.
\newcommand{\CaptionFigQQa}{Les zones à problèmes pour l'exercice \ref{exo0043}.}
\input{auto/pictures_tex/Fig_QQa.pstricks}

\end{corrige}
