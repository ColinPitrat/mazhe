\begin{corrige}{CalculDifferentiel0007}

	Il s'agit d'un exercice de dérivation de fonction composée; une des choses importantes est de bien noter à quel point sont calculées les dérivées partielles : certaines sont calculées en $(x,y)$ et d'autre en $(x^2-y^2,2xy)$. 

	Écrivons
	\begin{equation}
		g(x,y)=(f\circ\varphi)(x,y)
	\end{equation}
	avec $\varphi(x,y)=(x^2-y^2,2xy)$. Étant donné que nous allons en avoir beaucoup besoin, calculons directement les dérivées de $\varphi$ :
	\begin{equation}
		\begin{aligned}[]
			\frac{ \partial \varphi_1 }{ \partial x }&=2x	&\frac{ \partial \varphi_2 }{ \partial x }&=2y\\
			\frac{ \partial \varphi_1 }{ \partial y }&=-2y	&\frac{ \partial \varphi_2 }{ \partial y }&=2x.
		\end{aligned}
	\end{equation}
	En utilisant la formule donné au théorème \ref{ThoDerDirFnComp}, nous avons
	\begin{equation}
		\begin{aligned}[]
			\frac{ \partial g }{ \partial x }(x,y)&=\frac{ \partial f }{ \partial x_1 }\big( \varphi(x,y) \big)\frac{ \partial \varphi_1 }{ \partial x }(x,y)+\frac{ \partial f }{ \partial x_2 }\big( \varphi(x,y) \big)\frac{ \partial \varphi_2 }{ \partial x }(x,y)\\
			&=2x\frac{ \partial f }{ \partial x_1 }\big( \varphi(x,y) \big)+2y\frac{ \partial f }{ \partial x_2 }\big( \varphi(x,y) \big).
		\end{aligned}
	\end{equation}
	Afin de trouver $\frac{ \partial^2g }{ \partial x^2 }(x,y)$, il s'agit maintenant de calculer la dérivée de $\frac{ \partial g }{ \partial x }(x,y)$ par rapport à $x$. Pour ce faire, nous voyons l'expression
	\begin{equation}
		\frac{ \partial f }{ \partial x_1 }\big( \varphi(x,y) \big)
	\end{equation}
	comme la fonction composée de $\frac{ \partial f }{ \partial x_1 }$ et de $\varphi$. Par souci de simplification des notations, nous allons adopter les notations suivantes :
	\begin{equation}
		\begin{aligned}[]
			\partial_1f&=\frac{ \partial f }{ \partial x_1 }&\partial_2f&=\frac{ \partial f }{ \partial x_2 }\\
			\partial^2_{11}f&=\frac{ \partial^2f }{ \partial x_1^2}&\partial^2_{22}f&=\frac{ \partial^2f }{ \partial x_2^2 }\\
			\partial^2_{12}f&=\frac{ \partial^2f }{ \partial x_1\partial x_2}&\partial^2_{21}f&=\frac{ \partial^2f }{ \partial x_2\partial x_1 }.
		\end{aligned}
	\end{equation}
	Étant donné que la fonction $f$ est $C^2$, nous avons $\partial^2_{12}f=\partial^2_{21}f$ par le théorème \ref{Schwarz}. En utilisant en même temps la formule de dérivation de produit,
	\begin{equation}
		\begin{aligned}[]
			\frac{ \partial^2g }{ \partial  x}(x,y)&=2x\frac{ \partial  }{ \partial x }\left( \frac{ \partial f }{ \partial x_1 }\big( \varphi(x,y) \big) \right)\\
			&\quad+2\frac{ \partial f }{ \partial x_1 }\big( \varphi(x,y) \big)\\
			&\quad+2y\frac{ \partial  }{ \partial x }\left( \frac{ \partial f }{ \partial x_2 }\big( \varphi(x,y) \big) \right)\\
			&=2x\left( \frac{ \partial^2f }{ \partial x_1\partial x_1 }\big( \varphi(x,y) \big)\frac{ \partial \varphi_1 }{ \partial x }(x,y)+\frac{ \partial^2f }{ \partial x_2\partial x_1 }\big( \varphi(x,y) \big)\frac{ \partial \varphi_2 }{ \partial x }(x,y)      \right)\\
			&\quad +2\frac{ \partial f }{ \partial x_1 }\big( \varphi(x,y) \big)\\
			&\quad +2y\left( \frac{ \partial^2f }{ \partial x_1\partial x_2 }\big( \varphi(x,y) \big)\frac{ \partial \varphi_1 }{ \partial x }(x,y)+\frac{ \partial^2f }{ \partial x_2\partial x_2 }\big( \varphi(x,y) \big)\frac{ \partial \varphi_2 }{ \partial x }(x,y)       \right)\\
			&=4x^2\partial^2_{11}f\big( \varphi(x,y) \big)+8xy\partial^2_{12}f\big( \varphi(x,y) \big)+2\partial_1f\big( \varphi(x,y) \big)+4y^2\partial^2_{22}f\big( \varphi(x,y) \big).
		\end{aligned}
	\end{equation}
	En ce qui concerne le calcul de $\frac{ \partial^2g }{ \partial y^2 }(x,y)$, nous avons le même genre de jeu :
	\begin{equation}
		\begin{aligned}[]
			\frac{ \partial g }{ \partial y }(x,y)=-2y\frac{ \partial f }{ \partial x_1 }\big( \varphi(x,y) \big)+2x\frac{ \partial f }{ \partial x_2 }\big( \varphi(x,y) \big).
		\end{aligned}
	\end{equation}
	La dérivée de cela par rapport à $y$ donne
	\begin{equation}
		\begin{aligned}[]
			\frac{ \partial^2g }{ \partial y^2 }(x,y)&=-2\partial_1f\big( \varphi(x,y) \big)-2y\frac{ \partial  }{ \partial y }\left( \frac{ \partial f }{ \partial x_1 }\big( \varphi(x,y) \big) \right)+2x\frac{ \partial  }{ \partial y }\left( \frac{ \partial f }{ \partial x_2 }\big( \varphi(x,y) \big) \right)\\
			&=-2\partial_1f\big( \varphi(x,y) \big)+4y^2\partial^2_{11}f\big( \varphi(x,y) \big)-8xy\partial^2_{12}f\big( \varphi(x,y) \big)+4x^2\partial^2_{22}f\big( \varphi(x,y) \big).
		\end{aligned}
	\end{equation}
	En effectuant la somme, nous trouvons
	\begin{equation}
		\Delta g(x,y)=\frac{ \partial^2g }{ \partial x^2 }+\frac{ \partial^2g }{ \partial y^2 }=4(x^2+y^2)\Delta f(x^2-y^2,2xy).
	\end{equation}
	Notez que nous n'avons pas $\Delta g=4(x^2+y^2)\Delta f$.
	
	
	
\end{corrige}
