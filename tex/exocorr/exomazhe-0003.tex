% This is part of (almost) Everything I know in mathematics
% Copyright (c) 2016
%   Laurent Claessens
% See the file fdl-1.3.txt for copying conditions.

\begin{exercice}\label{exomazhe-0003}

    Soient \( x=0.1\times 10^{21}\) et \( y=0.5\times 10^20\) et les expressions 
    \begin{enumerate}
        \item
            \( z_1=\frac{ x-y }{ y }+\frac{ x+y }{ x }\)
        \item
            \( z_2=\frac{ x^2+y^2 }{ xy }\).
    \end{enumerate}
    Ces deux expressions sont algébriquement équivalentes.
    
    \begin{enumerate}
        \item
            Calculer les valeurs.
        \item
            On suppose une machine en précision simple. Laquelle des deux expressions est préférable ?
    \end{enumerate}

\corrref{mazhe-0003}
\end{exercice}
