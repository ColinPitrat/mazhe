% This is part of Exercices et corrigés de CdI-1
% Copyright (c) 2011, 2019
%   Laurent Claessens
% See the file fdl-1.3.txt for copying conditions.

\begin{corrige}{OutilsMath-0079}

    Le dessin est sur la figure \ref{LabelFigExoMagnetique}.
    \newcommand{\CaptionFigExoMagnetique}{La façon naturelle de décrire la situation sont les coordonnées cylindriques.}
    \input{auto/pictures_tex/Fig_ExoMagnetique.pstricks}

    Écrire la situation en coordonnées cylindriques est relativement facile. D'abord le vecteur courant est $Ie_z$. Le vecteur $d$ est un multiple de $e_r$ parce que ce dernier est exactement fait pour pointer vers l'axe vertical. En ce qui concerne le multiple à fixer, la longueur de $d$ est $r$ (par définition de la coordonnées cylindrique $r$) et sa direction est du fil vers le point, c'est-à-dire dans la même direction que $e_r$. Nous avons donc
    \begin{equation}
        d=re_r.
    \end{equation}
    Nous avons par conséquent
    \begin{equation}
        B(r,\theta,z)=\frac{ re_r\times e_z }{ r^2 }=\frac{ e_{\theta} }{ r }.
    \end{equation}
    N'oubliez pas de vérifier pourquoi $e_r\times e_z=e_{\theta}$. Calculer la divergence et le rotationnel de ce champ se fait en utilisant les formules \eqref{EqDivEnCylonf} et \eqref{EqRotationnelCylin} en posant $B_r=0$, $B_{\theta}=-\frac{1}{ r }$ et $B_z=0$. Nous trouvons
    \begin{equation}
        \begin{aligned}[]
            \nabla\cdot B&=0\\
            \nabla\times B&=0.
        \end{aligned}
    \end{equation}
    
    \begin{remark}
        La divergence d'un champ magnétique est \emph{toujours} nulle. Le fait que le rotationnel soit nul par contre est juste une propriété du cas particulier que nous regardons ici.
    \end{remark}

\end{corrige}
