\begin{corrige}{LimiteContinue0001}

	En coordonnées paramétriques, une droite passant par l'origine est de la forme $\gamma(t)=(at,bt)$ pour des constantes $a$ et $b$. Sur cette droite, la fonction vaut
	\begin{equation}
		f\big( \gamma(t) \big)=f(at,bt)=\frac{ abt^2 }{ a^2t^2+b^2t^2 }=\frac{ ab }{ a^2+b^2 }.
	\end{equation}
	Nous voyons donc que le long de la droite de paramètres $a$, $b$, la fonction est constante et vaut $ab/(a^2+b^2)$.
	
	Note : pour obtenir ce résultat sans utiliser d'équations paramétrique, on pouvait poser $y=px+m$, et traiter le cas particulier de la droite verticale à part.

	Le corollaire \ref{CorMethodeChemin} montre alors que la fonction n'a pas de limite pour $(x,y)\to (0,0)$ parce que nous avons trouvé toute une série de chemins le long desquels les limites sont différentes. Par exemple pour $a=0$ et $b=1$, nous avons $\lim_{t\to 0} f(\gamma(t))=0$ et avec $a=b=1$, nous avons $\lim_{t\to 0} (f\circ\gamma)(t)=\frac{ 1 }{2}$.

\end{corrige}
