\begin{corrige}{analyseCTU-0105}

    \begin{enumerate}
  \item La seule difficult\'e dans cet exercice est de trouver la d\'ecomposition en somme \'el\'ement simples de la fonction rationnelle $\displaystyle \frac{15x^2-4x-81}{(x-3)(x+4)(x-1)}$. Heureusement le d\'enominateur (de degr\'e 3) nous est donn\'e d\'ej\`a sous la forme d'une produit de polyn\^omes de degr\'e 1. Le num\'erateur est un polyn\^ome de degr\'e 2. Cela implique qu'il nous suffira de chercher des \'el\'ements simples de la forme $\frac{\text{constante}}{\text{poly de degr\'e 1}}$. On veut trouver trois constantes non nulles $A, \, B$ et $C$ telles que 
    \begin{equation*}
      \frac{15x^2-4x-81}{(x-3)(x+4)(x-1)} = \frac{A}{(x-3)} + \frac{B}{(x+4)} + \frac{C}{(x-1)} .
    \end{equation*}
Nous \'ecrivons la somme des trois fractions dans le membre de droite
\begin{equation*}
  \begin{aligned}
    &\frac{A}{(x-3)} + \frac{B}{(x+4)} + \frac{C}{(x-1)}= \frac{A(x+4)(x-1) + B(x-3)(x-1)+ C(x-3)(x+4) }{(x-3)(x+4)(x-1)}\\
    &= \frac{A(x^2+3x-4) + B(x^2-4x+3)+ C(x^2 +x-12) }{(x-3)(x+4)(x-1)} \\ 
    &= \frac{(A+B+C)x^2+ (3A-4B+C)x + (-4A +3B-12C)}{(x-3)(x+4)(x-1)},
  \end{aligned}
\end{equation*}
et nous obtenons les trois \'equations suivantes 
\begin{equation*}
  \begin{cases}
    &   A+B+C = -15, \\
    &   3A-4B+C = -4, \\
    &   -4A +3B-12C = -12.
  \end{cases}
\end{equation*}
Apr\`es quelques calculs on trouve $A=3$, $B=5$ et $C=7$. 

Nous pouvons revenir au calcul de primitive 
\begin{equation*}
  \begin{aligned}
    \int &\frac{15x^2-4x-81}{(x-3)(x+4)(x-1)}\, dx = \int \frac{3}{(x-3)} \, dx +  \int \frac{5}{(x+4)} \, dx + \int \frac{7}{(x-1)} \, dx\\
& = 3\ln(|x-3|) + 5\ln(|x+4|) + 7\ln(|x-1|)+C. 
  \end{aligned}
\end{equation*}
  \item La fraction rationnelle $\displaystyle \frac{1}{x^6+x^4}$ est le produit entre $\displaystyle \frac{1}{x^2+1}$ et $\displaystyle \frac{1}{x^4}$. Pour trouver sa d\'ecomposition en \'el\'ements simples
    \begin{equation*}
      \frac{p(x)}{x^2+1} + \frac{q(x)}{x^4},
    \end{equation*}
o\`u $p$ et $q$ sont deux fonctions polynomiales, il faut remarquer que pour calculer la somme $p(x)$ doit multiplier un terme de degr\'e 4 et $q(x)$ un terme de degr\'e 2. Comme on veut que le degr\'es des deux produits soit le m\^eme nous allons prendre $q(x)$ de degr\'e $2$ et $p(x)$ de degr\'e $0$. On a alors 
\begin{equation*}
   \frac{1}{x^6+x^4} = \frac{A}{x^2+1} + \frac{Bx^2 + Cx +D}{x^4},  
\end{equation*}
avec $A, B, C$ et $D$ \`a d\'eterminer dans $\mathbb{R}$. \`A partir de 
\begin{equation*}
   \frac{1}{x^6+x^4} = \frac{Ax^4 + Bx^4+ Bx^2 + Cx^3 + Cx +Dx^2 +D }{(x^2+1)x^4}, 
\end{equation*}
il est facile de voir que $D=1$, $C=0$, $B=-1$ et $A=1$, donc notre primitive est 
\begin{equation*}
  \int \frac{1}{x^6+x^4}\, dx = \int \frac{1}{x^2+1}\, dx  + \int \frac{-x^2  +1}{x^4} \, dx =\arctan(x) + \frac{1}{x} -\frac{x^{-3}}{3} + C.
\end{equation*}
  \end{enumerate}
\end{corrige}
