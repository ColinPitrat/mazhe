% This is part of Exercices et corrigés de CdI-1
% Copyright (c) 2011, 2019
%   Laurent Claessens
% See the file fdl-1.3.txt for copying conditions.

\begin{corrige}{OutilsMath-0090}

    Les dérivées partielles sont données par
    \begin{equation}
        \begin{aligned}[]
            \frac{ \partial f }{ \partial x }&=\frac{ -2xz }{ (x^2+y^2)^2 },&\frac{ \partial f }{ \partial y }&=\frac{ -2yz }{ (x^2+y^2)^2 },&\frac{ \partial f }{ \partial z }&=\frac{1}{ x^2+y^2 }.
        \end{aligned}
    \end{equation}
    Nous avons donc
    \begin{equation}
        \nabla f(a,b,c)=\begin{pmatrix}
            \frac{ -2ac }{ (a^2+b^2)^2 }    \\ 
            \frac{ -2bc }{ (a^2+b^2)^2 }    \\ 
            \frac{ 1 }{ a^2+b^2 }    
        \end{pmatrix}
    \end{equation}
    et le produit scalaire avec $(1,2,-3)$ est alors 
    \begin{equation}
        \nabla f(a,b,c)\cdot\begin{pmatrix}
            1    \\ 
            2    \\ 
            -3    
        \end{pmatrix}=
            \frac{ -2ac }{ (a^2+b^2)^2 }  +2        \frac{ -2bc }{ (a^2+b^2)^2 }    -3        \frac{ 1 }{ a^2+b^2 }    .
    \end{equation}
    La valeur de $df_{(a,b,c)}(v)$ est exactement la même.

    La circulation de $\nabla f$ le long d'un chemin $\sigma\colon \mathopen[ a , b \mathclose]\to \eR^3$ est donnée par $f\big( \sigma(b) \big)-f\big( \sigma(a) \big)$. Ici nous avons donc
    \begin{equation}
        \int_{\sigma}\nabla f=f\big( \sigma(2\pi) \big)-f\big( \sigma(0) \big)=0.
    \end{equation}
    
    La circulation d'un champ de vecteur conservatif le long d'un chemin fermé est toujours nulle.


\end{corrige}
