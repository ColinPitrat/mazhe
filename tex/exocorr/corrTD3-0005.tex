% This is part of Exercices de mathématique pour SVT
% Copyright (c) 2010
%   Laurent Claessens et Carlotta Donadello
% See the file fdl-1.3.txt for copying conditions.

\begin{corrige}{TD3-0005}

	Pour les premiers termes nous avons
	\begin{equation}
		\begin{aligned}[]
			u_0&=100\\
			u_1&=20+b\\
			u_2&=2(20+b)+b=40+3b\\
			u_3&=2(40+3b)+b=80+7b\\
			u_4&=2(80+7b)+b=160+15b.
		\end{aligned}
	\end{equation}
	
	Une façon de voir le concept de récurrence est de considérer que nous avons deux suites. La première est celle de l'énoncé :
	\begin{equation}
		\begin{cases}
			u_{n+1}=au_n+b	&	\forall n\in\eN_0\\
			u_0=x,
		\end{cases}
	\end{equation}
	et la seconde est 
	\begin{equation}
		v_n=a^n\left( x+\frac{ b }{ a-1 } \right)-\frac{ b }{ a-1 }.
	\end{equation}
	Ce que nous devons faire est de montrer que $u_n=v_n$ pour tout $n$.

	Pour $n=0$, cela est vrai parce que $u_n=x$ (par définition) tandis qu'en posant $n=0$ dans la définition de $v_n$ nous trouvons 
	\begin{equation}
		v_0=a^0\left( a+\frac{ b }{ a-1 } \right)-\frac{ b }{ a-1 }=x
	\end{equation}
	parce que $a^0=1$.

	Supposons que pour un certain $k$, nous ayons $u_n=v_n$, et montrons qu'alors $u_{k+1}=v_{k+1}$. Notre supposition nous dit que
	\begin{equation}
		u_k=v_k=a^k\left( x+\frac{ b }{ a-1 } \right)-\frac{ b }{ a-1 }.
	\end{equation}
	Maintenant, nous utilisons la formule de définition de $u_{k+1}$ :
	\begin{equation}
		\begin{aligned}[]
			u_{k+1}&=au_k+b\\
			&=a\left[ a^k\left( x+\frac{ b }{ a-1 } \right)-\frac{ b }{ a-1 } \right]+b\\
			&=a^{k+1}\left( x+\frac{ b }{ a-1 } \right)-\frac{ ab }{ a-1 }+b\\
			&=a^{k+1}\left( x+\frac{ b }{ a-1 } \right)+\frac{ -ab+b(a-1) }{ a-1 }\\
			&=a^{k+1}\left( x+\frac{ b }{ a-1 } \right)+\frac{ -b }{ a-1 }\\
			&=v_{k+1}.
		\end{aligned}
	\end{equation}
	Donc nous avons bien trouvé que $u_{k+1}=v_{k+1}$ sous l'hypothèse que $u_k=v_k$. 

	La preuve par récurrence est terminée : nous sommes maintenant sûr que $u_n=v_n$ pour tout $n$.
\end{corrige}
