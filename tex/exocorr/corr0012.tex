% This is part of Exercices et corrigés de CdI-1
% Copyright (c) 2011
%   Laurent Claessens
% See the file fdl-1.3.txt for copying conditions.

\begin{corrige}{0012}

\begin{enumerate}
\item Nous savons que $(k+2)/k\to 1$. Prenons donc $K$ tel que $k>K$ implique $(k+2)/k>\frac{ 1 }{2}$. D'autre part, $\cos(k\pi)=(-1)^{k+1}$, donc à tout moment de la suite, il y a un élément plus petit que $-1/2$ et un autre plus grand que $1/2$. Il n'y a donc pas de convergence parce qu'une telle suite ne peut pas être de Cauchy.

\item On utilise la proposition de la page 39 du cours qui dit que la limite d'un quotient est le quotient des limites. Or, la suite $\frac{1}{ k^3 }+\frac{1}{ k }+1$ tend vers $1$ et la suite $\frac{ 5 }{ k^3 }+2$ tend vers $2$. La suite proposée tend donc vers $1/2$.

\item Le secret est de mettre en évidence le terme de plus haut degré, et de simplifier :
\begin{equation}
	x_k=\frac{ k^3(1+\frac{1}{ k^2 }+\frac{1}{ k^3 }) }{ 5+\frac{ 2 }{ k^3 } },
\end{equation}
ensuite on se souvient que la limite du quotient est le quotient des limites (quand elles existent, ce qui est le cas après simplification par $k^3$). Nous obtenons donc convergence vers $1/5$.

\item 
En mettant $k$ en évidence et en simplifiant, nous tombons sur la suite
\begin{equation}
	x_k=\frac{    1+\frac{ (-1)^k }{ k }   }{  1-\frac{ (-1)^k }{ k }  },
\end{equation}
où les limites du numérateur et du dénominateur existent indépendamment. La limite est donc le quotient des limites, c'est à dire $1$.
\item  Cette suite est dans $\eC$. La sous-suite $i^{4k}$ est constante (et vaut $1$), tandis que la sous-suite $i^{4k+1}$ est constante et vaut $i$. Cette suite n'est donc pas de Cauchy.
\item Cette suite diverge (tend vers l'infini).
\item Nous avons $x_2=2$, et ensuite $x_k>2^{(2^k)}$. Cela se prouve par récurrence, en effet si $x_k>2^{(2^k)}$,
\begin{equation}
	x_{k+1}=x_k^2+1>\left( 2^{(2^k)} \right)^2+1>2^{(2^{k+1})}.
\end{equation}
Étant donné que la suite $y_k=2^{(2^k)}$ tend vers l'infini, la suite des $x_k$ (qui est toujours plus grande) tend également vers l'infini.

\begin{alternative}
Considérons la suite $x_{k+1}=x_k^2+1$. Si cette suite a une limite $x$, alors la suite $x\mapsto x_k^2$ a pour limite limite $x^2$. Pour chaque $k$, nous avons $x_{k+1}-x_k^2=1$, de sorte que la suite $k\mapsto x_{k+1}-x_k^2$ est constante et a $1$ pour limite. Mais la différence de deux suites convergentes a pour limite la différence des limites, de telle sorte qu'en passant à la limite, la suite $x\mapsto x_{k+1}-x_k^2$ a pour limite $x-x^2$.

En d'autres termes, si nous supposons que $x_k\to x$, alors nous passons à la limite dans la relation de récurrence
\begin{equation}
	x_{k+1}=x_k^2+1,
\end{equation}
et nous trouvons l'équation
\begin{equation}
	x=x^2+1
\end{equation}
que doit satisfaire le candidat limite $x$. Mais il est vite vu que cette équation n'a pas de solution réelle. La suite des $x_k$ ne peut donc pas converger.

Nous referons abondamment usage de cette technique pour résoudre l'exercice \ref{exo0022}.
\end{alternative}

\end{enumerate}

\end{corrige}
