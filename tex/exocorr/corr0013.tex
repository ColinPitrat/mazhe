% This is part of Exercices et corrigés de CdI-1
% Copyright (c) 2011
%   Laurent Claessens
% See the file fdl-1.3.txt for copying conditions.

\begin{corrige}{0013}

\begin{enumerate}
\item Tout nombre entier peut être écrit sous la forme $2n$ ou $2n+1$. Soit $\epsilon>0$, et $K$ tel que $k>K$ implique que $x_{2k}$ et $x_{2k+1}$ soient plus petit que $\epsilon$. Dans ce cas, à partir de $2K$, la suite des $x_k$ est plus petite que $\epsilon$.
\item Si $p_i$ est la suite des nombres premiers, alors une suite qui a les propriétés imposées peut très bien avoir $x_{p_i}=1$.
\item Oui : la suite $1,2,1,2,1,2,\ldots$
\item idem.
\item La suite $-1,0,1,-2,-1,0,1,2,-3,-2,-1,0,1,2,3,\ldots$ qui consiste à énumérer de $-n$ à $n$ puis de $-(n+1)$ à $n+1$ et ainsi de suite.
\item Soient $a>b$, deux valeurs atteintes une infinité de fois par la suite. Soit $r\in\eR$. Prouvons que la suite ne peut pas converger vers $r$. Disons que $r\neq a$, dans ce cas, si $| r-a |=\sigma$, pour tout $K>0$, il existe un $k>K$ tel que $x_k=a$ et donc tel que $| x_k-r |\leq\epsilon$ pour $\epsilon=\sigma/2$ par exemple. Même raisonnement si $r\neq b$.

Donc une suite périodique qui converge ne peut pas prendre deux valeurs distinctes (ici : $a$ et $b$). Nous en déduisons qu'elle doit être constante.

\item faux, comme le montre le contre-exemple de la suite constante $x_k=-1$. Nous avons $\| x_k \|\to| 1 |$ et pourtant $x_k$ ne converge pas vers $1$.

\item La sous-suite bornée est monotone et bornée, donc elle converge. Soit $a$ sa limite. Supposons pour fixer les idées que la suite est monotone croissante. La suite $x_k$ ne peut pas prendre de valeurs plus grande que $a$. En effet si elle prend la valeur $a+\delta$, elle ne peut plus, ensuite, prendre de valeurs intermédiaires entre $a$ et $a+\delta$ parce qu'elle est monotone croissante. Par conséquent, il n'est pas possible d'avoir une sous-suite convergente parce que cette dernière devrait avoir des termes tels que $| a-x_k |<\delta/2$.

La suite elle-même est donc bornée par $a$ et converge donc.

\item Pour trouver un exemple, il suffit de trouver une suite qui fait deux pas en avant et un pas en arrière. Par exemple $u_n=n+(-1)^n$. Nous avons
\begin{equation}
	u_{n+1}-u_n=
\begin{cases}
    -1	&	\text{si }n \text{est pair}\\
    3	&	 \text{si }n \text{est impair},
\end{cases}
\end{equation}
donc cette suite n'est jamais monotone.
\end{enumerate}

\end{corrige}
