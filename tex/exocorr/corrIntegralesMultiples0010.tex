\begin{corrige}{IntegralesMultiples0010}

L'intégrale à calculer est 
\[
\int_{D} 1 dA.
\] 
Comme vous pouvez voire dans la figure, la region $D$ n'est pas une region du premier ou du deuxième type. Cependant $D$ jouit d'une certaine régularité parce que son bord est donné par deux morceaux d'hyperbole et deux morceaux de parabole. En fait, la méthode la plus simple pour décrire $D$ est d'introduire des nouvelles variables adaptées à sa geométrie. Nous choisissons $u=y/x^2$ et $v=xy$. On peut alors écrire $D=\{(u,v) \,:\, u\in[a,b], \: v\in[c,d]\}$. 

La transformation inverse, qui nous donne $x$ et $y$ en fonction de $u$ et $v$, est $x=(v/u)^{1/3}$ et $y=(uv^2)^{1/3}$. Il faut bien comprendre que ces deux transformations sont définies et de classe $\mathcal{C}^1$ seulement hors de l'origine, d'où l'importance des hypothèses  $0< a\leq b$ et $0< c\leq d$. Le jacobien à considérer est 
\begin{equation}
  J=\det
  \begin{pmatrix}
    \frac{\partial x}{\partial u} & \frac{\partial x}{\partial v}\\
\frac{\partial y}{\partial u} & \frac{\partial y}{\partial v}\\
  \end{pmatrix}=\frac{1}{3u}.
\end{equation}

Notre intégrale est donc 
\[
\int_a^b\int_c^d\frac{1}{3u}\, dv du= \frac{d-c}{3}\ln\left(\frac{b}{a}\right).
\] 
\end{corrige}

