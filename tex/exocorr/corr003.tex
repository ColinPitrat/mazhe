\begin{corrige}{003}
A point $z\in S^1$ can be written under the form $z=e^{i\theta_0}$. A tangent vector of $S^1$ at the point $z$ can be expressed by means of the derivative of a path $\dpt{ X }{ \eR }{ S^1 }$ such that $X(0)=z$. Let us consider a general path $X(t)=e^{i\theta(t)}$ with $\theta_0=\theta_0$. Its tangent vector is
\begin{equation} \label{eq:ecrX}
  X=\Dsdd{ e^{i\theta(t)} }{t}{0}=i\theta'(0)e^{i\theta_0}=\theta'(0)e^{i\big( \frac{ \pi }{ 2 }+\theta_0 \big)}.
 \end{equation}

Let us point out the fact that the velocity $\theta'(0)$ of the path describes the norm of the tangent vector while $e^{i\theta_0}$ is the point of $S^1$ on which $X$ is fixed. The differential of $\phi$ at $z$ on the vector $X$ reads
\begin{equation}  \label{eq:dphiz}
\begin{split}
d\phi_zX&=\Dsdd{ \phi(X(t)) }{t}{0}\\
		&=\Dsdd{ e^{iN\theta(t)} }{t}{0}\\
		&=N\theta'(0)e^{i(N-1)\theta_0}e^{i\big( \frac{ \pi }{ 2 }+\theta_0 \big)}\\
		&=Nz^{N-1}X.
\end{split}
\end{equation}

Now let us see $TS^1$ as subset of $\eR^4$. Any $X\in TS^1$ can be written as $(z,v)$ where $z\in S^1\subset\eC\simeq\eR^2$ is the point at which $X$ is ``fixed'' and $v\in \eC\simeq\eR^2$ is the vector itself. Relation \eqref{eq:ecrX} contains strong constraints on elements $(z,v)\in\eR^4$: if $z=e^{i\theta_0}=(\cos \theta,\sin\theta)$, the element $v$ must be of the form $he^{i\left( \frac{ \pi }{ 2 }+\theta_0 \right)}=(-h\sin\theta,h\cos\theta)$. Now $TS^1$ is seen as the subset of $\eR^4$ of vectors of the form
\[ 
  (z,X)=(\cos\theta,\sin\theta,-h\sin\theta,h\cos\theta).
\]
At this point it is important to underline two points.
\begin{itemize}
\item The map $h$ can be negative as well as positive.
\item An element in $\eC$ does not uniquely determines an element in $TS^1$ although tangent spaces of $e^{i\theta}$ and $e^{i(\pi+\theta)}$ look like the same (draw a picture if you are unsure). When one deals with a tangent vector, one has to keep trace of the point on which the vector is fixed. If one does not, one can believe that $T_{z_1}S^1=T_{z_2}S^1$ and then that finally, $TS^1=\eC$, see figure \ref{LabelFigALIzHFm}.
\end{itemize}


%The result is on figure \ref{LabelFigALIzHFm}. % From file ALIzHFm
\newcommand{\CaptionFigALIzHFm}{A possible mistake is to confuse $T_{z_1}S^1$ and $T_{z_2}S^1$.}
\input{auto/pictures_tex/Fig_ALIzHFm.pstricks}

The way to see a cylinder (of radius $1$) as a subset of $\eR^3$ is to take the coordinates $(\cos\theta,\sin\theta,h)$. A diffeomorphism $\dpt{ f }{ TS^1 }{ \text{Cyl} }$ is given by
\begin{equation}
f(\cos\theta,\sin\theta,-h\sin\theta,h\cos\theta)=(\cos\theta,\sin\theta,h).
\end{equation}
We have to check that it is diffeomorphic: it is smooth, bijective and the inverse is smooth too. This needs the ``new'' definition of smooth map. Indeed the map $f$ is defined on a rather non trivial subset of $\eR^4$: the set of $(x,y,u,v)$ such that
\[ 
%\begin{split}
x^2+y^2=1 \quad \text{and} \quad
\begin{cases}
 v=-ux/y&\text{if }y\neq 0\\
 v\in\eR&\text{if }y=0
\end{cases}
%\end{split}
\]
which is not open. We have to consider pointwise a neighbourhood and a prolongation. Let's take the part $A_0\equiv x\neq 0,\,y\neq 0$ of the domain of $f$. On this part of the domain, $f$ reads $(x,y,u,v)\mapsto(x,y,\frac{ v }{ x })$. We can find an open set $A\subset\eR^4$ which contains $A_0$ and such that $x$ and $y$ don't vanish on $A$. The prolongation of $f$ from $A_0$ to $A$ by the same formula $(x,y,u,v)\mapsto(x,y,\frac{ v }{ x })$ is a smooth map.

When $x=0$, the map $f$ reads $(0,1,0,h)\mapsto (1,0,h)$ which can be prolongated to the smooth map $(x,y,z,h)\mapsto (y,x,h)$. The same idea is true on $y=0$ where $f$ reads $(1,0,0,h)\mapsto(1,0,h)$.


The inverse map $f^{-1}$ is $(x,y,h)\mapsto(x,y,-hy,hx)$ with the constraint $x^2+y^2=1$ on the domain.

The following diagram is the \emph{definition} of the dotted line; this is what one means when one says ``$d\phi$ seen on the cylinder'':
\[
\xymatrix{ p\in\Cyl \ar[r]^{f^{-1}}\ar@{.>}[d]_{d\phi}	& f^{-1}(p)\in TS_z^1 \ar[d]^{d\phi_z} \\ 
               (f\circ d\phi\circ f^{-1})(p)\in \Cyl	& d\phi_z\big( f^{-1}(p) \big)\in T_{\phi(z)}S^1 \ar[l]^-{f}.    }
\]
In this, we suppose that $p\in\Cyl$ is the image by $f$ of a tangent vector on $S^1$ at the point $z$.

In order to express the $d\phi_z$ of equation \eqref{eq:dphiz}  in terms of the cylinder, we pick $X=(-\sin\theta,\cos\theta,r)\in\text{Cyl}$, i.e. $X=he^{i( \frac{ \pi }{ 2 }+\theta )}\in T_{e^{i\theta}}S^1$ and we apply $d\phi_z$ on it:
\[
\begin{split}
d\phi_z\left( he^{i\big( \frac{ \pi }{ 2 }+\theta \big)} \right)&=Ne^{i(N-1)\theta}he^{i\big( \frac{ \pi }{2}+\theta \big)}\\
								&=Nhe^{i\big( \frac{ \pi }{2}+N\theta \big)}\\
								&=(-\sin N\theta,\cos N\theta,Nh).
\end{split}
\]

\end{corrige}
