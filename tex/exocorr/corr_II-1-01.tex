% This is part of the Exercices et corrigés de CdI-2.
% Copyright (C) 2008, 2009,2016
%   Laurent Claessens
% See the file fdl-1.3.txt for copying conditions.


\begin{corrige}{_II-1-01}

Ceci est une équation à variables séparées, c'est-à-dire du type $y'=u(t)f(y)$. Ici, nous avons $u(t)=t$ et $f(y)= e^{y}$, et la primitive de $1/f$ est donnée par
\begin{equation}
	G(y)=\int e^{-y}=- e^{-y}.
\end{equation}
La fonction $y$ est une solution de l'équation que nous regardons si et seulement s'il existe une constante $c\in\eR$ telle que
\begin{equation}		\label{Eq101EyC}
	- e^{-y}=\frac{ t^2 }{ 2 }+C.
\end{equation}
Étant donné qu'une exponentielle est toujours positive, une solution $y_C$ ne peut exister que lorsque $| t |<\sqrt{2C}$. En renommant $C\to C/2$ et en prenant le logarithme de la relation \eqref{Eq101EyC}, nous trouvons
\begin{equation}
	y_C(t)=-\ln\left( \frac{ C-t^2 }{ 2 } \right),
\end{equation}
qui est valable sur l'intervalle $| t |<\sqrt{C}$. Afin de trouver celle qui vérifie $y_C(0)=0$, nous devons résoudre
\begin{equation}
	y_c(0)=-\ln\left( \frac{ C }{ 2 } \right)=0
\end{equation}
par rapport à $C$, dont la solution est évidement $C=2$. La solution recherchée est donc
\begin{equation}
	\begin{aligned}
		y_2\colon ]-\sqrt{2},\sqrt{2}[&\to \eR \\
		t&\mapsto -\ln\left( \frac{ 2-t^2 }{2} \right).
	\end{aligned}
\end{equation}

\end{corrige}
