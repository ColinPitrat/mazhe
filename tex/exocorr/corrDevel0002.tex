% This is part of Exercices et corrigés de CdI-1
% Copyright (c) 2011-2012
%   Laurent Claessens
% See the file fdl-1.3.txt for copying conditions.

\begin{corrige}{Devel0002}


Nous devons écrire
\begin{equation}
	P(x-a)=f(a)+f'(a)(x-a)+\frac{ f''(x) }{ 2! }(x-a)^2
\end{equation}
avec $a=0$ et $f(x)= e^{x}$. Étant donné que les dérivées de tout ordre de l'exponentielle sont $e^x$, nous avons toujours $f^{(k)}(0)=1$, et donc
\begin{equation}
	P(x)=1+x+\frac{ x^2 }{ 2 }.
\end{equation}

Les graphes sont sur la figure \ref{LabelFigLaurin}.
\newcommand{\CaptionFigLaurin}{En bleu : la fonction $x\mapsto e^x$, en cyan l'approximation d'ordre \( 0\), en vert son développement d'ordre $1$ et en rouge celui d'ordre $2$.}
\input{auto/pictures_tex/Fig_Laurin.pstricks}

Afin d'évaluer l'erreur commise entre $e^{0,1}$ et $P(0,1)$, nous calculons le reste donné par la proposition \ref{PropResteTaylorc}. Il existe un $c$ entre $0$ et $0,1$ tel que
\begin{equation}
	R_2(x)=\frac{ e^c }{ 3 }\cdot 0,001,
\end{equation}
où $c$ est entre $0$ et $0,1$. La valeur de $e^c$ est bornée par $e<3$, et donc
\begin{equation}
	R_2(0,1)<0,001,
\end{equation}
et l'estimation
\begin{equation}
	e^{0,1}=1+x+\frac{ x^2 }{ 2 }=1+0,1+\frac{ 0,01 }{ 2 }=1,105
\end{equation}
est acceptable. En effet, la valeur « exacte » est $1,10517$.

\end{corrige}
