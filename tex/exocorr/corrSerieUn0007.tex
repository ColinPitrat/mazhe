% This is part of Exercices et corrections de MAT1151
% Copyright (C) 2010
%   Laurent Claessens
% See the file LICENCE.txt for copying conditions.

\begin{corrige}{SerieUn0007}

	\begin{enumerate}

		\item
			Nous supposons que $a$ est une constante et nous devons calculer
			\begin{equation}
				K=\sup_{| d-d_0 |<\eta}\frac{ | a^d-a^{d_0} | }{ | d-d_0 | }.
			\end{equation}
			Si $a>1$, cette la fonction est croissante\footnote{JE NE VOIS PAS POURQUOI, MAIS EN LA TRAÇANT, ÇA A L'AIR D'ÊTRE LE CAS.} et donc le supremum est atteint lorsque $d=d_0+\eta$ et le conditionnement absolu vaut
			\begin{equation}
				K_{abs}^{\eta}=\frac{ |a^{d_0}(a^{\eta}-1)| }{ \eta }.
			\end{equation}
			Dans ce cas, le conditionnement relatif vaut
			\begin{equation}
				K_{rel}(d)=\frac{ |d(a^{\eta}-1)| }{ \eta }.
			\end{equation}

			Nous pouvons obtenir une majoration de $K_{rel}$ en utilisant le théorème des accroissements finis. En effet, il existe un $c\in\mathopen[ d , d_0 \mathclose]$ tel que
			\begin{equation}
				a^d=a^{d_0}+f'(c)(d-d_0)
			\end{equation}
			où $f(x)=a^x$ et $f'(c)=\ln(a)a^c$. Donc
			\begin{equation}
				K_{abs}=\sup_{| d-d_0 |<\eta}\ln(a)a^c.
			\end{equation}
			Lorsque $a>1$, la fonction $\ln(a)a^c$ est croissante en $c$ et donc nous avons
			\begin{equation}
				K_{abs}(d_0)\leq \ln(a)a^{d_0+\eta}
			\end{equation}
			et
			\begin{equation}
				K_{rel}(d)\leq \ln(a)da^{\eta}.
			\end{equation}

			Il y a encore une troisième façon de travailler, en utilisant le corollaire \ref{CorConditionnementNormeNabla}. Nous savons alors que $K_{\text{abs}}(d)\simeq| x'(d) |$, c'est-à-dire
			\begin{equation}
				K_{\text{abs}}(d)^{\eta}\simeq| \ln(a)a^d |
			\end{equation}
			lorsque $\eta$ est petit. Le conditionnement relatif vaut alors
			\begin{equation}
				K_{\text{rel}}\simeq d| \ln(a) |.
			\end{equation}
			Donc plus $a$ est proche de $1$, mieux le problème est conditionné, et plus $d$ est petit, mieux c'est aussi.

		\item
			Lorsque $x(d)=d+1$, nous avons
			\begin{equation}
				K^{\eta}_{abs}(d_0)=\sup_{| d-d_0 |<\eta}\frac{ | x(d)-x(d_0) | }{ | d-d_0 | }=1,
			\end{equation}
			et donc
			\begin{equation}
				K^{\eta}_{rel}=\frac{| d| }{ |d+1| }.
			\end{equation}

	\end{enumerate}

\end{corrige}
