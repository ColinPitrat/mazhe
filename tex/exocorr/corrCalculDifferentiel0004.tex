\begin{corrige}{CalculDifferentiel0004}

	\begin{enumerate}
		\item
			La petite formule $2x^2y\leq x^4+y^2$ est un cas particulier de la formule $a^2+b^2\geq 2ab$ qui vient simplement du fait que $a^2-2ab+b^2=(a-b)^2\geq 0$. Tester la continuité de $f$ en $(0,0)$ revient à calculer sa limite lorsque $(x,y)\to (0,0)$ et puis de regarder si la limite est égale à la valeur de la fonction en $(0,0)$. Ici, $f(0,0)=0$, donc nous devons voir si $\lim_{(x,y)\to(0,0)}f(x,y)=0$ ou non. Étant donné que 
			\begin{equation}
				\frac{ 2x^2y }{ x^4+y^2 }\leq 1,
			\end{equation}
			nous avons toujours
			\begin{equation}
				0\leq | f(x,y) |\leq \left| \frac{ x^3y }{ x^4+y^2 } \right| \leq\left| \frac{ x }{ 2 } \right| \to 0.
			\end{equation}
			Le nombre $| f(x,y) |$ est par conséquent coincé entre $0$ et une fonction qui tend vers zéro. La règle de l'étau conclut que la limite existe au vaut zéro.

		\item
			Les dérivées partielles en $(0,0)$ se calculent par la définition :
			\begin{equation}
				\partial_xf(0,0)=\lim_{h\to 0} \frac{ f(h,0)-f(0,0) }{ h }=0,
			\end{equation}
			et
			\begin{equation}
				\partial_yf(0,0)=\lim_{h\to 0} \frac{ f(0,h)-f(0,0) }{ h }=0.
			\end{equation}
			Donc les deux dérivées partielles existent et sont nulles. Le «candidat» différentiel est donc $T(u)=0$. Nous testons le critère de différentiabilité \eqref{EqCritereDefDiff} avec $T=0$ :
			\begin{equation}
				\begin{aligned}[]
				\lim_{(h_1,h_2)\to (0,0)} &\frac{ | f\big( (0,0)+(h_1,h_2) \big)-f(0,0)-T(h_1,h_2) | }{ \| (h_1,h_2) \|_{\eR^2} }\\
					&=\lim_{(h_1,h_2)\to(0,0)}\left| \frac{ h_1^3h_2 }{ (h_1^4+h_2^2) }\frac{1}{ \sqrt{h_1^2+h_2^2} } \right|.
				\end{aligned}
			\end{equation}
			En suivant le chemin $(t,t)$, nous trouvons la limite zéro. Par contre en suivant le chemin $(t,t^2)$, nous trouvons la limite $\frac{ 1 }{2}$. La limite n'existe donc pas et la fonction n'est pas différentiable en $(0,0)$.
	\end{enumerate}

\end{corrige}
