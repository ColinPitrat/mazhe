% This is part of the Exercices et corrigés de mathématique générale.
% Copyright (C) 2009,2013
%   Laurent Claessens
% See the file fdl-1.3.txt for copying conditions.
\begin{corrige}{3}

Les paramètres de la boîte sont le rayon $R$ et la hauteur $h$ qui sont contraints par $V=\pi R^2h$, et on veut minimiser
\begin{equation}
	S(R,h)=2\pi Rh+2\pi R^2.
\end{equation}
Cela ressemble à une fonction à deux variables ($R$ et $h$), mais ce n'en est pas une. En effet, dès que $R$ est choisit, $h$ est automatiquement choisit à cause de la contrainte $V=\pi R^2h$. Nous pouvons donc remplacer $h$ par $V/\pi R^2$ dans la fonction $S(R,h)$, et nous trouvons la surface en fonction de $R$ tout seul :
\begin{equation}
	S(R)=\frac{ 2V }{ R }+2\pi R^2.
\end{equation}
Pour minimiser, nous cherchons l'annulation de la dérivée
\begin{equation}
	S'(R)=-\frac{ 2V }{ R^2 }+4\pi R.
\end{equation}
Nous trouvons $S'(R)=0$ pour
\begin{equation}
	R=\left( \frac{ V }{ 2\pi } \right)^{1/3},
\end{equation}
qui est la seule solution réelle. Nous croyons que cela est un minimum et non un maximum parce que $S(R)\to\infty$ pour $R\to\pm\infty$. Nous retrouvons la hauteur que l'on cherche en remettant cette valeur de $R$ dans l'équation qui lie $R$ et $h$ :
\begin{equation}
	h=\frac{ V }{ \pi R^2 }=2^{-1/3}\left( \frac{ V }{ \pi } \right)^{4/3}.
\end{equation}

\end{corrige}
