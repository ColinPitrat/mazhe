\begin{corrige}{controlecontinu0013}

Il nous faut calculer $\int_{E}1\,dV$. Le volume $E$ peut être décrit en coordonnées sphèriques de la façon suivante. D'abord $E$ est contenu dans le premier octant, c'est-à-dire que les valeurs de $\theta$ à considérer sont entre $0$ et $\pi/2$, et que $z$ ne peut que être positif. De plus, $E$ est contenu entre le deux sphères centrées dans l'origine de rayons $1$ et $2$. Cela nous dit que $1<r<2$ pour tous le points de $E$. Enfin on traduit léquation du cône en coordonées sphèriques et on obtient
\[
2\cos^2(\phi)= \sin^2(\phi),
\]
soit $\tan(\phi)=\sqrt{2}$. À chaque point de $E$ correspond un valeur de $\phi$ compris entre $0$ et $\arctan(\sqrt{2})$. 

L'intégrale à calculer est alors 
\[
\int_0^{\arctan(\sqrt{2})}\int_0^{\pi/2}\int_{1}^{2} r^2\sin{\phi}\, dr\, d\theta\, d\phi = \frac{7\pi}{6}(\cos(\arctan(\sqrt{2}))-1). 
\]

\end{corrige}
