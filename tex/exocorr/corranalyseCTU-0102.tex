% This is part of Analyse Starter CTU
% Copyright (c) 2015
%   Laurent Claessens,Carlotta Donadello
% See the file fdl-1.3.txt for copying conditions.

\begin{corrige}{analyseCTU-0102}    

\begin{enumerate}
      \item La fonction $\sinh$ est d\'erivable (donc continue) et sa d\'eriv\'ee est $\cosh$, qui est une fonction strictement positive. Donc $\sinh$ admet une fonction réciproque par le th\'eor\`eme de la bijection. 
      \item Pour montrer que $f(y) = \ln(y + \sqrt{y^2+1})$ est la fonction réciproque de $\sinh$ nous consid\'erons les fonctions compos\'ees $y\mapsto \sinh(f(y))$ et $x\mapsto f(\sinh(x))$. La fonction $f$ est la r\'eciproque de $\sinh$ si et seulement si ces deux fonctions sont respectivement l'indentit\'e de $y$ et de $x$. 
        \begin{align*}
          &\sinh(f(y)) = \frac{1}{2}\left(y+\sqrt{y^2 +1}- \frac{1}{y+\sqrt{y^2 +1}} \right) =\frac{2y^2 + 2y\sqrt{y^2+1}}{2(y+\sqrt{y^2 +1})} = y ;
        \end{align*}
        \begin{align*}
          & f(\sinh(x)) = \ln\left(\frac{e^x - e^{-x}}{2} + \sqrt{\left(\frac{e^x - e^{-x}}{2}\right)^2 +1}\right) \\
          &=\ln\left(\frac{e^x - e^{-x}}{2} + \sqrt{\left(\frac{e^x + e^{-x}}{2}\right)^2}\right)  = \ln\left(\frac{2e^x }{2}\right) = x.
        \end{align*}
      \item Calculons $f'(y)$  en utilisant son expression analytique
        \begin{align*}
          f'(y) = \frac{1}{y+\sqrt{y^2 +1}} \left(1+ \frac{y}{\sqrt{y^2 +1}}\right)= \frac{y+\sqrt{y^2 +1}}{(y+\sqrt{y^2 +1})\sqrt{y^2 +1}} = \frac{1}{\sqrt{y^2 +1}}.
        \end{align*}

 La formule de dérivation d'une fonction composée nous dit que 
 \begin{equation*}
   f'(y) = \frac{1}{\cosh(f(y))}
 \end{equation*}
ensuite on sait que $f$ est la fonction r\'eciproque de $\sinh$ 
 et les propriétés des fonctions $\sinh$ et $\cosh$ vues dans l'exercice 19
%~\ref{exostarterST-0015} 
 nous disent que $\cosh^2(x) -\sinh^2(x) = 1$. On a donc 
 \begin{equation*}
   \cosh(f(y)) = \sqrt{1+ \sinh^2(f(y))} = \sqrt{1+y^2}.
 \end{equation*}
    \end{enumerate}
\end{corrige}
