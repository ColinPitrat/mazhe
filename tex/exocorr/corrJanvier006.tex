% This is part of the Exercices et corrigés de mathématique générale.
% Copyright (C) 2009
%   Laurent Claessens
% See the file fdl-1.3.txt for copying conditions.
\begin{corrige}{Janvier006}


Un hexagone régulier de côté $a$ est la réunion de 6 triangles équilatéraux de
côté $a$. Chacun de ces triangles a une base $a$ et une hauteur $\sqrt{a^2 -
\frac{a^2}4} = \frac{\sqrt{3}a}2$ (par Pythagore), donc l'aire de l'hexagone est~:
\begin{equation*}
  3 a \frac{\sqrt{3}a}2 = \frac{3\sqrt 3}{2} a^2
\end{equation*}


\end{corrige}
