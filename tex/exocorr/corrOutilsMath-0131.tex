% This is part of Outils mathématiques
% Copyright (c) 2011
%   Laurent Claessens
% See the file fdl-1.3.txt for copying conditions.

\begin{corrige}{OutilsMath-0131}

    Les dérivées partielles sont classiques. En ce qui concerne \( \nabla\cdot(\nabla f)(a)\), il faut comprendre qu'il y a implicitement des parenthèses de la façon suivante :
    \begin{equation}
        \left( \nabla\cdot (\nabla f) \right)(a).
    \end{equation}
    Cela donne
    \begin{equation}
        \nabla \cdot\nabla f=\nabla\cdot\begin{pmatrix}
            \frac{ \partial f }{ \partial x }    \\ 
            \frac{ \partial f }{ \partial y }    
        \end{pmatrix}=\frac{ \partial^2f }{ \partial x^2 }+\frac{ \partial^2f }{ \partial y^2 },
    \end{equation}
    c'est à dire que nous devons calculer la somme des dérivées secondes.

    Voici comment on peut faire les calculs avec Sage.
    \begin{verbatim}
----------------------------------------------------------------------
| Sage Version 4.8, Release Date: 2012-01-20                         |
| Type notebook() for the GUI, and license() for information.        |
----------------------------------------------------------------------
sage: f(x,y)=x**2*sin(y)+exp(cos(x))
sage: f.diff(x)
(x, y) |--> 2*x*sin(y) - e^cos(x)*sin(x)
sage: f.diff(y)
(x, y) |--> x^2*cos(y)

sage: f.diff(x)(pi,-pi/4)
-pi*sqrt(2)
sage: f.diff(y)(pi,-pi/4)
1/2*pi^2*sqrt(2)

sage: f.diff(x)(pi,-pi/4)+5*f.diff(y)(pi,-pi/4)
5/2*pi^2*sqrt(2) - pi*sqrt(2)

sage: D=f.diff(x,2)+f.diff(y,2)
sage: D
(x, y) |--> -x^2*sin(y) + e^cos(x)*sin(x)^2 - e^cos(x)*cos(x) + 2*sin(y)
sage: D(pi,-pi/4)
1/2*pi^2*sqrt(2) - sqrt(2) + e^(-1)
    
    \end{verbatim}

    Vous noterez la notation \info{f.diff(x,2)} pour calculer la dérivée seconde de \( f\) par rapport à \( x\).

    Les résultats sont
    \begin{equation}
        df_a(u)=\frac{ 5 }{ 2 }\pi^2\sqrt{2}-\pi\sqrt{2},
    \end{equation}
    puis
    \begin{equation}
        \nabla\cdot(\nabla f)=-x^2\sin(y)+ e^{\cos(x)}\sin^2(x)- e^{\cos(x)}\cos(x)+2\sin(y).
    \end{equation}
    En calculant cela au point \( a\) nous trouvons
    \begin{equation}
        \nabla\cdot(\nabla f)(a)=\frac{ 1 }{2}\pi^2\sqrt{2}+e^{-1}.
    \end{equation}

\end{corrige}
