\begin{corrige}{CalculDifferentiel0006}

	Nommons $u$, $v$ et $w$ les variables de $f$. La matrice jacobienne de la fonction $g\colon \eR^3\to \eR$ est la matrice colonne
	\begin{equation}
		\begin{pmatrix}
			\displaystyle\frac{ \partial g }{ \partial x }(x,y,z)	\\ 
			\displaystyle\frac{ \partial g }{ \partial y }(x,y,z)	\\ 
			\displaystyle\frac{ \partial g }{ \partial z }(x,y,z)	
		\end{pmatrix}.
	\end{equation}
	Les dérivées de $g$ se calculent en termes de celles de $f$ en utilisant la règle des fonctions composées :
	\begin{equation}
		\begin{aligned}[]
			\frac{ \partial g }{ \partial x }(x,y,z)&=\frac{ \partial f }{ \partial u }\big( \varphi(x,y,z) \big)\underbrace{\frac{ \partial \varphi_u }{ \partial x }(x,y,z)}_{=2x}\\
			&\quad+\frac{ \partial f }{ \partial v }\big( \varphi(x,y,z) \big)\underbrace{\frac{ \partial \varphi_v }{ \partial x }(x,y,z)}_{=0}\\
			&\quad+\frac{ \partial f }{ \partial w }\big( \varphi(x,y,z) \big)\underbrace{\frac{ \partial \varphi_w }{ \partial x }(x,y,z)}_{=-2x}\\
			&=2x\big( \partial_uf(\varphi)+\partial_wf(\varphi) \big)
		\end{aligned}
	\end{equation}
	où nous avons noté, sur la dernière ligne, simplement $\varphi$ au lieu de $\varphi(x,y,z)$ pour alléger la notation. De la même façon pour les autres nous trouvons
	\begin{equation}
		\begin{aligned}[]
			\partial_yg(x,y,z)&=2y\big( -\partial_uf(\varphi)+\partial_vf(\varphi) \big)\\
			\partial_zg(x,y,z)&=2z\big( -\partial_vf(\varphi)+\partial_wf(\varphi) \big).
		\end{aligned}
	\end{equation}
	
	Nous calculons maintenant la somme de ces trois expressions en posant $x=y=z=t$. Les termes partent deux à deux et il ne reste que zéro.
\end{corrige}
