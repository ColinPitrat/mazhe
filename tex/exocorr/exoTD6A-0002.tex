\begin{exercice}\label{exoTD6A-0002}
 
  \begin{enumerate}
  \item Vérifier que la fonction $x_1(t)=\sin(t)$ est une solution de l'équation différentielle 
    \begin{equation}\label{arcsin}
      \frac{dx}{dt}= \sqrt{1-x^2}, 
    \end{equation} 
    pour $t$ dans l'intervalle $\displaystyle ]-\frac{\pi}{2},\frac{\pi}{2} [$. 
  \item Vérifier que les deux fonctions constantes $x_2(t)= 1$ et $x_3(t)=-1$ sont des solutions de \eqref{arcsin} pour tout $t$.
  \item Dessiner les trajectoires des trois solutions. 
  \item Est-il possible de prolonger la solution $x_1$ après $t=\frac{\pi}{2}$ ? Avant $t=-\frac{\pi}{2}$ ? Discuter. 
  \item Est-il possible de trouver une solution si la donnée de départ est $x(0)= 2$ ? 
  \item Est-il possible de trouver une solution si la donnée de départ est $x(0)= \alpha$, avec $\alpha$ quelconque dans l'intervalle $[-1, 1]$ ? Ce point est difficile si on ne connaît pas la méthode de séparation de variables. 
  \item Dessiner ``toutes'' les trajectoires trouvées.  
  \end{enumerate}
  
\corrref{TD6A-0002}

\end{exercice}
