% This is part of Exercices de mathématique pour SVT
% Copyright (c) 2010
%   Laurent Claessens et Carlotta Donadello
% See the file fdl-1.3.txt for copying conditions.

\begin{corrige}{TD6b-0004}

	\begin{enumerate}
		\item
			L'équation homogène correspondante est
			\begin{equation}		\label{Eqsqeqhomi}
				y'_H-2ty_H=0.
			\end{equation}
			En remplaçant $y'_H$ par $\frac{ dy_H }{ dt }$ et en passant tous les $t$ à droite et tous les $y$ à gauche,
			\begin{equation}
				\frac{ dy_H }{ y }=2tdt.
			\end{equation}
			En intégrant des deux côtés (c'est-à-dire en prenant les primitives et en n'oubliant pas la constante d'intégration), nous trouvons $\ln(y_H)=t^2+C$. Nous avons donc
			\begin{equation}
				y_H(t)= e^{t^2} e^{C}.
			\end{equation}
			Afin de simplifier les notations, nous allons écrire
			\begin{equation}
				y_H= Ke^{t^2}
			\end{equation}
			avec $K= e^{C}$. Pour chaque $K\in \eR$, la fonction $y_H(t)=K e^{t^2}$ vérifie l'équation homogène \eqref{Eqsqeqhomi}.

			Pour résoudre l'équation complète (non homogène), nous utilisons la \emph{méthode de variations des constantes}, c'est-à-dire que nous cherchons les solutions sous la forme $y(t)=K(t) e^{t^2}$. Nous remplaçons cette expression pour $y$ dans l'équation de départ :
			\begin{equation}
				K' e^{t^2}+2t e^{t^2}-2tK e^{t^2}=t
			\end{equation}
			où nous avons utilisé le fait que $y'=K' e^{t^2}+2Kt e^{t^2}$. Après simplifications, nous trouvons
			\begin{equation}
				K'=t e^{-t^2}.
			\end{equation}
			La fonction $K$ est donc une primitive de la fonction $t e^{-t^2}$, c'est-à-dire
			\begin{equation}
				K=\int t e^{-t^2}dt.
			\end{equation}
			Cette intégrale est l'intégrale de l'exercice \ref{exoTD5-0005}.\ref{Itemintizcxexsq}. La réponse est
			\begin{equation}
				K(t)=-\frac{ 1 }{2} e^{-t^2}+C.
			\end{equation}
			La solution de l'équation différentielle proposée est donc
			\begin{equation}
				y(t)=K(t) e^{t^2}=\big( -\frac{ 1 }{2} e^{-t^2}+C \big) e^{t^2},
			\end{equation}
			ou encore, après simplifications, 
			\begin{equation}
				y(t)=-\frac{ 1 }{2}+C e^{t^2}.
			\end{equation}
			
			Nous pouvons vérifier que cela est bien la solution de l'équation proposée en remplaçant dans l'équation de départ. D'abord $y'=2tC e^{t^2}$, donc
			\begin{equation}
				y'-2ty=2tC e^{t^2}-2t\big( -\frac{ 1 }{2}+C e^{t^2} \big)=2tC e^{t^2}+t-2tC e^{t^2}=t,
			\end{equation}
			ce qu'il fallait. Nous avons donc bien, pour chaque $C$, une solution de l'équation différentielle $y'-2ty=t$.

			Maintenant, nous demandons en plus que $y(1)= e^{-1/2}$. Le point à comprendre est que nous avons une solution à l'équation différentielle \emph{sans contrainte} pour chaque $C$. Nous pouvons fixer $C$ de telle manière à avoir une solution qui satisfait la contrainte :
			\begin{equation}
				y(1)=-\frac{ 1 }{2}+C e^{1}=-\frac{ 1 }{2}+eC.
			\end{equation}
			Si nous voulons que cela soit égal à $ e^{-1/2}$, nous devons résoudre l'équation
			\begin{equation}
				-\frac{ 1 }{2}+eC= e^{-1/2}.
			\end{equation}
			La solution est
			\begin{equation}
				C= e^{-1}\big(  e^{-1/2}+\frac{ 1 }{2} \big).
			\end{equation}
		\item

            \begin{verbatim}
----------------------------------------------------------------------
| Sage Version 4.7.1, Release Date: 2011-08-11                       |
| Type notebook() for the GUI, and license() for information.        |
----------------------------------------------------------------------
sage:  t=var('t')
sage: y=function('y',t)
sage: DE=diff(y,t)+y*tan(t)-sin(2*t)
sage: desolve(DE,[y,t])       
(c - 2*cos(t))/sec(t)
            \end{verbatim}
            La solution est 
            \begin{equation}
                y(t)=\big( -2\cos(t)+C \big)\cos(t).
            \end{equation}
            Pour obtenir la valeur de la constante \( C\) nous utilisons la condition initiale :
            \begin{equation}
                y(0)=(-2+C)=6
            \end{equation}
            par conséquent \( C=8\).

		\item
			Ici nous avons seulement une équation homogène. Il ne faudra donc pas utiliser la méthode de la variation des constantes. D'abord nous écrivons l'équation différentielle sous la forme
			\begin{equation}
				\frac{ dy }{ dt }=-y\cos(t),
			\end{equation}
			ensuite nous remettons tous les $y$ à gauche et tous les $t$ à droite :
			\begin{equation}
				\frac{ dy }{ y }=-\cos(t)dt.
			\end{equation}
			En intégrant des deux côtés, nous avons
			\begin{equation}
				\ln(y)=-\sin(t)+C
			\end{equation}
			et donc
			\begin{equation}		\label{Eqsbzqiiikesint}
				y= Ke^{-\sin(t)}
			\end{equation}
			où nous avons posé $K= e^{C}$.
			
			Pour vérifier le résultat, nous commençons par calculer la dérivée :
			\begin{equation}
				y'=-K\cos(t) e^{-\sin(t)},
			\end{equation}
			et ensuite nous remettons dans l'équation :
			\begin{equation}
				y'+y\cos(t)=-K\cos(t) e^{-\sin(t)}+K e^{-\sin(t)}\cos(t)=0,
			\end{equation}
			comme il se doit.

			Nous devons maintenant trouver pour quelle valeur de la constante $K$ nous avons $y(0)=10$. Pour cela nous posons $t=0$ dans l'équation \eqref{Eqsbzqiiikesint} :
			\begin{equation}
				y(0)=K e^{-\sin(0)}=K.
			\end{equation}
			Il faut donc poser $K=10$ pour obtenir $y(0)=10$. La réponse est donc
			\begin{equation}
				y(t)=10 e^{-\sin(t)}.
			\end{equation}
		\item
			L'équation homogène à résoudre est
			\begin{equation}
				t^3y'_H+(2-3t^2)y_H=0.
			\end{equation}
			En séparant les variables,
			\begin{equation}		\label{EqsbzqdyHiv}
				\frac{ dy_H }{ y_H }=-\frac{ 2-3t^2 }{ t^3 }.
			\end{equation}
			La primitive du membre de droite est presque l'exercice \ref{exoTD5-0005}.\ref{ItemTDczcii}. En intégrant \eqref{EqsbzqdyHiv}, nous trouvons
			\begin{equation}
				\ln(y_H)=3\ln(t)+t^{-2}+C,
			\end{equation}
			donc
			\begin{equation}
                y_H(t)=K e^{3\ln(t)+t^{-2}}=K\left(  e^{\ln(t)} \right)^3 e^{t^{-2}}=Kt^3 e^{1/t^2}.
			\end{equation}
			Cela est la solution générale de l'équation homogène associée à notre problème.

			En ce qui concerne le problème non homogène (c'est-à-dire avec le second membre), la technique de variation des constantes nous indique de chercher une fonction $K(t)$ pour laquelle la solution serait sous la forme
			\begin{equation}		\label{EqsbzqivyKt}
				y(t)=K(t)t^3 e^{t^{-2}}.
			\end{equation}
			Afin de trouver la fonction $K$, nous injectons \eqref{EqsbzqivyKt} dans l'équation différentielle à résoudre : nous calculons $t^3y'+(2-3t^2)y$ et nous imposons que le résultat soit $t^3$. Pour la dérivée de $y$ nous avons
			\begin{equation}
				\begin{aligned}[]
					y'&=K't^3 e^{t^{-2}}+K\left( t^3 e^{t^{-2}} \right)'\\
					&=K't^3 e^{t^{-2}}+K\left( 3t^2e^{t^{-2}}+t^3(-2)t^{-3}e^{t^{-2}} \right)\\
					&=e^{t^{-2}}\big[ K't^3+K(3t^2-2) \big].
				\end{aligned}
			\end{equation}
			Maintenant nous pouvons calculer l'équation
			\begin{equation}
				\begin{aligned}[]
					t^3y'+(2-3t^2)y&=t^3e^{t^{-2}}\big[ K't^3+K(3t^2-2) \big]+(2-3t^2)Kt^3e^{t^{-2}}\\
					&=t^3e^{t^{-2}}t^3K'.
				\end{aligned}
			\end{equation}
			La fonction $K$ est donc contrainte par la relation $t^3e^{t^{-2}}t^3K'=t^3$, c'est-à-dire
			\begin{equation}
				K'= e^{-t^{-2}}t^{-3}.
			\end{equation}
			La fonction $K$ est donc donné par l'intégrale
			\begin{equation}
				K=\int e^{-t^{-2}}t^{-3}dt
			\end{equation}
			qui s'obtient en posant $u=t^{-2}$, $dt=\frac{ du }{ -2t^{-3} }$. Le résultat est que
			\begin{equation}
				K(t)=\frac{ 1 }{2}\left(  e^{-t^{-2}}+C \right)
			\end{equation}
			Maintenant la solution générale de notre équation différentielle s'écrit
			\begin{equation}
				y(t)=K(t)t^3 e^{t^{-2}}=\frac{ t^3+Ct^3 e^{t^{-2}} }{ 2 }.
			\end{equation}
            Remarquons que cette solution n'est pas valable en \( t=0\).
			
	\end{enumerate}

\end{corrige}
