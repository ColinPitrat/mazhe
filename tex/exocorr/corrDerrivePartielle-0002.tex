% This is part of the Exercices et corrigés de mathématique générale.
% Copyright (C) 2010
%   Laurent Claessens
% See the file fdl-1.3.txt for copying conditions.

\begin{corrige}{DerrivePartielle-0002}

	Le vecteur directeur qu'on nous donne est 
	\begin{equation}
		v=P-Q=(-7,5,1).
	\end{equation}
	Nous devons trouver un point du graphe où le vecteur normal est parallèle à $v$. Pour ce faire, nous calculons le vecteur normal au point $\big( x,y,z(x,y) \big)$ pour un $(x,y)$ quelconque, et puis nous fixerons $x$ et $y$.

	Les dérivées partielles de $z$ par rapport à $x$ et $y$ sont
	\begin{equation}
		\begin{aligned}[]
			\frac{ \partial z }{ \partial x }&=8x\\
			\frac{ \partial z }{ \partial y }&=18y.
		\end{aligned}
	\end{equation}
	Au point $(x,y)$ cela donne les deux vecteurs tangents 
	\begin{equation}
		\begin{aligned}[]
			t_1&=(1,0,8x)\\
			t_2&=(0,1,8y).
		\end{aligned}
	\end{equation}
	Le vecteur normal est donné par
	\begin{equation}
		n=t_1\times t_2=\begin{vmatrix}
			e_x	&	e_y	&	e_z	\\
			1	&	0	&	8x	\\
			0	&	1	&	18y
		\end{vmatrix}=\begin{pmatrix}
			-8x	\\ 
			-18y	\\ 
			1	
		\end{pmatrix}.
	\end{equation}
	
	Afin que $n$ soit parallèle à $v$ (c'est-à-dire soit un multiple) nous cherchons à résoudre l'équation
	\begin{equation}
		\begin{pmatrix}
			-7	\\ 
			5	\\ 
			1	
		\end{pmatrix}=\lambda
		\begin{pmatrix}
			-8x	\\ 
			-18y	\\ 
			1	
		\end{pmatrix}.
	\end{equation}
	Il faut résoudre cela par rapport à $\lambda$, $x$ et $y$. Il n'y a cependant que $x$ et $y$ qui nous intéressent.

	La dernière ligne montre que $\lambda=1$ ensuite les deux autres lignes fournissent les équations
	\begin{subequations}
		\begin{numcases}{}
			-7=-8x\\
			5=-18y,
		\end{numcases}
	\end{subequations}
	et donc le point recherché est $(\frac{ 7 }{ 8 },-\frac{ 5 }{ 18 })$.

\end{corrige}
