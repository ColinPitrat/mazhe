% This is part of Mes notes de mathématique
% Copyright (c) 2011-2014
%   Laurent Claessens
% See the file fdl-1.3.txt for copying conditions.

\begin{corrige}{0028}


\begin{enumerate}
\item $\lim_{(x,y)\rightarrow (0,0)} \dfrac{x^2y}{x^4+y^6+1}$

	Cette fonction est le quotient de deux polynômes. Le polynôme au dénominateur est non nul en $(0,0)$, donc la fonction est continue en $(0,0)$, et 
\begin{equation}
\lim_{(x,y)\rightarrow (0,0)}f(x,y)= f(0,0) = 0.
\end{equation}
Nous avons utilisé le fait qu'une fonction est continue en un point si et seulement si elle vaut sa limite en ce point.


\item  $\lim_{(x,y)\rightarrow (0,0)} \dfrac{x^2-xy+y^3}{\sqrt{x^2+y^2}}$

	On va appliquer la règle de l'étau trois fois, aux trois fonctions:
	\[\dfrac{x^2}{\sqrt{x^2+y^2}}, \, \dfrac{|xy|}{\sqrt{x^2+y^2}}, \, \dfrac{|y^3|}{\sqrt{x^2+y^2}}\]

    \[\begin{array}{cccccc} 

	\dfrac{x^2}{\sqrt{x^2+y^2}}&\leq& \dfrac{x^2+y^2}{\sqrt{x^2+y^2}} &=& \sqrt{x^2+y^2} & \longrightarrow_{(x,y)\rightarrow (0,0)}0\\

	\dfrac{|xy|}{\sqrt{x^2+y^2}} & \leq &  \dfrac{|y|\sqrt{x^2+y^2}}{\sqrt{x^2+y^2}} &=& |y|  &\longrightarrow_{(x,y)\rightarrow (0,0)}0\\

	\dfrac{|y|^3}{\sqrt{x^2+y^2}} &\leq& \dfrac{|y|(x^2+y^2)}{\sqrt{x^2+y^2}} &=& |y|\sqrt{x^2+y^2} & \longrightarrow_{(x,y)\rightarrow (0,0)}0

    \end{array} \]
	ce qui prouve que  $\lim_{(x,y)\rightarrow (0,0)} \dfrac{x^2-xy+y^3}{\sqrt{x^2+y^2}} = 0$

\begin{alternative}
Considérons une boule de rayon $\epsilon$ autour de $(0,0)$. Tout point à l'intérieur de cette boule est de la forme $(r\cos\theta,r\sin\theta)$ avec $r\in[0,\epsilon]$ et $\theta\in[0,2\pi]$. Sur un tel point, la fonction vaut
\begin{equation}
	\begin{aligned}[]
	\left| \frac{ r^2\cos^2\theta -r^2\cos\theta\sin\theta+3r^3\sin^3\theta }{ r }\right|&=r\big| \cos^2\theta-\cos\theta\sin\theta+3r\sin^3\theta \big|\\
				&\leq 5r\leq 5\epsilon.
	\end{aligned}
\end{equation}

La fonction est donc bornée par $5\epsilon$ dans la boule de rayon $\epsilon$, et converge donc vers zéro.
\end{alternative}



\item   $\lim_{(x,y)\rightarrow (0,0)} \dfrac{x-y}{\ln(x^2+y^2+1)}$
																	    
Nous allons appliquer la méthode décrite au point \ref{SecLimVarsPlus}. L'ensemble des valeurs que $f$ atteint dans une boule de rayon $\delta$ autour de $(0,0)$ se calcule en utilisant les coordonnées polaires. En effet, tout point de cette boule s'écrit sous la forme $(x,y)=(r\cos\theta,r\sin\theta)$ pour $r\in[0,\delta]$ et $\theta\in[0,2\pi]$. Nous avons donc
\begin{equation}
	E_{\delta}=\big\{  \frac{ r(\cos\theta-\sin\theta) }{ \ln(r^2+1) } \big\}_{\substack{r\in]0,\delta]\\\theta\in[0,2\pi]}}.
\end{equation}
Remarquez que nous avons volontairement écrit $r\in]0,\delta]$ en excluant le $0$ de l'intervalle. De toutes façons, $f$ n'existe pas en $(0,0)$. Notons tout de suite que pour tout $\delta$, la valeur $0$ est dans $E_{\delta}$, en prenant $\sin\theta=\cos\theta$.

Voyons ce que nous pouvons trouver d'autre dans cet ensemble. Il est assez naturel, pour des questions de facilité, d'essayer avec $\sin\theta=0$ et $\cos\theta=1$. Pour tout $\delta$, l'ensemble $E_{\delta}$ contient tout les nombres
\begin{equation}
	\frac{ r }{ \ln(r^2+1) }
\end{equation}
avec $r\in]0,\delta]$. Il n'est pas très difficile, en utilisant la \href{http://fr.wikipedia.org/wiki/Règle_de_L'Hôpital}{règle de l'Hopital} de voir que 
\begin{equation}
	\lim_{r\to 0}\frac{ r }{ \ln(r^2+1) }=\infty.
\end{equation}
L'ensemble $E_{\delta}$ contient donc des valeurs arbitrairement élevées en même temps que $0$. Il ne peut donc pas y avoir convergence.


\begin{alternative}
	
	Bien que 	$\dfrac{x}{\ln(x)} \longrightarrow_{x\rightarrow 0^+}1$, on va voir que la limite n'existe pas ici. Prenons deux	 	manières de tendre vers $(0,0)$ qui donneront deux limites pour $f$ différentes. Ceci suffira pour prouver que la fonction n'admet pas de limite en $(0,0)$.

    \[\begin{array}{cccc} 

	x= 0, \; y\rightarrow  0^+ & \lim_{(x,y)\rightarrow (0,0)} \dfrac{x-y}{\ln(x^2+y^2+1)} & =& -\lim_{y\rightarrow 0^+}\dfrac{y}{\ln(1+y^2)} \\
	 & &=^H&-\lim\dfrac{1+y^2}{2y} \;= \; -\infty \\

	x\rightarrow 0^+, \; y=0 & \lim_{(x,y)\rightarrow (0,0)} \dfrac{x-y}{\ln(x^2+y^2+1) } & = & \lim_{x\rightarrow 0^+}\dfrac{x}{\ln(1+x^2)} \\
	& &=^H&\lim\dfrac{1+x^2}{2x} \;= \; +\infty 
     \end{array}\]

\end{alternative}


\item   $\lim_{(x,y)\rightarrow (1,1)} \dfrac{x-y}{\ln(x^2+y^2-1)}$

Passer en coordonnées polaires demande de poser $x=1=r\cos\theta$ et $y=1+r\sin\theta$. Nous trouvons donc l'expression suivante pour $f(r,\theta)$ (où le centre des coordonnées polaires est $(1,1)$, et non $(0,0)$):
\begin{equation}
	\frac{ r(\cos\theta-\sin\theta) }{ \ln\big( r^2+2r(\cos\theta+\sin\theta)+1 \big) }.
\end{equation}
Avec $\theta=\pi/4$, nous trouvons que cela vaut toujours $0$, ce qui prouve que dans tout voisinage de $(1,1)$, la fonction $f(x,y)$ prend la valeur zéro. En prenant $\cos\theta=1$ et $\sin\theta=0$, nous trouvons par contre
\begin{equation}
	f(r)=\frac{ r }{ \ln(r^2+2r+1) }=\frac{ r }{ 2\ln(r+1) },
\end{equation}
donc la limite pour $r\to 0$ vaut $1$. Cela prouve que, en choisissant un voisinage de $(1,1)$ assez petit, la fonction $f$ prend des valeurs arbitrairement proches de $1$. Cela rend impossible la condition \eqref{Eqvmoinsrapplimdeux} pour quelque $l$ que ce soit.

\begin{alternative}
	Prenons deux manières de tendre vers $(1,1)$ qui donneront deux limites pour $f$ différentes. Ceci suffira pour prouver que la fonction n'admet pas de limite en $(1,1)$.

    \[\begin{array}{cccc} 

	x= 1, \; y\rightarrow  1 & \lim_{(x,y)\rightarrow (1,1)} \dfrac{x-y}{\ln(x^2+y^2-1)} & =& \lim_{y\rightarrow 1}\dfrac{1-y}{\ln(y^2)} \\
	 & &=^H&\lim\dfrac{-1}{-2y/y^2} \;= \; \f{1}{2}\\

	x\rightarrow 1, \; y=1 & \lim_{(x,y)\rightarrow (1,1)} \dfrac{x-y}{\ln(x^2+y^2-1) } & = & \lim_{x\rightarrow 1}\dfrac{x}{\ln(x^2)} \\
	& &=^H&\lim\dfrac{1}{-2x/x^2} \;= \; -\f{1}{2} 
     \end{array}\]
										    
\end{alternative}
									
	\item $\lim_{(x,y)\rightarrow (0,0)} \dfrac{xy^2}{x^2+y^4}$

	Prenons deux manières de tendre vers $(0,0)$ qui donneront deux limites pour $f$ différentes. Ceci suffira pour prouver que la fonction n'admet pas de limite en $(0,0)$.

    \[\begin{array}{cccccc} x=y^2 & \lim_{(x,y)\rightarrow (0,0)} \dfrac{xy^2}{x^2+y^4} & =& \lim_{y\rightarrow 0}\dfrac{y^4}{2y^4} &=& \f{1}{2} \\

        x=0, \; y\rightarrow 0 & \lim_{(x,y)\rightarrow (0,0)} \dfrac{xy^2}{x^2+y^4} & =& \lim_{y\rightarrow 0}\dfrac{0}{y^4} &=& 0 \end{array}\]

\end{enumerate}

\end{corrige}
