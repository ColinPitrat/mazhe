% This is part of Exercices et corrigés de CdI-1
% Copyright (c) 2011,2015
%   Laurent Claessens
% See the file fdl-1.3.txt for copying conditions.

\begin{corrige}{0019}

\begin{enumerate}
\item Formellement, une suite $(x_k)_{k\in\eN}$ est une fonction $f : \eN \to \eR : k \mapsto x_k$, et le résultat de l'exercice \ref{exo0004} s'applique. On en déduit que pour tout indice $j$, nous avons
 \begin{equation*}
    \sup\{ x_k + y_k \tq k \geq j \} \leq   \sup\{ x_k \tq k \geq j \} +  \sup\{ y_k \tq k \geq j \}.
  \end{equation*}
Ici, pour chaque $j$, nous avons pris $E_j=\{ k\tq k\geq j \}$ en guise de $E$ de l'exercice \ref{exo0004}. Il suffit maintenant de passer à la limite $(j \to \infty)$ pour obtenir le résultat.
\item
Prendre $x_k = (-1)^k$ et $y_k = -x_k$. L'inégalité devient $0 <
  2$.

\item Si l'une des suites, disons $x_k$, converge, alors
  \begin{equation*}
    \limsup (x_k) = \lim (x_k) = - \lim(-x_k) = - \limsup (-x_k)
  \end{equation*}
  et on en déduit que
  \begin{equation*}
    \begin{split}
      \limsup (y_k) &= \limsup(y_k + x_k + (-x_k))\\
      &\leq \limsup(y_k + x_k) + \limsup (-x_k)\\
      &= \limsup(y_k + x_k) - \limsup (x_k)
    \end{split}
  \end{equation*}
  ce qui fournit l'inégalité inverse de celle obtenue au premier
  point, d'où l'égalité.

\item Prendre $x_k = (-1, 0, -1, 0, \ldots)$ et $y_k =
  x_k$. L'inégalité devient $1 > 0$.

\item 

Ce qui a fait fonctionner l'exemple du point précédent, c'est qu'un produit de nombres négatifs a donné un nombre positif, ce qui a permit, dans le produit,  de passer au dessus de zéro, tandis que dans chacune des suites séparément, la limite supérieure était zéro. Ce stratagème ne fonctionne pas sans nombres négatifs.

%On veut pouvoir appliquer la propriété
%  \begin{equation*}
%    (0 \leq a \leq b \text{~et~}0\leq a^\prime \leq b^\prime)
%    \Rightarrow 0 \leq a a^\prime \leq b b^\prime
%  \end{equation*}
%  et on va donc supposer $x_k, y_k \geq 0$. Montrons que cette
%  condition répond à la question posée :

Nous allons voir que la condition $x_k,y_k$ répond à la question posée. Soit $l \in \eN$ fixé. Pour tout $k \geq l$, nous savons
  \begin{equation*}
    0 \leq x_k \leq \sup\{ x_i \tq i \geq l\} \text{~et~} 0
    \leq y_k \leq \sup\{ y_i \tq i \geq l\}
  \end{equation*}
  et on en déduit%, toujours pour $k \geq l$,
  \begin{equation*}
    0 \leq x_k y_k \leq \sup\{ x_i \tq i \geq l\}  \sup\{ y_i \tq i \geq l\}
  \end{equation*}
  % c'est-à-dire que $\sup\{ x_i \tq i \geq l\} \sup\{ y_i \tq
  % i \geq l\}$ majore l'ensemble $\{x_ky_k \tq k \geq l \}$,
  et donc
  \begin{equation*}
    \sup \{x_ky_k \tq k \geq l \} \leq \sup\{ x_i \tq i \geq l\}  \sup\{ y_i
    \tq i \geq l\}
  \end{equation*}
  d'où le résultat attendu en prenant la limite pour $l \rightarrow
  +\infty$.

\item Prendre $x_k = (1, 0, 1, 0, \ldots)$ et $y_k = (0, 1, 0, 1, 0,
  \ldots)$, c'est-à-dire la même suite décalée d'un indice pour que le
  produit soit la suite nulle. L'inégalité devient $0 < 1$.

\item 
Supposons que l'une des suites converge, par exemple $(x_k) \to r$. Discutons deux cas : si $r \neq 0$, on a
  \begin{equation}
    \limsup (x_k) = \lim (x_k) = \frac{1}{\lim(\frac{1}{x_k})} = \frac{1}{\limsup(\frac{1}{x_k})}
  \end{equation}
  et on en déduit
  \begin{equation}
    \begin{split}
      \limsup (y_k) &= \limsup(y_k \cdot x_k / x_k))\\
      &\leq \limsup(y_k x_k) \limsup \left(\frac1{x_k}\right)\\
      &= \frac{\limsup(y_k x_k)}{\limsup (x_k)}
    \end{split}
  \end{equation}
  ce qui fournit l'inégalité inverse de \eqref{EqLimSupxyLeqLSxLSy}, d'où l'égalité. Lorsque $r = 0$, l'inégalité déjà obtenue est en fait
  \begin{equation}
    \limsup (x_ky_k) \leq 0
  \end{equation}
  mais d'après la condition sur $x_k$ et $y_k$, on sait aussi que
  \begin{equation}
    0\leq\limsup (x_ky_k)
  \end{equation}
  et on en déduit l'égalité.
\end{enumerate}

\end{corrige}
