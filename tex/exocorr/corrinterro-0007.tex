% This is part of Exercices de mathématique pour SVT
% Copyright (C) 2010
%   Laurent Claessens et Carlotta Donadello
% See the file fdl-1.3.txt for copying conditions.

\begin{corrige}{interro-0007}

	\begin{enumerate}
		\item
			Il n'y a pas d'indétermination : lorsque $x$ tend vers l'infini, $x^3$ tend vers $-\infty$.
		\item
			Mise en évidence de $x$ au numérateur et au dénominateur et simplification par $x$ :
			\begin{equation}
				\frac{ x-2 }{ 2x-3 }=\frac{ x\left( 1-\frac{ 2 }{ x } \right) }{ x\left( 2-\frac{ 3 }{ x } \right)=\frac{ 1-\frac{ 2 }{ x } }{ 2-\frac{ 3 }{ x } } }.
			\end{equation}
			Lorsque nous faisons la limite $x\to \infty$, les fractions $\frac{ 2 }{ x }$ et $-\frac{ 3 }{ x }$ tendent vers zéro et il ne reste que
			\begin{equation}
				\lim_{x\to \infty} \frac{ 1-\frac{ 2 }{ x } }{ 2-\frac{ 3 }{ x } }=\frac{ 1 }{2}.
			\end{equation}
		\item
			Ici par contre, il n'y a pas d'indéterminations : si on remplace $x$ par zéro, nous obtenons immédiatement $\frac{ 2 }{ 3 }$.
		\item
			En remplaçant, nous avons $\frac{ 0 }{ 0 }$, et par conséquent une indétermination à lever. Nous pouvons factoriser le numérateur en $x^2-4=(x+2)(x-2)$ et simplifier la fraction par $x-2$ :
			\begin{equation}
				\frac{ x^2-4 }{ (x-2)(x+1) }=\frac{ (x+2)(x-2) }{ (x-2)(x+1) }=\frac{ x+2 }{ x+1 },
			\end{equation}
			et
			\begin{equation}
				\lim_{x\to 2} \frac{ x+2 }{ x+1 }=\frac{ 4 }{ 3 }.
			\end{equation}
		\item
			La fonction $\frac{ \sin(x) }{2}$ est une fonction périodique et oscille donc tout le temps. Elle n'a pas de limite.
	\end{enumerate}

\end{corrige}
