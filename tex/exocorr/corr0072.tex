% This is part of Exercices et corrigés de CdI-1
% Copyright (c) 2011,2014
%   Laurent Claessens
% See the file fdl-1.3.txt for copying conditions.

\begin{corrige}{0072}

Quelque remarques. N'oubliez pas que l'adhérence d'un ensemble non vide n'est \emph{jamais} vide : l'ensemble lui-même est toujours dans son adhérence.
\begin{enumerate}

\item
\item
\item
L'ensemble $\eZ$ n'a pas d'intérieur. En effet, prenons $n\in\eZ$ et considérons la boule $B(n,e)$. Cette boule contient évidement des éléments qui ne sont pas dans $\eZ$. Nous avons donc prouvé qu'aucune boule centrée en $n$ n'est entièrement comprise dans $\eZ$. Cela prouve que $\eZ$ n'as pas d'intérieur.

L'adhérence de $\eZ$ est réduite à $\eZ$ lui-même parce que si $x\notin\eZ$, il existe un $r$ tel que $B(x,r)\cap\eZ=\emptyset$, ce qui prouve que $x$ n'est pas adhérent à $\eZ$.

\item
L'ensemble $\eQ$ n'as pas d'intérieur pour la même raison que $\eZ$ : il existe un élément hors de $\eQ$ aussi proche que l'on veut de tout rationnel. L'adhérence de $\eQ$, par contre est tout $\eR$. En effet, si $x\in\eR$ et si $r>0$, la boule $B(x,r)$ intersecte toujours $\eQ$.
\item
\item
Notez que
\begin{equation}
	\bigcap_{n=1}^{\infty}\mathopen]-\frac{1}{ n },\frac{1}{ n }\mathclose[=\{ 0 \}.
\end{equation}
\end{enumerate}

Les réponses complètes sont dans le tableau suivant. Parmi les propriétés d'être ouvert, fermé, borné, compact ou connexe par arcs (c.p.a), les ensembles cités jouissent des propriétés mentionnées mais pas des autres.

\noindent
%\begin{bigcenter}
  %\begin{tabular}{|l|*{3}{|>{$}c<{$}}|l|}
  \begin{tabular}{|c||c|c|c|c|}
    \hline
    Ex. & \text{Intérieur} & \text{Adhérence} & \text{Frontière} & Propriétés\\
    \hline
    (a) & $]-\sqrt3,\sqrt3\mathclose[\cup\mathopen]10,11\mathclose[$ &%
    $\mathopen[-\sqrt3,\sqrt3\mathclose]\cup\mathopen[10,11\mathclose]$ &%
    $\{-\sqrt3,\sqrt3,10,11\}$&borné\\
    (b) & $\mathopen]2;3\mathclose[\setminus\{e\}$ & $[2;3]$ & $\{2, e, 3\}$ &
    borné\\
    (c) & $\emptyset$ & $\eZ$ & $\eZ$ & fermé\\
    (d) & $R$ & $R$ & $\emptyset$ & ouvert, fermé, c.p.a\\
    (e) & $\emptyset$ &
    $\left\{ \sfrac1i \tq i \in \eZ_0  \right\} \cup \{0\}$&
    $\left\{ \sfrac1i \tq i \in \eZ_0  \right\} \cup \{0\}$&\\
    (f) & $\emptyset$ & $\{0\}$ & $\{0\}$ &
    \begin{tabular}{l}
      fermé, borné,\\
      compact, c.p.a
    \end{tabular}\\
    \hline
  \end{tabular}
%\end{bigcenter}

\end{corrige}
