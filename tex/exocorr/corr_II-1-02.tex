\begin{corrige}{_II-1-02}

Ceci est encore une équation à variables séparées, dont la solution se recherche en suivant la même méthode que pour l'exercice \ref{exo_II-1-01}. En suivant la même notation, nous avons $u(t)=1$ et $f(y)=3y^{2/3}$, ce qui donne $U(t)=t$ et $G(y)=y^{1/3}$. Il faut faire ici une remarque : dans la proposition 1 de la page 311, nous demandons que $f(\eta)\neq 0$ pour tout $\eta\in J$ où $J$ est l'ensemble d'arrivée de la fonction $y$. Les solutions que nous allons trouver devront donc être restreintes au domaine $y(t)\neq 0$. Nous y reviendrons.

La solution générale est donc fournie par l'équation $y^{1/3}=t+C$, c'est à dire
\begin{equation}		\label{EqGen102}
	y(t)=(t+C)^3.
\end{equation}
La solution telle que $y(0)=27$ est donc
\begin{equation}
	\begin{aligned}
		y\colon \eR&\to \eR \\
		t&\mapsto (t+3)^3. 
	\end{aligned}
\end{equation}

Toute solution $y$ tel que $y(\tau)\neq 0$ pour un certain $\tau$ a la forme \eqref{EqGen102} sur $\eR\setminus\{ -C \}$. Cette solution s'étend immédiatement à $\eR$ tout entier. La seule solution que le théorème nous a donc fait oublier est la solution identiquement nulle.

 À part la solution identiquement nulle, la condition $y(0)=0$ accepte la solution $y(t)=t^3$, qui est celle avec $C=0$.

\end{corrige}
% This is part of the Exercices et corrigés de CdI-2.
% Copyright (C) 2008, 2009
%   Laurent Claessens
% See the file fdl-1.3.txt for copying conditions.


