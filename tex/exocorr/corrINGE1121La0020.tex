% This is part of the Exercices et corrigés de mathématique générale.
% Copyright (C) 2009-2010,2017
%   Laurent Claessens
% See the file fdl-1.3.txt for copying conditions.


\begin{corrige}{INGE1121La0020}

	Pour les valeurs propres, nous calculons le polynôme caractéristique :
	\begin{equation}
		P_A(\lambda)=\det\begin{pmatrix}
			2-\lambda	&	0	&	\alpha	\\
			0	&	-1-\lambda	&	0	\\
			0	&	0	&	2-\lambda
		\end{pmatrix}
		=
		-(2-\lambda)^2(1+\lambda).
	\end{equation}
	Cela fournit les valeurs propres $\lambda_1=2$ (de multiplicité $2$) et $\lambda_2=-1$.

	Commençons par trouver les vecteurs propres correspondants à la valeur propre $\lambda_2=-1=$. Pourquoi nous commençons par celle-là ? Parce qu'il n'y a pas de discussion sur $\alpha$ pour lui. La matrice du système est
	\begin{equation}
		\begin{pmatrix}
			3	&	0	&	\alpha	\\
			0	&	0	&	0	\\
			0	&	0	&	3
		\end{pmatrix}.
	\end{equation}
	La troisième ligne dit tout de suite que $z=0$m et donc la première ligne donne $x=0$ (sans discussions). Ensuite, $y$ est sans contraintes. L'espace propre de la valeur propre $-1$ est donc toujours engendré par
	\begin{equation}
		v_{-1}=\begin{pmatrix}
			0	\\ 
			1	\\ 
			0	
		\end{pmatrix}.
	\end{equation}

	En ce qui concerne l'espace propre pour la valeur $\lambda_1=2$, nous avons le système de matrice
	\begin{equation}
		\begin{pmatrix}
			0	&	0	&	\alpha	\\
			0	&	-3	&	0	\\
			0	&	0	&	0
		\end{pmatrix}.
	\end{equation}
	La seconde ligne dit que $y=0$. La première dit que $\alpha z=0$. Il y a donc deux possibilités :
	\begin{enumerate}

		\item
			$\alpha\neq 0$. Alors $z=0$ est obligatoire et l'espace propre est engendré par
			\begin{equation}
				v_{2}=\begin{pmatrix}
					1	\\ 
					0	\\ 
					0	
				\end{pmatrix}.
			\end{equation}
			Dans ce cas, même si la valeur propre était de multiplicité deux, il n'y a que une seule dimension de vecteurs propres.
		\item
			Si $\alpha=0$, alors on ne va pas discuter longtemps : la matrice est déjà diagonale depuis le début. On lit tout de suite dessus que les vecteurs propres sont les trois vecteurs de la base canonique.

	\end{enumerate}
	
\end{corrige}
