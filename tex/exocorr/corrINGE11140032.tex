% This is part of Un soupçon de physique, sans être agressif pour autant
% Copyright (C) 2006-2009
%   Laurent Claessens
% See the file fdl-1.3.txt for copying conditions.


\begin{corrige}{INGE11140032}

	Pour la continuité, ces exercices se réduisent à calculer la limite des fonctions en le point à problème, et vérifier si cette limite est égale ou non à la valeur de la fonction en ce point, en vertu du théorème qui lie la limite et la continuité.

	\begin{enumerate}

		\item
			Cette fonction est définie partout, et vu que $\sin(x)/x\to 1$, elle est continue partout.
		\item
			Elle est définie partout, et elle est continue parce que 
			\begin{equation}
				\lim_{x\to 0} x\sin(\frac{1}{ x })=0.
			\end{equation}
			Cette limite a été calculée à l'exercice \ref{exoINGE11140031}\ref{ItemexoINGE_11140031xsinusx}.
		\item
			La fonction n'est définie que pour $x\geq 0$. De plus,
			Nous avons que 
			\begin{equation}
				\lim_{x\to 0} \frac{ \sqrt{x+1} }{ x }=\infty,
			\end{equation}
			donc cette fonction n'est pas continue en $0$. 

		\item
			La fonction est définie sur tout $\eR$. Il faut maintenant regarder les deux points de discontinuité possibles. Le premier revient à calculer la limite
			\begin{equation}
				\lim_{x\to 0} \frac{ x+1 }{ x }\sin\left( \frac{ x }{ x+1 } \right),
			\end{equation}
			qui est un cas du type $\frac{ \sin(A) }{ A }$ avec $A$ qui tend vers zéro. Nous savons que cette limite vaut $1$. Nous avons donc que $\lim_{x\to 0} f(x)=1$, mais $f(0)=0$. La fonction n'est donc pas continue en $0$.

			Pour le second point, nous devons calculer
			\begin{equation}
				\lim_{x\to -1} \frac{ x+1 }{ x }\sin\left( \frac{ x }{ x+1 } \right).
			\end{equation}
			Pour cette limite, nous utilisons la règle de l'étau. Le sinus reste borné entre $-1$ et $1$, tandis que le facteur $\frac{ x+1 }{ x }$ tend vers zéro. Nous avons donc $\lim_{x\to -1} f(x)=0$ et donc la fonction est continue en $-1$.

		\item
			Cette fonction est définie et continue sur tout $\eR$ parce qu'elle est une composée de fonctions continues~: la valeur absolue et l'exponentielle.

		\item
			Cette fonction est définie tant que $1+| x |>0$, ce qui est toujours le cas. Elle est également continue partout.
			
	\end{enumerate}

\end{corrige}
