% This is part of the Exercices et corrigés de mathématique générale.
% Copyright (C) 2010
%   Laurent Claessens
% See the file fdl-1.3.txt for copying conditions.

\begin{corrige}{DerrivePartielle-0010}

	En utilisant le théorème de Pythagore, la fonction $f$ s'écrit
	\begin{equation}
		f(x,y,z)=x^2+y^2+(z-2)^2,
	\end{equation}
	mais comme nous ne nous intéressons qu'aux points tels que $z=xy$, nous regardons la fonction
	\begin{equation}
		f(x,y)=x^2+y^2+(xy-2)^2.
	\end{equation}
	Nous devons étudier les extrema de cette fonction.

	Le code Sage qui fait les calculs est :

	\VerbatimInput[tabsize=3]{tex/sage/exo109.sage}

	La sortie :

	\VerbatimInput[tabsize=3]{tex/sage/exo109.txt}

	Il faut rejeter les solutions qui contiennent le nombre imaginaire \verb+I+. Pour le reste, la solution est là.

	Le point délicat est de résoudre les équations
	\begin{subequations}
		\begin{numcases}{}
			(y^2+1)x-2y=0\\
			(x^2+1)y-2x=0.
		\end{numcases}
	\end{subequations}
	Nous isolons $x$ dans la première équation :
	\begin{equation}
		x=\frac{ 2y }{ (y^2+1) },
	\end{equation}
	et nous le remettons dans la seconde :
	\begin{equation}
		\left( \frac{ 4y^2 }{ (y^2+1)^2 }+1 \right)y-\frac{ 4y }{ y^2+1 }=0.
	\end{equation}
	La première solution qu'on voit est $y=0$ qui implique $x=0$. Le point $(0,0)$ est donc un point critique.

	Si $y\neq 0$ nous pouvons simplifier par $y$ et mettre au même dénominateur $(y^2+1)^2$ :
	\begin{equation}
		4y^2+(y^2+1)^2-4(y^2+1)=0.
	\end{equation}
	En simplifiant nous trouvons une équation bicarrée : $y^4+2y^2-3=0$. Nous posons $u=y^2$ et nous trouvons l'équation $u^2+2u-3=0$. En ce qui concerne $u$ nous avons les solutions $u=1$ et $u=-3$. La solution $u=-3$ est à rejeter parce que $u=y^2$ (c'est sans doute ici qu'arrivent les solutions complexes trouvées par Sage). Reste donc $y=\pm 1$.

\end{corrige}
