% This is part of the Exercices et corrigés de CdI-2.
% Copyright (C) 2008, 2009
%   Laurent Claessens
% See the file fdl-1.3.txt for copying conditions.


\begin{corrige}{_I-3-7}

Le développement de cet exercices est semblable à celui de l'exercice \ref{exo_I-3-6}. D'abord, nous étudions la convergence de l'intégrale de la dérivée
\begin{equation}
	\frac{ \partial  }{ \partial m }\left( \frac{ \cos(mx) }{ 1+x^2 } \right)=-\frac{ x\sin(mx) }{ 1+x^2 }.
\end{equation}
Pour cela, nous utilisons le critère d'Abel avec
\begin{equation}
	\begin{aligned}[]
		\varphi(m,x)	&=-\sin(mx)\\
		\psi(m,x)	&=\frac{ x }{ 1+x^2 }.
	\end{aligned}
\end{equation}
Une primitive de $\sin(mx)$ est donnée par $-\frac{ \cos(mx) }{ m }$, donc
\begin{equation}		\label{EqPrimIntI378}
	\int_0^X\sin(mx)dx=\frac{1}{ m }\big( 1-\cos(mX) \big)
\end{equation}
qui ne peut pas être bornée indépendamment de $m$. Le critère d'Abel ne peut donc être appliqué que dans un compact. Nous faisons donc à nouveau le coup du compact, et nous appliquons Abel sur chaque compact (en la variable $m$, pas $x$ !!) de $]0,\infty[$. Le cas $m=0$ devra donc être traité à part. Si $m$ est borné vers le bas par $m_0$, alors nous pouvons borner \eqref{EqPrimIntI378} par
\begin{equation}
	| \int_0^X\sin(mx)dx |\leq \frac{ 2 }{ m_0 }.
\end{equation}
Par ailleurs, les fonctions $m\mapsto \psi(m,x)$ sont constantes (et valent $x/(1+x^2)$) et tendent uniformément vers zéro quand $x$ tend vers l'infini. Le critère d'Abel s'applique donc sur le compact dont le minimum est $m_0$ et nous trouvons que l'intégrale
\begin{equation}
	\int_0^{\infty}\frac{ x\sin(mx) }{ 1+x^2 }dx
\end{equation}
est uniformément convergente sur tout compact. Nous en concluons que $I(m)=-\frac{ \partial J }{ \partial m }(m)$ est continue. Et nous avons
\begin{equation}
	I(m)=\begin{cases}
	\frac{ \pi }{2} e^{-m}	&	\text{si }x>0\\
	0	&	 \text{si }x=0.
\end{cases}
\end{equation}

\end{corrige}
