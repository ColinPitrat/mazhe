% This is part of the Exercices et corrigés de mathématique générale.
% Copyright (C) 2010
%   Laurent Claessens
% See the file fdl-1.3.txt for copying conditions.

\begin{corrige}{FoncDeuxVar0024}

	Si nous notons $u$ et $v$ les variables de $f$, nous écrivons la formule générale
	\begin{equation}
		\frac{ \partial F }{ \partial \rho }=\frac{ \partial f }{ \partial u }\big( x(\rho,\theta),y(\rho,\theta) \big)\frac{ \partial x }{ \partial \rho }(\rho,\theta)+=frac{ \partial f }{ \partial v }\big( x(\rho,\theta),y(\rho,\theta) \big)\frac{ \partial y }{ \partial \rho }(\rho,\theta)
	\end{equation}
	Nous avons
	\begin{equation}
		\begin{aligned}[]
			\frac{ \partial f }{ \partial u }(u,v)&=\frac{1}{ u-v }	&\frac{ \partial f }{ \partial v }(u,v)&=\frac{-1}{ u-v }\\
			\frac{ \partial x }{ \partial \rho }&=\cos(\theta)	&\frac{ \partial y }{ \partial \rho }&=\sin(\theta),
		\end{aligned}
	\end{equation}
	et donc
	\begin{equation}		\label{EqVQdFdtho}
		\begin{aligned}[]
			\frac{ \partial F }{ \partial \rho }(\rho,\theta)&=\frac{1}{ x-y }\cos(\theta)+\frac{ -1 }{ x-y }\sin(\theta)\\
			&=\frac{ \cos(\theta) }{ \rho(\cos\theta-\sin\theta) }-\frac{ \sin(\theta) }{ \rho(\cos\theta-\sin\theta) }\\
			&=\frac{1}{ \rho }.
		\end{aligned}
	\end{equation}

	Avec le même genre de calculs;
	\begin{equation}
		\frac{ \partial F }{ \partial \theta }=\frac{ \sin(\theta)+\cos(\theta) }{ \sin(\theta)-\cos(\theta) }.
	\end{equation}

	En substituant $\rho=1$ et $\theta=$ dans l'équation \eqref{EqVQdFdtho}, nous avons
	\begin{equation}
		\frac{ \partial F }{ \partial \rho }(0,1)=\frac{1}{ 1 }=1.
	\end{equation}
	

\end{corrige}
