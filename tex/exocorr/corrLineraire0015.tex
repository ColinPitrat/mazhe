% This is part of the Exercices et corrigés de mathématique générale.
% Copyright (C) 2009,2016
%   Laurent Claessens
% See the file fdl-1.3.txt for copying conditions.
\begin{corrige}{Lineraire0015}

	Une condition très intéressante pour savoir si trois vecteurs forment une base et de les mettre dans une matrice et de voir le déterminant. S'il est non nul, c'est une base; s'il est nul, ce n'est pas une base. Ici, on a
	\begin{equation}
		\begin{vmatrix}
			0	&	5	&	a	\\
			1	&	-2	&	b	\\
			-1	&	1	&	c
		\end{vmatrix}=
		-5(c+b)+a(1-2).
	\end{equation}
	La condition pour que les trois soient non libres est que le déterminant soit nul, c'est-à-dire
	\begin{equation}
		a+5(b+c)=0.
	\end{equation}
	

\end{corrige}
