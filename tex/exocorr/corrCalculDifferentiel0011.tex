\begin{corrige}{CalculDifferentiel0011}

	Passons aux coordonnées polaires, c'est à dire posons
	\begin{equation}
		\begin{aligned}[]
			x&=r\cos(\theta)	&r&=(x^2+y^2)^{1/2}\\
			y&=r\sin(\theta)	&\theta&=\arctan\left( \frac{ y }{ x } \right).
		\end{aligned}
	\end{equation}
	Nous considérons la fonction $\tilde f$ définie par
	\begin{equation}		\label{EqeCDchemuudevarfftrtxy}
		\tilde f(r,\theta)=f\big( x(r,\theta),y(r,\theta) \big),
	\end{equation}
	et ensuite nous voyons $\tilde f$ comme une fonction composée. Nous avons
	\begin{equation}
		\begin{aligned}[]
			\frac{ \partial f }{ \partial x }&=\frac{ \partial \tilde f }{ \partial r }\frac{ \partial r }{ \partial x }+\frac{ \partial \tilde f }{ \partial \theta }\frac{ \partial \theta }{ \partial x }\\
			&=\partial_r\tilde f\frac{ x }{ \sqrt{x^2+y^2} }+\partial_{\theta}\tilde f\frac{1}{ 1+\left( \frac{ y }{ x } \right)^2 }\left( -\frac{ y }{ x^2 } \right)\\
			&=\partial_r\tilde f\frac{ x }{ \sqrt{x^2+y^2} }-\partial_{\theta}\tilde f\frac{ y }{ x^2+y^2 }.
		\end{aligned}
	\end{equation}
	En ce qui concerne la dérivée par rapport à $y$, nous faisons
	\begin{equation}
		\frac{ \partial f }{ \partial y }=\partial_r\tilde f\frac{ y }{ \sqrt{x^2+y^2} }+\partial_{\theta}\frac{ x }{ x^2+y^2 }.
	\end{equation}
	Nous pouvons maintenant chercher à résoudre l'équation \eqref{EqeCDuueqares}. Nous l'écrivons sous la forme
	\begin{equation}
		1=\frac{ x }{ \sqrt{x^2+y^2} }\frac{ \partial f }{ \partial x }+\frac{ y }{ \sqrt{x^2+y^2} }\frac{ \partial f }{ \partial y }.
	\end{equation}
	Nous avons donc
	\begin{equation}
		1=\left( \frac{ x^2 }{ x^2+y^2 }+\frac{ y^2 }{ x^2+y^2 } \right)\partial_{r}\tilde f+\left( \frac{ -yx+xy }{ (x^2+y^2)^{3/2} } \right)\partial_{\theta}\tilde f
	\end{equation}
	L'équation se réduit donc à
	\begin{equation}		\label{EqeCDuueqsimp}
		\frac{ \partial \tilde f }{ \partial r }=1,
	\end{equation}
	et par conséquent
	\begin{equation}
		\tilde f(r,\theta)=r+c(\theta).
	\end{equation}
	Pour obtenir cela, nous avons pris l'intégrale de \eqref{EqeCDuueqsimp} par rapport à $r$ et nous avons considéré que la constante d'intégration pouvait dépendre de $\theta$. Nous trouvons maintenant la forme générale de $f$ en utilisant la définition \eqref{EqeCDchemuudevarfftrtxy} «à l'envers» :
	\begin{equation}
		f(x,y)=\tilde f\big( r(x,y),\theta(x,y) \big)=r(x,y)+c\big( \theta(x,y) \big)=\sqrt{x^2+y^2}+c\big( \arctan\left( \frac{ y }{ x } \right) \big).
	\end{equation}
	Au final, ce que nous avons prouvé, c'est que pour toute fonction $c\colon \eR\to \eR$, la fonction
	\begin{equation}
		f(x,y)=\sqrt{x^2+y^2}+c\big( \arctan\left( \frac{ y }{ x } \right) \big)
	\end{equation}
	est une solution de l'équation proposée. De plus, toutes les solutions s'écrivent de cette manière pour une certaine fonction $c$.

	Étant donné que nous demandons des fonction $C^1$, nous demandons que la fonction $c$ soit $C^1$.

\end{corrige}
