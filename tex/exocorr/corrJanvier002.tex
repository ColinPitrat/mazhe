% This is part of the Exercices et corrigés de mathématique générale.
% Copyright (C) 2009
%   Laurent Claessens
% See the file fdl-1.3.txt for copying conditions.
\begin{corrige}{Janvier002}


Commençons par mettre le nombre proposé sous forme trigonométrique $\rho e^{i\theta}$. La norme de $1+\sqrt{3}$ est $\sqrt{1+(\sqrt{3})^2}=2$. Afin de trouver l'angle $\theta$, nous savons que
\begin{equation}
	1+\sqrt{3}i=2(\frac{ 1 }{2}+\frac{ \sqrt{3} }{ 2 }i),
\end{equation}
qui doit être égal à $2 e^{i\theta}$. L'angle $\theta$ est donc l'angle qu'il faut pour que $\cos(\theta)=\frac{1}{ 2 }$ et $\sin(\theta)=\frac{ \sqrt{3} }{ 2 }$, c'est à dire $\theta=\pi/3$.

Nous devons donc calculer $\left( 2 e^{i\pi/3} \right)^{39}=2^{39} e^{39i\pi/3}=-2^{39}$, parce que $\frac{ 39\pi }{ 3 }=12\pi$, et que l'angle $12\pi$ est le même que l'angle $\pi$ et que $ e^{i\pi}=-1$.

\end{corrige}
