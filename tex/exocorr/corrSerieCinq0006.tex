% This is part of Exercices et corrections de MAT1151
% Copyright (C) 2010
%   Laurent Claessens
% See the file LICENCE.txt for copying conditions.

\begin{corrige}{SerieCinq0006}

	Notons d'abord que la matrice $T^tT$ est toujours symétrique parce que $(T^tT)^t=T^tT$.

	Prouvons maintenant par récurrence que la matrice $A$ peut être écrite sous la forme $T^tT$ où $T$ est triangulaire. Supposons que $A$ puisse être écrite sous la forme
	\begin{equation}
		A=\begin{pmatrix}
			T_0^tT_0	&	\begin{matrix}
				x_1	\\ 
				\vdots	\\ 
				x_n	
			\end{matrix}\\ 
			x_1\ldots x_n	&	x_{n+1}	
		\end{pmatrix}
	\end{equation}
	où $T_0$ est une matrice triangulaire. Nous allons chercher la matrice triangulaire $T$ telle que $T^tT=A$ sous la forme
	\begin{equation}
		T=\begin{pmatrix}
			 *	&	*	&	*	&	t_1	\\
			 0	&	*	&	*	&	\vdots	\\
			 0	&	0	&	a	&	t_n	\\ 
			 0	&	0	&	0	&	t_{n+1}	 
		 \end{pmatrix}
	\end{equation}
	où les $*$ (ainsi que le $a$) forment la matrice $T_0$.

	En égalisant $A=T^tT$, nous trouvons d'abord
	\begin{equation}	\label{EqCCtnpuxnpu}
		t_{n+1}^2=x_{n+1}.
	\end{equation}
	Mais nous avons vu que $x_{n+1}>0$ en tant qu'élément diagonal de $A$. Par conséquent la relation \eqref{EqCCtnpuxnpu} déterme bien $t_{n+1}$. Nous avons même le choix du signe. Quel luxe !
	
	Ensuite nous trouvons $x_n=t_n(a+t_{n+1})$, ce qui fournit $t_n$ sous la forme
	\begin{equation}
		t_n=\frac{ x_n }{ a+t_{n+1} }.
	\end{equation}
	Nous avons dit que nous avions le choix du signe pour $t_{n+1}$. Nous le choisissons positif. Comment prouver que le dénominateur n'est pas nul ? En fait $a$ est l'élément en bas à droite de la matrice $T_0$ qui est la matrice qui réalise $T_0^tT=A_0$, et donc nous pouvons supposer que $a$ est strictement positif exactement comme nous venons de prendre $t_{n+1}$ strictement positif. En effet, $a$ n'est rien d'autre que le «$t_{n+1}$» du pas de récurrence précédent.

\end{corrige}
