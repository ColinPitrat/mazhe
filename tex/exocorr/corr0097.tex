% This is part of Exercices et corrigés de CdI-1
% Copyright (c) 2011
%   Laurent Claessens
% See the file fdl-1.3.txt for copying conditions.

\begin{corrige}{0097}

Prouvons que $f$ est nécessairement monotone, c'est à dire que $f(y)-f(x)$ a un signe constant tant que $y>x$. L'application $(x,y)\mapsto f(y)-f(x)$ est une application continue de $I\times I$ vers $\eR$ qui ne passe par zéro que lorsque $x=y$ (parce que $f$ est inversible), donc elle ne peut pas changer de signe dans la partie $y>x$.

Maintenant, la proposition \ref{PropIntContMOnIvCont} prouve l'exercice.

\end{corrige}
