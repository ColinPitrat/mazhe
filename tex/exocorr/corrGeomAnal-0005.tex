\begin{corrige}{GeomAnal-0005}
Nous avons quatre propriétés à vérifier. Dans la suite $v$ est un élément de $\eR^3$.
\begin{enumerate}
\item L'application $\| . \|_{\infty}$ sur $\eR^3$ prends ses valeurs en $\eR^+$ :
  \begin{itemize}
  \item la valeur absolue est une fonction non négative : pour tout $r$ dans $\eR$ on a $|r|= r$ si $r$ est positif et $|r|= -r$ si $r$ est négatif. Il est clair que $|r|=0$ si et seulement si $r=0$ ;
  \item le maximum parmi trois nombres $\geq 0$ ne peut que être $\geq 0$.
  \end{itemize}
\item $\| v \|_{\infty}= 0$ si et seulement si $v=(0, 0, 0)^t$. 
  \begin{itemize}
  \item $\| (0,0,0)^t \|_{\infty}= \max\{0, 0 ,0\}= 0$ ;
  \item soit $v$ un vecteur non nul tel que $\| v \|_{\infty}= 0$. Si $v$ est non nul au moins une parmi les composantes de $v$ est non nulle. Sans perdre en généralité on dira qu'il s'agit de $v_1$. Alors $|v_1|> 0$. La définition de maximum nous dit que $|v_1|\leq \| v \|_{\infty}$. On a obtenu la suite d'inégalités $0\geq v_1>0$. Absurde.
  \end{itemize}
\item Pour tout $\lambda$ in $\eR$ on a  $\| \lambda v \|_{\infty}=|\lambda|\| v \|_{\infty}$. 
  \begin{itemize}
  \item soit $v=(v_1, v_2, v_3)^t$, alors $\lambda v= (\lambda v_1, \lambda v_2, \lambda v_3)^t$. Nous savons  que $|\lambda v_i|= |\lambda| |v_i|$ et que la multiplication fois un nombre positif ne change pas le sens des inégalités : si $a$ et $b$ sont dans $\eR$ et  $a<b$, alors $|\lambda| a < |\lambda| b$. On obtient alors 
\[
\| \lambda v \|_{\infty}=\max\{|\lambda| |v_1|,|\lambda| |v_2|, |\lambda| |v_3| \}= |\lambda| \max\{|v_1|, |v_2|, |v_3| \}=|\lambda|\| v \|_{\infty}.
\]   
  \end{itemize}
\item Inégalité triangulaire : $ \|v+w \|_{\infty}\leq\| v \|_{\infty}+\| w \|_{\infty}$.
  \begin{itemize}
  \item soient $v$ et $w$ deux vecteurs dans $\eR^3$. La somme $v+w$ est un vecteur de composantes $(v_1+w_1,v_2+w_2,v_3+w_3)$. On a alors 
\[
\|v+w \|_{\infty}=\max_{1\leq i\leq 3}\{|v_i+w_i|\}.
\]
Pour fixer les idées nous pouvons supposer que le maximum soit $|v_1+w_1|$. Une propriété de la valeur absolue nous dit que $|v_1+w_1|\leq|v_1|+|w_1|$. Pour conclure il nous suffit de remarquer que $|v_1|\leq \max\{|v_1|, |v_2|, |v_3| \}=\| v \|_{\infty}$ et, de même,  $|w_1|\leq \| w \|_{\infty}$.  
  \end{itemize}
\end{enumerate}

La boule unité est un cube centré dans l'origine et de côté 2.  

   
 

\end{corrige}
