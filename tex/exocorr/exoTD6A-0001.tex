\begin{exercice}\label{exoTD6A-0001}
  
Soient $\alpha$ et $k$ deux nombres réels positifs fixés. Vérifier que la fonction
\begin{equation}
  x(t)= k e^{\ln\left(\frac{C}{k}\right)e^{-\alpha t}}
\end{equation}
est une solution de l'équation différentielle 
\begin{equation}
  \frac{dx}{dt}=\alpha x\ln\left(\frac{k}{x}\right),
\end{equation}
pour tout choix de $C>0$, $C\in\mathbb{R}$. 

Calculez la limite de $x(t)$ lorsque $t$ tend vers $+\infty$ et vérifiez que la constante $C$ est la valeur de la fonction $x(t)$ à l'instant $t=0$.
  
\corrref{TD6A-0001}

\end{exercice}
