% This is part of the Exercices et corrigés de CdI-2.
% Copyright (C) 2008, 2009
%   Laurent Claessens
% See the file fdl-1.3.txt for copying conditions.


\begin{exercice}\label{exo_I-3-10}

Soit la \defe{fonction d'Euler}{Euler!fonction}
\begin{equation}
	\Gamma(x)=\int_0^{\infty} e^{-t}t^{x-1}dt.
\end{equation}
\begin{enumerate}
\item démontrer que $\Gamma(x)\in C^{\infty}\big( ]0,\infty[ \big)$,
\item prouver la relation
\begin{equation}		\label{EqGammaFacto}
	\Gamma(x+1)=x\Gamma(x),
\end{equation}
en déduire que  $\Gamma(n)=(n-1)!$ pour tout entier $n\geq 1$.

\item montrer que $\Gamma(\frac{1 }{2})=2\int_0^{\infty} e^{-u^2}du$,
\item montrer que 
\begin{equation}
	\big( \Gamma(\frac{ 1 }{2}) \big)^2=\lim_{R\to\infty}\iint_D e^{-(u^2+v^2)}du\,dv
\end{equation}
où $D$ est un disque de rayon $R$ centré à l'origine. En déduire que $\Gamma(\frac{ 1 }{2})=\sqrt{\pi}$.

\item
Calculer
\begin{equation}
	\int_{0}^{\infty} e^{-t}\sqrt{t}dt.
\end{equation}

\item
montrer que 
\begin{equation}
	\int_0^1x(\ln x)^{-1/3}dx=\frac{ -2^{1/3} }{ 2 }\Gamma(\frac{ 2 }{ 3 }).
\end{equation}

\end{enumerate}


\corrref{_I-3-10}
\end{exercice}
