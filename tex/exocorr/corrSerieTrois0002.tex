% This is part of Exercices et corrections de MAT1151
% Copyright (C) 2010
%   Laurent Claessens
% See the file LICENCE.txt for copying conditions.

\begin{corrige}{SerieTrois0002}

	Nous allons noter $f$ au lieu de $x$ la fonction recherchée parce que sinon il y a des problèmes de notation à n'en plus finir parce que la donnée est notée $x_0$. Nous nous voyons mal écrire des expressions comme $x(x)(t)$. Au lieu de ça, nous écrivons $f_{x_0}(t)$ la solution au temps $t$ du problème avec $x_0$ comme paramètre.

	La solution exacte du problème est donnée par
	\begin{equation}
		f_{x_0}(t)=x_0 e^{at}\cos(t).
	\end{equation}
	Pour trouver le conditionnement absolu, il faut étudier le rapport
	\begin{equation}
		\frac{ \| f_x-f_{x_0} \| }{ | x-x_0 | }
	\end{equation}
	où la norme du numérateur est la norme sur l'espace de fonctions considéré (par exemple la norme supremum). Nous avons que
	\begin{equation}
		f_x(t)-f_{x_0}(t)=(x-x_0) e^{at}\cos(t),
	\end{equation}
	et donc
	\begin{equation}
		\| f_x-f_{x_0} \|=| x-x_0 |\cdot\| t\mapsto  e^{at}\cos(t) \|
	\end{equation}
	parce que les normes sont $\eR$-linéaire nonobstant une valeur absolue. Nous trouvons donc
	\begin{equation}
		K_{\text{abs}}(x_0)=\| t\mapsto  e^{at}\cos(t) \|.
	\end{equation}
	Notons que cela est infini pour la norme supremum si $a>0$ et si on considère $\eR$ comme domaine de variation de $t$. Quoi qu'il en soit, pour le conditionnement relatif,
	\begin{equation}
		\| f_{x_0} \|=| x_0 |\cdot\| t\mapsto e^{at}\cos(t) \|,
	\end{equation}
	et donc
	\begin{equation}
		K_{\text{rel}}(x_0)=\| t\mapsto  e^{at}\cos(t) \|\cdot\frac{ | x_0 | }{ \| f_{x_0} \| }  =1.
	\end{equation}
	Au passage, il y a une simplification par l'infini si $a>0$.

\end{corrige}
