% This is part of the Exercices et corrigés de mathématique générale.
% Copyright (C) 2009-2011
%   Laurent Claessens
% See the file fdl-1.3.txt for copying conditions.
\begin{corrige}{General0025}

La fonction est $y=\sqrt{x}e^x$, qui n'existe pas en dessous de $x=0$, qui vaut $0$ en $x=0$ et qui est croissante. En utilisant la formule générale du solide de révolution, 
\begin{equation}
	V=\pi\int_0^1x e^{2x}dx,
\end{equation}
qui est à intégrer par partie. Nous trouvons
\begin{equation}
	V=\pi\left[ \frac{ (2x-1) e^{2x} }{ 4 } \right]_0^1=\pi\left( \frac{ e^2 }{ 4 }+\frac{1}{ 4 } \right).
\end{equation}

\end{corrige}
