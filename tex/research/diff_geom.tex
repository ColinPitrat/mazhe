
%%%%%%%%%%%%%%%%%%%%%%%%%%
%
   \subsection{Notions of topology}
%
%%%%%%%%%%%%%%%%%%%%%%%%

Let $X$ be a set. A \defe{topology}{} on $X$ is a choice of a set $\tau$ of parts of $X$ such that

\begin{itemize}
\item $X\in\tau$, $\emptyset\in\tau$,
\item $\bigcap_{\text{finite}}A_i\in\tau$ if $A_i\in\tau$,
\item $\bigcup A_i\in\tau$ if $A_i\in\tau$.
\end{itemize}
Elements of $\tau$ are said \defe{open}{} in $X$ for the topology $\tau$. Let us point out the difference between the second and the third point: the third point allows infinite unions. If we consider $A_i=]0,1+\frac{1}{ i }[\subset\eR$, we have
\[ 
  \bigcap_{i\in\eN_0}A_i=]0,1]
\]
which is not open in $\eR$.

If $E\subset X$, the \defe{induced}{} topology on $E$ by $X$ is the one for which a subset $A$ of $E$ is open in $E$ if and only if there exists a subset $\mU$ of $X$ which is open in $X$ and such that $A=E\cap\mU$. In other words, $A\in\tau_E$ if and only if $A=E\cap\mU$ for a certain $\mU\in\tau_X$.

When $X$ and $Y$ are topological spaces, we say that $\dpt{ f}{ X }{ Y }$ is \defe{continuous}{} if for all $A\in\tau_Y$, $f^{-1}(A)\in\tau_X$, that is, if the inverse image of any open set is open. Let us see a non trivial example. 

We consider $X$ as $\eR$ endowed with the usual topology and $Y$ as $\eR$ with the topology $\tau_Y=\{ \emptyset,\eR \}$. We have $X=Y$ as sets, but $X\neq Y$ as topological spaces. The identity map $\dpt{ \id }{ X }{ Y }$ is continuous because $\id^{-1}(\emptyset)=\emptyset\in\tau_X$ and $\id^{-1}(\eR)=\eR\in\tau_X$. But the identity $\dpt{ \tilde\id }{ Y }{ X }$ is not continuous because $\tilde\id^{-1}(]0,1[)=\,]0,1[$ which is not open in $Y$.


%%%%%%%%%%%%%%%%%%%%%%%%%%
%
   \subsection{Differential, path and extensions}
%
%%%%%%%%%%%%%%%%%%%%%%%%

Let's take $X_0\in T_{x_0}C$, a map $\dpt{ f }{ C }{ C' }$ and an extension $\hat f$ of $f$ on a neighbourhood $V$ around $x_0$. If one consider the path $\dpt{ \eta }{ \eR }{ V }$, $\eta(t)=x_0+tX_0$, we know that
\[ 
\begin{split}
  d\hat f_{x_0}\big(\eta'(0)\big)&=\Dsdd{ \hat f\big( \eta(t) \big) }{t}{0}\\
				&=\frac{ \partial \hat f }{ \partial x^i }\frac{ d\eta^i }{ dt }.
\end{split}
\]
On the latter expression, we see that if $\eta(0)=\gamma(0)=x_0$ and $\gamma'(0)=\eta'(0)$, then $df_{x_0}\big( \gamma'(0) \big)=df_{x_0}\big( \eta'(0) \big)$. In particular, we can chose a path $\gamma$ such that $\gamma(t)\in c$ for all $t$. For this path, we can put $f\big( \gamma(t) \big)$ in the derivative instead of $\hat f\big( \gamma(t) \big)$.

Hence we can consider the differential of a map $\dpt{ f }{ C }{ C' }$ as an object intrinsically defined on $C$ without extensions. So we set $df_{x_0}\colon T_{x_0}C\to T_{f(x_0)}C$,
\[ 
  df_{x_0}(\eta'(0))=\Dsdd{ (f\circ\eta)(t) }{t}{0}.
\]

%%%%%%%%%%%%%%%%%%%%%%%%%%
%
   \subsection{Vector fields}
%
%%%%%%%%%%%%%%%%%%%%%%%%


A \emph{tangent vector field}  along an embedded curve $C$ in $\eR^2$ is a rule which assigns to every point
$x\in C$ a vector $X_x$ tangent to $C$ at $x$. Moreover, one requires the dependence on $x$
``to be smooth''. 

More precisely, let $T(C)$ denote the set of all vectors tangent to $C$:
\begin{equation} \label{eq_TC_bigcup}
 T(C)=\bigcup_{x\in C}T_x(C) 
\end{equation}
where the union is a \emph{disjoint} union. A \defe{tangent vector field }{} on $C$ is a map $\dpt{ X }{ C }{ T(C) }$, $x\mapsto X_x$ such that
\begin{enumerate}
\item for all $x$, one has

\begin{equation} \label{eq_XxTxC}
 X_x\in T_x(C), 
\end{equation}

\item the map $C\to \eR^2\times\eR^2$,

\begin{equation} \label{eq_xmapstoxXx}
 x\mapsto(x,X_x) 
\end{equation}
is smooth.

The latter is a non trivial point since $C$ is not open in $\eR^2$.

\end{enumerate}

\begin{exercice}
Some students noticed that equation \eqref{eq_xmapstoxXx} was a wrong because of \eqref{eq_XxTxC}: firstly $T_xC$ is vector space of dimension one (so $X_x$ cannot belong to a $\eR^2$) and secondly, the union \eqref{eq_TC_bigcup} being a disjoint union, if $x$ and $y$ are different points of $C$, $X_x$ and $X_y$ belongs to two different spaces: $T_xC$ and $T_yC$.

Try to understand the problem and to find a solution.
\end{exercice}

%%%%%%%%%%%%%%%%%%%%%%%%%%
%
   \subsection{Cohomology}
%
%%%%%%%%%%%%%%%%%%%%%%%%

When $C$ is a curve in $\eR^2$, the \defe{cohomology}{} of $C$ is the quotient 
\[ 
  H^1(C)=\frac{ \Omega^1(C) }{ d C^{\infty}(C) }
\]
of differential forms on $C$ by forms which are differential of some functions. When $\alpha\in\Omega^1(C)$, 
\[ 
  [\omega]=\{ \omega+df\,| f\in C^{\infty}(C) \}.
\]
We are going to study two examples: $H^1(\eR)$ and $H^1(S^1)$.


\subsubsection{Cohomology of $\eR$}
%--------------------------------

We begin by defining the form $dt$ on $\eR$ by the following action of a tangent vector $\dot\gamma(t)=v\in\eR$: $dt(\dot\gamma(t))=v$. Each form on $\eR$ can be written as
\[ 
  \alpha=a\,dt
\]
where $a$ is a smooth function on $\eR$. Why? Because for each $x\in\eR$, $T^*_x\eR$ has dimension one. We are going to prove the following:

\begin{theorem}
   The cohomology of $\eR$ is zero:
\[ 
  H^1(\eR)=0.
\]
\end{theorem}

In other words, each form $\alpha$ on $\eR$, reads $df$ for a suitable function $f\in C^{\infty}(\eR)$. For the proof, we explicitly give a function $f$ such that $\alpha=df$ for any $\alpha\in\Omega^1(\eR)$. 

Let $\alpha\in\Omega^1(\eR)$ and let us write it as $\alpha_x=a(x)\,dt\in T_x^*(\eR)$. We pose
\[ 
  f(x)=\int_0^xa(u)\,du
\]
and we compute that $df=\alpha$. Let $\gamma$ be a path in $\eR$,
\begin{equation}
\begin{split}
df_{\gamma(0)}\dot\gamma(0)&=\Dsdd{ (f\circ\gamma)(t) }{t}{0}\\
		&=\Dsdd{ \int_0^{\gamma(t)}a(u)\,du }{t}{0}\\
		&=a(\gamma(0))\gamma'(0).
\end{split}
\end{equation}
On the other hand,
\begin{equation}
\begin{split}
\alpha_{\gamma(0)}\dot\gamma(0)&=a(x)\big( dt\dot\gamma(0) \big)\\
			&=a(\gamma(0))\dot\gamma(0);
\end{split}
\end{equation}
this is the same!


\subsubsection{Cohomology of $S^1$}
%--------------------------------

In the whole part about $S^1$, $\gamma$ will denote the path
\[ 
  \gamma(t)=e^{it},
\]
and $\sigma$ will denote a parametrization of $S^1$ used in integrals.

We begin to show that $H^1(S^1)\neq0$: there exists some $1$-forms on $S^1$ that cannot be written under the form $df$. For this, we begin to prove that

\begin{proposition} \label{prop_intdfz}
For all $f\in C^{\infty}(S^1)$,
\[ 
  \int_{S^1}df=0.
\]
\end{proposition}

\begin{proof}
If $\dpt{ \sigma }{ [0,2\pi] }{ S^1 }$ is a parametrization of $S^1$, we have
\begin{equation}
\begin{split}
  \int_{S^1}df&=\int_{[0,2\pi]}df_{\sigma(t)}\dot\sigma(t)\,dt\\
		&=\int_0^{2\pi}\Dsdd{ (f\circ\sigma)(u) }{u}{t}\,dt\\
		&=(f\circ\sigma)(2\pi)-(f\circ\sigma)(0)\\
		&=0.
\end{split}
\end{equation}
\end{proof}

We define the differential $1$-form $\alpha$ by
\[ 
  \alpha(\dot\gamma(t))=1.
\]
So
\[ 
  \int_{s^1}\alpha=\int_0^{2\pi}\alpha_{\sigma(t)}(\dot \sigma(t))\,dt=2\pi\neq0.
\]
Then this $\alpha$ is not an exact form.

There exists an alternative way to define this form. We consider the function $\dpt{ \theta }{ S^1 }{ \eR }$ given by 
\begin{equation} \label{eq_deftheta}
\theta(e^{i\eta})=\eta.
\end{equation}
 This is the function which gives the angular coordinate of a point in $S^1$. As a map from $S^1$ to $\eR$, one can compute its differential applied on a tangent vector; for example on $\dot\gamma(t)$:
\[ 
  d\theta\big( \dot\gamma(t) \big)=\Dsdd{ \theta(e^{it}) }{t}{0}=\Dsdd{ t }{t}{0}=1,
\]
so this is the previously defined form $\alpha$. By the way, it proves that this form is a smooth differential form.

We have proved that $d\theta$ is the differential of the function $\theta$ and that it is not an exact form (because its integral is not zero). It is a problem. In fact, equation \eqref{eq_deftheta} doesn't \emph{globally} defines a function on $S^1$ because $\theta(e^{i\cdot 0})=0$ but $\theta(e^{i\cdot 2\pi})=2\pi$. In other words, $\theta(1)$ is not well defined. Then we can only define the function $\theta$ on a local chart. For example on the half circle, we define $\dpt{ \theta }{ S^1 }{ ]-\frac{ \pi }{ 2 },\frac{ \pi }{2}[ }$ by $\theta(e^{i\sigma})=\sigma$. This is a well defined function.

The following theorem is the main part of the proof that $H^1(S^1)=\eR$.

\begin{theorem}
   A form $\alpha\in\Omega^1(S^1)$ fulfil
\[ 
  \int_{S^1}\alpha=0
\]
if and only if there exists a function $\dpt{ f }{ S^1 }{ \eR }$ such that $df=\alpha$.
\end{theorem}
Proposition  \ref{prop_intdfz} is the first half of the proof. Let's see the second one

\begin{proof}
Let $\alpha\in\Omega^1(S^1)$ such that $\int_{S^1}\alpha=0$ . It can be written under the form 
\[ 
  \alpha_{e^{i\theta}}=a(e^{i\theta})\,d\theta
\]
and we define $\dpt{ f }{ S^1 }{ \eR }$ by
\[
f(e^{i\theta})=\int_0^{\theta}a(e^{i\sigma})\,d\sigma.
\]
Since $\alpha_{e^{i\theta}}\big( \dot\gamma(\sigma) \big)=a(e^{i\theta})\,d\theta\big( \dot\gamma(\sigma) \big)=a(e^{i\theta})$, in fact $f$ is 
\begin{equation}
f(e^{i\theta})=\int_0^{\theta}\alpha.
\end{equation}
It is important to remark that this function has not the problem of definition of the previously $\theta$ because $\int_{S^1}\alpha=0$.

Now we want to prove that $df=\alpha$. For this we consider a path $c(t)=e^{i\theta(t)}$ in $S^1$ and we compute $df_{e^{i\theta(0)}}c'(0)$. We have:
\begin{equation}
\begin{split}
  df_{e^{i\theta(0)}}c'(0)&=\Dsdd{ (f\circ c)(t) }{t}{0}\\
			&=\Dsdd{ \int_0^{\theta(t)}a(e^{i\sigma})\,d\sigma }{t}{0}\\
		&=a(e^{i\theta(0)})\theta'(0).
\end{split}
\end{equation}
On the other hand,
\begin{equation}
\begin{split}
  \alpha_{e^{i\theta(0)}}c'(0)&=a(e^{i\theta(0)})\,d\theta\big( c'(0) \big)\\
		&=a(e^{i\theta(0)})\Dsdd{ \theta(e^{i\theta(t)}) }{t}{0}\\
		&=a(e^{i\theta(0)})\Dsdd{ \theta(t) }{t}{0}\\
		&=a(e^{i\theta(0)})\theta'(0).
\end{split}
\end{equation}
This concludes the proof.

\end{proof}

It is now easy to see that 
\[ 
  I:[\alpha]\mapsto\int_{S^1}\alpha 
\]
defines a vector space isomorphism between $H^1(S^1)$ and $\eR$. We check that $I$ is well defined: if $[\alpha]=[\beta]$, we must have $I([\alpha])=I([\beta])$. It is correct because $[\alpha]=[\beta]$ means that there exists a function $f$ such that $\beta= \alpha+df$. Then 
\[ 
  I([\beta])=\int_{S^1}\beta=\int_{S^1}(\alpha+df)=  \int_{S^1}\alpha=I([\alpha]). 
\]


%%%%%%%%%%%%%%%%%%%%%%%%%%
%
   \subsection{Integral curve}
%
%%%%%%%%%%%%%%%%%%%%%%%%

A map $\dpt{ c }{ \eR }{ \eR^n }$ is an \defe{integral curve}{} of the vector field $X\in\cvec(\eR^n)$ if for all $t$ (in the domain of $c$),
\begin{equation}  \label{eq_defintcurv}
c'(t)=X_{c(t)}.
\end{equation}
As a vector field, $X$ is a map $\dpt{ X }{ \eR^n }{ \eR^n }$, so equation \eqref{eq_defintcurv} reads
\[ 
  c'(t)=X\big( c(t) \big).
\]
Now, ``to be an integral curve'' means ``to satisfy a differential equation''. You know some existence and unicity theorems; for example this one:

\begin{theorem}  \label{tho_exuniintcur}
  Let $\dpt{ u }{ I }{ \eR }$ and $\dpt{ f }{ J }{ \eR }$ where $J$ is open in $\eR$ and $I$ has non empty interior. We consider the following differential equation:
\begin{equation}\label{eq:eqdiff}
y'(t)=u(t)f(y(t)).
\end{equation}
Suppose that $u$ is continuous on $I$ and $f$ is continuous on $J$ with $f(\eta)\neq 0$ for all $\eta\in J$. Let $t_0\in I$ and $y_0\in J$. Then there exists an interval $I'\subset I$ with non empty interior with $t_0\in I'$ and a $C^1$ function $\dpt{ y }{ I' }{ J }$ such that 

\begin{enumerate}
\item $y$ is solution of \eqref{eq:eqdiff} on $I'$ and fulfils the initial condition  $y(t_0)=y_0$,
\item if $z$ is a solution of \eqref{eq:eqdiff} in an interval $I''\subset I$ with $t_0\in I''$ and $z(t_0)=y_0$, then $I''\subset I'$ and $z(t)=y(t)$ for $t\in I''$.
\end{enumerate}

\end{theorem}

When one says that an embedded curve $C$ is an integral curve of a vector field $X$, one means that there exists a system of charts $\dpt{ \varphi_{\alpha} }{ \mU_{\alpha} }{ C }$ on this curve which solve condition  \eqref{eq_defintcurv} as map $\dpt{ \varphi_{\alpha} }{ \mU_{\alpha} }{ \eR^n }$. In particular, $X$ must be tangent to $C$ on each point.

%%%%%%%%%%%%%%%%%%%%%%%%%%
%
   \subsection{Christoffel symbols}
%
%%%%%%%%%%%%%%%%%%%%%%%%

When $f\colon \mU\to \eR^3$ is a parametrization of a surface $S$ in $\eR^3$, there exists two interesting tangent vectors fields:
\begin{equation} \label{eq_basvecto}
 e_x|_{x_0,y_0}= \Dsdd{ f(x_0+t,y_0) }{t}{0}\text{ and } e_y|_{(x_0,y_0)}=\Dsdd{ f(x_0,y_0+t) }{t}{0}.
\end{equation}
They form a basis of $T_{f(x_0,y_0)}S$. These vectors are often called $\partial_i$. You can ask yourself why this partial derivative notation.

The problem of finding the first fundamental  form (see exercise \ref{exo011}) reduce to computes products $e_1\cdot e_j$. These products form the matrix $g$:
\[ 
  g_{ij}:=e_i\cdot e_j.
\]
 Now, there are no reasons for $\partial_i e_j$ to be a tangent vector. In fact, it is even difficult to \emph{define} the object $\partial_i e_j$ because it should be something like
\[ 
  \partial_i e_j=\lim_{t\to 0}\frac{ e_j(x+t)-e_j(x) }{ t }.
\]
But $e_j(x)$ and $e_j(x+t)$ are not vectors on the same points, therefore the difference on the right hand side is not well defined as difference in $TS$. This difference is however well defined in the ambient space $\eR^3$. So in order to define the derivative of a tangent vector field, we have to consider the surface as a part of $\eR^3$. In other words, we have to extend the coordinate system $(x,y)\to S$ into a coordinate system $(x,y,z)\to\eR^3$ which reduce to original coordinates when $z=z_0$ and such that the vector $e_z$ is normal.

As an example, polar coordinates on the (unit) sphere in $\eR^2$ are given by 
\[ 
  (\varphi,\theta)\to (\cos\varphi\sin\theta,\cos\varphi\cos\theta,\sin\varphi)
\]
and an extension of the system is the full spherical coordinates on $\eR^3$:
\[ 
   (\varphi,\theta,r)\to (r\cos\varphi\sin\theta,r\cos\varphi\cos\theta,r\sin\varphi);
\]
it is easy to see that $e_r$ is normal. Now, our new set of $e_i$'s form a basis of the whole $\eR^3$, then we define the \defe{Christoffel symbols}{} $\Gamma_{ij}^{k}$ of the new coordinates system
\begin{equation} \label{eq_def_sch}
\partial_ie_j=\sum_k\Gamma_{ij}^{k}e_k.
\end{equation}
In this formula, $k$ is summed over \emph{all} coordinates, including the normal one. 

 There exists a funny formula to explicitly compute these Christoffel symbols. We begin by taking the derivative $\partial_k$ of $g_{ij}=e_i\cdot e_j$, where $k$, $i$ and $j$ are tangent coordinates. So we do not take care about the normal coordinate. Remark that in formula
\[
\partial_kg_{ij}=\Gamma_{ki}^{l}e_l\cdot e_j+e_i\cdot\Gamma_{kj}^{l}e_l,
\]
the summed index $l$ take \emph{a priori} all values including the normal coordinate, but the scalar product with $e_j$ and $e_i$ kill the normal component. So in the sequel we are left with only tangent coordinates. Replacing the products $e_l\cdot e_j$ and $e_i\cdot e_l$ by the corresponding elements of the matrix $g$,
\[ 
  \partial_kg_{ij}=\Gamma_{ki}^{l}g_{lk}+\Gamma_{kj}^{l}g_{il}.
\]
Computing the combination $\partial_kg_{ij}+\partial_ig_{jk}-\partial_jg_{ki}$ and taking the symmetry property $\Gamma_{ij}^{k}=\Gamma_{ji}^{k}$, $g_{ij}=g_{ji}$ into account, we find the relation
\[ 
  2\Gamma_{ki}^{l}g_{lj}=\partial_kg_{ij}+\partial_ig_{jk}-\partial_jg_{ki}.
\]
If we write $g^{ab}:=(g^{-1})_{ab}$, we can rewrite it as
\begin{equation}  \label{eq_Gamaform}
\Gamma_{ij}^{k}=\frac{ 1 }{2}g^{lk}\big( \partial_ig_{jl}+\partial_jg_{li}-\partial_lg_{ij} \big).
\end{equation}
This is the promised formula. Now we are able to write down elements $\Gamma_{ij}^{k}$ by simply derive and invert the matrix $g$.

In formula \eqref{eq_Gamaform}, indices (including the summed one) only take values in tangent coordinates. The normal one is completely forgotten.


\paragraph{Information}
%----------------------

Here is the definition given in the oral course. Let $r:U\subset\eR^3\to\Sigma\subset\eR^3$ be a chart on an embedded surface $\Sigma$.
Set $r_i:=\partial_{x_i}r\;;\;N:=\frac{1}{\|r_1\times r_2\|}r_1\times r_2\quad(i=1,2)$. Then the \defe{Christoffel symbols}{}
$\Gamma_{ij}^k$ are defined by the relations:
\[
\partial_{x_i}(r_j)\;:=\;\sum_k\Gamma_{ij}^kr_k\;+\;\nu_{ij}N\;.
\]

Do  you understand why the latter formula has two terms while formula  \eqref{eq_def_sch} has only one? Is there \emph{really} one term less in \eqref{eq_def_sch}?

\subsection{Commutator of vector field}
%--------------------------------------

Let $M$ be a surface and $X$, $Y\in\cvec(M)$. We want to define $[X,Y]\in\cvec(M)$, so for each $x\in M$ and $f\in C^{\infty}(M)$ we have to define $[X,Y]_xf$. We know that $X_x$ is an element in $T_xM$ which acts on $f$ by
\[ 
  X_x(f)=\Dsdd{ f\big( X_x(t) \big) }{t}{0}=df_xX_x\quad\in\eR,
\]
so one can consider $Xf$ as a function, $(Xf)(x)=X_xf$. We can apply the vector $Y_x$ to this function and get a real:
\[ 
\begin{split}
  Y_x(Xf)&=\Dsdd{ (Xf)(Y_x(t)) }{t}{0}\\
	&=\Dsdd{ f\big( X_{Y_x(t)}(u) \big) }{t}{0}{s}{0}.
\end{split}  
\]
This allows us to define $[X,Y]_xf=X_x(Yf)-Y_x(Xf)$. If, in a local coordinate system, $X=X^i\partial_i$ and $Y=Y^j\partial_j$, we have
\[ 
\begin{split}
(XY)f&=X^i\partial_i(Y^j\partial_jf)\\
	&=X^i(\partial_iY^j)\partial_jf+X^iY^j\partial^2_{ij}f.
\end{split}  
\]
The same computation with $X\leftrightarrow Y$ shows that the second derivative disappears in the commutator $[X,Y]$. So $[X,Y]$ is a combination of the $e_i$'s and therefore is a vector fields.

The set of vector fields endowed with the bracket $[.,.]$ is an algebra. It is even a \emph{Lie} algebra because the bracket 
\begin{enumerate}
\item is bilinear,
\item is skew-symmetric: $[X,Y]=-[Y,X]$,
\item fulfills the Jacobi identity: $[ [X,Y],Z]+[ [Y,Z],Z]+[ [Z,X],Y]=0$,
\end{enumerate}
for all $X$, $Y$, $Z\in\cvec(M)$.

%%%%%%%%%%%%%%%%%%%%%%%%%%
%
   \section{Exercises}
%
%%%%%%%%%%%%%%%%%%%%%%%%

\Exo{002}

\Exo{001}

\Exo{004}

\Exo{017}

\Exo{003}

\Exo{005}

\Exo{007}

\Exo{006}

\Exo{009}

\Exo{010}

\Exo{008}

\Exo{011}

\Exo{012}

\Exo{013}

\Exo{014}

\Exo{015}

\Exo{016}
