Elements of K-theory are taken from \cite{VB_and_K,Landi}.

%%%%%%%%%%%%%%%%%%%%%%%%%%
%
   \section{The group \texorpdfstring{$K_0$}{K0}}
%
%%%%%%%%%%%%%%%%%%%%%%%%

Let $\cA$ be an unital $C^*$-algebra and $ M_N(\cA)$, the $C^*$-algebra of $N\times N$ matrices with coefficients in $\cA$. We have
\begin{equation}
 M_N(\cA)\simeq \cA\otimes_{\eC} M_N(\eC).
\end{equation}
The isomorphism being given by
\[ 
  a\mapsto \sum_{ij}a_{ij}\otimes_{\eC}E_{ij}
\]
if $a_{ij}\in\cA$ are the coefficients of $a\in M_N(\cA)$. Two projectors $p$, $q\in M_N(\cA)$ are \defe{equivalent in the sense of Murray-von Neumann}{equivalent!Murray-von Neumann} if there exists a matrix $u\in M_N(\cA)$ such that $p=u^*u$ and $q=uu^*$. In that case, the matrix $u$ is a partial isometry by lemma \ref{LemPartIsomCstar}. Notice that this notion of equivalence was already developed on page \pageref{PgEaivVNMurray} when speaking of projectors in von~Neumann algebras.

We consider 
\[ 
  M_{\infty}(\cA)=\bigcup_{N=1}^{\infty}M_N(\cA),
\]
with the inclusion map
\begin{equation}
\begin{aligned}
 \phi\colon M_N(\cA)&\to M_{N+1}(\cA) \\ 
a&\mapsto \phi(a)=
\begin{pmatrix}
a&0\\
0&0
\end{pmatrix}. 
\end{aligned}
\end{equation}
We consider the equivalence $p\sim q$ in $M_{\infty}(\cA)$ if and only if there exists $u\in M_{\infty}(\cA)$ such that $p=u^*u$ and $q=uu^*$. The class of $q$ is denoted by $[q]$ and the set of classes is denoted by $V(\cA)$\nomenclature{$V(\cA)$}{A class of projectors defined in order to introduce K-theory.}. The set $V(\cA)$ is a semigroup for the addition
\[ 
  [p]+[q]=
\left[
\begin{pmatrix}
p&0\\
0&q
\end{pmatrix}
\right].
\]
Some elements are not invertible.

\begin{probleme}
Pour bien faire, il faut encore prouver que cette somme est bien définie.
\end{probleme}


\begin{proposition}
The semigroup $V(\cA)$ is abelian. 
\end{proposition}

\begin{proof}
If $p=uu^*$ and $q=vv^*$ and $a=\begin{pmatrix}
u\\&v
\end{pmatrix}$, we have
\[ 
  [p]+[q]=\left[  \begin{pmatrix}
uu^*\\&vv^*
\end{pmatrix} \right]
=
\left[  
\begin{pmatrix}
u\\&v
\end{pmatrix}
\begin{pmatrix}
u^*\\&v^*
\end{pmatrix}
 \right]
=[aa^*]=[a^*a],
\]
but
\[ 
  a^*a=\begin{pmatrix}
&v^*\\u^*
\end{pmatrix}
\begin{pmatrix}
&u\\v
\end{pmatrix}
=
\begin{pmatrix}
v^*v\\&u^*u
\end{pmatrix}
=
\begin{pmatrix}
q\\&p
\end{pmatrix}.
\]

\end{proof}

We pose, on $V(\cA)\times V(\cA)$, the equivalence relation
\[ 
  \big( [p],[q] \big)\sim\big( [p'],[q'] \big)
\]
if and only if there exists $[r]\in V(\cA)$ such that
\begin{equation}		\label{EqConfEquivpqrK}
[p]+[q']+[r]=[p']+[q]+[r].
\end{equation}
Now the group $K_0(\cA)$ is defined as
\[ 
  K_0(\cA)=\frac{\big( V(\cA)\times V(\cA) \big)}{\sim}.
\]
That construction is not a particularity of K-theory. When $S$ is a semigroup, the group $K(S)=S\times S/\sim$ where $(s,t)\sim (u,v)$ if and only if there exists a $r\in S$ such that $s+v+r=t+u+r$ is the \defe{Grothendiek group}{grothendiek group} of the semigroup $S$. The group $K(S)$ is always abelian. For example, $K(\eN,+)=\eZ$ and $K(\eZ,\cdot)=\eQ$.


An addition in $K_0(\cA)$ is defined by 
\[ 
  \Big[ [p],[q] \Big]+\Big[ [p'],[q'] \Big]=\Big[ [p]+[p'],[q]+[q'] \Big].
\]
From definition we have $\Big( [p],[p] \Big)\sim\Big( [q],[q] \Big)$ and $\Big[ [p],[p] \Big]$ is the neutral for addition in $K_0(\cA)$. The invert is given by
\[ 
  -\Big[ [p],[q] \Big]=\Big[ [q],[p] \Big].
\]
We have a natural homomorphism
\begin{equation}
\begin{aligned}
 \kappa_{\cA}\colon V(\cA)&\to K_0(\cA) \\ 
[p]&\mapsto  \Big[ [p],[0] \Big]. 
\end{aligned}
\end{equation}
We say that $V(\cA)$ has the \defe{simplification property}{simplification property} if 
\[ 
  [p]+[r]=[q]+[r]\,\Rightarrow\,[p]=[q].
\]

\begin{proposition}
This map is injective if and only if $V(\cA)$ has the simplification property.
\end{proposition}

\begin{proof}
First, we suppose that $\kappa_{\cA}$ is injective and we suppose the equality $[p]+[r]=[q]+[r]$ for a certain $[r]\in V(\cA)$. This implies $\Big[ [p],[0] \Big]=\Big[ [q],[0] \Big]$ and then injectivity of $\kappa_{\cA}$ gives $[p]=[q]$ and the simplification property is proved.

Now we suppose that $\cA$ is such that $V(\cA)$ fulfils the simplification property. Let $[p]$ and $[q]$ such that $\kappa_{\cA}([p])=\kappa_{\cA}([q])$, or $\Big[ [p],[0] \Big]=\Big[ [q],[0] \Big]$ which means that $\Big( [p],[0] \Big)\sim\Big( [q],[0] \Big)$. From the definition of the classes, it provides the existence of a $[r]\in V(\cA)$ such that $[p]+0+[r]=[q]+[r]$ which in turn proves that $[p]=[q]$ by the simplification rule.
\end{proof}

Notice that every element in $K_0(\cA)$ can be written under the form of a difference
\[ 
  \kappa_{\cA}([p])-\kappa_{\cA}([g]),
\]
namely,
\[ 
\begin{split}
\Big[ [p],[q] \Big]=\Big[ [p],[0] \Big]+\Big[ [0],[q] \Big]=\kappa_{\cA}[p]-\Big[ [q],[0] \Big]
	=\kappa_{\cA}[p]-\kappa_{\cA}[q].
\end{split}  
\]

%----------------------------------------------------------------------------------------------------------------------------
\subsection{Example}

In the case $\cA=\eC$, the space $V(\eC)$ is determined by the simple following result.
\begin{proposition}
Two projectors of $p,q\in M_N(\eC)$ are equivalent if and only if the dimensions of their image are equal.
\end{proposition}

\begin{proof}
From what was said about projectors on page \pageref{PgProjPositif} , there exists a $u\in M_N(\eC)$ such that $p=uu^*$. Moreover $p$ and $q$ are both diagonalisable, and from the dimension assumption, their diagonal form are the same. In particular, $\tr(p)=\tr(q)$ is their common dimension of image space. 

If $p=uu^*$, then $p$ can be written as $u_Au_A^*$ with $u_A=uA$ for every matrix $A$ such that $AA^*=\mtu$. We are going to find such a matrix $A$ for which $q=u_A^*u_A$, i.e. $q=A^*(u^*u)A$. Notice that $u^*u$ and $A$ are diagonalisable; the proposition should be proved if their diagonal form are the same: in that case $A$ is the unitary matrix which diagonalises $u^*u$ in the basis in which $q$ is diagonal.

We have $(u^*u)^2=u^*uu^*u=u^*pu$, and using the fact that $p^2=p$, we find $(u^*u)^3=(u^*u)^2$, so that the eigenvalues of $u^*u$ are $0$ and $1$. From that and the assumptions, we deduce
\[ 
  \dim\big( \Image(u^*u) \big)=\tr(u^*u)=\tr(uu^*)=\tr(p)=\dim\big( \Image(p) \big)=\dim\big( \Image(q) \big),
\]
so that $\dim\big( \Image(q) \big)=\dim\big( \Image(u^*u)\big)$ and they have the same diagonal form.
\end{proof}

Since two projectors are equivalent if and only if the dimensions of their respective image are equal, the set of equivalence classes is $V(\eC)=\eN$. The Grothendieck group associated with the semigroup $\eN$ is $\eZ$, so that $K_0(\eC)=\eZ$.

%----------------------------------------------------------------------------------------------------------------------------
\subsection{Universality}

\begin{proposition}
Let $G$ be an abelian group and $\Phi\colon V(\cA)\to G$, a semigroup homomorphism such that $\Phi\big( V(\cA) \big)$ is invertible in $G$. Then $\Phi$ extends in an unique way to an homomorphism $\Psi\colon K_0(\cA)\to G$ such that $\Psi\circ\kappa_{\cA}=\Phi$.
\end{proposition}
In other words, one can invent the map $\Psi$ is such a way the following diagram commutes:
\[ 
\xymatrix{%
   K_0(\cA) \ar[d]_{\kappa_{\cA}}\ar@{.>}[rd]^{\Psi}		\\
   V(\cA) \ar[r]_{\Phi}	&	G.
}
\]

\begin{proof}
For unicity, consider $\Psi_1,\Psi_2\colon K_0(\cA)\to G$, two maps extending the map $\Phi$, so
\begin{equation}
\Psi_1\left( \big[ [p],[q] \big] \right)=\Psi_1\big( \kappa_{\cA}[p]-\kappa_{\cA}[q] \big)= \Phi[p]-\Phi[q]=\Psi_2\left( \big[ [p],[q] \big] \right),
\end{equation}
where we used the assumption $\Psi_1\circ\kappa_{\cA}=\Phi$.

For existence, we define $\Psi\colon K_0(\cA)\to G$ by the formula
\begin{equation}
	\Psi\left( \big[ [q],[p] \big] \right)=\Phi[q]-\Phi[p].
\end{equation}
Let us prove that it is well defined. Suppose that $\big( [p],[q] \big)\sim \big( [p'],[q'] \big)$, and consider a $[r]$ such that condition \eqref{EqConfEquivpqrK} holds. Since $\Phi\big( V(\cA) \big)$ is invertible in $G$, it makes sense to compute
\begin{equation}
\begin{split}
\Psi\left( \big[ [p'],[q'] \big] \right)-\Psi\left( \big[ [p],[q] \big] \right)&=\Phi[p']-\Phi[q']-\Phi[p]+\Phi[q]\\
			&=-\big( \Phi[p']+\Phi[q]+\Phi[r] \big)+\Phi[p']+\Phi[q]+\Phi[r]\\
			&=\Phi[0]\\
			&=0
\end{split}
\end{equation}
where we used the fact that $G$ is abelian in order to rearrange terms and the fact that $\Phi$ is an homomorphism. We have $\Psi\circ\kappa_{\cA}=\Phi$ because
\[ 
  \Psi\big( \kappa_{\cA}[p] \big)=\Psi\left( \big[ [p],[0] \big] \right)=\Phi[p].
\]
\end{proof}

\begin{proposition}
If $\alpha\colon \cA\to \cB$ is an homomorphism of $C^*$-algebra, then the map
\begin{equation}
\begin{aligned}
 \alpha_*\colon V(\cA)&\to V(\cB) \\ 
   [a]&\mapsto [\alpha(a)] 
\end{aligned}
\end{equation}
is a well defined group homomorphism which extends to a group homomorphism $\alpha_*\colon K_0(\cA)\to K_0(\cB)$.
\end{proposition}

\begin{proof}
Let consider a projector $(a_{ij})\in M_{\infty}(\cA)$. Since $\alpha$ is an homomorphism, the matrix $\alpha(a_{ij})$ is a projector of $M_{\infty}(\cB)$. Since $\alpha$ preserves the involution (by definition of a $C^*$-algebra homomorphism), we have that $a\sim b$ implies $\alpha(a)\sim\alpha(b)$. Thus the induced map $\alpha_*\colon V(\cA)\to V(\cB)$ is an homomorphism.

In order to get the extension, we use the universality property. We know that $\alpha$ is a semigroup homomorphism and $\Phi=\kappa_{\cB}\circ\alpha_*$ is a semigroup homomorphism from the semigroup $V(\cA)$ and the group $K_0(\cB)$. Then universality provides $\Psi\colon K_0(\cA)\to K_0(\cB)$.
\end{proof}

\begin{proposition}
If $\cA$ is a $C^*$-algebra obtained as the inductive limit 
\[ 
  \cA=\lim_{\rightarrow}\{ \cA_i,\Phi_{ij} \}_{i,j\in\eN},
\]
then $\{ K_0(\cA_i),(\Phi_{ij})_* \}$ is an inductive system of groups and 
\begin{equation}
K_0(\cA)=K_0\big( \lim_{\rightarrow}\cA_i \big)=\lim_{\rightarrow} K_0(\cA_i).
\end{equation}
\end{proposition}

%+++++++++++++++++++++++++++++++++++++++++++++++++++++++++++++++++++++++++++++++++++++++++++++++++++++++++++++++++++++++++++
					\section{Vector bundle and K-theory}
%+++++++++++++++++++++++++++++++++++++++++++++++++++++++++++++++++++++++++++++++++++++++++++++++++++++++++++++++++++++++++++

We follow \cite{VB_and_K} in which one finds the proofs that are omitted here. In this section, we only deal with complex vector bundles over compact Hausdorff base. We accept that, when the basis is non connected, the vector bundle has different dimensions on different components. 

We denote by $\epsilon^n\to X$ the trivial bundle of dimension $n$ over $X$ and we say that two vector bundles $E_1$ and $E_2$ are \defe{stably isomorphic}{stably isomorphic} if there exists a $n$ such that $E_1\oplus\epsilon^n\simeq E_2\oplus\epsilon^n$. We write it by $E_1\simeq_s E_2$\nomenclature{$E_1\simeq_s E_1$}{Vector bundles $E_1$ and $E_2$ are stably isomorphic}.

We also define the relation $E_1\sim E_2$\nomenclature{$E_1\sim E_2$}{Vector bundle $E_1$ and $E_2$ are equivalent} if and only if $E_1\oplus\epsilon^n\simeq E_2\oplus\epsilon^m$ for some $n$ and $m$. One can prove that $\simeq_s$ and $\sim$ are equivalence relations.

\begin{proposition}
If $X$ is compact and Hausdorff, the set of equivalence classes for $\sim$ is an abelian group for the direct sum.
\end{proposition}

\begin{proof}
No proof.
\end{proof}
That group is denoted by $\tilde K(X)$\nomenclature{$\tilde K(X)$}.  

The set of equivalence classes for $\simeq_s$ cannot be a group because the inverse does not exists. Indeed if $E\oplus E'\simeq_s\epsilon^0$, then we have a $n$ such that $E\oplus E'\oplus\epsilon^n=\epsilon^n$, which implies that $E$ and $E'$ are zero dimensional.

\begin{proposition}
The set of equivalence classes with respect to $\simeq_s$ has the simplification property
\begin{equation}
F\oplus E_1\simeq_s F\oplus E_2\,\Rightarrow E_1\simeq_s E_2
\end{equation}
on every compact of $X$.
\end{proposition}

\begin{proof}
Using proposition \ref{PropEoplusEprimetriv}, we can add on both sides of $F\oplus E_1\simeq_s F\oplus E_2$ a vector bundle $F'$ such that $F\oplus F'=\epsilon^n$. Thus we get $E_1\oplus\epsilon^n\simeq_s E_2\oplus\epsilon^n$, which means that $E_1\simeq_s E_2$.
\end{proof}

We know that the positive rational numbers $\eQ^+$ are build from the integers by taking the pairs $(a,b)\in\eN^2$ and the quotient by
\[ 
  (a,b)\sim(c,d)\,\Leftrightarrow\, ad=bc.
\]
We perform the same construction with vectors bundle and we define $K(X)=\{ (E,E') \}/\sim$\nomenclature{$K(X)$}{} where
\begin{equation}
(E_1,E_1')\sim (E_2,E_2')\,\Leftrightarrow\, E_1\oplus E_2'\simeq_s E_2\oplus E_1'.
\end{equation}
Let us prove that that relation is transitive. For this, we suppose $(E_1,E'_1)\sim (E_2,E'_2)$, and $(E_2,E'_2)\sim(E_/,E'_3)$, i.e. there exists integers $n$ and $q$ such that
\[ 
E_1\oplus E'_2\oplus\epsilon^n\oplus E'_3\oplus\epsilon^q	=E_2\oplus E'_1\oplus\epsilon^n\oplus E'_3\oplus\epsilon^q.
\]
Now we use the relation $E_2\oplus E'_3\oplus\epsilon^q=E_3\oplus E'_2\oplus\epsilon^q$ in the right hand side and we use the simplification property by $E'_2$, we get
\[ 
  E_1\oplus E'_3\oplus\epsilon^{n+q}=E_3\oplus E'_1\oplus\epsilon^{n+q},
\]
which means that $(E_1,E_1')\sim (E_3,E'_3)$. Notice that the capacity assumption was used when we made the simplification by $E_2$. The simplification would have added a $\epsilon^m$ which was implicitly absorbed in the $n$ or the $q$.

In that context, the class of the pair $(E,E')$ is often denoted by $E-E'$. The set $K(X)$ becomes a group for the addition
\begin{equation}
(E_1-E_1')+(E_2-E_2')=(E_1\oplus E_2)-(E'_1\oplus E'_2).
\end{equation}
The zero of the group $K(X)$ is $E-E$ for any $E$ and the inverse of $E-E'$ is $E'-E$. Since $E\oplus(E'\oplus F)=E'\oplus(E\oplus F)$, we have
\begin{equation}
E-E'=(E\oplus F)-(E'\oplus F).
\end{equation}
Taking an inverse of $E'$ as $F$, we find that $E-E'=E\oplus F-\epsilon^n$, so that every element of $K(X)$ can be written as a difference 
\[ 
E-\epsilon^n. 
\]

\begin{proposition}		\label{PropvphomKtK}
The map
\begin{equation}
\begin{aligned}
 \varphi\colon K(X)&\to \tilde K(X) \\ 
   E-\epsilon^n&\mapsto [E]_{\sim} 
\end{aligned}
\end{equation}
is a surjective homomorphism.
\end{proposition}

\begin{proof}
We prove first that it is well defined. Suppose that $E-\epsilon^n=E'-\epsilon^m$, so $E\oplus \epsilon^m=E'\oplus \epsilon^m$, so that $E'\sim E$. Surjectivity is clear.
\end{proof}

The kernel of that map of made from the differences $E-\epsilon^n$ with $E\sim \epsilon^0$, or in other words, $E\simeq_s\epsilon^m$ for a certain $m$,
\begin{equation}
\ker(\varphi)=\{ \epsilon^m-\epsilon^n \}\subset K(X).
\end{equation}
This is isomorphic to $\eZ$.

%---------------------------------------------------------------------------------------------------------------------------
					\subsection{Reduced group}
%---------------------------------------------------------------------------------------------------------------------------

Let us fix a base point $x_0\in X$. We have a natural homomorphism $K(X)\to K(x_0)=\eZ$. By the proposition \ref{PropvphomKtK}, this homomorphism restricts to an isomorphism $\{ \epsilon^m-\epsilon^n \}\to \eZ$. Thus we have a decomposition
\begin{equation}
		K(X)=\tilde K(X)\oplus\eZ
\end{equation}
which depends on the choice of the base point $x_0$. For this reason, we says that $\tilde K(X)$ is \defe{reduced}{reduced K group} of $K(X)$.

Since $K(X)$ is an additive group, we can turn it into a ring by the following multiplication:
\begin{equation}
(E_1-E_1')(E_2-E_2')=E_1\otimes E_2 -E_1\otimes E'_2-E'_1\otimes E_2+E_1'\otimes E'_2.
\end{equation}
One can check that $K(X)$ becomes a ring with $\epsilon^1$ as identity.

%---------------------------------------------------------------------------------------------------------------------------
					\subsection{Functorial description}
%---------------------------------------------------------------------------------------------------------------------------

If $f\colon X\to Y$ is a map between two compact Hausdorff topological spaces, we have an induces map
\begin{equation}
\begin{aligned}
 f^*\colon K(Y)&\to K(X) \\ 
   E-E'&\mapsto f^*(E)-f^*(E') 
\end{aligned}
\end{equation}
which satisfies
\begin{enumerate}
\item $f^*(E_1\oplus E_2)\simeq f^*(E_1)\oplus f^*(E_2)$,
\item $f^*(E_1\otimes E_2)=f^*(E_1)\otimes f^*(E_2)$,
\end{enumerate}
and is then a ring homomorphism. We are then lead to see $K$ as a functor between 
\begin{itemize}
\item the category of compact Hausdorff topological spaces with continuous maps as arrows, $\catC$, and
\item the category of rings, $\catD$.
\end{itemize}
The functor $K$ makes the correspondence $X\to K(X)$ for the objects and for the arrows,
\begin{equation}
\begin{aligned}
 Kf\colon K(X)&\to K(Y) \\ 
   KF&\mapsto (f^{-1})^*. 
\end{aligned}
\end{equation}
Since $K(\id_X)=\id_{K(X)}$ and $K(g\circ f)=Kg\circ Kf$ for every $f\colon X\to Y$ and $g\colon Y\to Z$, the operation $K$ is a functor.

%---------------------------------------------------------------------------------------------------------------------------
					\subsection{External product}
%---------------------------------------------------------------------------------------------------------------------------

One has the external product 
\begin{equation}
\begin{aligned}
 \mu\colon K(X)\otimes K(Y)&\to K(X\times Y) \\ 
   \mu(a\otimes b)&= \pr_X^*(a)\pr^*_Y(b) 
\end{aligned}
\end{equation}
where $\pr_X$ and $\pr_Y$ are the projections of $X\times Y$ onto $X$ and $Y$. So we have $\pr_X^*\colon K(X)\to K(X\times Y)$ and the product $\pr_X^*(a)\pr^*_Y(b) $ is the internal product in $K(X\times Y)$. It is a general fact that tensor product of rings is a ring with the rule
\begin{equation}
	(a\otimes b)(c\otimes d)=ac\otimes bd.
\end{equation}
In our case, $K(X)\otimes K(Y)$ is a ring. We are going to prove that $\mu$ is a ring homomorphism. Let us denote by $f^*(E)$ the function given by proposition \ref{PropmapfEEsun}. We have
\[ 
  \mu\big( (a\otimes b)(c\otimes d) \big)=\mu(ac\otimes bd)=\pr^*_X(ac)\pr^*_Y(bd).
\]
Now suppose that $a=E_a-E'_a$ and $c=E_c-E'_c$. We have $ab=E_a\otimes E_c-E_a\otimes E'_c=E'_a\otimes E_c+E'_a\otimes E'_c$, and using the properties \eqref{EqPropfstarEVect}, we find
\begin{align*}
 \mu\big( (a\otimes b)(c\otimes d) \big)	&=\pr^*_X(a)\pr^*_X(c)\pr^*_Y(b)\pr^*_Y(d)\\
						&=\pr^*_X(a)\pr^*_Y(b)\pr^*_X(c)\pr^*_Y(d)\\
						&=\mu(a\otimes b)\mu(c\otimes d),
\end{align*}
and $\mu$ is an homomorphism. We have in particular an homomorphism
\begin{equation}		\label{EamuhomSdeuxX}
  \mu\colon K(X)\otimes K(S^2)\to K(X\otimes S^2).
\end{equation}
The main content of \defe{Bott periodicity}{Bott periodicity} is to prove that \eqref{EamuhomSdeuxX} is in fact an isomorphism.


%---------------------------------------------------------------------------------------------------------------------------
					\subsection{Clutching function}
%---------------------------------------------------------------------------------------------------------------------------

Let $p\colon E\to X$ be a vector bundle and $f\colon E\times S^1\to E\times S^1$ be an homomorphism where $E\times S^1$ denotes the bundle $(p\times\id)\colon E\times S^1\to X\times S^1$. For each $x\in X$ and $z\in S^1$, the map $f$ produces an isomorphism $f(x,z)\colon p^{-1}(x)\to p^{-1}(x)$. Notice that, the $S^1$ part being trivial, we immediately restrict to $X$. If not, we would have written $(p\times \id)^{-1}(x,z)$.

If we identify the boundary of the two copies of $D^2$ in $D^2\cup D^2$ (i.e. we identify the two copies of $S^1$), what we obtain is the sphere $S^2$. So we take two copies of $E\times D^2$ and we identify the boundaries $S^1$ with $f$. What we get is a vector bundle over $X\times S^1$ that we name $[e,f]$\nomenclature{$[E,f]$}{vector bundle over $X\times S^1$}. The function $f$ is the \defe{clutching}{clutching function} for $[E,f]$.

Let $f_t\colon E\times S^1\to E\times S^1$, be an homotopy of clutching functions. It allows us to build a vector bundle on $X\times S^2\times I$ ($I$ is the interval $[0,1]$ in which $t$ varies) which restricts to $[E,f_0]$ and $[E,f_1]$ on $X\times S^2\times \{ 0 \}$ and $X\times S^2\times \{ 1 \}$. So we have $[E,f_0]\simeq[E,f_1]$.

