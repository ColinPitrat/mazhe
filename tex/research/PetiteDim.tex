\section[Low dimensional anti de Sitter spaces]{Group realization of low dimensional anti de Sitter spaces}
%++++++++++++++++++++++++++++++++++++++++++++++++++++++++++++++++++

\subsection{Two dimensional anti de Sitter}		\label{SubsecTwoDimAdSAdGH}
%------------------------------------------

Using notations and conventions of section~\ref{SecToolSL}, the set $\Ad(G)H$ is a subset of $\gsl(2,\eR)$ made of elements of norm $8$. One can show by brute force computation or using the commutation relations that
\[
\begin{split}
 \Ad( e^{x_KT} e^{x_NE})H&=\big( -\sin(2x_K)x_N+\cos(2x_K) \big)H\\
			&\quad+\big( -\cos(2x_K)x_N-\sin(2x_K) \big)(E+F)\\
			&\quad-x_NT,
\end{split}
\]
which is a, following the Killing form \eqref{EqBHEFTsldR},  general element of norm $8$ in $\gsl(2,\eR)$. Thus as sets, $AdS_2$ is $\Ad(KN)H$.  Since $A$ is the stabilizer of $H$ for the adjoint action of $G$ on $H$ and $G=ANK=KNA$, we also have
\[
  AdS_2=G/A=\Ad(KN)H=\Ad(G)H.
\]
That provides isomorphisms
\begin{equation}
\begin{aligned}
 \phi\colon [0,\pi[\times\eR&\to AdS_2 \\
(x_K,x_B)&\mapsto \Ad( e^{xK T} e^{xNE})H,
\end{aligned}
 \end{equation}
or
\begin{equation}	 \label{EqCylAdSDeux}
\begin{aligned}
 \phi\colon Cyl&\to AdS_2 \\
(\theta,h)&\mapsto \Ad( e^{\frac{ \theta }{ 2 }T} e^{hE})H.
\end{aligned}
\end{equation}
where $Cyl$ is the usual cylinder in $\eR^3$.

\subsection{Three dimensional anti de Sitter}		\label{SubsecGpAdsDeux}
%--------------------------------------------

The space $AdS_3$ is the hyperboloid
\begin{equation} \label{hyperboloide}
 u^2 + t^2 - x^2 - y^2 = 1
 \end{equation}
embedded in $\eR^{2,2}$. There exists a bijection between $\eR^{2,2}$ and the two by two real matrices given by
\[
 v= \begin{pmatrix}
u\\t\\x\\y
\end{pmatrix}
\mapsto
g(v)=\begin{pmatrix}
u+x&y+t\\
y-t&u-x
\end{pmatrix}.
\]
As far as norm is concerned, we have $\| v \|=\det g(v)$. Among these matrices, the ones of $\SL(2,\eR)$ are given by the condition $\det g=1$, which is precisely the equation of the hyperboloid in $\eR^{4}$. That shows that $AdS_3=\SL(2,\eR)$.

From \eqref{hyperboloide}, the isometry group of $AdS_3$, denoted by $\Iso(G)$, is the group $O(2,2)$. It is locally isomorphic to $G \times G$, through the action
\begin{equation}
 (G \times G) \times G \longrightarrow G \colon((g_L,g_R),z)
 \rightarrow g_l \, z g_R^{-1},
 \end{equation}
 which corresponds to the identity component of $Iso(G)$ (from the bi-invariance of the Killing metric), and because of the Lie algebra isomorphism
\begin{equation}\label{Iso}
 \Phi: \mG \times \mG \rightarrow iso(G) : (X,Y)
 \rightarrow  \overline{X} - \underline{Y},
\end{equation}
where $\overline{X}$ (resp. $\underline{Y}$) denotes the right invariant (resp. left invariant) vector field on $G$ associated with the element $X$ (resp. $Y$) of its Lie algebra.

\section{Simple example on \texorpdfstring{$AdS_{2}$}{AdS2}}\label{sec_AdSdeux}
%----------------------------------------------------------------

As is subsection~\ref{SubsecTwoDimAdSAdGH}, we see $AdS_2$ as $\Ad(G)H$. By definition the singularity is composed of closed orbits of $AN$ and $A\bar{N}$ for the adjoint action on $AdS_2$, and the notion of fundamental field is
\begin{equation}
H^*_x=\Dsdd{\Ad(e^{-tH})x}{t}{0} =-[H,x].
\end{equation}

A basis of the Lie algebra $\sA\oplus\sN$ is given by $\{E,H\}$. So $x$ will belong to a closed orbit if and only if $E_x^*\wedge H^*_x=0$. If we put $x=x_HH+x_EE+x_FF$, the computation is
\[
E_x^*\wedge H^*_x=[E,x]\wedge[H,x]
                 =4x_Hx_F (E\wedge F)+2x_Ex_F (H\wedge E)-2x_F^2 (H\wedge F).
\]
It is zero if and only if $x_F=0$. The closed orbit of $A\bar{N}$ is given by the same computation with $H^*_x\wedge F^*_x$. The part of these orbits contained in $AdS_2$ is the one with norm $8$:
\[
B(x,x)=8(x_H^2+x_Ex_F)\stackrel{!}{=}8.
\]
In both cases, it imposes $x_H=\pm 1$, and the closed orbits in $AdS_2$ are given by
\begin{subequations}\label{EqQuatreLInesDeux}
\begin{align}
\hS_{AN}&=\pm H+\lambda E\\
\hS_{A\bar N}&=\pm H+\lambda F,
\end{align}
\end{subequations}
with $\lambda\in\eR$. The singularity is then the union $\hS_{AN}\cup\hS_{A\bar N}$ of four lines in the hyperboloid.

\begin{proposition} \label{PropAdSDeuxJannule}
     The singularity of $AdS_2$ can equivalently be described as
    \begin{equation}\label{condHH}
        \hS=\{x \in \Ad(G)H \tq \|H^*_x\|=0\},
    \end{equation}
\end{proposition}
This result is also proved for \( AdS_3\) in corollary~\ref{CorJannsingul}.

\begin{proof}
The condition \eqref{condHH} on $x$ reads
\begin{equation}\label{eq:BHxHx}
B([H,x],[H,x])=0.
\end{equation}
The most general\footnote{It is actually \emph{more} than the most general element to be considered because our space is $\Ad(G)H$, which is only a part of $\sldr$.} element $x$ in $\sldr$ is $x=x_AH+x_NE+x_FF$. We have $[x,H]=-2x_NE+2x_FF$, so that the condition \eqref{eq:BHxHx} becomes $x_Nx_F=0$. The two possibilities are $x=x_AH+x_NE$ and $x=x_AH+x_FF$. The singularity in $\gsl(2,\eR)$ is composed of the planes $(H,F)$ and $(H,E)$. The intersection between the plane $(H,F)$ and the hyperboloid is given by the equation
\[
B(aH+bF,aH+bF)=8
\]
whose solutions are $a=\pm 1$. The same is also true for the plane $(H,E)$. So we find that the set \eqref{condHH} is exactly the four lines  \eqref{EqQuatreLInesDeux}.

\end{proof}

One can check that light cone of a given point of the hyperboloid is given by the two straight lines trough the point; so it automatically intersects the singularity. As conclusion, every point of $AdS_2$ belong to the black hole. For this reason we say that there is no black hole in the two dimensional case because the black hole should be the whole space while one ask the singularity, the black hole and the complete space to be different. See also the condition \eqref{EqhSssubBH}.

%+++++++++++++++++++++++++++++++++++++++++++++++++++++++++++++++++++++++++++++++++++++++++++++++++++++++++++++++++++++++++++
\section{Still \texorpdfstring{$AdS_2$}{AdS2}}
%+++++++++++++++++++++++++++++++++++++++++++++++++++++++++++++++++++++++++++++++++++++++++++++++++++++++++++++++++++++++++++

If $h$ is the generator of $\SO(1,1)$ in the quotient $AdS_2=\SO(2,1)/\SO(1,1)$, then we have
\begin{equation}
	AdS_2=\frac{ \SO(2,1) }{ \SO(1,1) }=\Ad(G)h
\end{equation}
by the isomorphism
\begin{equation}
	\psi[g]=\Ad(g)h.
\end{equation}
The multiplication table of the algebra $\so(2,1)$ is
\begin{equation}
	\begin{aligned}[]
		[q_0,q_1]&=h\\
		[h,q_0]&=q_1\\
		[h,q_1]&=q_0.
	\end{aligned}
\end{equation}

Knowing the adjoint map, we can compute the Killing for. In particular we have, in the basis $q_0$, $q_1$, $h$,
\begin{equation}
	B=\begin{pmatrix}
		  -2	&	0	&	0\\
		    0	&	2	&	0\\
		     0	&	0	& 	2
	     \end{pmatrix}.
\end{equation}
Thus the norm on $\so(2,1)$ is
\begin{equation}
	\| aq_0+bq_1+ch\| =2(-a^2+b^2+c^2),
\end{equation}
and
\begin{equation}
	AdS_2=\{ aq_0+bq_1+ch\tq -a^2+b^2+c^2=2 \}.
\end{equation}

If $x=\Ad(g)h$, we consider the adjoint action of $G$ over $x$ and the fundamental vector fields
\begin{equation}
	X^*_x=\Dsdd{  e^{-tX}\cdot\Ad(g)h }{t}{0}=\Dsdd{  e^{-\ad(tX)}\Ad(g)h }{t}{0}=-\ad(X)x,
\end{equation}
so we have
\begin{equation}
	X^*_{\Ad(g)h}=-\ad(X)\Ad(g)h.
\end{equation}
For the metric, the first choice is to set
\begin{equation}
	B(X^*_x,Y^*_x)=B(\Ad_{g^{-1}}X,\Ad_{g^{-1}}Y),
\end{equation}
but that definition would not take the quotient into account. Indeed, we have $B(h,h)\neq 0$ while
\begin{equation}
	h^*_h=-\ad(h)h=0.
\end{equation}
Thus we consider the metric
\begin{equation}
	B(X^*_x,Y^*_y)=B\big( \pr_{\sQ}(\Ad_{g^{-1}}X ),\pr_{\sQ}(\Ad_{g^{-1}}Y ), \big).
\end{equation}
We have in particular
\begin{equation}
	\| X_h^* \|=\| (aq_0+bq_1+ch)_h^* \|=(aq_0+bq_1)\cdot(aq_0+bq_1)=-a^2+b^2.
\end{equation}
So the time-like direction at $h$ is $q_0^*$. We can ask ourself in which direction it goes:
\begin{equation}
	(q_0)^*_h=\Dsdd{ \Ad( e^{-tq_0})h }{t}{0}=q_1.
\end{equation}
That provides the orientation of time at the point $\Ad(e)h$. Now we would like to know the orientation of time on other points, like at the point
\begin{equation}
	\Ad( e^{bq_1})h=\cosh(b)h-\sinh(b)q_0.
\end{equation}
Since $\Ad( e^{-bq_1})q_1=q_1$, we have
\begin{equation}
	\| q_1^* \|_{\Ad( e^{bq_1})h}=\| q_1 \|=1,
\end{equation}
the vector $q_1^*$ is not time-like at the point $\Ad( e^{bq_1})h$. On the other hand, we have
\begin{equation}
	\Ad( e^{-bq_1})q_0=\cosh(b)q_0+\sinh(b)h,
\end{equation}
so
\begin{equation}
	\| q_0^* \|_{\Ad( e^{bq_1})h}=\| \pr_{\sQ}\big( \cosh(b)q_0+\sinh(b)h \big) \|=-2\cosh(b),
\end{equation}
which is time-like.

\section{Causal structure of \texorpdfstring{$AdS_2$}{AdS2}}  \label{SecAdS2}
%++++++++++++++++++++++++++++++++++++

 As far as I know, results about $AdS_2$ comes from discussions with Pierre Bieliavsky. The approach that I followed for $AdS_3$ is the one of \cite{BTZB_un,BTZB_deux} although results are already know from \cite{BTZ_un,BTZ_deux}.

The two dimensional case does not present black hole structure, but the geometric setting is very clear as it is no more that intersections of lines in an hyperboloid.

We consider the following situation: $G=SL(2,\eR)$, $H=\SO(1,1)=A$ (the $A$ of Iwasawa for $G$), $AdS_2\simeq G/A$. The generators of $SL(2,\eR)$ are given as usual by $E,F,H$ and the generators of the groups $A$, $N$, $K$ are $H$, $E$, and $E-F$. We set
\[
\mS=R\cdot A\cup \ovR\cdot A=\pi(N)\cup\pi(\ovN).
\]
where the bar denotes the conjugation by the Cartan involution $\theta$. The light cone of $\pi(g)$ in $AdS_2$ is given by
\begin{equation}
C^+_{\pi(g)}=\{ \pi(ge^{tE})\cup\pi(ge^{tF}) \}_{t\in\eR^+}
\end{equation}
The study of the causal structure on this space is the following question: which are the $\pi(g)\in AdS_2$ such that $C^+_{\pi(g)}\cap\mS\neq \emptyset$? One can write $g=kna$ and forget the $a$-part because all the expressions appears in a $\pi$.

Let us consider the ray $\pi(ge^{tE})=\pi(kne^{tE})$. The aim is to see what choice of $k\in K$ and $n\in N$ makes $\pi(kne^{tE})\in\mS$ for a certain $t$. If we set
\[
n=\begin{pmatrix}
1 & n \\
0 & 1
\end{pmatrix}  \qquad k=\begin{pmatrix}
\cos k & \sin k \\
-\sin k & \cos k
\end{pmatrix},
\]
we find a general element of the ray (before the projections by $\pi$) under the form
\[
kne^{tE}=\begin{pmatrix}
\cos k & (t+n)\cos k+\sin k \\
-\sin k & -(t+n)\sin k+\cos k
\end{pmatrix}
\]
Taking the quotient into account, we have to see which are the $k\in[0,2\pi]$, $n\in\eR$ such that
\begin{equation}\label{eq:chem_tE}
\pi(kne^{tE})=\begin{pmatrix}
\cos k & (t+n)\cos k+\sin k \\
-\sin k & -(t+n)\sin k+\cos k
\end{pmatrix}
\begin{pmatrix}
e^a & 0 \\
0 & e^{-a}
\end{pmatrix}
\end{equation}
can be fitted on $\begin{pmatrix}
1 & l \\
0 & 1
\end{pmatrix}$ by a suitable choice of $t\in\eR^+$ and $a\in\eR$. The equations are
\begin{subequations}
\begin{align}
e^a\cos k&=1                                   \label{eq:sys_d_ai}\\
-e^a\sin k&=0                                  \label{eq:sys_d_aiii}   \\
e^{-a}\big(  (t+n)\cos k+\sin k  \big)&=l      \label{eq:sys_d_aii}\\
e^{-a}\big(  -(t+n)\sin k+\cos k  \big)&=1.    \label{eq:sys_d_aiv}
\end{align}
\end{subequations}
Equation \eqref{eq:sys_d_aiii} yields $\sin k=0$, so that \eqref{eq:sys_d_ai} imposes the choice $a=0$ and $\cos k=1$. This makes $k=\mtu$, so that $\pi(g)\in\mS$. Then as far as the singularity $A\cdot N$ is concerned, the direction $e^{tE}$ is safe.

Now we study the possibility to fall into $A\cdot \ovN$ by walking in the direction $e^{tE}$, so we have to fit equation \eqref{eq:chem_tE} with $\begin{pmatrix}
1 & 0 \\
f & 1
\end{pmatrix} $:
\begin{subequations}
\begin{align}
e^a\cos k&=1                                   \label{eq:sys_d_bi}\\
-e^a\sin k&=f                                  \label{eq:sys_d_biii}   \\
e^{-a}\big(  (t+n)\cos k+\sin k  \big)&=l      \label{eq:sys_d_bii}\\
e^{-a}\big(  -(t+n)\sin k+\cos k  \big)&=1.    \label{eq:sys_d_biv}
\end{align}
\end{subequations}
Condition \eqref{eq:sys_d_biv} makes $\cos k<0$ completely safe. Then we suppose $\cos k>0$ and $e^{a}=1/\cos k$. On the other hand, \eqref{eq:sys_d_biii} is always true because $f$ can take any real value. The two last equations give the system
\begin{equation}
\left\{
\begin{aligned}
(t+n)\cos k+\sin k&=0\\
-(t+n)\sin k+\cos k&=0
\end{aligned}
\right.
\end{equation}
The first gives $t=-\frac{\sin k+n\cos k}{\cos k}$. When substituted in the second equation, we find $\sin^2 k+\cos^2 k=0$.

Then the directions $e^{tE}$ is completely secure.

The second possible direction in the light cone is $e^{tF}=\begin{pmatrix}
1 & 0 \\
f & 1
\end{pmatrix}$. It is rather easy to find
\[
kne^{tF}=\begin{pmatrix}
(1+nt)\cos k+t\sin k & n\cos k+\sin k \\
-(1+nt)\sin k+t\cos k & -n\sin k+\cos k
\end{pmatrix}
\]
The equations which explain us how to to fall into $\begin{pmatrix}
1 & l \\
0 & 1
\end{pmatrix}$ are
\begin{subequations}
\begin{align}
t(\cos k-n\sin k)&=\sin k\\
e^a\big( (1+nt)\cos k+t\sin k  \big)&=1\\
e^{-a}\big( n\cos k+\sin k  \big)&=l\\
e^{-a}(-n\sin k+\cos k)&=1.
\end{align}
\end{subequations}
Substituting $(\cos k-n\sin k)$ from the first to the fourth equation, we find $e^a=\sin k/t$. The third equations is always true because $l$ can take any value (note that $\sin k$ must be nonzero). If we suppose $\cos k-n\sin k\neq 0$, we find that the direction $e^{tF}$ falls into $\begin{pmatrix}
1 & l \\
0 & 1
\end{pmatrix}$ after a time $t=\frac{\sin k}{\cos k-n\sin k}$. Examining the case of $\cos k-n\sin k=0$, we see that it is safe with respect to the hole $\begin{pmatrix}
1 & l \\
0 & 1
\end{pmatrix}$ in the direction $e^{tF}$.

The last point is to check what are the points which fall into $\begin{pmatrix}
1 & 0 \\
c & 1
\end{pmatrix}$ in the direction $e^{tF}$. The equations are
\begin{subequations}
\begin{align}
e^a\big(t\cos k-(1+nt)\sin k\big)&=c\\
e^a\big( (1+nt)\cos k+t\sin k  \big)&=1\\
e^{-a}\big( n\cos k+\sin k  \big)&=0\\
e^{-a}(-n\sin k+\cos k)&=1.
\end{align}
\end{subequations}
We immediately find that $n\cos k+\sin k\neq 0$ is safe and that $e^a=1/\cos k$. Substituting, we find $-n\sin k\cos k=1-\cos^2k$ and $\sin k\neq 0$ is obligatory. Then there are no hope to fall down in this part of the back hole.

As conclusion, the only way for a light ray to fall into the black hole is to begin at $\pi(kn)$ with $\cos k-n\sin k\neq 0$ and to follow the direction $e^{tF}$ during a time
\[
t=\dfrac{\sin k}{\cos k-n\sin k}.
\]

\subsection{Second point of view}
%--------------------------------

\begin{proposition}
Any point in the physical space can be written as $\Ad(ak)H$, with $k\in]0,\pi/2[$.
\label{prop:AdAK}
\end{proposition}

\begin{proof}
The physical space contains the curve  $\cos\beta H+\sin\beta(E+F)$ with $\beta\in]0,\pi[$, which is exactly\footnote{Remark that the whole closed curve with $\beta$ running from $0$ to $2\pi$ is double covered by $K$.} $\Ad(k)H$ for $k\in]0,\pi/2[$. It is also the intersection of $AdS_2$ and the part of $\sldr$ contained between the planes $(E,H)$ and $(F,H)$. If we use the coordinates $x,y,z$ on $\sldr$ (a point is given by $\overline{ r }=(x,y,z)=xH+yE+zF$), the physical space is given by the relations
\begin{numcases}{}
x^2+yz=1\\
y>0\\
z>0.
\end{numcases}
The first equation gives a $\beta$ such that $x=\cos^{2}\beta$, $yz=\sin^2\beta$. It is always possible to define a $a\in\eR$ such that $y=e^{2a}\sin\beta$ and $z=e^{-2a}\sin\beta$. Finally, the physical space is parameterized by
\begin{equation}
   (\beta,a)\mapsto\cos\beta H+\sin\beta(e^{2a}E+e^{-2a}F).
\end{equation}
    From commutation relations in $\sldr$, one finds
\begin{subequations}
\begin{align}
\Ad(e^{aH})E&=e^{2a}E,\\
\Ad(e^{aH})F&=e^{-2a}F,
\end{align}
\end{subequations} and then
\begin{equation}
\begin{split}
\Ad(ak)H&=\Ad(e^{aH})(\cos\beta H+\sin\beta(E+F) )\\
    &=\cos\beta H+\sin\beta(e^{2a}E+e^{-2a}F).
\end{split}
\end{equation}
\end{proof}

\subsection{Light cone}
%---------------------

The light-like vectors of $\sldr$ are $E$ and $F$, so the light cone of the point $\Ad(g)H$ has two parts:
\[
\Ad(g)\Ad(e^{tE})H\text{ and } \Ad(g)\Ad(e^{tF})H
\]
whose are better written under the more compact form
\begin{equation}
C^+_{\Ad(g)H}=\{\Ad(g)\Ad(e^{t\epsilon E})H\}_{%
\begin{subarray}{l}
t>0\\
\epsilon=\id,\theta
\end{subarray}
}
\end{equation}
where $\epsilon$ is the identity or the Cartan involution ($\theta(X)=-X^t$). Since $H^*_x=-[H,x]$, the equation of the cone which falls into the hole is
\begin{equation}
\|[H,\Ad(g)\Ad(e^{t\epsilon E})H]\|=0.
\end{equation}
It is somewhat easy to remark that for all $X,Y$ in a Lie algebra $\lG$ and all automorphism $\varphi$ the formula, $\varphi(\Ad(e^X)Y)=\Ad(e^{\varphi X})(\varphi Y)$ holds.
Then
\begin{equation}
\Ad(e^{t\epsilon E})H=s(\epsilon)\epsilon(\Ad(e^{tE})H)
\end{equation}
with
\[
  s(\epsilon)=
\begin{cases}
1&\text{if }\epsilon=\id,\\
-1&\text{if }\epsilon=\theta.
\end{cases}
\]
In order to see if the cone intersects the singularity, we have to solve --with respect to $t$-- the equation
\begin{equation}
B
\left(%
\big[  \Ad(g)s(\epsilon)\epsilon(\Ad(e^{tE})H),H\big],\ldots
\right)=0
\end{equation}
where the dots means that the same lies in the second entry of the Killing form. Using the invariance of the Killing form, the fact that $\epsilon^{-1}=\epsilon$ and
    equation \eqref{eq_AdetE} that $\Ad(e^{tE})H=H-2tE$, we rewrite the condition under the form
\begin{equation}
	B\big(	[H-2tE,s(\epsilon)\epsilon^{-1}\Ad(g^{-1})H],\ldots\big)=0.
\end{equation}
If we pose $g=a^{-1} n^{-1} k^{-1}$ and $x=s(\epsilon)\epsilon kn\cdot H=hH+lE+fF$, the condition becomes
\begin{equation}
\begin{split}
	B\big( [H-2tE,x],[H-2tE,x] \big)&=32f^2t^2-64fht-32lf
\end{split}
\end{equation}
The conclusion is that the point $\Ad(g)H$ with $g=a^{-1} n^{-1} k^{-1}$ falls into the hole in the direction $\epsilon$ if and only if
\[
 s(\epsilon)\epsilon kn\cdot H=hH+lE+fF
\]
with $l+h^2<0$.

\begin{probleme}
Il faudrait d\'emontrer que lesdits points ne sont jamais dans la lamelle.
\end{probleme}

\subsection{The graphical way}
%---------------------------

\begin{probleme}
C'est quoi la motivation pour dire que cette partie de l'espace est physique ?
\end{probleme}

Proposition~\ref{prop:AdAK} fix the form of a light cone in the physical space:
\begin{equation}
C_x^+=\Ad(a)C_c^+,
\end{equation}
which is of course also a line. A line can intersect an other line at only one point. Then if the light cone of a point intersect the singularity at two points, then it intersects the \emph{two} lines which delimits the physical space.
\begin{equation}
\sharp(C^+_x\cap\scrS)=\sharp\Ad(a)(C^+_c\cap\scrS)
                    =\sharp(C_c^+\cap\scrS)
                    =2.
\end{equation}
Then every point in the physical space begins and finishes their live in the singularity.


\subsection{The black hole}
%--------------------------


The light cone of the point $\Ad(ak)H$ --which is a general point of the physical part-- is given by $\Ad(ak)s(\epsilon)\epsilon(\Ad(e^{tE})H)$. The computation of $\Ad(ak)(H-2tE)$ and $-\Ad(ak)(-H+2tF)$ gives
\begin{subequations}  \label{eq:lin_light}
\begin{align}
(\cos(2k)-t\sin(2k))H-e^{2a}(\sin(2k)+2t\cos^{2}k)E-e^{-2a}(\sin(2k)-2t\sin^2k)F  \label{eq:lin_light_a}
\intertext{and}
(\cos(2k)-t\sin(2k))H-e^{2a}(\sin(2k)-2t\sin^{2}k)E-e^{-2a}(\sin(2k)+2t\cos^2k)F
\end{align}
\end{subequations}
With respect to $t$, these are two straight lines,  so they the intersection of $AdS_2$ and the tangent plane to $AdS_2$ at $\Ad(ak)H$.

It allows us immediately to infer the non-existence of a black hole structure for this choice of singularity. The light cone at $x\in AdS_2$ is given by the tangent plane $C$ of $AdS_2$ at $x$. The part of the singularity passing by $H$ is given by a vertical plane $S$. The intersection of these two planes is a line, and the intersection of a line with $AdS_2$ is two points. Then each of the two lines of $C\cap AdS_2$ intersect one of the two lines of $S\cap AdS_2$. The same is true for the other part of the singularity.

The conclusion is that both two lines of the light cone intersect the singularity passing by $H$ \emph{and} the one passing by $-H$.  So any point comes from the singularity and returns to the singularity; no point is connected to the infinity.

The $AdS_2$-black hole is doomed.

\subsection{Closed orbits}
%------------------------

Let us show that the singularity are the closed orbits of $AN$
and $A\bar{N}$ for the adjoint action on $AdS_2=\Ad(G)H$. A basis of $AN$
is given by $\{E,H\}$. So $x$ will belong to a closed orbit if and only if
$E_x^*\wedge H^*_x=0$. If we put $x=x_HH+x_EE+x_FF$, the computation is
\begin{equation}
\begin{split}
E_x^*\wedge H^*_x&=[E,x]\wedge[H,x]\\
                 &=4x_Hx_F E\wedge F+2x_Ex_F H\wedge E-2x_F^2 H\wedge F.
\end{split}
\end{equation}
It is zero if and only if $x_F=0$. The closed orbit of $A\bar{N}$ is given by the same computation with $H^*_x\wedge F^*_x$. The part of these orbits contained in $AdS_2$ is the one with norm $8$:
\begin{equation}
B(x,x)=8(x_H^2+x_Ex_F).
\end{equation}
In both cases, it gives $x_H=\pm 1$, and the closed orbits in $AdS_2$ are given by
\begin{subequations}
\begin{align}
\pm H&+\lambda F\\
\pm H&+\lambda E,
\end{align}
\end{subequations}
whose just are the singularity previously given in \eqref{EqQuatreLInesDeux}.


\subsection{Time orientation}
%----------------------------

The tangent vectors to the two lines of equations \eqref{eq:lin_light} give a basis of the tangent space $T_{\Ad(ak)H}AdS_2$:
\begin{subequations}
\begin{align}
-\sin(2k)H+2e^{2a}\sin^2kE+2e^{-2a}\cos^2kF
\intertext{and}
-\sin(2k)H+2e^{2a}\sin^2kE-2e^{-2a}\cos^2kF. \label{eq:deux_vec_tg}
\end{align}
\end{subequations}
The norm of the first one is $8\sin^22k+32\sin^2k\cos^2k>0$ while the one of the second is $8\sin^22k(1-2\cos^2k)$ whose sign is not well defined. We use the first one to define a \defe{time-orientation}{time!orientation!on $AdS_2$} on $AdS_2$:
\begin{equation}
T=-\sin(2k)H+2e^{2a}\sin^2kE+2e^{-2a}\cos^2kF.
\end{equation}
A vector $X\in T_xAdS_2$ is \defe{future directed}{} when $B(T,X)>0$. Note that the vector \eqref{eq:deux_vec_tg} is also future directed. Then the given light cone is well the future light cone.

\begin{remark}
We had identified $(\partial_H,\partial_E,\partial_F)$ with $(H,E,F)$ as in any vector space.
\end{remark}

At $H$, the future oriented light-like vectors are $E$ and $F$ because $T_H=2F$. It is important to remark that the ones at $-H$ are the same $E$ and $F$. The light cone at $-H$ ``go back'' to $H$! It can seem counter-intuitive, but it is necessary because, in order to be a physical space, the part $\dpt{k}{\frac{\pi}{2}}{\pi}$ must contains its own future. So the time orientation of the $AdS_2$-light cone in $\sldr$ should be a little complicated\ldots and at least uneasy to guess on a picture.

Let's see for example the destiny of a guy walking at $E+F$. Particularising \eqref{eq:lin_light}, we find the two lines
\begin{subequations}
\begin{align}
 a&\equiv tH+(1-t)E+(1+t)F\\
 b&\equiv tH+(1+t)E+(1-t)F.
\end{align}
\end{subequations}
The time orientation is given by $H+E+F$ and one can check that $a$ and $b$ are future-directed. When $t=1$, the light cone intersect $H+2F$  and $H+2E$ and when $t=-1$, it intersects $-HG+2E$ and $-H+2F$.

In the general case, with $\dpt{k}{\frac{\pi}{2}}{\pi}$, the line \eqref{eq:lin_light_a} does

\begin{itemize}
\item intersect $-H+lE$ at $l>0$ and $t=\frac{\sin 2k}{2\sin^2k}<0$,
\item intersect $H+fF$ at $f>0$ and $t=-\frac{\sin k}{\cos k}>0$,
\item don't intersect $-H+fE$,
\item don't intersect $H+lE$.
\end{itemize}


\subsection{Singularity and physical space}
%------------------------------------------

The two dimensional case is very special because it doesn't present a black hole structure. The particular structure directly appears in the groupal formalism\footnote{See section~\ref{secBTZ} for notations related to $SL(2,\eR)$.}.

We look at $M=\Ad(G)H$ on which $G=SL(2,\eR)$ acts by the adjoint action. The set $A=e^{\eR H}$, which is the abelian part of $G$ with respect to the Iwasawa decomposition, is the stabilizer of $H$. Hence the theorem~\ref{tho:homeo_action} makes, up to a double covering,
\begin{equation}
AdS_2=G/A=\Ad(G)H.
\end{equation}
The double covering is the fact that $SL(2,\eR)=\SO(2,2)/\eZ_2$, so that $\Ad(G)H$ is a double covering of $AdS_2$. This is the reason why proposition~\ref{prop:AdAK} works with $k\in]0,\pi/2[$ instead of $]0,\pi[$.

Since the Killing form is $\Ad$-invariant, $G/A$ is a sphere for the metric $B$.

 In the basis $\{H,E,F\}$ of $SL(2,\eR)$, the matrix of the Killing form is given by
\begin{equation}
B=
\begin{pmatrix}
8&&\\
&&4\\
&4&
\end{pmatrix}
\end{equation}
while the basis  $\{H,E+F,E-F\}$ gives
\[
B=
\begin{pmatrix}
8\\
&8\\
&&-8
\end{pmatrix},
\]
so that we have the following isometry, $(\sldr,B)\sim(\eR^3,\eta_{1,2})$. The latter form of the metric allows us to visualize $AdS_2$ as an hyperboloid in $\eR^3$.

 We will use the Cartan involution $\theta(X)=-X^t$, so that $\bar N$ denotes $\theta(N)=\{  e^{fF} \}$.

\begin{remark}
The vector field $H^*$ is a Killing because it is the tangent vector of the flow of the one-parameter isometry group $\dpt{\psi_t}{AdS_2}{AdS_2}$ given by $\psi_t(x)=\Ad(e^{tH})x$.
\end{remark}

Another way to express the singularity is
 \begin{equation}
\Ad(e^{nE})(\pm H) \textrm{ and }
\Ad(e^{fF})(\pm H),
\end{equation}
which shows that these are orbits of $AN$ and $A\bar{N}$. Indeed, as $\Ad(a)$ fixes $H$, we can write $\Ad(an)H=\Ad(ana^{-1})H$. Using the CBH formula we find
\[
ana^{-1}=e^{nE+2anE+\ldots}=e^{ne^{2a}E}=n'\in N.
\]
The same can be done with $f$. So $\Ad(an)H=\Ad(n')H$ and $\Ad(af)H=\Ad(f')H$. This shows that for all $n\in N$ and $a\in A$, there exists $n'\in N$ such that
\begin{subequations} \label{eq:singuAdd}
\begin{align}
\Ad(an)H&=\Ad(n')H \intertext{ The same is true with $f$:}
\Ad(af)H&=\Ad(f')H.
\end{align}
\end{subequations}

In the basis $E$, $F$, $H$ the singularity is made up from four lines with angle $45\degree$ trough $H$ and $-H$. They divide the space $AdS_2$ into four pieces. We define the \defe{physical space}{physical space} as the part of $AdS_2$ contained between $H+\lambda E$ and $-H+\lambda E$.   The $K$ part of $SL(2,\eR)$ gives a double covering of this curve. The part contained between the parts $H+\lambda F$ and $-H+\lambda F$ of the singularity should be another choice of physical space.

\subsection{Light cone}
%---------------------

The light-like vectors of $\sldr$ are $E$ and $F$, so the light cone of point $\Ad(g)H$ consists in two parts:
\[
\begin{split}
\Ad(g)\Ad(e^{tE})H,\\
\Ad(g)\Ad(e^{tF})H.
\end{split}
\]
It is best rewritten in the compact form
\begin{equation}
C^+_{\Ad(g)H}=\{\Ad(g)\Ad(e^{t\epsilon E})H\}_{%
\begin{subarray}{l}
t>0\\
\epsilon=\id,\theta
\end{subarray}
}
\end{equation}
where $\epsilon$ is the identity or the Cartan involution.

It is somewhat easy to remark that for all $X,Y$ in a Lie algebra and all automorphism $\varphi$, the formula $\varphi(\Ad(e^X)Y)=\Ad(e^{\varphi X})(\varphi Y)$ holds. Then
\begin{equation}
\Ad(e^{t\epsilon E})H=s(\epsilon)\epsilon(\Ad(e^{tE})H)
\end{equation}
with
\[
  s(\epsilon)=
\begin{cases}
1&\text{if }\epsilon=\id,\\
-1&\text{if }\epsilon=\theta.
\end{cases}
\]
Since $H^*_x=-[H,x]$, the intersection of the light cone with the singularity is expressed, using proposition~\ref{PropAdSDeuxJannule}, as
\begin{equation}
    \|[H,\Ad(g)\Ad(e^{t\epsilon E})H]\|^2=0.
\end{equation}

\section{Causal structure of \texorpdfstring{$AdS_3$}{AdS3}}
%+++++++++++++++++++++++++++

\subsection{Isometries of \texorpdfstring{$AdS_3$}{AdS3}}
%--------------------------------

We consider the group
\[
   G:=\SL(2,R)\simeq AdS_3=\{A\in GL(2,\eR):\det A=1\}
\]
on which we put the Killing metric in order to use the fact that $\Aut(G)\subset\Iso(G)$ (equation \eqref{eq:Aut_Iso}). For notational convenience, we write $\uG:=G\times G$.

\begin{lemma}
The group $\SL(2,R)\times\SL(2,R)$ is locally isomorph to $O(2,2)\subseteq \Iso(G)$.
\end{lemma}

\begin{proof}

We consider the action $\dpt{\psi}{\uG\times G}{G}$ defined by
\[
   \psi(g_1,g_2,x)=\psi(g_1,g_2)x=g_1xg_2^{-1}.
\]
Since $\psi(g_1,g_2)=L_{g_1}\circ R_{g_2^{-1}}$, the Killing form is invariant under $\psi(g_1,g_2)$ (cf. theorem~\ref{tho:bi_invariance}). Thus $\psi(g_1,g_2)\in\Iso(G)$. Now we define $\tX$: the left invariant vector field generated by $X\in\mG$ and $\utX$, the right invariant one.
\[
   dR\utilde X=\utilde X,\quad dL\tX=\tX.
\]
We proof that the differential of $\psi$ is $d\psi(X,Y)=\utX-\tX$. Consider the path $X_t$ and $Y_t$ in $G$. One has $\psi(X_t,Y_t,x)=X_txY_t^{-1}$. Taking the $d/dt$ of this,
\begin{equation}
\begin{split}
d\psi_{(e,e,x)}(X,Y,0)&=\Dsdd{\psi(X_t,Y_t,x)}{t}{0}\\
                      &=\Dsdd{X_tx}{t}{0}+\Dsdd{xY_t^{-1}}{t}{0}\\
                      &=\utX_x-\tY_x.
\end{split}
\end{equation}
From this to $d\psi_e(X,Y)=\utX-\tY$, there is just a notational game.

On the other hand, $\dim(\mG\times\mG)=6$ because a basis of $\mG$ is for example $\{E,H,F\}$ (cf. page \pageref{SecToolSL}). But the dimension of $\Lie{O(2,2)}$ is also $6$. Indeed, if we denotes by $(t,u,x,y)$ the coordinates in the $4$-dimensional space on which $O(2,2)$ acts, one has the rotations in the planes $(x,y)$, $(x,t)$, $(x,u)$, $(y,t)$, $(y,u)$, $(t,u)$, and some reversals. But the reversals are not connected to identity (because determinant is $-1$), then they don't appear in the Lie algebra.
The conclusion is\quext{Il faut encore conclure, et \c{c}a se fait dans le papier des proceedings de Stephane. Mais de toutes fa\c{c}ons pour déformer, tout ce dont nous avons besoin c'est d'avoir une action par isométries de $\uG$. Et \c{c}a, on l'a.}
\[
\mG\times\mG\simeq\Lie{\Iso(G)}.
\]

\end{proof}

\subsection{Orbits in \texorpdfstring{$AdS_3$}{AdS3}}
%----------------------------

We know that $\uG:=\SL(2,R)\times\SL(2,R)$ acts by isometries on $G=\SL(2,R)$. The question is now to see what are the orbits of this actions and are they open? We will use the techniques developed in section~\ref{SecDNqdJOp}. Let us take some notations: we denote the action by $\dpt{\tau}{\uG\times G}{G}$, $\tau((g,h),x)=gxh^{-1}$. This is the same as the former $\psi$. Clearly, $\uAN$ denotes $AN\times AN$ with respect to the decomposition $ANK$ of $\SL(2,R)$ (see~\ref{SecToolSL}) The orbits of $\uAN$ in $G$ are
\[
   \tau_{ (an,(a'n')^{-1}) }(x)=anxa'n'.
\]

\subsubsection{Two closed orbits}
%///////////////////////////////

Let us first consider the simplest case: $x\in AN$. Since $AN$ is a group, the orbit $\tau_{\uG}x$ is $AN$ itself.

The second easy case is $x\in\mZ(AN)$. In this case, the orbit is $ANx$. But it is rather clear that $AN$ is closed in $G$ because it is the exponential of $\mA\oplus\mN$ which is closed in $\sldr$, as any subspace of a vector space.

Now, let us find  ``natural'' representatives of these orbits. For the first one, the choice is clear, it is $\mtu\in AN$. For the second one, we have to find a matrix in $G$ which commutes with the whole $G$.
Equation \eqref{eq:expo_ANK} gives the explicit matrices of $\SL(2,R)$. It is not hard to see that only $\pm\mtu$ works:
\[
   \mZ(G)=\{\pm\mtu\}.
\]
Finally, our two first orbits are $\pm AN$. These are closed in $G$.

\subsubsection{Two open orbits}
%/////////////////////////////

Consider a $k\in K$. We well see at page \pageref{pg:orbit_ssvar} that $\tau_{\uG}k$ is a submanifold of $G$, thus it will be open if and only if it has dimension three. If $\uX\in(\mA\oplus\mN)\times(\mA\oplus\mN)$, we define its \defe{fundamental vector field}{fundamental!vector field} $\uX^*_x$ by
\[
\uX^*_x=\Dsdd{ \tau_{e^{t\uX}}(x) }{t}{0}
\]
for any $x\in G$, and we take the notation
\[
    \mO=\tau_{\uG}(k).
\]

On the other hand, we know that the tangent space of an orbit for a group action is spanned by the fundamentals fields for this action. Hence, the action  is generated by the left and right actions, and the fundamentals fields are the corresponding invariant fields:
\begin{equation}
T_k(\mO)=\Span\{ \utH_k,\tH_k,\utE_k,\tE_k \},
\end{equation}
where we recall that $\mG=\SL(2,R)$ and $\mA=\eR H$, $\mN=\eR E$.

\begin{remark}
Be careful on this fact: in the expression ``$T_k(\mO)$'', the $k$ appears two times: the first is the index $k$ to $T$ and the second is in the definition of $\mO$. We could work with any $T_s(\mO)$, but $k$ is the only element of $G$ which \emph{certainly} is in $\mO$. So we will work with $T_k(\mO)$.
\end{remark}


As last notation, we define $X^{\flat}(Y)=B(X,Y)$. We consider the $3$-form
\[
   \nu=\tE^{\flat}\wedge\tH^{\flat}\wedge\tF^{\flat}.
\]
Since $E$, $H$ and $F$ are linearly independent, this form is nondegenerate. Thus $\nu_k(A,B,C)=0$ if and only if $A$, $B$ and $C$ are linearly independent in $T_kG$. Let us compute:
\[
\nu_k(\utH_k,\tH_k,\tE_k)=\nu_e(dL_{k^{-1}}\utH_k,H,E)
\]
because $\tH$ and $\tE$ are left invariants. But $\utH$ is right invariant, so that $dL_{k^{-1}}\utH=\Ad(k)H$. Finally,
\begin{equation}
\nu_k(\utH,\tH,\tE)=\nu_e(\Ad(k)H,H,E).
\end{equation}


As we know that\footnote{In view of \eqref{eq:expo_ANK}, it is possible to determine the exact form of $\alpha$ and $\beta$; it will be done soon.} $\Ad(k)H=\alpha H+\beta(E+F)$ (for $k\in K$), we begin to restrict, \emph{with} loss of generality,  ourself to a $k_0$ such that $\Ad(k_0)H\neq\alpha H$. In this case,

\begin{equation}
   \nu_{k_0}(\utH,tH,tE)=\nu_e(\beta E,H,F)\neq 0.
\end{equation}
The conclusion is that the orbit of $k_0$ is open because it is three-dimensional.

If on the other hand we chose $k_1$ such that $\Ad(k_1)H=-H$,
\[
   \nu_{k_0}(\tE,\utE,\tH)=\nu_e(E,\Ad(k_1)E,H)\neq 0.
\]
Then, such a $k_1$ gives an other open orbit.

\subsubsection{Do we have all the orbits?}
%////////////////////////////////////////////

Loosely said, we will show that the orbit of $k_0$ contains at least the half of $K$, and then the half of $G=ANK$. It will then be clear that the orbit of $k_1$ takes the remaining part.

An easy computation shows that if $k\in K$ is as in equation \eqref{eq:expo_ANK},
\[
\Ad(k)H=\alpha H+\beta(E+F)
\]
with $\alpha=\cos^2t-\sin^2t$ and $\beta=-2\sin t\cos t$. On the one hand, the condition $\Ad(k_0)H\neq \alpha H$ is fulfilled for example by
\[
   k_0=\frac{\sqrt{2}}{2}\begin{pmatrix}
1 & -1 \\
1 & 1
\end{pmatrix}.
\]
On the other hand, a general element of $AN$ can be written as $\begin{pmatrix} e^a&le^a\\0&e^{-a}  \end{pmatrix}$, so that a general element of $ANk_0AN$ is
\[
   \frac{\sqrt{2}}{2}\begin{pmatrix}
     e^{a+b}(l+1) & me^{a+b}(l+1)+e^{a-b}(l-1)\\
     e^{b-a}      & me^{b-a}+e^{-(a+b)}
                     \end{pmatrix}.
\]
With a suitable choice of $a$ and $b$, one can give to $e^{a+b}$ and $e^{a-b}$ any positive value. Thus for any $m$, $l\in\eR$ and $x,y>0$, one can find
\begin{equation}
   \frac{\sqrt{2}}{2}\begin{pmatrix}
    x(l+1) & mx(l+1)+y^{-1}(l-1)\\
      y    & my+x^{-1}
                     \end{pmatrix}
\end{equation}
as general element of $\tau_{\uAN}(k_0)$. Taking
\begin{equation}
\begin{split}
  y&=-\frac{2}{\sqrt{2}}\sin t\\
  x&=\frac{\cos t}{\sqrt{2}}\\
  m&=\tan t\\
  l&=1,
\end{split}
\end{equation}
one finds the matrix $\begin{pmatrix} \cos t & \sin t\\
                                      -\sin t & \cos t \end{pmatrix}$
for any $\sin t >0$. In this sense, we say that one has at least \emph{the half} of $K$ in $\tau_{\uG}(k_0)$. Then, by re-acting with $AN$, one finds the half of $G=ANK$.

\subsection{The metric}
%----------------------

We compute the metric in the basis $\{\dtau,\du,\dphi\}$ at the point $z=e^{\phi H}e^{eE}e^{\tau T}e^{\alpha\phi H}$:
\begin{equation}
	\begin{aligned}[]
   dL_{z^{-1}}\dtau&=\Dsdd{  e^{-\alpha\phi H}e^{-\tau T}e^{-uE}e^{-\phi H}e^{\phi H}e^{uE}e^{(\tau+t)T}e^{\alpha\phi H}  }{t}{0}\\
             &=\Ad(e^{-\alpha\phi H})T.
	\end{aligned}
\end{equation}
In the same way,
\begin{equation}
	\begin{aligned}[]
  dL_{z^{-1}}\du&=\Ad(e^{-\alpha\phi H}e^{-\tau T})E\\
  dL_{z^{-1}}\dphi&=\Ad(z^{-1})H+\alpha H.s
	\end{aligned}
\end{equation}
In the computation of $dL_{z^{-1}}\dphi$, we used the Leibnitz rule. The rest is a rather mechanical computation using the $\Ad$-invariance of $B$ and the fact that, by definition if $B_z$, $B_z(X,Y)=B_e(dL_{z^{-1}}X,dL_{z^{-1}}Y)$. The following relations will be useful:
\begin{subequations}
\begin{align}
  \Ad(e^{uE})T&=T-nH+n^2E\\
  \Ad(e^{uE})H&=H+2nE\\
  \Ad(e^{-\tau T})E&=-\sin\tau\cos\tau H+\cos^2\tau E-\sin^2\tau F
\end{align}
\end{subequations}
\begin{align}
  B(\dtau,\dtau)&=B(T,T)=-8\\
\begin{split}
  B(\dtau,\du)  &=B(T,\Ad(e^{-\tau T})E)\\
                &=B(\Ad(e^{\tau T})T,E)\\
		&=B(E-F,E)=4,
\end{split}\\
\begin{split}
     B(\dtau,\dphi)&=B(\Ad(e^{-\alpha\phi H})T,\alpha H)\\
                     &\quad +B(T, \Ad(e^{-\tau T}e^{-uE}e^{-\phi H})H  )\\
		   &=\underbrace{B(T,\alpha H)}_{=0}+B( \Ad(e^{uE})T,\Ad(e^{-\phi H})H )\\
		   &=B(\Ad(e^{uE})T,H)\\
		   &=-4n,
\end{split}\\
  B(\du,\du)&=B(E,E)=0,\\
\begin{split}
  B(\du,\dphi)&=B(  E,\Ad(e^{-uE}e^{-\phi H})H )\\
              &\quad+B( \Ad(e^{-\tau T})E,\alpha H )\\
	      &=B(-\sin\tau\cos\tau H+\cos^2\tau E-\sin^2\tau F,H)\\
	      &=-2\alpha\cos(2\tau).
\end{split}
 \end{align}

 % Il faut encore taper le B(\dphi,\dphi) dès le haut de /84
