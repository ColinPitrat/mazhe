% This is part of (almost) Everything I know in mathematics and physics
% Copyright (c) 2013-2014
%   Laurent Claessens
% See the file fdl-1.3.txt for copying conditions.

\begin{abstract}

We are now going to apply deformation theory to the physical part of the $AdS_4$ black hole. The first idea was to deform the $AN$ of $\SO(2,l-1)$ and to deform $AdS_l$ by action of this group (see section~\ref{SecDefAction} for an introduction to deformation by group action). We show in section~\ref{SecUnifSOdn} that this procedure is possible.

Instead of that, we will only deform an open orbit of $AN$ in the four dimensional case\footnote{But the structure of algebra \eqref{EqTableSOIwa}, promises easy higher dimension generalisation.}. There are two reasons for that. First a physical domain of the black hole is contained in an open orbit of $AN$; and second it reveals possible to deform such a domain by action of a four dimensional group. Deforming a four dimensional space by a four dimensional group instead of a six dimensional one is a matter of ``no waste'' of dimensions.

The main lines of the construction are the following:
\begin{itemize}
\item We pick an open orbit $\mU$ of $AN$ in $AdS_4$, and we select a point $[u]\in\mU$.
\item We compute the stabilizer $S$ of $[u]$ in $AN$, in such a way that, as homogeneous space, $\mU=AN/S$. We consider the ``remaining group'' $R'$ of $R$ when one removes $S$ from $R$.
\item We prove that $R'$ acts freely and transitively on $\mU$, so that $\mU$ is globally of group type (definition~\ref{DefGlobGpType}).
 \item It turns out that $R'$ does not accept any symplectic structure; hence we will search for other groups acting transitively on $\mU$, and show that one and only one of them accepts a symplectic structure.
\item The latter group turns out to be a split extension of an Heisenberg group (see appendix~\ref{SecExtHeiz}) for which we know a deformation.
\end{itemize}

This part is published in \cite{articleBVCS}.
\end{abstract}

\section{Group structure on the open orbits}
%++++++++++++++++++++++++++++++++++++++++++
\label{SecGpStructOuvertOrb}

 \subsection{Global structure}
%----------------------------

The following definition formalises the idea for a manifold to be ``like a group''. We will prove that the physical domain of our anti de Sitter black hole is of this type.
\begin{definition}
A $m$-dimensional homogeneous space $M$ is \defe{locally of (symplectic) group type}{locally!group type} if there exists a Lie (symplectic) subgroup $R$ of the group of automorphisms of $M$ which acts freely on one of its orbits in $M$. The homogeneous space is \defe{globally of group type}{globally group type} if $R$ has only one orbit. In this case, for every choice of a base point $\mfo$ in $M$, the map $R\to M:g\mapsto g.\mfo$ is a diffeomorphism.
            \label{DefGlobGpType}
\end{definition}

Lie groups are themselves examples of symmetric spaces (globally) of group type.  In the symplectic situation, however, a symplectic symmetric Lie group must be abelian (\cite{ThzPierre}, page 12). We will see in what follows other non-abelian examples.

In the context of our anti de Sitter black hole (in particular when one has causal issues in mind), it is not important to deform the whole space but it is sufficient\footnote{We will however point out in the perspectives (section~\ref{SecConcPerspAd}) that a quantization of the whole space has a real interest.} to only deform one open orbit of $AN$. Indeed, if an observer begins his life somewhere in the physical space (hence in an open orbit of $AN$), he will never exit the orbit because one open orbit of $AN$ is bounded by closed orbit of $AN$ which are singular.

Let us recall that the solvable part of the Iwasawa decomposition of $\so(2,3)$ may be realized with a nilpotent part $\sN$ and an abelian one $\sA$  with elements
\begin{subequations}
\begin{align}
\sA&=\{ J_1, J_2\}  &\sN&=\{W,V,M,L\}
\end{align}
\end{subequations}
and the commutator table
\begin{subequations}    \label{EqTableRappelSO}
\begin{align}
[V,W]&=M &[V,L]&=2W\\
[ J_1,W]&=W       &[ J_2,V]&=V\\
[ J_1,L]&=L           &[ J_2,L]&=-L\\
[ J_1,M]&=M           &[ J_2,M]&=M,
\end{align}
\end{subequations}
where we know that that $W$, $J_{1}\in\sH$, and $J_{2}\in\sQ$. We pick the point
\[
  [u]=
\Bigg[
\begin{pmatrix}
0&1\\
-1&0\\
&&\mtu_{3\times 3}
\end{pmatrix}
\Bigg].
\]
This is an element of $K$ which, as already mentioned in page \pageref{PgNoticeKpassung}, therefore does not belong to a closed orbit of $AN$, neither to a one of $A\bar N$. Hence $[u]$ lies in the physical part of $AdS_4$. We denote by $\mU$ the $AN$-orbit of $[u]$.

Elements of the stabilizer of $[u]$ in $\SO(2,3)$ are elements $r$ such that $r\cdot[u]=[u]$, i.e. elements for which there exists  $h$ in $H$ such that $ru=uh$. It is easy to see that $J_{2}$ once again belongs to the Lie algebra of the stabilizer. The new element for $AdS_{4}$ is $V$. Indeed $V^{3}=0$, so the exponential is a three term sum:
\[
   e^{aV}=
\begin{pmatrix}
b+1&0&-b&0&a\\
0&1&0&0&0\\
b&0&1-b&0&a\\
0&0&0&1&0\\
a&0&-a&0&1
\end{pmatrix}
\]
where $b=a^{2}/2$. We hope that $[ e^{aV}u]=[u]$, i.e. that
\[
   e^{aV}u=uh
\]
for a certain $h\in H$. A direct computation of $h=u^{-1} e^{aV}u$ gives
\[
  h=
\begin{pmatrix}
1   &   0   &0  &0  &0\\
0   &   b+1 &-b &0  &a\\
0   &   b   &1-b    &0  &a\\
0   &   0   &0  &1  &0\\
0   &   a   &-a &0  &1
\end{pmatrix}
\]
which is a matrix of $H$ because it leaves unchanged the first basis vector.

The stabilizer cannot contain more than two generators because an open orbit must be four dimensional. The stabilizer of $[u]$ in $G=\SO(2,3)$ is thus the group generated by $ e^{aJ_{2}}$ and $ e^{bV}$ plus eventually a discrete set making $S$ non connected.  The group $S$ is in fact connected because
\begin{equation}
   S=\{ r\in R\tq r\cdot[u]=[u] \}
    =\{ r\in R\tq \AD(u^{-1})r\in H \}.
\end{equation}
Since $R$ is an exponential group, we have $S=\exp\sS$ where
\[
  \sS=\{ X\in\sR\tq\Ad(u^{-1})X\in\sH \}=\Ad(u)\sH.
\]
The set $\sS$ being connected (because it is the image by a continuous map of the connected set $\sH$), $S$ is connected too.

 The open orbit that we are studying is thus realised as the homogeneous space $\mU=AN/S$. An important result is the fact that what we obtain by simply removing $J_2$ and $V$ from the table \eqref{EqTableRappelSO} is still an algebra. The orbit $\mU$ is therefore isomorphic to the group $R'$ generated by the Lie algebra $\sR'=\{ J_1,W,M,L \}$. The table of $\sR'$ is
\begin{subequations}
\begin{align}
  [J_{1},W]&=-W\\
[J_{1},L]&=L\\
[J_{1},M]&=M.
 \end{align}
\end{subequations}
From construction, $R'\cap S=\{ e \}$. Unfortunately, using the conditions \eqref{EqDefAlgSymple}, we find that in order to be compatible with the Lie algebra structure, the form $\omega$ of the algebra must satisfy $\omega(W,M)=\omega(W,L)=\omega(M,L)=0$, so that it is degenerate.  The action of $R'$ on $\mU$ enjoys however some remarkable properties.
\begin{proposition}
The action
\begin{equation}
\begin{aligned}
 \tau\colon R'\times \mU&\to \mU \\
r[r_0u]&= [rr_0u]
\end{aligned}
\end{equation}
is free and simply transitive.
\label{PropURsimptra}
\end{proposition}

\begin{proof}

First, we prove that the action of $R'$ is transitive. As an algebra, $\sR$ is a split extension $\sR=\sS\oplus_{\ad}\sR'$.  Hence, as group, $R=S R'$, or equivalently $R=R'S$. That proves that the action is transitive.

If the action is not simply transitive, there exists $x\in\mU$ and $r$, $r'\in R'$ such that $\tau_rx=\tau_{r'}x$. Since the action of $R'$ is transitive, we have a $r_1\in R'$ such that $x=r_1[u]$. In this case, the element $r_1^{-1}r^{-1}r'$ of $R'$ fixes $[u]$, but $R'\cap S=\{ e \}$. Then one deduces that $r'=rr_1$, so that $[rr_1u]=[r'r_1u]=[rr_1^2u]$. It follows that $r_1$ fixes $[u]$, and thus that $r_1=e$, so that $r=r'$.

For freeness remark that, in a neighbourhood of $e$, the neutral $e$ itself is the only element trivially acting on $[u]$.

\end{proof}

As corollary, the orbit $\mU$ is locally of group type $R'$.

\begin{proposition}     \label{PropmUsimpl}
The orbit $\mU$ is simply connected.
\end{proposition}

\begin{proof}
    The \href{http://en.wikipedia.org/wiki/Fibration}{fibration}\index{fibration} $S\to R\to\mU$ induces the long exact sequence of cohomology groups
    \[
      H^{0}(\mU)\to H^{0}(R)\to H^{0}(S)\to H^{1}(\mU)\to H^{1}(R)\to\ldots
    \]
    The group $R$ being connected and simply connected, the sequence shows that $H^{0}(S)\simeq H^{1}(\mU)$, but we already mentioned that $S$ is connected, so $H^1(\mU)=0$.
\end{proof}

\begin{probleme}
Il faut une citation pour le coup de la suite qui découle de la fibration.
\label{ProbFibra}
\end{probleme}


\begin{corollary}
The open orbit $\mU$ is globally of group type.
\label{CormUgloGppasSym}
\end{corollary}

\begin{proof}
It is immediately apparent from proof of proposition~\ref{PropURsimptra}. Since
\[
  R'[u]=R'S[u]=R[u]=\mU,
\]
the group $R'$ acts freely on $\mU$ and has only one orbit.
\end{proof}

Remark  that it remains to be proved that $\mU$ is globally of \emph{symplectic} group type. For that, there should be a symplectic form on $R'$. Exploiting the fact that $\Span\{W,M,L \}$ is a three-dimensional abelian subalgebra of $\sR'$, it is easy to see that $\sR'$ does not accept a symplectic form. Hence  corollary ~\ref{CormUgloGppasSym} does not prove that $\mU$ is globally of symplectic group type. The lack of symplectic form on the algebra reflects on $\mU$ as manifold by the following lemma, and motivates the search for other four-dimensional groups than $R'$ acting transitively on $\mU$.

\begin{lemma}
The open orbit $\mU=R\cdot [u]$ does not admit any $R$-invariant symplectic form.
\end{lemma}

\begin{proof}
Let $\omega^{\mU}$ be such an invariant symplectic form and $\omega^{R'}$ be the pull-back of $\omega^{\mU}$ by the action: $\omega^{R'}=\tau^*\omega^{\mU}$.
We have $d\tau\circ dL_{r'}=dL_{r'}\circ d\tau$ because $\tau(r'X(t))=[r'X(t)u]=r'\tau(X(t))$, thus
\[
  L_{r'}^*\omega^{R'}=(\tau\circ L_{r'})^*\omega^{\mU}=(L_{r'}\circ\tau)^*\omega(\mU)=\omega^{R'},
\]
so that $\omega^{R'}$ is a $R'$-invariant symplectic form on $R'$. But we saw that such a form does not exist.
\end{proof}

\begin{proposition}
The $R$-homogeneous space $\mU$ admits a unique structure of globally group type symplectic symmetric space. The latter is isomorphic to $(\SUR_0,\omega,s)$ described in appendix~\ref{SecDefSURme}.
\label{GT}
\end{proposition}
The next few pages are dedicated to prove this proposition and to give explicit algebra whose group gives the answer. We are searching for $4$-dimensional groups $\tilde R$ which
\begin{itemize}
\item has a free and simply transitive action on $\mU$, i.e. $\tilde{R}[u]=R[u]$,
\item admits a symplectic structure,
\end{itemize}
and we want it to be unique.  As already mentioned, the algebra $\sR'$ fails to fulfil the symplectic condition. The algebra $\tilde{\sR}=\Span\{ A,B,C,D \}$ of a group which fulfils the first condition must at least act transitively on a small neighbourhood of $[u]$ and thus be of the form
\begin{subequations}  \label{EqGeneAlgabcd}
\begin{align}
 A&=J_{1}+aJ_{2}+a'V\\
 B&=W+bJ_{2}+b'V\\
 C&=M+cJ_{2}+c'V\\
 D&=L+dJ_{2}+d'V.
\end{align}
\end{subequations}
Indeed, in a first attempt, we choose an algebra for which each of $A$, $B$, $C$ and $D$ contains a combination of $J_{1}$, $W$, $M$ and $L$. We consider the matrix of coefficients of $J_{1}$, $W$, $M$ and $L$ in $A$, $B$, $C$ and $D$. If the determinant of this matrix is zero, then one of the lines can be written as combination of the three others. In this case the action can even not be locally  transitive because the algebra only spans three directions actually acting ($J_{2}$ and $V$ have no importance here). So the determinant is non vanishing. In this case, the inverse of this matrix is a change of basis which puts $A$, $B$, $C$ and $D$ under the form  \eqref{EqGeneAlgabcd}.

The problem is now to fix the parameters $a,a',b,b',c,c',d,d'$ in such a way that the space $\Span\{ A,B,C,D \}$ becomes a Lie algebra (i.e. it closes under the Lie bracket) which admits a symplectic structure and whose group acts transitively on $\mU$. We will begin by proving that the surjectivity condition imposes $b=c=d=0$. Then the remaining conditions for $\tilde \sR$ to be an algebra are easy to solve by hand.

First, remark that $A$ acts on the algebra $\Span\{ B,C,D \}$ because $J_1$ does not appear in $[\sR,\sR]$. Hence we can write $\tilde{\sR}=\eR A\oplus_{\ad}\Span\{ B,C,D \}$ and, a subalgebra of a solvable exponential Lie algebra being a solvable exponential algebra, a general element of the group $\tilde{R}$ reads $\tilde r (\alpha,\beta,\gamma,\delta)= e^{\alpha A} e^{\beta B+\gamma C+\delta D}$. Our strategy will be to split this expression in order to get a product $S R'$ (which is equivalent to a product $R'S $). As Lie algebra, $\Span\{ B,C,D \}\subset\eR J_2\oplus_{\ad}\{ W,M,L,V \}$. Hence there exist functions $w$, $m$, $l$, $v$ and $x$ of $(\alpha,\beta,\gamma,\delta)$ such that
\begin{equation} \label{EqGeneRi}
 e^{\beta B+\gamma C+\delta D}= e^{xJ_2} e^{wW+mM+lL+vV}.
\end{equation}
We are now going to determine $l(\alpha,\beta,\gamma,\delta)$ and study the conditions needed in order for $l$ to be surjective on $\eR$. Since $J_2$ does not appear in any commutator, the Campbell-Baker-Hausdorff formula yields $x=\beta b+\gamma c+\delta d$. From the fact that $[J_2,L]=-L$, we see that the coefficient of $L$ in the left hand side of \eqref{EqGeneRi} is $-l(1- e^{-x})/x$. The $V$-component in the exponential can also get out without changing the coefficient of $L$. We are left with $\tilde r(\alpha,\beta,\gamma,\delta)= e^{\alpha A} e^{xJ_2} e^{yV} e^{w'W+m'M+lL}$ where $w'$ and $m'$ are complicated functions of $(\beta,\gamma,\delta)$ and $l$ is given by
\begin{equation}
l(\beta,\gamma,\delta)=\frac{ -\delta (\beta b+\gamma c+\delta d) }
{ 1- e^{-\beta b-\gamma c-\delta d} },
\end{equation}
which is only surjective when $b=c=d=0$. Taking the inverse, a general element of $\tilde R[u]$ reads $\big[ e^{ -wW-mM-lM} e^{j_1J_1}u \big]$, where the range of $l$ is not the whole $\eR$. Since the action of $R'$ is \emph{simply} transitive, $\tilde R$ is not surjective on $R[u]$ when $l(\alpha,\gamma,\delta)$ is not surjective on $\eR$.

When $b=c=d=0$, the conditions for \eqref{EqGeneAlgabcd} to be an algebra are easy to solve, leaving only two \emph{a priori} possible two-parameter families of algebras.

{ \renewcommand{\theenumi}{\arabic{enumi}.}
\begin{enumerate}
\item
The first one is the following:

\label{PgAlgUn}
\begin{subequations}
 \begin{align*}
A&=J_{1}+\frac{ 1 }{2}J_{2}+sV  &[A,B]&=B+sC\\
B&=W                &[A,C]&=\frac{ 3 }{2}C\\
C&=M                &[A,D]&=2sB+\frac{ 1 }{2}D\\
D&=L+rV             &[B,D]&=-rC.
\end{align*}
\end{subequations}
with $r\neq 0$. The general symplectic form on that algebra is given by
\begin{equation}
\omega_{1}=\begin{pmatrix}
0   &-\alpha        &-\beta &-\gamma\\
\alpha  &0          &0  &\frac{ 2\beta r }{ 3 }\\
\beta   &0          &0  &0\\
\gamma  &-\frac{ 2\beta r }{ 3 }    &0  &0
\end{pmatrix},
\end{equation}

\[
\det\omega=\left( \frac{ 2\beta r }{ 3 } \right)^{2},
\]
Conditions: $\beta\neq 0$, $r\neq 0$. That algebra will be denoted by $\sR_{1}$. The analytic subgroup of $R$ whose Lie algebra is $\sR_1$ is denoted by $R_1$.  One can eliminate the two parameters in algebra $\sR_{1}$ by the isomorphism
\begin{equation}        \label{EqIsomRUnrs}
\phi=
\begin{pmatrix}
1&0&0&0\\
0&1&0&4s\\
0&2sr&1/r&4s^{2}/r\\
0&0&0&1
\end{pmatrix}
\end{equation}
which fixes $s=0$ and $r=1$. The algebra $\sR_1$ is thus isomorphic to
\begin{equation}
\begin{aligned}[]
 [A,B]&=B            &[A,C]&=\frac{ 3 }{ 2 }C\\
[A,D]&=\frac{ 1 }{ 2 }D      &[B,D]&=-C.
\end{aligned}
\end{equation}

Comparing with equation \eqref{EqrhoBmudz}, one recognizes the one-dimensional extension of Heisenberg algebra with parameters $d=3/4$, $\mu=0$ and $\BX=\begin{pmatrix}
1&0\\0&1/2
\end{pmatrix}$. Hence $R_1$ is isomorphic to $\SUR_0$ and, by the way,  we have a product on that group (see appendix~\ref{SecExtHeiz}).
\item
The second algebra whose group acts simply transitively on $\mU$ is:
\begin{align*}
A&=J_{1}+rJ_{2}+sV      &[A,B]&=B+sC\\
B&=W                &[A,C]&=(r+1)C\\
C&=M                &[A,D]&=2sB+(1-r)D.\\
D&=L
\end{align*}
There is no way to get a nondegenerate symplectic form on that algebra.

\end{enumerate}
}       % Fin du groupe pour que l'énumération se fasse en chiffres arabes

From proposition~\ref{PropSymplestarEG}, the symplectic structure to be chosen on $\sR_1$ is $\delta C^*$ and lemma~\ref{LemJumpCoadOrb} shows that we are able to quantize\footnote{by opposition to \emph{deform}: there are no symplectic condition in deformation.} $\sR_1$ with any symplectic form in the coadjoint orbit $\delta\big( C^*\circ\Ad(g) \big)$ with $g\in R_1$. The coadjoint adjoint action of $R_1$ on $\sR_1$ can be computed using the fact that $\sR_1$ splits into four parts; the non trivial results are
\begin{align*}
\Ad( e^{dD}A)&=A-\frac{ d }{2}D         &\Ad( e^{aA})B&= e^{a}B\\
\Ad( e^{cC})A&=A-\frac{ 3c }{2}C        &\Ad( e^{aA})C&= e^{3a/2}C\\
\Ad( e^{bB})D&=D-bC             &\Ad( e^{aA})D&= e^{a/2}D\\
\Ad( e^{bB})D&=D-bC.
\end{align*}
Direct computations show that
\begin{equation}
\begin{split}
    \Ad\big(  e^{aA} e^{bB} e^{cC} e^{dD} \big)(x_AA+x_BB&+x_CC+x_DD)\\
            &=x_AA+ e^{a}(x_B-x_Ab)B\\
            &\quad + e^{3a/2}\Big( x_C-\frac{ 3x_Ac }{2}-bx_D +\frac{ bdx_A }{2} \Big)C\\
            &\quad+ e^{a/2}\Big( x_D-\frac{ dx_A }{2} \Big)D,
\end{split}
\end{equation}
so that, with more compact notations,
\begin{align}
\big( C^*\circ\Ad(g) \big)(X)=\big( x_C-\frac{ 3x_Ac-bx_D+\frac{ bdx_A }{2} }{2} \big) e^{3a/2},
\end{align}
and the symplectic forms that we are able to deform are given by $\delta\big( C^*\circ\Ad(g) \big)$. It provides a two-parameter familly of symplectic forms
\begin{equation}
\omega_1^g=
\begin{pmatrix}
0&0&\beta&\gamma\\
0&0&0&-2\beta/3\\
-\beta&0&0&0\\
-\gamma&2\beta/3&0&0
\end{pmatrix},
\end{equation}
\[
  \det\omega_1^g=\frac{ 4\beta^4 }{ 9 }.
\]



It turns out that the action of the group $R_1$ has good properties that are given in the following proposition.
\begin{proposition}
The action of $R_1$ on $\mU$ is free and simply transitive.
\label{PropCRunXXX}
\end{proposition}

\begin{proof}

First remark that the algebra $\sR_1$ can be written as a split extension:
\[
  \sR_1=\eR A\oplus_{\ad} \eR D\oplus_{\ad}\Span\{ B,C \},
\]
hence a general element of $R_1$ reads
 \begin{equation}
r_1(a,b,c,d)= e^{aA} e^{dD} e^{bW} e^{cM}.
\end{equation}
The work is now to expand it by replacing $A$, $B$, $C$, $D$ by their values in function of $J_1$, $W$, $M$, $L$, $J_2$ and $V$, and then to try to put all elements of $\sS$ on the left. This is done by virtue of Campbell-Baker-Hausdorff formula. The fact that $\Span\{ W,M,N,V \}$ is nilpotent dramatically reduces the difficulty. We have
\[
  \ln( e^{drV} e^{dL} e^{wW} e^{mM})= drV+dL+(d^{2}r+w)W+(\frac{1}{ 6 }d^{2}r^{2}+m+\frac{ 1 }{2}drw)M.
\]
One can find $m$ and $w$ (functions of $d$) such that the right hand side reduces to $drV+dL$. Hence we have, for some auxiliary functions $w$ and $m$,
\[
   e^{drV+dL}= e^{drV} e^{dL} e^{w(d)W} e^{m(d)M}
\]
and a general element of $R_1$ reads
\begin{equation}
 e^{asV+\frac{ a }{2}J_2} e^{drV} e^{aJ_1} e^{dL} e^{\big( w(d)+b \big)W} e^{\big( m(d)+c \big)M}=s(a,d)r'(a,b,c,d)
\end{equation}
with $s\in S$ and $r'\in R'$ which defines a bijective map $r_1\mapsto r'$ from $R_1$ to $R'$. This proves the transitivity of the action of $R_1$.

 For freeness, just remark that in a neighbourhood of $e$, no element of $R_1$ (but $e$) leaves $[u]$ unchanged.
\end{proof}

The conclusion is that $R_1$ is the group $\tilde{R}$ that we were searching for and that it is unique (up to the two-parameter isomorphism \eqref{EqIsomRUnrs}) as symplectic subgroup of $AN$ acting transitively on $\mU$. It concludes the proof of proposition~\ref{GT}.

\subsection{Alternative more intrinsic proofs}      \label{subSecAltreintr}
%----------------------------------------------

\begin{proposition}
Let $J\in Z(K)$ whose associated conjugation coincides with the Cartan involution: $\AD(J)=\theta$ and $u\in\SO(2,l-1)$ such that $u^2=J$ and $u\in e^{\sQ}\cap K$. Then the $AN$-orbit of $[u]$ is open.
\end{proposition}

\begin{probleme}
Il faut prouver cela. Pour cela, je crois que ce qu'il faut faire, c'est dire que Helgason ou Loos assure l'existence d'un $J\in Z(K)$ tel que $\theta=\AD(J)$ pour le groupe, et $\theta=\Ad(J)$ pour l'algebre. Ensuite, vu que les orbites fermées de $AN$ et $A\bar N$ sont $\pm AN$ et $\pm A\bar N$, évidement un élément d'une orbite fermée ne peut pas avoir son carré dans $K$.
\label{ProbAdJthetaj}
\end{probleme}

\begin{proof}
Let us find the Lie algebra $\sS$ of the stabilizer $S$ of $[u]$. First, the Cartan involution $X\mapsto -X^{t}$ is implemented as $\AD(J)$ with
\[
  J=\begin{pmatrix}
-\mtu_{2\times 2}\\
&\mtu_{3\times 3}
\end{pmatrix}
\]
which satisfies $u^{2}=J$ and $\sigma(u)=u^{-1}$ because $u\in\sQ$. Now, $\AD(u^{-1})r\in H$ if and only if $\sigma\Big( \AD(u^{-1})r \Big)=\AD(u^{-1})r$. Using the fact that $\sigma$ is an involutive automorphism, we see that this condition is equivalent to
\begin{equation}
\theta\sigma r=r.
\end{equation}

On the one hand the Cartan involution $\theta$ restricts on $\sA$ to $\theta|_{\sA}=-\id$ because $\sA\subset\sP$; and on the other hand, $\sigma(\sA)=\sA$ because $J_1\in\sH$ and $J_2\in\sQ$. So $\sigma$ splits $\sA$ into two parts:  $\sA=\sA^{+}\oplus\sA^{-}$ with $\sA^+=\sA\cap\sH=\eR J_1$ and $\sA^-=\sA\cap\sQ=\eR J_2$. Let $\beta_{1},\beta_{2}\in\sA^*$ be the dual basis: $\beta_{i}(J_{j})=\delta_{ij}$.  We know that $W\in\sG_{\beta_{1}}$, $V\in\sG_{\beta_{2}}$, $L\in\sG_{\beta_{1}-\beta_{2}}$, and $M\in\sG_{\beta_{1}+\beta_{2}}$.  The set of simple roots is given by
\[
  \Delta=\{ \alpha=\beta_{1}-\beta_{2},\,\beta=\beta_{2} \},
\]
and the positive roots are
\[
 \Phi^{+}=\{ \alpha,\beta,\alpha+\beta,\alpha+2\beta \},
\]
in terms of whose, the space $\sN$ is given by
\begin{align*}
W&\in\sG_{\alpha+\beta}&V&\in\sG_{\beta}\\
L&\in\sG_{\alpha}&M&\in\sG_{\alpha+2\beta}.
\end{align*}
Since $(\sigma^*\beta)(h_1J_1+h_2J_2)=\beta_{2}(h_1J_1-h_2J_2)=-h_2$, we find
 we find
\begin{align*}
 \sigma^*\beta&=-\beta\\
\sigma^*\alpha&=\alpha+2\beta\\
\sigma^*(\alpha+\beta)&=\alpha+\beta\\
\sigma^*(\alpha+2\beta)&=\alpha.
\end{align*}
We are now able to identify the set $\sS=\{ X\in\sA\oplus\sN\tq \sigma\theta X=X \}$. Let us take $X\in\sR=\sA\oplus\sN$ and apply $\sigma\theta$:
\begin{equation}
\begin{split}
  X&=X^++X^-+X_{\alpha}+X_{\beta}+X_{\alpha+\beta}+X_{\alpha+2\beta},\\
\theta X&=-X^+-X^-+Y_{-\alpha}+Y_{-\beta}+Y_{-\beta}+Y_{-(\alpha+\beta)}+Y_{-(\alpha+2\beta)},\\
\sigma\theta X&=-X^++X^-+Z_{-(\alpha+2\beta)}+Z_{\beta}+Z_{-(\alpha+\beta)}+Z_{-\alpha}
\end{split}
\end{equation}
where $X_{\varphi}$, $Y_{\varphi}$ and $Z_{\varphi}$ denote elements of $\sG_{\varphi}$, and $X^{\pm}$ denote the component $\sA^{\pm}$ of $X$.

It is immediately apparent that $\sigma\theta X^-=X^-$, so that $X^-\in\sS$.  The only other component common to $X$ and $\sigma\theta X$ is in $\sG_{\beta}$, but   it is \emph{a priori} not clear that $X_{\beta}=Z_{\beta}$. We know however that $\sigma\theta V=\alpha V$ because $\sG_{\beta}$ has only one dimension. Using the fact that $\sigma$ and $\theta$ are commuting involutions, it is apparent that $\alpha=\pm 1$. Decomposing $V$ into $V=V_{\sH}+V_{\sQ}$, we have $\theta\sigma V=\theta(V_{\sH}+V_{\sQ})$ which has to be equal to $V$ or $-V$. Thus there are only two possibilities
\begin{align*}
  \theta V_{\sH}&=V_{\sH}&\text{or} &&\theta V_{\sH}&=-V_{\sH}\\
\theta V_{\sQ}&=-V_{\sQ}&       &&\theta V_{\sQ}&=V_{\sQ}.
\end{align*}
If one compares the commutator table of $\SO(2,3)$ with the one of $\SO(2,2)$, one sees that $V$ is not present in $\SO(2,2)$. Since $\sH$ possesses every purely spatial rotation generators, the orthogonal complement $\sQ$ contains the time-time rotation as only rotations. Other components of $\sQ$ are boost. In particular, $V_{\sQ}$ is zero or a boost generator. In the latter case, $\theta V_{\sQ}=-V_{\sQ}$, and the conclusion is that $\sigma\theta V=V$. In the other case, $V_{\sQ}=0$ implies that $V\in\sH$ which is impossible because $[J_2,V]=0$ while $J_2\in\sQ$ and $[\sQ,\sH]\subset\sQ$.



The stabilizer $S$ is thus generated by $J_2$ and $\sG_{\beta}=\eR V$, i.e.
\begin{equation}
\sS=\Span\{ J_2,V \}.
\end{equation}
The stabilizer of $[u]$ being two-dimensional, the orbit of $[u]$ is four-dimensional and is then open in $AdS_4$.

\end{proof}

Notice that in contrast to the first way to find $\sS$, this time we have no eventually double covering problems.

Let $\tilde R$ be a subgroup of $R$ whose Lie algebra is a complement of $\sS$ in $\sR$, i.e. $\tilde{\sR}\oplus\sS=\sR$. This group does not act transitively on $\mU$ if and only if the boundary of $\tilde R[u]$ is non empty. Let $x_0=\tau_{r'_0}[u]$ belong to that boundary with $r'_0\in R'$. On that point, the fundamental fields of $\tilde R$ are not surjective on the tangent space of $\mU$:
\[
\begin{split}
\ker\big[ \tilde{\sR}&\to T_{x_0}\mU \big] \neq\{ 0 \}\\
 Y&\mapsto Y^*_{x_0}.
\end{split}
\]
Let $Y\in\tilde R$ belongs to this kernel: $Y^*_{x_0}=0$. Since the linear map $\big( d\tau_{r_0'^{-1}} \big)_{x_0}$ is nondegenerate, $Y^*_{x_0}$ vanishes if and only if $\big( d\tau_{r_0'^{-1}} \big)_{x_0}(Y^*_{x_0})=0$, but
\begin{equation}
\begin{split}
 \big( d\tau_{r_0'^{-1}} \big)_{x_0}(Y^*_{x_0})=-\Dsdd{ r_0'^{-1} e^{tY}r_0'[u] }{t}{0}
        &=\big( \Ad(r_0'^{-1})Y \big)^*_{[u]}\\
        &=\pr_{\tilde{\sR}}\big( \Ad(r_0'^{-1})Y \big)
\end{split}
\end{equation}
because, on the point $[u]$, the action to take the fundamental field is nothing else than the projection parallel to $\sS$. Hence the group $\tilde R$ is not surjective if and only if
\[
  \big( \Ad(R')\sS \big)\cap\tilde{\sR}\neq\{ 0 \}.
\]
We are now going to determine $\Ad(R')\sS$. Let $X=X^{-}+X_{\beta}\in\sS$ and act with an element of $R'=\exp\big( \sA^{+}\oplus\sG_{\alpha}\oplus\sG_{\alpha+\beta}\oplus\sG_{\alpha+2\beta}\big)$:
\[
\begin{aligned}
   \Ad( e^{H^{+}+Y_{\alpha}+Y_{\alpha+\beta}+Y_{\alpha+2\beta}})(X^{-}+X_{\beta})
        &=X^{-}+X_{\beta}\\
        &+\underbrace{\big[ H^{+}+Y_{\alpha}+Y_{\alpha+\beta}+Y_{\alpha+2\beta},X^{-}+X_{\beta} \big]}_{N'}\\
        &\qquad+\frac{ 1 }{2}\big[  H^{+}+Y_{\alpha}+Y_{\alpha+\beta}+Y_{\alpha+2\beta},N'\big]\\
        &\qquad+\ldots
\end{aligned}
\]
The computation of $N'$ is as follows:
\begin{align*}
[H^+,X^-]&=0                    &[H^{+},X_{\beta}]&=0\\
[Y_{\alpha},X^{-}]&=-\alpha(X^{-})Y_{\alpha}    &[Y_{\alpha},X_{\beta}]&=Z_{\alpha+\beta}\\
[Y_{\alpha+\beta},X^-]&=-(\alpha+\beta)(X^{-})Y_{\alpha+\beta}=0    &[Y_{\alpha+\beta},X_{\beta}]&=Z_{\alpha+2\beta}\\
[Y_{\alpha+2\beta},X^-]&=-(\alpha+2\beta)(X^{-})Z'_{\alpha+2\beta}  &[Y_{\alpha+2\beta},X_{\beta}]&=0,
\end{align*}
so $N'=-\alpha(X^{-})Y_{\alpha}+Z_{\alpha+\beta}-Z_{\alpha+2\beta}-(\alpha+2\beta)(X^{-})Z'_{\alpha+2\beta}$. Since $\beta(H^{+})=0$, the computation of $[H^{+},N']$, produces terms like $[H^{+},X_{\alpha+\beta}]=(\alpha+\beta)(H^{+})X_{\alpha+\beta}=\alpha(H^{+})X_{\alpha+\beta}$. Therefore, $[H^{+},N']=\alpha(H^{+})N'$ and
\begin{equation}  \label{EqElsInterditsSu}
\begin{split}
  \Ad( e^{H^{+}+Y})(X^{-}+X_{\beta})&=X+N'+\sum_{k\geq1}\frac{1}{ (k+1)! }\alpha(H^{+})^{k}N'\\
        &=X+\frac{  e^{\alpha(H^{+})}-1 }{ \alpha(H^{+}) } N'
\end{split}
\end{equation}
What we have proven is the following result.

\begin{proposition}
The group $\tilde R$ acts transitively on $\mU$ if and only if the Lie algebra $\tilde{\sR}$ does not contain elements of the form
\[
  X+\frac{  e^{\alpha(H^+}-1 }{ \alpha(H^+) }N'
\]
with $X\in\sS$ and $Y\in\sR'$; the element $N'$ being given by
\[
  N'=-\alpha(X^{-})Y_{\alpha}+Z_{\alpha+\beta}-Z_{\alpha+2\beta}-(\alpha+2\beta)(X^{-})Z'_{\alpha+2\beta}
\]
where $X=X^-+X_{\beta}$ is the decomposition of an element of $\sS$ and $Z_{\varphi}$ are elements of their respective root spaces $\sG_{\varphi}$.
\end{proposition}

One can distinguish three case: the first is $X=X^{-}\in\sA^{-}$, the second is $X=X_{\beta}\in\sG_{\beta}$ and the last one is $X=X^{-}+X_{\beta}$ ($X^{-}\neq  0\neq X_{\beta}$).

In the first case, formula \eqref{EqElsInterditsSu} forbids $\tilde{\sR}$ to contains elements of the form
\begin{equation}
  X^{-}+\sG_{\alpha}\oplus_{\alpha+2\beta}.
\end{equation}
The second case forbids elements of the form
\begin{equation}
  X_{\beta}+\sG_{\alpha+\beta}\oplus\sG_{\alpha+2\beta},
\end{equation}
and the third case forbids
\begin{equation}
  X^{-}+X_{\beta}\oplus\sG_{\alpha}\oplus\sG_{\alpha+\beta}\oplus\sG_{\alpha+2\beta}.
\end{equation}
We can extract constraints on the coefficients of algebra \eqref{EqGeneAlgabcd} from that analysis. The third interdiction makes that a linear combination of $J_2$ and $V$ in an element of $\tilde{\sR}$ can only occur in $A$, so that
\[
  bb'=cc'=dd'=0.
\]
The second interdiction says that $B$ and $C$ cannot contain $V$ alone, so $b'=c'=0$. Finally, the first condition imposes $c=d=0$ because $C$ and $D$ cannot contain $J_2$ alone. The remaining constraints for \eqref{EqGeneAlgabcd} to be an algebra are easy to solve by hand. The results are the same two algebras as previously found.

\subsection{Local group structure}
%------------------------------

We saw in proposition~\ref{PropCRunXXX} that the open orbit $\mU$ can be identified with the group generated by the algebra $\{ J_{1},W,M,L \}$.

We want to find an algebra (whose group is) acting transitively on a neighbourhood of $[u]$ in the $AN$ orbit of $[u]$ and which admits a symplectic form. Let $A,B,C,D$ be a basis of this algebra. For local transitivity, each of them must contains at least one of $J_{1},W,M$ and $L$. As in the previous case, the most general algebra to be studied is
\begin{subequations}   \label{EqAlgGEnennsy}
\begin{align}
 A&=J_{1}+aJ_{2}+a'V\\
 B&=W+bJ_{2}+b'V\\
 C&=M+cJ_{2}+c'V\\
 D&=M+dJ_{2}+d'V.
\end{align}
\end{subequations}
Among such algebras, we will have to check surjectivity of the action and the possibility to endow with a symplectic form.

If we impose that $\Span\{ A,B,C,D \}$ is a subalgebra for the bracket inherited from $\mathfrak{so}(2,3)$, we find a lot of conditions on the coefficients $a$, $b$, $c$, $d$, $a'$, $b'$, $c'$ and $d'$. If, for example, we look at $[A,B]$, we find
\[
  [A,B]=W+a'M+ab'V-a'bV.
\]
In this combination, the coefficient of $W$ is $1$ and the one of $M$ is $a'$, so the only possibility is $[A,B]=B+a'C$. This leads to the following conditions (equating coefficients of $J_{2}$ and $V$):
\begin{subequations} \label{SubEqSystemeAlgTN}
\begin{align}
 b+a'c&=0\\
b'+a'c'&=ab'-a'b.
\end{align}
Proceeding in a similar way for the six different commutators, we find:

For $[A,C]$
\begin{align}
ac'-a'c&=(a+1)c'\\
(a+1)c&=0,
\end{align}
for $[A,D]$
\begin{align}
(1-a)d+2a'b&=0\\
ad'-a'd&=2a'b'+(1-a)d',
\end{align}
for $[B,C]$
\begin{align}
(b-c')c&=0\\
(b-c')c'&=bc'-b'c,
\end{align}
for $[B,D]$
\begin{align}
-d'c+2b'b-bd&=0\\
-d'c'-bd'+2(b')^{2}&=bd'-b'd,
\end{align}
for $[C,D]$
\begin{align}
cd'&=c'b'\\
cd&=c'b.
\end{align}
\end{subequations}
Solutions of these equations\footnote{from now until the determination of symplectic forms, all results are computed by Maxima~\cite{Maxima}.}, parametrized by reals $r$ and $s$ and the corresponding commutators are listed below.

 The next step is to determine which of these algebras admit a compatible symplectic structure in the sense of definition~\ref{DefSympleAlg}.
 For this, we just have to consider a general skew-symmetric matrix
\[
  \omega=
\begin{pmatrix}
0&-\alpha&-\beta&-\gamma\\
\alpha&0&-\delta&-\sigma\\
\beta&\delta&0&-\epsilon\\
\gamma&\sigma&\epsilon&0
\end{pmatrix}
\]
 and, for each algebra, solve the four constrains. In the first algebra (see below), we find for example
\[
  \omega_{1}([A,B],C)+\omega_{1}([B,C],A)+\omega_{1}([C,A],B)=-\frac{ -5\omega_{1}(C,B) }{ 2 }\stackrel{!}{=}0,
\]
so that $\omega_{1}(C,B)=0$. Full results are listed below (the symplectic matrices are written in the basis $\{ A,B,C,D \}$).  We see in particular that most of the solutions reduce to the \emph{canonical algebra}, $\sR_c$ given by
\begin{align*}
[a,b]&=b
&[a,c]&=2c
&[c,d]&=c.
\end{align*}
Let us compute the metric on these groups too. Let $G_{L}$ be one of these groups (the groups corresponding to algebras one to seven, see below). We have a diffeomorphism between $G_{L}$ and the orbit $\mU$ of $[u]\in AdS_4$,
\begin{equation}
\begin{aligned}
 \phi\colon G_{L}&\to \mU \\
r&\mapsto [ru].
\end{aligned}
\end{equation}
We define the metric on $G_{L}$ at $e$ by
\begin{equation}   \label{EqgeABBphi}
  g_{e}(A,B):=B_{\phi(e)}\big( d\phi(A),d\phi(B) \big)
\end{equation}
where
\[
  d\phi(A)=\Dsdd{ \phi\big( A(t) \big) }{t}{0}=\Dsdd{ \big( \pi\circ R_{u} \big)\big( A(t) \big) }{t}{0}=d\pi\,dR_{u}A.
\]
By definition of the metric on the cosets (cf equation \eqref{eq:scal_TgM}) we have
\begin{align*}
B_{[g]}(d\pi X,d\pi Y)&=B_{g}(\pr X,\pr Y)\\
        &=B_{e}(dL_{g^{-1}}\pr X,dL_{g^{-1}}\pr Y)
\end{align*}
Using that in equation \eqref{EqgeABBphi}, we find ($e$ is the neutral of $G_{L}$):
 \[
\begin{split}
  B_{e}(A,B)&=B_{[u]}(d\pi dR_{u}A,d\pi dR_{u}B)\\
        &=B_{u}(\pr dR_{u}A,\pr dR_{u}B)\\
        &=B_{e}(dL_{u^{-1}}\pr_{\sQ_{u}}dR_{u}A,dL_{u^{-1}}\pr_{\sQ_{u}}dR_{u}B)
\end{split}
\]
where $\pr$ is the projection on $\sQ_{u}=dL_{u}\sQ$. The latter formula allows to easily compute the metric on the various groups.

\let\ANCtheenumi\theenumi
 \renewcommand{\theenumi}{\arabic{enumi}.}
\begin{enumerate}
\item As first solution, we find of course the same algebra $\sR_1$ as the one of page \pageref{PgAlgUn}.
\item The second solution is also the same as the previously found one.
\item The third solution is
\begin{align*}
A&=J_{1}+J_{2}+sV       &[A,B]&=B+sC\\
B&=W-\frac{ r }{2}V     &[A,C]&=2C\\
C&=M                &[A,D]&=2sB\\
D&=L+rJ_{2}         &[B,D]&=-rB\\
 &              &[C,D]&=-rC,
\end{align*}
\begin{equation}
\omega=\begin{pmatrix}
0   &-\alpha        &-\beta &-\gamma\\
\alpha  &0          &0  &\frac{ \beta rs-2\alpha r  }{ 2 }\\
\beta   &0          &0  &\frac{ \beta r }{2}\\
\gamma  &-\frac{ \beta rs-2\alpha r  }{ 2 } &-\frac{ \beta r }{2}   &0
\end{pmatrix},
\end{equation}
\[
  \det\omega=\frac{ \beta^{4}r^{2}s^{2} }{ 4 }-\frac{ \alpha\beta^{3}r^{2}s }{ 2 }+\frac{ \alpha^{2}\beta^{2}r^{2} }{ 4 },
\]
Conditions: $r\neq 0$, $\beta\neq 0$ and $\alpha\neq\beta r$. The map $\phi_3\colon \sR_3\to \sR_c$
\[
  \phi_3=
\begin{pmatrix}
1   &   0   &   0   &   r\\
0   &   1/2s    &   0   &   1\\
0   &   s   &   1   &   s^2\\
0   &   0   &   0   &   r
\end{pmatrix}.
\]
($\det\phi_3=r/2s$) provides an isomorphism between $\sR_3$ and the canonical algebra.
\item The fourth solution is
\begin{align*}
A&=J_{1}+J_{2}+sV       &[A,B]&=B+sC\\
B&=W                &[A,C]&=2C\\
C&=M                &[A,D]&=2sB\\
D&=L+rJ_{2}+rsV         &[B,D]&=-rsC\\
 &              &[C,D]&=-rC,
 \end{align*}
\begin{equation}
\omega=\begin{pmatrix}
0   &-\alpha        &-\beta &-\gamma\\
\alpha  &0          &0  &\frac{ \beta r s  }{ 2 }\\
\beta   &0          &0  &\frac{ \beta r }{2}\\
\gamma  &-\frac{ \beta r s  }{ 2 }  &-\frac{ \beta r }{2}   &0
\end{pmatrix},
\end{equation}
\[
\det\omega=\frac{ \beta^{4}r^{2}s^{2} }{ 4 }-\frac{ \alpha\beta^{3}r^{2}s }{ 2 }+\frac{ \alpha^{2}\beta^{2}r^{2} }{ 4 }.
\]
Conditions:  $r\neq 0$, $\beta\neq 0$ and $\alpha\neq\beta s$.  The map $\phi_4\colon \sR_4\to \sR_c$
\[
  \phi_{4}=
\begin{pmatrix}
2&0&0&-r\\
0&1&1/s&rs\\
0&1&0&2rs\\
3&0&0&-r
\end{pmatrix}
\]
($\det\phi_4=-r/s$)  provides an isomorphism between $\sR_{4}$ and the canonical algebra.
\item The fifth solution is
\begin{align*}
A&=J_{1}-J_{2}+sV       &[A,B]&=B+sC\\
B&=W-rsJ_{2}+rs^{2}V        &[A,D]&=2sB+2D\\
C&=M+rJ_{2}-rsV         &[B,D]&=2rs^{2}B+rs^{3}C+rsD\\
D&=L+rs^{2}J_{2}-rs^{3}V    &[C,D]&=-2rsB-rs^{2}C-rD,\\
\end{align*}
\begin{equation}
\omega=\begin{pmatrix}
0&-\alpha&-\beta&\frac{ \beta r s^{2}+2\alpha r s+2\epsilon }{ r }\\
\alpha&0&0&\epsilon s\\
\beta&0&0&-\epsilon\\
-\frac{ \beta r s^{2}+2\alpha r s+2\epsilon }{ r }&-\epsilon s&\epsilon&0
\end{pmatrix}
\end{equation}
\begin{equation}
\det\omega=\beta^{2}\epsilon^{2}s^{2}+2\alpha\beta\epsilon^{2}s+\alpha^{2}\epsilon^{2},
\end{equation}
Conditions:  $r\neq 0$, $\epsilon\neq 0$, $\alpha\neq -\beta s$.  The map $\phi_5\colon \sR_5\to \sR_c$
\begin{equation}
\phi_{5}=
\begin{pmatrix}
1&0&0&0\\
0&1&0&2s\\
0&0&0&-1\\
0&-rs&r&rs^{2}
\end{pmatrix}
\end{equation}
($\det\phi_{5}=r$) provides an isomorphism between $\sR_{5}$ and the canonical algebra.
\item The sixth solution is
\begin{align*}
A&=J_{1}+J_{2}      &[A,B]&=B\\
B&=W            &[A,C]&=2C\\
C&=M            &[C,D]&=-rC,\\
D&=L+rJ_{2}
\end{align*}
\begin{equation}
\omega=\begin{pmatrix}
0   &-\alpha        &-\beta &-\gamma\\
\alpha  &0          &0  &0\\
\beta   &0          &0  &\frac{ \beta r }{2}\\
\gamma  &0          &-\frac{ \beta r }{2}   &0
\end{pmatrix},
\end{equation}
\begin{equation}
\det\omega=\left( \frac{ \alpha\beta r }{ 2 } \right)^2,
\end{equation}
Conditions: $\alpha\neq 0$, $\beta\neq 0$ and $r\neq 0$. This algebra is isomorphic to the next one.
\item The seventh solution is
\begin{align*}
A&=J_{1}-J_{2}      &[A,B]&=B\\
B&=W            &[A,D]&=2D\\
C&=M+rJ_{2}     &[C,D]&=-rD,\\
D&=L
\end{align*}
\begin{equation}
\omega=\begin{pmatrix}
0   &-\alpha        &-\beta &-\gamma\\
\alpha  &0          &0  &0\\
\beta   &0          &0  &\frac{ \gamma r }{2}\\
\gamma  &0          &-\frac{ \gamma r }{2}  &0
\end{pmatrix},
\end{equation}
\begin{equation}
\det\omega=\pm\left( \frac{ \alpha\gamma r }{ 2 } \right),
\end{equation}
and the conditions are $\alpha\neq 0$, $\gamma\neq 0$, $r\neq 0$.  The map $\phi_7\colon \sR_7\to \sR_c$
\[
  \phi_{7}=
\begin{pmatrix}
1&0&0&0\\
0&1&0&0\\
0&0&0&1\\
0&0&r&0
\end{pmatrix}
\]
($\det\phi_{7}=-r$) provides an isomorphism between $\sR_{7}$ and the canonical algebra.
\end{enumerate}
\let\ANCtheenumi\theenumi
All these algebras are solvable of order two (the commutators of commutators vanish) --- but not nilpotent.

\begin{proposition}
Surjectivity of the action of the group $R_3$.
\end{proposition}

\begin{proof}
The Lie algebra $\sR_3$ can be written as a sequence of split extensions
\[
  \sR_3=\eR A\oplus_{\ad}\eR D\oplus_{\ad}\Span\{ B,C \},
\]
thus a general element of $R_3$ reads
\begin{equation}  \label{Eqelgener31}
r_3(a,b,c,d)= e^{aJ_1+aJ_2+asV} e^{dL+drJ_2} e^{bW-\frac{ br }{2}V} e^{cM}.
\end {equation}
When $[X,Y]=sY$, one has the formula
\begin{equation}
 \ln(e^{X} e^{Y})=X+\frac{ 1- e^{s} }{ s }Y.
\end{equation}
This allows us to write
\[
  \ln( e^{xJ_2} e^{yL})= xJ_2-\frac{ 1- e^{-x} }{ x }yL,
\]
and to conclude that we can make the splitting
\[
   e^{dL+drJ_2}= e^{drJ_2} e^{f(d)L}
\]
where $f(d)=d^{2}r/( e^{-dr}-1)$. The general element \eqref{Eqelgener31} becomes
\[
  r_3(a,b,c,d)= e^{aJ_2+asV} e^{drJ_2} e^{aJ_1} e^{f(d)L} e^{bW-brV/2} e^{cM}.
\]
Up to redefinition of $c$, we can split $ e^{bW-brV/2}$ into $ e^{bW} e^{-brV/2}$, and a new redefinition allows us to commute $ e^{-brV/2}$ and $ e^{bW}$. Up to new redefinitions we are left with
\[
  r_3(a,b,c,d)=s(a,b,d) e^{aJ_1} e^{f(d)L} e^{bW} e^{cM}
\]
where $s(a,b,d)$ is an element of $S$ which depends on $a$, $b$ and $d$. So a general element of $R_3[u]$ is
\begin{equation}
  [ e^{cM} e^{bW} e^{-f(d)L} e^{aJ_1}u]
\end{equation}
where $f$ fails to be surjective.
\end{proof}

%%%%%%%%%%%%%%%%%%%%%%%%%%
%
   \section{Deformation of \texorpdfstring{$AdS_4$}{AdS4}}
%
%%%%%%%%%%%%%%%%%%%%%%%%


\begin{abstract}
Ceci contient les rebuts de démonstrations et de choses non retenues pour la ligne droite de ma thèse.
\end{abstract}

A description of the black hole construction is given in section~\ref{SecCausal}. As far as Iwasawa decomposition is concerned, we recall that we fix choices in such a way that we have
\begin{subequations}
\begin{align}
\sN&=\{W_i,V_j,M,L\}&i=5,\ldots, l+1\\
\sA&=\{J_1,J_2\}
\end{align}
\end{subequations}
 with the commutator table
\begin{subequations} % \label{EqTableSOIwa}
\begin{align}
[V_i,W_j]&=\delta_{ij}M &[V_j,L]&=2W_j\\
[J_1,W_j]&=W_j       &[J_2,V_i]&=V_i\\
[J_1,L]&=L           &[J_2,L]&=-L\\
[J_1,M]&=M           &[J_2,M]&=M
\end{align}
\end{subequations}
  where $\sN$ is the nilpotent part of $\SO(2,l-1)$ and $\sA$ the abelian one. Notice that $W,J_{1}\in\sH$, and $J_{2}\in\sQ$.

\begin{proposition}\label{PropRsurSglobgroup}
The homogeneous space $\mU=R/S$ is globally of group type.
\end{proposition}

\begin{proof}
The fact that $J_{2}$ and $V$ do not act on $[u]$ makes that \emph{locally} the group corresponding to the algebra $\Span\{ J_{1},W,M,L \}$ is diffeomorphic to $\mU$. It is however not clear that the action will be \emph{globally} transitive.

We know that $u=\exp q_0$ and
\begin{subequations}
\begin{align}
[q_0,[q_0,J_{2}]]&=\left( \frac{ \pi }{2} \right)^{2}J_{2},         \label{EqqqJ2a} \\
\pr_{\sQ}[q_0,[q_0,V]]&=\left( \frac{ \pi }{2} \right)^{2}\pr_{\sQ}V,       \label{EqqqVb}
\end{align}
\end{subequations}
Now, $r$ belongs to $S$ when there exists a $h\in H$ such that $ru=uh$, or when $\AD(u)r\in H$. Thus we are lead to compute thinks like
\[
  \AD( e^{q_0})r
\]
and check for which $r$ this belongs to $H$. We will prove that this condition is satisfied for $ e^{aJ_{2}}$ and $ e^{bV}$. We have to study the $\sQ$-component of $ e^{\ad(q_0)X}$, and $ e^{X}$ will belong to $S$ when this component vanishes. Since $J_2\in\sQ$, we see that, in the expansion of $ e^{\ad(q_0)}J_2$, the even terms are in $\sQ$ while the odd ones are in $\sH$. Now equation \eqref{EqqqJ2a} makes that the odd terms provide the expansion of $\cos(\pi/2)=0$.

For $V$, we remark that it has a component $V_{\sQ}$ and $V_{\sH}$, the fact that $[q_0,V]\in\sH$ makes that once again the odd terms does not pose any problem. Equation \eqref{EqqqVb} gives the expansion of $\cos(\pi/2)$ for the even terms.

So the stabilizer of $[u]$ is the group generated by $ e^{aJ_2}$ and $ e^{bV}$. We denote by $\mS$ the Lie algebra:
\[
  \mS=\Span\{ J_2,V \}.
\]
 The stabilizer cannot be larger because $R$ has $6$ dimensions while $\mU$ has $4$ dimensions. Now we consider
\[
  \sR'=\Span\{ J_1,W,M,L \},
\]
and the first important remark is that it is an algebra. So it is immediately clear that locally, $\mU$ has the structure of the corresponding group $R'$. Let us prove that the action of $R'$ on $\mU$ is globally transitive, i.e. $R'[u]=R[u]$. For that, just remark that $\sS$ acts on $\sR'$, so
\[
  \sR=\sS\oplus_{\ad}\sR'
\]
and, as far as groups are concerned, we conclude that $R=SR'$ and hence that $R=R'S$. This proves that $R'$ acts transitively on $\mU$. We will see later that this group is not satisfactory because it does not posse a symplectic form.

We know that $R'$ acts on $\mU=R/S$ freely because none of the elements of $R'$ acts as identity on $[u]$. This proves that $\mU$ is locally of group type. It is also globally of group type because $\mU=R'[u]$ is only the orbit of $[u]$.

\end{proof}

We try to deform the open orbit of $AdS_4$ seen as the group of the first algebra (see page  \pageref{PgAlgUn}). We rewrite this algebra, that we name $\mR_{1}$, as
\begin{subequations}   \label{SubEqTablealgun}
\begin{align}
  [p,q]&=-rE&r\neq 0\\
[A,p]&=p+sE\\
[A,E]&=\frac{ 3 }{2}E\\
[A,p]&=2sp+\frac{ 1 }{2}q,
\end{align}
\end{subequations}
because we are going to see it as a split extension of the Heisenberg algebra spanned by $\{ p,q,E \}$. The first work is to identity the parameters $(d\BX,d\mu,2d)$ of the derivation which extends the Heisenberg algebra to $\mR_{1}$. The condition is that the commutator
\[
[  aA+v_1p+v_2q+zE,a'A+v_1'p+v_2'q+z'E  ]_{(d\BX,d\mu,2d)}
\]
seen as split extension (equation \eqref{EqGeneExtHeizCom})
\[
\begin{split}
d\BX\big( a(v_1'p-v_2'q)-a'(v_1p-v_2q) \big)&+ d\mu\big( a(v_1'p+v_2'q)-a'(v_1p+v_2q) \big)E\\
                    &+2d(az'-a'z)E\\
                    &+\Omega(v_1p+v_2q,v_1'p+v_2'q) E
\end{split}
\]
equals the same commutator given by the table \eqref{SubEqTablealgun}:
\begin{align*}
  &p\big(av_1'+2sav_2'-a'v_1-2a'v_2s\big)+q\big( \frac{ a }{2}v_2'-\frac{ a' }{2}v_2 \big)\\
&+E\big( sav_1'+\frac{ 3a }{2}z'-a'v_1s-v_1v_2'r+v_2v_1'r-\frac{ 3a' }{2}z \big).
\end{align*}
Equating the terms with $z$ in the coefficient of $E$, we find
\begin{equation}
d=3/4.
\end{equation}
The terms with $p$ and $q$ give the matrix $\BX$ by
\[
  d\BX\begin{pmatrix}
av_1'-a'v_1\\
av_2'-a'v_2
\end{pmatrix}=
\begin{pmatrix}
av_1'-a'v_1+2s(av_2'-a'v_2)\\
\frac{ 1 }{2}(av_2'-a'v_2)
\end{pmatrix},
\]
from what it follows that
\begin{equation}
\BX=\begin{pmatrix}
4/3&8s/3\\
0&2/3
\end{pmatrix}.
\end{equation}
When we pose $v_2=v_2'$, we find
\begin{equation}
\mu(p)=4s/3
\end{equation}
and with $v_1=v_1'=0$, we find
\begin{equation}
\mu(q)=0.
\end{equation}
The symplectic form on $V=\Span\{ p,q \}$ is
\begin{equation}
 \Omega=\begin{pmatrix}
0&-r\\
r&0
\end{pmatrix}.
\end{equation}
One can check that the condition $\Omega(\id-\BX)+(\id-\BX)^{t}\Omega=0$ is satisfied, so that $\bar{\BX}\in\gsp(V,\Omega)$. Therefore, the algebra $\mR_{1}$ is a split extension of $\pH_{n}$ in the sense of section~\ref{SecExtHeiz}. The symplectic form on $\mR_1$ is given by proposition \eqref{PropSymplestarEG}:
 \begin{equation}
\omega_{1}=\begin{pmatrix}
0   &-\alpha        &-\beta &-\gamma\\
\alpha  &0          &0  &\frac{ 2\beta r }{ 3 }\\
\beta   &0          &0  &0\\
\gamma  &-\frac{ 2\beta r }{ 3 }    &0  &0
\end{pmatrix},
\end{equation}
and the symplectic form on $\mR_1$ seen as extension of Heisenberg algebra is given by equation \eqref{PropSymplestarEG} which in the present case reads, in the basis $\{ A,p,E,q \}$,
\begin{equation}
-\delta E^*=
\begin{pmatrix}
0&s&3/2&0\\
-s&0&0&-r\\
-3/2&0&0&0\\
0&r&0&0
\end{pmatrix}.
\end{equation}
We see that the latter is a particular case of the general one with $\beta=-3/2$, $\alpha=-s$ and $\gamma=0$. In order to prove that $\omega_{1}$ is a Chevalley coboundary for the trivial representation, we search for an element $\xi\in\mR_1$ such that $\delta\xi^*=\omega_{1}$, see equation \eqref{EqDefcochaintrivC}. We pose
\[
  \xi=\xi_{A}A+\xi_{p}p+\xi_{E}E+\xi_{q}q
\]

\begin{probleme}
\emph{\ldots we search for an element\ldots} est-ce qu'il faut vraiment le \emph{for}??
\end{probleme}

and we impose that
\begin{equation}
  \delta\xi^*(X,Y)=-\xi^*([X,Y])=\omega_{1}(X,Y).
\end{equation}
A resolution with Maxima (see appendix~\ref{SecAppMaxima}) gives the result
\begin{equation}  \label{EqSolxialgun}
\xi_{p}=\frac{ 2\beta s-3\alpha }{ 3 },\quad\xi_{E}=-\frac{ 2\beta }{ 3 },\quad\xi_{q}=-\frac{ 6\gamma+8 \beta s^{2}-12\alpha s }{ 3 },
\end{equation}
while $\xi_{A}$ has no constrains\footnote{This is clear because $A$ does not appear in the commutators.}. This proves that $\omega_{1}$ is a Chevalley $2$-coboundary.

A problem is now to adapt the results of theorem~\ref{ThoDefHeizsansB} and~\ref{ThoDefoHeizAvecB} to the new symplectic form which is not the one of proposition~\ref{PropSymplestarEG}.

First, we fix the element $g\in\mR_1$ such that \label{PgAdgXEbbekl}
\begin{equation}
(\Ad(g)^*\delta\xi^*)(X)=E^*(X),\quad\text{or}\quad  \xi^*(\Ad(g)X)=E^*(X).
\end{equation}
We search $g$ under the form
\[
  g= e^{aA} e^{v_1p+v_2q+zE}
\]
  Let us begin with $X=E$: $\xi^*(\Ad(g)E)=1$. Using the commutation relations, $Ad(g)E= e^{\ad(aA)}E= e^{3a/2}E$. It fixes the value of $a$:
\begin{equation}   \label{Eqleadegxistar}
 a=-\frac{ 2 }{ 3 }\ln\big( -\frac{ 2\beta }{ 3 } \big),
\end{equation}
if $\beta<0$. The parameters $v_1$ and $v_2$ are more difficult:
\[
\begin{split}
  \Ad(g)p&= e^{\ad(aA)}(p+rE)\\
    &= e^{\ad(aA)}p+rv_2 e^{3a/2}E
\end{split}
\]
The first terms of $ e^{\ad(aA)}p$ are
\begin{subequations}  \label{EqPremtermsAdgp}
\begin{align}
p\\
a(p+sE)\\
\frac{ a^{2} }{2}(p+sE+s\frac{ 3 }{2}E)&=\frac{ a^{2} }{2}p+\frac{ a^{2} }{2}\big( 1+\frac{ 3 }{2} \big)sE\\
\frac{ a^{3} }{ 3! }(p+sE+s\frac{ 3 }{2}E)+\frac{ a^{3} }{ 3! }s\frac{ 3 }{2}E&=\frac{ a^{3} }{ 3! }p+\frac{ a^{3} }{ 3 }\big( 1+\frac{ 3 }{2}+\frac{ 3 }{2}\frac{ 3 }{2} \big)sE.
\end{align}
\end{subequations}
At each step, the first term gives the expansion of $ e^{a}p$ and the term with $sE$ of $[A,p]$ can be regrouped with the other to give, in the term of $a^{n}$,
\[
  \big( 1+\frac{ 3 }{ 2 }+\cdots+\big( \frac{ 3 }{ 2 } \big)^{n-1} \big)sE.
\]
Using the well know sum formula $S_{n}(x)=1+x+\cdots+x^{n}=\frac{\displaystyle x^{n+1}-1 }{\displaystyle x-1 }$, the term with $E$ in $ e^{\ad(aA)}p$ is
\[
  \frac{ a^{n} }{ n! }S_{n-1}\big(\frac{ 3 }{2}\big)sE=\frac{ a^{n}-1 }{ x-1 }
\]
where $x=3/2$. Putting all terms together,
\[
\begin{split}
 e^{\ad(aA)}p&= e^{a}p+2\sum_{n=0}^{\infty}\frac{ a^{n} }{ n! }x^{n}sE-2\sum_{n=0}^{\infty}\frac{ a^{n} }{ n! }sE\\
    &= e^{a}p+2 e^{3a/2}sE-2 e^{a}sE.
\end{split}
\]
Computations with Maxima give:
\[
\begin{split}
\Ad(g)p=gpg^{-1}= e^{a}p+2 e^{3a/2}sE-2 e^{a}sE+v_2r e^{3a/2}E,
\end{split}
\]
which must satisfy $\xi^*\big( \Ad(g)p \big)=E^*(p)=0$. So $v_2$ is given by the equation
\begin{equation}
 e^{a}\xi_{p}+2s\xi_{E}\big(  e^{3a/2}- e^{a} \big)+\xi_{E}v_2r e^{3a/2}=0
\end{equation}
where $a$ is given by equation \eqref{Eqleadegxistar}. For $\Ad(g)q$, we find that $v_1$ is solution of
\begin{equation}
4s\xi_{p}( e^{a}- e^{a/2})+\xi_{E}\big( - e^{3a/2}rv_1+4 e^{3a/2}s^{2}-8 e^{a}s^{2}+4 e^{a/2}s^{2} \big)+\xi_{q} e^{a/2}=0.
\end{equation}

\subsection{Isomorphism with \texorpdfstring{$\sR$}{R}}
%----------------------------------

Let $\pH_{1}=V\oplus\eR E$ be the Heisenberg algebra and its extension $\mF(\mtu,0,2)$, i.e.
\[
  \ad(A)=\begin{pmatrix}
1\\&1\\&&2
\end{pmatrix}
\]
on $\pH_{1}$. We extend $\mF$ by a two dimensional algebra $\mE=\Span\{ E_{1},E_{2} \}$ with
\[
  [E_{1},E_{2}]=2E_{2}
\]
and
\begin{equation}
\ad(E_{1})\begin{pmatrix}
p\\q
\end{pmatrix}=\begin{pmatrix}
p\\-q
\end{pmatrix},
\quad
\ad(E_{2})\begin{pmatrix}
p\\q
\end{pmatrix}
=\begin{pmatrix}
q\\0
\end{pmatrix}.
\end{equation}
All this produces a new algebra
\[
  \mE\oplus_{\rho}\mF=\Span\{ E_{1},E_{2},A,p,q,E \}
\]
whose full commutator table is
\begin{align*}
[E_{1},E_{2}]&=2E_{2}&[A,p]&=p&[E_{2},q]=p      \\
[E_{1},p]&=p&[A,p]&=p&[p,q]=E           \\
[E_{1},q]&=-q&[A,E]&=2E
\end{align*}
One can check that the following is an isomorphism $\alpha\colon \mE\oplus_{\rho}\mF\to \sR$:
\begin{subequations}\label{EqDefalphaisomRF}
\begin{align}
\alpha(E_{1})&=J_{1}-J_{2}&\alpha(E_{2})&=L/2\\
\alpha(A)&=J_{1}+J_{2}&\alpha(p)&=W\\
\alpha(q)&=-V&\alpha(E)&=M.
\end{align}
\end{subequations}
Trough this isomorphism, we say that $\sR=\mE\oplus_{\rho}\mF$. The projection $\sR\to\mF$ is given by $E_{1}\to 0$, $E_{2}\to 0$ or, trough $\alpha$, $L\to 0$ and $J_{1}-J_{2}\to 0$. After projection, the table \eqref{EqTableSOIwa} becomes
\begin{subequations}
\begin{align}
[V,W]&=M
&[H_{2},W]&=W\\
[H_{2},V]&=V
&[H_{2},M]&=2M,
\end{align}
\end{subequations}
which is the table of $\mF(\mtu,0,2)$. Projections of algebras $\mR_{j}$ are:

\subsubsection{Algebra 1}

\begin{align*}
A&\to \frac{ 3 }{4}H_{2}+rV\\
B&\to W\\
C&\to M\\
D&\to sV.
\end{align*}
Not surjective on $\mF$ when $s=0$.

\subsubsection{Algebra 3}

\begin{align*}
A&\to H_{2}+sV\\
B&\to W-\frac{ r }{ 2 }V\\
C&\to M\\
D&\to \frac{ r }{ 2 }H_{2}.
\end{align*}
Not surjective on $\mF$ when $s=0$.

\subsubsection{Algebra 4}

\begin{align*}
A&\to H_{2}+sV\\
B&\to W\\
C&\to M\\
D&\to \frac{ r }{ 2 }H_{2}+rsV.
\end{align*}
Not surjective on $\mF$ when $s=0$.

\subsubsection{Algebra 5}

\begin{align*}
A&\to sV\\
B&\to W-\frac{ rs }{ 2 }H_{2}+rs^{2}V\\
C&\to M+\frac{ r }{2}-rsV\\
D&\to \frac{ rs^{2} }{ 2 }H_{2}-rs^{3}V.
\end{align*}
Not surjective on $\mF$ when $s=0$.

The pathologic cases are $\mR_{3}$, $\mR_{4}$, $\mR_{5}$ and $\mR_{6}$ with $s=0$.  They are each isomorphic to the algebra generated by $\{ A,B,C,D \}$ and subject to the relations
\begin{align*}
[A,B]&=B\\
[A,C]&=2C\\
[C,D]&=-rC.
\end{align*}



We consider the map
\begin{equation}
\begin{aligned}
 \varphi_j\colon R_j&\to F(\mtu,0,2) \\
\varphi_j(r_{j})&= \AD(r_{j1})r_{j0}
\end{aligned}
\end{equation}
where $r_{j0}$ and $r_{j1}$ are defined by the condition $\alpha(r_{j1}r_{j0})=r_{j}$. This map actually takes its values in $F$ because, by construction, $\ad(\mE)\mF\subset\mF$.

\begin{theorem}
The map $\varphi_j$ is bijective.
\end{theorem}

\begin{proof}
Let us first assume that $\varphi_j(r_{j})=\varphi_j(s_{j})$. In this case, we have
\[
  r_{j1}r_{j0}r_{j1}^{-1}=s_{j1}s_{j0}s_{j1}^{-1},
\]
but $r_{j1}r_{j0}=\alpha^{-1}(r_{j})$ and $s_{j1}s_{j0}=\alpha^{-1}(s_{j})$, thus
\[
  \alpha^{-1}(s_{j}^{-1}r_{j})=s_{j1}^{-1}r_{j1},
\]
or
\[
  r_{j}s_{j}^{-1}=\alpha(s_{j1}^{-1}r_{j1})
\]
The left-hand side of the latter equation is an element of $R_{j}$ while the right-hand side is an element of $\alpha(E)$, i.e. of the group generated by $J_{1}-J_{2}$ and $L$. One can check that the intersection between $\mR_{j}$ and $\mE$ reduces to $\{ 0 \}$, so $r_{j}=s_{j}$. This proves injectivity.
\end{proof}

\subsection{The symplectic issue}
%-------------------------------

When one pull back the kernel from $F(\mtu,0,2)$ to $R_{j}$, there is no reason why the symplectic form $\delta E^*$ on $F$ maps to the general symplectic form on $R_j$. The issue is to push forward the symplectic form $\omega_{j}$ of $R_j$ to $F$ by the diffeomorphism $\varphi_j$ and find a kernel on $F$ for this new symplectic form instead if the canonical one.

We have the following two maps:
\[
\begin{split}
\varphi_j&\colon R_j\to F(\mtu,0,2)\\
d\alpha&\colon \mE\oplus_{\rho}\mF\to \sR
\end{split}
\]
where $d\alpha$ is the isomorphism \eqref{EqDefalphaisomRF}.
The new symplectic form on $F$ is defined by
\begin{equation}
\Omega_{j}(X_0,X_0')=\omega_{j}\big( d\varphi_j^{-1}X_0,d\varphi_j^{-1}X_0' \big)
\end{equation}
with $X_0$, $X_0'\in\mF\subset\mE\oplus_{\rho}\sR$. First, remark that when $X_1\in\mE$ and $X_0\in\mF$, we have
\begin{equation}
\Dsdd{ \alpha\big(  e^{tX_1} e^{tX_0} \big) }{t}{0}=d\alpha(X_1+X_0)
\end{equation}
which is surjective on $d\alpha(\mE\oplus_{\rho}\mF)$. Second, $d\varphi_j$ is the projection because
\begin{align*}
d\varphi_j\left( \Dsdd{ \alpha( e^{tX_1} e^{tX_0}) }{t}{0} \right)&=\Dsdd{ \AD( e^{tX_1} e^{tX_0}) }{t}{0}\\
    &=X_0.
\end{align*}
We conclude that
\[
  d\varphi_j^{-1}(X_0)=d\alpha(X_1+X_0)\in\sR_j
\]
where the ``$\in\sR_j$'' is part of the formula: this fixes the choice of $X_1\in\mE$.


Let us compute for example $d\varphi_1^{-1}(A)$. Using the value of $d\alpha(A)$, we find
\[
  d\varphi_1^{-1}(A)=d\alpha(X_1)+J_{1}+J_{2}
\]
where $X_1\in\mE$ must be chosen in such a way that the whole belongs to $\sR_{1}$.

Let us for example compute $d\varphi^{-1}_{1}(A)=d\alpha(X_1)+J_{1}+J_{2}\in\sR_{1}$. In order to fix $X_1$, we we suppose $X_1=xE_{1}+yE_{2}$, we write
\[
  d\alpha(X_1)=xJ_{1}-xJ_{2}+\frac{ y }{2}L
\]
and we solve
\[
  d\alpha(X_1)+J_{1}+J_{2}=aA+bB+cC+dD
\]
with respect to $x,y,a,b,c,d$. This is a linear system of $6$ equations for $6$ variables. The solution is $b=c=0$, $x=1/3$, $a=4/3$, $d=-4rs/3$ and $y=-8rs/3$. So we have
\begin{equation}
d\varphi_{1}^{-1}(A)=\frac{ 4 }{ 3 }J_{1}+\frac{ 2 }{ 3 }J_{2}-\frac{ 4rs }{ 3 }L=\frac{ 4 }{ 3 }A-\frac{ 4rs }{ 3 }D.
\end{equation}


\section{Isospectral deformations of \texorpdfstring{$M$}{M}}
%-------------------------------------------

In this section, we present a modified version of the oscillatory integral product \eqref{PRODUCT} leading to a left invariant associative algebra structure on the space of square integrable functions on $\SUR_0$. Why is it better that the initial product defined over smooth compactly supported functions? The motivation of considering the square integrable functions is the fact that the spectral triple defined in noncommutative geometry contains the space of square integrable spinors (see the book \cite{ConnesNCG}). The fact to stabilize the space of square integrable functions is then an indispensable step in order to put our results in the framework of spectral geometry.

\begin{theorem}

 Let $u$ and $v$ be smooth compactly supported functions on $R_0$. Define the following three-point functions:
 \begin{equation}
\begin{split}
S:=& S_V\big(\cosh(a_1-a_2)x_0, \cosh(a_2-a_0)x_1, \cosh(a_0-a_1)x_2\big)\\
&-\bigoplus_{0,1,2}\sinh\big(2(a_0-a_1)\big)z_2;
\end{split}
\end{equation}
and
\[
\begin{split}
A:= \Big[&\cosh\big(2(a_1-a_2)\big)\cosh(2(a_2- a_0))\cosh(2(a_0-a_1))\\
& \big[\cosh(a_1-a_2)\cosh(a_2- a_0)\cosh(a_0-a_1)\big]^{\dim R_0-2}\Big]^{\frac{1}{2}}.
\end{split}
\]
Then the formula
\begin{equation}\label{HILB}
u\star^{(2)}_\theta v:=\frac{1}{\theta^{\dim R_0}}
\int_{R_0\times R_0}Ae^{\frac{2i}{\theta}S}u\otimes v
\end{equation}
extends to $L^2(R_0)$ as a left invariant associative Hilbert algebra structure. In particular, one has the strong closedness property:
\begin{equation*}
\int u\star^{(2)}_\theta v=\int uv.
\end{equation*}
\label{thmL2}
\end{theorem}
\begin{proof}
The oscillatory integral product \eqref{PRODUCT} may be obtained by intertwining the Weyl product on the Schwartz space $\swS$ (in the Darboux global coordinates \eqref{DARBOUX}) by the following integral operator \cite{Biel-Massar}:
\begin{equation*}
\tau:=F ^{-1}\circ(\phi_\theta^{-1})^\star\circ F,
\end{equation*}
$F $ being the partial Fourier transform with respect to        the central variable $z$:
\begin{equation*}
F (u)(a,x,\xi):=\int e^{-i\xi z}u(a,x,z){\rm d}z;
\end{equation*}
and $\phi_\theta$ the one parameter family of diffeomorphism(s):
\begin{equation*}
\phi_\theta(a,x,\xi)=(a,\frac{1}{\cosh(\frac{\theta}{2}\xi)}x,
\frac{1}{\theta}\sinh(\theta\xi)).
\end{equation*}
Set $\mJ:=|(\phi^{-1})^\star\mbox{Jac}_\phi|^{-\frac{1}{2}}$ and observe that for all $u\in C^\infty\cap L^2$, the function $\mJ\,(\phi^{-1})^\star u$ belongs to $L^2$.  Indeed, one has
\begin{equation*}
\int|\mJ\,(\phi^{-1})^\star u|^2=\int|\phi^\star \mJ|^2\,|\mbox{Jac}_\phi|\,|u|^2=\int|u|^2.
\end{equation*}
Therefore, a standard density argument yields the following isometry:
\begin{equation*}
T_\theta:L^2(R_0)\longrightarrow L^2(R_0):
u\mapsto F ^{-1}\circ m_\mJ\circ(\phi^{-1})^\star\circ F (u),
\end{equation*}
where  $m_{\mJ}$ denotes the multiplication by $\mJ$.  Observing that $T_\theta=F ^{-1}\circ m_{\mJ}\circ F \circ\tau$, one has $\star^{(2)}_\theta=F ^{-1}\circ m_{\mJ}\circ F (\star_\theta)$.  A straightforward computation (similar to the one in \cite{StrictSolvableSym})     then yields the announced formula.
\end{proof}

Let us point out two facts with respect to the above formulas:
\begin{enumerate}
\item The oscillating three-point kernel $A\exp \big (\frac{2i}{\theta}\,S\big)$ is symmetric under cyclic permutations.
\item The above oscillating integral formula gives rise to a strongly closed, symmetry invariant, formal star product on the symplectic symmetric space $(R_0,\omega,s)$.
\item
    Importance of left invariance will be explained in the subsection~\ref{subsecImpLeftInvarDstar}.
\end{enumerate}

\begin{proposition}
The space $L^2(R_0)^{\infty}$ of smooth vectors in $L^2(R_0)$ of the left regular representation closes as a subalgebra of $(L^2(R_0),\star^{(2)}_\theta)$.
\end{proposition}

\begin{proof}
First, observe that the space of smooth vectors may be described  as the intersection of the spaces $\{V_n\}$ where  $V_{n+1}:=(V_n)_1$,   with $V_0:=L^2(R_0)$ and $(V_n)_1$ is defined as the space of elements $a$ of $V_n$ such that, for all $X\in\sR_0$, $X.a$ exists as an element of $V_n$ (we endow it with the projective limit topology).                             %%%%

Let thus $a,b\in V_1$. Then, $(X.a)\star b+a\star(X.b)$ belongs to $V_0$.  Observing that $D \subset V_1$ and approximating $a$ and $b$ by sequences $\{a_n\}$ and $\{b_n\}$ in $D $, one gets (by continuity of $\star$): $(X.a)\star b+a\star(X.b)=\lim(X.a_n\star b_n+a_n\star(X.b_n))= \lim X.(a_n\star b_n)=X.(a\star b)$. Hence $a\star b$ belongs to $V_1$.  One then proceeds by induction.
\end{proof}

\subsection{Complement: importance of left invariance}
%-------------------------------------------------------
\label{subsecImpLeftInvarDstar}

This complement intends to shortly explain why to express the Dirac operator under the form of \hyperlink{HyperDefLeftInvar}{left invariant} vectors (the $\tilde t_i$ in equation \eqref{EqDiracAdsquatre}).

Let $R$ be a Lie group with a simply transitive action on the manifold $M$, and suppose that the Dirac operator on $M$ reads
\[
  D|_M=\sigma^i\tilde X_i
\]
where $\sigma^i$ are constant and $\tilde X_i$ a basis of $\mR$ of left invariant vector fields.

\begin{proposition}
In this setting, if $\star^R$ is a right invariant product on $R$, the operator $D$ is a derivation.
\end{proposition}

\begin{proof}
    Take $u\in\Fun(M)$ and a spinor $\varphi$. First $\tilde X_i$ acts by derivation on the right invariant product $\star^R$. Indeed proposition~\ref{prop:def_stn} says that if one defines $u\ast_{\nu} v=T_{\nu}(T_{\nu}^{-1}u\stM T_{\nu}^{-1}v)$, then $\tilde X$ is a derivation of $\ast_{\nu}$ when $T_{\nu}$ is chosen in such a way that $T_{\nu}\rho_{\nu}(X)T_{\nu}^{-1}=\tilde X$. Hence have
\[
\begin{split}
D(u\star^R\varphi)&=\sigma^i\tilde X_i\cdot(u\star^R\varphi)\\
        &=\sigma^i(\tilde X_i\cdot u)\star^R\varphi+\sigma^i u\star^R\tilde X_i\cdot \varphi\\
        &=(Du)\star^R\varphi+u\star^R D\varphi.
\end{split}
\]

\end{proof}

This result implies in turn that the operator $[D,u]$ is bounded because it is the multiplication operator by $Du$. Indeed
\[
\begin{split}
  [D,u]_{\star^R}\varphi
        &=D(u\star^R \varphi)-u\star^R D\varphi\\
        &= (Du)\star^R\varphi+u\star^R D\varphi-u\star^R D\varphi\\
        &=(Du)\star^R\varphi.
\end{split}
\]

\section{Perspectives}
%+++++++++++++++++++++

A main achievement of spectral non-commutative geometry is the ability of retrieving the original Riemannian manifold from the data of the spectral triple. Such a result does not exist in the case of $AdS$ because the latter is a non-compact \emph{pseudo}-Riemannian manifold. The main lines of such a reconstruction method can however be foreseen in the case of anti de Sitter space.
\begin{itemize}
\item Knowing the family of products $\star^{(2)}_{\theta}$, we know in particular the usual commutative product of functions. That should allow us to find back the manifold $AdS_4$.
\item It is possible to extract the data of the curvature of the manifold from the data of its Dirac operator as the non-differential part of its square. That part will of course appear to be constant and negative (because we know that we were starting from anti de Sitter).
\end{itemize}

We only quantized an open orbit of $AdS_4$ because it is a whole physical domain. Quantization of the full space could be very interesting because of a special effect of the noncommutative product: two functions with disjoint supports can have a non vanishing product. What about the physical significance of that property when one multiplies a function supported in the singularity by a function supported in the physical part?

There is another reason to study the quantization of the full space. We will show in section~\ref{SecUnifSOdn} that a deformation of the full space by action of the Iwasawa component of $\SO(2,l-1)$ is possible. That quantization has the advantage of deforming the space by the action of the group which is precisely defining the singularity. In other words the \emph{same} group can describe a singularity and a quantization. A work to be done is to try and recover the special causal structure from the data of the quantized manifold. That structure must be in some way contained in the spectral triple.


\section{Spin structure on the black hole}
%+++++++++++++++++++++++++++++++++++++++++

The group $R_1$ defines the open orbit with the simple action $\tau(r)[r'u]=[rr'u]$ which fulfills $\pi\circ L_g=\tau(g)\circ\pi$. So we will first study the spin frame and spin bundle over $R_1$, and then bring the structure over the open orbit $\mU$.

\subsection{Frame bundle over \texorpdfstring{$R_1$}{R1}}
%-----------------------------------------------------------

A frame at $r\in R_1$ is an isometry between $\eR^{1,3}$ and $T_rR_1$. Since $dL_r$ is such an isometry, any isometry between $\sR_1$ and $T_rR_1$ reads $a=dL_ra'$ with $a'\in\SO(\sR_1)$. So we consider an isometry $\sigma\colon \eR^{1,3}\to \sR_1$ and a basis at $r$ reads
\begin{equation}   \label{EqDefBaseGenebvR1}
    b(v)=dL_r\circ\sigma a(v)
\end{equation}
with $a\in \SO(1,3)$. The basis defined by equation \eqref{EqDefBaseGenebvR1} will be denoted by $b_a(r)$. Here, we will restrict our bundle to $a\in \SO_0(1,3)$. The fact that $b_a$ is a global section of the frame bundle indicates us that the bundle is trivial by theorem~\ref{ThoYPrincBTrivSect}. The trivialization map is
\begin{equation}
\begin{aligned}
 \beta\colon B(R_1)&\to R_1\times \SO_0(1,3) \\
b_a(r)&\mapsto (r,a).
\end{aligned}
\end{equation}


\subsection{Frame bundle over \texorpdfstring{$\mU$}{U}}
%-----------------------------------------------------------

The frame bundle over $[ru]\in \mU$ is the set of isometries $\eR^{1,3}\to T_{[ru]}\mU$. If $\tilde\sigma$ is a given isometry
\[
  \tilde\sigma\colon \eR^{1,3}\to T_{[u]}\mU
\]
(for example a numeration of the fundamental fields), the fiber over $[ru]$ is given by maps of the form
\begin{equation}
\tilde b_a(r)v=d\tau(r)\tilde\sigma av.
\end{equation}
This is isomorphic to $R_1\times\SO_0(1,3)$, in consequence of what we conclude that the frame bundles over $\mU$ and $R_1$ are the same.
