% This is part of (almost) Everything I know in mathematics
% Copyright (c) 2013-2014
%   Laurent Claessens
% See the file fdl-1.3.txt for copying conditions.

\section{Representation}
%+++++++++++++++++++++++

\subsection{Representation of involutive algebra}
%-------------------------------------------------


Let $\cA$ be an involutive algebra and $H$ a Hilbert space. A \defe{representation}{representation!of involutive algebra} of $\cA$ in $H$ is a map $\pi\colon \cA\to \mL(H)$ such that
\begin{equation}
\begin{aligned}
  \pi(A+B)&=\pi(A)+\pi(B),&\pi(\lambda A)&=\lambda\pi(A),\\
\pi(AB)&=\pi(A)\circ\pi(B),&\pi(A^*)&=\pi(A)^*.
\end{aligned}
\end{equation}

A linear form $f\colon \cA\to \eR$ on the involutive algebra $\cA$ is \defe{positive}{positive!form on involutive algebra} if $f(A)\geq 0$ whenever $A>0$.

\begin{lemma}
If $\rho\colon \eM_n(\eC)\to \End(V)$ is a representation of the matrix algebra $\eM_n(\eC)$ on the finite dimensional space $V$, then there exists an isomorphism $V\to \eC^n\oplus\ldots\oplus\eC^n$ which intertwines $\rho$ to a multiple of the standard representation of the matrices on $\eC^n$.
\end{lemma}


\begin{proposition}
Let $\cA$ be an involutive algebra. We have
\begin{enumerate}
\item \label{itemi_prop_invalgrepr} If $\pi$ is a representation of $\cA$ in $H$ and if $\xi\in H$, then $A\mapsto\scal{ \pi(A)\xi }{ \xi }$ is a positive form on $\cA$.
\item  \label{itemii_prop_invalgrepr} Let $\pi$ and $\pi'$ be two representations of $\cA$ in $H$ and $H'$ respectively, and $\xi,\xi'$, two corresponding totalizing vectors. If $\scal{ \pi(A)\xi }{ \xi }=\scal{ \pi'(A)\xi' }{ \xi' }$ for all $A\in\cA$, then there exists an unique isometry $H\to H'$ which transforms $\pi$ into $\pi'$ and $\xi$ into $\xi'$, i.e.
\[ 
 \begin{split}
U\pi(A)U^{-1}&=\pi'(A)\\
U\xi&=\xi'
\end{split} 
\]
\end{enumerate}
 \label{prop_invalgrepr}
\end{proposition}


\begin{proof}
 For the first point, it is easy:
\[ 
  \scal{ \pi(A^*A)\xi }{ \xi }=\scal{ \pi(A^*)\pi(A)\xi }{ \xi }=\scal{ \pi(A)\xi }{ \pi(A)\xi }=\| \pi(A)\xi \|^2\geq 0.
\]
We used property $\pi(A^*)=\pi(A)^*$.

For the second point, $\overline{ \pi(\cA)\xi }=H$ because $\xi$ is totalizing. We have
\begin{equation} \label{eq_piAxipreis}
\scal{ \pi(A)\xi }{ \pi(B)\xi }=\scal{ \pi(B^*A)\xi }{ \xi }
        =\scal{ \pi'(B^*A)\xi' }{ \xi' }
        =\scal{ \pi'(A)\xi }{ \pi'(B)\xi' },
\end{equation}
but vectors of the form $\pi(A)\xi$ are everywhere dense in $H$ (and $\pi'(A)\xi'$ in $H'$), so we have an isomorphism
\begin{equation}
\begin{aligned}
 U\colon H&\to H' \\ 
\pi(A)\xi&\mapsto \pi'(A)\xi' 
\end{aligned}
\end{equation}
Equation \eqref{eq_piAxipreis} shows that $U$ is an isometry; it is linear because of linearity of $\pi$. The map $U$ is surjective because $\xi'$ is totalizing and injective because if $\pi'(A)\xi'=\pi'(B)\xi'$, $\pi'(A-B)=0$ which proves that $A=B$ because $\pi'$ is linear.

Now we check that this $U$ is the searched map. For all $A$, $B\in\cA$, 
\begin{equation}
\begin{split}
  \big[ U\circ\pi(A) \big](\pi(B)\xi)&=U\pi(AB)\xi=\pi'(AB)\xi'\\
        &=\pi'(A)\pi'(B)\xi'=\big[ \pi'(A)\circ U \big](\pi(B)\xi),
\end{split}
\end{equation}
so $U$ transforms $\pi$ into $\pi'$. In order to prove that $U\xi=\xi'$, we will use the fact that $U$ is an isometry:
\[ 
 \scal{ \xi' }{ \pi'(A)\xi' }=\scal{ \xi }{ \pi(A)\xi }
        =\scal{ U\xi }{ U\pi(A)\xi }
        =\scal{ U\xi }{ \pi'(A)\xi' }.
\]

For unicity, notice that equation $U\big( \pi(A)\xi \big)=\pi'(A)\xi'$ defines $U$ on a dense subspace of $H$. Continuity finishes to fix $U$.

\end{proof}

With the same notations,  the map $A\to\scal{ \pi(A) }{ A }$ is the form \defe{associated}{associated form with a representation} with representation $\pi$ and the vector $\xi$. Let $\cB$ be an involutive subalgebra of $\mL(H)$ and $\xi$, any element in $H$. We denote by $\omega_{\xi}$ the form on $\cB$ defined by
\begin{equation}
\omega_{\xi}(A)=\scal{ A\xi }{ \xi }
\end{equation}
for all $A\in\cB$. A positive form $\eta$ on $\cB$ is a \defe{vector}{vector!form} form if there exists a $\xi\in H$ such that $\eta=\omega_{\xi}$.

\begin{proposition}
Let $\cA$ be an involutive Banach algebra with approximate unit $(u_i)$,$\pi$ a nondegenerate representation of $\cA$ in $H$, $\xi\in H$ and $f$ the positive form defined from $\pi$ and $\xi$. We have
\begin{equation}
\| f \|=\scal{ \xi }{ \xi }.
\end{equation}
\end{proposition}

\begin{proof}
Point \ref{itemv_prop_invaddunit} of proposition  \ref{prop_invaddunit} states that if $(u_i)_{i\in J}$ is an approximate unit in $\cA$, then 
\[ 
  f(u_j)\to\| f \|\quad\text{and}\quad f(u_j^*u_j)\to \| f \|.
\]
Hence $\| f \|=\lim f(u_j)=\lim\scal{ \pi(u_j)\xi }{ \xi }$.

\end{proof}

\begin{proposition}
Let $\cA_{\cun}$ be the involutive algebra deduced from $\cA$ by adding an unit and $\pi$, a representation of $\cA$ in $H$. There exists one and only one way to extend $\pi$ to a representation $\pi_{(\cun)}$ of $\cA_{\cun}$ in such a way that $\pi_{(\cun)}(\cun)=\id$.
\end{proposition}
The representation $\pi_{(\cun)}$ is the \defe{canonical extension}{canonic!extension of a representation} of $\pi$. When $\pi$ is a representation of $\cA$ in $H$, the set
\begin{equation}
 K=\{ \pi(A)\xi\tq A\in\cA,\xi\in H \}
\end{equation}
is a closed vector subspace of $H$. We say that $K$ is the \defe{essential subspace}{essential subspace of a representation} of $\pi$. The representation is \defe{nondegenerate}{nondegenerate!representation}\index{representation!nondegenerate} if $K=H$.

\begin{proposition}
Let $\cA$ be an involutive Banach algebra and $\pi$, a nondegenerate representation of $\cA$ on $H$. If $(u_i)$ is an approximate unit in $\cA$, then $\pi(u_i)$ strongly converges to $\id$ (see subsection \ref{subsec_topomL}). 
\end{proposition}

\begin{proof}
We have to prove that $\| \pi(u_i)\xi-\xi \|\to 0$ for any $\xi\in H$. From non degeneracy, the set $\{ \pi(B)\xi\, , B\in \cA \}$ is total in $H$, so it is sufficient to prove the convergence for $\xi$ of the form $\pi(B)\xi$. We have
\[ 
  \| \pi(u_iB)-\pi(B) \|\leq\| u_iB-B \|\to 0
\]
because $u_i$ is an approximate unit and relation \eqref{eq_morleqpi}. 

The fact that $\pi$ is nondegenerate makes
\[ 
  \{ \pi(A)\xi\tq A\in\cA,\,\xi\in H \}=H.
\]
If $\mU$ is an open set in $\mL(H)$ around $\id$, we have to prove that $\pi(u_i)\in\mU$ for all $i\geq i_0$. Open sets are taken in the sense of seminorms $s_{\xi}=\| A\xi \|$. The balls ---which are not open--- are of the form
\[ 
 \begin{split}
B(A;(\xi_j),(r_j))&=\{ X\in\mL(H)\tq s_{\xi_i}(X-A)<r_j\,\forall j \}\\
        &=\{ X\in\mL(H)\tq \| X\xi_j-A\xi_j \|<r_j\,\forall j \}.
\end{split} 
\]
If a sequence fall into the balls $B(A;(\xi_j),(r_j))$ for a fixed $A$ and if we consider an open set around this $A$, the latter open set will contain at least one of the $B(A;(\xi_j),(r_j))$, hence the sequence will fall into this open set too. So we have to prove that for all \emph{finite} sequence $(\xi_j)$ and $r_j$ ($\xi_j\in H$ and $r_j\in\eR^+$),
\[ 
  \pi(u_i)\in B(\id;(\xi_j),(r_j))
\]
when $i$ is large enough.

Let $B\in\cA$ and $\xi\in H$, we have already proved that $\| \pi(u_iB)-\pi(B) \|\leq\| u_iB-B \|\to 0$. We have to prove that $\| \pi(u_i)\xi_j-\xi_j \|< r_j$. Since $\pi(C)\zeta$ is a total set, by redefinition of $\xi_j$, we can write $\xi_j$ under the form $\pi(B_j)\xi_j$. So
\[ 
  \| \pi(u_i)\pi(B_j)\xi_j-\pi(B_j)\xi_j \|=\| \pi(u_iB_j)\xi_j-\pi(B_j)\xi_j \|=\| [\pi(u_iB_j)- \pi(B_j)]\xi_j \|,
\]
but we know that when $A$ is a bounded operator on a Hilbert space, $\| Av \|\leq \| A \|\,\| v \|$. Thus we have
\[ 
\| \pi(u_i)\pi(B_j)\xi_j-\pi(B_i)\xi_j \|=\| [\pi(u_iB_j)-\pi(B_i)]\xi_j \|
        =\| \pi(u_iB_j)-\pi(B_i) \|\,\| \xi_j \|.
\]
Since the sequence $(\xi_j)$ is finite, one can bound $\| \xi_j \|$ by a certain $M$, hence
\[ 
  \| \pi(u_i)\pi(B_j)\xi_j-\pi(B_i)\xi_j \|\leq M\| \pi(u_iB_j)-\pi(B_i) \|.
\]
When $i$ is large, the latter is as small as we want and in particular it can become smaller than all the $r_j$ of the sequence. This proves that $\pi(u_i)$ strongly converges to $\id$ in $\mL(H)$.
\end{proof}

\begin{proposition}[Another version of GNS construction]		\label{PropGNSanother}
Let $\cA$ be an involutive Banach algebra with an approximate unit and the following elements:
\begin{itemize}
\item $\cA_{\cun}$ the involutive algebra obtained by adding an unit to $\cA$,
\item $f$ a positive continuous form on $\cA$,
\item $\tilde f$ its canonical extension to $\cA_{\cun}$,
\item $N$ the left ideal of $\cA_{\cun}$ defined by $N=\{ A\in\cA_{\cun}\tq \tilde f(A^*A)=0 \}$,
\item $\cA'_f$ the pre-Hilbert  space $\cA_{\cun}/N$,
\item $\cA_f$ the Hilbert space obtained by completion of the previous one.
\end{itemize}
For each $A\in\cA_{\cun}$, let 
\begin{itemize}
\item $\pi'(A)$, the operator in $\cA_{\cun}/N$ obtained from quotient of the left multiplication by $A$ in $\cA_{\cun}$,
\item $\xi$, the canonical image of $\cun$ in $\cA'_f$.
\end{itemize}
In this setting we have
\begin{enumerate}
\item \label{itemi_prop_DixGNS}Each $\pi'(A)$ extends in one and only one way to a linear continuous representation of $\cA$ on $\cA_f$,
\item \label{itemii_prop_DixGNS}the map $A\to\pi(A)$ with $A\in\cA$ is a representation of $\cA$ in $\cA_f$,
\item \label{itemiii_prop_DixGNS}the vector $\xi$ is totalizing for $\pi(\cA)$,
\item \label{itemiv_prop_DixGNS} $\forall A\in\cA$, we have $f(A)=\scal{ \pi(A)\xi }{ \xi }$.
\end{enumerate}
\label{prop_DixGNS}
\end{proposition}

In this context, the vector $\xi$, being the representative of the identity, is sometimes called the \defe{vacuum}{vacuum!of a GNS representation} of the GNS representation.

\begin{proof}

	Let $\eta\in\cA_{\cun}/N$ and let's say $\eta=[B]$ for $B\in\cA_{\cun}$, $\pi'(A)\eta=[AB]$ and $\xi=[\cun]=\{ \cun+n\tq n\in N \}$. If $A\in N$, we have $\tilde f(A^*A)=0$, thus for all $B\in N$,
	\[ 
	  \big|  \tilde f\big( (BA)^*(BA) \big)   \big|=\big| \tilde f\big( A^*(B^*B)A \big) \big| \leq \| B^*B \|f(A^*A)=0.
	\]
	This proves that $N$ is an ideal. Now $\pi'(B)\xi=\pi'(B)[\cun]=[B]$, so
	\[ 
	\scal{ \pi'(A)\pi'(B)\xi }{ \pi'(A)\pi'(B)\xi }=\scal{ [AB] }{ [AB] }
			=\scal{ [B^*A^*AB] }{ [\cun] }
	\]
	where this product is defined by
	\begin{equation}
	  \scal{ [A]\, }{\, [B] }:=\tilde f(AB).
	\end{equation}
	So
	\[ 
	 \begin{split}
	 \scal{ \pi'(A)\pi'(B)\xi }{ \pi'(A)\pi'(B)\xi }&=\tilde f(B^*A^*AB)\\
			&\leq \| A^*A \|\tilde f(B^*B) \\
			&=\| A^*A \|\scal{ \pi'(Bj\xi) }{ \pi'(B)\xi }.
	\end{split} 
	\]
	It gives $\| \pi'(A)[B] \|\leq\| A^*A \|\,\| [B] \|$, and finally
	\begin{equation}
	  \| \pi'(A) \|\leq\| A^*A \|,
	\end{equation}
	which proves that $\pi'$ is continuous, and therefore bounded. But a bounded operator on a part of a Hilbert space may be extended to the whole space. This finish the proof of \ref{itemi_prop_DixGNS}.

	Now we prove that $\pi$ is a representation for the structure of involutive algebra. The algebra structure is clear. For involution,
	\[ 
	\scal{ \pi(A)\pi(B)\xi }{ \pi(C)\xi }	=\tilde f(C^*AB)
						=\tilde f\big( (A^*C)^*B \big)
						=\scal{ \pi(B)\xi }{ \pi(A^*)\pi(C)\xi },
	\]
	so $\pi(A)^*=\pi(A^*)$. Notice that there are no argument as ``$\pi(B)\xi$ is dense''; we just use the fact that $\pi(B)\xi=[B]$ is the most general element in $\cA_{\cun}/N$. Ok for point \ref{itemii_prop_DixGNS}.

	On the one hand, $\cA_{\cun}$ is everywhere dense in $\cA$. On the other hand, $\cA'_f$ is everywhere dense in $\cA_f$ from the definition of a completion. But $\pi(\cA)\xi$ is the image of $\cA$ in $\cA'_f$, so $\pi(\cA)\xi=\cA'_f$ is everywhere dense in $\cA_f$. This proves that $\xi$ is totalizing for $\pi(\cA)$.  This proves \ref{itemiii_prop_DixGNS}.

	Finally, for all $A\in\cA_{\cun}$, 
	\[ 
	  \scal{ \pi(A)\xi }{ \xi }=\tilde f(\cun^*A\cun)=\tilde f(A).
	\]
	  
\end{proof}
We say that the representation $\pi$ and the vector $\xi$ are defined from $f$. So we often write $\pi_f$ and $\xi_f$.

\subsection{Cyclic representations of \texorpdfstring{$C^*$}{C*}-algebra }
%-----------------------------------------------------------------------

\begin{definition}
A \defe{representation}{representation!of a $C^*$-algebra} of the $C^*$-algebra $\cA$ on a Hilbert space $\hH$ is a linear map $\dpt{\pi}{\cA}{\oB(\hH)}$ such that

\begin{enumerate}
\item $\pi(AB)=\pi(A)\circ\pi(B)$,
\item $\pi(A^*)=\pi(A)^*$
\end{enumerate}
for all $A$, $B\in\cA$. 
\end{definition}
Most of time, we will denote a representation by the pair $(\pi,\hH)$. When the represented $C^*$-algebra is ambiguous, we write $(\cA,\pi,\hH)$.

\begin{lemma}       \label{Lemrepresnormpresou}
A representation $\pi$ is continuous and fulfills
\begin{equation}
\| \pi(A) \|\leq \| A \|,
\end{equation}
moreover when the representation is faithful, we have $\| \pi(A) \|=\| A \|$.
\end{lemma}

\begin{proof}
The spaces $\cA$ and $\oB(\hH)$ are $C^*$-algebra and $\pi$ is a morphism proposition \ref{PropMDfqcUs} concludes. The second claims follows from lemma \ref{lem:injmorpisom}.
\end{proof}

Let $(\pi_1,\hH_1)$ and $(\pi_2,\hH_2)$ be two representations. They are \defe{equivalent}{equivalence!of representation of $C^*$-algebra } when there exists an unitary isomorphism $\dpt{U}{\hH_1}{\hH_2}$ such that
\begin{equation}
  U\pi_1(A)U^*=\pi_2(A)
\end{equation}
for all $A\in\cA$.

A representation $(\cA,\pi,\hH)$ is \defe{nondegenerate}{degenerated representation of $C^*$-algebra} if $0$ is the only vector to be cancelled by all $\pi(A)$. It is \defe{cyclic}{cyclic!representation} if there exists a \defe{cyclic vector}{cyclic!vector} $\Omega\in\hH$, i.e. the closure of $\pi(\cA)\Omega=\hH$.


\begin{lemma}
If $\pi$ is an irreducible representation on $\hH$, then any non zero vector is cyclic.
\end{lemma}

\begin{proof}
Let $v\neq 0$; if it were not cyclic, then $\overline{ \pi(\cA)v}$ should be a proper invariant  subspace of $\hH$.
\end{proof}

\subsection{Primitive spectrum}
%------------------------------
For this short note about primitive spectrum, we follow \cite{Landi}.

When $\cA$ is any (not specially commutative) $BC^*$-algebra, the \defe{primitive spectrum}{primitive spectrum}\index{spectrum!primitive} of $\cA$ is the set $\Prim\cA$\nomenclature{$\Prim$}{Primitive spectrum} of kernels of $*$ irreducible representations. An element in $\Prim\cA$ is a two-sided ideal.

There exists a suitable topology on this space, the \defe{hull-kernel topology}{hull-kernel topology} or \defe{Jacobson topology}{Jacobson topology}\index{topology!hull-kernel}\index{topology!Jacobson}. This is given by means of closure. If $W\subset\Prim\cA$, then we define the closure of $W$ by
\begin{equation}
  \overline{ W }=\{ \mI\in\Prim\cA\tq \cap W\subseteq \mI \}.
\end{equation}
where $\cap W=\cap_{\mJ\in W}\mJ$. The inclusion $\cap W\subseteq\mI$ is an inclusion of subsets of $\cA$.

\begin{proposition}
This definition defines a topology. Namely, it fulfils the Kuratowsky axioms:

\begin{enumerate}
\item\label{enu802i} $\overline{ \emptyset }=\emptyset$,
\item \label{enu802ii} $W\subseteq\overline{ W }$,
\item \label{enu802iii} $\overline{ \overline{ W } }=\overline{ W }$,
\item \label{enu802iv} $\overline{ W_1\cup W_2 }=\overline{ W_1 }\cup\overline{ W_2 }$.
\end{enumerate}
\end{proposition}

\begin{proof}
Points \ref{enu802i}  and \ref{enu802ii} are trivial. For \ref{enu802iii}, remark that $\cap W=\cap\overline{ W }$ (equality as subsets of $\cA$). Indeed, consider $A\in\cap W$. Then any $\mJ\in\overline{ W }$ contains $A$ and then $\cap\overline{ W }$ contains $A$. Now if $A\in\cap\overline{ W }$, then any $\mI$ such that $\cap W\subseteq\mI$ contains $A$ and then $A\in\cap W$.
 
The proof of \ref{enu802iv} is more complicated. If $V\subset W$, then $(\cap W)\subseteq(\cap V)$ and then $\overline{ V }\subseteq\overline{ W }$. Then $\overline{ W_i }\subseteq\overline{ W_1\cup W_2 }$ for each of $i=1,2$.

The inverse inclusion is as follows. Let $\mI$ be an ideal kernel of the irreducible representation $\pi$ of $\cA$ on the Hilbert space $\hH$. Suppose that $\mI\notin \overline{ W_1 } \cup \overline{ W_2 }$. Then we will prove that $\mI\notin\overline{ W_1\cup W_2 }$. There exists $A\in W_1$ and $B\in W_2$ such that $\pi(A)\neq 0$ and $\pi(B)\neq0$. Let $\xi\in\hH$ such that $\pi(A)\xi\neq0$. Since $\pi$ is irreducible, then $\pi(A)\xi$ is cyclic. Since $\pi(B)\neq 0$, then there exists a $\psi\in\hH$ such that $\pi(B)\psi\neq0$. Cyclicity of $\pi(A)\xi$ shows that there exists a $C$ such that $\pi(C)\pi(A)\xi$ is sufficiently close to $\psi$ to satisfy
\[ 
  \pi(B)\big( \pi(C)\pi(A)\xi \big)\neq0.
\]
 Then $BCA\notin\ker\pi=\mI$. Since $W_i$ are ideals, 
\[ 
  BCA\in(\cap W_1)\cap(\cap W_2)=\cap(W_1\cup W_2). 
\]
Then $BCA\in W_1$ and $\cap(W_1\cup W_2)\nsubseteq\mI$. Consequently, $\mI\notin\overline{ W_1\cap W_2 }$.


\end{proof}


\subsection{GNS construction}
%----------------------------

\begin{lemma}
Let $\mfM$ be a $*$-algebra in $\oB(\hH)$, $\psi\in\hH$ and $p$, the projection into the closure of $\mfM\psi$. Then $p\in\mfM'$, i.e. $[p,A]=0$ for all $A\in\mfM$.\label{lem_preGNS}
\end{lemma}

\begin{proof}
Let $A\in\mfM$; by definition of $p$, we have $Ap\hH\subseteq p\hH$. For a $A\in\mfM$, $Ap\hH=\{ AB\psi\tq B\in\mfM \}$, but $\mfM$ is an algebra, then $AB\in\mfM$ and $Ap\hH\subseteq\mfM\psi\subseteq p\hH$. If we define $p^{\perp}=\mtu-p$, we find $p^{\perp}Ap=0$.

 Indeed $(\mtu-p)Ap=Ap-pAp$ and $Apx-pApx$ can be computed by setting $x=B\psi$ for a certain $B$ in the closure of $\mfM$ (the part of $x$ ``outside'' $\mfM$ has no importance). Then
\begin{equation}
\begin{aligned}
   ApB \psi-pAB\psi&=AB\psi-pAB\psi  &&\textrm{because $B\psi\in\mfM$}\\
        &=AB\psi-AB\psi  &&\textrm{because $AB\psi\in\mfM\psi$}\\
        &=0.
\end{aligned}
\end{equation}
It shows that $ApB\psi-pAB\psi=0$ and then that $p^{\perp}Ap=0$. 

From this, we see that $Ap=pAp$. Let us now consider $A=A^*$ (for a general element in $\cA$, use the decomposition). We have $(Ap)^*=p^*A^*=pA$, but $(Ap)^*=(pAp)^*=pAp=Ap$. Then $[A,p]=0$.


\end{proof}


\begin{proposition}
Any nondegenerate representation is direct sum of cyclic representations.
\end{proposition}

\begin{proof}
We apply the lemma with $\mfM=\pi(\cA)$. The non degeneracy of $\pi$ makes $p$ non zero: $\mfM\psi$ is never zero. Now we consider the map $\dpt{\rho}{\cA}{\oB(\hH)}$, $\rho(A)=p \pi(A)$. This is a representations because
\begin{equation}
    \rho(AB)=p\pi(A)\pi(B)
        =p\pi(A)p\pi(B)
        =\rho(A)\rho(B),
\end{equation}
and
\begin{equation}
\begin{split}
\rho(A^*)&=p\pi(A^*)=\pi(A^*)p\\
        &=\pi(A)^*p=(p\pi(A))^*=\rho(A)^*.
\end{split}
\end{equation}
More precisely, $\rho$ is a representation on $p\hH$ and when $\pi(A)\in p\hH$, we have $\rho(A)=\pi(A)$. The representation $\rho$ is constructed in such a way that $\psi$ is a cyclic vector:
\[ 
  \rho(A)\psi=p\hH.
\]
The same construction with $\psi_2\in p^{\perp}\hH$ gives and going on gives the thesis.
\end{proof}


Let $(\cA,\pi,\hH)$ be a nondegenerate representation and $\Psi\in\hH$ a vector with norm $1$. The state $\psi$ given by formula
\[ 
  \psi(A)=\scal{\Psi}{\pi(A)\Psi}
\]
is the \defe{vector state}{vector!state}\index{state!vector} of $\Psi$ relative to $\pi$.

\begin{theorem}[GNS construction]       \label{ThoGNScontruction}
Let $\cA$ be an unital $C^*$-algebra  and $\dpt{\omega}{\cA}{\eC}$, a state

\begin{enumerate}
\item\label{GNSi} There exists a Hilbert space $\hH$, a representation $\dpt{\pi_{\omega}}{\cA}{\oB(\hH_{\omega})}$ and a cyclic unit vector $\Omega_{\omega}$ such that
\begin{equation}
  \omega(A)=\scal{\Omega_{\omega}}{\pi_{\omega}(A)\Omega_{\omega}}
\end{equation}
for all $A\in\cA$.
\item\label{GNSii} The triple $(\hH_{\omega},\pi_{\omega},\Omega_{\omega})$ is unique up to isomorphism in the following sense. Let $\hH$ be a Hilbert space, $\dpt{\pi}{\cA}{\oB(\cA)}$ a representation and $\Omega\in\hH$ an unit cyclic vector such that $\omega(A)=\scal{\Omega}{\pi(A)\Omega}$ for all $A\in\cA$; then there exists an unitary isomorphism $\dpt{u}{\hH_{\omega}}{\hH}$ such that $u(\Omega_{\omega})=\omega$ and
\[ 
  \pi(A)=u\pi_{\omega}(A)u^*
\]
for all $A\in \cA$.

\end{enumerate}
\label{tho:GNS}
\end{theorem}

\begin{proof}
From \eqref{eq:omABleq} we know that, if $A\in\mN_{\omega}$ and $B\in\cA$, then $\omega(B^*A)=0$. So we can define $\mN$ in the two equivalent ways:
\begin{equation}
    \begin{aligned}[]
        \mN_{\omega}&=\{ A\in\cA\tq \omega(A^*A)=0 \}\\
        &=\{ A\in\cA\tq \omega(B^*A)=0\textrm{ for all $B\in\cA$} \}.
    \end{aligned}
\end{equation}
From the second line, we see that $\mN_{\omega}$ is an ideal in $\cA$. The set $\mN_{\omega}$ is closed from continuity of $\omega$. We use the product defined by equation \eqref{eq:defprodetat}: $(A,B):=\omega(A^*B)$. It defines a sesquilinear form on the quotient $\cA/\mN_{\omega}$
\begin{equation}
\scal{ VA }{VB}=\omega(A^*B)
\end{equation}
where $\dpt{V}{\cA}{\cA/\mN_{\omega}}$ is the canonical projection $VA=A+\mN_{\omega}$. So $\cA/\mN_{\omega}$ is a pre-Hilbert space from which we build $\hH_{\omega}$ by completion. We define the cyclic vector $\Omega_{\omega}=V\cun\in\hH_{\omega}$.

For each $A\in\cA$, we define $\dpt{L_A}{\cA/\mN_{\omega}}{\cA/\mN_{\omega}}$ by
\begin{equation}  \label{eq:defpiomega}
L_AVB=V(AB). 
\end{equation}
We have
\begin{equation}
\begin{aligned}
\| VAB \|^2&=\scal{VAB}{VAB}^2\\
        &=\omega(B^*A^*AB)^2\\
        &\leq \| A \|^4\omega(B^*B)^2&&\textrm{corollary \ref{cor:BeAAeB}}\\
        &=\| A \|^4\| VB \|^2,
\end{aligned}
\end{equation}
then for all $\psi\in\cA/\mN_{\omega}$, $\| L_A\psi \|\leq \| A \|^2\| \psi \|$. We conclude that
\begin{equation}
  \| L_A \|\leq \| A \|^2.
\end{equation}
Then $L_A$ can be extended to a bounded operator $\pi_{\omega}(A)$ on the whole $\hH_{\omega}$. The map $\dpt{\pi_{\omega}}{\cA}{\oB(\hH_{\omega})}$ is a representation such that for all $A\in\cA$,
\begin{subequations}
\begin{align}
\omega(A)&=\scal{\Omega_{\omega}}{\pi_{\omega}(A)\Omega_{\omega}},\\
\overline{ \pi_{\omega}(A)\Omega_{\omega} }&=\overline{ \cA/\mN_{\omega} }=\hH_{\omega}.
\end{align}
\end{subequations}
It proves point \ref{GNSi}; we now turn our attention to \ref{GNSii}. For all $A$, $B\in\cA$, we have
\begin{equation}  \label{eq_r1903r4}
\scal{\pi_{\omega}(B)\Omega_{\omega}}{\pi_{\omega}(A)\Omega_{\omega}}=\scal{\Omega_{\omega}}{\pi_{\omega}(B^*A)\Omega_{\omega}}
        =\scal{V\cun}{V(B^*A)}
        =\omega(B^*A),
\end{equation}
but from assumptions, $\omega(A)=\scal{\Omega}{\pi(A)\Omega}$; then 
\begin{equation} \label{eq:BesAOmBeA}
  \omega(B^*A)=\scal{\Omega}{\pi(B^*A)\Omega}
        =\scal{\pi(B)\Omega}{\pi(A)\Omega}.
\end{equation}
We conclude that 
\begin{equation}
  \scal{\pi_{\omega}(B)\Omega_{\omega}}{\pi_{\omega}(A)\Omega_{\omega}}=
    \scal{\pi(B)\Omega}{\pi(A)\Omega}.
\end{equation}
From definition of a cyclic vector, $\pi_{\omega}(\cA)\Omega_{\omega}$ is dense in $\hH_{\omega}$ and $\pi(\cA)$ in $\hH$. This allows us to define $\dpt{u}{\hH_{\omega}}{\hH}$ by the condition
\[ 
  u\pi_{\omega}(A)\Omega_{\omega}=\pi(A)\Omega.
\]
It is a well defined Hilbert space isomorphism because if $\pi_{\omega}(A)\Omega_{\omega}=\pi_{\omega}(B)\Omega_{\omega}$, then equation (true for all $D\in\cA$) $\scal{\pi_{\omega}(B)\Omega_{\omega}}{\pi_{\omega}(D)\Omega_{\omega}}=\scal{\pi(B)\Omega}{\pi(D)\Omega}$ gives an equation of the form $\scal{x}{d}=\scal{y}{d}$ for all $d$ in a dense subset. This equation implies that $x=y$, or $\pi(A)\Omega=\pi(B)\Omega$. We know from general Hilbert space theory that a surjective isometry is unitary; this is the case of $u$.

\end{proof}

For later use, we mention that equation \eqref{eq:BesAOmBeA} gives
\begin{equation}  \label{eq:piomomaesm}
\| \pi_{\omega}(A)\Omega_{\omega} \|=\omega(A^*A)
\end{equation}
when $A=B$.

\begin{corollary}
Let $(\pi_i,\hH_i)$ $(i=1,2)$ be two cyclic representations with cyclic vectors $\Omega_i$. If for all $A\in\cA$, 
\[ 
  \omega_1(A):=\scal{\Omega_1}{\pi_1(A)\Omega_1}=\scal{\Omega_2}{\pi_2(A)\Omega_2}=:\omega_2(A),
\]
then they are equivalent representations.

\end{corollary}

\begin{proof}
The representation $\pi_i$ in $\hH_i$ is cyclic and induces a GNS representation $\pi_{\omega_i}$. Since $\omega_1=\omega_2$, these two GNS representations are the same and $\pi_1$ and $\pi_2$ are thus both equivalent to the same representation.

\end{proof}

The following is just a restatement of \ref{GNSii} of theorem \ref{tho:GNS}.
\begin{proposition}
If $(\cA,\pi,\hH)$ is a cyclic representation, then all the GNS constructions build from a vector state are unitary equivalent to $\pi$. \label{prop:cyclequivGNS}
\end{proposition}


\subsection{Universal representation}
%-----------------------------------

\begin{theorem}
Any $C^*$-algebra accepts an isometric representation on a Hilbert space.
\end{theorem}

\begin{proof}
Let $\cA_h$ be the real Banach space build from hermitian elements of $\cA$ and choose a non zero $A\in\cA$. The element $-A^*A$ does not belong to $\cA^+$. Since $\cA^+$ is a convex closed cone, there exists a linear continuous form $f_A$ on $\cA_h$ such that $f_A(B)\geq 0$ for all $B\in\cA^+$ and $f_A(-A^*A)<0$. One can identify $f_A$ to an hermitian form on $\cA$ because $B$ is decomposed as $B=A_1+iA_2$ with $A_1,A_2\in\cA_h$; we define $f_A(B)=f_A(B_1)+f_A(B_2)$. The function $f_A$ is also a positive form on $\cA$ because on any positive element $B^*B$, we have $f_A(B^*B)>0$.

Now we look at $\pi_A$, the representation defined by $f_A$. We have $f_A(B)=\scal{ \pi_A(B)\xi }{ \xi }$, thus
\[ 
  f_A(A^*A)=\scal{ \pi_A(A^*)\pi_A(A)\xi }{ \xi },
\]
which is zero if $\pi_A(A)=0$. Then $\pi_A(A)\xi\neq 0$ and we conclude that $\pi_A(A)\neq 0$.

We consider $\pi$, the direct sum of all the representations $\pi_A$ for all $A\in\cA$, $A\neq 0$. First we prove that $\pi$ is injective. Indeed if $\pi(B)=0$, we have $\pi_A(B)=0$ for all $A$; in particular 
\begin{equation}
  0=\scal{ \pi_A(B)\xi }{ \pi_A(B) }
        =\scal{ \pi_A(B^*B)\xi }{ \xi }
        =f_A(B^*B)
\end{equation}
by \ref{itemiv_prop_DixGNS} of proposition \ref{prop_DixGNS}. So if $\pi$ is not invertible, there exists a $B\neq 0$ such that for all $A$, $f_A(B^*B)=0$; this implies $\| B^*B \|=0$ and $B=0$. Contradiction.

The map $\pi$ is isometric as injective morphism: $\| \pi(A) \|=\| A \|$.
\end{proof}


The \defe{universal representation}{universal!representation} $\pi_u$ of a $C^*$-algebra $\cA$ is the direct sum of all the GNS representations $\pi_{\omega}$ with $\omega\in\etS(\cA)$. The representation space is 
\[ 
  \hH_u=\bigoplus_{\omega\in\etS(\cA)}\hH_{\omega}.
\]


\begin{theorem}[Gelfand-Neumark]
A $C^*$-algebra is isomorphic to a subalgebra of $\oB(\hH)$ for a certain Hilbert space $\hH$.
\end{theorem}

\begin{proof}
Let's show that $\hH=\hH_u$ and the isomorphism $\pi_u$ answer the question.

\subdem{Injective}
Let $A\in\cA$ such that $\pi_u(A)=0$; from definition of the direct sum and of universal representation,  $\pi_{\omega}(A)=0$ for all $\omega\in\etS(\cA)$. Using equation \eqref{eq:piomomaesm} we find that for such a $A$, we have $0=\| \pi_{\omega}(A)\Omega_{\omega} \|^2=\omega(A^*A)^2$. It is true for all $\omega\in\etS(\cA)$. Lemma \ref{lem:omAenomA} then shows that $\| A^*A \|=0$ and then that $A=0$ because of definition of a $C^*$-algebra. 

\subdem{Surjective}
The representation $\pi_u$ is not specially surjective on $\oB(\hH_u)$, but it is surjective on the subalgebra $\pi_u(\cA)$ which is enough for the present purpose.

\subdem{Morphism}
The map $\dpt{\pi_u}{\cA}{\oB(\hH_u)}$ is a morphism because it is a representation.


\subdem{Isometry}
Lemma \ref{lem:injmorpisom} says that an injective morphism of $C^*$-algebra is isometric.

\end{proof}

The universal representation is trivially faithful, but it is very huge. For example the smallest faithful representation of $\oB(\hH)$ is the definition representation on $\hH$.

\begin{corollary}
An operator $A$ is positive if and only if $\pi(A)\geq 0$ for all cyclic representation $\pi$.
\end{corollary}

\begin{proof}
Since $A$ is positive, then $\sigma(A)\subset\eR^+$ and $A^*=A$. In order to prove that $\pi_u(A)$ is positive, we have to show that $\sigma(\pi_u(A))\subset\sigma(A)$. Let $z\in\sigma(\pi_u(A))$: the operator $\pi_u(A)-z\mtu_u$ where $\mtu_u$ is the unit operator on $\hH_u$ is not invertible. From equation  \eqref{eq:defpiomega},  we have $\mtu_u=\pi_u(\mtu)=\sum_{\omega\in\etS(\cA)}\pi_{\omega}(\cun)$. 
 
Let $z\notin\sigma(A)$ and let us see that $z\notin\sigma(\pi_u(A))$. From assumption on $z$, there exists a $B\in\cA$ such that $B(A-z\cun)=(A-z\cun)=\cun$. Then $\pi_u(B)$ is the inverse of $\pi_u(A)-z\pi_u(\cun)$. It proves that $z\notin\sigma(\pi_u(A))$ and so that $\pi_u(A)$ is positive.

We now prove that $\pi_{\omega}(A)$ is positive for all GNS representation $\pi_{\omega}$. Since $\pi_u$ acts separately on each space $\hH_{\omega}$, all what we said about the invertibility about $\pi_u$ can be said for each $\pi_{\omega}$.

But we know that all cyclic representation is equivalent to a GNS representation from proposition \ref{prop:cyclequivGNS}. We just have to prove that positivity is conserved by equivalence. Suppose that it is not the case. Let $z\in\sigma(\pi_{\omega}(A))$ and $B\in\cA$ such that $B\big( \pi_{\omega}(A)-z\mtu_{\omega} \big)=\mtu_{\omega}$. Then 
\[ 
  UBU^*\big( \pi(A)-z\mtu_{\omega} \big)=\mtu_{\omega}
\]
and then $UBU^*$ is the inverse of $\pi(A)-z\mtu_{\omega}$ and $z\notin\sigma(\pi(A))$. Thus
\[ 
  \sigma(\pi(A))\subset\sigma(\pi_{\omega}(A))\subset\eR^+
\]
which proves that $\pi(A)$ is positive.

We now prove the second sense of the corollary. All GNS representation is cyclic, then $\pi_u(A)$ is positive as sum of positive representation. We want to deduce that $A\geq0$. Since $\pi_u(A)$ is positive, $\pi_u(A)=\pi_u(A)^*=\pi_u(A^*)$. This implies that $A=A^*$ because $\pi_u$ is injective. The positivity of $\pi_u(A)$ gives the existence of a $B$ such that  $\pi_u(A)=\pi_u(B)^*\pi_u(B)=\pi_u(B^*B)$. The injectivity then shows that $A$ is positive.

\end{proof}

The GNS construction is done for unital algebras with states. Since there exists a notion of state on non unital algebras, ones raises the question to generalization of the GNS construction to non unital algebras.

%%%%%%%%%%%%%%%%%%%%%%%%%%
%
   \section{Spaces of matrices}
%
%%%%%%%%%%%%%%%%%%%%%%%%

Let $\cA$ be a $C^*$-algebra and $n\in\eN$. The $C^*$-algebra $\mfM^n(\cA)$\nomenclature{$\mfM^n(\cA)$}{$C^*$-algebra of matrices} is the space on $n\times n$ matrices with entries in $\cA$.The multiplication is defined by
\begin{equation}
  (MN)_{ij}=\sum_kM_{ik}N_{kj}
\end{equation}
where the product in the right hand side is the (in general noncommutative) one in $\cA$. The involution is naturally given by
\begin{equation}
  (M^*)_{ij}=M_{ij}^*.
\end{equation}
One can identify $\mfM^n(\cA)$ to $\cA\otimes\mfM^n(\eC)$ by identifying\footnote{The matrix $E_{ij}$ is the matrix full of zero except a $1$ at position $ij$.} $E_{ij}\in\mfM^n(\cA)$ to $A\otimes E_{ij}\in\cA\otimes\mfM^n(\eC)$.

The Gelfand-Neumark gives the existence of a faithful representation $\pi$ of $\cA$ on $\hH$. If one sees elements of $\hH\otimes\eC^n$ as $n$-uples $(v_1,\ldots,v_n)$ where each $v_i\in\hH$, we can define $\pi_n$ on $\hH\otimes\eC^n$ by linear extension of 
\begin{equation} \label{eq:defreprezmfM}
  \big[\pi_n(M)v\big]_i:=\pi(M_{ij})v_j.
\end{equation}
The norm $\| M \|$ is defined as the norm of $\pi_n(M)$. Since $n<\infty$, $\pi_n(\mfM^n(\cA))$ is a closed $*$-algebra in $\oB(\hH\otimes\eC^n)$ and then $\mfM^n(\cA)$ is a $C^*$-algebra for this norm. Proposition \ref{prop:unicitenormcsa} states that the norm is unique and then that the norm $\| M \|=\pi_n(M)$ is independent of the choice of $\pi$.

\begin{definition}      \label{DefComplPositive}
    A linear map $\dpt{ q }{ \cA }{ \cB }$ is \defe{completely positive}{positive!completely!map between $C^*$-algebra}\index{completely!positive} if for all $n\in\eN$, the map $\dpt{ q_n }{ \mfM^n(\cA) }{ \mfM^n(\cB) }$ defined by
    \[ 
      (q_n(M))_{ij}=q(M_{ij})
    \]
    is positive. 
\end{definition}
As an example, a morphism $\varphi$ is always completely positive because if $a=b^*b$ in $\mfM^n(\cA)$, then $\varphi(a)=\varphi(b)^*\varphi(b)$ which is positive in $\mfM^n(\cB)$.

%%%%%%%%%%%%%%%%%%%%%%%%%%
%
   \section{Stinespring theorem}
%
%%%%%%%%%%%%%%%%%%%%%%%%

Let us state a classical result about Hilbert space

\begin{lemma}
If $K$ is closed in a Hilbert space $\hH$ and if the linear operator $\dpt{ T }{ \hH }{ \hH }$ fulfils

\begin{itemize}
\item $\scal{ Tx }{ y }=\scal{ x }{ Ty }$ for all $x$, $y\in\hH$,
\item $Ty=y$ for all $y\in K$
\item $Tz=0$ for all $z\in K^{\perp}$,
\end{itemize}
then $T$ is the projection on $K$.

\end{lemma}
This lemma allows us to check that $WW^*$ is the projection to the image of $W$. Let us prove that $W^*W$ is the projection on $K_1$. The first condition is clear. The third is satisfied by definition of $W$: if $z\in K_1^{\perp}$, then $W^*Wz=0$. For the second one, remark that if $y\in K_1$, $\scal{ W^*Wx }{ y }=\scal{ x }{ y }$ and if $y\in K_1^{\perp}$, then $\scal{ W^*Wx}{ y }=0=\scal{ x }{ y }$. In both cases $y\in K_1$ and $y\in K_1^{\perp}$, we have $\scal{ W^*Wx }{ y }=\scal{ x }{ y }$. It is sufficient to conclude that $W^*Wx=x$ because $\hH=K_1\oplus K_1^{\perp}$.

\begin{theorem} \index{Stinespring theorem}\index{theorem!Stinespring}
   Let $\dpt{ q }{ \cA }{ \cB }$ be a completely positive map between unital  $C^*$-algebra such that $q(\cun)=\cun$. We suppose that $\cB$ is given with a faithful representation $\cB\simeq\pi_{\chi}(\cB)\subseteq\oB(\hH_{\chi})$ for a certain Hilbert space $\hH_{\chi}$. Then there exists a Hilbert space $\hH^{\chi}$, a representation $\pi^{\chi}$ of $\cA$ on $\hH^{\chi}$ and a partial isometry $\dpt{ W }{ \hH_{\chi} }{ \hH^{\chi} }$ with $W^*W=\cun$ and
\begin{equation}  \label{eq:stinun}
  \pi_{\chi}(q(A))=W^*\pi^{\chi}(A)W
\end{equation}
for all $A\in\cA$.

Stated in an equivalent way, if we define $P=WW^*$ and $\tilde\hH_{\chi}=P\hH^{\chi}\subset\hH^{\chi}$, and $\dpt{ U }{ \hH_{\chi} }{ \tilde\hH_{\chi} }$ as the restriction of $W$ (in such a way that $U$ is unitary because $W$ is a partial isometry), then we have
\begin{equation} \label{eq:stindeux}
  U\pi_{\chi}(q(A))U^{-1}=P\pi^{\chi}(A)P.
\end{equation}
\label{tho:stinespring} 
\end{theorem}


\begin{proof}
On $\hH_{\chi}$ we have the scalar product $\scal{.}{.}_{\chi}$ and we define the sesquilinear form $\scal{.}{.}_0^{\chi}$ on $\cA\otimes\hH_{\chi}$ by sesquilinear extension of 
\[ 
  (A\otimes v,B\otimes w)_0^{\chi}=\scal{v}{\pi_{\chi}(q(A^*B))w}_{\chi}.
\]
 \subdem{This form is semi positive definite}
Let us compute
\begin{equation}
\sum_{ij}(A_i\otimes v_i,A_j\otimes v_j)_0^{\chi}=\sum_{ij}\scal{v_i}{ \pi(q(A_j^*A_j))v_j }_{\chi}.
\end{equation}
For this, we consider $a\in\mfM^n(\cA)$ with elements $a_{ij}=A_i^*A_j$. Let $\pi_n$ be the faithful representation \eqref{eq:defreprezmfM}  of $\mfM^n(\cA)$ on $\hH\otimes \eC^n$ defined by
\[ 
  [\pi_n(M)v]_i=\sum_j\pi(M_{ij})v_j\in\hH.
\]
where $\pi$ is a faithful representation of $\cA$. If $z\in\eC^n\otimes\hH$, we can define $(az)\in\eC^n\otimes\hH$ by 
\[ 
  (az)_i=\big( \pi_n(a)z \big)_i.
\]
So we have
\begin{equation}
 \scal{ z }{ az }=\sum_{ij}\scal{ z_i }{ \pi(a_{ij})z_j }
        =\sum_{ij}\scal{ \pi(A_i)z_i }{ \pi(A_j)z_j }
        =\| Az \|^2\geq0
\end{equation}
where we use the notation $Az=\sum_i\pi(A_i)z_i\in\hH$. The conclusion is that $a\geq0$. Since $q$ is completely positive, $b$ is positive if we define $b_{ij}=q(A_i^*A_j)$. Positivity of $b$ gives rise to an element $c\in\mfM^n(\cB)$ such that $n=c^*c$ where $(c^*)_{ij}=(c_{ji})^*$. From the representation $\pi_{\chi}$ of $\cB$, we can build the faithful representation $\pi'_n$ of $\mfM^n(\cB)$ on $\eC^n\otimes\hH_{\chi}$: $[\pi'_n(b)v]_i=\sum_j\pi_{\chi}(b_{ij})v_j$ where each $v_i$ now belongs to $\hH_{\chi}$.

We are now able to prove that $\scal{ . }{ . }_0^{\chi}$ is a positive form. Indeed 
\begin{equation}
\begin{split}
\sum_{ij}  ( A_i\otimes v_i , A_j \otimes v_j )_0^{\chi}
        &=\sum_{ij} \scal{ v_i }{ \pi_{\chi}\big( q(A_i^*A_j) \big)v_j }_{\chi}
        =\sum_{ij} \scal{ v_i }{ \pi_{\chi}(b_{ij})v_j }_{\chi}\\
        &=\sum_{ijk}\scal{ \pi_{\chi}(c_{ki})v_i }{ \pi_{\chi}(c_{kj})v_j }_{\chi}\geq0
\end{split}
\end{equation}
where the last equality comes from the fact that $(c^*c)_{ij}=\sum_k(c^*)_{ik}c_{kj}=\sum_k(c_{ki})^*c_{kj}$.  This proves that $( . , . )_0^{\chi}$ is positive semi definite.

\subdem{Definition of $\pi^{\chi}$.}

Now we denote by $\mN_{\chi}$ the null space of $( . , . )_0^{\chi}$. Let $\dpt{ V_{\chi} }{ \cA\otimes\hH_{\chi} }{ \cA\otimes\hH_{\chi}/\mN_{\chi} }$ be the canonical projection. We define
\begin{equation}
\scal{ V_{\chi}(A\otimes v) }{ V_{\chi}(B\otimes w) }^{\chi}:=( A\otimes v , B\otimes w )_0^{\chi}
\end{equation}
and we denote by $\hH^{\chi}$ the closure of $\cA\otimes\hH_{\chi}/\mN_{\chi}$ with respect to this scalar product. We can now define $\pi^{\chi}$, a representation of $\cA$ on $\cA\otimes\hH_{\chi}/\mN_{\chi}$ by linear extension of 
\begin{equation}
  \pi^{\chi}(A)V_{\chi}(B\otimes w)=V_{\chi}(AB\otimes w)
\end{equation}
which is well defined because $\pi^{\chi}(A)\mN_{\chi}\subseteq\mN_{\chi}$.


Let us now prove that $\| \pi^{\chi}(A) \|\leq \| A \|$. From equation \eqref{cor:BeAAeB} used in $\mfM^n(\cA)$ with $B=\cun_n$,we know that
\begin{equation} \label{eq:r502061}
  0\leq A^*A\cun_n\leq \| A \|^2\cun_n.
\end{equation}
Now we consider any $B_1,\cdots,B_n\in\cA$ and we build the matrix
\[ 
  b=
\begin{pmatrix}
B_1&\cdots&B_n\\
0&\cdots&0\\
\vdots&\ddots&\vdots\\
0&\cdots&0
\end{pmatrix},\quad
  b^*=
\begin{pmatrix}
B_1^*  & 0      & \ldots&0\\
\vdots &\vdots  & \ddots&\vdots\\
B^*_n  &0&\ldots & 0
\end{pmatrix}.
\]
We conjugate \eqref{eq:r502061} with $b$: 
\[ 
  0\leq b^*A^*Ab\leq \| A \|^2b^*b,
\]
but $q$ is completely positive, then it respects the inequality:
\[ 
  q_n(b^*A^*Ab)\leq \| A \|^2q_n(b^*b)
\]
where $\dpt{ q_n }{ \mfM^n(\cA) }{ \mfM^n(\cB) }$ is defined by $\big( q_n(M) \big)_{ij}=q(M_ij)$. The definition of complete positivity of $q$  is precisely positivity of $q_n$. We consider now the representation $\pi_{\chi}$ of $\cB$ on $\hH_{\chi}$. Since $a\geq0$, we can find a $e\in\mfM^n(\cB)$ such that $q_n(a)=e^*e$; then
\begin{equation}
\begin{split}
\sum_{ij}\scal{ v_i }{ \pi_{\chi}\big( q(a_{ij})v_j \big) }&=\sum_{ij}\scal{ v_i }{ \pi_{\chi}(e^*e)_{ij}v_j }\\
        &=\sum_{ijk}\scal{ v_i }{ \pi_{\chi}(e^*_{ki})\pi_{\chi}(e_{kj} }\\
        &=\sum_{ijk}\scal{ \pi_{\chi}(e)_{ki}v_i }{ \pi_{\chi}(e_{ij})v_j }\geq0.
\end{split}
\end{equation}

Let us consider $\Psi=\sum_iV_{\chi}B_i\otimes v_i$ and compute
\begin{equation}
\begin{split}
\| \pi^{\chi}(A)\Psi \|^2&=\sum_{ij}( AB_i\otimes v_i , AB_j\otimes v_j )_0^{\chi}\\
        &=\sum_{ij}\scal{ v_i }{ \pi_{\chi}\big( q(B_i^*A^*AB_j) \big)v_j }_{\chi}\\
        &\leq \| A \|^2\sum_{ij}\scal{ v_i }{ \pi_{\chi}\big( q(B^*_iB_j) \big)v_j }_{\chi}\\
        &=\| A \|^2\sum_{ij}\scal{ B_i\otimes v_i }{ B_j\otimes v_j }_{\chi}\\
        &=\| A \|^2( V_{\chi}\sum_i B_i\otimes v_i , V_{\chi}\sum_jB_j\otimes v_j )^{\chi}\\
        &=\| A \|^2\| \Psi \|^2.
\end{split}
\end{equation}
Then $\| \pi^{\chi}(A)\Psi \|\leq \| A \|^2\| \Psi \|$ which proves that
\begin{equation}
\| \pi^{\chi}(A) \|\leq\| A \|^2.
\end{equation}
The formula
\begin{equation}
  \pi^{\chi}(A)V_{\chi}(B\otimes w)=V_{\chi}(AB\otimes w)
\end{equation}
defines a continuous representation $\pi^{\chi}$ on $\cA\otimes\hH_{\chi}/\mN_{\chi}$ which can be extended to a continuous representation on the whole $\hH^{\chi}$. This extension fulfils $\pi^{\chi}(A^*)=\pi^{\chi}(A)^*$.

We define $\dpt{ W }{ \hH_{\chi} }{ \hH^{\chi} }$ by
\begin{equation}
  Wv=V_{\chi}\cun\otimes v.
\end{equation}

\subdem{The map $W$ is a partial isometry}

In order to prove that $W$ is a partial isometry, just compute
\begin{equation}
\begin{split}
( Wv , Ww )^{\chi}&=( V_{\chi}\cun\otimes v , V_{\chi}\cun\otimes w )^{\chi}\\
        &=( \cun\otimes v , \cun\otimes w )_0^{\chi}\\
        &=\scal{ v }{ w }_{\chi}.
\end{split}
\end{equation}

\subdem{Adjoint of $W$}

We claim that $\dpt{ W^* }{ \hH^{\chi} }{ \hH_{\chi} }$, $W^*V_{\chi} A\otimes v=\pi_{\chi}(q(A))v$ is the adjoint of $W$. Recall that the definition of the adjoint requires that 
\[ 
  \scal{ w }{ W^*\psi }_{\chi}=( Ww , \psi )^{\chi}
\]
for all $w\in\hH_{\chi}$ and all $\psi\in\hH^{\chi}=\overline{ A\otimes\hH_{\chi}/\mB_{\chi} }$. A $\psi\in\hH^{\chi}$ can be written under the form $V_{\chi}A\otimes v$ with $A\in\cA$ and $v\in\hH_{\chi}$; then
\begin{equation}
\begin{split}
\scal{ w }{ W^*V_{\chi}A\otimes v }_{\chi}&=\scal{ w }{ \pi_{\chi}(q(A))v }_{\chi}\\
        &=( \cun\otimes w , A\otimes v )_0^{\chi}\\
        &=( V_{\chi}\cun\otimes w , V_{\chi}A\otimes v )^{\chi}\\
        &=\scal{ Ww }{ \psi }
\end{split}
\end{equation}
as expected. One can check that $W^*W=\mtu$ and that $W^*\pi^{\chi}(A)W=\pi_{\chi}(q(A))$ because
\begin{equation}
  W^*\pi^{\chi}(A)Wv=W^*\pi^{\chi}(A)V_{\chi}(\cun\otimes v)
        =W^*V_{\chi}(A\otimes v)
        =\pi_{\chi}(q(A)).
\end{equation}

\subdem{Last point: \eqref{eq:stinun}$\Rightarrow$\eqref{eq:stindeux}}

Since $W$ is a partial isometry, $P=WW^*$ is the projection on the image of $W$ and $W^*W$ ($=\cun$) is the projector on the subspace of $\hH_{\chi}$ on which $W$ is isometric; this space is $\hH_{\chi}$ itself. The $\tilde\hH_{\chi}=P\hH^{\chi}=\hH^{\chi}$ and $U=W$. Then
\begin{equation}
  U\pi_{\chi}\big( q(A) \big)U^{-1}=W\pi_{\chi}\big( q(A) \big)W^*
        =WW^*\pi^{\chi}(A)WW^*
        =P\pi^{\chi}(A)P
\end{equation}
This concludes the proof of theorem \ref{tho:stinespring}.

\end{proof}


\begin{remark}
If on the one hand $q(\cun)$ is not $\cun$, then the construction works, but $W$ is no more a partial isometry and we have
\[ 
  \| W \|^2=\| q(\cun) \|,
\]
so $\hH_{\chi}$ can not be seen as a subspace of $\hH^{\chi}$ by the map $\dpt{W}{ \hH_{\chi} }{ \hH^{\chi} }$.

If on the other hand $\cA$ or $\cB$ is not unital, then works if $q$ can be extended (keeping positive) to the unitization of $\cA$ in such a way that it conserves the identity in (the unitization of ) $\cB$.
\end{remark}

\begin{proposition}
Any positive map $q\colon \cA\to \cB$ from a commutative unital $C^*$-algebra $\cA$ is completely positive.
\end{proposition}

The proof will be decomposed into several propositions. Let us begin by a remark: from theorem \ref{thoGelfand}, we can write $\cA=C(X)$ for a certain locally compact Hausdorff space $X$. So we can identify $\mfM^n(C(X))$ with $C(X,\mfM^n(\eC))$ because to each $a\in\mfM^n(C(X))$, ($a_{ij}$ is a map $\dpt{ a_{ij} }{ X }{ \eC }$) we make correspond the map $\dpt{ \eta }{ X }{ \mfM^n(\eC) }$ defined by $\eta(x)_{ij}=a_{ij}(x)$.


\begin{proposition}
The set of finite linear combinations of elements $F$ of the form
\[ 
  F(x)=\sum_i f_i(x)M_i
\]
with $f_i\in C(X)$ and $M_i\in\mfM^n(\eC)$ is dense in $C(X,\mfM^n(\eC))$.
 \label{prop:lencombpart}
\end{proposition}


\begin{proof}
Let $G\in C(X,\mfM^n(\eC))$ and $\varepsilon>0$. Continuity of $G$ makes the set
\[ 
  \mO_x^{\varepsilon}=\{ y\in X\tq \| G(x)-G(y) \|\leq\varepsilon \}
\]
open for all $x\in X$. These set give an open covering of the compact space $X$, then we can extract a finite subcovering and build an unity partition $\varphi_i$. We define $F_l\in C\big( X,\mfM^n(\eC) \big)$ by
\begin{equation} \label{eq:Fllim}
  F_l(x)=\sum_{i=1}^l\varphi_i(x)G(x_i).
\end{equation}
We have
\begin{equation}
\| F_l(x)-G(x) \|=\| \sum_{i=1}^l\varphi(x)( G(x_i)-G(x) ) \|
        \leq \sum \varphi_i(x)\| G(x_i)-G(x) \|
        \leq \sum\varphi_i\varepsilon
    =\varepsilon
\end{equation}
Then $\| F_j-G \|=\sum_{x\in X}\| F_l(x)-G(x) \|\leq\varepsilon$ and the sequence $F_l$ converges to $G$. It proves the density.
\end{proof}


\begin{proposition}
When $\{ M_i \}$ is a basis of $\mfM^n(\eC)$ composed with positive elements, an element $F\in C(X,\mfM^n(\eC))$ of the form $F(x)=\sum_if_i(x)M_i$ is positive if and only if it each of $f_i$ is positive.
\end{proposition}

\begin{proof}
We know that $F\in C(X,\mfM^n(\eC))$ is positive when $F(x)$ is positive in $\mfM^n(\eC)$ for all $x\in X$. We have $F(x)=\sum_if_i(x)M_i$, but positivity of $F(x)$ requires $F(x)=F(x)^*$ and then $f_i(x)=f_i(x)^*$ because $M_i$ is positive.

\end{proof}

\begin{probleme}
  \cite{Landsman} states a stronger result that seems wrong to me because \( -3+7\) is positive.
\end{probleme}

\begin{proposition}
If $G\in C(X,\mfM^n(\eC))$ is positive, then there exists a sequence $F_k\geq 0$ such that $\lim_{k\to\infty}F_k=G$
\end{proposition}

\begin{proof}
Each element $F_l$ \eqref{eq:Fllim} is positive because  $G(x_i)$ is positive for all $x_i$.
\end{proof}

\begin{probleme}
    There are too much unclear thinks in my mind; I do not finish the proof.
\end{probleme}

The main result is the following proposition.

\begin{proposition}
If $\cA$ is a commutative unital $C^*$-algebra, then any positive map $\dpt{ q }{ \cA }{ \cB }$ is completely positive.
\end{proposition}


\section{Representations}
%+++++++++++++++++++++++++

As notational convention, when $H$ is a Hilbert space, we denote by $\mL(H)$\nomenclature{$\mL(H)$}{Space of continuous endomorphisms of $H$} the set of the continuous endomorphism of $H$. Topology on $\mL(H)$ is discussed in subsection \ref{subsec_topomL}.

When $\pi$ is a representation of $\cA$ in $H$ and $\xi\in H$, the space $\overline{ \pi(\cA)\xi }$ is a closed subspace of $H$ stable under $\pi(\cA)$.

\begin{definition}
A vector $\xi\in H$ is said \defe{totalizing}{totalizing vector} for a
representation $\pi$ of $\cA$ if $\overline{ \pi(\cA)\xi }=H$.
\end{definition}

\begin{proposition}
Let $\cA$ be an involutive algebra, $H$ an hermitian space and $\pi$ a representation of $\cA$ in $H$. Then the following facts are equivalent 
:
\begin{enumerate}
\item The only closed subspace in $H$ which are stable for $\pi(\cA)$ 
are $\{o\}$ and $H$.
\item The subset of $\mL(H)$ which commutes with $\pi(\cA)$ is reduced 
to $\mC$.
\item Any non zero vector in $H$ is totalizing for $\pi$, or $\pi$ have 
dimension $1$.
\end{enumerate}
 \label{prop:reprez_topo}
\end{proposition}

\begin{proof}
We begin proving that (ii) implies (iii). Let $\xi\in H$, $\xi\neq 0$. 
If $\pi(\cA)\xi$ is not everywhere dense in $H$, (i) makes
$\pi(\cA)\xi=0$. Then $\eC\xi$ is stable under $\pi(\cA)$. But 
$\eC\xi\neq\{o\}$, then $\eC\xi=H$. Thus $H$ has dimension $1$ and $\pi$ 
is the null representation.

Now, we prove that (iii) implies (i). Let $K\neq\{o\}$ be a closed 
vector  
subspace of $H$ stable under $\pi(\cA)$. We have to show that $K=H$. If 
$\dim H=1$, it is obvious. Let us consider a non zero  $\xi\in K$. Since 
$K$ is stable, $\pi(\cA)\xi\subset K$, but (iii) implies
 $\overline{ \pi(\cA)\xi }=H$. Thus $K=H$.

We turn our attention to the equivalence between (i) and (ii). First 
(ii) implies (i). We consider $K$, a vector subspace of $H$ stable under 
$\pi(\cA)$. We want $K=\{o\}$ or $K=H$. Let us consider 
$\dpt{p_K}{H}{K}$, the orthogonal projection. Since $K$ is a vector 
subspace, it makes sense to write $H\ominus K$.

Let us consider $\xi\in K$ and $\eta\in H\ominus K$. For any $A\in\cA$, 
$\pi(A^*)\xi\in K$ because $\pi(A)\xi\in K$. This yields
\[
   \langle \pi(A)\eta|\xi\rangle =\langle\eta|\pi(A^*)\xi\rangle=0,
\]
then $\pi(A)\eta\in H\ominus K$. We can conclude that 
\[
   [p_K,\pi(\cA)]=0
\]
because $\pi(\cA)p_K\xi=\pi(\cA)\xi=p_K\pi(\cA)\xi$, $\pi(\cA)p_K\eta=0$  and $p_K\pi(\cA)\eta=0$ from $\langle \pi(\cA)\eta|\xi\rangle =0$.

From (ii), $p_K$ is then a scalar operator: $p_K=0$ or $p_K=id$, \emph{i.e.} $K=\{o\}$ or $K=H$.

Finally, we show the implication from (i) to (ii). Let $T$ be an element of $\mL(H)$ which commute with the whole $\pi(\cA)$; we have to show that $T$ is scalar. Since it is clear that $T+T^*$ and $T-T^*$ also commute with $\pi(\cA)$, we can suppose $T=T^*$.

We had shown that a projector $p_K$ commutes with $\pi(\cA)$ if $K$ is stable under $\pi(\cA)$, closed and a sub vector space of $H$. This is the case of the eigenspaces because $(T-\lambda\mtu)v=0$, then $(T-\lambda\mtu)\pi(\cA)v=0$ because $[T,\pi(\cA)]=0$.

The spectral projector of $T$ commute with $\pi(\cA)$. Thus, the eigenspaces $H_{\lambda}$ are stable under $\pi(\cA)$, thus (by (i)) these are only $\{o\}$ and $H$. In other words $(T-\lambda\mtu)v=0$ has solutions which are on two subspaces whose projectors are $0$ and $1$. On the space on which $p_{\lambda}=1$, $T=\lambda\mtu$ and the one where $p_{\lambda}=0$ is $\{0\}$ then $T=0$.

\begin{probleme}
    We have to show that every $v$ belong to one of these two spaces. Is it because $T$ is hermitian, or do we need the compact assumption?
\end{probleme}

\end{proof}

\begin{definition}
Let $\cA$ be an involutive algebra, $H$ a hermitian space and $\pi$ a representation of $\cA$ in $H$. We say that $\pi$ is \defe{topologically irreducible}{irreducible!topologically} if it fulfils proposition \ref{prop:reprez_topo}. The representation $\pi$ is \defe{algebraically irreducible}{irreducible!algebraically} if the only stable vector subspaces of $H$ under $\pi(\cA)$ are $\{0\}$ and $H$.
\end{definition}

When $\dim H=\infty$, the second notion is stronger because it excludes the case where one has an \emph{open} stable subspace. Point 2.8.4 in \cite{Dixmier} shows that in the case of $C^*$-algebras, a topologically irreducible representation is automatically algebraically irreducible, so one can simply speak about irreducible representations.

%%%%%%%%%%%%%%%%%%%%%%%%%
%
   \section{Pure states}
%
%%%%%%%%%%%%%%%%%%%%%%%%


A subset $C$ of a vector space is \defe{convex}{convex} if for all $\lambda\in[0,1]$ and $v$, $w\in C$, the element $\lambda v+(1-\lambda)w$ belongs to C.

An \defe{extreme point}{extreme point} of a convex set $K$ is an element $\omega\in K$ which can de written under the form $\omega=\lambda\omega_1+(1-\lambda)\omega_2$ with $\lambda\in[0,1]$ only for $\omega_1=\omega_2=\omega$.

\begin{definition}
An extreme point of the state space $\etS(\cA)$ is a \defe{pure state}{state!pure}\index{pure state} and states that are not pure are \defe{mixed states}{mixed!state}\index{state!mixed}.
\end{definition}

The set of extreme points of the convex set $K$ is denoted by $\partial_cK$ and is called the \defe{extreme boundary}{boundary!extreme} of $K$.  An extreme point in the state space $\etS(\cA)$ of a $C^*$-algebra is a \emph{pure state} and other states are \defe{mixed states}{mixed!state}\index{state!mixed}. As notation, $\partial_c(\etS(\cA))$ is denoted by $\etP(\cA)$ or simply $\etP$ when there are no ambiguity.

\subsection*{Example \texorpdfstring{$\cA=\eC\oplus\eC$}{A=C+C}}
 Points are given by $(\lambda,\mu)$. Let's consider the convex set of states $r\in[0,1]$ defined by $r(\lambda,\mu)=(1-r)\lambda-r\mu$.  Extreme points are given by $0$ and $1$: $0(\lambda, \mu)=\lambda$ and $1(\lambda,\mu)=\mu$.

\subsection*{Example: \texorpdfstring{$\cA=\mfM^2(\eC)$}{A=M2C}} 
We can identify $\mfM^2(\eC)$ with its dual in the following way. A linear form $\omega$ on $\cA$ can always be written as
\[ 
  \omega
\begin{pmatrix}
 A_{11}&A_{12}\\
A_{21}&A_{22}
\end{pmatrix}
=\omega_{11}A_{11} +\omega_{12}A_{21}+ \omega_{21}A_{12}+ \omega_{22}A_{22} 
\]
So to each form $\omega$, one can associate the matrix of $\omega_{ij}$ and the following holds:
\[ 
  \omega(A)=\tr(\omega A)
\]
The identification between $\mfM^2(\eC)$ and its dual is then well given by $\omega\simeq(\omega_{ij})$. Let us see in terms of this identification the set $\etS(\cA)$. The condition $\omega(A)\geq 0$ imposes to the matrix $(\omega_{ij})$ to be positive and the condition $\omega(\cun)=1$ imposes $\tr\omega=1$. Then $\etS(\cA)$ is parametrized by 
\[ 
  \rho=\frac{ 1 }{2}
\begin{pmatrix}
1+x&y+iz\\
y-iz&1-x
\end{pmatrix}
\]
where $x$, $y$, $z\in\eR$. The pure states are such matrices $\rho$ with $x^2+y^2+z^2=1$.

If $M$ is a $*$-algebra in $\oB(\hH)$, the \defe{commutant}{commutant} $M'$ is 
\[ 
  M'=\{ A\in\oB(\hH)\tq [A,m]=0    \forall\, m\in M \}.
\]

\begin{proposition} 
The following properties are equivalent:

\begin{enumerate}
 \item \label{enumgz} The representation $\pi(\cA)$ is irreducible in $\hH$.
\item\label{enumgi} The commutant of $\pi(\cA)$ in $\oB(\hH)$ is
\[ 
  \pi(\cA)'=\{ \lambda\cun\tq\lambda\in\eC \}.
\]
\item \label{enumgii} $\pi(\cA)''=\oB(\hH)$.

\item \label{enumgiii} Each vector $\Omega\in\hH$ is cyclic for $\pi(\cA)$.

\end{enumerate}
 \label{prop_equiv_rep_irred}
\end{proposition}

\begin{proof}
\ref{enumgii}$\Rightarrow$\ref{enumgi}
We have to see that an operator which commutes with the whole $\oB(\hH)$, then it is a multiple of identity. If $[a,A]=0$ for all $A\in\oB(\hH)$, then it commutes in particular with an operator $A$ which leaves the basis vector $e_{\beta}$ (and only this basis vector). In this case, $ae_{\beta}=\lambda_{\beta}e_{\beta}$. We conclude that $a$ must be diagonal. Since $a$ must also commute with an operator which leaves $e_{\alpha}+e_{\beta}$ unchanged, we conclude that $\lambda_{\alpha}=\lambda_{\beta}$, so that $a=\lambda\mtu$.

\ref{enumgz}$\Rightarrow$\ref{enumgi} Will be done later.
\ref{enumgi}$\Rightarrow$\ref{enumgz} Let us suppose that $\pi(\cA)'=\eC\cun$ and that $\pi$ is irreducible; we will find out a contradiction. We have a non trivial subspace of $\sH$ stable under $\pi(\cA)$. The projection operator on this space commutes with the whole $\pi(\cA)$ although it is not a multiple of identity.

\ref{enumgz}$\Rightarrow$\ref{enumgiii} We proceed by contradiction once again.  Let $\psi\in\hH$ such that $\pi(\cA)\psi$ is not dense in $\hH$ and $P$ be the projection on the closure of $\pi(\cA)\psi$. Lemma \ref{lem_preGNS} assures that $P\in\pi(\cA)'$. Consequently $\pi(\cA)'\neq \eC\cun$ and $\pi(\cA)$ is not irreducible by \ref{enumgi}.

\ref{enumgiii}$\Rightarrow$\ref{enumgz} If $\psi$ belongs to a (non trivial) invariant subspace, then $\pi(\cA)\psi$ cannot be dense because it is a proper subspace of $\hH$.

\end{proof}

\begin{probleme}
    There are still unfinished points in that proof.
\end{probleme}


Let us point out the following part of the proposition:

\begin{lemma}[Schur's lemma]
The representation $\pi$ on $\cA$ is irreducible if and only if the commutant of $\pi(\cA)$ in $\oB(\hH)$ is
 \[ 
  \pi(\cA)'=\{ \lambda\cun \}_{\lambda\in\eC}.
\]

\end{lemma}


\begin{lemma} 
Let $\hat Q$ be a bounded quadratic form on a Hilbert space $\hH$. There exists a bounded operator $Q$ on $\hH$ such that for each $\psi,\phi\in\hH$ we have 
\[
\hat Q(\psi,\phi)=\scald{ \psi}{Q\phi }
\]
 and $\| Q \|\leq C$ where $C$ is the ``bounding constant'': $| \hat Q(\psi,\phi) |\leq \| \psi \|\| \phi \|$.
Moreover if 
%
\begin{equation}  \label{eq_r19032}
\hat Q(\phi,\psi)=\overline{ \hat Q(\psi,\phi) }
\end{equation}
 the operator $Q$ will be selfadjoint.
\label{lem_r19031}
\end{lemma}

\begin{proof}
Let us fix a $\psi\in\hH$ and look at the map $\phi\mapsto\hat Q(\psi,\phi)$. It is a bounded form, so Riesz theorem gives the existence of a $\Omega\in\hH$ such that $\hat Q(\psi,\phi)=\scald{ \Omega }{ \phi }$. We can define $Q$ by $Q\psi=\Omega$. It is clear that is is self-adjoint if equation \eqref{eq_r19032} is satisfied. 

From equalities $\hat Q(\psi,\phi)=\scald{ \Omega }{ \phi }=\scald{ Q\psi }{ \phi }$, we find
 \begin{equation}
\begin{split}
\| Q\psi \|^1&=| \scald{ Q\psi }{ Q\psi } |\\
        &=| \hat Q(Q\psi,\psi) |\\
        \leq \| Q\psi \|\| \psi \|\\
        &=\| Q \|\| \psi \|^2.
\end{split}
\end{equation}
Taking supremum on $\| \psi \|=1$, we find $\| Q \|^2\leq C\| Q \|$ and then
\[ 
  \| Q \|\leq C.
\]

\end{proof}

\begin{theorem} 
   The GNS representation $\pi_{\omega}$ of a state $\omega\in\etS(\cA)$ is irreducible if and only if $\omega$ is pure.
\label{tho_GNS_irred_pure}
\end{theorem}

\begin{proof}
Let us begin to suppose that $\omega$ is a pure state and that $\pi_{\omega}$ is reducible. Then the projection $P$ onto the invariant subspace $K$ of $\pi(\cA)$ belongs to $\pi(\cA)'$ (see the proof of Schur's lemma). Let $\Omega_{\omega}$ be the cyclic vector of $\pi_{\omega}$. 

If $P\Omega_{\omega}=0$, then for each $A\in\cA$ we have
\[ 
  0=\pi_{\omega}P\Omega_{\omega}=P\pi_{\omega}(A)\Omega_{\omega}.
\]
Since $\Omega_{\omega}$ is cyclic, it proves that $P=0$ which is impossible if $\pi_{\omega}$ is reducible. For the same reason, $P^{\perp}\Omega_{\omega}$ is neither not possible because it should implies that $P=\cun$. 

We define the two following states on $\cA$:
\begin{subequations}
\begin{align}
  \psi(A)&=k\scald{ P\Omega_{\omega} }{ \pi(A)P\Omega_{\omega} }\\
  \psi^{\perp}(A)&=l\scald{ P^{\perp}\Omega_{\omega} }{ \pi_{\omega}(A)P^{\perp}\Omega_{\omega} }
\end{align}
\end{subequations}
From definition of a projection, we have $\scald{ Px }{ Py }=\scald{ P^*Px }{ y }=\scald{ Px }{ y }$. Taking any $k$, $\lambda=1/k$ and $l=1/(1-1/k)$, using the relation $\omega(A)=\scald{ \Omega_{\omega} }{ \pi_{\omega}(A)\Omega_{\omega} }$ we find
\[ 
  \omega=\lambda\psi+(1-\lambda)\psi^{\perp},
\]
which is in contradiction with the fact that $\omega$ is pure.

We now suppose that $\pi_{\omega}$ is irreducible, and that $\omega$ reads
\[ 
  \omega=\lambda\omega_1+(1-\lambda)\omega_2
\]
with $\lambda\in[0,1]$ and $\omega_1,\omega_2\in\etS(\cA)$. We will prove that $\omega_1$ is proportional to $\omega$, so that $\omega$ is pure. Since elements of $\etS(\cA)$ are positive and $(1-\lambda)\geq0$, the form $\omega-\lambda\omega_1=(1-\lambda)\omega_2$ is positive. Therefore for all $A\in\cA$, we have $\lambda\omega_1(A^*A)\leq\omega(A^*A)$. From equation \eqref{eq:omABleq}, we find
%
\begin{equation} \label{eq_r19031}
\begin{split}
| \lambda\omega_1(A^*B) |&\leq\lambda^2\omega_1(A^*A)\omega_1(B^*B)\\
        &\leq\omega(A^*A)\omega(B^*B).
\end{split}
\end{equation}
It allows us to define a quadratic form $\hat Q$ on $\pi_{\omega}(\cA)\Omega_{\omega}$ by
\begin{equation}
\hat Q\big( \pi_{\omega}(A)\Omega_{\omega},\pi_{\omega}(B)\Omega_{\omega} \big):=\lambda\omega_1(A^*B).
\end{equation}

\subdem{$\hat Q$ is well defined}

We have to prove that $\pi_{\omega}(A_1)=\pi_{\omega}(A_2)\Omega_{\omega}$ implies
\[ 
 \hat Q(\pi_{\omega}(A_1)\Omega_{\omega},\cdot)=\hat Q(\pi_{\omega}(A_2)\Omega_{\omega},\cdot).
\]
 Equation \eqref{eq:piomomaesm} gives
\[
  \| \pi_{\omega}(A)\Omega_{\omega} \|^2=\omega(A^*A)
\]
and makes that $\omega\big( (A_1-A_2)^*(A_1-A_2) \big)=0$. Thus, using \eqref{eq_r19031}, we have
\begin{equation}
\begin{split}
\Big| & \hat Q(\pi_{\omega}(A_1)\Omega_{\omega},\pi_{\omega}(B)\Omega_{\omega})-\hat Q(\pi_{\omega}(A_2)\Omega_{\omega},\pi_{\omega}(B)\Omega_{\omega})      \Big|^2\\
        &=\Big|   \lambda\omega_1(A_1^*B)-\lambda\omega_1(A_2^*B)    \Big|^2\\
        &=\Big|  \lambda\omega_1\big( (A_1-A_2)^*B \big)  \Big|^2
        \\&\leq0.
\end{split}
\end{equation}
The whole is finally zero.
 

\subdem{$\hat Q$ is bounded}

Equation \eqref{eq:piomomaesm} together with the equality $| \lambda\omega_1(A^*B) |^2\leq\omega(A^*A)\omega(B^*B)$ give
\begin{equation}
\begin{split}
  \Big|  \hat Q(\pi_{\omega}(A)\Omega_{\omega},\pi_{\omega}(B)\Omega_{\omega})   \Big|^2
    &=| \lambda\omega_1(A^*B) |^2\\
    &\leq\omega(A^*A)\omega(B^*B)\\
    &=\| \pi_{\omega}(A)\Omega_{\omega} \|^2\| \pi_{\omega}(B)\Omega_{\omega} \|^2.
\end{split}
\end{equation}
Therefore $| \hat Q(\psi,\phi) |^2\leq\| \psi \|\| \phi \|$ and $\hat Q$ is bounded.

The quadratic form $\hat Q$ can be continuously extended to the whole $\hH_{\omega}$. From general property $\omega(A^*)=\overline{ \omega(A) }$ (when $\omega$ is a state),
\[ 
  \hat Q(\phi,\psi)=\overline{ \hat Q(\psi,\phi) }.
\]
Lemma \ref{lem_r19031} immediately applies to our $\hat Q$, so we have an operator $Q$ such that $\scald{ \psi }{ Q\phi }=\hat Q(\psi,\phi)$. Therefore
%
\begin{equation} \label{eq_1903r3}
  \scald{ \pi_{\omega}(A)\Omega_{\omega}}{Q\pi_{\omega}(B)\Omega_{\omega}}=\lambda\omega_1(A^*B)
\end{equation}
Since $\pi$ is a representation, we have for all $A$, $B\in\cA$:
 \begin{equation}
\begin{split}
  \scald{ \pi_{\omega}(A)\Omega_{\omega} }{ Q\pi_{\omega}(B)\Omega_{\omega} }
     &=\hat Q\big( \pi_{\omega}(B^*A)\Omega_{\omega},\Omega_{\omega} \big)\\
        &=\scald{ \pi_{\omega}(B^*A)\Omega_{\omega} }{ Q\Omega_{\omega} }\\
        &=\scald{ \pi_{\omega}(A)\Omega_{\omega} }{ \pi_{\omega}(B)Q\Omega_{\omega} }.
\end{split}
\end{equation}
This proves that $[Q,\pi_{\omega}(C)]=0$ for each $C\in\cA$. Therefore $Q\in\pi_{\omega}(\cA)'$ and there exists a $t\in\eR$ such that $Q=t\cun$. Using equation \eqref{eq_1903r3} and  \eqref{eq_r1903r4}, we find 
%
\begin{equation}
\begin{split}
t\omega(A^*B)&=t\scald{ \pi_{\omega}(A)\Omega_{\omega} }{ \pi_{\omega}(B)\Omega_{\omega} }\\
&=\lambda\omega_1(A^*B)\\
&=\hat Q\scald{ \pi_{\omega}(A)\Omega_{\omega} }{ \pi_{\omega}(B)\Omega_{\omega} }\\
\end{split}
\end{equation}
So $t\omega(A^*B)=\lambda\omega_1(A^*B)$; thus $\omega$ and $\omega_1$ are proportional and the decomposition
%
\[ 
  \omega=\lambda\omega_1+(1-\lambda)\omega_2
\]
is only possible for $\omega_1=\omega_2=\omega$. This proves that $\omega$ is a pure stare.

\end{proof}


\begin{corollary}
If the representation $\big( \pi(\cA),\hH \big)$ is irreducible, then the GNS representation $\big( \pi_{\omega}(\cA),\hH_{\omega} \big)$ build from any vector state (corresponding to $\Psi\in\hH$ such that $\| \Psi \|=1$) is unitary equivalent to $(\pi(\cA),\hH)$.
 \label{cor_GNSirredst}
\end{corollary}

\begin{proof}
    Since $\pi$ is irreducible, any vector in $\hH$ is cyclic from proposition \ref{prop_equiv_rep_irred}. So $\pi$ is cyclic and proposition \ref{prop:cyclequivGNS} concludes.

\end{proof}

\begin{corollary}
All irreducible representation of a $C^*$-algebra is (up to an equivalence) the GNS construction from a pure state.
\end{corollary}


\begin{proof}
We know that an irreducible representation is unitary equivalent to a GNS representation, but the GNS representation will only be irreducible when $\omega$ is pure state.
\end{proof}


\begin{proposition} 
A state $\omega$ is pure if and only if for each positive functional $\rho$ such that $0\leq \rho\leq\omega$, there exists a $t\in\eR^+$ such that $\rho=t\omega$.
\label{prop_pureiff}
\end{proposition}

\begin{proof}
From corollary \ref{cor_csa_unit} and  proposition \ref{prop_st_unit_ext} we can suppose that $\cA$ is unital because if not, the notion of positivity is defined from unitization. When $\rho=0$ or $\rho=\omega$, the result is trivial. Now we suppose that $0\neq\rho\neq\omega$.

\subdem{Direct sense}

We know that $\omega$ is pure and $0\leq\rho\leq\omega$, $0<\rho(\cun)<1$ because $\omega-\rho$ is positive. Therefore
\[ 
  \| \omega-\rho \|=\omega(\cun)-\rho(\cun)=1-\rho(\cun),
\]
thus $\rho(\cun)=1$ should implies $\| \omega-\rho \|=0$ and then $\omega=\rho$. The possibility $\rho(\cun)=1$ is also not possible. Thus $\rho(\cun)$ is between $0$ and $1$.

This allows us to consider the states
\[ 
  \frac{ \omega-\rho }{ 1-\rho(\cun) },\quad\text{and}\quad\frac{ \rho }{ \rho(\cun) }.
\]
For $\lambda=1-\rho(\cun)$,
\[ 
  \omega=\lambda\frac{ \omega-\rho }{ 1-\rho(\cun) }-(1-\lambda)\frac{ \rho }{ \rho(\cun) }.
\]
Since $\omega$ is pure, it implies $\frac{ \omega-\rho }{ 1-\rho(\cun) }=\frac{ \rho }{ \rho(\cun) }$. So, $\rho=\rho(\cun)\omega$.

\subdem{Inverse sense}

Let us consider a decomposition $\omega=\lambda\omega_1+(1-\lambda)\omega_2$ of $\omega$. In the proof of theorem \ref{tho_GNS_irred_pure}, we find that $0\leq\lambda\omega_1<\omega$. Then the assumption says that $\lambda\omega_1=\omega=\omega_2$ and the normalization makes automatically $\omega_1=\omega=\omega_2$ which proves that $\omega$ is pure.

\end{proof}


\begin{lemma}
Let $\cA$ be a $C^*$-algebra and $\rho$, a positive form on $\cA$. If we pose $M=\ker(\rho)$, $N=\{ A\in\cA\tq \rho(A^*A)=0 \}$, then
\[ 
  N+N^*\subseteq M
\]
and if $\rho$ is pure, $N+N^*=M$.
\end{lemma}

\begin{proof}
The functional $\rho$ being positive, equation  \eqref{eq:defprodetat} holds. With $A=\cun$ we find 
\begin{equation}  \label{eq_pos_Bdtho}
| \rho(B)^2 |\leq\rho(\cun)\rho(B^*B)
\end{equation}
 and thus $\rho(A^*A)=0$ implies $\rho(A)=0$.

No proof for the second part.
\end{proof}


\begin{theorem}
The space of pure states of the (commutative) $C^*$-algebra $C_0(X)$ (the space of functions which are decreasing to zero at infinity) endowed with the relative $w^*$-topology is homeomorphic to $X$.
\end{theorem}

\begin{probleme}
    The following proof is buggy and very unsure.
\end{probleme}

\begin{proof}
The $C^*$-algebra  $\cA=C_0(X)$ is not specially unital. If it is not, Gelfand theorem \ref{thoGelfand} says that there exists a locally compact and Hausdorff space $Y$ such that $\cA$ is isomorphic to $C_0(Y)$. If $\cA=C_0(X)$ with a non compact $X$, we begin to prove that the unitization is $\cA_{\cun}=C(\tilde X)$ where $\tilde X$ is the one point compactification of $X$. From proposition \ref{prop_unitariz_csa}, we have an unique unital $C^*$-algebra $\cA_{\cun}$  with an isometric morphism $\cA\to\cA_{\cun}$ such that $\cA_{\cun}/\cA\simeq \eC$.

 Therefore, we have to prove that $C(\tilde X)$ is a suitable $\cA_{\cun}$ and so it will be the unique unitization of $C_0(X)$. The identity map $\id\colon C_0(X)\to C(\tilde X)$ works. We have to check that $C(\tilde X)/C_0(X)\simeq\eC$. By identifying all functions of $C(\tilde X)$ which only differ by a function of $C_0(X)$, there are in fact only one function for each complex number $z$: the which is constant (or another which is $z$ at $\infty$).

We have proved that if $\cA=C_0(X)$, then $\cA_{\cun}=C(\tilde X)$. From proposition \ref{prop_st_unit_ext} pure states on $C_0(X)$ uniquely extend to a pure state on $C(\tilde X)$. So if $X$ is non compact, we do not loss anything by considering $C(\tilde X)$ instead of $C_0(X)$. We still have to prove that taking $C(\tilde X)$ instead of $C(X)$ does not \emph{gain} anything: we must have $\etS\big( C_0(X) \big)=\etS\big( C(\tilde X) \big)$. In the case of unital $C^*$-algebras, pure states are linear functionals. The way to extend linear functionals from $\cA$ to $\cA_{\cun}$ is the same as the one to extend states.

If $X$ is compact, then $C_0(X)=C(X)$. Hence we are in both case ($X$ compact or not) reduced to prove the theorem for $C(X)$ with compact $X$.
 
Following proposition \ref{prop:comHauffhomeo}, $\Delta(C(X))$ is homeomorphic to $X$ because the latter is compact and Hausdorff. We have now to prove that there exists an homeomorphism between $\Delta(C(X))$ and set of pure states on $C(X)$. We are going to prove a bijection. Let on the one hand $\omega_x\in\Delta(C(X))$ be defined by
\begin{equation}
\begin{aligned}
 \omega_x\,:\,C(X)&\to \eC \\ 
\omega_x(f)&= f(x),
\end{aligned}
\end{equation}
and on the other hand a functional $\rho$ such that $0\leq\rho\leq\omega_x$. We have $\ker(\omega_x)\subset\ker(\rho)$. When $f$ is positive, $0\leq\rho(f)\leq\omega_x(f)$, so for any function,
\[ 
  o\leq\rho(f^*f)\leq\omega_x(f^*f),
\]
but $\omega_x$ is multiplicative, therefore $\rho(f^*f)\leq\omega_x(f^*)\omega_x(f)$. If $f\in\ker(\omega_x)$, we have $\rho(f^*f)=0$. Lemma concludes $\rho(f)=0$. So if $f\in\ker\omega_x$, we have $f(x)=0$. From theorem \ref{tho:ideal_kernel}, $\ker\omega_x$ is a maximal ideal while $\ker(\rho)$ is an ideal. So $\ker(\omega_x)$ is a maximal ideal contained in an ideal, therefore if $\rho\neq0$, $\ker(\omega_x)=\ker(\rho)$ which implies that $\rho=\lambda\omega_x$. Proposition \ref{prop_pureiff} concludes that $\omega_x$ is pure. 

\begin{probleme}
    Why is $\ker(\rho)$ an ideal?
\end{probleme}


Now we suppose that $\omega$ is pure and we take  $g\in C(X)$ such that $0\leq g\leq 1_X$ on $C(X)$, we define
\begin{equation}
\begin{aligned}
 \omega_g\,:\,C(X)&\to \eC \\ 
f&\mapsto \omega(fg) 
\end{aligned}
\end{equation}
 So  $\omega(f)-\omega_g(f)=\omega\big( f(1-g) \big)$ and if $f$ is positive, $\omega(f)-\omega_g(f)\geq O$. So for a certain $t\in\eR^+$, we have 
\[ 
  \omega_g=t\omega
\]
 because $0\leq\omega_g\leq\omega$ and proposition \ref{prop_pureiff}. In particular, $\ker (\omega_g)=\ker(\omega)$, hence when $f\in \ker(\omega)$, for any $g\in C(X)$, $fg \in \ker(\omega)$ because any function in $C(X)$ is a linear combination of functions $g$ with $0\leq g\leq 1_X$. This proves that $\ker (\omega)$ is an ideal. On the other hand, $\ker (\omega)$ is a maximal ideal because the kernel of any functional on a vector space has codimension 1, see page \pageref{pg_codimun}. Theorem \ref{tho:ideal_kernel} shows that $\omega$ is multiplicative. So $\omega\in\Delta\big( C(X) \big)$.

\end{proof}

\subsection{Existence of pure states}
%------------------------------------

It is possible for $\etS$ to do not contain pure states. It is the case when $\etS$ is a convex cone. Such a $C^*$-algebra  has no irreducible representations. We are going to prove that this case is not possible. We define the convex hull of the part $A$ of a vector space by
\[ 
  co(A)=\{ \lambda+(1-\lambda)w\text{ with }w\in A,\lambda\in[0,1] \}.
\]



\begin{theorem}
 A compact connected set $K$ in a locally convex vector space is the closure of the convex hull of its extreme points. In other words:
    \[ 
  K=\overline{ co(\partial_eK) }.
\]

\end{theorem}
\begin{proof}
No proof
\end{proof}


\begin{lemma}

 Let $\cA$ be an unital $C^*$-algebra  and $\cB$ an self-adjoint vector subspace of $\cA$ with $\cun\in\cB$. Let $F$ be the set of linear forms $g$ on $B$ such that


\begin{itemize}
\item  $g(A^*) = \overline{g(A)}$ for all $A\in\cB$,
 \item $g(A) \geq 0$ for all $A\in \cB\cap \cA^+$,
\item $g(\cun)=1$.
\end{itemize}
 Any element of $F$ can be extended into a state on $\cA$.
\label{lem_DixcBprol}
 \end{lemma}

\begin{theorem}  
For all $A\in\cA$ and a $a\in\sigma(A)$, we have a pure state $\omega_a$ on $\cA$ such that $\omega_a(A)=a$. There also exists a pure state $\omega$ such that $| \omega(A) |=\| A \|$.
\label{tho_existsetat}
\end{theorem}

\begin{proof}
Let us take an intermediary result in proof of lemma \ref{prop_st_unit_ext}:
\begin{equation}
\begin{aligned}
 \tilde\omega,:\,\eC A+\eC\cun&\to \eC \\ 
(\lambda A+\mu\cun)&\mapsto \lambda a+\mu 
\end{aligned}
\end{equation}
We extend this state by continuity and multiplicatively to $C^*(A,\cun)$ with formulas as $\tilde\omega_a(A^n)=a^n$. We have to check that this extension is pure: $\tilde\omega_a$ is positive and belongs to $\Delta(C^*(A,\cun))$. 

Positivity comes from assumption that $A\in\cA_{\eR}$. Indeed, $a\in\sigma(A)\subset\eR^+$ . So positives elements in $C^*(A,\cun)$ are even power (and completion) of $A$. So the images are even powers of  $a\in\eR$ and are therefore positives. 
The fact that $\tilde\omega_a$ is multiplicative on $C^*(A,\cun)$ comes from the fact that it is the same, in expressions as 
$\tilde\omega_a(xy)$, to distribute inside the $\tilde\omega_a$ and push out terms $a^n$ by linearity, or write $\tilde\omega_a(x)\tilde\omega_a(y)$ and distribute outside. So $\tilde\omega$ is a pure state in $C^*(A,\cun)$.

 We consider the set $K_a$ of extensions of $\tilde\omega_a$ which are states on $\cA$. The fact that $K_a$ is non empty comes from the lemma \ref{lem_DixcBprol}. 

Let us now prove that $K_a$ is convex. For, we take $\omega_1$ and $\omega_2$, two extensions of $\tilde\omega_a$ and we will prove that $\omega=\lambda\omega_1+(1-\lambda)\omega_2$ is an extension too. By linearity, $\omega$ is a state. It is an extension of $\tilde\omega_a$ because
\[ 
\begin{split}
\omega(\cun+A^2)&=\lambda\omega_1(\cun+A^2)+(A-\lambda)\omega_1(\cun+A^2)\\
        &=\lambda(1+a^2)+(1-\lambda)(1+a^2)\\
        &=1+a^2.
\end{split}  
\]

 In order to prove that $K_a$ is closed, we prove that its complement is open. Let $\omega\notin K_a$. We will prove that there exists $\varepsilon$ such that for all $\eta$ with $\| \omega-\eta \|\leq\varepsilon$. Let $\lambda$ be such that $\| \lambda A \|=1$ in such a manner that $\tilde\omega_a(\lambda A)=\lambda A$. Let $\omega(A)=s$ ($\omega\notin K_a$). We have
\[ 
\begin{split}
\| \omega-\eta \|&=\sup\{ | \omega-\eta |\text{ st } \| B \|=1 \}\\
        &\geq | (\omega-\eta)(\lambda A) |\\
        &=| \omega(\lambda A)-\eta(\lambda A) |,
\end{split}  
\]
but
\[ 
  | \omega(\lambda A)-\eta(\lambda A) |\leq \| \omega-\eta \|\leq\varepsilon.
\]
So $| s-\eta(\lambda A) |\leq \varepsilon$, and when $\varepsilon$ is small (for example when $\varepsilon<| s-\lambda a |$), $\eta(\lambda A)$ is close to $s$ which is different of $\lambda A$. This proves that $K_a$ is closed.

We know that $K_a$ is convex and closed. So it has at least one extreme point. Let $\omega_a$ be one of them. We are going to prove that it is also extreme in $\etS$. If not it can be decomposed as $\omega=\lambda\omega_1+(1-\lambda)\omega_2$. Taking the latter at $A$ shows that, on $C^*(A,\cun)$, the functionals $\omega$, $\omega_1$ and $\omega_2$ are equals. So $\omega_a$ cannot be an extreme point in $K_a$. This concludes the first part of the proof. 


Theorem \ref{tho:prop_sigma} says us that in a Banach algebra, $\sigma(A)$ is compact for all $A$. So there exists a $a\in\sigma(A)$ such that $r(A)=| a |$. With this $a$,
 \[ 
  | \omega_a(A) |=| A |=r(A)=\| A \|.
\]
because $A=A^*$.

\end{proof}

The Gelfand Neumark theorem was proved using lemma \ref{lem:omAenomA}. Now we have at hand a refining of this lemma. Hence we can use
\[ 
  \pi_r=\bigoplus_{\omega\in\etP(\cA)}\pi_{\omega}
\]
instead of the universal representation. The justification of this claim is that when $A$ is such that $\omega(A^*A)=0$ for all pure states, $\| A^*A \|=0$. Indeed $A^*A$ is positive, so there exists a pure state $\omega_a$ such that $\omega_a(A^*A)=\| A^*A \|$. Then $\| A^*A \|=0$.
 
Now we say that two states are \defe{equivalent}{equivalence!of states}\index{state!equivalent} if their GNS representations are equivalent. Gelfand Neumark theorem  says that
\[ 
  \cA\simeq\pi_r(\cA):=\bigoplus_{\omega\in[\etP(\cA)]}\pi_{\omega}(\cA).
\]
where $[\etP(\cA)]$ stands for the set of equivalences classes in $\etP(\cA)$.

\begin{proposition}
Any finite dimensional $C^*$-algebra is isomorphic to a direct sum of matricial algebras.
\end{proposition}
\begin{proof}
    No proof.
\end{proof}

\section{Generalization of matrix \texorpdfstring{$C^*$}{C}-algebra}
%+++++++++++++++++++++++++++++++++++++++++++++++

If we want to generalize the $C^*$-algebra $\mfM^n(\eC)$ to infinite dimensional Hilbert spaces, we first try to use the $C^*$-algebra of bounded operators. It does not work because such a $C^*$-algebra possesses many non equivalent representations on non separable Hilbert spaces.

\subsection{Example}
%------------------

Let us consider the $C^*$-algebra $\oB(\hH)$ and its definition representation which is obviously irreducible. From GNS construction and corollary \ref{cor_GNSirredst}, we know that all the GNS representations build from a vector state are equivalent to the definition one.

On the other hand, there exists some self-adjoint bounded operators with non empty continuous spectrum. The take $A\in\oB(\hH)$ and $a\in\sigma(A)$ such that there are no $\psi_a\in\hH$ such that $A\psi_a=a\psi_a$. Theorem \ref{tho_existsetat} gives a pure state $\omega_a$ such that $\omega_a(A)=a$, and hence a GNS representation $\pi_a$ on $\hH$. This representation is irreducible because $\omega_a$ is pure. In $\pi_a$, we have  a cyclic vector $\Omega_a$ such that
\[ 
  \scal{\Omega_a}{\pi_a(A)\Omega_a}=\omega_a(A)=a.
\]

%+++++++++++++++++++++++++++++++++++++++++++++++++++++++++++++++++++++++++++++++++++++++++++++++++++++++++++++++++++++++++++
\section{Tensor product}            \label{SecTensProdCSA}
%+++++++++++++++++++++++++++++++++++++++++++++++++++++++++++++++++++++++++++++++++++++++++++++++++++++++++++++++++++++++++++

If $\cA$ and $\cB$ are $C^*$-algebra, the \defe{tensor product}{tensor product!of $C^*$-algebra} is the completion of the space generated by the finite sums of the form $\sum_{i=1}^n A_i\otimes B_i$ with $A_i\in\cA$ and $B_i\in\cB$.

As an example of the importance of the completion, consider a compact group $G$ and $\cA=C(G)$ the $C^*$-algebra of continuous functions on $G$. We can build the map $\Delta\colon \cA\to \cA\otimes \cA$ by
\begin{equation}
    \Delta(f)(x,y)=f(xy)
\end{equation}
for every $x$ and $y$ in $G$ and $f\in C(G)$. The well-definiteness of $\Delta$ is due to the fact that $C(G)\otimes C(G)\simeq C(G\times G)$ by completion. This trick is used whenever we define a coproduct on a space of functions on a group, see for example definition \ref{DefHopfsurCG} and section \ref{SecExtenLemK} around equation \eqref{EqCABsimeqCACB}.

If $A$ and $B$ are manifolds, we have
\begin{equation}
    C^{\infty}(A)\otimes C^{\infty}(B)\simeq C^{\infty}(A\times B)
\end{equation}
by the map
\begin{equation}        \label{EqIsoCABCACBCstar}
    \begin{aligned}
        \varphi\colon  C^{\infty}(A)\otimes C^{\infty}(B)&\to  C^{\infty}(A\times B) \\
        \sum_ia_i\otimes b_i&\mapsto \Big[ (x,y)\mapsto\sum_ia_i(x)b_i(y) \Big]. 
    \end{aligned}
\end{equation}
The image by $\varphi$ of the \emph{algebraic} tensor product $ C^{\infty}(A)\otimes C^{\infty}(B)$ is dense in $ C^{\infty}(A\times B)$ as for example the polynomials are contained in the image. Indeed let $f(x,y)=\sum_{ij}f_{ij}x^iy^j$. The function $f$ is the image by $\varphi$ of
\begin{equation}        \label{EqDecompffklCABCACB}
    \sum_{kl} f_{kl} a_k\otimes b_l
\end{equation}
where $a_k(x)=x^k$ and $b_k(y)=y^k$. Thus, if we consider the $C^*$-algebraic tensor product, we have the equality.

See also \cite{Delaroche} for the sequel about tensor products. Let $\cA_1$ and $\cA_2$ be $C^*$-algebra and $\cA_1\odot\cA_2$ be their algebraic tensor product. There are at least two ways to define a $C^*$-norm on the $*$-algebra $\cA_1\odot\cA_2$.

\begin{enumerate}
    \item
        The \defe{maximal norm}{norm!maximal}\index{maximal!norm} of $A\in\cA_1\odot$ is defined by
        \begin{equation}
            \| A \|_{max}=\sup_{\pi}\| \pi(A) \|
        \end{equation}
        where the supremum is taken over all the representations\footnote{i.e. all the homomorphisms $\pi\colon \cA_1\odot\cA_2\to \opB(\hH)$.} of $\cA_1\odot\cA_2$ over some Hilbert space $\hH$. The \defe{maximal tensor product}{maximal!tensor product} is the completion of $\cA_1\odot\cA_2$ for that norm.

    \item
        The \defe{minimal norm}{minimal!norm} is obtained by taking the supremum only over the representations of the for $\pi_1\otimes \pi_2$:
        \begin{equation}
            \| x \|_{min}=\sup_{\pi_1,\pi_2}\| (\pi_1\otimes\pi_2)(x) \|
        \end{equation}
        where $\pi_i$ is a representation of $\cA_i$ on an Hilbert space $\hH_i$.
\end{enumerate}

\begin{lemma}
    If $\cA_1$ and $\cA_2$ are sub-$C^*$-algebra of $\opB(\hH_1)$ and $\opB(\hH_2)$, then $\cA_1\otimes_{min}\cA_2$ is the closure of $\cA_1\odot\cA_2$ seen as subalgebra of $\opB(\hH_1\otimes\hH_2)$.
\end{lemma}

%---------------------------------------------------------------------------------------------------------------------------
\subsection{Examples: Hopf algebra of functions}
%---------------------------------------------------------------------------------------------------------------------------
%\label{SubSecHoptUnivecvgp}

\begin{definition}		\label{DefHopfsurCG}
    Let $C(G)$ be the set of continuous functions on a topological group $G$. If we denote by $1$ the function $1\colon G\to \eC$, $1(x)=1$ for all $x\in G$, the following produces a structure of Hopf algebra\footnote{Definition \ref{DefHopfAlgebra}.} on $C(G)$:
\begin{enumerate}
		\item
			$(f\cdot g)(x)=f(x)g(x)$,
		\item
			$\eta(\lambda)=\lambda 1$, the unit,
		\item\label{ItemHopfCGiii}
			$(\Delta f)(x\otimes y)=f(xy)$,
		\item\label{ItemHopfCGiv}
			$\epsilon(f)=f(e)$,
		\item
			$S(f)(x)=f(x^{-1})$.
	\end{enumerate}
\end{definition}

The item \ref{ItemHopfCGiii} deserves some comments. If $f\in C(G)$, the element $\Delta(f)\in C(G)\otimes C(G)$ is given by $\Delta(f)=f_{(1)}\otimes f_{(2)}$ with the requirement that
\begin{equation}
	f_{(1)}(x)f_{(2)}(y)=f(xy).
\end{equation}
Such choice is possible since by density of the functions of the form $f(x)g(y)$ in $C(G\times G)$, see section \ref{SecTensProdCSA}.

Let us check that $(\id\otimes\epsilon)\Delta=\id$. First $(\id\otimes\epsilon)\big( f_{(1)}\otimes f_{(2)} \big)=f_{(1)}\otimes f_{(2)}(e)$. Now we use the identification between $C(G)\otimes \eC$ and $C(G)$ ($(f\otimes z)(g)=zf(g)$) in order to get
\begin{equation}
	\Big( (\id\otimes\epsilon)\big( f_{(1)}\otimes f_{(2)} \big) \Big)(g)=\big( f_{(1)}\otimes f_{(2)}(e) \big)(g)=f_{(1)}(g)f_{(2)}(e)=f(ge)=f(g).
\end{equation}

\begin{probleme}
	Unicity of $f_{(1)}$ and $f_{(2)}$ seems doubtful to me.
\end{probleme}


%+++++++++++++++++++++++++++++++++++++++++++++++++++++++++++++++++++++++++++++++++++++++++++++++++++++++++++++++++++++++++++
\section{Traces over $C^*$-algebra }
%+++++++++++++++++++++++++++++++++++++++++++++++++++++++++++++++++++++++++++++++++++++++++++++++++++++++++++++++++++++++++++
\label{SecTraceCstar}

Source: \cite{DixmierTrace}. For tracial functionals on von Neumann algebras, see section \ref{SecTracevonNeuman}.

\begin{definition}
    Let $\cA$ be a $C^*$-algebra. A \defe{trace}{trace!over $C^*$-algebra} on $\cA^+$ is a function $\tau\colon A^+\to \mathopen[ 0 , \infty \mathclose]$ such that
    \begin{enumerate}
        \item
            if $A$ and $B$ are in $\cA^+$, $\tau(A+B)=\tau(A)+\tau(B)$,
        \item
            If $A\in\cA^+$ and $\lambda\in\eR^+$, then $\tau(\lambda A)=\lambda\tau(A)$. Here if $\lambda=0$, we pose $0\times\infty=0$;
        \item
            If $Z\in\cA$, we have $\tau(ZZ^*)=\tau(Z^*Z)$.
    \end{enumerate}
    We say that $\tau$ is semifinite if for every $A\in\cA^+$,
    \begin{equation}
        \tau(A)=\sup\{ \tau(B)\tq B\in\cA^+,B\leq A,\tau(B)<\infty \}.
    \end{equation}
\end{definition}
We recall that for every $Z\in\cA$, we have $ZZ^*\in\cA^+$.

The following lemma is the lemma $3$ in \cite{DixmierTrace}.
\begin{lemma}       \label{LemTraceAplusextmlmn}
    Let $\cA$ be a $C^*$-algebra and $\tau$ a trace on $\cA^+$. Then the following hold.
    \begin{enumerate}
        \item
            The set
            \begin{equation}
                \mL=\{ A\in\cA\tq\tau(AA^*)<\infty \}
            \end{equation}
            is a bilateral ideal in $\cA$.
        \item
            The set $\mN=\langle \mL^2\rangle$ is the set of complex linear combinations of $\mN^+$: $\mN=\langle \mN^+\rangle.$
        \item
            We have
            \begin{equation}
                \mN^+=\{ A\in\cA^+\tq\tau(A)<\infty \}.
            \end{equation}
        \item
            There exists one and only one linear form $f$ on $\mN$ which coincides with $\tau$ on $\mN^+$.
        \item
            The linear form $f$ satisfies
            \begin{enumerate}
                \item
                    $f(A^*)=\overline{ f(A) }$;
                \item
                    $f(AB)=f(BA)$ for every $u$ and $v$ in $\mL$;
                \item
                    $f(ZA)=f(AZ)$ for every $A\in\mN$ and $Z\in \cA$.
            \end{enumerate}
            
    \end{enumerate}
    
\end{lemma}

\begin{proof}
    Let us begin by pointing out the fact that if $A\leq B$, then $\tau(A)\leq\tau(B)$ because of linearity: $\tau(B)=\tau(A)+\tau(B-A)\geq\tau(A)$ since $B-A\in\cA^+$.

    \begin{enumerate}
        \item
            If $A\in\mL$, then $A^*\in\mL$ because $\tau(A^*A)=\tau(AA^*)$ (by definition of a trace). If $A,B\in\mL$, we have
            \begin{equation}
                (A+B)(A+B)^*\leq 2(AA^*+BB^*),
            \end{equation}
            so that
            \begin{equation}
                \tau\big( (A+B)(A+B)^* \big)<2\tau(AA^*)+2\tau(BB^*)<\infty.
            \end{equation}
            This proves that $A+B\in\mL$.

            Let now $A\in\mL$ and $Z\in\cA$. Since $AZZ^*X^*\leq\| ZZ^* \|AA^*$, we have 
            \begin{equation}
                \tau\big( AZ(AZ)^* \big)<\infty.
            \end{equation}
            So $\mL$ is a right ideal in $\cA$. This is also a left ideal because $ZA=(A^*Z^*)^*$, but the fact that $A^*\in\mL$ implies $A^*Z^*\in\mL$, so that $ZA\in\mL$.
        \item
            An element $X$ in $\mN$ reads
            \begin{equation}
                X=\sum_{j=1}^{n}A_jB_j^*
            \end{equation}
            with $A_j,B_j\in\mL$. The \defe{polarization}{polarization} relation reads
            \begin{equation}        \label{EqPolaXAB}
                \begin{aligned}[]
                    4X&=\sum_j(A_j+B_j)(A_j+B_j)^*\\
                    &\quad+\sum_j(A_j-B_j)(A_j-B_j)^*\\
                    &\quad+\sum_j(A_j+iB_j)(A_j+iB_j)^*\\
                    &\quad+\sum_j(A_j-iB_j)(A_j-iB_j)^*.
                \end{aligned}
            \end{equation}
            So $X$ is a linear combination of elements in $\mN^+$ (that is elements of the form $AA^*$ with $A\in\mL$). Notice that $A\in\mL$ implies $iA\in\mL$ because
            \begin{equation}
                \tau\big( iA(iA)^* \big)=-\tau(iAiA^*)=\tau(AA^*)<\infty.
            \end{equation}
            This proves that $\mN\subset\langle \mN^+\rangle$. The fact that $\langle \mN^+\rangle$ is a subset of $\mN$ is by construction.
        \item
            An element $X$ in $\langle \mN^+\rangle$  reads $X=\sum_jA_jB^*_j$ where, for each $j$, we have $A_jB_j^*\in\mN^+$, in particular $A_jB^*_j=(A_jB^*_j)^*=B_jA_j^*$. We can still write down the polarization identity, but now the last two terms of \eqref{EqPolaXAB} give $2i(B_jA_j^*-A_jB_j^*)=0$. 

            Thus we have
            \begin{equation}
                \begin{aligned}[]
                    4X&=\sum_j(A_j+B_j)(A_j+B_j)^*-\sum_j(A_j-B_j)(A_j-B_j)^*\\
                    &\leq\sum_j(A_j+B_j)(A_j+B_j)^*,
                \end{aligned}
            \end{equation}
            but we already know that $\tau\big( \sum_j(A_j+B_j)(A_j+B_j)^* \big)<\infty$. Thus we have $\tau(X)<\infty$. So we proved that
            \begin{equation}
                \mN^+\subset\{ X\in\cA^+\tq\tau(X)<\infty \}.
            \end{equation}
            
            Let now $X\in\cA^+$ be such that $\tau(X)<\infty$. In order to prove that $X\in\mN^+$, it is sufficient to prove that $X\in\mN$. Since $X=X^*$, we can use the continuous functional calculus (see theorem \ref{ThoContFuncCalculus} and remark \ref{RemExpansionSqrtConCal}) in order to define $X^{1/2}$. We have
            \begin{equation}
                \tau\big( X^{1/2}(X^{1/2})^* \big)=\tau\big( X^{1/2}X^{1/2} \big)<\infty,
            \end{equation}
            so that $X^{1/2}\in\mL$.
        \item
            Since $\mN$ is generated by $\mN^{+}$, the functional $\tau$ on $\mN^+$ there is one and only one extension of $\tau$ to a linear functional $f$ on $\mN$.
        \item
            \begin{enumerate}
                \item
                    An element of $\mN$ is a linear combination of elements of $\mN^+$: $X=\sum_j\lambda_iA_i$ with $A_i\in\mN^+$. Thus using the linearity of $f$ and the properties of $\tau$, we have
                    \begin{equation}
                        \begin{aligned}[]
                            f(X^*)&=f\big( \sum_j\overline{ \lambda_j }A_j^* \big)\\
                            &=\sum_j\overline{ \lambda_j }f(A_j)\\
                            &=\sum_j\overline{ \lambda_j }\underbrace{\tau(A_j)}_{\in\eR^{+}}\\
                            &=\sum_j\overline{ \lambda_j \tau(A_j)}\\
                            &=\overline{ f(X) }.
                        \end{aligned}
                    \end{equation}
                    This is the first property we had to check.
                \item
                    If $A\in\mL$, we have $AA^*\in\mN$ and by definition of a trace,
                    \begin{equation}
                        f(AA^*)=\tau(AA^*)=\tau(A^*A)=f(A^*A).
                    \end{equation}
                    If $A$ and $B$ belong to $\mL$, we use the polarization identity:
                    \begin{equation}
                        4AB^*=(A+B)(A+B)^*-(A-B)(A-B)^*+i(A+iB)(A+iB)^*-i(A-iB)(A-iB)^*,
                    \end{equation}
                    and we do the same computation in order to get $f(AB^*)=f(B^*A)$ whenever $A$ and $B$ belong to $\mL$.
                \item
                    Let $Z\in\cA$. An element in $\mN$ reads $X=\sum_jA_jB_j$ with $A_j$ and $B_j$ in $\mL$. Since $\mL$ is an bilateral ideal in $\cA$ we have $ZA_j\in\mL$ and $B_jZ\in\mL$, thus we can make the following computation:
                    \begin{equation}
                        \begin{aligned}[]
                            f\big( Z\sum_jA_jB_j \big)&=\sum_jf\big( (ZA_j)B_j \big)\\
                            &=\sum_jf\big( B_j(ZA_j) \big)\\
                            &=\sum_jf\big( (B_jZ)A_j \big)\\
                            &=\sum_jf\big( A_j(B_jZ) \big)\\
                            &=f\big( (\sum_jA_jB_j)Z \big)\\
                            &=f(XZ).
                        \end{aligned}
                    \end{equation}              
            \end{enumerate}
            This concludes the proof of the lemma.
    \end{enumerate}
    
    
\end{proof}
