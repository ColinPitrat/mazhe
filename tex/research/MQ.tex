% This is part of Mes notes de mathématique
% Copyright (c) 2014
%   Laurent Claessens
% See the file fdl-1.3.txt for copying conditions.

\section{Mathematical framework of field theory}
%++++++++++++++++++++++++++++++++++++++++++++++

    This is a short review; the aim is to see why the quantum theory of fields needs representations of the Poincaré group. It will be mostly physics oriented. References dealing with field theory including gauge theory and representations are \cite{Sternberg,Preparation,Naber,QMVirtanen,MQSenechal,Weinberg,WormerAngular}.

\subsection{Axioms of the (quantum) relativistic field theory}
%-------------------------------------------------------------

The quantum mechanics is based on a few number of axioms:
\begin{enumerate}\label{pg:axiomes}
\item We have a Hilbert space $\pH$. A physical state is given by a \defe{ray}{ray} in $\pH$, i.e. a set 
       \[
           \rR=\{\xi\psi:|\xi|=1\}
       \]
for a certain $\psi\in\pH$ with $\scalh{\psi}{\psi}=1$. In other words, the set of physical sates is the quotient of the set of unital vectors in $\pH$ by the relation $\psi\sim\psi'$ if and only if $\psi=\xi\psi'$ for some unimodular complex number $\xi$. We denote by $\Ray\pH$\nomenclature{$\Ray\pH$}{Rays in a Hilbert space} the set of all rays in $\pH$.
       
\item\label{ax:vaps} The observables are represented by hermitian linear operators on $\pH$. A state $\rR$ has value $\alpha$ for the observable $A$ if $A\rR=\alpha\rR$, where the action of $A$ on the ray is obvious (and well defined because $A$ is linear).

\item If one has a system described by a state $\rR$, and if one want to measure if it is in one of the state $\rR_1,\ldots,\rR_n$ (orthogonal rays), the answer will be $\rR_i$ with probability 
        \[
	    P(\rR\to\rR_i)=|\scalh{\rR}{\rR_i}|^2.
        \]
If the $\rR_n$ form a complete system, one has a theorem which states that 
\[
   \sum_i P(\rR\to\rR_i)=1.
\]

\item\label{ax:reprez}  The rays of $\pH$ furnish a representation of the (identity component of) Poincaré group.
\end{enumerate}

This last point can look strange; we will see later (page \pageref{pg:poincare_act}) how it comes. It is the expression of a relativistic theory. That axiom is the reason why one make intensive use of representation theory in relativistic (quantum) field theory \ldots or maybe the intensive use of representation theory is the reason of that axiom. However, we will make an intensive use of representation theory developed in chapter \ref{ChapThoComsGroupes}.

\subsection{Symmetries and Wigner's theorem}
%-------------------------------------------

Consider the following situation: someone observes a system in a state $\rR$, and makes measures $P(\rR\to\rR_i)$. An other person observes the same system which is, for him, in a state $\rR'$ and observes $P(\rR'\to\rR'_i)$.

If two observers are related by a transformation of the Hilbert state which induces $\rR\to\rR'$ and $\rR_i\to\rR_i'$, there are said \defe{equivalent}{equivalence!of observer} if 
\begin{equation}\label{eq:sym_isom}
   P(\rR\to\rR_i)=P(\rR'\to\rR'_i).
\end{equation}

Let us say it more precisely from a mathematical point of view. A \defe{symmetry}{symmetry!of quantum system} is an invertible operator $\dpt{T}{\Ray\pH}{\Ray\pH}$ such that for any $\phi_i\in\rR_i$, $\phi_i'\in T\rR_i$ and $\phi''_i\in T^{-1}\rR_i$, 
\begin{equation}\label{eq:}
  |\scalh{\phi'_1}{\phi'_2}|^2=|\scalh{\phi_1}{\phi_2}|^2=|\scalh{\phi''_1}{\phi''_2}|^2
\end{equation}


\begin{remark}
Here, neither $\rR$ nor $\rR'$ are measurable: the $P$'s only are measurable.
\end{remark}

The following can be found in  \cite{Weinberg} p.91, \cite{Sternberg} p.354.
\begin{theorem}[Wigner]\label{tho:Wigner}
Any symmetry $T$ is induced by an operator $U$ on $\pH$ such that $\psi\in\rR$ implies $U\psi\in\rR'$. This operator is either unitary and linear, either anti-unitary and antilinear.
\end{theorem}

So, the symmetry operator must satisfy
\begin{subequations}
\begin{align}  
  \scalh{U\psi}{U\phi}&=\scalh{\psi}{\phi}\\
  U(\xi\psi+\eta\psi)&=\xi U\psi+\eta U\phi,
\end{align}  
\end{subequations}
or
\begin{subequations}
\begin{align}
  \scalh{U\psi}{U\phi}&=\scalh{\psi}{\phi}^*\\
  U(\xi\psi+\eta\psi)&=\xi^* U\psi+\eta^* U\phi.
\end{align}  
\end{subequations}
In the anti-linear case operator, we do not define $U^{\dag}$ by $\scalh{\phi}{U^{\dag}\psi}=\scalh{U\phi}{\psi}$ because the left-hand side should be anti-linear with respect to $\psi$ while the right-had should be linear. In place, for an antilinear operator $A$, we define $A^{\dag}$ by
\begin{equation} 
   \scalh{\phi}{A^{\dag}\psi}=\scalh{A\phi}{\psi}^*=\scalh{\psi}{A\phi}.
\end{equation}
In this way, the definitions of unitary and anti-unitary in term of dagger are the same: $U^{\dag}=U^{-1}$.

For any transformation $\dpt{T}{\Ray \pH}{\Ray \pH}$, the Wigner's theorem provides an operator $\dpt{U(T)}{\pH}{\pH}$ which induces $T$ on $\Ray$. If the operator $T$ depends on a parameter $\theta$, the operator $U(T(\theta))$ depends on $\theta$. If $T$ depends continuously on the parameter then the family $U(T(\theta))$ only contains unitary/linear operators or only antiunitary/antilinear operators.

In physical cases, $T(\theta)$ is mostly a Poincaré transformation: $\theta=(\Lambda,p)$. But $T(\mtu,0)$ is the identity which is represented by $U(\mtu,0)=\mtu$. Then all the (connected to identity) Poincaré transformations are represented by linear and unitary operators on $\pH$.

We will follow the proof given in \cite{Weinberg}. An other form of the proof can be found in \cite{Sternberg}. The latter use a slightly different formalism in the axioms of the quantum mechanics; this is explained in appendix \ref{app:Wigner}. It is now time to prove the theorem. 

\begin{proof}[Proof of Wigner's theorem]
We consider an orthonormal basis $\{\psi_k\}$ of $\pH$ with $\psi_k\in\rR_k$, and a choice of $\psi'_k\in T\rR_k$. From this and the assumptions, we have
\[
|\scalh{\psi'_k}{\psi'_l}|^2=|\scalh{\psi_k}{\psi_l}|^2=\delta_{kl}.
\]
Then $\scalh{\psi'_k}{\psi'_k}=0$ whenever $k\neq l$ and, since $\scalh{\psi'_k}{\psi'_k}$ is real and positive, $\scalh{\psi'_k}{\psi'_k}=1$. So $\scalh{\psi'_k}{\psi'_l}=\delta_{kl}$. 

The set $\psi_k'$ is also complete in $\pH$. Indeed suppose that we have a vector $\psi'\in\pH$ such that $\scalh{\psi'}{\psi'_k}=0$ for all $k$. If $\psi'\in\rR$, we consider a $\psi''\in T^{-1}\rR$ and we have
\[
   |\scalh{\psi''}{\psi_k}|^2=|\scalh{\psi'}{\psi'_k}|^2=0,
\]
which contradicts the fact that the $\psi_k$'s form a complete set. Now we have to fix a phase convention for the $\psi_k$. Since there are no canonical choice of phase, we fix with respect to an arbitrary one of the $\psi_k$, say $\psi_1$. We put
\begin{equation}
  \gamma_k=\us{\sqrt{2}}(\psi_1+\psi_k)\in\rC_k
\end{equation}
for $k\neq 1$. Any $\gamma'_k\in T\rC_k$ can be written in the basis $\{\psi'_k\}$:
\begin{equation}\label{eq:gamma_psi_k}
  \gamma'_k=\sum_l c_{kl}\psi'_l.
\end{equation}
From assumption \eqref{eq:sym_isom} and the fact that $|c_{kl}|^2=|\scalh{\gamma'_k}{\psi'_l}|^2$, we find, for $k,l\neq 1$
\[
  |c_{kl}|^2=\frac{1}{2}\delta_{kl}.
\]
We can choose the phase of $\gamma'_k$ and $\psi'_k$ in order to get $c_{kk}=c_{k1}=1/\sqrt{2}$. For this, we begin to fix $\gamma'_k$ in such a manner to get $c_{k1}=1/\sqrt{2}$ (from $|c_{k1}|=|\scalh{\gamma'_k}{\psi'_1}|$), and next we fix $\psi_k'$ for the $c_{kk}$. From now on, the so chosen $\gamma'_k$ and $\psi'_k$ are denoted by $U\gamma_k$ and $U\psi_k$. 

What we  did until now is to take a basis $\{\psi_k\}$ of $\pH$ and define $\gamma_k=1/\sqrt2(\psi_1+\psi_k)$. Next we had chosen the phases of $\psi'_k\in T\rR_k$ and $\gamma'_k\in T\rC_k$ in order to have
\begin{equation}\label{eq:c_kl}
\begin{aligned}
   c_{kk}=c_{k1}&=1/\sqrt 2&&\forall k,\\
   c_{kl}&=0 &&\text{if }l\neq k,l\neq 1.
\end{aligned}   
\end{equation}
This allows us to check a certain linearity for the operator $U$:
\begin{equation}
\begin{aligned}
U\left( \us{\sqrt 2}(\psi_k+\psi1)  \right)
   &=U\gamma_k\\
   &=\gamma'_k\\
   &=\us{\sqrt 2}\psi'_1+\us{\sqrt 2}\psi'_k&&\text{from \eqref{eq:gamma_psi_k} and \eqref{eq:c_kl}}\\
   &=\us{\sqrt{2}}\big(  U\psi_1+U\psi_k   \big).
\end{aligned}
\end{equation}
Now we have to build $U$ on a general vector $\psi=\sum_k\psi_k\in\mR$. Any vector $\psi'\in T\mR$ can be decomposed with respect to the basis $\{\psi'_k=U\psi_k\}$: 
\begin{equation}\label{eq:dev_Upsi}
  \psi'=\sum_k C'_kU\psi_k.
\end{equation}
From the conservation of probability $|\scalh{\psi_k}{\psi}|^2=|\scalh{U\psi_k}{\psi'}|^2$ and $|\scalh{\gamma_k}{\psi}|^2=|\scalh{U\gamma_k}{\psi'}|^2$, we find
\begin{subequations}\label{eq:deux_C_k}
\begin{align}
          |C_k|^2&=|C'_k|^2, \label{eq:deux_C_k_a} \\
       |C_k+C_1|^2&=|C_k'+C'_1|^2.
\end{align}
\end{subequations}
If one writes $C_k=a_k+ib_k$, one finds $\real(C_k/C_1)=(a_ka_1+b_kb_1)/|C_1|^2$. By doing the same with $C'_k$ and using \eqref{eq:deux_C_k},
\begin{equation}\label{eq:C_k_C_1}
  \real(C_k/C_1)=\real(C'_k/C'_1).
\end{equation}
Equation \eqref{eq:deux_C_k_a} also imposes
\begin{equation}\label{eq:frac_C_k}
 |C_k/C_1|^2=|C'_k/C'_1|^2,
\end{equation}
while compatibility between \eqref{eq:frac_C_k} and \eqref{eq:C_k_C_1} requires
\begin{equation}\label{eq:C_k_C_1_im}
  \imag(C_k/C_1)=\pm\imag(C'_k/C'_1).
\end{equation}
Equations \eqref{eq:C_k_C_1} and \eqref{eq:C_k_C_1_im} show that $C_k$ and $C'_k$ must satisfy 
\begin{subequations}
\begin{align}
  C_k/C_1&=C'_k/C'_1\\
\intertext{xor}  
  C_k/C_1&=(C'_k/C'_1)^*.
\end{align}
\end{subequations}
For a given $\psi$ we have to show that the choice must be the same for all the $C_k$\footnote{We will show later that for a given $T$, the choice must be the same for all the $\psi$.}. Let $l\neq k$ and suppose that $C_k/C_1=C'_k/C'_1$ and $C_l/C_1=(C'_l/C'_1)^*$; we will show that in this case, one of the two ratios is real. So we can suppose $k\neq 1\neq l$. We consider the vector $\Phi=\us{\sqrt 3}(\psi_1+\psi_k+\psi_l)$, 
\[
  \Phi'=\frac{\alpha}{\sqrt 3}\big( U\psi_1+U\psi_k+U\psi_l  \big)
\]
where $\alpha\in\eC$ satisfies $|\alpha|=1$. The conservation of probability $|\scalh{\Phi}{\psi}|^2=|\scalh{\Phi'}{\psi'}|^2$ gives $|C_1+C_k+C_l|^2=|C'_1+C'_k+C'_l|^2$. Since $|C_1|^2=|C'_1|^2$, we can divide the left hand side by $|C_1|^2$ and the right one by $|C'_1|^2$. We find
\[
\left|1+\frac{C_k}{C_1}+\frac{C_l}{C_1}\right|^2=\left|1+\frac{C'_k}{C'_1}+\frac{C'_l}{C'_1}\right|^2.
\]
Using the assumption $C_k/C_1=C'_k/C'_1$ and $C_l/C_1=(C'_l/C'_1)^*$, we are in a case of an equation of the form $|u+v|^2=|u+v^*|^2$ with $u$, $v\in\eC$. If we write $u=a+bi$ and $v=x+iy$, we find $b+y=\pm(b-y)$, so that it leaves the choice $y=0$ or $b=0$ which corresponds to $(C_k/C_1)\in\eR$ or $(C_l/C_1)\in\eR$. So the coefficients $C'_k$ ($k\neq 1$) in the expansion \eqref{eq:dev_Upsi} must satisfy
\begin{subequations}
\begin{align}
  C_k/C_1&=C'_k/C'_1\quad\forall k \label{eq:rap_C_a} \\
\intertext{xor}  
  C_k/C_1&=(C'_k/C'_1)^*\quad\forall k  \label{eq:rap_C_b}.
\end{align}
\end{subequations}
Note that the phase of $C_1$ is not yet fixed. We naturally choice $C_1=C'_1$ or $C_1={C'_1}^*$ following the case. We define $\dpt{U}{\pH}{\pH}$ by
\begin{subequations}\label{eq:def_U}
\begin{align}
   U\left( \sum_k C_k\psi_k \right) &= \sum_kC_kU\psi_k&&\text{if \eqref{eq:rap_C_a}},
\label{eq:def_U_a}   
\intertext{xor}  
  U\left( \sum_k C_k\psi_k \right) &= \sum_kC_k^*U\psi_k&&\text{if \eqref{eq:rap_C_b}}.
\label{eq:def_U_b}  
\end{align}
\end{subequations}
One can explicitly check that it preserves the probability because $|\scalh{\psi}{\psi_k}|^2=|C_k|^2$ while $|\scalh{U\psi}{U\psi_k}|$ is equal to $|C_k|^2$ or $|C^*_k|^2$ (which are the same) following the case \eqref{eq:def_U_a} or \eqref{eq:def_U_b}.

Now we have to prove that the choice \eqref{eq:def_U_a} or \eqref{eq:def_U_b} is fixed by the data of $T$ and must be the same for all the $\psi\in\pH$. For, let us consider two vectors $\phi=\sum A_k\psi_k$, $\varphi=\sum B_k\psi_k$ and suppose that
\[
   U\phi=\sum_k A_k U\psi_k\text{ but } U\varphi=\sum_kB^*_k U\psi_k.
\]   
In order to see that it is impossible, looks at the conservation of probability $|\sum_k A_kB^*_k|^2=|\sum_k A_kB_k|^2$, then
\begin{equation}
\sum_{kl}\big(   B^*_lB_kA_lA^*_k-B^*_lB_kA^*_lA_k    \big)=\sum_{kl}B^*_lB_k\imag(A_lA^*_k)=0.
\end{equation}
Since $A_lA^*l\in\eR$, we can regroup each term $(k,l)$ with the corresponding term $(l,k)$. We get
\begin{equation}\label{eq:imim_zero}
\begin{split}
  0=\sum_{kl}\imag(A_lA^*_k)(B^*_lB_k-B^*_kB_l)
   =\sum_{kl}\imag(A^*_kA_l)\imag(B^*_kB_l).
\end{split}   
\end{equation}
We can find a vector $\sum_kC_k\psi_k$ such that 
\begin{subequations}\label{eq:choix_C}
\begin{align}
 \sum_{kl}\imag(C^*_kC_l)\imag(A^*_kA_l)&\neq 0  \label{eq:choix_C_a}\\
\intertext{and}
 \sum_{kl}\imag(C^*_kC_l)\imag(B^*_kB_l)&\neq 0.
\end{align}
\end{subequations}
In order to see how to find such a vector, let us show that there always exists a choice $(i,j)$ such that $B^*_iB_j$ is not real. Let us say $B_1=x+iy$ and $B_k=a_k+bi$. If $y\neq 0$, the condition $\imag(B_1^*B_k)=0$ gives $B_k=\frac{b_k}{y}B_1$. It is always possible to find a sequence $(b_k)$ which gives $1$ as norm for $\sum B_k\psi_k$; the problem is not there. The problem is that $B_k/B_1\in\eR$, so that the choice \eqref{eq:def_U}  is not a true choice. For the same reason, all the $B^*_iB_k$ can't be pure imaginary.

Now we can find the vector which satisfy \eqref{eq:choix_C}. There are several cases. If there is a pair $(k,l)$ such that $A^*_kA_l$ and $B^*_kB_l$ are both complex, we can take all $C_i$'s zero for $k\neq i\neq l$ and choose $C_k$ and $C_l$ in such a way that $C^*_kC_l$ is not real. If there is a pair $(k,l)$ with $A^*_kA_l$ complex and $B^*_kB_l$ real, we consider a pair $(m,n)$ such that $B^*_mB_n$ is complex. If $A^*_mA_n$ is complex, we take all the $C_i$'s zero except $C_m$ and $C_n$ such that $\imag(C^*_mC_n)\neq 0$. If $A^*_mA_n$ is real, we take all the $C_i$'s zero except $C_k,C_l,C_m,C_n$ which we choose in such a way that $\imag(C^*_mC_n)\neq 0$ and $\imag(C^*_kC_k)\neq 0$.

Equation \eqref{eq:choix_C_a} makes that the same choice must be made for $\sum A_k\psi_k$ and $\sum C_k\psi_k$ (if it was not the case, we would have an equation of the form of \eqref{eq:imim_zero}). For the same reason, the same choice must be made for $\sum B_k\psi_k$ and $\sum C_k\psi_k$. So we conclude that the data of $T$ fixes the choice between \eqref{eq:def_U_a} and \eqref{eq:def_U_b} and that this choice must be the same for all the vectors of $\pH$.

We have to show that the possibility \eqref{eq:def_U_a} makes $U$ linear and unitary while the possibility \eqref{eq:def_U_b} makes $U$ antilinear and antiunitary. For we consider $\psi=\sum_k A_k\psi_k$ and $\phi=\sum_k B_k\psi_k$. If \eqref{eq:def_U_a} works,
\begin{equation}
\begin{split}
  U(\alpha\psi+\beta\phi)&=U\Big(  \sum_k(\alpha A_k+\beta B_k)\psi_k     \Big)\\
                         &=\sum_k(\alpha A_k+\beta B_k)U\psi_k\\
			 &=\alpha U\psi+\beta U\phi,
\end{split}
\end{equation}
and
\begin{equation}
\scalh{U\psi}{U\phi}=\sum_{kl}A^*_kB_l\scalh{U\psi_k}{U\psi_l}=\sum_kA^*_kB_k,
\end{equation}
so that $\scalh{U\psi}{U\phi}=\scalh{\psi}{\phi}$. Thus in this case $U$ is linear and unitary. In the case where \eqref{eq:def_U_a} works, the computations are almost the same:
\begin{equation}
\begin{split}
  U(\alpha\psi+\beta\phi)&=U\Big(  \sum_k(\alpha A_k+\beta B_k)\psi_k     \Big)\\
                         &=\sum_k(\alpha^* A_k^*+\beta^* B^*_k)U\psi_k\\
			 &=\alpha^* U\psi+\beta^* U\phi,
\end{split}
\end{equation}
and
\begin{equation}
\scalh{U\psi}{U\phi}=\sum_{kl}A_kB^*_l\scalh{U\psi_k}{U\psi_l}=\sum_kA_kB^*_k,
\end{equation}
so that $\scalh{U\psi}{U\phi}=\scalh{\psi}{\phi}^*$. In this case, $U$ is antilinear and antiunitary.

\end{proof}

\subsection{Projective representations}
%-------------------------------------------------

If $T_1(\rR_n)=\rR_n'$ and $\psi_n\in\rR_n$, then $U(T_1)\psi_n\in\rR_n'$. If $T_2(\rR')=\rR''$, then $U(T_2)U(T_1)\psi_n\in\rR_n''$. But $U(T_2T_1)\psi_n$ also belongs to $\rR_n''$. Then there exists a $\phi_n(T_2,T_1)\in\eR$ such that 
\[
   U(T_2)U(T_1)\psi_n=e^{i\phi_n(T_2,T_1)}U(T_2T_1)\psi_n.
\]
Note that for any $\psi\in \pH$, there exists a $\lambda\in\eR$ such that $\|\lambda\psi\|=1$. Since a real can be sent out the $U(T)$'s, for \emph{any} $\psi\in \pH$, there exists a $\phi$ which only depends on $\psi/\|\psi\|$ such that
\begin{equation}
    U(T_2)U(T_1)\psi=e^{i\phi(T_2,T_1)}U(T_2T_1)\psi
\end{equation}

\begin{proposition}
The $\phi$ doesn't depend at all on the $\psi$:
\begin{equation}
  U(T_2)U(T_1)=e^{i\phi(T_2,T_1)}U(T_2T_1).
\end{equation}
\end{proposition}

\begin{proof}
Let us consider a $\psi_A$ and a $\psi_B$ which are not proportional each other. One has a $\phi_{AB}(T_2,T_1)$ such that
\begin{equation}
\begin{split}
e^{i\phi_{AB}(T_2,T_1)}U(T_2T_1)(\psi_A+\psi_B)&=U(T_2)U(T_1)(\psi_A+\psi_B)\\
                                               &=e^{i\phi_A(T_2,T_1)}U(T_2T_1)\psi_A\\
                                               &\quad +e^{i\phi_B(T_2,T_1)}U(T_2T_1)\psi_B.
\end{split}
\end{equation}
Now, we apply $U(T_2T_1)^{-1}$ to both sides. If it is unitary, the $e^{i\phi}$ get out without problems; else is get out as $e^{-i\phi}$:
\begin{equation}
  e^{\pm i\phi_{AB}}(\psi_A+\psi_B)=e^{\pm i\phi_A}\psi_A+e^{\pm i\phi_B}\psi_B.
\end{equation}
Since $\psi_A$ and $\psi_B$ are linearly independent, the only solution is $e^{i\phi_{AB}}=e^{i\phi_A}=e^{i\phi_B}$.

\end{proof}

Since the operators $U(T)$ must only fulfil
\begin{equation}\label{eq:projectif}
   U(T_2)U(T_1)=e^{i\phi(T_2,T_1)}U(T_2,T_1),
\end{equation}
these form a \defe{projective representation}{projective!representation} of the symmetry group on the physical Hilbert space $\pH$.

\begin{remark}
In order to have some physical relevance, this demonstration supposes that a state $\psi_A+\psi_B$ exists in nature. If one can divide the particles in several ``incompatibles''\ classes labeled by $a,b$ such that $\psi_a+\psi_b$ doesn't exist, then equation \eqref{eq:projectif} is false and one has to write 
\[
   U(T_2)U(T_1)\psi_a=e^{i\phi_a(T_2,T_1)}U(T_2T_1)\psi_a
\]
because we can't show that $\phi_a=\phi_b$ from the simple fact that $\psi_a+\psi_b$ doesn't exist!

For example, physicists think that there are no superposition of state of integer and semi-integer spin.
\end{remark}

\begin{remark}
If the group satisfies some requirements, one can choose $\phi=0$. From now we suppose that we are in this case: we work with ``true''\ representations.
\end{remark}


\subsection{Representations and power expansions}
%------------------------------------------------

Let $G$ be an arc connected Lie group whose elements are denoted by $T(\theta)$ with $\theta$, a continuous family of parameters (from a local chart). The multiplication law is given by a function $\dpt{f}{\eR^n\times\eR^n}{\eR^n}$:
\begin{equation}\label{eq:T_groupe}
T(\theta')T(\theta)=T\big(f(\theta',\theta)\big).
\end{equation}
If $ \theta=0$ is the coordinate of the identity, 
\begin{equation}\label{eq:f_0}
f(0,\theta)=f(\theta,0)=\theta.
\end{equation}

We suppose that $G$ acts on the rays of a Hilbert space $\pH$, so that there are represented on $\pH$ by unitary operators $U\big(T(\theta)\big)$. We denote by $W$ the group of transformations of $\pH$; roughly speaking, 
\[
     W=U(G).
\]
Now, we are going to cheat a little. We know that there exists a normal neighbourhood\index{normal!neighbourhood} of $e$ in $W$. In simple words, the map $\dpt{\exp}{\lW}{W}$ is a diffeomorphism between the elements of $\lW$ ``close''\ to $0$ and the ones of $W$ close to $e$. By \textit{close to}, we mean that the components of $\theta$ are small enough. If $\{it_a\}$ is a basis of $\lW$, we define
\begin{equation}\label{eq:U_expo}
   U(T(\theta))=e^{i\theta^at_a}.
\end{equation}
In other words, one considers the exponential map for a neighbourhood of identity.

The cheat is the fact that $U(T(\theta))$ is actually defined by Wigner's theorem from the data of the group $G$. So equation \eqref{eq:U_expo} should be seen as a requirement in the choice of the basis $\{t_a\}$.

\begin{remark}
The $i$ in the exponential in \eqref{eq:U_expo} and in the definition of the basis $\{it_a\}$ is a convention in order the $t_a$'s to be hermitian. Indeed, the Lie algebra of a group of unitary matrices is made of \emph{anti}hermitian matrices.
\end{remark}

With all that,
\begin{equation}\label{eq:dev_U}
   U(T(\theta))=\mtu+i\theta^at_a+\frac{1}{2}\theta^b\theta^ct_{bc}+\ldots
\end{equation}
where $t_{bc}$ is defined (among other requirements) to absorb the ``intuitive''\ minus sign in the third term.

Now we are going to explore some consequences of equation \eqref{eq:T_groupe}. Equation \eqref{eq:f_0} makes the expansion of $f$ as
\begin{equation}\label{eq:dev_f}
f^a(\theta',\theta)=\theta^a+{\theta'}^a+f^a_{bc}{\theta'}^b\theta^c+\ldots
\end{equation}
From expansions \eqref{eq:dev_f} and \eqref{eq:dev_U} of $f$ and $U(T(\theta))$, ``group structure''\ equation \eqref{eq:T_groupe} gives (at order two):
\begin{equation}\label{eq:t_ab}
   t_{bc}=-t_bt_c-if^a_{bc}t_a
\end{equation}
and nothing for the first order. Then, providing that one knows the group structure (the $f$), one knows the second order of the representation from the first one.
From equation \eqref{eq:U_expo}, one finds the value of $t_{ab}$:
\[
   e^{i\theta^at_a}=1+i\theta^at_a+\frac{1}{2}(i)^2(\theta^at_a)(\theta^bt_b),
\]
up to constant coefficients, one can choose $t_{ab}$ to be symmetric with respect to $a$ and $b$:
\[
   t_{ab}=\frac{1}{2}(t_at_b+t_bt_a).
\]
Taking this convention and computing $t_{bc}-t_{cb}$ from \eqref{eq:t_ab}, we find
\begin{equation}
  [t_a,t_b]=iC_{ab}^ct_c
\end{equation}
with $C_{ab}^c=f_{ab}^c-f_{ba}^c$.

On the other hand, one knows that if a group is abelian, its algebra is also abelian; we can see it here by considering that if $G$ is abelian, $f(\theta,\theta')=f(\theta',\theta)$, then $f_{ab}^c$ is symmetric and $[t_a,t_b]=0$. We can say more about $f$ Since the $t_a$ commute, equations \eqref{eq:T_groupe} and \eqref{eq:U_expo} make that
\begin{equation}
  e^{if(\theta,\theta')^at_a}=e^{i\theta^at_a}e^{i{\theta'}^bt_b}\\
                             =e^{i(\theta^a+{\theta'}^a)t_a},
\end{equation}
so that 
\[
   f(\theta,\theta')=\theta+\theta'.
\]

\section{The symmetry group of nature}
%-------------------------------------

\subsection{Spin and double covering}\label{subsec:sym_nature}
%-----------------------------------

Some of literature carry an ambiguity in the choice of the right space-time symmetry group in the quantum field theory. A very good and deep discussion about the choice of the space-time symmetry group of nature is given in the book \cite{Naber} which will be used here. An other enlightening review can be found in \cite{ModavePoincarre}.

From a relativistic point of view, the group is the Poincaré group of all the maps $\eR^4\to\eR^4$ which leaves invariant the quantity $s^2=-t^2+x^2+y^2+z^2$.  At this point we can already make an important remark: the so defined quantity $s$ is in fact \emph{not} a relativistic invariant. Indeed if I follow a (spatially) closed path, I will measure $\Delta t\neq 0$ and $\Delta x=\Delta y=\Delta z=0$ because in \emph{my} frame, my displacement is zero. A guy who keeps at my starting point will measure (between the beginning and the end of my travel) $\Delta 't\neq 0$ and also $\Delta x=\Delta y=\Delta z=0$. If $s=s'$, then $\Delta t=\Delta t'$.

So the relativistic invariance is only local: $ds^2=ds'^2$, and as far as relativity is concerned, one can work with infinitesimal transformations only. In this case, the distinction between the \emph{groups} $L_+^{\uparrow}$ and $\SLdc$ is no relevant. Intuitively, we choose $L_+^{\uparrow}$ to be the space-time symmetry group. As we will see the difference will reveal to be crucial in relativistic field theory because $L_+^{\uparrow}$ has no half-integer spin representations.

This group naturally splits into two parts: the translations and the rotations (and boost). As far as I know, the translation part makes no difficulties. For the other one, there are some difficulties to find the \emph{minimal} group of symmetry. First, one often want to separate the space-time inversions $P$ and $T$ from the remaining: the group then becomes the homogeneous orthochrone Lorentz group $L_+^{\uparrow}$. An other often presented group is \ldots $\SLdc$. This is our choice here. The physical reason of this choice is all but immediate. As we will see during the following pages, an elementary particle is an irreducible representation of the symmetry group.

For massive particles, the relevant subgroup of $\SLdc$ reveals to be $SU(2)$. If we had chosen the most intuitive $L_+^{\uparrow}$, we would have found $\SO(3)$. There is an important difference between $SU(2)$ and $\SO(3)=SU(2)/\eZ_2$: the first one admits representations of any integer and half-integer spin while the second only posses the integer spin representations (cf. page \pageref{pg:reprez_SO3}).

Let us now be more precise about the relation between $L_+^{\uparrow}$ and $\SLdc$. A know result is
\[
   L_+^{\uparrow}=\frac{\SLdc}{\eZ_2}.
\]
Let $\dpt{\mSpin}{\SLdc}{L_+^{\uparrow}}$ be the surjective homomorphism with kernel $\pm\mtu_{2\times 2}$ giving this relation.
We will not give a complete proof, but we will explain how $\SLdc$ acts by isometries on $\eR^4$. First, we remark that there exists a bijection between $\eR^4$ and the $2\times 2$ complex hermitian matrices:
\begin{equation}
  v=\begin{pmatrix} t+z & x-iy\\x+iy&t-z\end{pmatrix}=\begin{pmatrix}t\\x\\y\\z\end{pmatrix}.
\end{equation}
If $\lambda\in\SLdc$, the matrix $\lambda v\lambda^{\dag}$ is also hermitian and $\|v\|^2=\det v$. Thus 
\begin{equation}
\begin{aligned}
 \Lambda(\lambda)  \colon \eR^4&\to \eR^4 \\ 
   v&\mapsto \lambda v\lambda^{\dag} 
\end{aligned}
\end{equation}
is a Lorentz transformation if and only if $|\det\lambda|=1$. Moreover,
\[
     \Lambda(\lambda\lambda')=\Lambda(\lambda)\Lambda(\lambda').
\]
If $\lambda'=e^{\phi}\lambda$, then $\Lambda(\lambda')=\Lambda(\lambda)$, thus it is natural to impose $\det v=1$ and to consider $\SLdc$ instead of $L(2,\eC)$ to fit $L_+^{\uparrow}$. Now, $\Lambda(\lambda)=\Lambda(-\lambda)$, and we wish to consider $\SLdc/\eZ_2$.

I think the problem is the following: as far as the action of the ``nature group''{} on the space-time is concerned, it is sufficient to consider $L_+^{\uparrow}$. But the group which acts on the state space is wider: it must be $SL(2,\eC)$.

From now, when we say ``Poincaré group'', we mean $\SLdc\times\eR^4$ while  ``Lorentz''\ means $\SLdc$ acting on $\eR^4$ by $\Lambda(\lambda)v=\lambda v\lambda^{\dag}$.

Let us continue the discussion of page \pageref{pg:reprez_SOt}. A know result is the fact that the map $\mSpin$ restricts to a surjective homomorphism $\dpt{\mSpin}{SU(2)}{\SO(3)}$ with kernel $\pm\mtu$ giving the relation $\SO(3)=SU(2)/\eZ_2$. If one considers a representation $\dpt{\rho}{\SO(3)}{GL(V)}$, then $\tilde\rho=\rho\circ\mSpin$ is a representation of $SU(2)$ on $V$. So every representation of $\SO(3)$ comes from a representation of $SU(2)$.

As far as the transformation rule of a (quantum mechanical) wave function under a rotation $R\in \SO(3)$ is concerned, one can see (it is done in \cite{Naber}) that the try
\[
   \begin{pmatrix}
     \psi_1\\
     \psi_2
   \end{pmatrix}\to
T(R)
\begin{pmatrix}
     \psi_1\\
     \psi_2
   \end{pmatrix}
\]
doesn't works if $T(R)$ is a representation of $\SO(3)$ on $\eC^2$. If one allows $T$ to be a representation of $SU(2)$, then our choice ---for an electron--- should naturally be the spin one half representation $T=D^{(1/2)}$. Let us do it. The remaining problem is the following. Let's consider that in a certain frame, an electron is described by the wave function $\begin{pmatrix} \psi_1&\psi_2\end{pmatrix}$, the question is to know the wave function observed by a guy which use another frame linked to the first frame by $R\in \SO(3)$. We always have exactly two elements in $\SU(2)$ projected to $R$ by $\mSpin$; namely $\mSpin(\pm g)=R$; so how to choose between
\[
   D^{(1/2)}(g)
\begin{pmatrix}
     \psi_1\\
     \psi_2
   \end{pmatrix}
\quad\text{and}\quad
   D^{(1/2)}(-g)
\begin{pmatrix}
     \psi_1\\
     \psi_2
   \end{pmatrix}\; ?
\]
The trick is to remark that a change of frame is not the mathematical process described by a single element $R$ of $\SO(3)$, but a physical \emph{continuous} process which begins at the identity and stops at $R$. In other word, we have to ask ourself \emph{how to go from a frame to another}? Taking as example the rotations around the $x$ axis, we can look at two different path in $\SO(3)$ from $\mtu$ to $\mtu$ given by the same expression
\[
  R_1(t)=R_2(t)=
\begin{pmatrix}
1&0&0\\
0&\cos t&\sin t\\
0&-\sin t&\cos t
\end{pmatrix},
\]
but considering $t\colon 0\to 2\pi$ for $R_1$ and $t\colon 0\to 4\pi$ for $R_2$. The covering map $\dpt{\mSpin}{SU(2)}{\SO(3)}$ allows us to lift any path in $\SO(3)$ to a path in $SU(2)$ in an unique way providing a starting point. In other words, if $\mSpin(g)=R$,
\[
\begin{split}
  \exists !\, \tilde{R}(t)\in SU(2)\text{ such that }\mSpin\circ\tilde{R}=R \text{ and } \tilde{R}(0)=\mtu,\\
  \exists !\, \tilde{R}(t)\in SU(2)\text{ such that }\mSpin\circ\tilde{R}=R \text{ and } \tilde{R}(0)=-\mtu.
\end{split}
\]
The question is now: how to choose the right path among these two? The answer comes from the homotopy of $\SO(3)$: the path $R_1$ and $R_2$ belongs to two different classes.

Considering the ``change of frame''{} as a continuous process, the initial point is naturally chosen to be $\mtu$. With this choice, the lift of $R_1$ and $R_2$ are given by
\[
  g_1(t)=g_2(t)=
\begin{pmatrix}
\cos\frac{t}{2}&-i\sin\frac{t}{2}\\
-i\sin\frac{t}{2}&\cos \frac{t}{2}
\end{pmatrix}
\]
with $\dpt{t}{0}{2\pi}$ for $g_1$ and $\dpt{t}{0}{4\pi}$ for $g_2$. In $SU(2)$, the ending point of $g_1$ is $-\mtu$ while the one of $g_2$ is $\mtu$.

It is still possible to say a lot of interesting thinks about the space-time symmetry group of nature; let's just conclude saying that $SU(2)$ is more adapted to the rotations of non zero spin than $\SO(3)$. (it is  not intuitive!)
\subsection{How to implement the Poincaré group}
%----------------------------------------------
\label{pg:poincare_act}

We are not making physics here, but differential geometry and group theory; so we will not discuss the physical relevance of the Poincaré group from a ``speed of light''\ point of view. We consider the \defe{Poincaré group}{Poincaré!group} as the group of all the affine isometries of metric $\eta=diag(-1,1,1,1)$ and the \defe{Lorentz group}{Lorentz!group}\index{group!Lorentz} as the subgroup of rotations and boost.

A Poincaré transformation of $\eR^4$ is given by $(\Lambda,a)$ with $\Lambda$ a $4\times 4$ matrix and $a\in\eR^4$, a translation vector. The composition of $(\Lambda,a)$ with $(\Lambda',a')$ is given by $(\Lambda'\Lambda,\Lambda'a+a')$, the inverse is $(\Lambda^{-1},-\Lambda^{-1} a)$, the neutral is $(\mtu,0)$, and $(\det\Lambda)^2=1$.

The axiom \ref{ax:reprez} at page \pageref{pg:axiomes} gives us a group of transformation of the rays in $\pH$ parametrised by $(\Lambda,a)$ such that
\begin{equation}
  T(\Lambda',a')T(\Lambda,a)=T(\Lambda'\Lambda,\Lambda'a+a'),
\end{equation}
$\dpt{T(\Lambda,a)}{\Ray \pH}{\Ray \pH}$. Then Wigner's theorem defines a representation of the Poincaré group on $\pH$ by unitary matrices:
\[
\psi\to U(\Lambda,a)\psi.
\]

\begin{remark}
Wigner only ensure existence of \emph{projective} representations. Here we suppose that our symmetry group (maybe slightly different that Poincaré) is such that any projective representations can be turn into a classical representation. We will therefore use the composition law
\begin{equation}\label{eq:composition_U}
  U(\Lambda',a')U(\Lambda)=U(\Lambda'\Lambda,\Lambda'a+a')
\end{equation}
instead of $U(\Lambda',a')U(\Lambda,a')=e^{i\phi(\Lambda,a,\Lambda',a')}U(\Lambda'\Lambda,\Lambda'a+a')$.
\end{remark}

By axiom, the (connected) Poincaré group acts on rays of $\pH$, and we have the representation $U$ which form a group acting on $\pH$. The Lie algebra acts also:
\begin{equation}
u\psi=\Dsdd{U(t)}{t}{0}\psi:=\Dsdd{U(t)\psi}{t}{0}.
\end{equation}
This definition is natural because $\pH$ is a vector space: it can be identified with its tangent space: $U(t)\psi$ is a path in $\pH$ and its derivative at $t=0$ is still a well defined element in $\pH$. Now recall that the operators $U$ are unitary, so that the corresponding operators $u$ are hermitian (therefore diagonalisable).

Let us consider an abelian subgroup $A$ of Poincaré with Lie algebra $\lA$. One can find a basis of $\pH$ made of common eigenvectors of a basis of $\lA$. In other words, one can find a basis of $\pH$ which simultaneously diagonalises all $\lA$. If $\{a_i\}$ is a basis of $\lA$, one can find a basis $\{\ket{\psi_{\lambda}}\}$ (here $\lambda$ labels a basis of $\pH$: it might take continuous values) such that
\begin{equation}
   a_i\ket{\psi_{\lambda}}=\lambda_i\ket{\psi_{\lambda}}.
\end{equation}

\subsection{Momentum operator}
%-----------------------------

Of course, there exists an abelian subgroup of Poincaré: the pure translations, $A=\{U(\mtu,a)\}$. A basis of the Lie algebra is given by four vectors labeled as $P^{\mu}$ and defined by 
\[
P^{\mu}=\Dsdd{U(\mtu,te^{\mu})}{t}{0}
\]
 where $e^{\mu}$ is the unit vector following the direction $\mu$ (for $\mu=0$, $e^0=(1,0,0,0)$). One can consider a basis which diagonalises the $P^{\mu}$'s: 
\begin{equation}
  P^{\mu}\ket{p,\sigma}=p\hmu\ket{p,\sigma}
\end{equation}
where by definition,
\begin{equation}\label{eq:def_P}
   P\hmu\ket{p,\sigma}=\Dsdd{U(\mtu,te\hmu)\ket{p,\sigma}}{t}{0}.
\end{equation}

\begin{remark}
Be careful on a point: we don't say anything about the symbol ``$p$''\ in the ket. The only property is that it labels a Hilbert space $\pH$. But nothing is already imposed to $\pH$: it must just carry a representation of the Poincaré group on its rays. In particular, it is \emph{a priori} false to say that $p$ is a ``momentum $4$-vector'' and that $p\hmu$ is a component of $p$. Naturally, our notations are adapted to think that! Maybe it is a pedagogical mistake; I don't know. 
\end{remark}

This remark can be disturbing: why is generally $\ket{p,\sigma}$ called ``a state of momentum $p$''? Since $U(\mtu,a)$ is unitary, $P\hmu$ is hermitian; the $p\hmu$ are eigenvalues for an hermitian operator, so by axiom \ref{ax:vaps} (page \pageref{pg:axiomes}) they are candidate to be physical values. But equation \eqref{eq:def_P} shows that $P\hmu$ is what a physicist should call an ``infinitesimal translation'', so that Noether suggests us to interpret the eigenvalue as momentum. We are safe!

The parameters $\sigma$ are not yet defined neither. It will come later. For the moment, we include into the definition of a \defe{one particle state}{state!one particle} that $\sigma$ takes discrete values.
 
 Since $U(\mtu,a)=e^{a_{\mu} P\hmu}$,
\[
   U(\mtu,a)\ket{p,\sigma}=e^{ia_{\mu} p^{\mu}}\ket{p,\sigma}.
\]
Now we are interested in the determination of $U(\Lambda,a)\ket{p,\sigma}$.

\begin{proposition}
The operators $P^{\mu}$ are subject to the ``transformation law''
\begin{equation}
   U(\Lambda,a)P^{\mu} U(\Lambda,a)^{-1}=\Lambda^{\mu}_{\nu} P^{\nu}.
\end{equation}
\end{proposition}

\begin{proof}
Since operators $U(\Lambda,a)$ are linear, they can be putted in the derivative which defines $P\hmu$. Using the composition law  \eqref{eq:composition_U} we find:
\begin{equation}
\begin{split}
   U(\Lambda,a)P\hmu U(\Lambda,a)^{-1}&=\Dsdd{ U(\Lambda,a)U(\mtu,te\hmu)U(\Lambda,a)^{-1} }{t}{0}\\
                                     &=\Dsdd{U(\mtu,t\Lambda e\hmu)}{t}{0}.
\end{split}				     
\end{equation}
The $\Lambda$ can be putted out of derivative; let us see it for a sum of two terms (here it is four):
\begin{equation}
\begin{split}
  \Dsdd{ U(\mtu,t(e\hmu+e\hnu)) }{t}{0}&=\Dsdd{U(\mtu,te\hmu)U(\mtu,te\hnu)}{t}{0}\\
                                      &=\Dsdd{U(\mtu,te\hmu)U(\mtu,0)}{t}{0}
				        +\Dsdd{U(\mtu,0)U(\mtu,te\hnu)}{t}{0}\\
				      &=P\hmu+P\hnu.
\end{split}
\end{equation}
Thus 
\begin{equation}
    \Dsdd{U(\mtu,\Lambda\hmu_{\nu} e\hnu)}{t}{0}=\Lambda\hmu_{\nu}\Dsdd{U(\mtu,te\hnu)}{t}{0}
                                              =\Lambda\hmu_{\nu} P\hnu.
\end{equation}
\end{proof}

\subsection{Pure Lorentz transformation}
%--------------------------------------

Now we consider a pure Lorentz transformation $U(\Lambda)\equiv U(\Lambda,0)$, and we want to look at $U(\Lambda)\ket{p,\sigma}$. In order to see its decomposition into others $\ket{k,\sigma'}$, we apply a $P\hmu$:
\begin{equation}
\begin{split}  
  P\hmu U(\Lambda)\ket{p,\sigma}&=U(\Lambda)\Big( U(\Lambda)^{-1} P\hmu U(\Lambda)  \Big)\ket{p,\sigma}\\
                                &=U(\Lambda) {(\Lambda^{-1})}\hmu_{\nu} P\hnu \ket{p,\sigma}\\
				&=(\Lambda^{-1})\hmu_{\nu} p\hnu U(\Lambda)\ket{p,\sigma}.
\end{split}				
\end{equation}
Thus the vector $U(\Lambda)\ket{p,\sigma}\in\pH$ has $(\Lambda^{-1})\hmu_{\nu} p\hnu$ as eigenvalue for $P\hmu$. If the $p\hmu$'s are seen as components of a $4$-vector $p$, one can write
\[
    P\hmu U(\Lambda)\ket{p,\sigma}=(\Lambda p)\hmu U(\Lambda)\ket{p,\sigma};
\]    
     thus we naturally write  
\begin{equation}     
  U(\Lambda)\ket{p,\sigma}=\sum_{\sigma'}C_{\sigma'\sigma}(\Lambda,p)\ket{\Lambda p,\sigma'}.
\end{equation}
       

Note that we had not yet given anything about the nature of the $p$ in the ket $\ket{p,\sigma}$ so we can \emph{define} the product $\Lambda p$ by the fact that the ket $\ket{\Lambda p,\sigma}$ has eigenvalue $(\Lambda^{-1})\hmu_{\nu} p\hnu$ for the operator $P\hmu$. So it is one of the $\ket{p',\sigma'}$.

\subsection{Rebuilding of a basis for \texorpdfstring{$\pH$}{H}}
%----------------------------------------

From general considerations about the Lorentz group (many physicists had written very better books than me about) anyone knows that the only functions of the $p\hmu$'s  which are invariant under all the Lorentz transformations are $p^2=\eta_{\mu}nu p\hmu p\hnu$ and the sign of $p^0$ when $p^2< 0$.

For any value of $p^2$ and sign of $p^0$, one consider a ``standard vector''\ $k$. For example:
\begin{subequations}
\begin{align}
   k&=(1,0,0,1)&&\text{for }p^2=0,\\
   k&=(1,0,0,0)&&\text{for }p^2<0,p^0>0\\
   k&=(-1,0,0,0)&&\text{for }p^2<0,p^0<0.
\end{align}   
\end{subequations}
With this convention, $p$ can be written as $p=L(p)k$ for a suitable Lorentz transformation $L(p)$. The vector $U(L(p))\ket{k,\sigma}$ has eigenvalue $L(p)k$ for the operator $P$, thus it is a linear combination of some $\ket{p,\sigma'}$.

Now we will cheat and redefine our basis of the Hilbert space $\pH$. First, we consider a fixed $k$; in other words, we build the state space for a given particle which has given momentum $p$.  The basis vectors must be eigenvectors for the fours operators $P\hmu$. As far as we say no more, any eigenvalue is possible. Thus our basis must be labelled by at least an element $p$ of $\eR^4$ with only one constraint: the value of $p^2$ (plus eventually the sign of $p^0$). So we define the $\ket{k,\sigma}$ to be such that
\[
   P\hmu\ket{k,\sigma}=k\hmu\ket{k,\sigma}.
\]
Since we know that with this definition of $\ket{k,\sigma}$, the eigenvalue of $U(L(p))\ket{k,\sigma}$ for $P\hmu$ is $p\hmu$, we \emph{define} $\ket{p,\sigma}$ as
\begin{equation}
  \ket{p,\sigma}=N(p)U(L(p))\ket{k,\sigma}.
\end{equation}
where $N(p)$ is a normalization to be discussed later. With this construction, we have an eigenvector for any possible eigenvalue for $P\hmu$. We have to show that these vectors are linearly independent.

The set of the $\ket{p,\sigma}$ with different $p$ is free in $\pH$ because they are eigenvectors for different eigenvalue of an hermitian operator\quext{I did not checked that it is sufficient}. There are no reason to think that the set of operators $P\hmu$ is complete; in other words, it remains not clear that there exist only one way to diagonalise the all the $P\hmu$. The function of the extra label $\sigma$ is to label different linearly independent vectors with same eigenvalue for $P$.

From now, we are interested in $\ket{k,\sigma}$ and $N(p)$.

\subsection{Little group}
%-----------------------

We have:
\begin{equation}
\begin{split}
  U(\Lambda)\ket{p,\sigma}&=N(p)U(\Lambda L(p))\ket{k,\sigma}\\
                          &=N(p)U(L(\Lambda p))U\big(  L(\Lambda p)^{-1}\Lambda L(p) \big)\ket{k,\sigma},
\end{split}
\end{equation}
So we will try to understand the operation $L(\Lambda p)^{-1}\Lambda L(p)$. First remark that
\[
   U(L(\Lambda p)^{-1})\ket{\Lambda p,\sigma}=N(\Lambda p)\ket{k,\sigma},
\]
and then compute:
\begin{equation}
\begin{split}
  U(L(\Lambda p)^{-1}\Lambda L(p) )N(p)\ket{k,0}&=U(L(\Lambda p)^{-1}\Lambda)\ket{p,\sigma}\\
                                               &=U(L(\Lambda p)^{-1})\sum_{\sigma'}
					         C_{\sigma'\sigma}(\Lambda,p)\ket{\Lambda p,\sigma'}\\
					       &=\sum_{\sigma'}C_{\sigma'\sigma}(\Lambda,p)
					         N(\Lambda p)\ket{k,\sigma'}.
\end{split}
\end{equation}

The \defe{little group}{little group} is the subgroup of the Lorentz transformations which leaves the chosen standard vector $k$ invariant: $Wk=k$. For any $W$ in the little group,
\[
   U(W)\ket{k,\sigma}=\sum_{\sigma'}D_{\sigma'\sigma}(W)\ket{k,\sigma'}
\]
With this definition, the $D$'s form a representation of the little group. Indeed for any $V,W$ in the little group,
\begin{equation}
\begin{split}
  \sum_{\sigma'}D_{\sigma'\sigma}(VW)\ket{k,\sigma'}&=U(VW)\ket{k,\sigma}\\
                                    &=U(V)\sum_{\sigma''}D_{\sigma''\sigma}(W)\ket{k,\sigma''}\\
				    &=\sum_{\sigma'\sigma''}D_{\sigma'\sigma''}(V)D_{\sigma''\sigma}(W)
				      \ket{k,\sigma'}.
\end{split}
\end{equation}
Since we want the $\ket{p,\sigma}$ with different $p$ and $\sigma$ to form a basis of $\pH$, they are linearly independent, then we can get rid of the sum over the $\sigma'$ and keep the equation
\[
D_{\sigma'\sigma}(VW)=\sum_{\sigma''}D_{\sigma'\sigma''}(VW)D_{\sigma''\sigma}(VW);
\]
if we adopt a more ``matricial''\ notation,
\begin{equation}
D(VW)=D(V)D(W).
\end{equation}

We are now able to perform a step in the study of the vector $U(\Lambda)\ket{p,\sigma}$. We naturally define $W(\Lambda,p)=L(\Lambda p)^{-1}\Lambda L(p)$. This belongs to the little group\footnote{Pay attention that $L(p)$ depends implicitly on the choice of $k$.}. Then,
\begin{equation}
\begin{split}
  U(\Lambda)\ket{p,\sigma}&=N(p)U(L(\Lambda p))U(W(\Lambda,p))\ket{k,\sigma}\\
                          &=N(p)\sum_{\sigma'}D_{\sigma'\sigma}(W)U(L(\Lambda p))\ket{k,\sigma'}\\
			  &=\frac{N(p)}{N(\Lambda p)}\sum_{\sigma'}D_{\sigma'\sigma}(W(\Lambda,p))
			      \ket{\Lambda p,\sigma'}.
\end{split}
\end{equation}

But we have no constraint on the $D$'s: it must just form a representation of the little group. Consequently, we are at a point in which our axioms are no more sufficient to continue the building of quantum field theory: we will get as many theories as representations of the little group.

The physical interpretation is the following\label{pg:phyz_reprez}: each type of particle has its own representation. When we consider a Hilbert space on which $U(\Lambda)$ acts via one given representation of the little group, we consider the Hilbert space which describes the corresponding particle. Note that the little group depends on the choice of $k$, and therefore depends on the particle which is studied (massive or not).

In this sense, a particle is a representation of the Poincaré group\quext{I think that the irreducibility of a representation is related to \emph{elementary} particles.}. In particular, the nature of the index $\sigma$ can change from the one representation to the other.

\begin{remark}
As far as normalization is concerned, we will pose
\[
  N(p)=\sqrt{k^0/p^0}.
\]
There are some good reasons to take it; but it is irrelevant from our group point of view of the theory.
\end{remark}

\subsection{Positive mass}
%------------------------

This is the easy case. The choice of standard momentum is $k=\begin{pmatrix}1&0&0&0\end{pmatrix}$. One could believe that the little group is $\SO(3)$. It would be the case if we had chosen $L_+^{\uparrow}$ instead of $\SLdc$ --see point \ref{subsec:sym_nature}. In our hermitian representation of $\eR^4$, $k=\mtu$. Then a matrix of $\SLdc$ which leaves it invariant fulfills
\[
   \lambda k\lambda^{\dag}=\lambda\lambda^{\dag}=\mtu,
\]
this is $\lambda\in SU(2)$. By the way, note that $\SO(3)=SU(2)/\eZ_2$.

The celebrated ``law of transformation''\ of a massive particle of spin $j$ (integer or half integer) under the Lorentz transformation $\Lambda$ is 
\begin{equation}
  U(\Lambda)\ket{p,\sigma}=\sqrt{\frac{ (\Lambda p)^0 }{p^0}}
       \sum_{\sigma'}D^{(j)}_{\sigma'\sigma}(W(\Lambda,p))\ket{\Lambda p,\sigma'}
\end{equation}
where $\sigma$ runs from $-j$ to $j$ by step of $1$.

\subsection{Null mass}
%---------------------

In the case of a null mass, the standard vector is $k=\qvect{1}{0}{0}{1}$ and an element of the little group fulfils $Wk=k$. As the little group is part of the Lorentz group, this is an isometry, so
\begin{subequations}
\begin{align}
  \scalh{Wt}{Wk}&=\scalh{t}{k}\\
  \scalh{Wt}{Wt}&=\scalh{t}{t},
\end{align}  
\end{subequations}
for any $t\in\eR^4$. Taking in particular $t=\qvect{1}{0}{0}{0}$,
\begin{subequations}
\begin{align}
  (Wt)^{\mu}k_{\mu}&=t\hmu k_{\mu}=-1\\
  (Wt)^{\mu}(Wt)_{\mu}&=t\hmu t_{\mu}=-1.  
\end{align}  
\end{subequations}
If we write $Wt=({a},{b},{c},{d})$,  the first relation gives $d=a-1$, so that  $Wt=({1+\xi},{\alpha},{\beta},{\xi})$, while the second one gives $\xi=(\alpha^2+\beta^2)/2$. The conclusion is that $W$ acts on $t$ as a certain Lorentz transformation $S(\alpha,\beta)$:
\begin{equation}
Wt=\begin{pmatrix}
     1+\xi\\
     \alpha\\
     \beta\\
     \xi
   \end{pmatrix}=
   \begin{pmatrix}
     1+\xi  & -\xi    & \alpha & \beta\\
     \alpha & -\alpha &   1    &   0\\
     \beta  & -\beta  &   0    &   1\\
     \xi    & (1+\xi) & \alpha & \beta.
   \end{pmatrix}
   \begin{pmatrix}
     1\\
     0\\
     0\\
     0
   \end{pmatrix}.
\end{equation}
Be careful: it doesn't means that $W=S$, but $Wt=St$. However it is an information: $S(\alpha,\beta)^{-1} W$ is a Lorentz transformation which leaves $t$ invariant. Then it is a spatial rotation. More precisely, since $W$ and $S$ conserve $\qvect{1}{0}{0}{1}$, it is a rotation around the $z$ axis: $S(\alpha,\beta)^{-1} W=R(\theta)$, and
\begin{equation}
  W(\theta,\alpha,\beta)=S(\alpha,\beta)R(\theta)
\end{equation}
is the most general element of the non massive little group.
