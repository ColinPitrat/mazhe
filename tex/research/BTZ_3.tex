\subsection{The metric}
%----------------------

We compute the metric in the basis $\{\dtau,\du,\dphi\}$ at the point $z=e^{\phi H}e^{eE}e^{\tau T}e^{\alpha\phi H}$:
\begin{subequations}
  \begin{align}
  \begin{split}
   dL_{z^{-1}}\dtau&=\Dsdd{  e^{-\alpha\phi H}e^{-\tau T}e^{-uE}e^{-\phi H}e^{\phi H}e^{uE}e^{(\tau+t)T}e^{\alpha\phi H}  }{t}{0}\\
             &=\Ad(e^{-\alpha\phi H})T.
\end{split}
\intertext{In the same way,}
  dL_{z^{-1}}\du&=\Ad(e^{-\alpha\phi H}e^{-\tau T})E\\
  dL_{z^{-1}}\dphi&=\Ad(z^{-1})H+\alpha H.s
  \end{align}
\end{subequations}
In the computation of $dL_{z^{-1}}\dphi$, we used the Leibnitz rule. The rest is a rather mechanical computation using the $\Ad$-invariance of $B$ and the fact that, by definition if $B_z$, $B_z(X,Y)=B_e(dL_{z^{-1}}X,dL_{z^{-1}}Y)$. The following relations will be useful:
\begin{subequations}
\begin{align}
  \Ad(e^{uE})T&=T-nH+n^2E\\
  \Ad(e^{uE})H&=H+2nE\\
  \Ad(e^{-\tau T})E&=-\sin\tau\cos\tau H+\cos^2\tau E-\sin^2\tau F
\end{align}
\end{subequations}
\begin{align}
  B(\dtau,\dtau)&=B(T,T)=-8\\
\begin{split}
  B(\dtau,\du)  &=B(T,\Ad(e^{-\tau T})E)\\
                &=B(\Ad(e^{\tau T})T,E)\\
		&=B(E-F,E)=4,
\end{split}\\
\begin{split}
     B(\dtau,\dphi)&=B(\Ad(e^{-\alpha\phi H})T,\alpha H)\\
                     &\quad +B(T, \Ad(e^{-\tau T}e^{-uE}e^{-\phi H})H  )\\
		   &=\underbrace{B(T,\alpha H)}_{=0}+B( \Ad(e^{uE})T,\Ad(e^{-\phi H})H )\\
		   &=B(\Ad(e^{uE})T,H)\\
		   &=-4n,
\end{split}\\
  B(\du,\du)&=B(E,E)=0,\\
\begin{split}
  B(\du,\dphi)&=B(  E,\Ad(e^{-uE}e^{-\phi H})H )\\
              &\quad+B( \Ad(e^{-\tau T})E,\alpha H )\\
	      &=B(-\sin\tau\cos\tau H+\cos^2\tau E-\sin^2\tau F,H)\\
	      &=-2\alpha\cos(2\tau).
\end{split}
 \end{align}

 % Il faut encore taper le B(\dphi,\dphi) dès le haut de /84
