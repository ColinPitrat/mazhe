% This is part of (almost) Everything I know in mathematics
% Copyright (c) 2013-2014,2017-2018
%   Laurent Claessens
% See the file fdl-1.3.txt for copying conditions.

\section{Ordering relation}
%+++++++++++++++++++++++++++++

We know from proposition~\ref{PropAplusConvexCone} that the set \(\cA^+\) of positive elements in the $C^*$-algebra \(\cA\) is a convex cone in \(\cA\). We saw in subsection\ref{SubsecPosiCconePartOrder} that in a real vector space a convex cone is equivalent to a notion of positivity and to a partial ordering. Here we consider the real vector space \(\cA_{\eR}\) and we say that $A\leq B$ if\nomenclature[C]{$\leq$}{Ordering relation in $C^*$-algebra} when $B-A\in\cA^+$. Thus we write
\begin{equation}
    \cA^+=\{ A\in\cA_{\eR}\tq  A\geq 0\}.
\end{equation}


\begin{proposition}     \label{PropAAsmAuAAu}
If $A=A^*$, then
\[
  -\| A \|\cun\leq A\leq\| A \|\cun.
\]
\end{proposition}

\begin{proof}
Since $A=A^*$ and $\| A \|\in\eR^+$, we know that $\| A \|\cun-A\in\cA_{\eR}$. We have to show that $\| A \|\cun-1$ is positive. In other words we have to show that its Gelfand transform is pointwise positive in $C(\sigma(A))$. We consider $\| A \|\cun-A$ as an element of $C^*(A,\cun)$ and the Gelfand transform gives
\[
  \| A \|1_{\sigma(A)}-\hat A.
\]
We will explain just after the proof why we write $1_{\sigma(A)}$ instead of $1$. If one apply this on an element of $\sigma(A)$, one has to remember that $\hat A(t)=t$, so $\hat A(t)$ is at most $r(A)$ because $\sigma(A)\subset\eR^+$. Since $r(A)\leq\| A \|$, one sees that
\begin{equation}
    \big( \| A \|1_{\sigma(A)}-\hat A \big)(t)=\| A \|-t,
\end{equation}
but $t\in\sigma(A)$ implies $t\leq r(A)$. Then the latter expression is positive and $\| A \|\cun-A\in\cA_{\eR}^+$.

\end{proof}

Let us now see why $\hat \cun=1_{\sigma(A)}$ in $C^*(A,\cun)$. First, we consider $\cun\in C^*(A,\cun)$; from properties of characters, $\hat\cun(\omega)=\omega(\cun)=1$. So considering $\cun\in C^*(A,\cun)$, the character $\hat\cun$ is defined by $\hat\cun(\omega)=1\in\eR$ for all $\omega\in\Delta(C^*(A,\cun))$. Now, we know that there exists an homeomorphism between $\Delta(C^*(A,\cun))$ and $\sigma(A)$. Under this homeomorphism, $\cun(t)=t$ for all $t\in\sigma(A)$.

Since there is a bijection between $\sigma(A)$ and $\Delta(C^*(A,\cun))$, it should make no sense to define $\hat\cun$ outside of $\sigma(A)$.

\begin{proposition} \label{prop:mBABineq}
If $-B\leq A\leq B$, then $\| A \|\leq\| B \|$.
\end{proposition}

\begin{proof}
The relations $-B\leq A\leq B$ and $-\| B \|\cun\leq B\leq\| B \|\cun$ give $-B\leq A\leq B\leq\| B \|\cun$. But $B\leq\| B \|\cun$ implies $-\| B \|\cun\leq-B$ from definition of a linear partial order. Then $-\| B \|\cun\leq A\leq\| B \|\cun$. All this make that $\sigma(A)\subseteq[-\| B \|,\| B \|]$ and then that $\| A \|\leq\| B \|$ because $r(A)=\| A \|$ when $A=A^*$ and $r(A)=\sup\{ | z |\tq z\in\sigma(A) \}$.
\end{proof}

\begin{proposition}
Let $A$, $B\in\cA^*$ such that $\| A+B \|\leq k$. Then $\| A \|\leq k$.
\end{proposition}

\begin{proof}
From assumptions, $A=A^*$, the $A\leq\| A \|\cun$. Taking this relation with $A+B$ instead of $A$, we get $A+B\leq\| A+B \|\cun\leq k\cun$. From linearity of the partial order, this implies $0\leq A\leq k\cun -B$. Since $k\geq 0$, $-k\cun\leq 0$ and $k\cun-B\leq k\cun$ because $0\leq B$. Finally, proposition~\ref{prop:mBABineq}  makes that $\| A \|\leq k\| \cun \|$ implies $\| A \|\leq k$.
\end{proof}

\begin{proposition}[\cite{DixmierTrace}] \label{PropMDfqcUs}
Let $\cA$ be an involutive Banach algebra, $\cB$ a $C^{*}$-algebra and $\dpt{\pi}{\cA}{\cB}$ a morphism. Then $\forall A\in\cA$,
\begin{equation}  \label{eq_morleqpi}
  \|\pi(A)\|\leq\|A\|
\end{equation}
and \( \pi\) is continuous.
\end{proposition}

\begin{proof}
If $B\in\cB$ is hermitian, $\|B^2\|=\|B^*B\|=\|B\|^2$. An induction shows that $\|B^{2n}\|^{-2n}=\|B\|$. With $n\to\infty$, the left hand side goes to the spectral radius (\ref{prop:An_usn}). Then
\begin{equation}
r(B)=\|B\|
\end{equation}
when $B$ is hermitian.

Let us now take $A\in\cA$. We have $\sigma_{\cB}(\pi(A))\subset \sigma_{\cA}(A)$ because $(\pi(A)-\lambda\mtu)v=\mtu$ for a $v\neq 0$ in $\cB$ implies $(A-\lambda\mtu)\pi^{-1}(v)=\mtu$ with $\pi^{-1}(v)\neq 0$ (because $\pi(\mtu)=\mtu$).

Remark that $(A-\lambda\mtu)(a)=0$ with $a\neq 0$

The continuity of \( \pi\) is now a consequence of lemma~\ref{lem:lin_vec_cont}.
\end{proof}

\begin{proposition}[\cite{DixmierTrace}] \label{prop:vp_geq}
Let $\cA$ be a $C^{*}$-algebra $\cB$ an involutive normed algebra and $\dpt{\varphi}{\cA}{\cB}$ an injective morphism. Then $\forall A\in\cA$,
\begin{equation}
   \|\varphi(A)\|\geq\|A\|.
\end{equation}
\end{proposition}

\begin{proof}
Consider $A\in\cA$, and suppose that $\|\varphi(A^*A)\|\geq\|A^*A\|$. Then
\[
\|A^2\|=\|A^*A\|\leq\|\varphi(A^*A)\|=\|\varphi(A)^*\varphi(A)\|\leq\|\varphi(A)\|^2,
\]
so that we can only consider the case where $A$ is hermitian. One can also consider only the restriction of $\varphi$ to the sub$C^{*}$-algebra generated by $A$ and then suppose that $\cA$ is commutative. In the same way, we consider only $\varphi(\cA)$ instead of $\cB$ so that $\cB$ can also be considered as commutative. In other words, we consider $C^*(A,\cA)$ and $C^*(\varphi(A),\mtu)$ instead of $\cA$ and $\cB$. But from now to the end of the proof, we will often write it as $\cA$ and $\cB$.

Now, we consider the completion of $\cB$ and we add an unit to $\cA$ and $\cB$. Then the structure spaces $\mS:=\Delta(C^*(A,\mtu))$ and $\mT:=\Delta(C^*(\varphi(A),\mtu))$ are compacts in the weak topology.

If $\chi\in\mT$, it is clear that $\chi\circ\varphi$ is a character because these are both multiplicative. Then $\chi\circ\varphi\in\mS$, and we denotes it by $\varphi'(\chi)$. It is easy to prove that $\dpt{\varphi'}{\mT}{\mS}$ is a continuous map. Indeed, an open set in $\mS=\Delta(C^*(A,\mtu))$ is of the form
\[
  \hB^{-1}(\mO)=\{ \omega\in\mS:\omega(B)\in\mO  \}
\]
 for $B\in C^*(B,\mtu)$. But
 \begin{equation}
\begin{split}
   {\varphi'}^{-1}(\hB^{-1}(\mO))&=\{ \chi\in\Delta(\cB):\varphi'(\chi)\in\hB^{-1}(\mO)  \}\\
                         &=\{ \chi\in\Delta(\cB): (\chi\circ\varphi)(B)\in\mO  \} \\
             &=\widehat{\varphi(B)}^{-1}(\mO),
\end{split}
\end{equation}
which is an open subset of $\mT=\Delta(\cB)$.

 Thus $\varphi'(\mT)$ is a compact subset of $\mS$.

Suppose that $\varphi'(\mT)\neq\mS$, and consider two functions $f,g$ on $\mS$ such that $fg=0$, but $f\neq 0$ and $g=1$ on $\varphi'(\mT)$. The Urysohn lemma gives a function $f$ which works for it: $f=0$ on $\varphi'(\mT)$ and $f\neq 0$ out of $\varphi'(\mT)$. Now, $f$, $g\in C(\Delta(\cA))$ but we have gives a homomorphism between $\cA$ and $C(\Delta(\cA))$. Then we have $A$, $B\in\cA$ such that $\hat{A}=f$,  $\hat{B}=g$ with $AB=0$, $x\neq 0$, $\chi(\varphi(y))=1$ for any $\chi\in\mT$. From this last equality one conclude that $\varphi(B)$ is invertible in $\cB$. But $\varphi(A)\varphi(B)=0$ and $\varphi(A)\neq 0$, which contradict.

Thus $\varphi'(\mT)=\mS$, so that $\forall A\in\cA$, $\forall\xi\in\mS$, there exists $\chi\in\mT$ such that $\xi(A)=\varphi'(\chi)(A)$. Finally, for any $x\in\cA$,
\begin{equation}
 \|A\|=\sup_{\xi\in\mS}|\xi(A)|
      =\sup_{\chi\in\mT}|\varphi'(\chi)(A)|
      =\sup_{\chi\in\mT}|\chi\circ\varphi(A)|
      \leq\|\varphi(A)\|.
\end{equation}

\end{proof}

\begin{proposition}
    Every \( C^*\)-algebra isomorphism is isometric.
\end{proposition}

\begin{proof}
    This is a combination of propositions~\ref{PropMDfqcUs} and~\ref{prop:vp_geq}.
\end{proof}

\begin{lemma}  \label{lem:injmorpisom}
An injective $C^*$-algebra morphism is is always an isometry and in particular, the image is closed.
\end{lemma}

\begin{proof}
Let $\varphi$ be the morphism and suppose that there exists a $B\in\cA$ such that $\| \varphi(B) \|\neq \| B \|$. Then
\[
  \| \varphi(B^*B) \|=\| \varphi(B) \|^2\neq\| B \|^2=\| B^*B \|.
\]
We pose $A=B^*B$ and we know that $\| A \|=\sqrt{r(A^*A)}$ because $A^*=A$. By definition, $r(B)=\sup\{ | z |\tq z\in\sigma(B)) \}$. If we apply $\sigma(f(A))=f(\sigma(A))$ to $f(t)=t^2$, we conclude that $\sigma(A)\neq\sigma(\varphi(A))$. We saw, during proof of~\ref{PropMDfqcUs}, that $\sigma(\varphi(A))\subseteq\sigma(A)$. Then we have a strict inclusion
\[
   \sigma(\varphi(A))\subset\sigma(A).
\]
Thus there exists a continuous function $f\neq 0$ on $\sigma(A)$ such that $f(x)=0$ when $x\in\sigma(\varphi(A))$. In particular $f(\varphi(A))=0$ and by~\ref{lem:fvpvpf}, we see that $\varphi(f(A))=0$. This is in contradiction with the injectivity of $\varphi$.

\end{proof}

\begin{corollary}
The image of a $C^*$-algebra morphism $\dpt{\varphi}{\cA}{\cB}$ is closed. In particular $\varphi(\cA)$ is a $C^*$-subalgebra of $\cB$.
\end{corollary}

\begin{proof}
We  define $\dpt{\psi}{\cA/\ker\varphi}{\cB}$ by $\psi([a])=\varphi(a)$. It is a (bijective) vector space isomorphism, and $\varphi=\psi\circ\tau$. Since $\varphi$ is a morphism, $\ket \varphi$ is an ideal and then $\cA/\ker\varphi$ is a $C^*$-algebra. Now $\varphi=\psi\circ\tau$ is an injective morphism of $C^*$-algebra and its image is closed: $\psi(\cA/\ket\varphi)$ is closed in $\cB$. Since $\varphi$ is a morphism, its range is a $*$-algebra in $\cB$ which is closed for the norm in $\cB$. With induced operation from $\cB$, $\varphi(\cA)$ becomes a $C^*$-algebra.
\end{proof}

\section{Approximate unit}
%++++++++++++++++++++++++++++++++

\begin{definition}
 A \defe{approximate unit}{approximate unit} of a normed algebra $\cA$ is a family $(u_i)_{i\in I}$ (where $I$ is an increasing filtered set, in order to makes sense to $i\to\infty$) such
that
\begin{enumerate}
\item  $\forall i$, $\|u_i\|\leq 1$,
\item  $\forall A\in\cA$, $\|u_iA-A\|\to 0$ and $\|Au_i-A\|\to 0$. \label{enuoii}
\end{enumerate}
\label{def:app_unit}
\end{definition}
It is clear that if $\cA$ has an unit $\mtu$, the choice $u_i=\mtu\,\forall i$ is an approximate unit.

\subsubsection*{Example}

Let $C_0(\eR)$, the $C^*$-algebra of continuous functions on $\eR$ such that for all $\varepsilon>0$, there exists a compact $K$ outside of which $| f(x) |<\varepsilon$. This has no unit, but we can build an approximate unit as following. Let $n\in\eN$ and define
\[
  \cun_n=
\begin{cases}
 1& \textrm{on $[-n,n]$}\\
 0&\textrm{when $| x |>n+1$}\\
\textrm{continuous}&\textrm{otherwhise}
\end{cases}
\]
It should be noted that the limit $\lim_{n\to\infty}\cun_n$ in the norm  $\| . \|_{\infty}$ is \emph{not} $1_{\eR}$.

\begin{proposition}
All non unital $C^*$-algebra admits an approximate unit. If $\cA$ is separable, then $\Lambda$ can be chosen countable.
\end{proposition}

\begin{proof}
Let $\Lambda$ be the set of finite subsets of $\cA$ endowed with the partial ordering given by inclusion. To each $\lambda=\{ A_1,\ldots,A_n \}\in\Lambda$, we define $B_{\lambda}=\sum_{i=1}^nA_i^*A_i$. From construction, $B_{\lambda}^*=B_{\lambda}$ and $B_{\lambda}\in\cA^+$. The latter point gives $\sigma(B_{\lambda})\subset\eR^+$. Si for all $z\in\eR^-$, the element $A-z\cun$ is invertible (in $\cA_{\cun}$) and $n^{-1}\cun+B_{\lambda}$ is invertible when $n<0$. It allows us to define
\[
  \cun_{\lambda}=B_{\lambda}(n^{-1}\cun+B_{\lambda})^{-1}.
\]
Since $B_{\lambda}=B_{\lambda}^*$ and $B_{\lambda}$ commutes with all functions of itself (as $(n^{-1}\cun+B_{\lambda})^{-1}$), it leads to $\cun_{\lambda}=\cun_{\lambda}^*$. The element $(n^{-1}\cun+B_{\lambda})^{-1}$ is computed in $\cA_{\cun}$, the it can be written as $C+\mu\cun$ for certain $C\in\cA$ and $\mu\in\eC$. However, $\cun_{\lambda}\in\cA$. Indeed
\[
  \cun_{\lambda}=B_{\lambda}(C+\mu\cun)=B_{\lambda}C+\mu B_{\lambda}\in\cA.
\]

Now we use the continuous functional calculus on $B_{\lambda}$ with $f(f)=\frac{t}{n^{-1}+t}$; more precisely, we will use formula $\sigma(f(B_{\lambda}))=f(\sigma(B_{\lambda}))$ where we know that $\sigma(B_{\lambda})\subset\eR^+$. We find $f(B_{\lambda})=B_{\lambda}(n^{-1}\cun+B_{\lambda})^{-1}=\cun_{\lambda}$. Then
\[
  f(\sigma(B_{\lambda}))\subset f(\eR^+)=[0,1].
\]
In particular, $\sigma(\cun_{\lambda})\subset[0,1]$.

In order to prove the second condition of definition~\ref{def:app_unit}, we pose $C_i=\cun_{\lambda} A_i-A_i$. One can prove that \quext{Je ne vois pas comment.}
\[
  \sum_{i=1}^n C_iC_i^*=n^{-2}B_{\lambda}(n^{-1}+t)^{-2}.
\]
 We consider the function $f(t)=n^{-2}t(n^{-1}+t)^{-2}$ on $\sigma(B_{\lambda})$, then $f\geq 0$ and in particular
\[
  \sup_{t\in\eR^+}| f(t) |=\frac{1}{4n}
\]
and $f(t)$ takes its maximum at $t=1/n$. Since $f$ is defined on $\sigma(B_{\lambda})$ which is a part of $\eR^+$, we have $\| f \|_{\infty}\leq 1/4n$. Then $\| n^{-2}B_{\lambda}(n^{-1}\cun+B_{\lambda})^{-2} \|\leq 1/4n$ and
\[
  \| \sum_i C_iC_i^* \|\leq\frac{1}{4n}.
\]
In particular, for each $i$, we have $\| C_iC_i^* \|\leq \frac{1}{4n}$.

Now consider $A\in\cA$. Then $A$ belongs to a $\lambda\in\Lambda$ and we can build a directed subset of $\Lambda$ in which $A$ is always present. For $n\to\infty$,
\begin{equation}
\begin{split}
  \lim_{\lambda\to\infty}\| \cun_{\lambda}-A \|^2&=\lim_{\lambda\to\infty}\| (\cun_{\lambda}A-A)^*(\cun_{\lambda}A-A) \|\\
                                                &=\lim_{\lambda\to\infty}\| C_iC_i^* \|\\
                                                &=0
\end{split}
\end{equation}
because $\lambda\to\infty$ needs $n\to\infty$ and $\| C_iC_i^* \|\leq 1/4n$.

If $\cA$ is separable, then we can find a countable set of $A_i$ dense in $\cA$. We build $\Lambda$ from them and the proof works because when $\lambda\to\infty$, any $A\in\cA$ is reached from density.

\end{proof}


\begin{lemma} \label{lem:taulimA}
Let $\{ \cun_{\lambda} \}$ be an approximate unit in the ideal $\cI$ and $A\in\cA$. Then
\begin{equation}
\| \tau(A)\|=\lim_{\lambda\to\infty}\| A-A\cun_{\lambda} \|.
\end{equation}

\end{lemma}


\begin{proof}
First remark that property~\ref{enuoii} of definition of an approximate unit don't imply that the above limit is zero because it holds when $\{ \cun_{\lambda} \}$ is an approximate unit in $\cA$.

Definition $\| \tau(A) \|:=\inf_{J\in\cI}\| A+J \|$ immediately gives
\begin{equation} \label{eq:ir512_2}
  \|\tau(A)\|\leq \| A-A\cun_{\lambda} \|.
\end{equation}

In order to prove the inverse inequality, we add (if necessary) an unit to $\cA$ and we consider a $J$ in $\cI$. Then
\begin{equation} \label{eq:ri512}
\begin{split}
  \| A-A\cun_{\lambda} \|&=\| (A+J)(\cun-\cun_{\lambda})+J(\cun_{\lambda}-\cun) \|\\
                         &\leq \| A+J \|\| \cun-\cun_{\lambda} \|+\| J\cun_{\lambda}-J \|.
\end{split}
\end{equation}

Remember equation \eqref{eq:norcin}: for all $c\geq \| A \|$, we have $\| c\cun-A \|\leq c$. Let us write it  with $\cun_{\lambda}$ instead of $A$ and $c=1$:
\[
  \| \cun-\cun_{\lambda} \|\leq 1.
\]
On the other hand, equation \eqref{eq:ri512} with $\lambda\to\infty$ (and then with $\| J\cun_{\lambda}-J \|\to 0$) gives
\[
  \lim_{\lambda\to\infty}\| A-A\cun_{\lambda} \|\leq\| A+J \|.
\]
From definition of the norm on $\cA/\cI$ and of an infimum, for all $\varepsilon>0$, there exists a $J\in \cI$ such that $\| \tau(A) \|+\varepsilon\geq\| A+J \|$. Let us fix an $\varepsilon$ and such a $J$. Then, using equation \eqref{eq:ir512_2} ,
\[
  \| A+J \|-\varepsilon\leq\| \tau(A) \|\leq\| A-A\cun_{\lambda} \|,
\]
 and then
\[
  \lim_{\lambda\to\infty}\| A-A\cun_{\lambda} \|-\varepsilon \leq\| \tau(A) \|\leq\| A-A\cun_{\lambda} \|,
\]
letting $\varepsilon\to 0$, it gives the lemma:
\[
  \| \tau(A) \|\geq \lim_{\lambda\to\infty}\| A-A\cun_{\lambda} \|.
\]
\end{proof}

\begin{proposition} \label{prop:ideal_Banach}
If $\cI$ is an ideal in a Banach algebra $\cA$, then the quotient $\cA/\cI$ becomes a Banach algebra with the norm
\begin{equation}  \label{eq:norm_ideal}
\|\tau(A)\|=\inf_{J\in\cI}\|A+J\|
\end{equation}
and the multiplication rule
\begin{equation}   \label{eq:prod_ideal}
\tau(A)\tau(B)=\tau(A)\tau(B).
\end{equation}
\end{proposition}

\begin{proof}
The fact that it is a Banach space is non trivial\cite{thomaslassen}. We begin to prove that \eqref{eq:prod_ideal} is a well defined product:
\begin{equation}
   \tau(A+J_1)\tau(A+J_2)=\tau(AB+AJ_2+J_1B+J_1J_2)
                         =\tau(AB)
\end{equation}
because $AJ_2+J_1B+J_1J_2\in\cI$.

Next we have to prove that $\|\tau(A)\tau(B)\|\leq\|\tau(A)\|\|\tau(B)\|$. Remark that
\begin{equation} \label{eq:tauAA}
\|\tau(A)\|\leq\|A\|
\end{equation}
because
\[
  \|\tau(A)\|=\inf\|A+J\|
             \leq\inf(\|A\|+\|J\|)
             =\|A\|+\inf\|J\|
             =\|A\|.
\]
Then $\forall\varepsilon>0$, there exists a $J\in\cI$ such that
\begin{equation} \label{eq:tauAeps}
\|\tau(A)\|+\varepsilon\geq\|A+J\|
\end{equation}
It also gives $\|\tau(A)\|=\|\tau(A+J)\|\leq\|A+J\|$. Take $A$, $B\in\cA$ and a $\varepsilon$ for which \eqref{eq:tauAeps} holds for both $A$ and $B$. Then
\begin{equation}
\begin{split}
\|\tau(A)\tau(B)\|&=\|\tau\big( (A+J_1)(A+J_2)\big)\|\\
                  &\leq\|(A+J_A)(B+J_2)\|\\
                  &\leq\|A+J_1\|\|B+J_2\|\\
                  &\leq(\|\tau(A)\|+\varepsilon)(\|\tau(B)\|+\varepsilon),
\end{split}
\end{equation}
which gives the result as $\varepsilon\to 0$.

\end{proof}

\begin{theorem} \label{tho_idautadjquo}
Let $\cI$ be an ideal in the $C^*$-algebra $\cA$. Then

\begin{enumerate}
\item the ideal $\cI$ is selfadjoint; in other words, if $A\in\cI$, then $A^*\in\cI$,  \label{enuni}
\item the quotient $\cA/\cI$ is a $C^*$-algebra for the norm \label{enunii}
\[
  \| \tau(A)\|:=\inf_{J\in\cI}\| A+J \| ,
\]
the multiplication
\[
  \tau(A)\tau(B):=\tau(AB),
\]
and the convolution
\[
  \tau(A)^*=\tau(A^*)
\]
where $\dpt{\tau}{\cA}{\cA/\cI}$ is the canonical projection.

\end{enumerate}
\end{theorem}

\begin{proof}
It is necessary to prove~\ref{enuni} before~\ref{enunii} because the point~\ref{enuni} proves the well definiteness of the involution. Let $\cI^*=\{ A^*\tq A\in\cI \}$ and $J\in\cI$. Then $J^*J\in\cI\cap\cI^*$ because an ideal is left and right ideal (it is easy to see that $\cI^*$ is an ideal).

We are going to prove that $\cI\cap\cI^*$ is a $C^*$-subalgebra of $\cA$. If $J\in \cI\cap\cI^*$,then $J^*\in \cI\cap\cI^*$ too. On the other hand, $\cI\cap\cI^*$ is closed linear subspaces of $\cA$ because $\cI$ and $\cI^*$ are such from definition of an ideal. Ideals are Banach space and it is clear that the conditions about norm and involutions are true in $\cI\cap\cI^*$.

So $\cI\cap\cI^*$ is a non unital $C^*$-algebra and posses an approximate unit $\{ \cun_{\lambda} \}$. Let $J\in\cI$; the following upper bound holds:
\begin{equation}
\begin{split}
\| J^*-J^*\cun_{\lambda} \|^2&=\| (J-\cun_{\lambda}J)(J^*-J^*\cun_{\lambda}) \|\\
                            &=\| (JJ^*-JJ^*\cun_{\lambda})-\cun_{\lambda}(JJ^*-JJ^*\cun_{\lambda}) \|\\
                            &\leq \| JJ^*-JJ^*\cun_{\lambda} \|+\| \cun_{\lambda}(JJ^*-JJ^*\cun_{\lambda}) \|\\
                            &\leq \| JJ^*-JJ^*\cun_{\lambda} \|+\| \cun_{\lambda} \|\| JJ^*-JJ^*\cun_{\lambda} \|.
\end{split}
\end{equation}
Since $JJ^*$ belongs to $\cI\cap\cI^*$ which is a $C^*$-algebra an\quext{Cette phase me semble avoir un problème.} which $\cun_{\lambda}$ is build, we recognize in the latter two terms something of the form $\| A\cun_{\lambda}-A \|$ whose limit is zero. Then $\lim_{\lambda\to\infty}\| J^*-J^*\cun_{\lambda} \|=0$.
 We have proved that $J^*\cun_{\lambda}\in\cI\cap\cI^*$ and in particular that $J^*\cun_{\lambda}\in\cI$ for all $\lambda$. Then $J^*$ is the limit of a sequence in $\cI$, but the latter is closed, then $J^*\in\cI$.

Now we prove~\ref{enunii}.
From proposition~\ref{prop:ideal_Banach}, the quotient space $\cA/\cI$ is a Banach algebra in the chosen product and  norm. We just have to prove that $\tau(A)^*=\tau(A^*)$ gives a well behaved involution on $\cA/\cI$, i.e. we have to show that $\| A^*A \|=\| A \|^2$ for $A\in\cA/\cI$, or $\| \tau(A)^*\tau(A) \|=\| \tau(A) \|^2$ for $A\in\cA$. Using lemma~\ref{lem:taulimA}, we can compute
\begin{equation}
\begin{aligned}
\| \tau(A)^2 \|&=\lim_{\lambda\to\infty}\| A-A\cun_{\lambda} \|^2&\textrm{The lemma}\\
               &=\lim_{\lambda\to\infty}\| (A-A\cun_{\lambda})^*(A-A\cun_{\lambda})&\textrm{from $\| A \|^2=\| A^*A \|$
in $\cA_{\cun}$} \\
               &=\lim_{\lambda\to\infty}\| (\cun-\cun_{\lambda})A^*A(\cun-\cun_{\lambda}) \|\\
               &\leq\lim_{\lambda\to\infty}\| \cun-\cun_{\lambda} \|\| A^*A(\cun-\cun_{\lambda}) \|\\
               &\leq\lim{\lambda\to\infty}\| A^*A(\cun-\cun_{\lambda}) \|&\textrm{because $\| \cun-\cun_{\lambda} \|\leq 1$}\\
               &=\lim_{\lambda\to\infty}\| A^*A-A^*\cun_{\lambda} \|\\
               &=\| \tau(A^*A) \|&\textrm{definition of the norm}\\
               &=\| \tau(A)^*\tau(A) \|.
\end{aligned}
\end{equation}
Now lemma~\ref{lem:STARAlC} makes the Banach $\cA/\cI$ a $C^*$-algebra.

\end{proof}

\begin{corollary}
    Morphism of $C^*$-algebras have following properties related to ideals:
    \begin{enumerate}
        \item The kernel of a morphism between two $C^*$-algebras is an ideal.  \label{enupi}
        \item Any ideal in a $C^*$-algebra is the kernel of a morphism.\label{enupii}
        \item Then any morphism has norm $1$.\label{enupiii}
    \end{enumerate}
\end{corollary}

\begin{proof}
   ~\ref{enupi} If $A\in\ker\varphi$, then $AB\in\ker\varphi$ because $\varphi(AB)=\varphi(A)\varphi(B)=0$. From proposition~\ref{PropMDfqcUs}, a morphism is continuous. The kernel of a continuous map is always continuous by the ``intermediate value'' property.

   ~\ref{enupii} Let $\cI$ be an ideal in the $C^*$-algebra $\cA$ and $\dpt{\tau}{\cA}{\cA/\cI}$, the canonical projection. We know that $\cA/\cI$ is a $C^*$-algebra and that $\tau$ is a morphism. Let us show that $\cI=\ker\tau$.

    On the one hand, if $J\in\cI$, then $\tau(J)=\tau(0)$ which is the zero of $\cA/\cI$. Let, on the other hand, $J\in\ker\tau$. Then $\tau(J)=0=\tau(0)$. The fact that $\tau(J)=\tau(0)$ shows that the difference between $0$ and $J$ is an element of $\cI$. In other words: $J\in\cI$.

    This and the fact that $\| \tau \|=1$ give~\ref{enupiii}.\quext{Je ne vois cependant pas pourquoi.}
\end{proof}


\begin{proposition}
Let $\cA$ be a $C^*$-algebra  and $M$ a two-sided ideal in $\cA$ everywhere dense in $\cA$. There exists an increasing filtering approximate unit of $\cA$ contained in $M$.

If $\cA$ is separable, we can ask the approximate unit to be indexed by $\eN$.

\end{proposition}
\quext{Ici, y'a un point pas clair dans Landsman 2.7.2. De toutes facons, c'est une d\'emonstration \`a refaire}
\begin{proof}
Let $\cA_{\cun}$ be the algebra obtained by adding an unit to $\cA$, and $\Lambda$ the set of finite parts of $M$ ordered by inclusion. For $\lambda=\{ A_1,\ldots, A_n \}$, we pose
\[
  v_{\lambda}=A_1A_1^*+\cdots+A_nA_n^*\in M
\]
 and
\[
  u_{\lambda}=v_{\lambda}\big( \frac{1}{ n }+v_{\lambda} \big)^{-1}
\]
with $A_i\in M$ and $\lambda\in\Lambda$. The fact that $A^*\in M$ when $A\in M$ comes from point~\ref{enuni} of theorem~\ref{tho_idautadjquo}.

\begin{probleme}
Cette démonstration n'est pas du tout finie.
\end{probleme}

\end{proof}

The same kind of proof gives the following

\begin{proposition}
Let $\cA$ be a $C^{*}$-algebra and $\mI$ a right ideal in $\cA$. There exists, in $\mI\cap\cA^+$, a family $(u_{\lambda})$ with $\lambda$ in a filtered set such that
\begin{enumerate}
\item $\|u_{\lambda}\|\leq 1$,
\item $\lambda\leq\mu$ implies $u_{\lambda}\leq u_{\mu}$,
\item $\forall A\in\overline{\mI}$, $\|u_{\lambda} A-A\|\to 0$.
\end{enumerate}
\end{proposition}

The definition of ``$A\leq B$''\ for $A,B$ in a $C^{*}$-algebra comes from the positivity notion.


%+++++++++++++++++++++++++++++++++++++++++++++++++++++++++++++++++++++++++++++++++++++++++++++++++++++++++++++++++++++++++++
\section{Ideal in Banach algebras}

\begin{definition}
An \defe{ideal}{ideal in Banach algebra} in a Banach algebra $\cA$ is a closed vector subspace $\cI\subseteq\cA$ such that $\forall I\in\cI$, $\forall A\in\cA$, $AI\in\cI$ and $IA\in\cI$.

A \defe{left ideal}{left!ideal} has just $AI\in\cI$.

A  \defe{maximum ideal}{maximum ideal} is an ideal $\cI\neq\cA$ for which there exists no strict intermediary ideal between $\cI$ and $\cA$, i.e. no ideal $\tilde{\cI}\neq\cI$  with $\tilde{\cI}\neq\cA$ and $\cI\subset\tilde{\cI}$.
\end{definition}

An ideal is a Banach algebra, and the only ideal in $\cA$ which contains an invertible element is $\cA$ itself.

If $\cA$ is unital, then $\cA/\cI$ is unital and his unit is given by $\tau(\cun)$. Let us prove that $\|\tau(\cun)\|=1$. From equation \eqref{eq:tauAA}, $\|\tau(\cun)\|\leq\|\cun\|=1$. On the other hand, from equation $\|\tau(A)\tau(B)\|\leq\|\tau(A)\|\|\tau(B)\|$ with $B=\cun$, we find $1\leq\|\tau(\cun)\|$.

\begin{corollary}
    Let $\cA$ and $\cB$ be $C^*$-algebras, $\varphi\colon \cA\to \cB$ a morphism and $\cI$ the kernel of $\varphi$. We consider the canonical decomposition of $\varphi$ into
    \begin{equation}
    \xymatrix{%
       \cA \ar[r]^{\tau}    & \cA/\cI\ar[r]^{\psi}  & \varphi(\cA)\ar[r]    & \cB.
    }
    \end{equation}
    Then $\cI$ is closed in $\cA$, $\varphi(\cA)$ is closed in $\cB$ and $\psi$ is an isometric isomorphism.
\end{corollary}

\begin{proof}
    A morphism of $C^*$-algebras is always continuous,%
    %TODO: give a justification.
    then $\cI=\ker\varphi=\varphi^{-1}(\{ 0 \})$ is closed because $\{ 0 \}$ is closed. The map $\psi\colon \cA/\cI\to \varphi(\cA)$ is injective because when $\psi([A])=\psi([B])$, we have $\varphi(A)=\varphi(B)$ and $A=B+I$ for a certain $I\in\cI$, hence $[A]=[B]$. Proposition~\ref{prop:vp_geq} gives $\| \varphi(A) \|\geq\| A \|$ while proposition~\ref{PropMDfqcUs} gives $\| \varphi(A) \|\leq \| A \|$. So $\varphi$ is isometric.

    Hence $\varphi(\cA)$ is complete and therefore closed in $\cB$.
\end{proof}


\section{States}
%+++++++++++++++

\begin{theorem}[Riesz representation theorem]
Let $X$ be a locally compact Hausdorff space and $\dpt{\Lambda}{C_0(X)}{\eC}$ be a positive linear functional. Then there exists a $\sigma$-algebra $\mM$ in $X$, which contains all Borel sets, and an unique measure $\mu$ on $\mM$ such that

\begin{enumerate}
\item $\Lambda f=\int_Xfd\mu$,
\item $\mu(K)<\infty$ for all compact $K\subset X$,
\item If $E\in\mM$, then
\[
  \mu(E)=\inf\{ \mu(V)\tq E\subset V,\; V\textrm{ open} \}
\]
\item If $E$ is open or if $E\in\mM$ with $\mu(E)<\infty$, then
\[
  \mu(E)=\sup\{ \mu(K)\tq K\subset E,\; K\textrm{ compact} \},
\]
\item The space $(X,\mM,\mu)$ is a complete measure space. That is, if $E\in\mM$, $A\subset E$ and $\mu(E)=0$, then $A\in\mM$.

\end{enumerate}
\end{theorem}

%+++++++++++++++++++++++++++++++++++++++++++++++++++++++++++++++++++++++++++++++++++++++++++++++++++++++++++++++++++++++++++
\section{States on unital \texorpdfstring{$C^*$}{C*}-algebras}
%+++++++++++++++++++++++++++++++++++++++++++++++++++++++++++++++++++++++++++++++++++++++++++++++++++++++++++++++++++++++++++

\begin{definition}      \label{DefStateUnital}
A \defe{state}{state!on unital $C^*$-algebra} on an \emph{unital} $C^*$-algebra $\cA$ is a linear map $\dpt{\omega}{\cA}{\eC}$ which is

\begin{itemize}
\item positive: $\omega(A)\geq 0$ for every $A\in\cA^+$,
\item normalised: $\omega(\cun)=1$.
\end{itemize}
We say that the state $\omega$ is \defe{faithful}{faithful!state} if $\omega(A^*A)=0$ implies $A=0$. We denote by $\etS(\cA)$ the space of all states on $\cA$.
\end{definition}
As an example, consider $\cA\subset\oB(\hH)$, a $C^*$-subalgebra of the bounded operators on the Hilbert space $\hH$. Then any $\psi\in\hH$ such that $\| \psi \|=1$ defines a state on $\cA$ by $\psi(A)=\scal{\psi}{A\psi}$. Indeed if $A\in\cA^+$, there exists a $B$ such that $A=B^*B$ and from definition of the involution on $\oB(\hH)$, we have $\psi(B^*B)=\| B\psi \|^2\geq 0$.

When a state is not faithful, the set on which $\omega(a^*a)=0$ is a vector space, and one usually takes the quotient of $\cA$ by that subspace.

As an example, let us prove the following.
\begin{proposition}                 \label{Propstaretattraces}
Every state in the $*$-algebra $\eM_n(\eC)$ of $n\times n$ matrices with complex coefficients read
\begin{equation}
\varphi(a)=\tr(as)
\end{equation}
for a certain positive matrix $s\in\eM_{n}(\eC)$ with $\tr(s)=1$. Moreover, every functional of that form is a state.
\end{proposition}

\begin{proof}
    Let us first prove the second claim. Let $s$ be a positive matrix with trace equal to $1$, we have
    \[
      \varphi(a^*a)=\tr(a^*as)=\tr(a^*as^{1/2}s^{1/2})=\tr(s^{1/2}a^*as^{1/2})
    \]
    which is positive. Here we used the fact that every positive element reads $a^*a$ and positivity of $s$ to define $s^{1/2}$.

    By linearity, a state on $\eM_n(\eC)$ must read $\varphi(a)=\sum_i\sum_j\varphi_{ij}a_{ij}$ for some coefficients $\varphi_{ij}$, so that $\varphi(a)=\tr(sa)$ where $s$ is the matrix of $(\varphi_{ij})$. The condition $\varphi(\mtu)=1$ immediately imposes $\tr(s)=1$. Now we study $\varphi(a^*a)=\tr(sa^*a)$. In the basis which diagonalises $a^*a$, we have $(a^*a)_{ij}=\delta_{ij}n_i$ (no sum) with $n_i>0$.  Thus
    \[
      \varphi(a)=\sum_k(sa^*a)_{kk}=\sum_ks_{kk}n_k
    \]
    which must be positive for every choice of positive $n_k$. That imposes for all $s_{kk}$ to be positive, so that $s$ is a positive matrix.
\end{proof}

Notice that $s$ being positive, it is diagonalisable, so that it is nothing else than a list of positive numbers.


\begin{theorem}
The space of states of $\cA=C(X)$ is the set of probability measures on $X$. We suppose that $X$ is compact and Hausdorff.
\end{theorem}

\begin{proof}
If $X$ is compact and Hausdorff, we know from a long time that $C(X)$ is an unital $C^*$-algebra with norm
\[
  \| f \|_{\infty}:=\sup_{x\in X}| f(x) |.
\]
We can apply the Riesz representation theorem: we have on $X$ a $\sigma$-algebra $\mM$ which contains all Borel sets and an (unique) measure $\mu$ on $\mM$ such that, among other properties,
\begin{equation}
  \omega(f)=\int_Xfd\mu.
\end{equation}
Since $X$ is compact, it fulfills $\mu(X)<\infty$. We have $\omega(\cun)=\int_X\cun d\mu=\mu(X)=1$. Then $\mu(X)=1$ and it is a probability measure.

On the other hand, it is clear that a probability is a state.

\end{proof}

Let $\omega\in\etS$ be positive. Then the definition\label{PgStateInn}
\begin{equation} \label{eq:defprodetat}
(A,B)_{\omega}:=\omega(A^*B)
\end{equation}
gives a \defe{pre-inner product}{pre-inner product}\label{pgdef_preinned}, i.e. an inner product without the condition $v^2=0\Rightarrow v=0$. Indeed, $(A,A)_{\omega}=\omega(A^*A)$, but $A^*A\in\cA^+$, then $\omega(A^*A)\geq 0$ because $\omega$ is a state. In the case of a faithful state, we get an inner product.

From Cauchy-Schwarz inequality,
\begin{equation} \label{eq:omABleq}
  | \omega(A^*B) |^2\leq \omega(A^*A)\omega(B^*B).
\end{equation}
Moreover, $\omega(A^*)=(A,\cun)_{\omega}$, then
\begin{equation} \label{eq:omABleqs}
  \omega(A^*)=\overline{\omega(A)}.
\end{equation}

\begin{proposition}
Let $\dpt{\omega}{\cA}{\eC}$, a linear map on an unital $C^*$-algebra $\cA$. It is positive if and only if $\omega$ is bounded and $\| \omega \|=\omega(\cun)$. In particular

\begin{enumerate}
\item A state in an unital $C^*$-algebra is bounded and has norm $1$, \label{71125ai}
\item an element $\omega\in\cA^*$ such that $\| \omega \|=\omega(\cun)=1$ is a state on $\cA$. \label{7125aii}
\end{enumerate}
\label{prop:linposboun}
\end{proposition}


\begin{proof}
\subdem{Direct sense}
Let $\omega$ be positive and begin by $A=A^*$. Then inequality $-\| A \|\cun\leq A\leq\| A \|\cun$ gives $\omega(A)\leq\| A \|\omega(1)$. Indeed
\begin{equation}
\begin{split}
  A\leq\| A \|\cun&\Rightarrow A-\| A \|\cun\leq 0\\
                  &\Rightarrow\omega(\| A \|\cun-A)\geq0\\
                  &\Rightarrow \| A \|\omega(\cun)\geq\omega(A).
\end{split}
\end{equation}
The same can be done with $-\| A \|\cun\leq A$ as starting point, then for $A=A^*$, we have $\| \omega(A) \|\leq \omega(\cun)\| A \|$.

Let us now consider any $A$. We know that $| \omega(A^*B) |^2\leq\omega(A^*A)\omega(B^*B)$. Writing it for $A=\cun $ and using $\| A^*A \|=\| A \|^2$, we find that $| \omega(B) |^2\leq\omega(\cun)^2\| B \|^2$,
and from the definition of the functional norm, it leads to
\begin{equation} \label{eq:irquatre}
\| \omega \|\leq\omega(\cun).
\end{equation}
A $C^*$-algebra is a normed vector space, then for all $A\in\cA$, there exists a $R\in\cA$ such that $A=\| A \|R$ and $\| R \|=1$. For this $R$, we have
\[
  \frac{| \omega(A) |}{\| A \|}=| \omega(R) |,
\]
and the definition of $\| \omega \|$ can be rewritten as
\[
  \| \omega \|=\sup\{ \frac{| \omega(A) |}{\| A \|}\tq A\in\cA \}.
\]
Now equation \eqref{eq:irquatre} gives $\| \omega \|\leq\omega(A)$. From definition of $\| \omega \|$, we also know that $\| \omega \|\geq \omega(\cun)$. Finally, $\| \omega \|=\omega(\cun)$.

\subdem{Inverse sense}

We know that $\omega$ is bounded and that $\| \omega \|=\omega(\cun)$ and we want to prove that $\omega$ is positive. Let $A\in\cA_{\eR}$ and let us decompose $\omega(A)=\alpha+i\beta$ where $\alpha,\beta\in\eR$. We can adapt the reasoning that around equation \eqref{eq:rcinq} to prove that $\beta=0$. Namely,
\begin{equation}
  | \omega(B+it\cun) |^1\leq \omega(\cun)^2\| B+it\cun \|
                        \leq \omega(\cun)^2(\| B \|^2+t^2),
\end{equation}
and for the self-adjoint element $B:=\omega(\cun)A-\alpha\cun$,
\[
  | \omega(B+it\cun) |^2=\beta^2+2\beta t\omega(\cun)+t^2\omega(\cun)^2.
\]
Combining the two equations, and taking into account the fact that $\omega(\cun)=\| \omega \|>0$,
\[
  \beta^2+\beta t\omega(\cun)\leq\omega(\cun)^2\| B \|^2.
\]
Since the right hand side don't depend on $t$, we conclude that $\beta=0$. This shows that $\omega$ is real on $\cA_{\eR}$. Now we are going to prove that $\omega(A)\geq 0$ when $A\geq 0$. Let $s>0$ be so small that $\| \cun-sA \|\leq 1$. Then
\[
  1\geq\| \cun-sA \|=\underbrace{\frac{\| \omega \|}{\omega(\cun)}}_{=1}\| \cun-sA \|\geq \frac{| \omega(\cun-sA) |}{\omega(\cun)}.
\]
We conclude that $| \omega(\cun)-s\omega(A) |\leq\omega(\cun)$, which is only possible if $\omega(A)\geq 0$. It proves that $\omega$ is positive.

Let us now check that points~\ref{71125ai} and~\ref{7125aii} are effectively obtained. We proved that for a state $\| \omega \|=\omega(\cun)$ and in the definition, we impose $\omega(\cun)=1$.

\end{proof}

\subsection{States on non unital \texorpdfstring{$C^*$}{C*}-algebras}
%----------------------------------------------

We now relax the unital hypothesis.

\begin{definition}  \label{DefApplPositive}
    A linear map $\dpt{q}{\cA}{\cB}$ between two $C^*$-algebra is \defe{positive}{positive!map between $C^*$-algebra} when $q(A)\geq 0$ in $\cB$ whenever $A\geq 0$ in $\cA$.
\end{definition}

\begin{proposition}
A positive map is bounded.\label{prop:posborn}
\end{proposition}

Since for linear map, to be bounded is equivalent to continuous, all positive map is continuous.

\begin{proof}
We first prove that a bounded map on $\cA^+$ is bounded on $\cA$. Let us decompose $A\in\cA$ into $A=A'+iA''$ with $A',A''\in\cA_{\eR}$ and use lemma~\ref{lem:AsAdecm} to write $A=A'_+-A'_-+iA''_+-iA''_-$.We have $\| A' \|\leq \| A \|$ and $\| A'' \|\leq \| A \|$. If $B$ is one of $A'_{\mp}$ or $A''_{\pm}$, lemma also says that $\| B \|\leq \| A \|$. Now let us suppose that $\| q(B) \| \leq C\| N \|$ for all $B\in\cA^+$ and a certain $C>0$. Then $\| q(A) \|\leq 4 C\| A \|$ and $q$ is then bounded.

We are now going to prove that a positive map $q$ is bounded. Suppose that $q$ is unbounded. In particular, it is not bounded in $\cA^+$ (if it were, it should be bounded everywhere) and there exists a sequence $(A_n)\in\cA^+_1$ such that $\| q(A_n) \|\geq n^3$ for all $n$. Here, the symbol $\cA_A^+$ denote the elements $A\in\cA^+$ for which $\| A \|\leq 1$.

Let us consider the series $\sum_{n=0}^{\infty}n^{-2}A_n$. Since $\| A_n \|\leq 1$, this converges to an element $A\in\cA^+$. If $q$ is positive, then $q(A)\geq n^{-2}q(A_n)$ and from property $-B\leq A\leq B\Rightarrow\| A \|\leq\| B \|$, we see that
\[
 \| q(A) \|\geq n^{-2}\| q(A_n) \|\geq n
\]
for all $n\in\eN$. Indeed, $q$ positive implies $q(A_n)\geq 0$ because $A_n\in\cA^+$. Now,
\begin{equation}
\begin{split}
&-q(A)\leq n^{-2}q(A_n)\leq q(A)\\
&\Rightarrow \| n^{-2}q(A_n) \|\leq\| q(A) \|\\
&\Rightarrow \| q(A) \|\geq n^{-2}\| q(A_n) \|\geq n^3.
\end{split}
\end{equation}
This proves that $\| q(A) \|$ is greater than $n^3$ for all $n$; this is impossible (because this is a finite number). Then $q$ is bounded on $\cA^+$ and on the whole $\cA$.

\end{proof}

If we consider $\cB=\eC$, states are particular cases of positive maps and then states on unital $C^*$-algebra are bounded and have norm $1$.

In order to define a state on a non-unital $C^*$-algebra, we can't take the normalization $\omega(\cun)=1$. But we just prove a condition to get an unit norm in the unital case. So we define


\begin{proposition}
Let $\cA$ be an involutive Banach algebra with unit $\cun$ such that $\| \cun \|=1$. If $f$ is a positive linear form, then it is continuous and $\| f \|=f(\cun)$. \label{prop_Dix214}
\end{proposition}
\begin{proof}
This statement is nothing else than propositions~\ref{prop:linposboun} and~\ref{prop:posborn}.
\end{proof}


\begin{proposition}
Let $\cA$ be an involutive Banach algebra with approximate unit and $\cA_{\cun}$ the involutive algebra deduced from $\cA$ by adding an unit. We consider $f$, a linear positive and continuous form on $\cA$. Then
\begin{enumerate}
\item \label{itemi_prop_invaddunit} $\forall A\in \cA$, $f(A^*)=\overline{ f(A) }$ and
\[
  | f(A) |^2\leq \| f \|f(A^*A),
\]
\item \label{itemii_prop_invaddunit} $| f(B^*AB) |\leq\| A \|f(B^*B)$,
\item \label{itemiii_prop_invaddunit} $\| f \|=\sup_{\substack{A\in\cA\\\| A \|\leq1} } f(A^*A)$,
\item \label{itemiv_prop_invaddunit} Let $(A_i)_{i\in I}$ be elements of $\cA$ indexed by the filtering set $I$ such that $\| A_i \|\leq 1$ and $f(A_i)\to\| f \|$. Then $f(A^*_iA_i)\to\| f \|$,
\item \label{itemv_prop_invaddunit} If $(u_j)_{j\in J}$ is an approximate unit of $\cA$, then $f(u_j)\to \| f \|$ and $f(u_j^*u_j)\to\| f \|$,
\item \label{itemvi_prop_invaddunit} The function $f$ can be extended to an unique positive form $\tilde f$ on $\cA_{\cun}$ in such a way that $\tilde f(\cun)=\| f \|$. If a positive form on $\cA_{\cun}$ extends $f$, it is everywhere bigger than $f$.
\item \label{itemvii_prop_invaddunit} With the same $(A_i)_{i\in I}$ that in~\ref{itemiv_prop_invaddunit}, we have $A_i\to\cun$ for the prehilbert structure defined by $\tilde f$ on $\cA_{\cun}$. Then $\cA$ is everywhere dense in $\cA_{\cun}$ for this structure.
\end{enumerate}

\label{prop_invaddunit}
\end{proposition}

\begin{proof}
The definition $(A,B):=f(A^*B)$ gives a pre-inner product (see page \pageref{pgdef_preinned}) because
\[
(\lambda A,B)=f(\overline{ \lambda }A^*B)=\overline{ \lambda }(A,B).
\]
 So we want to write
\[
  f(A^*)=f(A^*\cun)=(A,\cun)=\overline{ (\cun,A) }=\overline{ f(A) },
\]
but we do not have unit. We therefore have to use continuity of $f$ instead:
\[
f(A^*)=\lim(A^*u_j)
    =\lim\overline{ f(u_j^*A) }
    =\lim\overline{ f\big( (A^*u_j)^* \big) }
    =\overline{ f(A^{**}) }
    =\overline{ f(A) }.
\]
The second equality is not a particular case of $f(A^*)=\overline{ f(A) }$, but the fact that $f(B^*A)=\overline{ f(A^*B) }$ because $f$ defines a pre-inner product\quext{C'est un problème pcq sans ça, je ne vois pas comment démontrer que ce $f$ donne bien un pré produit scalaire.}.

Since we have a pre-inner product, we have Cauchy-Schwarz:
\[
  \lim| f(Au_j) |^2\leq\lim f(A^*A)f(u^*_ju_j),
\]
hence
\begin{equation}
| f(A) |^2=\lim| f(Au_j) |^2\leq f(A^*A)\lim f(u^*_ju_j).
\end{equation}
An involutive Banach algebra fulfils $\| AB \|\leq\| A \|\,\| B \|$ for all $A$, $B\in\cA$, so $\| u^*_ju_j \|\leq 1$ because definition of an approximate unit gives $\| u_j \|\leq 1$. We have
\[
  \| f \|=\sup_{\| A \|=1}| f(A) |
\]
Let us consider $\lambda\in\eR$ such that $\| \lambda u_j^*u_j \|=1$; we necessarily have $\lambda\geq 1$. We have
\[
  \| f \|\geq f(\lambda u^*_ju_j)=\lambda f(u^*_ju_j)\geq f(u^*_ju_j)
\]
because $f$ is positive and $u^*_ju_j\in\cA_{\eR}$. We conclude that
\[
  f(A^*A)\lim f(u^*_ju_j)\leq \| f \|f(A^*A)
\]
and this finally gives the first point:
\begin{equation}
  | f(A) |^2\leq \| f \|f(A^*A).
\end{equation}

\ref{itemvii_prop_invaddunit}$\Rightarrow$\ref{itemvii_prop_invaddunit}. Indeed,~\ref{itemi_prop_invaddunit} implies $\| f \|^2\leq\| f \|\lim f(A_i^*A_i)$ which in turn gives $\lim f(A^*_iA_i)\geq\| f \|$. But $\| A_i \|\leq 1$, so $\| f \|\geq\lim f(A_i^*A_i)$ and we conclude that $\| f \|=\lim f(A_i^*A_i)$.

\ref{itemiv_prop_invaddunit}$\Rightarrow$\ref{itemiii_prop_invaddunit}. On the one hand, for all $\| A \|\leq 1$, we have $f(A^*A)\leq\| f \|$; on the other hand we have a sequence $A_i$ such that $f(A_i^*A_i)\to\| f \|$. So $\| f \|$ is an upper bound for the set of $f(A^*A)$ and in the same time, it is in the adherence of this set. So $\| f \|$ is the supremum of this set.

Proof of~\ref{itemvi_prop_invaddunit}. Unicity is clear because the prescription $\tilde f(\cun)=\| f \|$ gives $\tilde f$ on a basis of $\cA_{\cun}$. For existence, we pick $(\lambda,A)=\lambda+A\in\cA_{\cun}$ with $\lambda\in\eC$ and $A\in\cA$ and we pose $\tilde f(\lambda+A)=\lambda\| f \|+f(A)$. It is linear, extends $f$ and $\tilde f(\cun)=\| f \|$. To get positivity,
\begin{equation}
\begin{split}
 \tilde f\big( (\lambda+A)^*(\lambda+A) \big)&=f(A^*A-\overline{ \lambda }A+\lambda A^*)+| \lambda |^2\| f \|\\
        &=f(A^*A)+2\real\overline{ \lambda }f(A)+| \lambda |^2\| f \|\\
        &\geq f(A^*A)-2| \lambda |\| f \|^{\frac{ 1 }{2}}f(A^*A)^{\frac{ 1 }{2}}+| \lambda |^2\| f \|\\
        &=\big[ f(A^*A)^{\frac{ 1 }{2}}-| \lambda |\,\| f \|^{\frac{ 1 }{2}} \big]^{\frac{ 1 }{2}}\\&\geq 0.
\end{split}
\end{equation}
Let $g$ be a positive form on $\cA_{\cun}$ which extends $f$. The involutive Banach algebra of $\cA$ is extended to $\cA_{\cun}$ by setting $\cun=1$. Proposition~\ref{prop_Dix214} and the fact that $g$ extends $f$ make $\| f \|\leq\| g \|=g(\cun)$. Hence
\[
  g\big( (\lambda+A)^*(\lambda+A) \big)\geq f\big( (\lambda+A)^*(\lambda+A) \big),
\]
and finally $g\geq\tilde f$.

Now we take $B\in\cA$ and we look at the form $g(A)=\tilde f(B^*AB)$. It is a positive form because
\[
  g(A^*A)=\tilde f(B^*A^*AB)=\tilde f\big( (AB)^*(AB) \big)\geq 0.
\]
For the norm of $g$, we have $\| g \|=g(\cun)=\tilde f(B^*B)$. Since $\| A^*A \|\cun\geq A^*A$, we find $\| A^*A \|f(\cun)\geq f(A^*A)$ and therefore
\[
  \| f \|\,\| A^*A \|\geq f(A^*A).
\]
We find
\[
  | f(B^*AB) |^2\leq f(B^*B)\| g \|\,\| A^*A \|,
\]
but $\| g \|=f(B^*B)$ and $A^*A\leq\| A \|^2$, so
\begin{equation}
| f(B^*AB) |\leq f(B^*B)\| A \|.
\end{equation}

Verification of~\ref{itemvii_prop_invaddunit}. The pre-Hilbert structure defined by $\tilde f$ is the convergence notion $A_i\to A$ when $f(A_i)\to f(A)$. The first part of statement~\ref{itemvii_prop_invaddunit} is thus just the fact that
\[
  \tilde f\big( (A_i-\cun)^*(A_i-\cun) \big)\to 0.
\]
\begin{probleme}
Pas fait le reste de la preuve. C'est dans \cite{Dixmier}.
\end{probleme}

\end{proof}


We consider an involutive Banach algebra $\cA$; $f$, a positive linear continuous form on $\cA$ and $\tilde f$ its extension to $\cA_{\cun}$ by proposition~\ref{prop_invaddunit}. We claim that
\[
  \{ A\in\cA\tq \tilde f(A^*A)=0 \}
\]
is a left ideal. Indeed let us show that $BA$ belongs to this set when $A$ does. We use~\ref{itemii_prop_invaddunit} of proposition~\ref{prop_invaddunit} to compute the norm on both side of the identity
\[
  \tilde f\big( (BA)^*(BA) \big)=\tilde f(A^*B^*BA).
\]
We find
\[
  | \tilde f\big( A^*(B^*B)A \big) |\leq \| B^*B \|f(A^*A)=0.
\]
Thus $\tilde f\big( A^*(B^*B)A \big)=0$.


\begin{definition}      \label{DefStateCSA}
A \defe{state}{state!on non-unital $C^*$-algebra} on a $C^*$-algebra is a linear map $\dpt{\omega}{\cA}{\eC}$ which is positive and has norm $1$.\label{def:etatnon}
\end{definition}
Proposition~\ref{prop:posborn} shows that a state is bounded; then one can impose the condition $\| \omega \|=1$. From proposition~\ref{prop:linposboun}, a state in the sense of definition~\ref{def:etatnon} on an unital  $C^*$-algebra is a state in the sense of definition ~\ref{DefStateUnital} because $1=\| \omega \|=\omega(\cun)$.

\begin{proposition}[\cite{Landsman}]
A state on a non unital $C^*$-algebra has an unique extension on the unitized algebra. \label{prop_st_unit_ext}
\end{proposition}

\begin{proof}
Let $\omega$ be a state on the $C^*$-algebra $\cA$ and let us define the extension
\[
  \omega_{\cun}(A+\lambda\cun):=\omega(A)+\lambda
\]
 on $\cA_{cun}$. It is clear that $\omega_{cun}=1$. We have now to show that $\omega_{cun}(A)\geq 0$ whenever $A\in\cA^+$. On $\cA$, the form $\omega$ is a state and the is bounded; let $(\cun_{\lambda})$ be an approximate unit in $\cA$. Since $\lim_{\lambda\to\infty}\| A-\cun_{\lambda}A \|=0$, we have $| \omega(A-A\cun_{\lambda}) |\to 0$. Let us use \eqref{eq:omABleq} and \eqref{eq:omABleqs} in the particular case $A=B$:
\[
  | \overline{\omega(A)}\omega(A) |^2\leq\omega(A^*A)^2,
\]
but since $\omega$ is positive, the right hand side is a positive number and we can remove the square in both sides; using the fact that for any complex number, $| z\overline{z} |=| z |^2$, we find
\[
  | \omega(A)^2 |\leq \omega(A^*A).
\]
Then, with $\omega_{\cun}$, we find
\[
  \omega_{\cun}\big( (A+\lambda\cun)^*(A+\lambda\cun) \big)\geq | \omega(A+\lambda\cun) |^2=| \omega(A)+\lambda |^2,
\]
but $(A+\lambda\cun)^*(A+\lambda\cun)$ is a general element of $\cA_{\cun}$. It proves that $\omega$ is positive.

\end{proof}


The following result gives a lot of examples of states.

\begin{lemma}
 For all $A\in \cA$ and $a\in\sigma(A)$, there exists a state $\omega_a$ on $\cA$ such that $\omega_a(A)=a$. When $A=A^*$, we can find a state $\omega$ such that $| \omega(A) |=\| A \|$. \label{lem:omAenomA}
\end{lemma}

\begin{proof}
If $\cA$ has no unit, we add one. We define $\dpt{\tilde\omega}{\eC A+\eC\cun}{\eC}$ by $\tilde\omega(\lambda A+\mu\cun):=\lambda a+\mu$. Since $a\in\sigma(A)$, we know that $\lambda a+\mu\in\sigma(\lambda A+\mu\cun)$ and then that $\lambda A+\mu\cun-(\lambda  a+\mu)\cun$ has no inverse.

When $\cA$ has an unit, we know that $\omega\in\Delta(\cA)\Rightarrow | \omega(A) |\leq \| A \|$. We want to get the same for our $\tilde\omega$. Looking at the proof of \eqref{eq:omAleqnA}, we see that we have to prove that $\tilde\omega(x)\neq 0$ whenever $x$ is invertible. Be careful on a point: the question is posed in the $C^*$-algebra $\eC A+\eC\cun$, not in $\cA$ or $\cA_{\cun}$. So we take an invertible element $\lambda A+\mu\cun$ . Then $-\mu/\lambda\neq a$ because it is not in $\sigma(A)$. So $\tilde\omega_a(\lambda A+\mu\cun)=\lambda a+\mu\neq 0$ and we can affirm that
\begin{equation}
| \tilde\omega_a(\lambda A+\mu\cun) |\leq \| (\lambda A+\mu\cun) \|.
\end{equation}
Since $\tilde\omega_a(\cun)=1$, we know that $\| \tilde\omega_a \|\geq 1$, but the equation above shows that $\| \tilde\omega_a \|\leq 1$. We conclude that $\| \tilde\omega_a \|=1$.

Hahn-Banach theorem~\ref{tho:hahnBanach} gives an extension $\omega_a$ of $\tilde\omega_a$ to the whole $\cA$ with norm $1$.
Point~\ref{7125aii} of proposition~\ref{prop:linposboun} shows that $\omega_a$ is a state on $\cA_{\cun}$ with $\omega_a(A)=\tilde\omega_a(A)=a$.

For the second assertion, we know that $\sigma(A)$ is a closed set, then there exists a $a\in\sigma(A)$ such that $r(A)=| A |$. For this $a$, $| \omega(A) |=| a |=r(a)=\| A \|$ because $\| A \|=r(A)$ when $A=A^*$.

Now if $\cA$ has no unit, we consider as $\omega_a$, the restriction to $\cA$ of the $\omega_a$ that we build on $\cA_{\cun}$.

\end{proof}

%+++++++++++++++++++++++++++++++++++++++++++++++++++++++++++++++++++++++++++++++++++++++++++++++++++++++++++++++++++++++++++
\section{Uniqueness of the norm}
%+++++++++++++++++++++++++++++++++++++++++++++++++++++++++++++++++++++++++++++++++++++++++++++++++++++++++++++++++++++++++++

\begin{proposition}         \label{prop:unicitenormcsa}
The norm on a $C^*$-algebra is unique in the sense that if $\cA$ is a $C^*$-algebra, one cannot find an other norm on $\cA$ (as Banach algebra) for which $\cA$ is a $C^*$-algebra.
\end{proposition}

\begin{proof}
    Let us consider an element such that $A=A^*$. Point~\ref{ItemSpecffSpecThoSpectral} of theorem~\ref{ThoSpectralTho} applied to $f=\id_{\sigma(A)}$, and the fact that $\| f \|_{\infty}=r$ imply that $\| A \|=r(A)$.

    Considering this argument with a general $A^*A$ instead of a particular $A$, we find that
    \begin{equation}
    \| A \|=\sqrt{r(A^*A)}.
    \end{equation}
    Since the spectral radius is determined by the Banach algebra structure only, this formula shows that the norm on a general element $A$ is unique.

\end{proof}

\begin{remark}
The way to complete a $*$-algebra is not unique. Thus one can construct several $C^*$-algebra from a given $*$-algebra.
\end{remark}

\subsection{Convexity}
%---------------------

A \defe{convex}{convex!subset} subspace $C$ of a vector space $V$ is a subset of $V$ such that for all $v$, $w\in C$ and $\lambda\in[0,1]$, we have $\lambda v+(1-\lambda)w\in C$.

If $p_i>0$, $\sum_ip_i=1$ and $v_i\in C$, then $\sum_i v_i\in C$.

\begin{lemma}
If $\cA$ is an unital $C^*$-algebra, then the space $\etS(\cA)$ of all states on $\cA$ is convex.
\end{lemma}

\begin{proof}
If $\omega$ and $\eta$ are states, then they are positive and $\omega(\cun)=\eta(\cun)=1$. Then $\lambda\omega+(1-\lambda)\eta$ is also positive and $\lambda\omega(\cun)+(1-\lambda)\eta(1)=1$.
\end{proof}

\begin{proposition}
In the non unital case, $\etS$ is convex too.
\end{proposition}

\begin{proof}
Let $\omega$ and $\eta$ be two states on a non unital $C^*$-algebra $\cA$ and $\omega_{cun}$, $\eta_{cun}$ their extensions to $\cA_{\cun}$. From lemma, $\xi_{cun}=\lambda\omega_{cun}+(1-\lambda)\eta_{cun}$ is a state on $\cA_{\cun}$. We want the restriction $\xi=\lambda\omega+(1-\lambda)\eta$ to be a state on $\cA$.

Since $\xi_{cun}(A+\mu\cun)=\lambda\omega(A)+(1-\lambda)\eta(A)+\mu$, the restriction reads
\[
  \xi(A)=\lambda\omega(A)+(1-\lambda)\eta(A).
\]
Then the unique extension of $\xi$ with same norm is precisely $\xi_{cun}$. It proves that $\| \xi \|=1$.
\end{proof}

The dual $\cB^*$ of a Banach space $\cB$ is the space of functional $\dpt{\rho}{\cB}{\eC}$ and a functional is by definition linear and continuous. If $\cA$ is unital, proposition~\ref{prop:linposboun} ensures that $\etS(\sA)\subset\cA^*$. Let us recall the $w^*$-limit\index{$w^*$-limit} in $\cA^*$. We say that $\omega_n\to \omega$ when for all $v\in\cA$, $\omega_n(A)\to \omega(A)$. It is clear that the $w^*$-limit preserves the positivity: if $\omega_n\geq0$ and $\omega_n\to \omega$, then $\omega\geq 0$. This proves that $\etS(\cA)$ is closed in $\cA^*$ for the $w^*$-topology. From normalization $\| \omega \|=1$, we conclude  that $\etS(\cA)$ is a closed set in the unit ball of $\cA^*$. From Banach-Alaoglu theorem, $\etS(\cA)$ is compact. What is proved is

\begin{proposition}
The space of states of an unital $C^*$-algebra is convex and compact.
\end{proposition}

The simplest example is $\cA=\eC$. From $\omega(1)=1$ and linearity, we deduce $\omega(i)=i\omega(1)=i$. Then $\omega(a+bi)=a+bi$ is the only state on $\eC$. An isolated point is compact and convex. Let us now consider a less trivial example.

We now consider $\cA=\eC^2$. We see $\eC^2$ as $\eC\oplus\eC$ and we write $z\dot + z'$ the element $(z,z')\in\eC^2$. We define the product by
\[
  (\lambda_1\dot +\mu_1)(\lambda_2\dot +\mu_2)=\lambda_1\lambda_2\dot +\mu_1\mu_2.
\]
From characterization $\cA^+=\{ B^*B\tq B\in\cA \}$, a generic element in $\cA^+$ reads
\[
  (\bar\lambda\dot +\bar\mu)(\lambda\dot +\mu)=\bar\lambda\lambda\dot +\bar\mu\mu.
\]
So positive elements in $\eC^2$ are $(\lambda,\mu)$ with $\lambda,\mu\geq 0$. In order for $\omega$ to be positive, we need $\omega(\lambda,\mu)=c\lambda+d\nu\geq 0$ for all $\lambda,\mu\geq0$. For normalization $\omega(1,1)=1$, we also need $c+d=1$. Then $\etS(\eC^2)$ can be identified with $[0,1]$ which is convex and compact.
