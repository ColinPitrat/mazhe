% This is part of (almost) Everything I know in mathematics
% Copyright (c) 2010-2017
%   Laurent Claessens
% See the file fdl-1.3.txt for copying conditions.

\section{Differentiable manifolds}
%+++++++++++++++++++++++++++++++++

More precisions in \cite{kobayashi}, chapter II (\S 1 and 2) and III. (\S 1, 2, 3 and 7); \cite{madore,Helgason} can also be useful. An other source for this chapter is \cite{ms_book}. A systematic exposition of manifolds and such can be found in \cite{dgbook}.

\subsection{Definition and examples}
%-----------------------------------

A $n$-dimensional \defe{differentiable manifold}{differentiable!manifold}\index{manifold} is a set $M$ and a system of charts $\{(\mU_{\alpha},\varphi_{\alpha})\}_{\alpha\in I}$ where each set $\mU_{\alpha}$ is open in $\eR^n$ and the maps $\dpt{\varphi_{\alpha}}{\mU_{\alpha}}{M}$ are injective and satisfy the three following conditions:

\begin{itemize}
\item every $x\in M$ is contained in at least one set $\varphi_{\alpha}(\mU_{\alpha})$,
\item for any two charts $\dpt{\varphi_{\alpha}}{\mU_{\alpha}}{M}$ and $\dpt{\varphi_{\beta}}{\mU_{\beta}}{M}$, the set
\[
   \varphi_{\alpha}^{-1}( \varphi_{\alpha}(\mU_{\alpha})\cap\varphi_{\beta}(\mU_{\beta}) )
\]
is an open subset of $\mU_{\alpha}$, 
\item the map
\[
  \dpt{  (\varphi_{\beta}^{-1}\circ\varphi_{\alpha})  }{   \varphi_{\alpha}^{-1}( \varphi_{\alpha}(\mU_{\alpha})\cap\varphi_{\beta}(\mU_{\beta})  )   }{\mU_{\beta}}
\]
is differentiable\footnote{In the sequel, by ``differentiable'' we always mean smooth. If this map is differentiable, $C^k$, analytic,\ldots then the manifold is said to be differentiable, $C^k$, analytic,\ldots} as map from $\eR^n$ to $\eR^n$.

\end{itemize}

Each time we say ``manifold``, we mean ``differentiable manifold``. We will only consider manifolds with Hausdorff topology (see later for the definition of a topology on a manifold). Any open set of $\eR^n$ is a differentiable manifold if we choose the identity map as chart system. Most of surfaces $z=f(x,y)$ in $\eR^3$ are manifolds, depending on certain regularity conditions on~$f$. 

If $M_1$ and $M_2$ are two differentiable manifolds, a map $\dpt{f}{M_1}{M_2}$ is \defe{differentiable}{differentiable!map} if $f$ is continuous and for each two coordinate systems $\dpt{\varphi_1}{\mU_1}{M_1}$ and $\dpt{\varphi_2}{\mU_2}{M_2}$, the map $\varphi_2^{-1}\circ f\circ\varphi_1$ is differentiable on its domain. One can show that if $\dpt{f}{M_1}{M_2}$ and $\dpt{g}{M_2}{M_3}$ are differentiable, then $\dpt{g\circ f}{M_1}{M_3}$ is differentiable.

\subsubsection{Example: the sphere}\index{sphere}

The sphere $S^n$ is the set
\[
  S^n=\{  (x_1,\ldots, x_{n+1})\in\eR^{n+1}\tq \|x\|=1  \}
\]
for which we consider the following open set in $\eR^n$:
\[
   \mU=\{  (u_1,\ldots,u_n)\in\eR^n\tq\|u\|<1  \}
\]
and the charts $\dpt{\varphi_i}{\mU}{S}$, and $\dpt{\tilde{\varphi}_i}{\mU}{S}$
\begin{subequations}
\begin{align}
   \varphi_i(u_1,\ldots,u_n)&=(u_1,\ldots,u_{i-1}, \sqrt{  1-\|u\|^2  },u_i,\ldots,u_n )\\
   \tilde{\varphi}_i(u_1,\ldots,u_n)&=(u_1,\ldots,u_{i-1}, -\sqrt{  1-\|u\|^2  },u_i,\ldots,u_n ).
\end{align}
\end{subequations}
These map are clearly injective. To see that $\varphi(\mU)\cup\tilde{\varphi}(\mU)=S$, consider $(x_1,\ldots,x_{n+1})\in S$. Then at least one of the $x_i$ is non zero. Let us suppose $x_1\neq 0$, thus $x_1^2=1-(x_2^2+\cdots+x_{n+1}^2)$ and
\begin{equation}\label{eq:xupm}
   x_1=\pm\sqrt{1-(\ldots)}.
\end{equation}
If we put $u_i=x_{i+1}$, we have $x=\varphi(u)$ or $x=\tilde{\varphi}(u)$ following the sign in relation \eqref{eq:xupm}. The fact that $\varphi^{-1}\circ\tilde{\varphi}$ and $\tilde{\varphi}^{-1}\circ\varphi$ are differentiable is a ``first year in analysis exercise``.

\subsubsection{Example: projective space}

On $\eR^{n+1}\setminus\{o\}$, we consider the equivalence relation $v\sim\lambda w$ for all non zero $\lambda\in\eR$, and we put
\[
  \eR P^n=\left(\eR^{n+1}\setminus\{o\}\right)/\sim.
\]
This is the set of all the one dimensional subspaces of $\eR^{n+1}$. This is the real \defe{projective space}{projective!real space} of dimension $n$. We set $\mU=\eR^n$ and
\[
  \varphi_i(u_1,\ldots,u_n)=\Span\{ (u_1,\ldots,u_{i-1},1,u_i,\ldots,u_n) \}.
\]
One can see that this gives a manifold structure to $\eR P^n$. Moreover, the map
		\begin{equation}
		\begin{aligned}
			A \colon S^n &\to \eR P^n\
			v&\mapsto \Span v
		\end{aligned}
	\end{equation}	
is differentiable.

Let us show how to identify $\eR\cup\{ \infty \}$ to $\eR P^1$, the set of directions in the plane $\eR^2$. Indeed consider any vertical line $l$ (which does contain the origin). A non vertical vector subspace of $\eR^2$ intersects $l$ in one and only one point, while the vertical vector subspace is associated with the infinite point.


\subsection{Topology on manifold and submanifold}
%------------------------------------------------

A subset $V\subset M$ is \defe{open}{topology!on manifold} if for every chart $\dpt{\varphi}{\mU}{M}$, the set $\varphi^{-1}(V\cap\varphi(\mU))$ is open in $\mU$.

\begin{theorem}
This definition gives a topology on $M$ which has the following properties:

\begin{enumerate}
\item the charts maps are continuous,
\item the sets $\varphi_{\alpha}(\mU_{\alpha})$ are open.
\end{enumerate}

\end{theorem}

\begin{proof}
First we prove that the open system defines a topology. For this, remark that $\varphi_{\alpha}^{-1}$ is injective (if not, there should be some multivalued points). Then $\varphi^{-1}(A\cap B)=\varphi^{-1}(A)\cap\varphi^{-1}(B)$. If $V_1$ and $V_2$ are open in $M$, then
\[
  \varphi^{-1}(V_1\cap V_2\cap\varphi(\mU))=\varphi^{-1}(V_1\cap\varphi(\mU))\cap\varphi^{-1}(V_2\cap\varphi(\mU))
\]
which is open in $\eR^n$. The same property works for the unions.

Now we turn our attention to the continuity of $\dpt{\varphi}{\mU}{M}$; for an open set $V$ in $M$, we have to show that $\varphi^{-1}(V)$ is open in $\mU\subset\eR^n$. But the definition of the topology on $M$, is precisely the fact that $\varphi^{-1}(V\cap\varphi(\mU))$ is open.
\end{proof}

If $M$ is a differentiable manifold and $N$, a subset of $M$, we say that $N$ is a \defe{submanifold}{submanifold} of dimension $k$ if $\forall\,p\in N$, there exists a chart $\dpt{\varphi}{\mU}{M}$ around $p$ such that
\[
   \varphi^{-1}(\varphi(\mU)\cap N)=\eR^k\cap\mU:=\{(x_1,\ldots,x_k,0\ldots,0)\in\mU\}.
\]

In this case, $N$ is itself a manifold of dimension $k$ for which one can choose the $\varphi$ of the definition as charts.

Let us consider $M$ and $N$, two differentiable manifolds, $\dpt{f}{M}{N}$ a $\Cinf$ map and $x\in M$. We say that $f$ is an \defe{immersion}{immersion} at $x$ if $\dpt{df_x}{T_xM}{T_{f(x)}N}$ is injective and that $f$ is a \defe{submersion}{submersion} if $df_x$ is surjective.

If $M$ and $M$ are two analytic manifolds, a map $\dpt{\phi}{M}{N}$ is \defe{regular}{regular}\label{PgDefRegular} at $p\in M$ if it is analytic at $p$ and $\dpt{d\phi_p}{T_pM}{T_{\phi(p)}N}$ is injective.

\begin{proposition}
Let $M$ be a submanifold of the manifold $N$. If $p\in M$, then there  exists a coordinate system $\{x_1,\ldots,x_n\}$ on a neighbourhood of $p$ in $N$ such that $x_1(p)=\ldots=x_n(p)=0$ and such that the set
\[
   U=\{q\in V\tq x_j(q)=0\,\forall\, m+1\leq j\leq n\}
\]
gives a local chart of $M$ containing $p$.
\label{prop:var_coord}
\end{proposition}

\begin{proof}
No proof.
\end{proof}

The sense of this proposition is that one can put $p$ at the center of a coordinate system on $N$ such that $M$ is just a submanifold of $N$ parametrised by the fact that its last $m-n$ components are zero.

Now we can give a characterization for a submanifold: $N$ is a submanifold of $M$ when $N\subset M$ (as set) and the identity $\dpt{\iota}{N}{M}$ is regular.\label{pg:caract_subvar}

\begin{proposition}
The own topology of a submanifold is finer than the induced one from the manifold.
\label{prop:topo_sub_manif}
\index{topology!on submanifold}
\end{proposition}

\begin{proof}
Let $M$ be a manifold of dimension $n$ and $N$ a submanifold\footnote{In the whole proof, we should say ``there exists a sub-neighbourhood such that\ldots``} of dimension $k<n$. We consider $V$, an open subset of $N$ for the induced topology, so $V=N\cap\mO$ for a certain open subset $\mO$ of $M$. The aim is to show that $V$ is an open subset in the topology of $N$. 

Let us define $\mP=\varphi^{-1}(\mO)$.  The charts of $N$ are the projection to $\eR^k$ of the ones of $M$. We have to consider $W=\varphi^{-1}(V)$, since $N$ is a submanifold, $\varphi^{-1}(\mO\cap N)=\eR^k\cap\mP$. It is clear that $W=\eR^k\cap\mP$ is an open subset of $\eR^k$ because it is the projection on the $k$ first coordinates of an open subset of $\eR^n$.

The subset $V$ of $N$ will be open in the sense of the own topology of $N$ if $\varphi'{}^{-1}(V\cap\varphi'(\mU'))$ is open in $\eR^k$ where $\varphi'$ is the restriction of $\varphi$ to his $k$ first coordinates: $\varphi'(a)=\varphi(a,0)$ and $\mU'$ is the projection of $\mU$.
\end{proof}


\begin{lemma}
Let $V,M$ be two manifolds and $\varphi\colon V\to M$, a differentiable map. We suppose that $\varphi(V)$ is contained in a submanifold $S$ of $M$. If $\dpt{\varphi}{V}{S}$ is continuous, then it is differentiable.
\label{lem:var_cont_diff}
\end{lemma}

\begin{remark}
The map $\varphi$ is certainly continuous as map from $V$ to $M$ (this is in the assumptions). But this don't imply that it is continuous for the topology on $S$ (which is the induced one from $M$). So the continuity of $\dpt{\varphi}{V}{S}$ is a true assumption.
\end{remark}

\begin{proof}
Let $p\in V$. By proposition \ref{prop:var_coord}, we have  a coordinate system $\{x_1,\ldots,x_m\}$ valid on a neighbourhood $N$ of $\varphi(p)$ in $M$ such that the set 
\[
  \{r\in N\tq x_j(r)=0\, \forall s<j\leq m  \}
\]
with the restriction of $(x_1,\ldots x_s)\in N_S$ form a local chart which contains $\varphi(p)$. From the continuity of $\varphi$, there exists a chart $(W,\psi)$ around $p$ such that $\varphi(W)\subset N_S$. The coordinates $x_j(\varphi(q))$ are differentiable functions of  the coordinates of $q$ in $W$. In particular, the coordinates $x_j(\varphi(q))$ for $1\leq j\leq s$ are differentiable and $\dpt{\varphi}{V}{S}$ is differentiable because its expression in a chart is differentiable.
\end{proof}

A consequence of this lemma: if $V$ and $S$ are submanifolds of $M$ with $V\subset S$, and if $S$ has the induced topology from $M$, then $V$ is a submanifold of $S$. Indeed, we can consider the inclusion $\dpt{\iota}{V}{S}$: it is differentiable from $V$ to $M$ and continuous from $V$ to $S$ then it is differentiable from $V$ to $S$ by the lemma. Thus $V=\iota^{-1}(S)$ is a submanifold of $S$ (this is a classical result of differential geometry).

\begin{proposition}
A submanifold is open if and only if it has the same dimension as the main manifold.
\label{prop:subvar_ouvert}
\end{proposition}

\begin{proof}
\subdem{Necessary condition}
We consider some charts $\dpt{\varphi_i}{U_i}{M}$ on some open subsets $U_i$ of $\eR^n$. If $N$ is open in $M$, then this can be written as 
\[
  N=\bigcup_iU_i.
\]
If we choose the charts on $M$ in such a manner that $\dpt{\varphi_i}{U_i\cap \eR^k}{N}$ are charts of $N$, we must have $\varphi_i(U_i\cap\eR^k)=\varphi_i(U_i)$. Then it is clear that $k=n$ is necessary.
\subdem{Sufficient condition}
If $N$ has same dimension as $M$, the charts $\dpt{\varphi_i}{U_i}{M}$ are trivially restricted to $N$.
\end{proof}


%+++++++++++++++++++++++++++++++++++++++++++++++++++++++++++++++++++++++++++++++++++++++++++++++++++++++++++++++++++++++++++
\section{Tangent and cotangent bundle}
%+++++++++++++++++++++++++++++++++++++++++++++++++++++++++++++++++++++++++++++++++++++++++++++++++++++++++++++++++++++++++++

\subsection{Tangent vector}
%--------------------------

As first attempt, we define a tangent vector of $M$ at the point $x\in M$ as the ``derivative'' of a path $\dpt{\gamma}{(-\epsilon,\epsilon)}{M}$ such that $\gamma(0)=x$. It is denoted by
\[
\gamma'(0)= \dsdd{\gamma(t)}{t}{0}.
\]
The question is to correctly define de derivative in the right hand side. Such a definition is achieved as follows. A \defe{tangent vector}{} to the manifold $M$ is a linear map $X\colon  C^{\infty}(M)\to \eR$ which can be written under the form
\begin{equation}  \label{eq_deftgpath}
   Xf=(f\circ X)'(0)=\Dsdd{f(X(t))}{t}{0}
\end{equation}
for a certain path $X\colon \eR\to M$. Notice the abuse of notation between the tangent vector and the path which defines it.

A more formal way to define a tangent vector is to say that it is an equivalent class of path in the sense that two path are equivalent if and only if they induced maps by \eqref{eq_deftgpath} are equals. 

Using the chain rule $d(g\circ f)(a)=dg(f(a))\circ df(a)$ for the differentiation in $\eR^n$, one sees that this equivalence notion doesn't depend on the choice of $\varphi$. In other words, if $\varphi$ and $\tilde{\varphi}$ are two charts for a neighbourhood of $x$, then $(\varphi^{-1} \circ\gamma)'(0)=(\varphi^{-1} \circ\sigma)'(0)$ if and only if $(\tilde{\varphi}^{-1} \circ\gamma)'(0)=(\tilde{\varphi}^{-1} \circ\sigma)'(0)$. The space of all tangent vectors at $x$ is denoted by $T_xM$. There exists a bijection $[\gamma]\leftrightarrow (\varphi^{-1}\circ\gamma)'(0)$ between $T_xM$ and $\eR^n$, so $T_xM$ is endowed with a vector space structure.

If $(\mU,\varphi)$ is a chart around $X(0)$, we can express $Xf$ using only well know objects by defining the function $\tilde f =f\circ\varphi$ and $\tX=\varphi^{-1}\circ X$
\[
  Xf=\Dsdd{ (\tilde f \circ\tX)(t) }{t}{0}=\left.\dsd{\tilde f }{x^{\alpha}}\right|_{x=\tX(0)}\left.\frac{d\tX^{\alpha}}{dt}\right|_{t=0}.
\]
In this sense, we write
\begin{equation}
  X=\frac{d\tX^{\alpha}}{dt} \dsd{}{x^{\alpha}}
\end{equation}
and we say that $\{\partial_1,\ldots,\partial_n\}$ is a basis of $T_xM$. As far as notations are concerned, from now a tangent vector is written as $X=X^{\alpha}\partial_{\alpha}$ where $X^{\alpha}$ is related to the path $\dpt{X}{\eR}{M}$ by $X^{\alpha}=d\tX^{\alpha}/dt$. We will no more mention the chart $\varphi$ and write
\[
  Xf=\Dsdd{f(X(t))}{t}{0}.
\]
Correctness of this short notation is because the equivalence relation is independent of the choice of chart. When we speak about a tangent vector to a given path $X(t)$ without specification, we think about $X'(0)$.

All this construction gives back the notion of tangent vector when $M\subset \eR^m$. In order to see it, think to a surface in $\eR^3$. A tangent vector is precisely given by a derivative of a path: if $\dpt{c}{\eR}{\eR^n}$ is a path in the surface, a tangent vector to this curve is given by
\[
   \lim_{t\to 0}\frac{c(t_0)-c(t_0+t)}{t}
\]
which is a well know limit of a difference in $\eR^n$.

\label{pg:vecto_vecto}Let us precise how does a tangent vector acts on maps others than $\eR$-valued functions. If $V$ is a vector space and $\dpt{f}{M}{V}$, we define
\[
   Xf=(Xf^i)e_i
\]
where $\{e_i\}$ is a basis of $V$ and the functions $\dpt{f^i}{M}{\eR}$, the decomposition of $f$ with respect to this basis. If we consider a map $\dpt{\varphi}{M}{N}$ between two manifolds, the natural definition is $Xf:=dfX$. More precisely, if we consider local coordinates $x^{\alpha}$ and a function $\dpt{f}{M}{\eR}$,
\begin{equation}\label{eq:dvp_phi}
   (d\varphi X)f=\Dsdd{  (f\circ\varphi\circ X)(t) }{t}{0}=\dsd{f}{x^{\alpha}}\dsd{\varphi^{\alpha}}{x\hbeta}\frac{dX\hbeta}{dt}.
\end{equation}
Now we are in a notational trouble: when we write $X=X^{\alpha}\partial_{\alpha}$, the ``$X^{\alpha}$``{} is the derivative of the ``$X^{\alpha}$``{} which appears in the path $X(t)=(X^1(t),\ldots,X^n(t))$ which gives $X$ by $X=X'(0)$. So equation \eqref{eq:dvp_phi} gives
\begin{equation}
   X(\varphi):=d\varphi X=X\hbeta(\partial_{\beta}\varphi^{\alpha})\partial_{\alpha}.
\end{equation}

\subsection{Differential of a map}
%------------------------------------------

Let $\dpt{f}{M_1}{M_2}$ be a differentiable map, $x\in M_1$ and $X\in T_xM_1$, i.e. $\dpt{X}{\eR}{M_1}$ with $X(0)=x$ and $X'(0)=X$. We can consider the path $Y=f\circ X$ in $M_2$. The tangent vector to this path is written $df_x X$.

\begin{proposition}
If $\dpt{f}{M_1}{M_2}$ is a differentiable map between two differentiable manifolds, the map
		\begin{equation}
		\begin{aligned}
			df_x \colon T_xM_1 &\to T_{f(x)}M_2\
			X'(0)&\mapsto (f\circ X)'(0)
		\end{aligned}
	\end{equation}	
is linear.
\end{proposition}

\begin{proof}
We consider local coordinates $\dpt{x}{\eR^n}{M_1}$ and $\dpt{y}{\eR^m}{M_2}$. The maps $\dpt{f}{M_1}{M_2}$ and $\dpt{y^{-1}\circ f\circ x}{\eR^n}{\eR^m}$ will sometimes be denoted by the same symbol $f$. We have $(x^{-1}\circ X)(t)=(x_1(t),\ldots,x_n(t))$ and $(y^{-1}\circ Y)(t)=\big( y_1(x_1(t),\ldots,x_n(t),\ldots, y_m(x_1(t),\ldots,x_n(t)  \big)$, so that
\[
  Y'(0)=\left(   \sum_{i=1}^n \dsd{y_1}{x_i}x_i'(0),\ldots,\sum_{i=1}^n \dsd{y_m}{x_i}x_i'(0)   \right)\in\eR^m
\]
which can be written in a more matricial way under the form
\[
   Y'(0)=\left( \dsd{y_i}{x_j}x'_j(0) \right).
\]
So in the parametrisations $x$ and $y$, the map $df_x$ is given by the matrix $\partial y^i/\partial x_j$ which is well defined from the only given of $f$.
\end{proof}


Let $\dpt{x}{\mU}{M}$ and $\dpt{y}{\mV}{M}$ be two charts systems around $p\in M$. Consider the path $c(t)=x(0,\ldots,t,\ldots 0)$ where the $t$ is at the position $k$. Then, with respect to these coordinates,
\[
  c'(0)f=\Dsdd{ f(c(t))  }{t}{0}=\dsd{f}{x^i}\frac{dc^i}{dt}=\dsd{f}{x^k},
\]
so $c'(0)=\partial/\partial x^k$. Here, implicitly, we wrote $c^i=(x^i)^{-1}\circ c$ where $(x^i)^{-1}$ is the $i$th component of $x^{-1}$ seen as element of $\eR^n$. We can make the same computation with the system $y$. With these abuse of notation,
\begin{equation}
   \dsd{}{x^i}=\sum_j\dsd{y^j}{x^i}\dsd{}{y^j}
\end{equation}
as it can be seen by applying it on any function $\dpt{f}{M}{\eR}$. More precisely if $\dpt{x}{\mU}{M}$ and $\dpt{y}{\mU}{M}$ are two charts (let $\mU$ be the intersection of the domains of $x$ and $y$), let $\dpt{f}{M}{\eR}$ and $\ovf=f\circ x$, $\tilde f =f\circ y$. The action of the vector $\partial_{x^i}$ of the function $f$ is given by
\[
  \partial_{x^i}f=\dsd{\ovf}{x^i}
\]
where the right hand side is a real number that can be computed with usual analysis on $\eR^n$. This real \emph{defines} the left hand side. Now, $\ovf=\tilde f \circ y^{-1}\circ x$, so that
\[
   \dsd{\ovf}{x^i}=\dsd{ (\tilde f \circ y^{-1}\circ x) }{x^i}=\dsd{\tilde f }{y^j}\dsd{y^j}{x^i}
\]
where $\dsd{\tilde f }{y^j}$ is precisely what we write now by $\partial_{y^j}f$ and $\dsd{y^j}{x^i}$ must be understood as the derivative with respect to $x^i$ of the function $\dpt{(y^{-1}\circ x)}{\eR^n}{\eR^n}$.

Let $\dpt{f}{M}{N}$ and $\dpt{g}{N}{\eR}$; the definitions gives
\[
  (df_xX)g=\Dsdd{(g\circ f)(X(t))}{t}{0}
          =\dsd{g}{y^i}\dsd{f^i}{x^{\alpha}}\frac{dX^{\alpha}}{dt}.
\]
This shows that $\dsd{f^i}{x^{\alpha}}\frac{dX^{\alpha}}{dt}$ is $(df_xX)^i$.  But $dX^{\alpha}/dt$ is what we should call $X^{\alpha}$ in the decomposition $X=X^{\alpha}\partial_{\alpha}$ then the matrix of $df$ is given by $\dsd{f^i}{x^{\alpha}}$. So we find back the old notion of differential.

\begin{remark}
If $X\in T_xM$ and $f$ is a \emph{vector valued} function on $M$, then one can define $Xf$ by exactly the same expression. In this case,
\[
  df X=\Dsdd{f(v(t))}{t}{0}=Xf.
\]
\end{remark}

A map $\dpt{f}{M_1}{M_2}$ is an \defe{immersion}{immersion} at $p\in M_1$ if $\dpt{df_p}{T_pM_1}{T_{f(p)}M_2}$ is injective. It is a \defe{submersion}{submersion} if $df_p$ is surjective.

\subsection{Tangent and cotangent bundle}
%--------------------------------------

If $M$ is a $n$ dimensional manifold, as set the tangent bundle\index{tangent!space} is the \emph{disjoint} union of tangent spaces
\[
  TM=\bigcup_{x\in M}T_xM.
\]

\begin{theorem}
	The tangent bundle admits a $2n$ dimensional manifold structure for which the projection 
	\begin{equation}
		\begin{aligned}
			\pi \colon TM &\to M\
			T_pM&\mapsto p
		\end{aligned}
	\end{equation}	
	is a submersion.
\end{theorem}

The structure is easy to guess. If $\dpt{\varphi_{\alpha}}{\mU_{\alpha}}{M}$ is a coordinate system on $M$ (with $\mU_{\alpha}\subset\eR^n$), we define $\dpt{\psi_{\alpha}}{\mU_{\alpha}\times \eR^n}{TM}$ by
\[
  \psi( \underbrace{x_1,\ldots x_n}_{\in\mU_{\alpha}},\underbrace{a_1,\ldots a_n}_{\in\eR^n}  )
          =\sum_i a_i\left.\dsd{}{x_i}\right|_{\varphi(x_1,\ldots,x_n)}.
\]
The map $\psi_{\beta}^{-1}\circ\psi_{\beta}$ is differentiable because
\[
(\psi_{\beta}^{-1}\circ\psi_{\beta})(x,a)=( y(x),\sum_i a_i\left.\dsd{y_j}{x_i}\right|_{y(x)}  )
\]
which is a composition of differentiable maps. The set $TM$ endowed with this structure is called the \defe{tangent bundle}{tangent!bundle}.

%---------------------------------------------------------------------------------------------------------------------------
\subsection{Vector space structure on the tangent space}
%---------------------------------------------------------------------------------------------------------------------------

If \( X,Y\in T_pM\) are tangent vectors, one can define \( X+Y\) and \( \lambda X\) for every \( \lambda\in\eR\). The second one is easy:
\begin{equation}
    \lambda X=\Dsdd{ X(\lambda t) }{t}{0}.
\end{equation}
In order to define the sum of two vectors one has to consider a neighbourhood \( \mU\) of \( p\) in \( M\) and a chart \( \varphi\colon \mU\to \mO\) where \( \mO\) is an open set in \( \eR^n\). Then one consider a basis \( \{ e_i \}_{1\leq i\leq n}\) of \( \eR^n\) at the point \( \varphi(p)\). With these choices we define the ``basis'' path
\begin{equation}
    \gamma_i(t)=\varphi^{-1}(te_i)
\end{equation}
and we write
\begin{equation}
    \partial_i=\frac{ \partial  }{ \partial x_i }=\Dsdd{ \varphi^{-1}(te_i) }{t}{0}.
\end{equation}
The vectors \( \partial_i\) form a basis of \( T_pM\) in the sense of the following lemma.

\begin{lemma}
    The action of a vector \( X\in T_pM\) on a function \( f\colon M\to \eR\) can be decomposed into
    \begin{equation}
        Xf=\sum_{i=1}^n X_i(\partial_if)
    \end{equation}
    with \( X_i\in\eR\)
\end{lemma}

\begin{proof}
    Let \( \varphi\colon M\to \eR^n\) be a chart of a neighbourhood of \( p\) with \( \varphi(p)=0\). We determine the value of \( X_i\) using the function
    \begin{equation}
        f_i(x)=\varphi(x)_i,
    \end{equation}
    that is the \( i\)th component of the point \( \varphi(x)\in\eR^n\). Then if we write \( \varphi\big( X(t) \big)=\sum_j a_j(t)e_j\) we have
    \begin{subequations}
        \begin{align}
            X(f_i)=\Dsdd{ f_i\big( X(t) \big) }{t}{0}=\Dsdd{ \big[ \sum_ja_j(t)e_j \big]_i }{t}{0}=\Dsdd{ ai_(t) }{t}{0}=a_i'(0).
        \end{align}
    \end{subequations}
    Notice that \( a_i(0)=0\) since \( X(0)=p\) and \( \varphi(p)=0\). The combination \( f\circ\varphi^{-1}\) is an usual function from \( \eR^n\) to \( \eR\), so that we can use the chain rule on it. The following computation thus make sense:
    \begin{subequations}
        \begin{align}
            Xf&=\Dsdd{ f\big( X(t) \big) }{t}{0}\\
            &=\Dsdd{ f\Big( \varphi^{-1}\varphi\big( X(t) \big) \Big) }{t}{0}\\
            &=\Dsdd{ (f\circ\varphi^{-1})\big( \sum_ja_j(t)e_j \big) }{t}{0}\\
            &=\sum_k \frac{ \partial (f\circ\varphi^{-1}) }{ \partial x_k }\big( \underbrace{\sum_ja_j(0)e_j}_{=\varphi(p)=0} \big)\underbrace{\frac{ d\big[ \sum_ja_j(t)e_j \big]_k  }{ dt }}_{=a'_k(0)}\\
            &=\sum_k a'_k(0)\frac{ \partial (f\circ\varphi^{-1}) }{ \partial x_k }(0).
        \end{align}
    \end{subequations}
    Now using the definition of a derivative of a function \( \eR^n\to \eR\) and of the ``basis'' tangent vector \( \partial_k\),
    \begin{subequations}
        \begin{align}
            \frac{ \partial (f\circ\varphi^{-1}) }{ \partial x_k }(0)&=\Dsdd{ (f\circ\varphi^{-1})(te_k) }{t}{0}\\
            &=\partial_k f
        \end{align}
    \end{subequations}
   At the end of the day we have
   \begin{equation}
       Xf=\sum_k a'_k(0)\partial_kf.
   \end{equation}
\end{proof}

This lemma allows us to define the sum in \( T_pM\) as
\begin{equation}
    \left( \sum_kX_k\partial_k \right)+\left( \sum_kY_k\partial_k \right)=\sum_k (X_k+Y_k)\partial_k
\end{equation}
when \( X_k\) and \( Y_k\) are reals.

The tangent space \( T_pM\) is thus a vector space.


\subsection{Commutator of vector fields}

If $X$, $Y\in\cvec(M)$, one can define the \defe{commutator}{commutator of vector fields} $[X,Y]$ in the following way. First remark that, if $\dpt{f}{M}{\eR}$, the object $X(f)$ is also a function from $M$ to $\eR$ by $X(f)(x)=X_x(f)$, so we can apply $Y$ on $X(f)$. The definition of $[X,Y]_x$ is
\begin{equation}
  [X,Y]_xf=X_x(Yf)-Y_x(Xf).
\end{equation}
If $X=X^i\partial_i$ and $Y=Y^j\partial_j$, then 
$XY(f)=X^i\partial_i(Y^j\partial_jf)
     =X^i\partial_i Y^j\partial_j f+X^iY^j\partial^2_{ij}f$.
From symmetry $\partial^2_{ij}f=\partial^2_{ji}f$, the difference $XYf-YXf$ is only $X^i\partial_iY^j-Y^i\partial_iX^j$, so that
\begin{equation}
  [X,Y]^i=XY^i-YX^i
\end{equation}
where $X^i$ and $Y^i$ are seen as functions from $M$ to $\eR$.


\subsection{Some Leibnitz formulas}

See \cite{kobayashi}, chapter I, proposition 1.4.

\begin{lemma}
If $M$ and $N$ are two manifolds, we have a canonical isomorphism
\[
     T_{(p,q)}(M\times N)\simeq T_pM+T_qN.
\]
\label{lemLeibnitz}
\end{lemma}

\begin{proof}
A $Z\in T_{(p,q)}(M\times N)$ is the tangent vector to a curve $(x(t),y(y))$ in $M\times N$. We can consider $X\in T_pM$ given by $X=x'(0)$ and $Y\in T_qN$ given by $Y=y'(0)$. The isomorphism is the identification $(X,Y)\simeq Z$. Indeed, let us define $\oX\in T_{(p,q)}(M\times N)$, the tangent vector to the curve $(x(t),q)$, and $\oY\in T_{(p,q)}(M\times N)$, the tangent vector to the curve $(p,y(t))$. Then $Z=\oX+\oY$ because for any $\dpt{f}{M\times N}{\eR}$,
\begin{equation}
 Zf=\dsdd{f(x(t),y(t))}{t}{0}
   =\dsdd{f(x(t),y(0))}{t}{0}+\dsdd{f(x(0),y(t))}{t}{0}
   =\oX f+\oY f.
\end{equation}
\end{proof}

\begin{proposition}[Leibnitz formula] \label{Leibnitz}
Let us consider $M,N,V$, three manifold; a map $\dpt{\varphi}{M\times N}{V}$ and a vector $Z\in T_{(p,q)}(M\times N)$ which corresponds (lemma \ref{lemLeibnitz}) to $(X,Y)\in T_pM+T_qN$.

If we define $\dpt{\varphi_1}{M}{V}$ and  $\dpt{\varphi_2}{N}{V}$ by $\varphi_1(p')=\varphi(p',q)$ and $\varphi_2(q')=\varphi(p,q')$, we have the \defe{Leibnitz formula}{Leibnitz formula}:
\begin{equation}
    d\varphi(Z)=d\varphi_1(X)+d\varphi_2(Y).
\end{equation}
\end{proposition}
\begin{proof}
 Since $Z=\oX+\oY$, we just have to remark that
\[
                  d\varphi(\oX)=\dsdd{\varphi(x(t),q)}{t}{0}=d\varphi_1(X),
\]
so $d\varphi(Z)=d\varphi(\oX+\oY)=d\varphi_1(X)+d\varphi_2(Y)$.
\end{proof}
One of the most important application of the Leibnitz rule is the corollary \ref{cor_PrincLeib} on principal bundles. 

\subsection{Cotangent bundle}

A form on a vector space $V$ is a linear map $\dpt{\alpha}{V}{\eR}$. The set of all forms on $V$ is denoted by $V^*$ and is called the \defe{dual space}{dual!of a vector space} of $V$. On each point of a manifold, one can consider the tangent bundle which is a vector space. Then one can consider, for each $x\in M$ the dual space $T^*_xM:=(T_xM)^*$ which is called the \defe{cotangent bundle}{cotangent bundle}. A $1$-\defe{differential form}{differential!form} on $M$ is a smooth map $\dpt{\omega}{M}{T^*M}$ such that $\omega_x:=\omega(x)\in T^*_xM$. So, for each $x\in M$, we have a $1$-form $\dpt{\omega_x}{T_xM}{\eR}$.

Here, the smoothness is the fact that for any smooth vector field $X\in\cvec(M)$, the map $x\to\omega_x(X_x)$ is smooth as function on $M$. One often considers vector-valued forms. This is exactly the same, but $\omega_xX_x$ belongs to a certain vector space instead of $\eR$. The set of $V$-valued $1$-forms on $M$ is denoted by $\Omega(M,V)$ \nomenclature{$\Omega(M,V)$}{$V$ valued $1$-forms} and simply $\Omega(M)$ if $V=\eR$
The cotangent space $T^*_pM$ of $M$ at $p$ is the dual space of $T_pM$, i.e. the vector space of all the (real valued) linear\footnote{When we say \emph{a form}, we will always mean \emph{a linear form}.} $1$-forms on $T_pM$. In the coordinate system $\dpt{x}{\mU}{M}$, we naturally use, on $T^*_pM$, the dual basis of the basis $\{\partial/\partial_{x^i},\ldots\partial/\partial_{x^i}\}$ of $T_pM$. This dual basis is denoted by $\{dx_1,\ldots,dx_n\}$, the definition being as usual:
\begin{equation}\label{eq:dx_v}
  dx_i(\partial^j)=\delta^j_i.
\end{equation}
The notation comes from the fact that equation \eqref{eq:dx_v} describes the action of the differential of the projection $\dpt{x_i}{\mU}{\eR}$ on the vector $\partial^j$.

If $(\mU_{\alpha},\varphi_{\alpha})$ is a chart of $M$, then the maps
		\begin{equation}
		\begin{aligned}
			\phi_{\alpha} \colon \mU_{\alpha}\times\eR^n &\to T^*M\
			(x,a)&\mapsto a^idx_i|_x
		\end{aligned}
	\end{equation}	
give to $T^*M$ a $2n$ dimensional manifold structure such that the canonical projection $\dpt{\pi}{T^*M}{M}$ is an immersion.

When $V$ is a finite-dimensional vector space, we denote by $V^*$ its dual\footnote{The vector space of all the linear map $V\to \eR$.} and we often use the identifications $V\simeq V^*\simeq T_vV\simeq T_wV\simeq T^*_vV$ where $v$ and $w$ are any elements of $V$. Note however that there are no \emph{canonical} isomorphism between these spaces, unless we consider some basis.

%+++++++++++++++++++++++++++++++++++++++++++++++++++++++++++++++++++++++++++++++++++++++++++++++++++++++++++++++++++++++++++ 
\section{Exterior calculus}
%+++++++++++++++++++++++++++++++++++++++++++++++++++++++++++++++++++++++++++++++++++++++++++++++++++++++++++++++++++++++++++

%--------------------------------------------------------------------------------------------------------------------------- 
\subsection{The exterior algebra}
%---------------------------------------------------------------------------------------------------------------------------

\begin{definition}[Exterior product]
    If $V$ is a vector space, we denote by $\Lambda^kV^*$ the space of all the $k$-form on $V$. We define the \defe{exterior product}{product!exterior} $\dpt{\wedge}{\Lambda^kV^*\times\Lambda^lV^*}{\Lambda^{k+l}V^*}$ by
    \begin{equation}
      (\omega^k\wedge\eta^l)(v_1,\ldots,v_{k+l})
      =\us{k!l!}\sum_{\sigma\in S_{k+l}} (-1)^{\sigma}   \omega(v_{\sigma(1)},\ldots,v_{\sigma(k)})\eta(v_{\sigma(k+1)},v_{\sigma(k+1)})
    \end{equation}
\end{definition}
If $\{e_1,\ldots,e_n\}$ is a basis of $V$, the dual basis $\{\sigma^1,\ldots,\sigma^n\}$ of $V^*$ is defined by $\sigma^i(e_j)=\delta^i_j$.

If $I=\{1\leq i_1\leq\ldots i_k\leq n\}$, we write $\sigma^I=\sigma^{i_1}\wedge\ldots\sigma^{i_k}$ any $k$-form can be decomposed as
\[
  \omega=\sum_{I}\omega_I\sigma^I.
\]
The exterior algebra is provided with the \defe{interior product}{interior!product} denoted by $\iota$. It is defined by\label{pg_DefProdExt}
\begin{equation}
\begin{aligned}
 \iota(v_0)\colon\Lambda^kW&\to \Lambda^{k-1}W \\ 
(\iota(v_0)\omega)(v_1,\ldots,v_{k-1})& =\omega(v_0,v_1,\ldots,v_{k-1}).
\end{aligned}
\end{equation}

\begin{lemma}
    Let \( \sigma\) be an element of the symmetric group\footnote{Definition \ref{DEFooJNPIooMuzIXd}.} of the set \( \{ a_1,\ldots, a_n \}\) where the \( a_i\) are integers. Then
    \begin{equation}
        (dx_{a_1}\wedge\ldots \wedge dx_{a_n})(e_{\sigma(a_1)},\ldots, e_{\sigma(a_n)})=(-1)^{\sigma}.
    \end{equation}
\end{lemma}

\begin{proof}
    We make it by induction over \( n\). With \( n=1\) the only permutation is the identity; the claim reduces to \( dx_{a_1}(e_{a_1})=1\). Let us try with \( n=2\). Up to renumbering we have
    \begin{equation}
        (dx_1\wedge dx_2)(e_1,e_2)=1
    \end{equation}
    and
    \begin{equation}
        (dx_1\wedge dx_2)(e_2,e_1)=-1.
    \end{equation}
    We pass to the induction. Let \( \sigma\in S_n\). We have
    \begin{subequations}
        \begin{align}
            (dx_{a_1}\wedge dx_{a_n})(&e_{\sigma(a_1)},\ldots, e_{\sigma(a_n)})=dx_{a_1}\wedge (dx_{a_n})(e_{\sigma(a_1)},\ldots, e_{\sigma(a_n)})\\
            &=\sum_{\phi\in S_n}(-1)^{\phi}\frac{1}{ (n-1)! }dx_{a_1}\big( e_{\phi\sigma(a_1)} \big)(dx_{a_2}\wedge\ldots\wedge dx_{a_n})(e_{\phi\sigma(a_2)},\ldots, e_{\phi\sigma(a_n)})\\
            &=\sum_{\phi\in S_n}(-1)^{\phi}\frac{1}{ (n-1)! }\delta_{a_1,\phi\sigma(a_1)}(-1)^{\phi\sigma}\\
            &=\sum_{\phi\in S_n}\delta_{a_1,\phi\sigma(a_1)}(-1)^{\sigma}\frac{1}{ (n-1)! }
        \end{align}
    \end{subequations}
    where we used the fact that the sign of a permutation provides a morphism between \( S_n\) and \( \{ -1,1 \}\) (proposition \ref{ProphIuJrC}\ref{ITEMooBQKUooFTkvSu}). In the sum over \( S_n\), only the \( \phi\) that make \( \sigma(a_1)\to a_1\) remain; there are \( | S_{n-1} |=(n-1)!\) such elements. Thus the whole evaluates to \( (-1)^{\sigma}\).
\end{proof}

\begin{lemma}[\cite{MonCerveau}]    \label{LEMooICRXooFKPCRd}
    Let \( \tau_i\colon \eR^n\to \eR^{n-1}\) defined by
    \begin{equation}
        \tau_i(v)_k=\begin{cases}
            v_k    &   \text{if } k<i\\
            v_{k+1}    &    \text{if } k\geq i\text{.}
        \end{cases}
    \end{equation}
    Then we have
    \begin{equation}
        (dx_1\wedge\ldots\wedge\widehat{dx_i}\wedge\ldots\wedge dx_n)(v_1,\ldots, \widehat{v_i},\ldots, v_n)=
        \det\Big(  \tau_i(v_1),\ldots, \widehat{\tau_i(v_i)},\ldots, \tau_i(v_n)  \Big)
    \end{equation}
    where the hat denotes a non present term.
\end{lemma}

\begin{proof}
    We extend \( \tau_i\) to the dual : \( \tau_i\colon(\eR^n)^*  \to (\eR^{n-1})^*\) is defined by
    \begin{equation}
        \tau_i(dx_k)=\begin{cases}
            dy_k    &   \text{if } k<i\\
            dy_{k-1}    &    \text{if } k>i
        \end{cases}
    \end{equation}
    (not defined on \( dx_i\)). It is easy to check that, if \( k\neq i\),
    \begin{equation}
        \tau_i(dx_k)\tau_i(v)=dx_k(v).
    \end{equation}
    The value of  $(dx_1\wedge\ldots\wedge\widehat{dx_i}\wedge\ldots\wedge dx_n)(v_1,\ldots, \widehat{v_i},\ldots, v_n)$ is a polynomial in the variables \( dx_k(v_l)\) (with \( k\neq l\)). Since \( dx_k(v_l)=\tau\i(dx_k)\big( \tau_iv_l \big)\), the same polynomial will give the value of
    \begin{equation}
        (\tau_idx_1\wedge\ldots\wedge \widehat{\tau_idx_i}\wedge\ldots\wedge \tau_idx_n  )(\tau_i v_1,\ldots, \widehat{\tau_iv_i},\ldots, \tau_iv_n).
    \end{equation}
    Thus we have
    \begin{subequations}
        \begin{align}
            (dx_1\wedge\ldots\wedge\widehat{dx_i}&\wedge\ldots\wedge dx_n)(v_1,\ldots, \widehat{v_i},\ldots, v_n)\\
            &=(\tau_idx_1\wedge\ldots\wedge \widehat{\tau_idx_i}\wedge\ldots\wedge \tau_idx_n  )(\tau_i v_1,\ldots, \widehat{\tau_iv_i},\ldots, \tau_iv_n)\\
            &=(dy_1\wedge\ldots\wedge dy_{n-1})(\tau_iv_1,\ldots,\widehat{\tau_iv_i},\ldots, \tau_iv_n) \label{SUBEQooQGSKooSgfxJh}\\
            &=\det\big( \tau_iv_1,\ldots, \widehat{\tau_i v_i},\ldots, \tau_iv_n \big)
        \end{align}
    \end{subequations}
    The last equality is because \eqref{SUBEQooQGSKooSgfxJh} is is a \( (n-1)\)-form applied to \( n-1\) vectors in \( \eR^{n-1}\) and so is the determinant. 
\end{proof}

%--------------------------------------------------------------------------------------------------------------------------- 
\subsection{Differential of \texorpdfstring{$k$}{k}-forms}
%---------------------------------------------------------------------------------------------------------------------------

The differential of a $k$-form is defined by the following theorem.

\begin{theorem}
Let $M$ be a differentiable manifold. Then for each $k\in \eN$, there exists an unique map
\[
  \dpt{d}{\Omega^k(M)}{\Omega^{k+1}(M)}
\]
such that

\begin{enumerate}
\item $d$ is linear,
\item for $k=0$, we find back the $\dpt{d}{\Cinf(M)}{\Omega^1(M)}$ previously defined,
\item if $f$ is a function and $\omega^k$ a $k$-form, then 
\begin{equation}
d(f\omega^k)=df\wedge\omega^k+fd\omega^k,
\end{equation}


\item $d(\omega^k\wedge\eta^l)=d\omega^k\wedge\eta^l+(-1)^k\omega^k\wedge d\eta^l$,
\item $d\circ d=0$.
\end{enumerate}
\end{theorem}

An explicit expression for $d\omega^k$ is actually given by
\begin{equation}
   d\omega^k=\sum d\omega_I\wedge dx^I
\end{equation}
if $\omega^k=\sum\omega_I dx^I$.
An useful other way to write it is the following. If $\omega$ is a $k$-form and $X_1,\ldots,X_{p+1}$ some vector fields,
\begin{equation}\label{eq:formule_domega}
\begin{split}
  (k+1)d\omega(X_1,\ldots,X_{p+1})&=\sum_{i=1}^{p+1}(-1)^{i+1}X_i\omega(X_1,\ldots\hat{X}_i,\ldots,X_{p+1})\\
                                  &\quad+\sum_{i<j}(-1)^{i+j}\omega([X_i,X_j],X_1,\ldots,\hX_i,\hX_j,\ldots,X_{p+1}).
\end{split}
\end{equation}
Let us show it with $p=1$. Let $\omega=\omega_i dx^i$ and compute $d\omega(X,Y)=\partial_i\omega_j(dx^i\wedge dx^j)(X,Y)$. For this, we have to keep in mind that the $\partial_i$ acts only on $\omega_j$ while, in equation \eqref{eq:formule_domega}, a term $X\omega(Y)$ means --pointwise-- the action of $X$ on the function $\dpt{\omega(Y)}{M}{\eR}$. So we have to use Leibnitz formula:
\[
  (\partial_i\omega_j)X^iY^j=(X\omega_j)Y^j
                            =X(\omega_j Y^j)-\omega_j XY^j.
\]
On the other hand, we know that $[X,Y]^i=XY^i-YX^i$, so
\begin{equation}
   d\omega(X,Y)=X\omega(Y)-Y\omega(X)-\omega([X,Y]).
\end{equation}

\subsubsection{Hodge dual operator}
%/////////////////////////////
Let us take a manifold $M$ endowed with a metric $g$.  We can define a map $\dpt{r}{T^*_xM}{T_xM}$ by, for $\alpha\in T^*_xM$,
\[
   \scal{r(\alpha)}{v}=\alpha(v).
\]
for all $v\in T_xM$, where $\scal{\cdot}{\cdot}$ stands for the product given by the metric $g$. If we have $\alpha,\beta\in T^*_xM$, we can define
\[
   \scal{\alpha}{\beta}=\scal{r(\alpha)}{r(\beta)}.
\]
With this, we define an inner product on $\Lambda^p(T^*_xM)$:
\[
   \scal{\alpha_1\wedge\ldots\alpha_p}{\beta_1\wedge\ldots\beta_p}=\det_{ij}\scal{\alpha_i}{\beta_j}.
\]

\begin{definition}      \label{DEFooUOJQooSzKjNR}
    The \defe{Hodge operator}{Hodge operator} is $\dpt{\hodge}{\Lambda^p(T^*_xM)}{\Lambda^{n-p}(T^*_xM)}$ such that for any $\phi\in\Lambda^p(T^*_xM)$,
    \begin{equation}
       \phi\wedge(\hodge\psi)=\scal{\phi}{\psi}\Omega=\langle \phi,\psi \rangle\sqrt{|\det(g)|}dx^1\wedge\ldots\wedge dx^n.
    \end{equation}
\end{definition}

\begin{example} \label{EXooCIYIooFPMLMU}
    We consider \( \eR^n\) with the euclidian metric. If \( \sigma=dx_j\), then we expect \( \hodge\sigma\) to be \( sdx_1\wedge\ldots\wedge \widehat{dx_j}\wedge\ldots\wedge dx_n\) for a certain factor \( s\) to be fixed (something like \( (-1)^j\)).

    For every \( 1\)-form \( \phi\) we need \( \phi\wedge(\hodge \sigma)=\langle \phi, \sigma\rangle dx_1\wedge\ldots\wedge dx_n\). A basis of \( \Wedge^1(TM)\) is \( \{ dx_k \}_{k=1,\ldots, n}\), so we test on \( dx_k\).

    First we have
    \begin{equation}
        \langle dx_k, dx_j\rangle =\langle e_k, e_j\rangle =\delta_{kj}.
    \end{equation}
    Then
    \begin{equation}
        s\,dx_k\wedge dx_1\wedge\ldots\wedge \widehat{dx_j}\wedge\ldots\wedge dx_n=s\delta_{kj}(-1)^{j+1}dx_1\wedge\ldots\wedge dx_n.
    \end{equation}
    Thus we need \( s=(-1)^{j+1}\) and we have
    \begin{equation}
        \hodge dx_j=(-1)^{j+1}dx_1\wedge\ldots\wedge \widehat{dx_j}\wedge\ldots\wedge dx_n.
    \end{equation}
\end{example}

\subsubsection{Volume form and orientation}
%//////////////////////////////////////////

Let $M$ be a $n$ dimensional smooth manifold. A \defe{volume form}{volume!form} on $M$ is a nowhere vanishing $n$-form and the manifold itself is said to be \defe{orientable}{orientable manifold} if such a volume form exists. Two volume forms $\mu_1$ and $\mu_2$ are describe the same orientation if there exists a function $f>0$ such that\footnote{Recall that the space of $n$-forms is one-dimensional.} $\mu_1=f\mu_2$. 

\begin{proposition}
There exists only two orientations on a connected orientable manifold.
\end{proposition}
\begin{probleme}
    Check if the statement of that proposition is correct. Find a reference.
\end{probleme}

One says that the \emph{ordered} basis $(v_1,\cdots,v_n)$ of $T_xM$ is \defe{positively oriented}{positive!orientation} with respect to the volume form $\mu$ is $\mu_x(v_1,\cdots,v_n)>0$.

\subsection{Musical isomorphism}\label{subsec_musique}\index{musical isomorphism}
%---------------------------------

In some literature, we find the symbols $v^{\flat}$ and $\alpha^{\sharp}$. What does it mean ? For $X\in\cvec(M)$ and $\omega\in\Omega^2(M)$, the \defe{flat}{flat} operation $v^{\flat}\in\Omega^1(M)$ is simply defined by the inner product:
\begin{equation}        \label{EQooBTWXooTqoNxa}
  v^{\flat}=i(v)\omega
\end{equation}
 In the same way, we define the \defe{sharp}{sharp} operation by taking a $1$-form $\alpha$ and defining $\alpha^{\sharp}$ by
\begin{equation}
   i(\alpha^{\sharp})\omega=\alpha.
\end{equation}
An immediate property is, for all $v\in\cvec(M)$, $v^{\flat\sharp}=v$, and for all $\alpha\in\Omega^1(M)$, $\alpha^{\sharp\flat}=\omega$.

\subsection{Pull-back and push-forward}

Let $\dpt{\varphi}{M}{N}$ be a smooth map, $\alpha$ a $k$-form on $N$,
and $Y$ a vector field on $N$. Consider the map $\dpt{d\varphi}{T_xM}{T_{\varphi(x)}M}$. The aim is to extend it to a map from the tensor algebra of ${T_xM}$ to the one of $T_{\varphi(x)}M$. See \cite{kobayashi} for precise definition of the tensor algebra.

The \defe{pull-back}{pull-back!of a $k$-form} of $\varphi$ on a $k$-form $\alpha$ is the map
\[
 \dpt{\varphi^*}{\Omega^k(N)}{\Omega^k(M)}
\]
 defined by
\begin{equation}\label{306e1}
 (\varphi^*\alpha)_m(v_1,\ldots,v_k)
 =\alpha_{\varphi(m)}(d\varphi_mv_1,\ldots,d\varphi_mv_k)
\end{equation}
for all $m\in M$ and $v_i\in\cvec(M)$.

Note the particular case $k=0$. In this case, we take --instead of $\alpha$-- a function $\dpt{f}{N}{\eR}$ and the definition \eqref{306e1} gives $\dpt{\varphi^*f}{M}{\eR}$ by
\[
     \varphi^*f=f\circ\varphi.
\]

The \defe{push-forward}{push-forward!of a $k$-form} of $\varphi$ on a $k$-form is the map
\[
 \dpt{\varphi_*}{\Omega^k(M)}{\Omega^k(N)}
\]
defined by $\varphi_*=(\varphi^{-1})^*$. For $v\in T_nN$, we explicitly have:
\[
                   (\varphi_*\alpha)_n(v)=\alpha_{\varphi^{-1}(n)}(d\varphi_n^{-1} v).
\]

Let now $\dpt{\varphi}{M}{N}$ be a diffeomorphism. The \defe{pull-back}{pull-back!of a vector field} of $\varphi$ on a vector field is the map
\[
           \dpt{\varphi^*}{\cvec(N)}{\cvec(M)}
\]
defined by
\[
              (\varphi^*Y)(m)=[(d\varphi^{-1})_m\circ Y\circ\varphi](m),
\]
or
\[
 (\varphi^*Y)_{\varphi^{-1}(n)}=(d\varphi^{-1})_nY_n,
\]
for all $n\in N$ and $m\in M$. Notice that \[\dpt{(d\varphi^{-1})_n}{T_nN}{T_{\varphi^{-1}(n)}M},\] and that  $\varphi^{-1}(n)$ is well defined because $\varphi$ is an homeomorphism.

The \defe{push-forward}{push-forward!of a vector field} is, as before, defined by $\varphi_*=(\varphi^{-1})^*$. In order to show how to manipulate these notations, let us prove the following equation:
\[ 
   f_{*\xi}=(df)_{\xi}.
\]
For $\dpt{\varphi}{M}{N}$ and $Y$ in $\cvec(N)$, we just defined $\dpt{\varphi^*}{\cvec(N)}{\cvec(M)}$, by
\begin{eqnarray}
 \label{2112r1}(\varphi^*Y)_{\varphi^{-1}(n)}=(d\varphi^{-1})_nY_n.
\end{eqnarray}
Take $\dpt{f}{M}{N}$; we want to compute $f_*=(f^{-1})^*$ with $\dpt{(f^{-1})^*}{\cvec(M)}{\cvec(N)}$. Replacing the ``$^{-1}$``\ on the right places, the definition \eqref{2112r1} gives us
\[
 \Big[(f^{-1})^*X\Big]_{f(m)}=(df)_mX_m,
\]
if $X\in\cvec(M)$, and $m\in M$.

We can rewrite it without  any indices: the coherence of the spaces automatically impose the indices: $(f^{-1})^*X=(df)X$. It can also be rewritten as $(f^{-1})^*=df$, and thus $f_*=df$. From there to $f_{* \xi}=(df)_{\xi}$, it is straightforward.

%+++++++++++++++++++++++++++++++++++++++++++++++++++++++++++++++++++++++++++++++++++++++++++++++++++++++++++++++++++++++++++ 
\section{Partition of unity}
%+++++++++++++++++++++++++++++++++++++++++++++++++++++++++++++++++++++++++++++++++++++++++++++++++++++++++++++++++++++++++++


\begin{definition}[\cite{ooQCDSooCpqDvB}]       \label{DEFooKFXLooFRLaBG}
    Let \( X\) be a topological space. A \defe{partition of unity}{partition!of unity} of \( X\) is a family of continuous functions \( \{ \psi_j \}_{j\in J}\) such that 
    \begin{enumerate}
        \item
            \( \psi_j\colon X\to \mathopen[ 0 , 1 \mathclose]\);
        \item
            for every \( x\in X\) there exists a neighbourhood of \( x\) in \( X\) in which only a finite number of the \( \psi_j\)'s is non zero;
        \item
            for every \( x\in X\) we have
            \begin{equation}
                \sum_{j\in J}\psi_j(x)=1.
            \end{equation}
    \end{enumerate}
\end{definition}

\begin{definition}[\cite{ooQCDSooCpqDvB}]
    Let \( X\) be a topological space and \( \{ U_i \}_{i\in I}\) be a locally finite covering of \( X\). A partition of unity is \defe{subordinate}{partition!of unity!subordinate} to that covering if it is indexed by \( I\) (\( J=I\) in the definition \ref{DEFooKFXLooFRLaBG}) and such that \( \supp(\psi_i)\subset U_i\) for every \( i\in I\).
\end{definition}

\begin{theorem}[\cite{ooQCDSooCpqDvB}]      \label{THOooPCHDooITWKpC}
    Let \( \Omega\) be an open set in \( \eR^d\) and \( \{ U_i \}_{i\in I}\) be an open covering of \( \Omega\). There exists
    \begin{enumerate}
        \item
            a \(  C^{\infty}\) partition of unity \( \{ \psi_j \}_{j\in J}\) such that \( \supp(\psi_j)\) is compact in one of the \( U_i\);
        \item       \label{ITEMooFGMJooQPLqGY}
            a \(  C^{\infty}\) partition of unity \( \{ \alpha_i \}_{i\in I}\) subordinated to the covering, such that for every compact \( K\) only a finite number of these \( \psi_i\)'s is non zero.
    \end{enumerate}
\end{theorem}

\begin{remark}
    This theorem does not furnish a smooth compactly supported partition of unity subordinated to the given covering. Either you choose the partition to be compactly supported, either you choose them subordinated to the covering.
\end{remark}

\begin{corollary}  \label{CORooMSWPooCxvuhm}
    If \( \Omega\) is bounded and \( \{U_i \}_{i\in I}\) is an open covering of \( \Omega\), there exists a partition of unity subordinated to \( \{ U_i \}_{i\in I}\) such that each \( \psi_i\) belongs to \( \swD(U_i)\).
\end{corollary}

\begin{proof}
    If \( \Omega\) is bounded in \( \eR^d\) we can consider \( U'_i=U_i\cap \Omega\) and use the point \ref{ITEMooFGMJooQPLqGY} of theorem \ref{THOooPCHDooITWKpC} for the covering \( \{ U'_i \}_{i\in I}\). So we have a partition of unity subordinated to that covering with supports in the \( U'_i\)'s. Since the support is closed and the \( U_i\)'s are bounded, the supports are compact. The functions of this partition of unity are also subordinated to the original \( U_i\)'s.
\end{proof}

%+++++++++++++++++++++++++++++++++++++++++++++++++++++++++++++++++++++++++++++++++++++++++++++++++++++++++++++++++++++++++++ 
\section{Integration of a differential form}
%+++++++++++++++++++++++++++++++++++++++++++++++++++++++++++++++++++++++++++++++++++++++++++++++++++++++++++++++++++++++++++

%--------------------------------------------------------------------------------------------------------------------------- 
\subsection{Open set in \( \eR^n\)}
%---------------------------------------------------------------------------------------------------------------------------

Let \( U\) be an open set of \( \eR^n\). A differential form of degree \( n\) over \( U\) can always be written under the form
\begin{equation}
    \omega_x=f(x)dx_1\wedge\ldots\wedge dx_n;
\end{equation}
this is proposition \ref{ProprbjihK}. 

\begin{definition}      \label{DEFooEYRFooRQTmRF}
    The integral of \( \omega\) on \( U\) is
    \begin{equation}
        \int_{U}f\,dx_1\wedge\ldots\wedge dx_n=\int_Uf
    \end{equation}
    The second integral is the integral of a function on \( \eR^n\), that is definition \ref{DefTVOooleEst} where the measure is the Lebesgue measure on \( \eR^n\).
\end{definition}

\begin{lemma}[Change of variable]       \label{LEMooNCYSooXtnCKq}
    Let \( f\colon V\to U\) be a diffeomorphism of open sets in \( \eR^n\) and \( \omega\) be a \( n\)-form on \( U\). Then we have
    \begin{equation}
        \int_U\omega=\int_{f^{-1}(U)}f^*\omega
    \end{equation}
    if \( \det(f)>0\). A sign change if \( \det(df)<0\).
\end{lemma}

\begin{proof}
    Let, for \( y\in U\), write the form \( \omega\) as \( \omega_y=h(y)dy_1\wedge\ldots\wedge dy_n\). Taking \( v_i\in \Gamma(TV)\) we have
    \begin{subequations}
        \begin{align}
            (f^*\omega)_x(v_1,\ldots, v_n)&=\omega_{f(x)}\big( df_xv_1,\ldots, df_xv_n \big)\\
            &=h\big( f(x) \big)\det\begin{pmatrix}
                df_xv_1    \\ 
                \vdots    \\ 
                df_xv_n    
            \end{pmatrix}\\
            &=(h\circ f)(x)\det(df_x)\det\begin{pmatrix}
                v_1    \\ 
                \vdots    \\ 
                v_n    
            \end{pmatrix}\\
            &=(h\circ f)(x)\det(df_x)(dx_1\wedge\ldots\wedge dx_n)(v_1,\ldots, v_n).
        \end{align}
    \end{subequations}
    Thus
    \begin{equation}
        f^*\omega= (h\circ f)\det(df)dx_1\wedge\ldots\wedge dx_n
    \end{equation}
    Using the usual change of variable theorem \ref{THOooUMIWooZUtUSg}\ref{ITEMooAJGDooGHKnvj} (and taking a sign if \( \det(df)<0\) because there is an absolute value in around the jacobian in \eqref{EQooLYAWooTArAZR}) :
    \begin{equation}
        \int_{f^{-1}(U)}f^*\omega=\int_V(h\circ f)\det(df)=\int_{f(V)}h=\int_Uh=\int_U\omega.
    \end{equation}
\end{proof}

That is for integrating a differential form on an open set of \( \eR^n\). In order to integrate on a manifold we ``simply'' use a pull-back with a chart system. There will be three complications
\begin{itemize}
    \item If an atlas is made from more than one chart, what about the intersections ?
    \item Independence with respect to the choice of the chart.
    \item Integrating a vector field (that is not obviously a \( n\)-form).
\end{itemize}

%--------------------------------------------------------------------------------------------------------------------------- 
\subsection{One chart on a manifold}
%---------------------------------------------------------------------------------------------------------------------------

We suppose \( (M,g)\) to be a \( n\)-dimensional Riemannian manifold and \( S\) to be a \( (n-1)\)-dimensional submanifold. We suppose that both are inside only one chart
\begin{equation}
    \phi\colon U\subset \eR^n\to M
\end{equation}
and
\begin{equation}
    \varphi\colon A\subset \eR^{n-1}\to S.
\end{equation}
We also consider a differential form \( \omega\in \Wedge^n(T^*M)\) and \( \sigma\in\Wedge^{n-1}(T^*M)\). These are respectively \( n\) and \( n-1\) differential forms on \( M\).  We also consider \( v\), a vector field on \( M\) and \( \tau\), a \( 1\)-form on \(M\).

Let us see what is possible to integrate.

\begin{definition}[\cite{ooMLEZooCKxedX}]       \label{DEFooPDRCooPiBklC}
    Let \( \omega\) be a \( n\)-form defined on \( \phi(U)\) (vanishing everywhere else). Its integral is :
    \begin{equation}
        \int_{\phi(U)}\omega=\int_U\phi^*\omega.
    \end{equation}
    The last integral is an integral of type \( \int_{U}F(x_1,\ldots, x_n)dx_1\wedge\ldots \wedge dx_n\) on an open set in \( \eR^n\). That is definition \ref{DEFooEYRFooRQTmRF}.
\end{definition}

This definition is nothing if it depend on the parametrisation. The following proposition show slightly more than the independence.
\begin{proposition}[\cite{MonCerveau}]       \label{PROPooNJCLooMqeeeX}
    Let be the charts \( \phi\colon U\to M\) and \( \psi\colon V\to N\) and a map \( f\colon M\to N\). The whole is supposed to be minimal :
    \begin{equation}
        f\big( \phi(U) \big)=\psi(V).
    \end{equation}
    Then we have the ``change of variable'' formula :
    \begin{equation}
        \int_{\phi(U)}\omega=\int_{\psi(V)}(f^{-1})^*\omega.
    \end{equation}
\end{proposition}

\begin{proof}
    By definition \( \int_{\phi(U)}\omega=\int_U\phi^*\omega\) and we have the diffeomorphism 
    \begin{equation}
        \phi^{-1}\circ f^{-1}\circ \psi\colon V\to U,
    \end{equation}
    so that we can use the result of lemma \ref{LEMooNCYSooXtnCKq} :
    \begin{equation}
        \int_U\phi^*\omega=\int_{(\phi^{-1}\circ f^{-1}\circ \psi)^{-1}(U)}  (\phi^{-1}\circ f^{-1}\circ \psi)^*\phi^*\omega=\int_{(\psi^{-1}\circ f\circ \phi )U}\psi^*(f^{-1})^*\omega=\int_{\psi^{-1}(N)}\psi^*(f^{-1})^*\omega.
    \end{equation}
    The last integral is the definition of an integral on \( N\) :
    \begin{equation}
        \int_{\psi^{-1}(N)}\psi^*(f^{-1})^*\omega=\int_N(f^{-1})^*\omega.
    \end{equation}
\end{proof}

Here is the lemma that shows the independence of definition \ref{DEFooPDRCooPiBklC} with respect to the change of chart system.
\begin{lemma}       
    Let \( \varphi\colon V\to M\) be a chart such that \( \varphi(V)\cap \varphi(U)=N\) is not empty. We define \( U'=\phi^{-1}(N)\) and \( V'=\varphi^{-1}(N)\). Then
    \begin{equation}        \label{EQooLSZMooPcyMWN}
        \int_{\phi(U')}\omega=\int_{\varphi(V')}\omega.
    \end{equation}
\end{lemma}
This lemma allows us to write \( \int_N\omega\) the common value of both sides of \eqref{EQooLSZMooPcyMWN}.

\begin{proof}
    Taking \( f=\id\) and two charts for the same open set in \( M\) in proposition \ref{PROPooNJCLooMqeeeX} shows the result.
\end{proof}
 
%--------------------------------------------------------------------------------------------------------------------------- 
\subsection{On manifold that require a finite atlas}
%---------------------------------------------------------------------------------------------------------------------------

We restrict ourself to manifolds that accept a finite atlas.

\begin{definition}      \label{DEFooITDTooWwrPPr}
    If \( \omega\) is a \( n\)-form on \( M\) and if \( \{ f_{\alpha} \} \) is a partition of unity\footnote{See theorem \ref{THOooPCHDooITWKpC}.} subordinate to the finite atlas \( \{ U_{\alpha} \}\) then
    \begin{equation}
        \int_M\omega=\sum_{\alpha}\int_{\phi_{\alpha}(U_{\alpha})}f_{\alpha}\omega.
    \end{equation}
\end{definition}

We show that this definition does not depend on the choice of the partition of unity.
\begin{lemma}[\cite{MonCerveau}] \label{LEMooCMIZooHhHaHV}
    The definition \ref{DEFooITDTooWwrPPr} is independent of the choice of atlas and partition of unity.
\end{lemma}

\begin{proof}
    Let \(  \{ U_{\alpha},\phi_{\alpha},f_{\alpha} \}_{\alpha\in A}  \) and \( \{ V_i,\varphi_i,g_i \}_{i\in I}\) be two choices of atlas, charts and subordinate partition of unity. We have to show that
    \begin{equation}        \label{EQooPVQZooHvbioJ}
        \sum_{\alpha\in A}\int_{\phi_{\alpha}(U_{\alpha})}f_{\alpha}\omega=\sum_{i\in I}\int_{\varphi_i(V_i)}g_i\omega.
    \end{equation}
    Since \( \{ g_i \}\) is a partition of unity,
    \begin{equation}
            \spadesuit=\sum_{\alpha}\int_{\phi_{\alpha}(U_{\alpha})}f_{\alpha}\omega=\sum_{\alpha}\int_{\phi_{\alpha}(U_{\alpha})}\sum_ig_if_{\alpha}\omega.
    \end{equation}
    Since the atlas are finite, the sums are finite and can be permuted with the integral. Moreover the function \( g_if_{\alpha}\) is nonzero only on \( \phi_{\alpha}(U_{\alpha})\cap\varphi_i(V_i)\) so that the integral can be taken on \( \phi_{\alpha}(U_{\alpha})\), \( \phi_{\alpha}(U_{\alpha})\cap\varphi_i(V_i)\) or \( \varphi_i(V_i)\). We have
    \begin{subequations}
        \begin{align}
            \spadesuit=\sum_i\sum_{\alpha}\int_{\phi_{\alpha(U_{\alpha})}}g_if_{\alpha}\omega&=  \sum_i\sum_{\alpha}\int_{\phi_{\alpha(U_{\alpha})}\cap \varphi_i(V_i)}g_if_{\alpha}\omega\\
            &=  \sum_i\sum_{\alpha}\int_{\varphi_i(V_i)}g_if_{\alpha}\omega\\
            &= \sum_i\int_{\varphi_i(V_i)}g_i\sum_{\alpha}f_{\alpha}\omega\\
            &=\sum_i\int_{\varphi_i(V_i)}g_i\omega.
        \end{align}
    \end{subequations}
\end{proof}
The common values of both sides of \eqref{EQooPVQZooHvbioJ} is denoted by \( \int_M\omega\).

The following is not really a definition, but a particular case of \ref{DEFooPDRCooPiBklC}. The integral of a \( n-1\)-form on a \( (n-1)\)-submanifold is
\begin{equation}        \label{EQooYPOGooRYOXQe}
    \int_S\sigma=\int_A\varphi^*\sigma.
\end{equation}
Once again the last integral is an integral of a \( n-1\)-form on an open set in \( \eR^{n-1}\).

\begin{definition}[\cite{ooMLEZooCKxedX}]       \label{DEFooAXFXooWiMLKP}
    The integral of a \( 1\)-form on a \( n-1\) dimensional submanifold is :
    \begin{equation}
        \int_S\tau=\int_S\hodge\tau
    \end{equation}
    where \( \hodge\) is the Hodge dual defined by \ref{DEFooUOJQooSzKjNR}. 
\end{definition}
The last integral is the integral of a \( (n-1)\)-form on a \( (n-1)\)-submanifold, given by \eqref{EQooYPOGooRYOXQe}.

\begin{definition}      \label{DEFooAXZGooJairMQ}
    The integral of a vector field on a \( (n-1)\)-submanifold is :
    \begin{equation}
        \int_Sv=\int_Sv^{\flat}
    \end{equation}
    where \( v^{\flat}\) is the \( 1\)-form defined by the musical isomorphism \eqref{EQooBTWXooTqoNxa}.
\end{definition}

The following proposition provides a much more explicit formula for the integral of a vector field. 

\begin{proposition}     \label{PROPooETLZooAVsrwy}
    Let \( \varphi\colon A\subset \eR^{n-1}\to \eR^n\) be an hypersurface and \( X\) be a vector field in \( \eR^n\). Then
    \begin{subequations}
        \begin{align}
            \int_SX&=\int_A\det\big( X,\frac{ \partial \varphi }{ \partial y_1 },\ldots, \frac{ \partial \varphi }{ \partial y_{n-1} } \big)  \label{SUBEQooWJSPooImJjQN}\\
            &=\int_A X\cdot\det\begin{pmatrix}
                e_1    &   \ldots    &   e_n    \\
                &   \partial_{y_1}\varphi    &       \\
                &    \vdots   &       \\
                &   \partial_{y_{n-1}}\varphi    &   
            \end{pmatrix}\\
            &=\int_A X\cdot n
        \end{align}
    \end{subequations}
    where \( \{ y_1,\ldots, y_{n-1} \}\) are the coordinates on \( A\) and \( n\) is the normal vector to the parametrization.
\end{proposition}
Note : thanks to lemma \ref{LEMooCMIZooHhHaHV}, the value of \( n\) can depend on the choice of coordinates, but the integral will not depend.

\begin{proof}
    If \( X=\sum_{i=1}^nX_i\partial_i\), then \( X^{\flat}=\sum_{i}X_idx_i\) and its Hodge dual is
    \begin{equation}
        \sum_{i}(-1)^i dx_1\wedge\ldots\wedge\widehat{dx_i}\wedge\ldots\wedge dx_n
    \end{equation}
    where the hat denotes a factor that is not present. Using the definitions \ref{DEFooAXZGooJairMQ}, \ref{DEFooAXFXooWiMLKP} and \ref{DEFooPDRCooPiBklC} it remains to integrate
    \begin{equation}
        \int_A\sum_i(-1)^iX_i\varphi^*\big( dx_1\wedge\ldots\wedge\widehat{dx_i}\wedge\ldots\wedge dx_n \big).
    \end{equation}
    If \( u_1,\ldots, u_{n-1}\) are vectors on \( A\) (that is on \( T_xA\) where \( x\) is the integration variable) we have
    \begin{subequations}
        \begin{align}
            \varphi^*(dx_1\wedge\ldots\wedge \widehat{dx_i}\wedge\ldots\wedge dx_n)(u_1,\ldots, u_{n-1})&= (dx_1\wedge\ldots\wedge \widehat{dx_i}\wedge\ldots\wedge dx_n)(d\varphi u_1,\ldots, d\varphi u_{n-1})\\
            &=\det\big( \tau_id\varphi u_1,\ldots, \tau_id\varphi u_{n-1} \big)
        \end{align}
    \end{subequations}
    where we used the lemma \ref{LEMooICRXooFKPCRd}. 

    What lies in the integral is the \( (n-1)\) differential form 
    \begin{subequations}        \label{EQooEVAPooSbRfaj}
        \begin{align}
           (u_1,\ldots, u_{n-1})\mapsto \sum_{i}(-1)^iX_i&\det\big(    \tau_id\varphi u_1,\ldots, \tau_id\varphi u_{n-1}  \big)\\
            &=\det\big( X,d\varphi u_1,\ldots, d\varphi u_{n-1} \big).
        \end{align}
    \end{subequations}
    Since this is a \( (n-1)\) differential form over \( \eR^{n-1}\), this has to be proportional to \( dy_1\wedge\ldots dy_{n-1}\). The proportionality factor is found by applying \eqref{EQooEVAPooSbRfaj} to the basis \( \{ e_1,\ldots, e_n \}\). Since \( d\varphi(e_i)=\frac{ \partial \varphi }{ \partial y_i }\) we have the proportionality factor
    \begin{equation}
        \det\left( X,\frac{ \partial \varphi }{ \partial y_1 },\ldots, \frac{ \partial \varphi }{ \partial y_n } \right)
    \end{equation}
    and the integral to be computed is
    \begin{equation}
        \int_A\det\left( X,\frac{ \partial \varphi }{ \partial y_1 },\ldots, \frac{ \partial \varphi }{ \partial y_n } \right)dy_1\wedge\ldots\wedge dy_{n-1}=\int_A\det\left( X,\frac{ \partial \varphi }{ \partial y_1 },\ldots, \frac{ \partial \varphi }{ \partial y_n } \right).
    \end{equation}
    The formula \eqref{SUBEQooWJSPooImJjQN} is proven. The two others are application of lemma \ref{LEMooFRWKooVloCSM}.
\end{proof}

\begin{example}
    Let us make the example with \( n=3\). We have
    \begin{equation}
        \varphi^*(dx\wedge dy)(v_1,v_2)=(dx\wedge dy)(d\varphi v_1 , d\varphi v_2)=\det
        \begin{pmatrix}
            d\varphi(v_1)_x    &   d\varphi(v_2)_x    \\ 
            d\varphi(v_1)_y    &   d\varphi(v_2)_y    
        \end{pmatrix},
    \end{equation}
    and then
    \begin{equation}
        \sum_i(-1)^iX_i \varphi^*(   \Wedge_{k\neq i}dx_k    )(v_1,v_2)=\sum_i(-1)^iX_i
        \begin{pmatrix}
            d\varphi(v_1)_x    &   d\varphi(v_2)_x    \\ 
            d\varphi(v_1)_y    &   d\varphi(v_2)_y    
        \end{pmatrix}=
        \det\begin{pmatrix}
             X_1  &   d\varphi (v_1)_x    &   d\varphi(v_2)_x    \\
             X_2  &   d\varphi (v_1)_y    &   d\varphi(v_2)_y    \\
             X_3  &   d\varphi (v_1)_z    &   d\varphi(v_2)_z   
        \end{pmatrix}
    \end{equation}
\end{example}

%--------------------------------------------------------------------------------------------------------------------------- 
\subsection{Integrating by part}
%---------------------------------------------------------------------------------------------------------------------------

\begin{proposition}[\cite{MonCerveau}]
    Let \( \Omega\) be an open set in an manifold \( M\) of dimension \( n\) and \( \varphi\colon A\to \eR^n\) be a parametrisation of the boundary \( \partial\Omega\) with tangent vector field \( n\) (defined on \( \partial\Omega\)). Let \( u,v\in  C^{\infty}(M)\). Then we have
    \begin{equation}        \label{EQooQSMNooKHwbqp}
        \int_{\partial \Omega}uv\,n_j=\int_{\Omega}\frac{ \partial u }{ \partial x_j }v+\int_{\Omega}u\frac{ \partial v }{ \partial x_j }.
    \end{equation}
\end{proposition}

\begin{proof}
    We use the Stokes formula (theorem \ref{ThoATsPuzF}) on the \( (n-1)\)-form
    \begin{equation}
        \omega=uv\,dx_1\wedge\ldots\wedge\widehat{dx_j}\wedge\ldots\wedge dx_n,
    \end{equation}
    and we know from example \ref{EXooCIYIooFPMLMU} that \( \omega=(-1)^{j+1}\hodge dx_j\). On the other hand,
    \begin{equation}
        d\omega=\sum_k\frac{ \partial (uv) }{ \partial x_k }dx_j\wedge dx_1\wedge\ldots\wedge\widehat{dx_j}\wedge\ldots\wedge dx_n=\frac{ \partial (uv) }{ \partial x_j }(-1)^{j+1}dx_1\wedge\ldots\wedge dx_n.
    \end{equation}
    We can use the Stokes formula :
    \begin{equation}
        \int_{\partial \Omega} uv dx_1\wedge\ldots\wedge\widehat{dx_j}\wedge\ldots\wedge dx_n=  (-1)^{j+1} \int_{\Omega}\frac{ \partial (uv) }{ \partial x_j }.
    \end{equation}
    The left-hand side can be transformed as
    \begin{equation}
        \int_{\partial\Omega}\hodge(dx_j)=\int_{\partial\Omega}uv\partial_j=\int_{\partial\Omega}uv\,n_j
    \end{equation}
    where we used the definition \ref{DEFooAXFXooWiMLKP} and the proposition \ref{PROPooETLZooAVsrwy}.

    The coefficients \( (-1)^{j+1}\) simplify and the derivation of product produce the result.
\end{proof}

\begin{example}     \label{EXooWLUVooNamnKG}
    If we integrate by part the function \( u\frac{ \partial^2 v }{ \partial x_j^2 }\) we have
    \begin{equation}
        \int_{\omega}u\frac{ \partial^2 }{ \partial x_j^2 }=-\int_{\Omega}\frac{ \partial u }{ \partial x_j }\frac{ \partial v }{ \partial x_j }+\int_{\partial \Omega}u\frac{ \partial v }{ \partial x_j }n_j.
    \end{equation}
    Summing over \( j\) we have the interesting formula
    \begin{equation}        \label{EQooJLDTooIMtxEX}
        \int_{\Omega}u\Delta v=-\int_{\Omega}\nabla u\cdot\nabla v+\int_{\partial \Omega}u\frac{ \partial v }{ \partial n }
    \end{equation}
    where \( \Delta v=\sum_j\frac{ \partial^2v }{ \partial x_j^2 }\) and \( \frac{ \partial v }{ \partial n }\) is a notation for \( \nabla v\cdot n\).
\end{example}

%+++++++++++++++++++++++++++++++++++++++++++++++++++++++++++++++++++++++++++++++++++++++++++++++++++++++++++++++++++++++++++ 
\section{Lie derivative}
%+++++++++++++++++++++++++++++++++++++++++++++++++++++++++++++++++++++++++++++++++++++++++++++++++++++++++++++++++++++++++++

Consider $X\in\cvec(M)$ and $\alpha\in\Omega^p(M)$. Let $\dpt{\varphi_t}{M}{M}$ be the flow of $X$. The \defe{Lie derivative}{Lie!derivative!of a $p$-form} of $\alpha$ is
\begin{equation}\label{liesurforme}
         \mL_X\alpha=\lim_{t\to 0}\us{t}[(\varphi^*_t\alpha)-\alpha]=\dsdd{\varphi^*_t\alpha}{t}{0}.
\end{equation}
More explicitly, for $x\in M$ and $v\in T_xM$,
\[
             (\mL_X\alpha)_x(v)=\lim_{t\to 0}\us{t}\left[(\varphi_t^*\alpha)_x(v)-\alpha_x(v)\right]
\]
In the definition of the \defe{Lie derivative}{Lie!derivative!of a vector field} for a vector field, we need an extra minus sign:
\begin{equation}		\label{EqDefLieDerivativeVect}
            (\mL_XY)_x=\dsdd{\varphi_{-t*}Y_{\varphi_t(x)}}{t}{0}.
\end{equation}
Why a minus sign ? Because $Y_{\varphi_t(x)}\in T_{\varphi_t(x)}M$, but $\dpt{(d\varphi_{-t})_a}{T_aM}{T_{\varphi_{-t}(a)}M}$ so that, if we want, $\varphi_{-t*}Y_{\varphi_t(x)}$ to be a vector at $x$, we can't use $\varphi_{t*}$.

These two definitions can be embedded in only one. Let $X\in\cvec(M)$ and $\varphi_t$ its integral curve\footnote{\textit{i.e.} for all $x\in M$, $\varphi_0(x)=x$ and $\dsdd{\varphi_{u+t}(x)}{t}{0}=X_{\varphi_u(x)}$.}\index{integral!curve}. We know that $\varphi_{t*}$ is an isomorphism $\dpt{\varphi_{t*}}{T_{\varphi^{-1}(x)}M}{T_xM}$. It can be extended to an isomorphism of the tensor algebras at $\varphi^{-1}(x)$ and $x$. We note it $\tilde{\varphi}_t$. For all tensor field $K$ on $M$, we define
\[
            (\mL_XK)_x=\lim_{t\to 0}[K_x-(\tilde{\varphi_t}K)_x].
\]

On a Riemannian manifold $(M,g)$, a vector field $X$ is a \defe{\href{http://en.wikipedia.org/wiki/Killing_vector_fields}{Killing vector field}}{killing!vector field} if $\mL_Xg=0$.



\begin{lemma}
Let $\dpt{f}{(-\epsilon,\epsilon)\times M}{\eR}$ be a differentiable map with $f(0,p)=0$ for all $p\in U$. Then there exists $\dpt{g}{(-\epsilon,\epsilon)\times M}{\eR}$, a differentiable map such that $f(t,p)=tg(t,p)$ and
\[
                g(0,q)=\left.\dsd{f(t,q)}{t}\right|_{t=0}.
\]
\end{lemma}
\begin{proof}
Take
\[
                g(t,q)=\int_0^1\dsd{f(ts,p)}{(ts)}ds,
\]
and use the change of variable $s\to ts$.
\end{proof}

\begin{lemma}
If $\varphi_t$ is the integral curve of $X$, for all function $\dpt{f}{M}{\eR}$, there exists a map $g$, $g_t(p)=g(t,p)$ such that
$f\circ\varphi_t=f+tg_t$ and $g_0=Xf$.
\end{lemma}

\begin{proof}
Consider $\overline{f}(t,p)=f(\varphi_t(p))-f(p)$, and apply the lemma:
\[
          f\circ\varphi_t=tg_t(p)+f(p).
\]
Thus we have
\[
      Xf=\lim_{t\to 0}\us{t}[f(\varphi_t(p))-f(p)]=\lim_{t\to 0}g_t(p)=g_0(p).
\]
\end{proof}

One of the main properties of the Lie derivative is the following:
\begin{theorem}		\label{ThoLieDerrComm}
Let $X$, $Y\in\cvec(M)$ and $\varphi_t$ be the integral curve of $X$. Then
\[
         [X,Y]_p=\lim_{t\to 0}\us{t}[Y-d\varphi_tY](\varphi_t(p)),
\]
or
\[
          \mL_XY=[X,Y].
\]

\end{theorem}
\begin{proof}
Take $\dpt{f}{M}{\eR}$ and the function given by the lemma: $\dpt{g_t}{M}{\eR}$ such that $f\circ \varphi_t=f+tg_t$ and $g_0=Xf$. Then put $p(t)=\varphi_t^{-1}(p)$. The rest of the proof is a computation:
\[
            (\varphi_{t*}Y)_pf=Y(f\circ\varphi_t)_{p(t)}=(Yf)_{p(t)}+t(Yg_t)_{p(t)},
\]
so
\begin{equation}
\begin{split}
  \lim_{t\to 0}\us{t}[Y_p-(\varphi_{t*}Y)_p]f&=\lim_{t\to 0}\us{t}[(Yf)_p-(Yf)_{p(t)}]-\lim_{t\to 0}(Yg_t)_{p(t)}\\
                                         &=X_p(Yf)-Y_pg_0\\
                                         &=[X,Y]_pf.
\end{split}
\end{equation}

\end{proof}

A second important property is
\begin{theorem}
For any function $f\colon M\to V$,
\[
           \mL_Xf=Xf.
\]
\end{theorem}

\begin{proof}
If $X(t)$ is the path which defines the vector $X$, it is obvious that at $t=0$, $X(t)$ is an integral curve to $X$, so that we can take $X(t)$ instead of $\varphi_t$ in \eqref{liesurforme}. Therefore we have:
\begin{equation}
    \mL_Xf=\dsdd{\varphi_t^*f}{t}{0}
          =Xf
\end{equation}
by definition of the action of a vector on a function.
\end{proof}
