% This is part of (almost) Everything I know in mathematics
% Copyright (c) 2014
%   Laurent Claessens
% See the file fdl-1.3.txt for copying conditions.

References about gravitation and noncommutative geometry are \cite{ConnesMotives,Landi,ConnesNCG,itoNCG_Varilly}. About the spectral action, one can consult \cite{SpectralActPrinciple}, with a prediction for the Higgs mass! I also would like to give the reference \cite{MrTopos} arguing that \wikipedia{en}{Topos}{topos} are the right direction to deal with quantum gravity.

\section{Inner and outer automorphisms}

Let $M$ be a paracompact smooth manifold. When $\cA$ is any unital $*$-algebra (as $ C^{\infty}(M)$), we define $\Aut(\cA)$ as the set of maps $\alpha\colon \cA\to \cA$ such that
\begin{align}
\alpha(ab)&=\alpha(a)\alpha(b),
&\alpha(\cun)&=\cun
 & \alpha(a^*)&=\alpha(a)^*.
\end{align}

\begin{proposition}
The group $\Diff(M)$ of diffeomorphism of $M$ is diffeomorphic to the group $\Aut\big(  C^{\infty}(M) \big)$. The isomorphism is given by $\varphi\mapsto\alpha_{\varphi}$ with
\[
  \alpha_{\varphi}(f)(x)=f\big( \varphi^{-1}(x) \big).
\]
\end{proposition}
\begin{proof}
No proof.
\end{proof}

For each $u\in U(\cA)$, we define the \defe{inner automorphism}{inner!automorphism} associated with $u$ by
\begin{equation}
\alpha_u(a)=uau^*.
\end{equation}
It satisfies $\alpha_{u^*}\circ\alpha_{u}=\alpha_u\circ\alpha_{u^*}=\id$. We denote by $\Inn(\cA)$\nomenclature{$\Inn(\cA)$}{Group of inner automorphisms of an algebra} the subgroup of inner automorphisms:
\[
  \Inn(\cA)=\{ \alpha_u\tq u\in U(\cA) \}.
\]
One can show that $\Inn(\cA)$ is a normal subgroup of $\Aut(\cA)$: for every $u\in U(\cA)$ and $\varphi\in\Aut(\cA)$, the automorphism $\varphi^{-1}\circ\alpha_u\circ\varphi$ belongs to $\Inn(\cA)$. So we can define the group of \defe{outer automorphism}{outer automorphism} as\nomenclature{$\Out(\cA)$}{Group of outer automorphisms of an algebra}
\[
  \Out(\cA)=\Aut(\cA)/\Inn(\cA),
\]
and we have the short exact sequence
\begin{equation}
\xymatrix{%
   \cun_{\Aut(\cA)} \ar[r]	&	\Inn(\cA)\ar[r]	&	\Aut(\cA)\ar[r]	&	\Out(\cA)\ar[r]&\cun_{\Inn(\cA)}.
}
\end{equation}
In the case of a commutative algebra, $\Inn(\cA)$ reduces to identity, so that $\Aut(\cA)=\Out(\cA)$.


\begin{proposition}
In the particular case of $\cA= C^{\infty}(M)$ for a smooth manifold $M$, we have
\[
  \Aut\big(  C^{\infty}(M) \big)=\Out\big(  C^{\infty}(M) \big)=\Diff(M).
\]
\end{proposition}

\begin{proof}

By proposition~\ref{PropcomCstarDelCeqX}, every character of $ C^{\infty}(M)$ is of the form $\omega_x$ for a certain $x\in M$. If $\alpha$ is an automorphism of $ C^{\infty}(M)$, then $\alpha^{-1}(\omega_x)$ defined by
\[
  \alpha^{-1}(\omega_x)f=\omega_x\big( \alpha^{-1}(f) \big)
\]
is still a character. Thus, there exists a $y\in M$ such that $\alpha^{-1}(\omega_x)=\omega_y$. That correspondence provides a continuous bijection $\phi\colon M\to M$ defined by
\[
  \alpha^{-1}(\omega+x)=\omega_{\phi(x)}.
\]
Applying a function $f\in C^{\infty}(M)$ to that equality, we find
\[
  \alpha^{-1}(f)(x)=f\big( \phi(x) \big),
\]
or
\begin{equation}
\alpha(f)(x)=f\big( \phi^{-1}(x) \big).
\end{equation}
Using the chain rule on that expression and the fact that $\alpha$ and $f$ are smooth, we find that $\phi$ is itself smooth. What we found is that $\alpha\leftrightarrow \phi$ is a group isomorphism between $\Aut\big(  C^{\infty}(M) \big)$ and $\Diff(M)$.
\end{proof}

Since $\Out(\cA)$ is the group of diffeomorphism in the commutative case, we want the remaining ($\Inn(\cA)$ in the noncommutative case) to be the group of internal automorphisms of the theory, namely the gauge transformations.

Of course, taking a diffeomorphism in general changes the metric, so that we have to use a principle to choose the correct physical metric of the theory. That principle is the minimizing of an action, namely the Einstein-Hilbert action
\[
  S_{EH}=\int_Mr(x)\sqrt{g(x)}dx
\]
where $r$ is the scalar curvature up to some constant factor. In the case of a boson theory, the correct action reveals to be the Yang-Mills one
\[
  S_{YM}=\int F\star F
\]
where $F$ is a curvature form.

Let $(\cA,\hH,\pi,D,J)$ be a real spectral triple (the representation $\pi$ of $\cA$ on $\hH$ is explicitly written). If $\alpha_u$ is an element of $\Inn(\cA)$, we define a new representation of $\cA$ on $\hH$ by
\[
  \pi_u=\pi\circ\alpha_u,
\]
and we have the adapted Dirac operator
\begin{equation}
D_u=D+A+\epsilon JAJ^*
\end{equation}
where $A=u[D,u^*]$. That Dirac operator is ``adapted'' in the sense of the following lemma.
\begin{lemma}
For every unitary element $u$ of $\cA$, the spectral triples $(\cA,\hH,\pi,D,J)$ and $(\cA,\hH,\pi_u,D_u,J)$ are equivalent and the unitary operator which provides the equivalence is $U=uJuJ^*$ where
\[
  D_u=D+u[D,u^*]\pm Ju[D,u^*]J^*
\]
with the sign $\pm$ is taken as $-$ if and only if $n=1\mod 4$.
\end{lemma}
\begin{proof}
No proof.
\end{proof}
More generally a \defe{inner fluctuation}{inner!fluctuation of the metric} of the metric a transformation of the form
\[
  D\mapsto D_A=D+A+JAJ^*
\]
where $A$ is any gauge potential. The \defe{spectral action principle}{spectral!action principle} claims that the gravitation coupled with the inner degrees of freedom is given by the action
\begin{equation}	\label{EqBozSpectralAct}
S_B(D,A)=\tr_{\hH}\chi\left( \frac{ D_A^2 }{ \Lambda^2 } \right)
\end{equation}
where $\tr$ is the usual trace on the Hilbert space $\hH$, $\Lambda$ is a mass scale parameter and $\chi$ is a function which cuts the eigenvalues of $D_A^2$ larger than $\Lambda^2$. Here the index $B$ refers to ``boson''.

\begin{proposition}
The spectral action \eqref{EqBozSpectralAct} remains unchanged under inner fluctuations
\[
  A\mapsto A^u=uAu^*+u[D,u^*]
\]
for every $u\in\Inn(\cA)$.
\end{proposition}
\begin{proof}
No proof.
\end{proof}
That lemma has the consequence that $A+JAJ^*=0$, so that there are no inner fluctuation of the metric in the case of commutative triple. If $A=\sum_ja_j[D,b_j]$, using the fact that $JJ^*=\cun$ and $J[D,b_j]J^*=[D,Jb_jJ^*]$, we find
\[
  JAJ^*=\sum_jJa_jJ^*[D,Jb_jJ^*],
\]
and with the identification $JaJ^*=a_j^*$, we find $JAJ^*=-A^*$, so that the commutative case does not have inner fluctuations.

\begin{probleme}
	Here we have $JAJ^*$ acting on the right as $A^*$. Maybe because of something like $Ja=a^*$, so that
	\[
		JAJ^*a=JA(a^*)=J(Aa^*)=aA^*.
	\]
	This makes the action of $JAJ^*$ commute with the one of $B$ which is on the left.
\end{probleme}

We know that the right representation of $\cA$ on $\hH$ is defined by $\xi b=b^0\xi$. Now, for $u\in U(\cA)$ we define the \defe{adjoint representation}{adjoint!representation!on Hilbert space} of $u$ by\nomenclature{$\Ad(u)\xi$}{Adjoint representation of $U(\cA)$ over $\hH$}
\begin{equation}
\Ad(u)\xi=u\xi u^*
\end{equation}

\begin{proposition}
Let $(\cA,\hH,D,J)$ be a real spectral triple. For all gauge potential $A\in\Omega^1_D(\cA)$ and for all $u\in U(\cA)$, we have
\begin{equation}
\Ad(u)(D+A+\epsilon'JAJ^{-1})\Ad(u^*)=D+\gamma_u(A)+\epsilon'J\gamma_u(A)J^{-1}
\end{equation}
where $\gamma_u(A)=u[D,u^*]+uAu^*$.
\end{proposition}
\begin{proof}
No proof.
\end{proof}

Moreover a inned fluctuation of an inner fluctuation is an inner fluctuation by the following proposition.
\begin{proposition}
Let $(\cA,\hH,D)$ be a spectral triple and $D_A=D+A$ for some potential vector $A$. Then for every $B\in\Omega^1_{D_A}(\cA)$ with $B=B^*$ (i.e. for every potential vector with respect to the new Dirac operator $D_A$) we have
\begin{equation}
  D_A+B=D+A'
\end{equation}
with $A'=A+B\in\Omega^1_D(\cA)$.
\end{proposition}
Remark that the inner fluctuation $D\mapsto D+A+\epsilon'JAJ^{-1}$ has to be seen as a two times process. First, we have the fluctuation $D\mapsto D+A$ which is a fluctuation relative to $\cA$ and then we have the fluctuation $D+A\mapsto D+A+\epsilon'JAJ^{-1}$ which is a fluctuation relative to the action of $\cA^0=J\cA J^*$. The result $D'=D+A+\epsilon' JAJ^{-1}$ is the most general operator such that $JD'J^{-1}=\epsilon D'$.

\section{Spectral action}
%---------------------------

One speaks about spectral action and its applications to physics in \cite{WhyStandard,SpectralActPrinciple,UnifActionFormula}

The only observables quantities in gravitation are the ones which are invariant under the full diffeomorphism group (that is the principle of gauge invariance). For example, if $K$ is the scalar curvature, and $F$ any function, the quantity
\[
  \int_M F(K)\sqrt{g}dx
\]
is an observable. That quantity is of course almost impossible to observe in real world experiments because it needs a knowledge of curvature on the hole space. That observable is highly non local.

The \defe{spectral action principle}{spectral!action principle} claims that the action to be used at the level of the functional intergral for quantum gravity after Wick rotation is given by
\begin{equation}
\tr f\left( \frac{ D }{ \Lambda } \right)
\end{equation}
where $D$ is the Dirac operator, $f$ is an even positive function and $\Lambda$ a real parameter.
\section{Gauge theory}
%------------------------

In order to build a gauge theory over $M$ for the group $G$, we begin by building a $G$-principal bundle over $M$ and we define $\Aut(P)$ as the group of automorphisms of $P$, i.e. the diffeomorphisms of $P$ which commute with the action of $G$. We consider $G_V$, the group of vertical automorphism of $P$. We have the short exact sequence
\begin{equation}
\xymatrix{%
   \mtu \ar[r] & G_V\ar[r]&\Aut(P)\ar[r]&	\Diff(M)\ar[r]&\mtu
}
\end{equation}
where the arrow $\alpha\colon \Aut(P)\to \Diff(M)$ is defined by $\varphi\mapsto\alpha_{\varphi}$,
\[
  \alpha_{\varphi}(x)=\pi\big( \varphi(\pi^{-1}x) \big)
\]
which is well defined because $\pi(\xi)=\pi(\xi')$ implies $\pi(\varphi \xi)=\pi(\varphi\xi')$. The map $\varphi$ belongs to the kernel of that arrow when $\alpha_{\varphi}(x)=x$ for every $x$. That map $\varphi$ must satisfies $\varphi(\pi^{-1}x)\in\pi^{-1}x$, so that $\varphi\in G_V$. Now the way to build a gauge theory for the group $G$ in the noncommutative geometry setting is as follows
\begin{enumerate}
\item we search for an algebra $\cA$ such that $\Inn(\cA)\simeq G_V$,
\item we build a spectral triple over $\cA$,
\item we compute the spectral action $S_B(D,A)$.
\end{enumerate}

Let $M$ be a Riemannian spin $4$-dimensional manifold and $(\cA,\hH,D)$ be its spectral triple.
\begin{proposition}
The spectral action
\[
  S_G(D,\Lambda)=\tr_{\hH}\chi\left( \frac{ D^2 }{ \Lambda^2 } \right))
\]
only depends on the spectrum of $D$.
\end{proposition}
\begin{proof}
No proof.
\end{proof}
We denote by $\Spec(M,D)$\nomenclature{$\Spec(M,D)$}{The spectrum of the Dirac operator over $M$} the spectrum of the Dirac operator with eigenvalues repeated as many times as their multiplicity. Two manifolds are said to be \defe{isospectral}{isospectral!manifold} when $\Spec(M,D)=\Spec(M',D')$. The proposition says that the spectral action is a spectral invariant. Notice that there exists isospectral manifold that are not isometric, so \emph{one cannot heard the shapes of a drum}. If $D$ and $D'$ are two isospectral Dirac operators corresponding to different metrics $g$ and $g'$, we have
\[
  S_B(D)=S_B(D'),
\]
but
\[
  S_{EH}(g)\neq S_{EH}(g').
\]
Thus the spectral invariance which leads to the spectral action $S_B$ is stronger than the usual diffeomorphism invariance which leads to the Einstein-Hilbert action $S_{EH}$.


\section{Scalar curvature}
%----------------------------

Let $(\cA,\hH,D)$ be a spectral triple of metric dimension $4$. We define the \defe{scalar curvature}{curvature!(scalar) of a spectral triple}\index{scalar!curvature of a spectral triple} as the function over $\cA$
\begin{equation}	\label{EqDefScalCurTriple}
\curR(a)=\dashint aD^{-2}.
\end{equation}
That definition is motivated by the following theorem.
\begin{theorem}
Let $(\cA,\hH,D)$ be the spectral triple of a $4$-dimensional spin Riemannian manifold. Then
\[
  \curR(f)=\frac{1}{ 24\pi^2 }\int_M f(x)s(x)\sqrt{g}dx
\]
where $s=-R$ is the scalar curvature.
\end{theorem}

Let us now study how does the curvature \eqref{EqDefScalCurTriple} changes under fluctuations of the form $D\mapsto D+A$. In the expansion of $\tr\big( f(D/\Lambda) \big)$, the coefficient of $\Lambda^2$ is given by $\dashint| D |^{-2}$, so that we have to look at the integral $\dashint(D+A)^2$.

\begin{proposition}
If we suppose that
\begin{itemize}
\item $\dashint AD^{-3}=0$ for all $A\in\Omega^1_D(\cA)$,
\item $\curR(a)=\dashint ads^2$ is a trace on $\cA$,
\end{itemize}
then
\begin{enumerate}
\item the term $\dashint (D+A)^{-2}$ is independent of $A$,
\item the functional $\curR$ is invariant under the inner fluctuation $D\mapsto D+A$.
\end{enumerate}
\end{proposition}



%+++++++++++++++++++++++++++++++++++++++++++++++++++++++++++++++++++++++++++++++++++++++++++++++++++++++++++++++++++++++++++
\section{Boundary conditions and heat kernel expansion}			\label{SecGravBoundCC}

%----------------------------------------------------------------------------------------------------------------------------
\subsection{Example}

As example of using the spectral action with heat kernel expansions and formulas \eqref{EqSeeyleydeWitt}, let us perform the ``warm up'' that Connes and Chamseddine propose in \cite{QGBoundaryTermsSpectralAction}, this is the usual Dirac operator
\begin{equation}		\label{EqFormgenDiracD}
D=\gamma^i(\partial_i+\omega_i)
\end{equation}
Let $(M,g)$ be a compact four dimensional Riemannian manifold and $V$, a vector bundle which carries a Clifford module structure, i.e. a map $\gamma\colon \Cliff(M)\to \End(V)$. If $\{ e_i \}$ is a local basis of $V$, we define $\gamma_i=\gamma(e_i)$ where we choose the basis $\{ e_i \}$ in such a way that $\gamma_i$ are constant, cf lemma~\ref{LemGammaBaseConstant}. In that case, we pose $\nabla_i=\partial_i+\omega_i$ and we have $D=\gamma^i\nabla_i$ as well as
\[
  D^2=\gamma^i\gamma^j\nabla_i\nabla_j.
\]
In order to make the link with the general Dirac type operator \eqref{EqFormGeneDirac}, we notice that what we do here is only to put ourself in a case where the element $r$ of $\Gamma(\End V)$ reads $\omega_i\gamma^i$ with $\omega_i\in C^{\infty}(M)$.

The computation of $[\nabla_i,\nabla_j]$ with $\nabla_i=\partial_i+\omega_i$ produces
\[
  [\nabla_i,\nabla_j]=(\partial_i\omega_j)-(\partial_j\omega_i)
\]
where $(\partial_i\omega_j)$ denotes the multiplication operator by the function $\partial_i\omega_j$. Thus we have
\begin{equation}
\begin{split}
D^2	&=\gamma^i\gamma^j\nabla+i\nabla_j\\
	&=(-2g^{ij}-\gamma^j\gamma^i)(\partial_i\omega_j-\partial_j\omega_i+\nabla_j\nabla_i)\\
	&=-2g^{ij}\nabla_j\nabla_i-\gamma^j\gamma^i(\partial_i\omega_j-\partial_j\omega_i)-\underbrace{\gamma^j\gamma^i\nabla_j\nabla_i}_{=D^2}.
\end{split}
\end{equation}
Thus we have
\begin{equation}
D^2=-g^{ij}\nabla_i\nabla_j-\frac{ 1 }{2}\gamma^i\gamma^j(\partial_i\omega_j-\partial_j\omega_i).
\end{equation}
This is a general form for the square of operator of the form \eqref{EqFormgenDiracD}. The operator we look at now is the one given by a form $\omega$ which satisfies the differential equation
\begin{equation}
\gamma^i\gamma^j(\partial_i\omega_j-\partial_j\omega_i)=\frac{ R }{ 2 },
\end{equation}
so that $D^2=-(g^{ij}\nabla_i\nabla_j-\frac{1}{ 4 }R)$, which immediately gives us $E=-\frac{1}{ 4 }R$ in the formulas \eqref{EqSeeyleydeWitt}. Using proposition~\ref{PropCondBordphiform} on our present Dirac operator (here, $\phi=0$) we find
\begin{equation}
	S=-\frac{1}{ 2 }\Pi_+\big(\gamma_n\gamma_a\nabla_a\chi \big)\Pi_+.
\end{equation}
We also have $\nabla=\nabla'$ and $\Phi=0$ as well as $\nabla'_a\chi=K_{ab}\chi\gamma^n\gamma^b$ where $K_{ab}$ is the \defe{extrinsic curvature}{extrinsic curvature}\index{curvature!extrinsic} defined by
\[
  K_{ab}=\Gamma^n_{ab}.
\]
We also define the scalar quantity $K=h^{ab}K_{ab}$ and it is easy to see that $\gamma^a\gamma^bK_{ab}=-K$. Using these formulas,
\[
\begin{split}
S	&=-\frac{ 1 }{2}\Pi_+(\gamma_n\gamma^a\nabla'_a\chi)\Pi_+
	=-\frac{ 1 }{2}(\gamma_n\gamma^aK_{ab}\chi\gamma^n\gamma^b)\Pi_+\\
	&=\frac{ 1 }{2}\Pi_+(\gamma_n\gamma^a K_{ab}\gamma^b\chi\gamma^n)\Pi_+
	=-\frac{ 1 }{2}\Pi_+(K\gamma_n\chi\gamma^n)\Pi_+\\
	&=\frac{ 1 }{2}\Pi_+K\gamma_n\gamma^n\chi
	=-\frac{ 1 }{2}\Pi_+K\chi\Pi_+,
\end{split}
\]
where we used the commutation relations $\gamma^n\gamma^b=-\gamma^b\gamma^n$, $\chi\gamma^b=\gamma^b\chi$, $\chi\gamma^n=\gamma^n\chi$ and $\gamma_n\gamma^n=-1$. Using now the definition $\Pi_+=\frac{1}{2}(\id+\chi)$ and the property $\chi^2=\id$, we conclude
\begin{equation}
S	=	-\frac{ 1 }{2}K\Pi_+.
\end{equation}

Since we are working on a four dimensional space, $\tr(\id)=4$, and $\tr(S)=-\frac{ 1 }{2}J\tr(\Pi_+)=-K$ because $\Pi_+$ is a projector on a space which has half of the dimension of the total space and $K$ is a scalar. For the same reason, $\tr(E)=-\frac{ 1 }{ 4 }\tr(R)=-\frac{ 1 }{ 4 }R\tr(\id)=-R$. The formulas \eqref{EqSeeyleydeWitt} reduce to
\begin{subequations}
\begin{align}
a_0(\Delta,\chi)	&=\frac{1}{ 4\pi^2 }\int_M\sqrt{g}d^4x,\\
a_2(\Delta,\chi)	&=-\frac{1}{ 48\pi^2 }\left( \int_M R\sqrt{g}d^4x + 2\int_{\partial M}K\sqrt{h}d^3x \right).
\end{align}
\end{subequations}

We are interested in the following case.

\begin{theorem}			\label{ThoExpActSpect}
Let $(\cA,\hH,D)$ be a spectral triple such that the expansion \eqref{EqDevHeatDsquare} holds. Then the spectral action $S=\tr\big( f(D/\Lambda) \big)$ can be expanded with respect to $\Lambda$ in the following way:
\begin{equation}
\tr\big( f(D/\Lambda) \big)\sim\sum_{\beta\in\Pi}f_{\beta}\Lambda^{\beta}\dashint| D |^{-\beta}+f(0)\zeta_D(0)+\ldots
\end{equation}
where the sum is taken over the dimension spectrum of the triple. The function $f_{\beta}$ is given by
\begin{equation}			\label{Eqfbetaintdonne}
  f_{\beta}=\int_0^{\infty}f(v)v^{\beta-1}dv.
\end{equation}
\end{theorem}

Notice that from the definition of the dimension spectrum, we have $\real(\beta)\geq 0$ and the negatives power involve all the Taylor expansion of $f$ at zero.

\begin{proof}
Let us act on a test function $k(u)$ that we expand as
\[
  k(u)=\int_0^{\infty} e^{-su}h(s)ds,
\]
and we formally write $k(t\Delta)=\int_)^{\infty} e^{-st\Delta}h(s)ds$, so that from the hypothesis we have
\begin{equation}
\tr\big( k(t\Delta) \big)\sim \sum_{alpha}t^{\alpha}\int_0^\infty s^{\alpha}h(s)ds.
\end{equation}
When $\alpha<0$, we have the formula
\[
  s^{\alpha}=\frac{1}{ \Gamma(-\alpha) }\int_0^{\infty} e^{-sv}v^{-\alpha-1}dv,
\]
so that
\begin{align*}
\int_0^{\infty}s^{\alpha}h(s)ds	=\frac{1}{ \Gamma(-\alpha) }\int_0^{\infty}\int_0^{\infty} e^{-sv}v^{-\alpha-1}dv\,h(s)ds
				=\frac{1}{ \Gamma(-\alpha) }\int_0^{\infty}k(v)v^{-\alpha-1}dv.
\end{align*}
That shows the asymptotic expansion
\begin{equation}		\label{EqTrktDeldv}
\tr\big( k(t\Delta) \big)\sim \sum_{\alpha}a_{\alpha}t^{\alpha}\frac{1}{ \Gamma(-\alpha) }\int_0^{\infty}k(v)v^{-\alpha-1}dv.
\end{equation}
By result \eqref{LemiizetaDzero}, the term $\alpha=0$ reduces to
\[
  \zeta_D(0)\int_0^{\infty}k(v)v^{-1}dv.
\]
\begin{probleme}
Cela devrait être $\zeta_D(0)f(0)$, mais je ne vois pas trop comment \ldots
\end{probleme}
Using now the fact that $2 a_{\alpha}/\Gamma(-\alpha)=\Res_{s=-2\alpha}\zeta_D(s)$, expression \eqref{EqTrktDeldv} becomes
\begin{equation}		\label{EqinterExptrbiftD}
  \tr\big( k(t\Delta) \big)\sim \sum_{\alpha}t^{\alpha}\frac{ 1 }{2}\Res_{s=-2\alpha}\zeta_D(s)\int_0^{\infty}k(v)v^{-\alpha-1}dv.
\end{equation}
Let us study the terms with $\alpha>0$ in the expansion \eqref{EqinterExptrbiftD}. The summation set in the expansion \eqref{EqDevHeatDsquare} can be very big, but equation \eqref{EqinterExptrbiftD} shows that only the $\alpha$ on which we have pole are to be taken into account, so from definition~\ref{DefDimSpec} of dimension spectrum, the sum over $\alpha$ reduces to a sum over $\alpha\in\Pi$.

If we pose $s'=s+2\alpha$, we have
\[
  \Res_{s=-2\alpha}\tr\big( | D |^{-s} \big)=\Res_{s'=0}\tr\left( | D |^{2\alpha}| D |-s' \right)=\dashint| D |^{2\alpha}
\]
If we pose $\beta=-2\alpha$, we find $\dashint| D |^{-\beta}$.

\end{proof}

Let us summarize what we obtained up to here. Let $D$ be an elliptic operator on $M$ and suppose $\partial M=\emptyset$. If we pose $\Delta=D^2$, the Mellin transform\index{Mellin transform} provides the following relation between $\tr(\Delta^{-z})$ (with $z\in\eC$) and $\tr( e^{-t\Delta})$:
\begin{equation}		\label{EaIntDevuspi}
\tr(\Delta^{-z})\Gamma(z)=\frac{1}{ \sqrt{\pi} }\int_0^{\infty}\tr( e^{-t\Delta})t^{z-1}dt.
\end{equation}
We know an heat kernel expansion for $\tr( e^{-t\Delta})$ and we know, from theorem~\ref{ThoExpActSpect} (expression \eqref{Eqfbetaintdonne}), the values of $a_{\alpha}$ when $\tr( e^{-tD^2})\sim\sum_{\alpha}a_{\alpha}t^{\alpha}$. So we know what to put into the integral~\ref{EaIntDevuspi}.

%++++++++++++++++++++++++++++++++++++++++++++++++++++++++++++++++++++++++++++++++++++++++++++++++++++++++++++++++++++++++++++
\section{Boundary}

The spectral action principle is given by $S=\tr\big( f(D/\Lambda) \big)$ with an even function $f\colon \eR\to \eR$ such that $f(0)=1$. For practical computations, the simplest is to take $f(x)= e^{-x^2}$ and to male an heat kernel expansion:
\[
  \tr\big( f(D/\Lambda) \big)\sim \Lambda^{(n/2)}Vol(M)+\Lambda^{(n/2)-1}\int_Mr(x)dx+\Lambda^{(n/2)-2}\ldots
\]
If the manifold $M$ has boundary, we pose the \defe{elliptic conditions}{elliptic!boundary condition}
\begin{equation}
\dom(D)=\{ s\tq Ts|_{\partial M}=s|_{\partial M} \}
\end{equation}
where $T=\gamma c(n)$. Here $\gamma$ is the parity and $c(n)$ is the multiplication by the normal vector to the boundary. In particular $T^2=\id$. When such conditions are imposed, one still have an heat kernel expansion under the form
\begin{align}
\tr\big( f(D/\Lambda) \big)&\sim \Lambda^{(n/2)}Vol(M)\\
				&+\Lambda^{(n-1)/2}\cdot 0&\textrm{this term is zero for some reasons}\\
				&+\Lambda^{(n/2)/2}\int_M \textrm{scalar curvature}\\
				&+\Lambda^{(n-3)/2}\int_{\partial M}\textrm{mean curvature}.
\end{align}
One can prove that the two latter terms are divergent on an asymptotically flat space.
