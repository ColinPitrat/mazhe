% This is part of (almost) Everything I know in mathematics and physics
% Copyright (c) 2013-2014,2016
%   Laurent Claessens
% See the file fdl-1.3.txt for copying conditions.

%+++++++++++++++++++++++++++++++++++++++++++++++++++++++++++++++++++++++++++++++++++++++++++++++++++++++++++++++++++++++++++
\section{BTZ from the structure of \texorpdfstring{$ \so(2,n)$}{so(2,n)}}
%+++++++++++++++++++++++++++++++++++++++++++++++++++++++++++++++++++++++++++++++++++++++++++++++++++++++++++++++++++++++++++

\begin{abstract}
	In this section, we study the relevant structure of the algebra $\so(2,n)$ which makes the BTZ black hole possible in the anti de Sitter space $AdS=SO(2,n)/SO(1,n)$. We pay a particular attention to the reductive Lie algebra structures of $\so(2,n)$ and we study how this structure evolves when one increases the dimension.

	We define the singularity as the closed orbits of the Iwasawa subgroup of the isometry group of anti de Sitter, but we insist on an alternative (closely related to the original conception of the BTZ black hole) way to describe the singularity as the loci where the norm of fundamental vector field vanishes. We provide a manageable Lie algebra oriented formula which describes the singularity and we use it in order to derive the existence of a black hole and to give a geometric description of the horizon.

\end{abstract}


\section{Introduction}
\label{LONGSecSumStructExist}

\subsection{Anti de Sitter space and the BTZ black hole}

The anti de Sitter space (hereafter abbreviated by $AdS$ or $AdS_l$) is a static solution to the Einstein's equations that describes a universe without mass. It is widely studied in different context in mathematics as well as in physics.

The BTZ black hole in $AdS$, initially introduced in \cite{BTZ_un,BTZ_deux} and then described and extended in various ways \cite{HolstPeldan,Aminneborg,Madden}, is an example of black hole structure which does not derives from a metric singularity, but from a causal issue.

The point of view we consider here grown from the papers \cite{BTZB_deux,Keio} in the case of $AdS_3$ and insists on the homogeneous space structure and the action of Iwasawa groups, in particular the singularity was described by looking at the closed orbits of the action of the Iwasawa group of $SO(2,2)$ on $AdS_3$. A dimensional generalization was then performed in \cite{lcTNAdS} (see also \cite{These} and the references therein for for a longer review).

One of the motivation in going that way is to embed the study of BTZ black hole into the noncommutative geometry and singleton physics \cite{BTZ_WZW,articleBVCS} since the deformation and the black hole are defined in a compatible way. Indeed the deformation is performed using the action of the same group as the groups which produce the singularity.

%
\subsection{The way we describe the BTZ black hole}
%

The anti de Sitter space $AdS_l$ is the surface in $\eR^{l+1}$ described by the equation
\begin{equation}
	AdS_l\equiv t^2+u^2-x_1^2-\ldots-x_{l-1}^2=1.
\end{equation}
This space can be investigated in the framework of homogeneous spaces since it reads as the following quotient of groups:
\begin{equation}
	AdS_l=\frac{ \SO(2,l-1) }{ \SO(1,l-1) }=G/H.
\end{equation}
We denote by $\sG=\so(2,l-1)$ and $\sH=\so(1,l-1)$ the Lie algebras and by $\pi$ the projection $G\to G/H$. The class of $g$ will be written $[g]$ or $\pi(g)$. We choose an involutive automorphism $\sigma\colon \sG\to \sG$ which fixes elements of $\sH$, and we call $\sQ$ the eigenspace of eigenvalue $-1$ of $\sigma$. Thus we have the reductive decomposition
\begin{equation}		\label{LONGEqIntroRedDecompHQLieAlg}
	\sG=\sH\oplus\sQ.
\end{equation}
The compact part of $\SO(2,l-1)$ decomposes into $K=\SO(2)\times\SO(l-1)$. We denote by $\sK$ the Lie algebra of $K$. Let $\theta$ be a Cartan involution which commutes with $\sigma$, and consider the corresponding Cartan decomposition
\begin{equation}        \label{LONGEqIntroRedDecompKPLieAlg}
	\sG=\sK\oplus\sP,
\end{equation}
where $\sK$ is the $+1$ eigenspace of $\theta$ and $\sP$ is the $-1$ eigenspace. A maximal abelian algebra $\sA$ in $\sP$ has dimension two and one can choose a basis $\{ J_1,J_2 \}$ of $\sA$ in such a way that $J_1\in\sH$ and $J_2\in\sQ$.

Now we consider an Iwasawa decomposition
\begin{equation}
	\sG=\sK\oplus\sA\oplus\sN,
\end{equation}
where $\sK$ and $\sA$ are the compact and abelian parts while $\sN$ is the nilpotent part corresponding to a choice of positivity on the set of roots. We denote by $\sR$ the Iwasawa component $\sR=\sA\oplus\sN$. We are also going to use the algebra $\bar\sN=\theta\sN$ and the corresponding Iwasawa component $\bar\sR=\sA\oplus\bar\sN$.

The Iwasawa groups $R=AN$ and $\bar R=A\bar N$ are naturally acting on anti de Sitter by $r[g]=[rg]$. It turns out that each of these two action has exactly two closed orbits, regardless to the dimension we are looking at. The first one is the orbit of the identity and the second one is the orbit of $[k_{\theta}]$ where $k_{\theta}$ is the element which generates the Cartan involution at the group level: $k_{\theta}gk_{\theta}^{-1}=\theta(g)$ ($g\in G$). In a suitable choice of matrix representation, the element $k_{\theta}$ is the block-diagonal element which is $-\mtu$ on $\SO(2)$ and $\mtu$ on $\SO(l-1)$. The $A\bar N$-orbits of $\mtu$ and $k_{\theta}$ are also closed. Moreover we have
\begin{equation}
	\begin{aligned}[]
		[A\bar N k_{\theta}]&=[k_{\theta}AN]\\
		[AN k_{\theta}]&=[k_{\theta}A\bar N]
	\end{aligned}
\end{equation}
because $A$ is invariant under the adjoint action of $k_{\theta}$ and, by definition, $\theta(N)=\bar N$. We define as \defe{singular}{Singular point} the points of the closed orbits of $AN$ and $A\bar N$ in $AdS$.

The Killing form of $\SO(2,l-1)$ induces a Lorentzian metric on $AdS$. The sign of the squared norm of a vector thus divides the vectors into three classes:
\begin{equation}
	\begin{aligned}[]
		\| X \|^2&>0&\rightarrow&&\text{time like,}\\
		\| X \|^2&<0&\rightarrow&&\text{space like,}\\
		\| X \|^2&=0&\rightarrow&&\text{light like.}
	\end{aligned}
\end{equation}
A geodesic is time (resp. space, light) like if its tangent vector is time like (resp. space, light).

If $E_1$ is a nilpotent element in $\sQ$, then every nilpotent in $\sQ$ are given by $\{ \Ad(k)E_1 \}_{k\in\SO(l-1)}$. These elements are also all the light like vectors at the base point. A light like geodesic trough the point $\pi(g)$ in the direction $\Ad(k)E_1$ is given by
\begin{equation}
	\pi(g e^{s\Ad(k)E_1}).
\end{equation}
One says that  points with $s>0$ are in the \defe{future}{Future} of $\pi(g)$ while points with $s<0$ are in the \defe{past}{Past} of $\pi(g)$.

We say that a point in $AdS_l$ belongs to the \defe{black hole}{Black hole} if every light like geodesics trough that point intersect the singularity in the future. We call \defe{horizon}{Horizon} the boundary of the set of points in the black hole. One says that there is a (non trivial) black hole structure when the horizon is non empty or, equivalently, when there are some points in the black hole, and some outside.

All these properties can be easily checked using the matrices given in \cite{These,lcTNAdS}. In this optic, I wrote a program using Sage\cite{Sage} which checks all the properties that are shown in this paper. It will be published soon.

As far as notations are concerned, we denote by $X_{\alpha\beta}$ the basis of $\sN$ and $\bar\sN$ corresponding to our choice of Iwasawa decomposition. That basis is chosen in such a way that $\ad(J_1)X_{\alpha\beta}=\alpha X_{\alpha\beta}$ and $\ad(J_2)X_{\alpha\beta}=\beta X_{\alpha\beta}$.

%
\subsection{Organization of the paper}
%

The main goal of this paper is to reorganize all this structure in a coherent way. Then we use it efficiently in order to define the singularity of the BTZ black hole, to prove that one has a genuine black hole in every dimension, and to determine the horizon.

In section~\ref{LONGSecProgressRidMatrices}, we list the commutators of $\so(2,n)$ with respect to its root spaces and we organize them in such a way to get a clear idea about the evolution of the structure when the dimension increases. We prove that, when one passes from $\so(2,n)$ to $\so(2,n+1)$, one gets four new vectors in the root spaces and that these are Killing-orthogonal to the vectors existing in $\so(2,n)$ (this is the ``dimensional slice'' described in subsection~\ref{LONGSubSecDimensionalSlices}).

We give in subsection~\ref{LONGSubSecReductiveDecompQ} an original way to describe the space $\sQ$ without reference to $\sH$. The space $\sQ$ is usually described as a complementary of $\sH$. Here we show that it can be described by means of the root spaces and the Cartan involution $\theta$. The space $\sH$ is then described as $\sH=[\sQ,\sQ]$. In some sense, we describe the quotient space $AdS=G/H$ directly by its tangent space $\sQ$ without passing trough the definition of $H$. Of course, the knowledge of $\sH$ will be of crucial importance later.


The subsection~\ref{LONGSubSecPropRedDecompQH} is devoted to the proof of many properties of the decompositions $\sG=\sH\oplus\sQ$ and $\sG=\sK\oplus\sP$.


The first important result is the proposition~\ref{LONGXUnALaTwistingSuperCool} that shows that the elements of $\sQ$ are $\ad$-conjugate to each others: there exist elements of the adjoint group which are intertwining the elements of $\sQ$.

Using that property, we compute the norm of these elements and we identify the nilpotent vectors in $\sQ$ (these are the light-like vectors). In the same time, we prove that the space $G/H$ is Lorentzian and we provide an orthogonal basis of $\sQ$.

The second central result is the fact that nilpotent elements in $\sQ$ are of order two: if $E\in\sQ$ is nilpotent, then $\ad(E)^3=0$. That result will be used in a crucial way in the proof of the black hole existence, as well as in the study of its properties.

In section~\ref{LONGSecBlacHole}, we define and study the structure of the BTZ black hole in the anti de Sitter space. First we identify the closed orbits of the Iwasawa group (theorem~\ref{LONGThoOrbitesOuverttes}) and we define them as singular. In a second time, we provide an alternative description the singularity: theorem~\ref{LONGThosSequivJzero} shows that the singularity can be described as the loci of points at which a fundamental vector field has vanishing norm. We also provide in lemma~\ref{LONGLemExpressionCoolNormJUn} a convenient way to compute that norm on arbitrary point of the space.

We prove, in section~\ref{LONGSubSecExistenceTrouNoir}, that our definition of singularity gives rise to a genuine black hole in the sense that there exists points from which some geodesics escape the singularity in the future and there exists some points from which all the geodesics are intersecting the singularity in the future.

In section~\ref{LONGSecHorizonSansMatrices}, we provide a geometric description of the horizon (theorem~\ref{LONGThoCausalPasseParR}). We show that, if we see $AdS_l$ as a subset of $AdS_{l+1}$, the space $AdS_{l+1}$ is generated by the action on $AdS_l$ by a one parameter group which leaves the singularity invariant. This action thus leaves invariant the whole causal structure and we are able to express the horizon in any dimension from the well known horizon in $AdS_3$.
%
\section{Structure of the algebra}
%
\label{LONGSecProgressRidMatrices}

%
\subsection{The Iwasawa component}
%

Our study of $AdS_l=\SO(2,l-1)/\SO(1,l-1)$ will be based on the properties of the algebra $\sG=\so(2,l-1)$ endowed with a Cartan involution $\theta$ and an Iwasawa decomposition $\sG=\sA\oplus\sN\oplus\sK$. In this section we want to underline the most relevant facts for our purpose. The part we are mainly interested in is the Iwasawa component $\sA\oplus\sN$ where
\begin{subequations}		\label{LONGEqLeANEnDimAlg}
\begin{align}
	\sN&=\{ X^{k}_{+0},X^{k}_{0+},X_{++},X_{+-} \}\\
	\sA&=\{ J_1, J_2\},
\end{align}
\end{subequations}
where $k$ runs %
\footnote{The ``new'' vectors which appear in $AdS_l$ with respect to $AdS_{l-1}$ are $X_{0\pm}^{l-1}$ and $X_{\pm 0}^{l-1}$. Such an element appears for the first time in $AdS_4$ and is not present when one study $AdS_3$.} %
from $3$ to $l-1$. The commutator table is
\begin{subequations}  \label{LONGEqTableSOIwa}
	\begin{align}
		[X_{0+}^{k},X_{+0}^{k'}]&=\delta_{kk'}X_{++}		&[X_{0+}^{k},X_{+-}]&=2X_{+0}^{k}\\
		[ J_1,X_{+0}^{k}]&=X_{+0}^{k}				&[ J_2,X_{0+}^{k}]&=X_{0+}^{k}\\
		[ J_1,X_{+-}]&=X_{+-}					&[ J_2,X_{+-}]&=-X_{+-}\\
		[ J_1,X_{++}]&=X_{++}					&[ J_2,X_{++}]&=X_{++}.
	\end{align}
\end{subequations}
We see that the Iwasawa algebra belongs to the class of $j$-algebras whose Pyatetskii-Shapiro decomposition is
\begin{equation}
	\sA\oplus\sN=(\sA_1\oplus_{\ad}\sZ_1)\oplus_{\ad}\big( \sA_2\oplus_{\ad}(V\oplus \sZ_2) \big),
\end{equation}
with
\begin{subequations}
	\begin{align}
		\sA_1&=\langle H_1\rangle		&\sA_2&=\langle H_2\rangle\\
		\sZ_1&=\langle X_{+-}\rangle		&\sZ_2&=\langle X_{++}\rangle\\
		&					&V&=\langle X_{0+}^{k},X_{+0}^{k}\rangle_{k\geq 3}
	\end{align}
\end{subequations}
where
\begin{equation}
	\begin{aligned}[]
		H_1&=J_1-J_2\\
		H_2&=J_1+J_2.
	\end{aligned}
\end{equation}

The general commutators of such an algebra are
\begin{subequations}		\label{LONGsubEqsGenPySO}
\begin{align}
	[H_1,X_{+-}]		&=2X_{+-}	&	[H_2,X_{0+}^{k}]	&=X_{0+}^{k}				&[H_1,V]	&\subset V	\\
				&		&	[H_2,X_{+0}^{k}]	&=X_{+0}^{k}				&[X_{+-},V]	&\subset V	\\
				&		&	[H_2,X_{++}]		&=2X_{++}								\\
				&		&	[X_{0+}^k,X_{+0}^l]	&=\Omega(X_{0+}^k,X_{+0}^l)X_{++}
\end{align}
\end{subequations}
In the case of $\so(2,n)$, we have the following more precise relations:
\begin{subequations}		\label{LONGSubEqsPlusPresPySO}
	\begin{align}
		[H_1,X^k_{0+}]&=-X^k_{0+}\\
		[X_{+-},X^k_{0+}]&=-2X^k_{+0}
	\end{align}
\end{subequations}
and the link between $\sN$ and $\tilde\sN$ is given by
\begin{subequations}		\label{LONGSubEqsThethaPySO}
	\begin{align}
		[\theta X_{+0}^{k},X_{++}]	&=2X_{0+}^{k}\\
		[\theta X_{0+}^{k},X^k_{0+}]	&=2J_2\\
		[\theta X_{++},X_{++}]		&=4H_2=4(J_1+J_2)\\
		[\theta X_{++},X^{k}_{0+}]	&=2X_{-0}^{k}\\
		[\theta X_{+-},X_{+-}]		&=4H_1=4(J_1-J_2)\\
		[\theta X_{+-},X^{k}_{+0}]	&=2X_{0+}^{k}.
	\end{align}
\end{subequations}
The relations between the higher dimensional root spaces are
\begin{equation}
	\begin{aligned}[]
		[X_{0+}^i,X_{-0}^j]&=-\delta_{ij}X_{-+}\\
		[X_{0+}^i,X_{+0}^j]&=\delta_{ij}X_{++}\\
		[X_{+0}^i,X_{0+}^j]&=-\delta_{ij}X_{++}\\
		[X_{+0}^i,X_{0-}^j]&=\delta_{ij}X_{+-}.
	\end{aligned}
\end{equation}
%
%
The space $\mZ_{\sK}(\sA)$ of the elements of $\sK$ which commute with all the elements of $\sA$ is given by the elements $r_{ij}=\frac{ 1 }{2}[X_{0+}^i,X_{0+}^j]$ for every $i,j\geq 3$, $i\neq j$ realise the $\so(l-1)$ algebra. We will say more about them in subsection~\ref{LONGSubSec_Thecompactpart}.


We deduce the following relations that will prove useful later
\begin{equation}
	\begin{aligned}[]
		[\theta X_{0+}^k,X_{++}]&=-2X^k_{+0}\\
		[\theta X_{+0}^k,X_{+-}]&=-2X^k_{0-}\\
		[\theta X_{++},X_{+0}^k]&=-2X^k_{0-}\\
		[X_{-+},X^k_{0-}]&=-2X^k_{-0}.
	\end{aligned}
\end{equation}


%
%
\subsection{The compact part}
%
\label{LONGSubSec_Thecompactpart}

The compact part of $\so(2,l-1)$, the algebra $\so(2)\oplus\so(l-1)$ is well known. What is interesting from our point of view is to write the commutation relations between the elements of $\so(l-1)$ and the roots.

We define the following elements that are non vanishing:
\begin{equation}
	[X_{0+}^i,X_{0-}^j]=[X_{0-}^i,X_{0+}^j]=2r_{ij}.
\end{equation}
One immediately has $\theta r_{ij}=r_{ij}$, so that $r_{ij}\in\sK$. We also have $r_{ij}\in\sG_0$ so that $r_{ij}\in\mZ_{\sK}(\sA)$ and they act on the root spaces. The action is given by
\begin{subequations}
	\begin{align}		\label{LONGEqComsRRN}
        [r_{ij},X_{+0}^k]&=0		&\text{if }i,j,k \text{ are different}\\
        [r_{ij},X_{0+}^k]&=0		&\text{if } i, j,k \text{ are different}\\
        [r_{ij},X_{+0}^j]&=X_{+0}^i	&\text{if }i\neq j\\
        [r_{ij},X_{0+}^j]&=-X_{0+}^i	&\text{if }i\neq j\\
		[r_{ij},X_{\pm\pm}]&=0.			\label{LONGsubeqrXpmpm}
	\end{align}
\end{subequations}
The elements $r_{ij}$ satisfy the algebra of $\so(n)$.

\begin{remark}
	If $\sigma$ is an involutive automorphism which commutes with $\theta$ and such that $\sigma J_1=J_1$, $\sigma J_2=-J_2$, then one has $\sigma r_{ij}=r_{ij}$. We will see later that this fact makes $r_{ij}\in\sH$.
\end{remark}

We know\footnote{See \cite{Berndt} for example.} that $\sG=\sG_0\oplus\sN\oplus\bar\sN$ where $\sG_0=\sA\oplus\mZ_{\sK}(\sA)$. Let us perform a dimension count in order to be sure that the vectors $r_{ij}$ generate $\mZ_{\sK}(\sA)$. When we are working with $AdS_l$, we have
\begin{equation}
	\begin{aligned}[]
		\dim(\sA)&=2\\
		\dim(\tilde\sN_2)&=4\\
		\dim(\bigoplus_{k}\tilde\sN_k)&=4(l-3)\\
		\dim(\langle r_{ij}\rangle)&=\frac{ 1 }{2}(l-4)(l-3).
	\end{aligned}
\end{equation}
The last line comes from the fact that we have the elements $r_{34}$, $r_{35}$,\ldots,$r_{45}$,\ldots The first such element appears in $AdS_5$. Making the sum, we obtain $\frac{ l(l+1) }{ 2 }$, which is the dimension of $\so(2,l-1)$. Thus we have
\begin{equation}
	\sG=\mZ_{\sK}(\sA)\oplus\sA\oplus\tilde\sN_2\oplus\bigoplus_{k}\tilde\sN_k
\end{equation}
where $\mZ_{\sK}(\sA)$ is generated by the elements $r_{ij}$.
%

%
\subsection{Dimensional slices}
%
\label{LONGSubSecDimensionalSlices}

Since the set $\mZ_{\sK}(\sA)$ of elements in $\sK$ that commute with elements of $\sA$ is a part of $\sH$ (see later), they will have almost no importance in the remaining\footnote{We will however need them in the computation of the coefficients \eqref{LONGEqCoefsabcBE}.}. The most important part of $\sG$ is
\begin{equation}		\label{LONGEqDecomDimQlipm}
    \sA\oplus\sN\oplus\bar\sN=\underbrace{\langle J_1,J_2,X_{\pm,\pm}\rangle}_{\text{for every dimension}}\oplus\underbrace{\langle X_{0\pm}^{4},X_{\pm 0}^{4}\rangle}_{\text{for }\so(2,\geq 3)}\oplus\ldots\oplus\underbrace{\langle X_{0\pm}^l,X_{\pm 0}^{l}\rangle}_{\text{for } \so(2, l-1)}.
\end{equation}
We use the following notations in order to make more clear how does the algebra evolve when one increases the dimension:
\begin{equation}
	\begin{aligned}[]
		\sN_2&=\langle X_{+-},X_{++}\rangle,		&\sN_k&=\langle X^k_{0+},X_{+0}^k\rangle	\\
		\bar\sN_2&=\langle X_{-+},X_{--} \rangle, 	&\bar\sN_k&=\langle X_{0-}^k,X_{-0}^k\rangle\\
		\tilde\sN_2&=\langle \sN_2,\bar\sN_2\rangle,	&\tilde\sN_k&=\langle \sN_k,\bar\sN_k\rangle
	\end{aligned}
\end{equation}
for $k\geq 3$. The relations are
\begin{equation}		\label{LONGEqsCommWithtsNDeuxkA}
	\begin{aligned}[]
		[\tilde\sN_2,\tilde\sN_2]&\subseteq\sA
		&[\tilde\sN_2,\tilde\sN_k]&\subseteq\tilde\sN_k\\
		[\tilde\sN_k,\tilde\sN_{k}]&\subseteq\sA\oplus\tilde\sN_2
		&[\tilde\sN_k,\tilde\sN_{k'}]&\subset \mZ_{\sK}(\sA) \\
	\end{aligned}
\end{equation}

As a consequence of the splitting and the commutation relations, we have many Killing-orthogonal subspaces in $\so(2,l-1)$:
\begin{subequations}
	\begin{align}
		\tilde\sN_k\perp\tilde\sN_{k'}		\label{LONGEqtsnkperprtsnkp}	\\
		\sA\perp\tilde\sN_2			\label{LONGEqAperpNTrois}		\\
		\sA\perp\tilde\sN_k.	\label{LONGEqAperpNk}				\\
		\tilde\sN_2\perp\tilde\sN_k	\label{LONGEqNTtroisperpNk}
	\end{align}
\end{subequations}

In order to check there relations, look at the trace of action of $\ad(X)\circ\ad(Y)$ on the various spaces:
\begin{equation}
	\ad(\sA)\circ\ad(\tilde\sN_2) \colon
	\begin{cases}
		\sA\to\tilde\sN_2\to\tilde\sN_2\\
		\tilde\sN_2\to\sA\to 0\\
		\tilde\sN_k\to\tilde\sN_k\to\tilde\sN_k\\
		\mZ_{\sK}(\sA)\to\tilde\sN_2\to\tilde\sN_2,
	\end{cases}
\end{equation}
and
\begin{equation}
	\ad(\sA)\circ\ad(\tilde\sN_k)\colon
	\begin{cases}
		\sA\to\tilde\sN_k\to\tilde\sN_k\\
		\tilde\sN_2\to\tilde\sN_k\to\tilde\sN_k\\
		\tilde\sN_k\to\sA\oplus\tilde\sN_2\to\tilde\sN_2\\
		\langle r_{kl}\rangle\to\tilde\sN_l\to\tilde\sN_l
	\end{cases}
\end{equation}
and
\begin{equation}
	\ad(\tilde\sN_2)\circ\ad(\tilde\sN_k)\colon
	\begin{cases}
		\sA\to\tilde\sN_k\to\tilde\sN_k\\
		\tilde\sN_2\to\tilde\sN_k\to\tilde\sN_k\\
		\tilde\sN_k\to\sA\oplus\tilde\sN_2\to\tilde\sN_2\oplus\sA\\
		\langle r_{kl}\rangle\to\tilde\sN_l\to\tilde\sN_l\oplus\sA
	\end{cases}
\end{equation}
and
\begin{equation}
	\ad(J_1)^2|_{\tilde\sN_2}=\ad(J_2)^2|_{\tilde\sN_2}=\id|_{\tilde\sN_2}.
\end{equation}
%
As a consequence,
\begin{equation}		\label{LONGEqDecompGPourKillOrtho}
	\sG=\mZ_{\sK}(\sA)\oplus\sA\oplus\tilde\sN_2\bigoplus_{k\geq 3}\tilde\sN_k
\end{equation}
is a Killing-orthogonal decomposition of $\so(2,l-1)$.

%
\subsection{Reductive decomposition}
%
\label{LONGSubSecReductiveDecompQ}

Related to the decompositions \eqref{LONGEqIntroRedDecompHQLieAlg} and \eqref{LONGEqIntroRedDecompKPLieAlg} we introduce the following notations: $X_{\sK}$, $X_{\sP}$ will denote the $\sK$ and $\sP$-components of the vector $X$ with respect to the decomposition $\sG=\sK\oplus\sP$. We define $X_{\sH}$ and $X_{\sQ}$ similarly with respect to the decomposition $\sG=\sH\oplus\sQ$. In the same way we define $X_{\sA}$, $X_{\sZ(\sK)}$ and so on by the Iwasawa decomposition
\begin{equation}
    \sG=\sA\oplus\sN\oplus\sK=\sA\oplus\sN\oplus\sZ(\sK)\oplus\so(l-1).
\end{equation}

Let $\sQ$ be the following vector subspace of $\sG$:
\begin{equation}		\label{LONGEqDecQEspacesCools}	%
	\sQ=\big\langle \mZ(\sK),J_2,[\mZ(\sK),J_1],(X_{0+}^k)_\sP\big\rangle_{k\geq 3}.
\end{equation}
Then we choose a subalgebra $\sH$ of $\sG$ which, as vector space, is a complementary of $\sQ$. In that choice, we require that there exists an involutive automorphism $\sigma\colon \sG\to \sG$ such that
\begin{equation}
	\sigma=(\id)_{\sH}\oplus(-\id)_{\sQ}.
\end{equation}
In that case the decomposition $\sG=\sH\oplus\sQ$ is reductive, i.e. $[\sQ,\sQ]\subset\sH$ and $[\sQ,\sH]\subset\sQ$.




From definition \eqref{LONGEqDecQEspacesCools}, it is immediately apparent that one has a basis of $\sQ$ made of elements in $\sK$ and $\sP$, so that one immediately has
\begin{equation}
	[\sigma,\theta]=0.
\end{equation}

If $X\in\sG$, the projections are given by
\begin{equation}		\label{LONGEqProjHQPKsigmatheta}
	\begin{aligned}[]
		X_{\sH}&=\frac{ 1 }{2}(X+\sigma X),		&X_{\sK}&=\frac{ 1 }{2}(X+\theta X),\\
		X_{\sQ}&=\frac{ 1 }{2}(X-\sigma X),		&X_{\sP}&=\frac{ 1 }{2}(X-\theta X).
	\end{aligned}
\end{equation}
In particular $\theta\sH\subset\sH$ since $\theta$ and $\sigma$ commute.

We introduce the following elements of $\sQ$:
\begin{equation}			\label{LONGAlignPRedDefQQ}
	\begin{aligned}[]
		q_0&=(X_{++})_{\sK\sQ}\\
		q_1&=J_2\\
		q_2&=-[J_1,q_0]=-(X_{++})_{\sP\sQ}\\
		q_k&=(X^k_{0+})_{\sP}.
	\end{aligned}
\end{equation}
In order to see that $[J_1,q_0]=(X_{++})_{\sP\sQ}$, just write the $\sP\sQ$-component of the equality $[J_1,X_{++}]=X_{++}$. We will prove later that this is a basis and that each of these elements correspond to one of the spaces listed in \eqref{LONGEqDecQEspacesCools}.

Since $X_{\sP}=(\theta X-X)/2$, we have
%
\begin{equation}
	[q_i,q_j]=-\frac{1}{ 4 }\big( [X_{0+}^i,X_{0-}^j]+[X_{0-}^i,X_{0+}^j] \big)=r_{ij}.
\end{equation}%
%
%

From equations \eqref{LONGEqAperpNTrois} and \eqref{LONGEqAperpNk}, we have $q_1\perp q_2$ and $q_2\perp q_k$. Using the other perpendicularity relations $\sK\perp\sP$ and \eqref{LONGEqtsnkperprtsnkp}, \eqref{LONGEqAperpNTrois}, \eqref{LONGEqAperpNk},
%
%
we see that the set $\{ q_i \}_{0\leq i\leq l-1}$ is orthogonal.

The space $\sH$ is defined as generated by the elements
\begin{equation}			\label{LONGAlignPremDefHH}
	\begin{aligned}[]
		J_1&		&r_k&=[J_2,q_k]			\\
		p_1&=[q_0,q_1]	&p_k&=[q_0,q_k]			\\
		s_1&=[J_1,p_1]	&s_k&=[J_1,p_k].
	\end{aligned}
\end{equation}
Elements \eqref{LONGAlignPRedDefQQ} and \eqref{LONGAlignPremDefHH} will be studied in great details later.%

%
\subsubsection{Remark on the compact part}
%
\label{LONGSubSubSecRemCompPart}

Elements of $\sK$ are elements of the form $X+\theta X$. A part of the elements inside $\mZ_{\sK}(\sA)$, these elements are of two kinds:
\begin{subequations}
	\begin{align}
		X_{++}+X_{--}\\
		X_{+-}+X_{-+}
	\end{align}
\end{subequations}
on the one hand, and
\begin{subequations}
	\begin{align}
		X^k_{0+}+X^k_{0-}\\
		X^k_{+0}+X^k_{-0}
	\end{align}
\end{subequations}
on the other hand. The first two are commuting, so that $\mZ(\sK)$ is two dimensional when one studies $AdS_3$. That correspond to the well known fact that the compact part of $\so(2,2)$ is $\so(2)\oplus\so(2)$ which is abelian. These elements, however, do not commute with the two other. For example, the combination
\begin{equation}
	\big( X_{++}+X_{--}\big)-\big(X_{+-}+X_{-+}\big)
\end{equation}
does not commute with the elements of the second type. Now, one checks that the combination
\begin{equation}		\label{LONGEqZKChoixblabl}
	\big( X_{++}+X_{--}\big)+\big( X_{+-}+X_{-+})
\end{equation}
commutes with all the other, so that it is the generator of $\mZ(\sK)$ for $AdS_{\geq 4}$. This corresponds to the fact that the compact part of $\so(2,n)$ is $\so(2)\oplus\so(n)$. In other terms,
\begin{equation}		\label{LONGEqDecompZKenDeuxSuivantDim}
	\mZ(\sK)=\langle X_{++}+X_{--}+X_{+-}+X_{-+}\rangle\oplus\underbrace{\langle X_{++}+X_{--}-X_{+-}-X_{-+}\rangle}_{\text{only for }AdS_3}.
\end{equation}


Notice that, for $AdS_{\geq 4}$, we can define $q_0=(X_{++})_{\mZ(\sK)}$ as $\sK=\so(2)\oplus\so(l-2)$ for $AdS_l$. The case of $AdS_3$ is particular because $\mZ(\sK)$ is of dimension two and we have to set by hand what part of $\mZ(\sK)$ belongs to $\sQ$ (the other part belongs to $\sH$). From what is said around equation \eqref{LONGEqZKChoixblabl}, we know that $q_0$ is a multiple of $X_{++}+X_{--}+X_{+-}+X_{-+}$.

Dimension counting shows that $\dim\sQ=l$ and general theory of homogeneous spaces shows that $\sQ$ has to be seen as the tangent space of the manifold $G/H$.

%
\subsection{Properties of the reductive decompositions}
%
\label{LONGSubSecPropRedDecompQH}

We are considering the two reductive decompositions $\sG=\sH\oplus\sQ=\sK\oplus\sP$ in the same time as the root space decomposition \eqref{LONGEqDecompGPourKillOrtho}. We are now giving some properties of them.


We know that $\sK\cap\sQ=\langle q_0\rangle$ belongs to $\tilde\sN_2$. As a consequence, the elements $X^k_{\alpha 0}$ and $X^k_{0\alpha}$ have no $\sK\sQ$-components and
\begin{equation}
	\begin{aligned}[]
		[J_2,\tilde\sN_k]_{\sP\sH}&=0\\
		[J_2,\tilde\sN_k]_{\sP\sQ}&=0\\
		\pr_{\sP\sQ}X_{\alpha 0}^k&=0\\
		\pr_{\sP\sH}X_{0\alpha }^k&=0.
	\end{aligned}
\end{equation}
Since $X^k_{\alpha 0}$ and $X^k_{0\alpha}$ are not  eigenvectors of $\theta$, they have a non vanishing $\sP$-component. We deduce that
\begin{equation}				\label{LONGEqXalphazeroaduPH}
	\begin{aligned}[]
		\pr_{\sP\sH}X^k_{\alpha 0}\neq 0\\
		\pr_{\sP\sQ}X^k_{0 \alpha }\neq 0.
	\end{aligned}
\end{equation}

As a consequence of compatibility between $\theta$ and $\sigma$, we have
\begin{equation}
	\begin{aligned}[]
		[J_1,(X_{\alpha\beta})_{\sH}]&=\alpha (X_{\alpha\beta})_{\sH}\\
		[J_1,(X_{\alpha\beta})_{\sQ}]&=\beta (X_{\alpha\beta})_{\sQ}
	\end{aligned}
\end{equation}
and
\begin{equation}		\label{LONGSubEqsJdeuxXalphaneta}
	\begin{aligned}[]
		[J_2,(X_{\alpha\beta})_{\sH}]&=\beta (X_{\alpha\beta})_{\sQ}\\
		[J_2,(X_{\alpha\beta})_{\sQ}]&=\alpha (X_{\alpha\beta})_{\sH}.
	\end{aligned}
\end{equation}
So $X_{\sQ}$ itself is an eigenvector of $\ad(J_1)$. In the same way, we prove that
\begin{equation}		\label{LONGEqJUnXabPK}
	\begin{aligned}[]
		[J_1,(X_{\alpha\beta})_{\sP}]=\alpha(X_{\alpha\beta})_{\sK}\\
		[J_1,(X_{\alpha\beta})_{\sK}]=\alpha(X_{\alpha\beta})_{\sP}
	\end{aligned}
\end{equation}
because $J_1\in\sP$.

\begin{corollary}		\label{LONGCorHPHKQPQKXuu}
	The vector $X_{++}$ has non vanishing components in $\sH\cap\sP$, $\sH\cap\sK$, $\sQ\cap\sP$ and $\sQ\cap\sK$.
\end{corollary}

\begin{proof}
	Since $\ad(J_2)$ inverts the $\sH$ and $\sQ$-components of $X_{++}$ (equation \eqref{LONGSubEqsJdeuxXalphaneta}), they must be both non zero. In the same way $\ad(J_1)$ inverts the components $\sP$ and $\sK$ of vectors of $\sH$ and $\sQ$ (equations \eqref{LONGEqJUnXabPK}).
\end{proof}

\begin{lemma}		\label{LONGLEmDesZPP}
	We have	$(X^{k}_{0+})_{\sK\sQ}=(X^{k}_{0+})_{\sP\sH}=0$ and consequently, $(X^k_{0+})_{\sP}=(X^k_{0+})_{\sQ}$.
\end{lemma}

\begin{proof}
	Consider the decomposition of the equality $[J_1,X^{k}_{0+}]=0$ into components $\sP\sQ$, $\sP\sH$, $\sK\sQ$, $\sK\sH$. Since $J_1\in\sP\cap\sH$, the $\sK\sH$ and $\sP\sQ$ components are
	\begin{subequations}
		\begin{align}
			[J_1,(X^{k}_{0+})_{\sP\sH}]&=0\\
			[J_1,(X^{k}_{0+})_{\sK\sQ}]&=0.
		\end{align}
	\end{subequations}
	In the same way, using the fact that $J_2\in\sP\cap\sQ$, we have
	\begin{subequations}	\label{LONGEqDeuxJUDzpKq}
		\begin{align}
			[J_2,(X_{0+}^{k})_{\sP\sH}]&=(X_{0+}^{k})_{\sK\sQ}\\
			[J_2,(X^{k}_{0+})_{\sK\sQ}]&=(X^{k}_{0+})_{\sP\sH}.		\label{LONGsubEqDeuxJUDzpKqb}
		\end{align}
	\end{subequations}
	Since $\dim(\sK\cap\sQ)=1$, the component $(X^{k}_{0+})_{\sK\sQ}$ has to be a multiple of $(X_{++})_{\sK\sQ}$. Thus we have
	\begin{equation}
		0=[J_1,(X^{k}_{0+})_{\sK\sQ}]=\lambda [J_1,(X_{++})_{\sK\sQ}]=\lambda (X_{++})_{\sP\sQ},
	\end{equation}
	but $(X_{++})_{\sP\sQ}\neq 0$, thus $\lambda=0$ and we conclude that $(X_{0+}^{k})_{\sK\sQ}=0$. Now, equation \eqref{LONGsubEqDeuxJUDzpKqb} shows that $(X_{0+}^{k})_{\sP\sH}=0$.
\end{proof}

\begin{lemma}				\label{LONGLemSigmaXzpBien}
	We have $\sigma X^k_{0+}=X^k_{0-}$.
\end{lemma}

\begin{proof}
	Since we know that $\sigma\sG_{\alpha\beta}\subset\sG_{\alpha,-\beta}$, the work is to fix the sign in
	\begin{equation}		\label{LONGEqsigmaXzppmthetaXzp}
		\sigma X^k_{0+}=\pm X^k_{0-}=\pm\theta X^k_{0+}.
	\end{equation}
	Lemma~\ref{LONGLEmDesZPP} states that $(X_{0+}^k)_{\sP}=(X_{0+}^k)_{\sQ}$. Thus the $\sQ$-component of $\theta X^k_{0+}$ is $-(X^k_{0+})_{\sQ}$, which is also equal to the $\sQ$-component of $\sigma(X^k_{0+})$. That  fixes the choice of sign in equation \eqref{LONGEqsigmaXzppmthetaXzp}.
\end{proof}

The following is an immediate corollary of lemma~\ref{LONGLemSigmaXzpBien} and the fact that $\theta$ fixes $\sP$ and $\sK$ while $\sigma$ fixes $\sH$ and $\sQ$.

\begin{corollary}		\label{LONGCorXzpHQPKXzm}
	We have
	\begin{subequations}
		\begin{align}
			(X^k_{0+})_{\sH}	&=(X^k_{0-})_{\sH}\\
			(X^k_{0+})_{\sQ}	&=-(X^k_{0-})_{\sQ}\\
			(X^k_{0+})_{\sP}	&=-(X^k_{0-})_{\sP}\\
			(X^k_{0+})_{\sK}	&=(X^k_{0-})_{\sK}.
		\end{align}
	\end{subequations}
\end{corollary}

\begin{proof}
	Since $\sigma$ acts as the identity on $\sH$ and changes the sign on $\sQ$, we have
	\begin{equation}
		\sigma X^k_{0+}=\sigma\big(  (X^k_{0+})_{\sH}+(X^k_{0+})_{\sQ}  \big)=(X^k_{0+})_{\sH}-(X^k_{0+})_{\sQ},
	\end{equation}
	but lemma~\ref{LONGLemSigmaXzpBien} states that $\sigma X^k_{0+}=X^k_{0-}=(X^k_{0-})_{\sH}+(X^k_{0-})_{\sQ}$. Equating the $\sH$ and $\sQ$-component of these two expressions of $\sigma X^k_{0+}$ brings the two first equalities.

	The two other are proven the same way. We know that $\theta X^k_{0+}=X^k_{0-}$, but
	\begin{equation}
		\theta X^k_{0+}=\theta\big( (X^k_{0+})_{\sP}+(X^k_{0+})_{\sK} \big)=-(X^k_{0+})_{\sP}+(X^k_{0+})_{\sK}.
	\end{equation}
	The two last relations follow.
\end{proof}

%
\subsubsection{An interesting basis of \texorpdfstring{$\sQ$}{Q}}
%
\label{LONGSubSubSecInterestingBasisQ}

Being the tangent space of $AdS$, the space $\sQ$ is of a particular importance. Let us now have a closer look at the vectors that we already mentioned in equations \eqref{LONGAlignPRedDefQQ}:
\begin{subequations}				\label{LONGEqBasQQzi}
	\begin{align}
		q_0&=(X_{++})_{\sK\sQ}\\
		q_1&=J_2\\
		q_2&=-[J_1,q_0]=-(X_{++})_{\sP\sQ}\\
		q_k&=(X^k_{0+})_{\sQ}		&\text{lemma~\ref{LONGLEmDesZPP}}.
	\end{align}
\end{subequations}

By lemma~\ref{LONGLEmDesZPP}, and the discussion about $\mZ(\sK)$ (equation \eqref{LONGEqDecompZKenDeuxSuivantDim}), we can express the elements $q_i$ without explicit references to $\sQ$ itself and each element corresponds to a particular space (once again, the choice of $q_0$ is not that simple in $AdS_3$):

%
\begin{subequations}		\label{LONGEqBasQQziPlusMieux}%
	\begin{align}
		q_0&=(X_{++})_{\mZ(\sK)}&\in \sK\cap\sQ\cap\tilde\sN_2					\\
		q_1&=J_2&\in\sQ\cap\sA\\
		q_2&=-[J_1,q_0]	&\in\sP\cap\sQ\cap\tilde\sN_2	\label{LONGsubEqJUnQZQDeux}		\\
		q_k&=(X^k_{0+})_{\sP}&\in\sP\cap\sQ\cap\tilde\sN_k.
	\end{align}
\end{subequations}
These elements correspond to the expression \eqref{LONGEqDecQEspacesCools}.
%
The compact part isomorphic to $\so(2,l-1)$ is then generated by the elements%

%
\begin{remark}

The space $\mZ(\sK)$ is given by the structure of the compact part of $\so(2,n)$, the elements $(X_{0+}^k)_{\sP}$ are defined from the root space structure of $\so(2,n)$ and the Cartan involution. The elements $J_1$ and $J_2$ are a basis of $\sA$. However, we need to know $\sH$ in order to distinguish $J_1$ from $J_2$ that are respectively generators of $\sA_{\sH}$ and $\sA_{\sQ}$.

Thus the basis
\eqref{LONGEqBasQQziPlusMieux} %
 %
is given in a way almost independent of the choice of $\sH$.
\end{remark}
%
%
\begin{corollary}		\label{LONGCorQdansPetK}
	We have $q_0\in\sK$ and $q_i\in\sP$ if $i\neq 0$. The set $\{ q_0,q_1,\ldots,q_l \}$ is a basis of $\sQ$. Moreover we have $\sQ\cap\tilde\sN_k=\langle q_k\rangle$.
\end{corollary}
%
\begin{proof}
	The first claim is a direct consequence of the expressions \eqref{LONGEqBasQQziPlusMieux}. Linear independence is a direct consequence of the spaces to which each vector belongs. A dimensional counting shows that it is a basis of $\sQ$.
\end{proof}

%
\subsubsection{Magic intertwining elements}
%

The vectors $q_i$ have the property to be intertwined by some elements of $\sH$. Namely, the adjoint action of the elements
%
\begin{subequations}\label{LONGAlignDefMagicIntert}
	\begin{align}
		X_1=p_1&=-[J_2,q_0]	&\in\sP\cap\sH\cap\tilde\sN_2\\
		X_2=s_1&=[J_1,X_1]	&\in\sK\cap\sH\cap\tilde\sN_2\\
		X_k=-r_k&=-[J_2,q_k]	&\in\sK\cap\sH\cap\tilde\sN_k				\label{LONGEqDefXkCommeComm}
	\end{align}
\end{subequations}%
%
intertwines the $q_i$'s in the sense of the following proposition.
%
\begin{proposition}[Intertwining properties]			\label{LONGXUnALaTwistingSuperCool}
	The elements defined by equation \eqref{LONGAlignDefMagicIntert} satisfy
	\begin{multicols}{2}
		\begin{subequations}				\label{LONGEqCalculBBBJUnUnNirme}
			\begin{align}
				\ad(J_1)q_0&=-q_2		\label{LONGEqCalculBBBJUnUnNirmeA}\\
				\ad(J_1)q_2&=-q_0.		\label{LONGEqCalculBBBJUnUnNirmeB}
			\end{align}
		\end{subequations}
		\begin{subequations}				\label{LONGEqSubEqbXUnqZero}
			\begin{align}
				\ad(X_1)q_1&=q_0		\label{LONGSubEqbXZeroqUn}\\
				\ad(X_1)q_0&=q_1,		\label{LONGSubEqbXUnqZero}
			\end{align}
		\end{subequations}
		\begin{subequations}
			\begin{align}
				\ad(X_2)q_2&=q_1		\label{LONGSubEqXdeuxQdeuxa}\\
				\ad(X_2)q_1&=-q_2		\label{LONGSubEqXdeuxQun}
			\end{align}
		\end{subequations}
		\begin{subequations}				\label{LONGEqSubEqbXkqZero}
			\begin{align}
				\ad(X_k)q_k&=-q_1.		\label{LONGSubEqbXIMoinsqZero}\\
				\ad(X_k)q_1&=q_k		\label{LONGSubEqbXkQunQk}
			\end{align}
		\end{subequations}
	\end{multicols}
\end{proposition}


\begin{proof}


	Equation \eqref{LONGEqCalculBBBJUnUnNirmeA} is by definition while equation \eqref{LONGEqCalculBBBJUnUnNirmeB} follows from the first one and the fact that $\ad(J_1)^2$ acts as the identity on $\tilde\sN_2$.

	The equality \eqref{LONGSubEqbXZeroqUn} is a direct consequence of the fact that $\ad(J_2)^2$ is the identity on $\tilde\sN_2$, so that
	\begin{equation}
		[X_1,q_1]=-\big[ [J_2,q_0],q_1 \big]=\ad(J_2)^2q_0=q_0.
	\end{equation}

	For the relation \eqref{LONGSubEqbXUnqZero}, first remark that, since $q_0=(X_{++})_{\sK\sQ}$, we have
	\begin{equation}
		X_1=-(X_{++})_{\sP\sH}
	\end{equation}
	and we have to compute
	\begin{equation}
		[X_1,q_0]=-\big[ (X_{++})_{\sP\sH},(X_{++})_{\sK\sQ} \big]
	\end{equation}
	Using the projections \eqref{LONGEqProjHQPKsigmatheta}, we have
	\begin{equation}
		\begin{aligned}[]
			(X_{++})_{\sP\sH}&=\frac{1}{ 4 }(X_{++}+\sigma X_{++}-\theta X_{++}-\sigma\theta X_{++})\\
			(X_{++})_{\sK\sQ}&=\frac{1}{ 4 }(X_{++}-\sigma X_{++}+\theta X_{++}-\sigma\theta X_{++})
		\end{aligned}
	\end{equation}
	We compute the commutator taking into account the facts that $\sigma$ is an automorphism and that, for example, $[X_{++},\sigma X_{++}]=0$ because $\sigma X_{++}\in\sG_{(+-)}$. What we find is
	\begin{equation}
		\big[ (X_{++})_{\sP\sH},(X_{++})_{\sK\sP} \big]=\frac{1}{ 4 }\frac{ 1 }{2}\Big( [X_{++},\theta X_{++}]-\sigma [X_{++},\theta X_{++}] \Big)=\frac{1}{ 4 }[X_{++},\theta X_{++}]_{\sQ}.
	\end{equation}
	Since $[X_{++},X_{--}]=-4(J_1+J_2)$, we have $[X_1,q_0]=J_2=q_1$ as expected.


	For equation \eqref{LONGSubEqXdeuxQdeuxa} we use the Jacobi relation and the relation \eqref{LONGEqCalculBBBJUnUnNirmeB}.
	\begin{equation}
		\begin{aligned}[]
			[q_2,X_2]&=\big[ q_2,[J_1,p_1] \big]\\
				&=-\big[ J_1,[p_1,q_2] \big]-\big[ p_1,[q_2,J_1] \big]\\
				&=-[p_1,q_0]\\
				&=-q_1
		\end{aligned}
	\end{equation}
	For equation \eqref{LONGSubEqXdeuxQun}, we use the definition of $X_2$, the Jacobi identity and the facts that $[p_1,J_2]=q_0$ and $[J_1,q_0]=-q_2$.

	We pass now to the fourth pair of intertwining relations. By definition, $q_k=(X^k_{0+})_{\sP}$, but taking into account the fact that $J_2\in\sP$ we can decompose the relation $[J_2,X_{0+}]=X_{0+}$ into
	\begin{subequations}
		\begin{align}
			[J_2,(X_{0+})_{\sP}]&=(X_{0+})_{\sK}		\label{LONGEqJdeuxXzpsPsK}\\
			[J_2,(X_{0+})_{\sK}]&=(X_{0+})_{\sP}.
		\end{align}
	\end{subequations}
	Equation \eqref{LONGEqJdeuxXzpsPsK} tolds us that
	\begin{equation}
		X_k=-(X_{0+})_{\sK}.
	\end{equation}
	Now we have to compute $[X_k,q_k]=-\big[ (X_{0+})_{\sK},(X_{0+})_{\sP} \big]$. We know that $[X_{0+},X_{0-}]=-2J_2\in\sP$. Thus corollary~\ref{LONGCorXzpHQPKXzm} brings
	\begin{equation}
		-2J_2=\big[ (X_{0+})_{\sK},(X_{0-})_{\sP} \big]+\big[ (X_{0+})_{\sP},(X_{0-})_{\sK} \big]=-2\big[ (X_{0+})_{\sK},(X_{0+})_{\sP} \big]=2[X_k,q_k],
	\end{equation}
	and the result follows.

	For equation \eqref{LONGSubEqbXkQunQk}, we have to compute $[X_k,q_1]=[J_2,(X_{+0}^k)_{\sK}]$. The $\sP$-component of $[J_2,X_{0+}^k]=X_{0+}^k$ is exactly
	\begin{equation}
		[J_2,(X_{0+}^k)_{\sK}]=(X_{0+}^k)_{\sP}=q_k.
	\end{equation}

\end{proof}

These intertwining relations will be widely used in computing the norm of the vectors $q_i$ in proposition~\ref{LONGPropBaseQOrtho} as well as in some other occasions.

Let us now give a few words about the existence and unicity of these elements. The fact that there exists an element $X_1$ such that $\ad(J_2)X_1=q_0$ comes from the decomposition \eqref{LONGEqsDecopmQXpmpm} and the fact that each $X_{\pm\pm}$ is an eigenvector of $\ad(J_2)$. It is thus sufficient to adapt the signs in order to manage a combination of $X_{++}$, $X_{+-}$, $X_{-+}$ and $X_{--}$ on which the adjoint action of $J_2$ creates $q_0$. However, the fact that this element has in the same time the ``symmetric'' property $\ad(X_1)q_0=q_1$ could seem a miracle. See theorem~\ref{LONGThoAdSqIouZero}.

\begin{lemma}
	An element $X_1$ such that $\ad(X_1)q_1=q_0$ can be chosen in $\sP\cap\sH\cap\tilde\sN_2$. Moreover, this choice is unique up to normalisation.
\end{lemma}

\begin{proof}
	Unicity is nothing else than the fact that $\dim(\sP\cap\sH\cap\tilde\sN_2)=1$. Indeed, since $\sG=\sA\oplus\tilde\sN\oplus\mZ_{\sK}(\sA)$ and $\sA\subset\sP$, we have $\sK\subset\tilde\sN$. Dimension counting shows that $\dim(\tilde\sN_2\cap\sH)=2$ (because $\dim(\tilde\sN_2)=4$ and $q_0,q_2\in\sQ\cap\tilde\sN_2$). As we are looking in $\tilde\sN_2$, we are limited to elements in $\so(2,2)$ (not the higher dimensional slices), so that we can consider $\sK=\so(2)\oplus\so(2)$. One of these two $\so(2)$ factors belongs to $\sH$, so that $\dim(\sK\cap\sH\cap\tilde\sN_2)=1$  and finally $\dim(\sP\cap\sH\cap\tilde\sN_2)=1$.

	Let now $X_1$ be such that $[X_1,q_1]=q_0$. If $X_1$ has a component in $\sQ$, that component has to commute with $q_1$ (if not, the commutator $[X_1,q_1]$ would have a $\sH$-component). So we can redefine $X_1$ in order to have $X_1\in\sH$.

	In the same way, a $\sA$-component has to be $J_1$ (because $J_2\in\sQ$) which commutes with $q_1$. We redefine $X_1$ in order to remove its $J_1$-component. We remove a component in $\tilde\sN_k$ because $[\tilde\sN_2,\tilde\sN_k]\subset\tilde\sN_k$, and a $\sK$-component can also be removed since its commutator with $q_1$ would produce a $\sP$-component. We showed that $X_1\in\sP\cap\sH\cap\tilde\sN_2$.
\end{proof}

\begin{lemma}		\label{LONGLemChoixDeXk}
	An element $X_k$ such that $\ad(X_k)q_1=q_k$ can be chosen in $\sK\cap\sH\cap\tilde\sN_k$.
\end{lemma}

\begin{proof}
	The proof is elementary in tree steps using the fact that $q_1\in\sP\cap\sQ\cap\sA$:
	\begin{enumerate}
		\item
			A $\sP$-component can be annihilated because $[\sP,\sP]\subset\sK$ while $q_k\in\sP$,
		\item
			a $\sQ$-component can be annihilated because $[\sQ,\sQ]\subset\sH$ while $q_k\in\sQ$,
		\item
			if $k'\neq k$, a $\tilde\sN_{k'}$-component can be annihilated because $[\tilde\sN_{k'},\sA]\subset\tilde\sN_{k'}$ while $q_k\in\tilde\sN_k$.
	\end{enumerate}
\end{proof}


%
\subsubsection{Killing form and orthogonality}
%




We \emph{define} the norm of an element in $\sG=\so(2,n)$ as
\begin{equation}	\label{LONGEqDefNormeKillingSix}
	\| X \|=-\frac{ 1 }{2n}B(X,X).
\end{equation}
Notice that $q_0$ belongs to the compact part of $\sG$, so that its Killing norm is negative, so that $\| q_0 \|$ is positive.


\begin{proposition}		\label{LONGPropBaseQOrtho}
	We have $\| q_0 \|=1$ and $\| q_i \|=-1$ ($i\neq 0$). As a consequence, the space $G/H$ is Lorentzian.
\end{proposition}


\begin{proof}
	We begin by computing the norm of $q_1=J_2$. The Killing form $B(J_2,J_2)=\tr\big( \ad(J_2)\circ\ad(J_2) \big)$ is the easiest to compute in the basis $\mZ_{\sK}(\sA)\oplus\sA\oplus\sN\oplus\bar\sN$ of eigenvectors of $J_2$. If we look at the matrix of $\ad(J_2)\circ\ad(J_2)$, we have one $1$ at each of the positions of $X_{++}$, $X_{+-}$, $X_{-+}$ and $X_{--}$. Moreover, for each higher dimensional slice, we get additional $2$ because of $X_{0+}^k$ and $X_{+0}^k$. When one looks at $\so(2,n)$ we have $n-2$ higher dimensional slices, so that
	\begin{equation}
		B(J_2,J_2)=4+2(n-2)=2n.
	\end{equation}

	The result is that $B(q_1,q_1)=6$, so that $\| q_1 \|=-1$.

	We are going to propagate that result to other elements of the basis, using the``magic'' intertwining elements $X_1$, $X_k$ and $J_1$.

	Using left invariance of the Killing form, we find
	\begin{equation}		\label{LONGEqCalculBBBqZeroqUnNirme}
		B(q_0,q_0)=B\big( q_0,-\ad(J_1)q_2 \big)=B\big( \ad(J_1)q_0,q_2 \big)=-B(q_2,q_2),
	\end{equation}
	so that $\| q_0 \| =-\| q_2 \|$.

	Now, the same computation with $X_1$ and $X_k$ instead of $J_1$ show that $\| q_0 \|=-\| q_1 \|$ and $\| q_1 \|=\| q_k \|$.
\end{proof}



\begin{remark}		\label{LONGRemBProdScal}
	Using the fact that the basis $\{ q_i\}$ is orthonormal, we can decompose an element of $\sQ$ by the Killing form. One only has to be careful on the sign: if $X=aq_0+\sum_{i>0}b_iq_i$, we have
	\begin{equation}		\label{LONGEqsabKillProjComp}
		\begin{aligned}[]
			a&=B(X,q_0)\\
			b_i&=-B(X,q_i).
		\end{aligned}
	\end{equation}
\end{remark}


\begin{remark}	\label{LONGRemOrdreNilpotentQ}
	As a consequence, a light like direction reads, up to normalization, $E(w)=q_0+\sum_{i=1}^{l-1}w_iq_i$ with $w\in S^{l-2}$.
\end{remark}

%
\subsubsection{Other properties}
%


\begin{lemma}		\label{LONGLemXZUAHetQ}
	We have $\sigma X_{\alpha\beta}\in\sG_{(\alpha,-\beta)}$. In particular, $X^k_{0+}$ has non vanishing components in $\sH$ and in $\sQ$.
\end{lemma}

\begin{proof}
	If one applies $\sigma$ to the equality $[J_2,X_{\alpha\beta}]=\beta X_{\alpha\beta}$, we see that $\sigma X_{\alpha\beta}$ is an eigenvector of $\ad(J_2)$ with eigenvalue $-\beta$. The same with $\ad(J_1)$ shows that $\sigma X_{\alpha\beta}$ has $+1$ as eigenvalue. Thus $\sigma X_{\alpha\beta}\in\sG_{(\alpha,-\beta)}$.

	In particular, $\sigma X^k_{0+}\neq \pm X^k_{0+}$ so that it does not belongs to $\sH$ nor to $\sQ$.
\end{proof}

Notice that, as corollary, we have
\begin{equation}
	\sigma X_{\alpha,\beta}=\pm X_{\alpha,-\beta}.
\end{equation}


\begin{lemma}				\label{LONGLemSigmaXppEgalXPm}
	We have $(X_{++})_{\sQ}=(X_{+-})_{\sQ}$ or, equivalently, $\sigma X_{++}=-X_{+-}$.
\end{lemma}

\begin{proof}
	Since $q_1=J_2\in\sA$ and $q_k\in\tilde\sN_k$, the $\sQ$-component of $X_{++}$ and $X_{+-}$ are only made of $q_0$ and $q_2$.
	We are	going to prove the following three equalities.
	\begin{enumerate}

		\item\label{LONGItemBpmqDeux}
			$B(X_{+-},q_2)=B(X_{+-},q_0)$
		\item
			$B(X_{++},q_2)=B(X_{++},q_0)$
		\item
			$B(X_{++},q_0)=B(X_{+-},q_0)$
	\end{enumerate}

	The first point is proved using the fact that $q_2=[q_0,J_1]$ and the $\ad$-invariance of the Killing form:
	\begin{equation}
		B(X_{+-},q_2)=-B\big( X_{+-},\ad(J_1)q_0 \big)=B\big( \ad(J_1)X_{+-},q_0 \big)=B(X_{+-},q_0).
	\end{equation}
	One checks the second point in the same way. For the third equality, we know from decomposition \eqref{LONGEqDecompZKenDeuxSuivantDim} that $q_0$ is a multiple of $X_{++}+X_{--}+X_{+-}+X_{-+}$. If the multiple is $\lambda$, $B(X_{++},q_0)=\lambda B(X_{++},X_{--})$ and $B(X_{+-},q_0)=\lambda B(X_{+-},X_{-+})$. Thus we have to prove that the traces of the operators
	\begin{equation}
		\begin{aligned}[]
			\gamma_1&=\ad(X_{++})\circ\ad(X_{--})\\
			\gamma_2&=\ad(X_{+-})\circ\ad(X_{-+})
		\end{aligned}
	\end{equation}
	are the same. That trace is straightforward to compute on the natural basis of $\sG=\mZ_{\sK}(\sA)\oplus\sA\oplus\sN\oplus\bar\sN$.
	%
	The only elements on which $\ad(X_{--})$ is not zero are $\sA$, $X^k_{0+}$, $X^k_{+0}$ and $X^k_{++}$, while for $\ad(X_{-+})$, the only non vanishing elements are $\sA$, $X^k_{0-}$, $X^k_{+0}$ and $X_{+-}$. From equation \eqref{LONGsubeqrXpmpm}, we have $\gamma_1(r_{ij})=\gamma_2(r_{ij})=0$. Using the commutation relations, we find
	\begin{subequations}
		\begin{align}
			\gamma_1J_1&=[X_{++},X_{--}]=-4(J_1+J_2) \\
			\gamma_1J_2&=[X_{++},X_{--}]=-4(J_1+J_2) \\
			\gamma_1X^k_{0+}&=2[X_{++},X^k_{-0}]=-4X^k_{0+} \\
			\gamma_1X^k_{+0}&=-2[X_{++},X^k_{0-}]=-4X^k_{+0} \\
			\gamma_1X_{++}&=[X_{++},4(J_1+J_2)]=-8X_{++}.
		\end{align}
	\end{subequations}
	Thus $\tr(\gamma_1)=-24$. The same computations bring
	\begin{subequations}
		\begin{align}
			\gamma_2J_1&=[X_{+-},X_{-+}]=-4(J_1-J_2)\\
			\gamma_2J_2&=[X_{+-},X_{-+}]=4(J_1-J_2)\\
			\gamma_2X^k_{0-}&=-2[X_{+-},X^k_{-0}]=-4X^k_{0-}\\
			\gamma_2X_{+-}&=[X_{+-},4(J_1-J_2)]=-8X_{+-}\\
			\gamma_2X^k_{+0}&=2[X_{+-},X^k_{0+}]=-2X^k_{+0},
		\end{align}
	\end{subequations}
	and $\tr(\gamma_2)=-24$. Thus we have
	\begin{equation}
		\pr_{\mZ(\sK)}(X_{++})=\pr_{\mZ(\sK)}(X_{+-}).
	\end{equation}
	%
\end{proof}

Notice that the lemma is trivial if we consider that $X_{++}-X_{+-}$ belongs to $\sH$ by definition of $\sH$. From a $AdS$ point of view, we define $AdS=G/H$ and we have to define $H$, so from that point of view, lemma~\ref{LONGLemSigmaXppEgalXPm} is by definition. However, the direction we have in mind is to use the more generic tools as possible. From that point of view, the fact to set $\mZ(\sK)\subset\sQ$ is more intrinsic than to set $X_{++}-X_{+-}\in\sH$.


\begin{proposition}		\label{LONGPropXmpXppqq}
	We have $(X_{++})_{\sQ}=(X_{+-})_{\sQ}=q_0-q_2$.
\end{proposition}

\begin{proof}
	Using the remark~\ref{LONGRemBProdScal}, the three Killing forms computed in the proof of lemma~\ref{LONGLemSigmaXppEgalXPm} are expressed under the form
	\begin{subequations}
		\begin{align}
			(X_{+-})_{q_0}&=-(X_{+-})_{q_2}\\
			(X_{++})_{q_0}&=-(X_{++})_{q_2}\\
			(X_{++})_{q_0}&=(X_{+-})_{q_2}.
		\end{align}
	\end{subequations}
	Consequently, we have $(X_{++})_{\sQ}=\lambda(q_0-q_2)$ and $(X_{+-})_{\sQ}=\lambda(q_0-q_2)$ for a constant $\lambda$ to be fixed. It is fixed to be $1$ by the facts that, by definition, $q_0=(X_{++})_{\sK\sQ}$ and $q_2\in\sP$.
\end{proof}

\begin{lemma}		\label{LONGLemComJDeuxQ}
	We have
	\begin{equation}
		\begin{aligned}[]
			[J_2,q_0]&=(X_{++})_{\sH\sP}\neq 0\\
			[J_2,q_1]&=0\\
			[J_2,q_2]&=(X_{++})_{\sH\sK}\neq 0\\
			[J_2,q_k]&=(X^k_{0+})_{\sH}\neq 0\\
			[J_1,p_1]&=-(X_{++})_{\sK\sH}\neq 0
		\end{aligned}
	\end{equation}
	where $k\geq 3$.
\end{lemma}

\begin{proof}
	Using the fact that $J_2\in\sQ\cap\sP$ and that $X_{++}$ has non vanishing components ``everywhere'' (corollary~\ref{LONGCorHPHKQPQKXuu}), we have
	\begin{equation}
		\begin{aligned}[]
			[J_2,q_0]&=[J_2,(X_{++})_{\sK\sQ}]=(X_{++})_{\sP\sH}\neq 0\\
			[J_2,q_2]&=[J_2,(X_{++})_{\sP\sQ}]=(X_{++})_{\sK\sH}\neq 0\\
			[J_1,p_1]&=[J_1,(X_{++})_{\sP\sH}]=-(X_{++})_{\sK\sH}\neq 0\\
			[J_2,q_k]&=[J_2,(X_{0+}^k)_{\sQ}]=(X^k_{0+})_{\sH}\neq 0		&\text{lemma~\ref{LONGLemXZUAHetQ}}\\
		\end{aligned}
	\end{equation}
\end{proof}

\begin{lemma}		\label{LONGLemNonHXaz}
	If $\alpha\neq 0$, then $X_{\alpha 0}^k\in\sH$.
\end{lemma}

\begin{proof}
	The element $\pr_{\sQ}X^k_{\alpha 0}$ is a combination of $q_i$. Since $\ad(J_2)\pr_{\sQ}X^k_{\alpha 0}=0$, we must have $(X^k_{\alpha 0})_{\sQ}=\lambda J_2$ by lemma~\ref{LONGLemComJDeuxQ}. Using the fact that $J_1\in\sH$, the $\sQ$-component of the equality $[J_1,X^k_{\alpha 0}]=\alpha X^k_{\alpha 0}$ becomes
	\begin{equation}
		[J_1,\lambda J_2]=\alpha\lambda J_2.
	\end{equation}
	The left-hand side is obviously zero, so that $\lambda=0$ which proves that $X^k_{\alpha 0}\in\sH$.
\end{proof}

Applying successively the projections \eqref{LONGEqProjHQPKsigmatheta}, and lemma~\ref{LONGLemSigmaXppEgalXPm}, we write the basis elements of $\sQ$ in the decomposition $\sG=\sG_0\oplus\sN\oplus\bar\sN$:
\begin{subequations}			\label{LONGEqsDecopmQXpmpm}
	\begin{align}
		q_0&=\frac{1}{ 4 }(X_{++}+X_{+-}+X_{-+}+X_{--}),		\\
		q_1&=J_2,							\\
		q_2&=\frac{1}{ 4 }(-X_{++}-X_{+-}+X_{-+}+X_{--}),		\\
		q_k&=\frac{ 1 }{2}(X^{k}_{0+}-X^{k}_{0-})
	\end{align}
\end{subequations}
with $k\geq 3$. Notice that none of them has component in $\mZ_{\sK(\sA)}$.

These decompositions allow us to compute the commutators $[q_i,q_j]$ and $[q_i,J_p]$. Instead of listing here every commutation relations, we will only write the ones we use when we need them.


\begin{lemma}		\label{LONGLemQzQdeuxJun}
	We have $[q_0,q_2]=-J_1$.
\end{lemma}

\begin{proof}
	The proof is exactly the same as the one of equation \eqref{LONGSubEqbXUnqZero} in lemma~\ref{LONGXUnALaTwistingSuperCool}. Here we use
	\begin{equation}
		(X_{++})_{\sP\sQ}=\frac{1}{ 4 }\big( X_{++}-\sigma X_{++}-\theta X_{++}+\sigma\theta X_{++} \big)
	\end{equation}
	and we find
	\begin{equation}
		[q_0,q_2]=-\big[ (X_{++})_{\sK \sQ},(X_{++})_{\sP\sQ} \big]=-\frac{1}{ 4 }[\theta X_{++},X_{++}]_{\sH}=-J_1.
	\end{equation}
\end{proof}

\begin{lemma}
We have
	\begin{equation}		\label{LONGEqXunQdeuxcommutent}
		[X_1,q_2]=[X_1,q_k]=0
	\end{equation}
	for $k\geq 3$.
\end{lemma}
\begin{proof}
	The proof is elementary:
	\begin{equation}
		\begin{aligned}[]
			[X_1,q_2]&\in[\sP\cap\sH\cap\tilde\sN_2,\sP\cap\sQ\cap\tilde\sN_2]\subset\sK\cap\sQ\cap\sA=\{ 0 \}\\
			[X_1,q_k]&\in[\sP\cap\sH\cap\tilde\sN_2,\sP\cap\sQ\cap\tilde\sN_k]\subset\sK\cap\sQ\cap\tilde\sN_k=\{ 0 \}.
		\end{aligned}
	\end{equation}
\end{proof}

The following is a first step in the proof of theorem~\ref{LONGThoAdESqqq}.
\begin{corollary}			\label{LONGCorAdQUncarreqi}
	We have $\ad(J_1)|_{\tilde\sN_2}^2=\ad(J_2)|_{\tilde\sN_2}^2=\id$ and $\ad(q_1)^2q_i=q_i$.
\end{corollary}

\begin{proof}
	The action of $\ad(q_1)^2$ is to change two times the sign of the components $X_{\alpha -}$. Thus $\ad(q_1)^2=\id$ on $\tilde\sN_2$. The result is now proved for $i=0,1,2$. For the higher dimensions, we use the fact that $J_2=q_1$ and we find
	\begin{equation}
		q_k=[X_k,q_1]=-\big[ [q_1,X_k],q_1 \big]=\ad(q_1)^2q_k
	\end{equation}
	as claimed.

	Moreover, the elements of $\tilde\sN_2$ are build of elements of the form $X_{\pm\pm}$, so that $\ad(J_1)^2$ changes at most twice the sign.

\end{proof}

\begin{lemma}		\label{LONGLemXkqzerozero}\label{LONGLemXkJunzero}
	We have
	\begin{subequations}
		\begin{align}
			[X_k,q_0]=[X_k,J_1]=[X_k,q_2]=0\\
			[J_1,q_k]=0.		\label{LONGEqJUnqkzero}
		\end{align}
	\end{subequations}
\end{lemma}

\begin{proof}
	The first claim is proved in a very standard way:
	\begin{equation}
		[X_k,q_0]\in[\sK\cap\sH\cap\tilde\sN_k,\sK\cap\sQ\cap\tilde\sN_2]\subset\sK\cap\sQ\cap\tilde\sN_k=\{ 0 \}.
	\end{equation}

	For the second commutator, we use the Jacobi identity and the definition $X_k=-[J_2,q_k]$:
	\begin{equation}		\label{LONGEqJ1J2qkzero}
		\big[ J_1,[J_2,q_k] \big]=-\big[ J_2,[q_k,J_1] \big]-\big[ q_k,\underbrace{[J_1,J_2]}_{=0} \big],
	\end{equation}
	while
	\begin{equation}
		[q_k,J_1]\in[\sP\cap\sQ\cap\tilde\sN_k,\sP\cap\sH\cap\sA]\subset\sK\cap\sQ\cap\tilde\sN_k=\{ 0 \}.
	\end{equation}
	That proves \eqref{LONGEqJUnqkzero} in the same time.

	For the third commutator, remark that, since $q_2=[q_0,J_1]$, we have
	\begin{equation}
		[X_k,q_2]=-\big[ q_0,[J_1,X_k] \big]-\big[ J_1,[X_k,q_0] \big].
	\end{equation}
	which is zero by the two first claims.
\end{proof}

\begin{proposition}			\label{LONGEtOrdreDeux}
	If $E$ is nilpotent in $\sQ$, then $\ad(E)^3=0$.
\end{proposition}

\begin{proof}
	If $E_1$ is a nilpotent element of $\sQ$, then every nilpotent elements in $\sQ$ are of the form $\lambda\Ad(k)E_1$ for some $k\in K$ and $\lambda\in\eR$\cite{These}. It is then sufficient to prove that one of them is of order two. The element
	\begin{equation}		\label{LONGEqDecqzmoinsqDeux}
		q_0-q_2=\frac{ 1 }{2}(X_{++}+X_{+-}),
	\end{equation}
	is obviously of order two because the eigenvalue for $\ad(J_1)$ increases by one unit at each iteration of $\ad(q_0-q_2)$.
\end{proof}

\begin{lemma}
	We have $[J_1,q_k]=0$.
\end{lemma}

\begin{proof}
	The proof is standard:
	\begin{equation}
		[J_1,q_k]\in[\sH\cap\sP\cap\sA,\sQ\cap\sP\cap\tilde\sN_k]\subset\sQ\cap\sK\cap\tilde\sN_k=\{ 0 \}.
	\end{equation}
\end{proof}

The following theorem, which relies on the preceding lemmas, will be central in computing the Killing form which appears in the characterization of theorem~\ref{LONGThosSequivJzero}.
%
\begin{theorem}			\label{LONGThoAdESqqq}
	We have
	\begin{equation}
		\ad(q_i)^2q_j=q_j
	\end{equation}
	if $i\neq j$ and $i\neq 0$. If $i=0$, we have
	\begin{equation}
		\ad(q_0)^2q_j=-q_j.
	\end{equation}
\end{theorem}


\begin{proof}
	The case $i=1$ is already done in corollary~\ref{LONGCorAdQUncarreqi}.

	We propagate that result to the other $\ad(q_i)^2$ with the intertwining elements $J_1$, $X_1$ and $X_k$.

	Let us compute $\ad(q_0)^2q_i=\ad\big([X_1,q_1]\big)^2q_i$ using twice the Jacobi identity and the properties of $X_1$ (in order to be more readable, we write $XY$ for $[X,Y]$)
	\begin{equation}		\label{LONGEqAdqZsqqi}
		\begin{aligned}[]
			\ad(q_0)^2q_i	&=	(X_1q_1)\Big( (X_1q_1)q_i \Big)\\
					&=	-(X_1q_1)\Big( (q_1q_I)X_1+(q_iX_1)q_1 \Big)\\
					&=(q_1q_i)\big( X_1(X_1q_1) \big)+(q_iX_1)\big( q_1(X_1q_1) \big)\\
					&\quad + X_1\big( (X_1q_1)(q_1q_i) \big)+q_1\big( (X_1q_1)(q_iX_1) \big)\\
					&=(q_1q_i)q_1-\ad(X_1)^2q_i+X_1\big( q_0(q_1q_i) \big)+q_1\big( q_0(q_iX_1) \big).
		\end{aligned}
	\end{equation}

	If $i=1$, the only non vanishing term is $-\ad(X_1)^2q_1=-q_1$.  Thus $\ad(q_0)^2q_1=-q_1$.

	If $i=2$, the relation \eqref{LONGEqXunQdeuxcommutent} annihilates the second and fourth terms while $[q_1,q_2]$ commutes with $q_0$ because $q_0\in\mZ(\sK)$. We are thus left with the term $-q_2$, and $\ad(q_0)^2q_2=-q_2$.

	If $i=k\geq 3$, we find
	\begin{equation}
		\ad(q_0)^2q_k=-\ad(q_1)^2q_k-\ad(X_1)^2q_k+X_1\big( q_0(q_1q_k) \big)+q_1\big( q_0(q_kX_1) \big).
	\end{equation}
	Since $[q_1,q_k]\in\sK$, it commutes with $q_0$. Using the fact that $[X_1,q_k]=0$, we get $\ad(q_0)^2q_k=-q_k$.

	Let us perform the same computations  as in \eqref{LONGEqAdqZsqqi} with $q_k$ ($k\geq 3$) instead of $q_0$ and $X_k$ (equations \eqref{LONGEqSubEqbXkqZero}) instead of $X_1$. What we get is
	\begin{equation}
		\ad(q_k)^2q_i=\ad(q_1)^2q_i-\ad(X_k)^2q_i+X_k\big( q_k(q_1q_i) \big)+q_1\big( q_k(q_iX_k) \big).
	\end{equation}

	If we set $i=0$, taking into account the commutator $[X_k,q_0]=0$, we have
	\begin{equation}		\label{LONGEqkdeuxqzerointer}
		\ad(q_k)^2q_0=\ad(q_1)^2q_0+X_k\big( q_k(q_1q_0) \big).
	\end{equation}
	As already proved, the first term is $q_0$. Now,
	\begin{equation}
		\big[ q_k,[q_1,q_0] \big]\in\sK\cap\sQ\cap\tilde\sN_k=\{ 0 \},
	\end{equation}
	so that the second term in \eqref{LONGEqkdeuxqzerointer} is zero. Thus we proved that $\ad(q_k)^2q_0=q_0$.

	If we set $i=1$, taking into account the relations \eqref{LONGEqSubEqbXkqZero}, we find
	\begin{equation}
		\ad(q_k)^2q_1=-\ad(X_k)^2q_1+q_1\big( q_k(q_1X_k) \big)=q_1.
	\end{equation}

	If we set $i=2$ and using the fact that $[X_k,q_2]=0$, we find
	\begin{equation}
		\ad(q_k)^2q_2=q_2-q_k\big( X_k(q_1q_2) \big).
	\end{equation}
	Using once again the Jacobi identity inside the big parenthesis, we find $2q_2-\ad(q_k)^2q_2$. This proves that $\ad(q_k)^2q_2=q_2$.


	We turn now our attention to $\ad(q_2)^2q_i$. We perform the same computation, using the intertwining property \eqref{LONGEqCalculBBBJUnUnNirme} of $J_1$. What we get is
	\begin{equation}		\label{LONGEqadqqqDeuxIGene}
		\ad(q_2)^2q_i=(J_1q_i)(q_0q_2)-\ad(q_0)^2q_i+q_0\big( q_2(J_1q_i) \big)+J_1\big( q_2(q_iq_0) \big).
	\end{equation}

	If we set $i=1$, we use the already proved property $\ad(q_0)^2q_1=-q_1$, and we obtain
	\begin{equation}
		\ad(q_2)^2q_1=(J_1q_1)(q_0q_2)+q_1+q_0\big( q_2(J_1q_1) \big)+J_1\big( q_2(q_1q_0) \big).
	\end{equation}
	We claim that all of these terms are zero except of $q_1$. First, $\big[ q_2,[q_1,k_k] \big]\in\big[ \tilde\sN_2,[\sA,\tilde\sN_2] \big]\subset\sA$.	Thus the last term vanishes .The commutator $[J_1,q_1]$ vanishes because $q_1=J_2$. We are done with $\ad(q_2)^2q_1=q_1$.

	If we set $i=k$ ($k\geq 3$) in \eqref{LONGEqadqqqDeuxIGene}, we use $\ad(q_0)^2q_k=-q_k$ and what we find is
	\begin{equation}		\label{LONGEqadqqqdeuxkk}
		\ad(q_2)^2q_k=(J_1q_k)(q_0q_2)+q_k+q_0\big( q_2(J_1q_k) \big)+J_1\big( q_2(q_kq_0) \big).
	\end{equation}
	We already know that $[J_1,q_k]=0$. We have $\big[ q_2,[q_k,q_0] \big]=0$ because
	\begin{equation}
		\begin{aligned}[]
			\big[ q_2,[q_k,q_0] \big]&\in\big[ \sP\cap\sQ\cap\tilde\sN_2,[\sP\cap\sQ\cap\tilde\sN_k,\sK\cap\sQ\cap\tilde\sN_2] \big]\\
						&\subset[\sP\cap\sQ\cap\tilde\sN_2,\sP\cap\sH\cap\tilde\sN_k]\\
						&\subset\sK\cap\sQ\cap\tilde\sN_k=\{ 0 \}.
		\end{aligned}
	\end{equation}
	The remaining terms in \eqref{LONGEqadqqqdeuxkk} are $\ad(q_2)^2q_k=q_k$.

	In order to compute $\ad(q_2)^2q_0$, we write $q_0=\ad(X_1)q_1$. Using twice the Jacobi identity, we get
	\begin{equation}
		\ad(q_2)^2q_0=X_1\big( (q_1q_2)q_2 \big)+q_1\big( (X_1q_2)q_2 \big)+(q_1q_2)(X_1q_2)+(X_1q_2)(q_2q_1).
	\end{equation}
	Using the fact that $[X_1,q_2]=0$, we are left with
	\begin{equation}
		\ad(q_2)^2q_0=X_1\big( \ad(q_2)^2q_1 \big)=[X_1,q_1]=q_0
	\end{equation}
	as desired.
\end{proof}


%
\subsection{A convenient basis for the root spaces and computations}
%
\label{LONGSubSecMOreConvBasisBlbla}
%

The most natural basis of $\tilde\sN_2$ is
\begin{equation}
	\tilde\sN_2=\langle X_{++},X_{+-},X_{-+},X_{--}\rangle,
\end{equation}
but the multiple commutators of these elements with $q_0$ reveal to require some work.

We provide in this section an other basis for $\tilde\sN$ that corresponds to the decomposition $\sK\oplus\sP$. Since $q_0$ is central in $\sK$, the exponential $e^{xq_0}X$ is trivial when $X\in\sK$ and, since $q_0\in\sK$, the commutator $[q_0,X]$ remains in $\sP$ when $X\in\sP$.

Here is the new basis:
%
\begin{equation}		\label{LONGEqSuperBaseeB}
	\eB=\{J_1,J_2, q_0,q_2,p_1,s_1,q_k,p_k,r_k,s_k \}_{k=3,\ldots,l-1}.
\end{equation}
%
where
\begin{subequations}		\label{LONGSubEqsBaseSuperTsNTrois}\label{LONGEqDecompqprskXX}
	\begin{align}
		q_0&		&&=\frac{1}{ 4 }(X_{++}+X_{+-}+X_{-+}+X_{--})	&\in\sK\cap\sQ\cap\tilde\sN_2\\
		q_2&=[q_0,J_1]	&&=\frac{1}{ 4 }(-X_{++}-X_{+-}+X_{-+}+X_{--})	&\in\sP\cap\sQ\cap\tilde\sN_2\\
		p_1&=[q_0,q_1]	&&=\frac{1}{ 4 }(-X_{++}+X_{+-}-X_{-+}+X_{--})	&\in\sP\cap\sH\cap\tilde\sN_2	\label{LONGEqqzqupuDef}\\
		s_1&=[J_1,p_1]	&&=\frac{1}{ 4 }(-X_{++}+X_{+-}+X_{-+}-X_{--})	&\in\sK\cap\sH\cap\tilde\sN_2\\
		q_k&		&&=\frac{ 1 }{2}(X_{0+}^k-X_{0-}^k)	&\in\sP\cap\sQ\cap\tilde\sN_k	\\
		p_k&=[q_0,q_k]	&&=\frac{ 1 }{ 2 }(X_{-0}^k-X_{+0}^k)	&\in\sP\cap\sH\cap\tilde\sN_k	\\
		r_k&=[J_2,q_k]	&&=\frac{ 1 }{2}(X^k_{0+}+X^k_{0-})	&\in\sK\cap\sH\cap\tilde\sN_k	\\
		s_k&=[J_1,p_k]	&&=-\frac{ 1 }{2}(X^k_{-0}+X^k_{+0})	&\in\sK\cap\sH\cap\tilde\sN_k	\\
		r_{ij}&=[q_i,q_j]&&					&\in\sK\cap\sH\cap\sG_0.
	\end{align}
and
	\begin{align}
		J_1&\in\sP\cap\sH\cap\sA\\
		q_1=J_2&\in\sP\cap\sQ\cap\sA
	\end{align}
\end{subequations}
Notice that the elements $p_1$ and $s_1$ are non vanishing by lemma~\ref{LONGLemComJDeuxQ}. We have
\begin{subequations}
	\begin{align}
		\sP\cap\sQ\cap\tilde\sN_2&=\langle q_2\rangle	&	\sK\cap\sQ\cap\tilde\sN_2&=\langle q_0\rangle\\
		\sP\cap\sQ\cap\tilde\sN_k&=\langle q_k\rangle	&	\sK\cap\sQ\cap\tilde\sN_k&=\emptyset\\
		\sP\cap\sQ\cap\sA&=\langle J_2\rangle	&	\sK\cap\sQ\cap\sA&= \emptyset\\
		\sP\cap\sH\cap\tilde\sN_2&=\langle p_1\rangle	&	\sK\cap\sH\cap\tilde\sN_2&=\langle s_1\rangle\\
		\sP\cap\sH\cap\tilde\sN_k&=\langle p_k\rangle	&	\sK\cap\sH\cap\tilde\sN_k&=\langle r_k,s_k\rangle\\
		\sP\cap\sH\cap\sA&=\langle J_1\rangle	&	\sK\cap\sH\cap\sA&=\emptyset
	\end{align}
\end{subequations}

The decomposition of $\tilde\sN_2$ into $\sK\oplus\sP$ is
\begin{equation}		\label{LONGEqDecomptsNTroisKP}
	\tilde\sN_2=\langle q_0,s_1\rangle\oplus \langle q_1,p_1\rangle.
\end{equation}


The decomposition of $\tilde\sN_k$ into $\sK\oplus\sP$ is
\begin{equation}		\label{LONGEqDecomptsNkKP}
	\tilde\sN_k =\langle r_k,s_k,r_{ij}\rangle_{k,i,j\geq 3}\oplus \langle q_k,p_k\rangle.
\end{equation}


We are now going to compute all the Killing form and commutators in this basis.
\begin{proposition}		\label{LONGPropBprsk}
	We have
	\begin{equation}
		\begin{aligned}[]
			B(p_k,p_k)&=-B(q_0,q_0)\\
			B(r_k,r_k)&=B(q_0,q_0)\\
			B(s_k,s_k)&=B(q_0,q_0),
		\end{aligned}
	\end{equation}
	and then
	\begin{equation}	\label{LONGeqNormInHigherDimensionalSlices}
		-\| p_k \|^2=\| r_k \|^2=\| s_k \|^2=1.
	\end{equation}
\end{proposition}

\begin{proof}
	Using the definitions \eqref{LONGEqDecompqprskXX} and the theorem~\ref{LONGThoAdESqqq}, we have
	\begin{subequations}		\label{LONGSubEqsBpprrssk}
	\begin{equation}
			B(p_k,p_k)=B\big( \ad(q_k)q_0,\ad(q_k)q_0 \big)
					=-B\big( \ad(q_k)^2q_0,q_0 \big)
					=-B(q_0,q_0),
	\end{equation}
	and
	\begin{equation}
		B(r_k,r_k)=B\big( \ad(q_1)q_k,\ad(q_1)q_k \big)=-B(q_k,q_k)=B(q_0,q_0).
	\end{equation}
	and
	\begin{equation}
		B(s_k,s_k)=-B\big( \ad(J_1)^2p_k,p_k \big)=-B(p_k,p_k)=B(q_0,q_0).
	\end{equation}
	\end{subequations}
\end{proof}
Notice that we are not surprised by the positivity of the norms of $r_k$ and $s_k$ because they belong to the compact part of the algebra.

\begin{proposition}
	The Killing norm in the space $\mZ_{\sK}(\sA)$ are given by
	\begin{equation}
		B(r_{ij},r_{kl})=\begin{cases}
			B(q_0,q_0)	&	\text{if }\{ i,j \}=\{ k,l \}\\
			0	&	 \text{otherwise}.
		\end{cases}
	\end{equation}
\end{proposition}

\begin{proof}
	First, we have
	\begin{equation}
		\begin{aligned}[]
			B(r_{ij},r_{ij})&=B\big( [q_i,q_j],[q_i,q_j] \big)\\
					&=-B\big( \ad(q_i)^2q_j,q_j \big)\\
					&=-B(q_j,q_j)\\
					&=B(q_0,q_0).
		\end{aligned}
	\end{equation}
	For the mixed case we have
	\begin{equation}
			B(r_{ij},r_{ik})=-B\big( \ad(q_i)^2q_j,q_k \big)=0.
	\end{equation}
	We suppose now that $i$, $j$, $k$ and $l$ are four different numbers. The action of $\ad(r_{kl})$ on $\sA$ is zero because $r_{kl}\in\mZ(\sA)$. From \eqref{LONGEqComsRRN}, the action of $\ad(r_{ij})\circ\ad(r_{kl})$ on $\sN$ is zero. Since the elements $r_{ij}$ satisfy the algebra of $\so(n)$, we have $\ad(r_{ij})\circ\ad(r_{kl})r_{mn}=0$ when $i$, $j$, $k$ and $l$ are four different numbers. Finally we have
	\begin{equation}
		B(r_{ij},r_{kl})=0.
	\end{equation}
\end{proof}

\begin{lemma}		\label{LONGLempunjdeuxqzero}
	We have $[p_1,J_2]=q_0$.
\end{lemma}

\begin{proof}
	The lemma comes from theorem~\ref{LONGThoAdESqqq} because
	\begin{equation}
		[p_1,J_2]=-[q_1,p_1]=\ad(q_1)^2q_0=q_0.
	\end{equation}
\end{proof}

\begin{lemma}				 \label{LONGlemJDeuxqDeuxsUn}
	We have $s_1=[J_2,q_2]$.
\end{lemma}

\begin{proof}
	We use the definition $p_1=[q_0,q_1]$ and the Jacobi identity:
	\begin{equation}
		[J_1,p_1]=\big[ J_1,[q_0,q_1] \big]=-\big[ q_0,[q_1,J_1] \big]-\big[ q_1,[J_1,q_0] \big].
	\end{equation}
	The first terms vanishes because $q_1\in\sA$ while $[J_1,q_0]=-q_2$ by definition.
\end{proof}

\begin{proposition}		\label{LONGPropBJpsun}
	We have
	\begin{equation}
		\begin{aligned}[]
			B(J_1,J_1)&=-B(q_0,q_0)\\
			B(p_1,p_1)&=-B(q_0,q_0)\\
			B(s_1,s_1)&=B(q_0,q_0),
		\end{aligned}
	\end{equation}
	and then
	\begin{equation}
		-\| J_1 \|^2=-\| p_1 \|^2=\| s_1 \|^2=1.
	\end{equation}
\end{proposition}

\begin{proof}
	Using $J_1=[q_0,q_2]$ (lemma~\ref{LONGLemQzQdeuxJun}) we find
	\begin{equation}
		B(J_1,J_1)=B\big( \ad(q_2)q_0,\ad(q_2)q_0 \big)=-B\big( \ad(q_2)^2q_0,q_0 \big)=-B(q_0,q_0).
	\end{equation}
	In much the same way, using the definition of $p_1$ and $s_1=[q_1,q_2]$ (lemma~\ref{LONGlemJDeuxqDeuxsUn}), we find $B(p_1,p_1)=B(q_1,q_1)$ and $B(s_1,s_1)=-B(q_2,q_2)$.

\end{proof}

\begin{lemma}		\label{LONGLemJunrkzero}
	We have $[J_1,r_k]=0$.
\end{lemma}

\begin{proof}
	Using the definition of $r_k$ and the Jacobi identity,
	\begin{equation}
		[J_1,r_k]= \big[ J_1,[J_2,q_k] \big]=-\big[ J_2,[q_k,J_1] \big]-\big[ q_k,[J_1,J_2] \big]=0
	\end{equation}
	because of equation \eqref{LONGEqJUnqkzero} and the fact that $\sA$ is abelian.
\end{proof}

Now, the Killing norms of the basis $\eB$ can be computed.
\begin{theorem}		\label{LONGThoBaisXXorthoigher}
	The basis
	\begin{equation}
		\eB=\{ q_0,q_2,p_1,s_1,q_k,p_k,r_k,s_k,J_1,J_2 \}.
	\end{equation}
	given by the definitions \eqref{LONGEqSuperBaseeB}  is orthonormal and
	\begin{equation}
		\begin{aligned}[]
			\| J_1 \|^2=\| q_1 \|^2=\| q_2 \|^2=\| p_1 \|^2=\| q_k \|^2=\| p_k \|^2&=-1\\
			\| r_{ij} \|=\| q_0 \|^2=\| s_1 \|^2=\| r_k \|^2=\| s_k \|^2&=1
			\end{aligned}
	\end{equation}
\end{theorem}

\begin{proof}
	The norms are given by the propositions~\ref{LONGPropBprsk},~\ref{LONGPropBJpsun}, and~\ref{LONGPropBaseQOrtho}.

	For the orthogonality, we know that $\sP\perp\sK$, $\sQ\perp\sH$ (from general theory) as well as $\tilde\sN_2\perp\tilde\sN_k$, $\tilde\sN_2\perp\sA$ and $\tilde\sN_k\perp\sA$ (equations \eqref{LONGEqtsnkperprtsnkp}, \eqref{LONGEqAperpNTrois}, \eqref{LONGEqAperpNk} and \eqref{LONGEqNTtroisperpNk}). Thus, among the elements of the basis $\eB$, the two only ones that could not orthogonal are $r_k$ and $s_k$. However, we have $B(r_k,s_k)=0$ because
	\begin{equation}
		B(r_k,s_k)=B\big( r_k,\ad(J_1)p_k \big)=-B\big( \ad(J_1)r_k,p_k \big)=0
	\end{equation}
	by lemma~\ref{LONGLemJunrkzero}.

\end{proof}


It turns out that we are able to compute all the commutators using the following techniques
\begin{enumerate}

	\item
		the orthonormality of the basis, theorem~\ref{LONGThoBaisXXorthoigher} among with the $\ad$-invariance of the Killing form
	\item
		the Jacobi identity
	\item
		the theorem~\ref{LONGThoAdESqqq} and the commutators of lemmas~\ref{LONGLempunjdeuxqzero},~\ref{LONGlemJDeuxqDeuxsUn} and~\ref{LONGLemJunrkzero}.

\end{enumerate}
Computing all the commutators that way is quite long and very few interesting. The interesting point is that it is possible. You can immediately jump to subsection~\ref{LONGSubSecSomeExpo}.

\begin{enumerate}
	\item$\ad(q_0)J_1=q_2$. Definition.
	\item$\ad(q_0)J_2=p_1$. Definition.
	\item$\ad(p_1)J_1=-s_1$. Definition.
	\item$\ad(q_0)q_k=p_k$\label{LONGItemComqzpk}. Definition.
	\item$\ad(q_k)J_2=-r_k$. Definition.
	\item$\ad(p_k)J_1=-s_k$\label{LONGItemCompkJun}. Definition.
	\item$\ad(p_1)J_2=q_0$\label{LONGItemCompunJdeux}. Lemma~\ref{LONGLempunjdeuxqzero}.
	\item$\ad(q_2)J_2=-s_1$. Lemma~\ref{LONGlemJDeuxqDeuxsUn}.
	\item$\ad(J_1)r_k=0$\label{LONGItemComJunrk}. Lemma~\ref{LONGLemJunrkzero}.

	\item$\ad(q_0)p_1=-J_2$\label{LONGItemComqzpun}. We have $\ad(q_0)p_1=\ad(q_0)^2q_1=-q_1$.
	\item$\ad(q_0)s_1=0$. By the usual techniques, we get $[q_0,s_1]\in\sK\cap\sQ\cap\sA=\{ 0 \}$. The following few are obtained in the same way.
	\item$\ad(q_0)r_k=0$\label{LONGItemComqzrk}.
	\item$\ad(q_0)s_k=0$.
	\item$\ad(q_2)p_1=0$.
	\item$\ad(q_k)J_1=0$.
	\item$\ad(q_2)p_k=0$.
	\item$\ad(q_2)p_k=0$.
	\item$\ad(p_1)q_k=0$\label{LONGItemCompunqk}.
	\item$\ad(J_2)p_k=0$.
	\item$\ad(q_2)s_1=-J_2$. We know that $[q_2,s_1]\in\sP\cap\sQ\cap\sA=\langle J_2\rangle$. Thus $[q_0,s_1]$ is a multiple of $J_2$. The coefficient is given by$B\big( [q_2,s_1],J_2 \big)/B(J_2,J_2)$. Using the $\ad$-invariance of the Killing form, we are left to compute $B(s_1,\ad(q_2)J_2)$. Lemma~\ref{LONGlemJDeuxqDeuxsUn} shows then that the coefficient we are searching if $B(s_1,s_1)/B(J_2,J_2)=1$.
	\item\label{LONGItemComqdeuxqk}$\ad(q_2)q_k=-s_k$. The spaces show that $[q_2,q_k]\in\sK\cap\sH\cap\tilde\sN_k=\langle r_k,s_k\rangle$. Thus we have to check the two possible components. First, $B\big( [q_2,q_k],r_k \big)=-B\big( q_2,\ad(q_k)^2J_2 \big)=0$ by theorem~\ref{LONGThoAdESqqq}. For the second, we use the definition of $s_k$ and the Jacobi identity:
		\begin{equation}
			\begin{aligned}[]
				B\big( [q_2,q_k],s_k \big)&=B\big( [q_2,q_k],\ad(J_1)[q_0,q_k] \big)\\
				&=B\big( [q_2,q_k],-\ad(q_0)\underbrace{[q_k,J_1]}_{=0}-\ad(q_k)\underbrace{[J_1,q_0]}_{=-q_2} \big)\\
				&=-B\big( \ad(q_2)q_k,\ad(q_2)q_k \big)\\
				&=B\big( q_k,\ad(q_2)^2q_k \big)\\
				&=B(q_k,q_k).
			\end{aligned}
		\end{equation}
		Finally, what we have is
		\begin{equation}
			[q_2,q_k]=\frac{ B(q_2,q_2) }{ B(s_k,s_k) }s_k=-s_k.
		\end{equation}
	\item$\ad(q_2)r_k=0$. From the spaces, $[q_2,r_k]\in\langle q_k\rangle$, but
		\begin{equation}
			B\big( \ad(q_2)r_k,q_k \big)=-B\big( r_k,[q_2,q_k] \big)=B(r_k,s_k)=0.
		\end{equation}
	\item$\ad(q_2)s_k=-q_k$. From the spaces, $[q_2,s_k]\in\sP\cap\sQ\cap\tilde\sN_k=\langle q_k\rangle$, so we compute
		\begin{equation}
			B\big( [q_2,s_k],q_k \big)=-B(s_k,[q_2,s_k])=B(s_k,s_k),
		\end{equation}
		and
		\begin{equation}
			[q_2,s_k]=\frac{ B(s_k,s_k) }{ B(q_k,q_k) }q_k=-q_k.
		\end{equation}
	\item$\ad(q_2)r_k=0$. From the spaces, $[q_k,q_2]\in\sP\cap\sQ\cap\tilde\sN_k=\langle q_k\rangle$. Thus we compute
		\begin{equation}
			B\big( [q_2,r_k],q_k \big)=-B\big( r_k,[q_2,q_k] \big)=B(r_k,s_k)=0
		\end{equation}
		where we used the item~\ref{LONGItemComqdeuxqk}.
	\item\label{LONGItemComjunpun}$\ad(J_1)p_1=s_1$. From the spaces, $[J_1,p_1]\in\langle s_1\rangle$. We have
		\begin{equation}
			B\big( [J_1,p_1],s_1 \big)=-B\big( p_1,[J_1,s_1] \big)=-B\big( p_1,\ad(J_1)^2p_1 \big)=-B(p_1,p_1),
		\end{equation}
		thus
		\begin{equation}
			[J_1,p_1]=-\frac{ B(p_1,p_1) }{ B(s_1,s_1) }s_1=s_1.
		\end{equation}
	\item\label{LONGItemComskqk}$\ad(s_k)q_k=-q_2$. From the spaces, $[s_k,q_k]\in\langle J_2,q_2\rangle$. Thus we have two Killing forms to compute. The first is
		\begin{equation}
			B\big( [s_k,q_k],J_2 \big)=B\big( s_k,[q_k,J_2] \big)=B(s_k,r_k)=0
		\end{equation}
		were we used the definition of $r_k$. The second is
		\begin{equation}
			B\big( [s_k,q_k],q_2 \big)=B\big( s_k,[q_k,q_2] \big)=B(s_k,s_k).
		\end{equation}
		where we used item~\ref{LONGItemComqdeuxqk}. Thus we have
		\begin{equation}
			[s_k,q_k]=\frac{ B(s_k,s_k) }{ B(q_2,q_2) }q_2=-q_2
		\end{equation}
	\item$\ad(s_k)J_2=0$. From the spaces, $[s_k,J_2]\in\langle q_k\rangle$, but
		\begin{equation}
			B\big( \ad(s_k)J_2,q_k \big)=-B\big( J_2,[s_k,q_k] \big)=-B(J_2,q_2)=0
		\end{equation}
		where we used item~\ref{LONGItemComskqk}.
	\item$\ad(q_2)s_k=-q_k$. From the spaces, $[q_2,s_k]\in\langle q_k\rangle$. We have
		\begin{equation}
			B\big( \ad(q_2)s_k,q_k \big)=-B\big( s_k,\ad(q_2)q_k \big)=B(s_k,s_k)
		\end{equation}
		where we used item~\ref{LONGItemComqdeuxqk}. Thus
		\begin{equation}
			[q_2,s_k]=\frac{ B(s_k,s_k) }{ B(q_k,q_k) }q_k=-q_k.
		\end{equation}
	\item$\ad(p_1)s_1=-J_1$. From the spaces, $[p_1,s_1]\in\langle J_1\rangle$. We have
		\begin{equation}
			B\big( [p_1,s_1],J_1 \big)=-B\big( s_1,[p_1,J_1] \big)=B(s_1,s_1),
		\end{equation}
		where we used item~\ref{LONGItemComjunpun}. Thus,
		\begin{equation}
			[p_1,s_1]=\frac{ B(s_1,s_1) }{ B(J_1,J_1) }J_1=-J_1.
		\end{equation}
	\item$\ad(p_1)r_k=p_k$\label{LONGItemCompunrk}. We use the definition of $r_k$ and Jacobi:
		\begin{equation}
			[p_1,r_k]=\big[ p_1,[J_2,q_k] \big]=-\big[ J_2,\underbrace{[q_k,p_1]}_{=0} \big]-\big[ q_k,\underbrace{[p_1,J_2]}_{=q_0} \big]=-[q_k,q_0]=p_k.
		\end{equation}
		where we used items~\ref{LONGItemCompunqk},~\ref{LONGItemCompunJdeux} and~\ref{LONGItemComqzpk}.
	\item$\ad(q_k)J_2=-r_k$\label{LONGItemComkJdeux}. From the spaces, $[q_k,J_2]\in\langle r_k,s_k\rangle$. First, we have
		\begin{equation}
			B\big( [q_k,q_2],s_k \big)=-B\big( J_2,\underbrace{[q_k,s_k]}_{=-q_2} \big)=0
		\end{equation}
		where we used item~\ref{LONGItemComskqk}. Secondly we have
		\begin{equation}
			B\big( [q_k,J_2],r_k \big)=B\big( q_k,\underbrace{[J_2,r_k]}_{\ad(J_2)^2q_k} \big)=B(q_k,q_k)
		\end{equation}
		Thus
		\begin{equation}
			[q_k,J_2]=\frac{ B(q_k,q_k) }{ B(r_k,r_k) }r_k=-r_k.
		\end{equation}
	\item$\ad(p_1)p_k=r_k$\label{LONGItemCompunpk}. We use the definition of $p_k$ and the Jacobi identity:
		\begin{equation}
			\begin{aligned}[]
				[p_1,p_k]&=-\big[ q_0,\underbrace{[q_k,p_1]}_{=0} \big]-\big[ q_k,\underbrace{[p_1,q_0]}_{=J_2} \big]\\
					&=-[q_k,J_2]\\
					&=r_k
			\end{aligned}
		\end{equation}
		where we used items~\ref{LONGItemCompunqk},~\ref{LONGItemComqzpun} and~\ref{LONGItemComkJdeux}.
	\item$\ad(p_1)s_k=0$. From the spaces, $[p_1,s_k]\in\langle p_k\rangle$. We have
		\begin{equation}
			B\big( [p_1,s_k],p_k \big)=-B\big( s_k,\underbrace{[p_1,p_k]}_{=0} \big)=0
		\end{equation}
		where we used the item~\ref{LONGItemCompunpk}.
	\item$\ad(s_1)q_k=0$. From the spaces, $[s_1,q_k]\in\langle q_k\rangle$. We have
		\begin{equation}
			B\big( [s_1,q_k],q_k \big)=B\big( s_1,[q_k,q_k] \big)=0.
		\end{equation}
	\item$\ad(s_1)p_k=0$\label{LONGItemComsunpk}. From the spaces, $[s_1,p_k]\in\langle p_k\rangle$. We have
		\begin{equation}
			B\big( [s_1,p_k],p_k \big)=B\big( s_1,[p_k,p_k] \big)=0.
		\end{equation}
	\item$\ad(s_1)r_k=s_k$. From the spaces, $[s_1,q_k]\in\langle r_k,s_k\rangle$. Using Jacobi,
		\begin{equation}
			\begin{aligned}[]
				[s_1,r_k]&=\big[ [J_1,p_1],r_k \big]\\
				&=-\big[ \underbrace{[p_1,r_k]}_{=p_k},J_1 \big]-\big[ \underbrace{[r_k,J_1]}_{=0},p_1 \big]\\
					&=-[p_k,J_1]\\
					&=s_k
			\end{aligned}
		\end{equation}
		where we used items~\ref{LONGItemCompunrk},~\ref{LONGItemComJunrk} and the definition of $s_k$.
	\item$\ad(s_1)J_1=-p_1$\label{LONGItemComsunJun}. We have $[s_1,J_1]=-\ad(J_1)^2p_1=-p_1$ by theorem~\ref{LONGThoAdESqqq}.
	\item$\ad(s_1)s_k=-r_k$\label{LONGItemComsunsk}. We use the definition of $s_k$ and Jacobi:
		\begin{equation}
			\begin{aligned}[]
				[s_1,s_k]&=\big[ s_1,[J_1,p_k] \big]\\
				&=-\big[ J_1,\underbrace{[p_k,s_1]}_{=0} \big]-\big[ p_k,\underbrace{[s_1,J_1]}_{=-p_1} \big]\\
					&=[p_k,p_1]\\
					&=-r_k
			\end{aligned}
		\end{equation}
		where we used items~\ref{LONGItemComsunpk},~\ref{LONGItemComsunJun} and~\ref{LONGItemCompunpk}.
	\item$\ad(J_2)s_k=0$. From the spaces, $[J_2,s_k]\in\langle q_k\rangle$. We have
		\begin{equation}
			B\big( [J_2,s_k],q_k \big)=B\big( J_2,\underbrace{[s_k,q_k]}_{=-q_2} \big)=0
		\end{equation}
		where we used item~\ref{LONGItemComskqk}.
	\item$\ad(q_k)p_k=-q_0$. Using the definition of $p_k$ and the theorem~\ref{LONGThoAdESqqq},
		\begin{equation}
			[q_k,p_k]=\big[ q_k[q_0,q_k] \big]=-\ad(q_k)^2q_0=-q_0.
		\end{equation}
	\item$\ad(q_k)r_k=-J_2$\label{LONGItemComqkrk}. Using the definition of $r_k$ and theorem~\ref{LONGThoAdESqqq}, we have
		\begin{equation}
			\big[ q_k,[J_2,q_k] \big]=-\ad(q_k)^2J_2=-J_2.
		\end{equation}
	\item$\ad(p_k)r_k=-p_1$. Using the definition of $r_k$ and Jacobi,
		\begin{equation}
			\begin{aligned}[]
				[p_k,r_k]&=-\big[ pk,[q_0,q_k] \big]\\
				&=\big[ q_0,\underbrace{[q_k,r_k]}_{=-J_2} \big]+\big[ q_k,\underbrace{[r_k,q_0]}_{=0} \big]\\
					&=[J_2,q_0]\\
					&=-p_1
			\end{aligned}
		\end{equation}
		where we used the items~\ref{LONGItemComqkrk},~\ref{LONGItemComqzrk} and the definition of $p_1$.
	\item$\ad(p_k)s_k=-J_1$. The spaces show that $[p_k,s_k]\in\langle J_1,p_1\rangle$. We have
		\begin{equation}
			B\big( [p_k,s_k],J_1 \big)=-B\big( s_k,\underbrace{[p_k,s_1]}_{=-s_k} \big)=B(s_k,s_k)
		\end{equation}
		and
		\begin{equation}
			B\big( [p_k,s_k],p_1 \big)=-B\big( s_k,\underbrace{[p_k,p_1]}_{=-r_k} \big)=0
		\end{equation}
		where we used the items~\ref{LONGItemCompkJun} and~\ref{LONGItemCompunpk}. Thus
		\begin{equation}
			[p_k,s_k]=\frac{ B(s_k,s_k) }{ B(J_1,J_1) }J_1=-J_1.
		\end{equation}
	\item$\ad(r_k)s_k=s_1$. From the spaces, $[r_k,s_k]\in\langle s_1\rangle$. We have
		\begin{equation}
			B\big( [r_k,s_k],s_1 \big)=B\big( r_k,\underbrace{[s_k,s_1]}_{=r_k} \big)=B(r_k,r_k)
		\end{equation}
		where we used item~\ref{LONGItemComsunsk}. Thus
		\begin{equation}
			[r_k,s_k]=\frac{ B(r_k,r_k) }{ B(s_1,s_1) }s_1=s_1.
		\end{equation}
	\item$\ad(J_1)q_2=-q_0$. Using the definition of $q_2$ and theorem~\ref{LONGThoAdESqqq},
		\begin{equation}
			[J_1,q_2]=-\ad(J_1)^2q_0=-q_0.
		\end{equation}
	\item$\ad(J_1)s_k=p_k$. From the spaces, $[J_1,s_k]\in\langle p_k\rangle$. We have
		\begin{equation}
			B\big( [J_1,s_k],p_k \big)=-B\big( s_k,\underbrace{[J_1,p_k]}_{=s_k} \big)=-B(s_k,s_k)
		\end{equation}
		where we used the definition of $s_k$.
	\item$\ad(J_2)p_1=-q_0$\label{LONGItemComJdeuxpun}. Using the definition of $p_1$ and theorem~\ref{LONGThoAdESqqq}, $[J_2,p_1]=-\ad(q_1)^2q_0=-q_0$.
	\item$\ad(J_1)s_1=q_2$. We use the definition of $s_1$ and Jacobi:
		\begin{equation}
			\begin{aligned}[]
				[J_2,s_1]&=\big[ J_2,[J_1,p_1] \big]\\
				&=-\big[ J_1,\underbrace{[p_1,J_2]}_{=q_0} \big]-\big[ p_1,\underbrace{[J_2,q_1]}_{=0} \big]\\
				&=-[J_1,q_0]\\
				&=q_2
			\end{aligned}
		\end{equation}
		where we used item~\ref{LONGItemComJdeuxpun} and the definition of $q_2$.
	\item$\ad(J_2)r_k=q_k$. Using the definition of $r_k$ and theorem~\ref{LONGThoAdESqqq}, $[J_2,r_k]=\ad(q_1)^2q_k=q_k$.
\end{enumerate}

\begin{proposition}
	We have $\sH=[\sQ,\sQ]$.
\end{proposition}

\begin{proof}
	The inclusion $[\sQ,\sQ]\subset\sH$ is by construction. Now every elements in the basis \eqref{LONGAlignPremDefHH} can be expressed in terms of commutators in $\sQ$ because
	\begin{subequations}
		\begin{align}
			J_1&=[q_0,q_2]\\
			s_k&=[q_k,q_2]\\
			s_1&=[J_2,q_2]	\label{LONGEqJdqdsu}
		\end{align}
	\end{subequations}
\end{proof}

%
\subsection{Properties of the basis}
%


The basis $\eB$ is motivated by the fact that $\ad(q_0)^2q_k=-q_k$, so that $ e^{\ad(xq_0)}$ is easy to compute on $q_k$ and $p_k$. Moreover, $r_k$ and $s_k$ belong to $\sK$, so that $[q_0,r_k]=[q_0,s_k]=0$. The drawback of that decomposition is that the basis elements do not belong to $\sN$ or $\bar\sN$ while it will be useful to have basis elements in $\sN$ and $\bar\sN$, among other for theorem~\ref{LONGThoOrbitesOuverttes}.

At a certain point, we are going to compute the exponentials $ e^{\ad(xq_0)}X$ when $X$ runs over $\sN$ and $\bar\sN$. We are going to extensively use the commutation relations listed in \eqref{LONGsubEqsGenPySO}, \eqref{LONGSubEqsPlusPresPySO} and \eqref{LONGSubEqsThethaPySO}. A particular attention will be devoted to the projection over $\sQ$ which will be central in determining the open and closed orbits of $AN$ in $G/H$.

\begin{lemma}		\label{LONGLemDecomptsNDanseB}
	The decomposition of $\tilde\sN_2$ with respect to $\eB$ is
	\begin{subequations}		\label{LONGSubeqsDecompXqps}
		\begin{align}
			X_{++}&=q_0-q_2-p_1-s_1\\
			X_{+-}&=q_0-q_2+p_1+s_1
		\end{align}
	\end{subequations}
	and the decomposition of $\tilde\sN_k$ with respect to $\eB$ is
	\begin{subequations}
		\begin{align}
			X_{0+}^k & = q_k+r_k	&    X_{0-}^k&=-q_k+r_k		\label{LONGsubEqqrkXzpegal}	\\
			X_{+0}^k & = -p_k-s_k	&	X_{-0}^k & = p_k-s_k
		\end{align}
	\end{subequations}

\end{lemma}

\begin{proof}


	Using known commutator and the fact that $[\ad(J_1),\ad(J_2)]=0$ on $\tilde\sN_2$, we find the following commutators:
	\begin{subequations}
		\begin{align}
			[J_1,q_0]&=-q_2	&&	&[J_2,q_0]&=-p_1			\label{LONGsubEqJunqzJdeuxqzmoinspun}\\
			[J_1,q_2]&=-q_0	&&	&[J_2,q_2]&=s_1\label{LONGsubEqJunqzJdeuxqzmoinspdeux}\\
			[J_1,p_1]&=s_1	&&	&[J_2,p_1]&=-q_0\\
			[J_1,s_1]&=p_1	&&	&[J_2,s_1]&=q_2.
		\end{align}
	\end{subequations}
	From these properties, we deduce that $q_0-q_2-p_1-s_1$ is proportional to $X_{++}$. Since, by definition, $q_0$ is the $\sK\sQ$-component of $X_{++}$, the proportionality factor is $1$. We also know that  $X_{+-}$ is proportional to $q_0-q_2+p_1+s1$. Since $q_0-q_2=(X_{++})_{\sQ}=(X_{+-})_{\sQ}$ (proposition~\ref{LONGPropXmpXppqq}), the proportionality coefficient is $1$. The relations \eqref{LONGSubeqsDecompXqps} are now proved.

	For the elements of $\tilde\sN_k$, the commutation relations give
	\begin{subequations}
		\begin{align}
			[J_1,q_k+r_k]&=0		&[J_1,p_k-s_k]&=s_k-p_k\\
			[J_2,q_k+r_k]&=q_k+r_k		&[J_2,p_k-s_k]&=0
		\end{align}
	\end{subequations}
	so that
	\begin{equation}
		\begin{aligned}[]
			q_k+r_k&\propto X_{0+}^k\in\sN\\
			s_k+p_k&\propto X^k_{+0}\in\sN\\
			p_k-s_k&\propto X^k_{-0}\in\bar\sN
		\end{aligned}
	\end{equation}

	We have $r_k=[J_2,q_k]\in\sK\cap\sH$, so that the $\sP$-component of $q_k+r_k$ is $q_k$. But $q_k=(X_{0+}^k)_{\sP}$ is the $\sP$-component of $X_{0+}^k$. The proportionality between $q_k+r_k$ and $X_{0+}^k$ together with the equality of their $\sP$-component provide the equality \eqref{LONGsubEqqrkXzpegal}.

	For the two other, let us suppose that
	\begin{subequations}
		\begin{align}
			X_{+0}^k&=a(p_k+s_k)\\
			X_{-0}^k&=b(p_k-s_k).
		\end{align}
	\end{subequations}
	In this case, we have
	\begin{equation}
			(X_{+0}^k)_{\sP}=\frac{ 1 }{2}(X_{+0}^k-\theta X_{+0}^k)
				=\frac{ 1 }{2}\big( (a-b)p_k+(a+b)s_k \big),
	\end{equation}
	so that $a=-b$ because $s_k\in \sK$. Now let us look at the $\sK\sQ$-component of the equality $[X_{+0}^k,X_{0+}^k]=-X_{++}$ taking into account the fact that $X_{+0}^k\in\sH$ and $(X_{0+}^k)_{\sK\sQ}=0$. What we have is $\big[ (X_{+0}^k)_{\sP\sH},(X_{0+}^k)_{\sP\sQ} \big]=-q_0$, but $(X_{0+}^k)_{\sP}=q_k$ and $(X_{+0}^k)_{\sP}=ap_k$, so that $[ap_k,q_k]=-q_0$. If we replace $p_k$ by its definition $[q_0,q_k]$, we get
	\begin{equation}
		a\big[ [q_0,q_k],q_k \big]=a\ad(q_k)^2q_0=-q_0,
	\end{equation}
	so that $a=-1$.

	The last point comes from $X_{0-}^k=\theta X_{0+}^k$.

\end{proof}
Notice that this result was already obvious from the decompositions given in \eqref{LONGEqDecompqprskXX}.


%
\subsection{Some exponentials}
%
\label{LONGSubSecSomeExpo}

It will be important to compute the element $ e^{\ad(xq_0)}X$ when $X$ runs over the vectors of $\eB$.
%
The action of $e^{xq_0}$ on $\sA$ is
\begin{subequations}			\label{LONGSubEqsAdxqzJJ}
	\begin{align}
		e^{\ad(xq_0)}J_1=\cos(x)J_1+\sin(x)q_2\\
		e^{\ad(xq_0)}J_2=\sin(x)p_1+\cos(x)q_1,
	\end{align}
\end{subequations}
on $\tilde\sN_2$ we have
\begin{subequations}					\label{LONGSubEqsexpxzqpsdzuu}
	\begin{align}
		e^{\ad(xq_0)}q_2&=\cos(x)q_2-\sin(x)J_1\\
		e^{\ad(xq_0)}q_0&=q_0\\
		e^{\ad(xq_0)}p_1&=\cos(x)p_1-\sin(x)q_1 		\label{LONGEqAdqzpUn}\\
		e^{\ad(xq_0)}s_1&=s_1,
	\end{align}
\end{subequations}
and the action on $\tilde\sN_k$ is
\begin{subequations}		\label{LONGEqExpAdqkpk}
	\begin{align}
		e^{\ad(xq_0)}q_k&=\cos(x)q_k+\sin(x)p_k\\
		e^{\ad(xq_0)}p_k&=\cos(x)p_k-\sin(x)q_k\\
		e^{\ad(xq_0)}p_k&=\cos(x)p_k-\sin(x)q_k.
	\end{align}
\end{subequations}

Combining with lemma~\ref{LONGLemDecomptsNDanseB},
\begin{subequations}			\label{LONGEqExpoQzSurNk}
	\begin{align}
		e^{\ad(xq_0)}X_{0+}^k=e^{\ad(xq_0)}(q_k+r_k)&=r_k+\cos(x)q_k+\sin(x)p_k		\label{LONGEqExpoQzSurNka}\\
		e^{\ad(xq_0)}X_{+0}^k=-e^{\ad(xq_0)}(s_k+p_k)&=-s_k-\cos(x)p_k+\sin(x)q_k.\label{LONGEqExpoQzSurNkb}
	\end{align}
\end{subequations}
The same way,
\begin{subequations}			\label{LONGEqExpoQzN}
	\begin{align}
		e^{\ad(xq_0)}X_{++}&=q_0+\sin(x)q_1-\cos(x)q_2+\sin(x)J_1-\cos(x)p_1\\
		e^{\ad(xq_0)}X_{+-}&=q_0-\sin(x)q_1-\cos(x)q_2+\sin(x)J_1+\cos(x)p_1.
	\end{align}
\end{subequations}
The projections on $\sQ$ of all these combinations are immediate.

%
\subsection{Classification of the basis by the spaces}
%

The basis \eqref{LONGEqDecompqprskXX} allows to generalize the theorem~\ref{LONGThoAdESqqq}.

By very definition, we have $\ad(\sP)^2\sP\subset\sP$, $\ad(\sP)^2\sK\subset\sK$ and the same for the couple $(\sH,\sQ)$. The relations \eqref{LONGEqsCommWithtsNDeuxkA} say that the same is true with the triple $(\tilde\sN_2,\tilde\sN_k,\sA)$, i.e. $\ad(\sA)^2\tilde\sN_k\subset\tilde\sN_k$ and $\tilde\sN_k(\tilde\sN_2)\subset\tilde\sN_2$.

\begin{theorem}		\label{LONGThoAdSqIouZero}
	The basis $\eB$ has the property to be stable under the commutators: $[X,Y]\in\{0,\pm\eB\}$ when $X,Y\in\eB$. Moreover, we have
	\begin{equation}		\label{LONGEqadXsqYzYmY}
		\ad(X)^2Y=\begin{cases}
			0	&	\text{if }\ad(X)Y=0\\
			Y	&	\text{if }X\in\sP\\
			-Y	&	 \text{if }X\in\sK.
		\end{cases}
	\end{equation}
\end{theorem}

\begin{proof}
	This theorem can be immediately checked using the commutators. It is however instructive to see that equation \eqref{LONGEqadXsqYzYmY} can be checked from few considerations. First, remark that, if we look at the decompositions $\sG=\sQ\oplus\sH=\sP\oplus\sK=\mZ_{\sK}(\sA)\oplus\sA\oplus\tilde\sN_2\oplus\tilde\sN_k$, the commutation relations \eqref{LONGEqsCommWithtsNDeuxkA}, we find
	\begin{equation}
		\ad(\tilde\sN_2)\circ\ad(\tilde\sN_2)\colon
		\begin{cases}
			\tilde\sN_2\to\sA\to\tilde\sN_2\\
			\tilde\sN_k\to\tilde\sN_k\to\tilde\sN_k\\
			\sA\to\tilde\sN_2\to\sA,
		\end{cases}
	\end{equation}
	\begin{equation}
		\ad(\tilde\sN_k)\circ\ad(\tilde\sN_k)\colon
		\begin{cases}
			\tilde\sN_2\to\tilde\sN_k\to\sA\oplus\tilde\sN_2\\
			\tilde\sN_k\to(\sA\oplus\tilde\sN_2)\to\sA\oplus\tilde\sN_k\\
			\sA\to\tilde\sN_k\to\sA\oplus\tilde\sN_2,
		\end{cases}
	\end{equation}
	\begin{equation}
		\ad(\sA)\circ\ad(\sA)\colon
		\begin{cases}
			\tilde\sN_2\to\tilde\sN_2\to\tilde\sN_2\\
			\tilde\sN_k\to\tilde\sN_k\to\tilde\sN_k\\
			\sA\to 0.
		\end{cases}
	\end{equation}
	Thus we have $\ad(X)^2Y=\lambda Y$ whenever we are not in the cases $(X,Y)\in(\tilde\sN_k,\tilde\sN_2)$ and $(X,Y)\in(\tilde\sN_k,\tilde\sN_k)$. We should check that these cases cannot bring a $\sA$-component. We should also check that, since $\sK\cap\sH\cap\tilde\sN_k$, is two dimensional, $\ad(X)^2r_k$ has no $s_k$-component as well as $\ad(X)^2s_k$ has no $r_k$-component.

	Let us suppose that the spaces fit. We have $\ad(X)^2Y=\lambda Y$ and we still have to check the values of $\lambda$. Since $\eB$ is orthonormal, we have
	\begin{equation}
		\lambda=\frac{ B\big( \ad(X)^2Y,Y \big) }{ B(Y,Y) }=-\frac{ B\big( \ad(X)Y,\ad(X)Y \big) }{ B(Y,Y) }.
	\end{equation}
	First, remark that this is zero if and only if $\ad(X)Y=0$. Now, there are $4$ possibilities following that $X,Y$ belong to $\sP$ or $\sK$ because we know that $B|_{\sK}<0$ and $B|_{\sP}>0$. The result is that $\lambda$ is positive when $X\in\sP$ and negative when $X\in\sK$. Now, the fact that $\eB$ is stable under the commutators implies that $\ad(X)^2Y\in\{ 0,\pm \eB \}$, so that $\lambda\in\{ 0,1,-1 \}$.
\end{proof}
