% This is part of (almost) Everything I know in mathematics
% Copyright (c) 2013-2014
%   Laurent Claessens
% See the file fdl-1.3.txt for copying conditions.

Bibliography for Clifford algebras, spin group and related topics are \cite{memP,Michelson,Witkowski,mellor,ResEtaDiracType}. More algebraic point of view  can be found in \cite{Fult,Chevalley}. More details about ``square rooting'' second order differential operators are in \cite{Bronn}. For physical concerns, the reader should refer to \cite{Weinberg,Peskin,schwabl}. 

\section{Invitation: Clifford algebra in quantum field theory}\index{quantum!field theory}
%++++++++++++++++++++++++++++++++++++++++++++++++

\label{Secqft}
\subsection{Schrödinger, Klein-Gordon and Dirac}
%----------------------------------

The origin of the Klein-Gordon equation is almost the same as the one of the Schrödinger: one replace physical functions by operators. For a free particle, the correspondence are
\begin{align*}
 \textrm{energy}&& E&\rightarrow i\hbar\dsd{}{t},\\
 \textrm{momentum}&& \overline{p}&\rightarrow -i\hbar\overline{ \nabla }.
\end{align*}
The Schrödinger equation\index{equation!Schrödinger} (which is the non relativistic quantum wave equation) comes from replacement in the non non relativistic expression of the Hamiltonian
\[
E=\frac{\overline{p}^2}{2m}\longrightarrow\left(\partial_t-\frac{i\hbar}{2m}\nabla^2\right)\psi=0,
\]
while the Klein-Gordon\index{equation!Klein-Gordon} one (which is the relativistic quantum wave equation) comes from the relativistic corresponding equation:
\[
E^2=\overline{p}^2c^2+m^2c^4\longrightarrow\left(\partial\hmu\partial_{\mu}+(\frac{mc}{\hbar})^2\right)\psi=0.
\]

This is a second order differential equation; there are however no ``law of nature''{} which forbid a first order equation. We try
\[
 i\hbar\dsd{\psi}{t}
 =\left(\frac{\hbar c}{i}\alpha^k\partial_k+\beta mc^2\right)\psi\equiv \hat{H}\psi.
\]

There are some physical constraints on the coefficients $\alpha^k$ and $\beta$. We will study one of them: we want the components of $\psi$ to satisfy the Klein-Gordon equation, so that the plane waves fulfill the fundamental relation $E^2=p^2c^2+m^2c^4$.

In order to see the implications of this constraint on the coefficients, we apply two times the operator $\hat{H}$, and we compare the result with the Klein-Gordon equation. We find:
\begin{subequations}
\begin{align}
 \alpha^i\alpha^j+\alpha^j\alpha^i&=2\delta^{ij}\mtu,\\
 \alpha^i\beta+\beta\alpha^i&=0,\\
 (\alpha^i)^2=\beta^2&=\mtu.
\end{align}
\end{subequations}
%
If we define $\gamma^0=\beta$ and $\gamma^i=\beta\alpha^i$, we find that the matrices $\gamma^{\mu}$ have to give a representation of the Clifford algebra\footnote{Don't be afraid with the extra minus sign: the quantum field theory is most written with the metric $(+,-,-,-)$ instead of $(-,+,+,+)$.}:
\begin{equation}\label{cliffphys}
	\gamma\hmu\gamma\hnu+\gamma\hnu\gamma\hmu=2\eta^{\mu\nu}\mtu.
\end{equation}
The Dirac equation reads
\[
\left(-i\gamma\hmu\partial_{\mu}+\frac{mc}{\hbar}\right)\psi=0.
\]
If we want to perform some computation with the quantum field theory, we need an explicit form for the $\gamma$'s; that's the reason why we study representations of the Clifford algebra. The \defe{Dirac operator}{dirac!operator} $\Dir$ is the operator which lies in the Dirac equation:
\begin{equation}
 \label{dirflat}\Dir=\sum_{\mu=0}^3\gamma\hmu\dsd{}{x\hmu}.
\end{equation}


\begin{definition}
Let $V$ be a (finite dimensional) vector space and $q$, a bilinear quadratic form over $V$. The \defe{Clifford algebra}{Clifford!algebra}\index{algebra!Clifford} $\Cliff(V,q)$ is the unital associative algebra generated by $V$ subject to the relation
\begin{equation}\label{501r1}
       v\cdot v=q(v)
\end{equation}
for all $v$ in $\Cliff(V,q)$. Here the dot denotes the algebra product and $q(v)$ means $q(v,v)$.
\end{definition}
Theorem proves \ref{tho_Cliffunif} proves unicity of such an algebra, so that it makes sense.

\begin{remark}
The relation \eqref{501r1} is no more a restriction for the elements in $\Cliff(V,q)$ than a restriction on the choice of the algebra product.
\end{remark}


An explicit construction of $\Cliff(V,q)$ can be achieved in the following way. We consider the tensor algebra $T(V)=\bigoplus_{n\geq 0}\left(\otimes^nV\right)=\eC\oplus V\oplus(V\otimes V)\oplus\ldots$ over $V$ the two-sided ideal $\mI$ generated by elements of the form $v\otimes v-q(v)1$. The  \defe{Clifford algebra}{Clifford!algebra}\index{algebra!Clifford} for $(V,q)$ is given by\nomenclature[G]{$\Cliff(p,q)$}{Clifford algebra of $\eR^{1,3}$}
\begin{equation}	\label{defI}
	\Cliff(p,q):=T(V)/\mI
\end{equation}
in which product of $\Cliff(V,q)$ is naturally defined by $[a]\otimes[b]=[a\otimes b]$ if $[a]$ is the class of $a\in T(V)$.

Let us now fix some notations more adapted to what we want to do. Let $V=\eR^{p,q}$ the vector space $\eR^{p+q}$ endowed with a diagonal metric which contains $p$ plus sign and $q$ minus signs. For $v$, $w\in V$, the inner product with respect to the metric $\eta$ of $v$ by $w$ will be denoted by $\eta(v,w)$.  The norm on $V$ will be defined by $\|v\|^2=-\eta(v,v)$. It is neither positive defined, nor negative defined. The explanation of the minus sign will come soon. The Clifford algebra is the quotient $\Cliff(p,q):=T(V)/\mI$ of the tensor algebra by the two-sided ideal $\mI$ generated by elements of the form
\[
	(v\otimes w)\oplus (w\otimes v)\oplus 2\eta(v,w)1
\]
 for $v,w$ in $V$. Depending on the context, we will often use the notations $\Cliff(\eta)$ or $\Cliff(V)$ or $\Cliff(p,q)$. The algebra product is $[x]\cdot[y]=[x\otimes y]$, $x$, $y\in T(V)$.  As long as $z\in V\subset\Cliff(p,q)$, the expression $\eta(z,z)$ is meaningful. The definition of $\Cliff$ is such that $z\cdot z=-\eta(z,z)$. This leads to the somewhat surprising formula  $z^2=\|z\|^2=-\eta(z,z)$.

\subsection{First representation}
%----------------------------------

Let $(V,g)$ be a metric vector space and $\Cliff(V,g)$ its Clifford algebra. For each $v\in V$, we define the two following elements of $\End_{\eR}(\Wedge V)$:
\begin{subequations}
\begin{align}
\epsilon(v)\big( u_1\wedge\cdots\wedge u_k \big)&=v\wedge u_1\wedge\cdots\wedge u_k\\
\iota(v)\big( u_1\wedge\cdots\wedge u_k \big)&=\sum_{j=1}^k(-1)^{j-1}g(u,u_j)u_1\wedge\cdots\wedge\hat u_j\wedge\cdots\wedge u_j.
\end{align}
\end{subequations}
One has $\epsilon(v)^2=0$ and $\iota(v)^2=0$ because $v\wedge v=0$. In order to understand the latter, we wonder what are the terms with $g(v,u_i)g(v,u_j)$ are in
\[ 
  \iota(v)^2\big( u_1\wedge\cdots\wedge u_k \big)=\sum_{l=1}^k(-1)^{j-1}g(v,u_j)\sum_{l=1}^{k-1}(-1)^{l-1}g(v,u_l)u_1\wedge\hat u_l\wedge\hat u_j\wedge\cdots\wedge u_k.
\]
Let's suppose $i<j$. The first term comes when the first $\iota(v)$ acts on $u_j$, its sign is given by $(-1)^{j-1}(-1)^{i-1}$. The second term has the same $(-1)^{i-1}$, but in this term, $u_j$ is on the position $j-1$ because $u_i$ has disappeared.

Now we use $c(v)=\epsilon(v)+\iota(v)$ which fulfils for all $u$, $v\in V$:
\[ 
\begin{split}
c(v)^2&=g(v,v,)1\\
c(u)v(v)+c(v)c(u)&=2g(u,v)1.
\end{split}  
\]
Therefore $c$ can be extended to a representation $c\colon \Cliff(V,g)\to \End(\Wedge V)$. If $\{ e_0,\cdots e_n \}$ is an orthonormal basis of $V$ (i.e. $g(e_i,e_j)=\eta_{ij}$); in this case the $c(e_j)$ are anticommuting and a basis of $\Cliff(V,g)$ is given by 
\begin{equation}
 \{ c(e_{k_1})\cdots c(e_{k_r})\tq 1\leq k_1<\cdots<k_r\leq n \}.
\end{equation}

\subsection{Some consequences of the universal property}
%---------------------------------------------------

The map $-\id|_V$ extends to $\alpha\in\Aut\big( \Cliff(V) \big)$,
\[ 
  \alpha(v_1\cdots v_r)=(-1)^rv_1\cdots v_r
\]
($v_i\in V$) and provides a graduation
\[ 
  \Cliff(V)=\Cliff^0(V)\oplus\Cliff^1(V).
\]
The map $\tau\colon \Cliff(V)\to \Cliff(V)$ extends $\id|_V$ to an anti-homomorphism:
\begin{equation}
\tau(v_1\cdots v_r)=v_r\cdots v_1.
\end{equation}

The \defe{complexification}{complexification!of Clifford algebra} of $\Cliff(V,g)$ is 
\[ 
  \CCliff(V,g):=\Cliff(V,g)\otimes_{\eR}\eC\simeq\Cliff(V^{\eC},g^{\eC}),
\]
the isomorphism being a $\eC$-algebra isomorphism. The $\eR$-linear operator $v\mapsto \overline{ v }$ in $V^{\eC}$ of complex conjugation extends to a $\eR$-linear automorphism $a\mapsto \overline{ a }$. We define the \defe{adjoint}{adjoint!in Clifford algebra} by
\begin{equation}
  a^*=\tau(\overline{ a })
\end{equation}

\subsection{Trace}
%---------------------

\begin{theorem}
There exists one an only one trace $\tr\colon \CCliff(V)\to \eC$ such that
\begin{enumerate}
\item $\tr(1)=1$,
\item $\tr(a)=0$ when $a$ is odd.
\end{enumerate}

\end{theorem}

\begin{proof}

Let $\{ e_1,\cdots,e_n \}$ be an orthonormal basis of $(V,g)$ and $a\in\CCliff(V)$. When one decomposes $a$ into the basis of $e_i$, one finds a lot of terms of each order. Since $\tr$ is a trace, when the $k_i$ are all different,
\[ 
\tr(e_{k_1}\cdots e_{k_{2r}})	=\tr(-e_{k_2}\cdots e_{k_{2r}} e_{k_1}
				=\tr(-e_{k_1}\cdots e_{k_{2r}})\\
\]
So the trace of any even element is zero. We decompose $a$ into
\[ 
  a=\sum_K a_K\prod_{i\in K}e_i
\]
where the sum is taken on the subsets of $\{ 1,\ldots, n \}$. A trace which fulfils the conditions must vanishes on even (but non zero) elements as well as on odd elements, so the only possible form is
\[ 
  \tr a=a_{\emptyset}.
\]
Notice that in order to get this precise form, we used $\tr(1)=1$ and linearity. This proves unicity and existence. Now we have to prove that this is a good definition in the sense that an other choice of basis gives the same result. So we take a new orthonormal basis
\[ 
  e'_j=\sum_{k=1}^nH_{jk}e_k
\]
with $H^tH=\mtu_{n\times n}$. Now we have
\[ 
  a=\sum_{K}^{}a_K\prod_{i\in K}e_i=\sum_{K}^{}a'_K\prod_{i\in K}e'_i,
\]
and we will prove that $a_{\emptyset}=a'_{\emptyset}$. Let's compute a lot:
\[ 
\begin{split}
   e_i'e_j'&=\sum_{k}^{}\sum_{l}^{}H_{ik}H_{jl}e_ke_l\\
		&=\sum_{k=l}^{}H_{ik}H_{jl}e_{k}e_{l}+\sum_{k\neq l}^{}H_{ik}H_{jl}e_{k}e_{l}\\
		&=\sum_{k}^{}H_{ik}H_{jk}1+\sum_{k\neq l}^{}H_{ik}H_{jl}e_{k}e_{l}\\
		&=(HH^t)_{ij}1+\sum_{k\neq l}^{}H_{ik}H_{jl}e_{k}e_{l}.
\end{split}  
\]
The sense of this formula is that when $i\neq j$, the product $e'_{i}e'_{j}$ has no term of order zero. In other terms, as long as we only have terms of order zero, one and two, a change $e\to e'$ does not change the term of order zero. We are now going to an induction proof: we want to prove that $e'_{j_{1}}\ldots e'_{j_{2r}}e'_{l}e'_{k}$ has no scalar term assuming that no even combination has scalar terms up to $2(r-1)$. It reads
\[ 
    \sum_{K \text{ even}}^{}a_{K}\prod_{i\in K} e_{i}e'_{l}e'_{k},
\]
therefore we just have to look at terms of the form
\[ 
  e_{j_{1}}\ldots e_{j_{2r}}\Big( (HH)^t_{kl}1-\sum_{i\neq j}^{}C_{kl}^{ij}e_{i}e_{j} \Big)
\]
where the $e_{j_{l}}$ are all different. The first term cannot produce a scalar term. In order to find a scalar term in $e'_{j_{1}}\ldots e'_{j_{2r}}e_{k}e_{l}$, we begin to look at terms whose decomposition of $e'_{j_{1}}\ldots e'_{j_{2r}}$ ends by $e_{l}e_{k}$, i.e.
\[ 
  H_{j_{2r-2}l}H_{j_{2r-1}k}e'_{j_{1}}\ldots e'_{2r-3}e_{l}e_{k}e_{k}e_{l}.
\]
The induction assumption says that there are no scalar term in $e'_{2r-3}e_{l}e_{k}e_{k}e_{l}$.

\end{proof}
One can prove that $\CCliff(C)$ is a Hilbert space with the scalar product 
\begin{equation}
\langle a |b\rangle=\tr(a^*b).
\end{equation}


Let $v\in V$ with $g(v,v)=1$ (thus in $\Cliff(V)$, we have $v^2=1$); since $v=\overline{ v }$, we have
\[ 
  a^*v=vv^*=v^2=1.
\]

\begin{lemma}
The maps $a\mapsto ua$ and $a\mapsto au$ are unitary if and only if $uu^*=u^*u=1$.
\end{lemma}

\begin{proof}
We pick $\lambda\in U(1)$ and $w=\lambda v\in V^{\eC}$ which fulfils $w^*w=1$. This is the most general element such that $ww^*=w^*w=1$. Now for an arbitrary $a$, $b\in\CCliff(V)$, we compute the two followings:
\[ 
  \langle wa|wb\rangle=\tr\big( (wa)^*wb \big)
		=\tr\big( a^*w^*wb \big)
		=\tr(a^*b)
		=\langle a|b\rangle,
\]
and
\[ 
  \langle aw|bw\rangle=\tr\big( w^*a^*bw \big)
		=\tr(ww^*a^*b)
		=\tr(a^*b)
		=\langle a|b\rangle.
\]
This proves that $a\mapsto wa$ and $a\mapsto aw$ are two unitary operators on the Hilbert space $\CCliff(V)$.

For the converse, we impose for all $a$, $b\in\CCliff(V)$:
\[ 
  \langle ua|ub\rangle=\tr(ba^*u^*u)\stackrel{!}{=}\tr(ba^*).
\]
In particular with $a^*b=1$, $\tr(u^*u)=\tr(1)=1$, thus the scalar part of $u^*u$ is $1$. So we write $u^*u=1+f$ where $f$ is non scalar, and for any $x\in\CCliff(V)$ , we have
\[
   \tr(x)=\tr(xu^*u)=\tr(x)+\tr(xf).
\]
We conclude that $\tr(xf)=0$, and therefore that $f=0$.
\end{proof}


\section{Spinor representation}
%+++++++++++++++++++++++++++++

For the spinor representation, we restrict ourself to the even case $p+q=2n$.

The aim of this subsection is to find some faithful representations of the complex Clifford algebra $\Cliff^{\eC}(p,q)$. In order to achieve this, we first consider $V^{\eC}$, the complex vector space of $V$ with an orthonormal basis $\{ e_1,\cdots,e_{p-1},e_p,\cdots,e_q  \}$. The metric is $\eta(e_k,e_k)=1$ and $\eta(e_{p+k},e_{p+k})=-1$ for $k=0,\cdots,p-1$. We use the following basis:
\begin{align}
f_k&=\frac{1}{2}(e_k+e_{p+k}),& g_k&=\frac{1}{2}(e_k-e_{p+k}),\\
 f_{p+s}&=\frac{1}{2}(e_{2p+2s}+ie_{2p+2s+1}),& g_{p+s}&=\frac{1}{2}(e_{2p+2s}-ie_{2p+2s})
\end{align}
for $k=0,\cdots,p-1$.
We note that $\{f_0,g_0\}$ spans a $\eC^2$-space which is $\eta$-orthogonal to the one which is spanned by $\{f_1,g_1\}$. The following two  spaces will prove to be useful:
\begin{subequations}
\begin{align}
  W           &=\Span_{\eC}\{f_0,f_1\}\simeq\eC^2,\\
 \underline{W}&=\Span_{\eC}\{g_0,g_1\}\simeq\eC^2.
\end{align}
\end{subequations}
\nomenclature{$W,\underline{W}$}{Totally isotropic subspace}
It is easy to compute the various products; among others we find
\begin{equation}
 \eta(f_0,f_0)=0,\quad
 \eta(f_1,f_0)=0,\quad
  \eta(f_1,f_1)=0;
\end{equation}
so that for any $w\in W$, we have $\scal{w}{w}=0$; for this reason, we say that $W$ is a \defe{completely isotropic}{isotropic!subspace!completely} subspace of $(V^{\eC},\eta^{\eC})$. The space $\underline{W}$ has the same property.

\begin{proposition}
We have
\begin{equation}
  \underline{W}\simeq W^*,
\end{equation}
where $W^*$ is the dual space of $W$. By $\simeq$ we mean that there exists a linear bijective map $\dpt{\psi}{\underline{W}}{W^*}$.
\end{proposition}
\begin{proof}
For each $\uw\in\uW$, we define $\dpt{\psi(\uw)}{W}{\eC}$ by
\[
   \psi(\uw)(w)=\eta(w,\uw).
\]
We first show that the map $\psi$ is injective. Let $\uw\in\uW$ be so that $\psi(\uw)=0$. Thus for all $v\in W$, we have
\begin{eqnarray}
   \label{3101r1}\psi(\uw)v=\eta(\uw,v)=0.
\end{eqnarray}
By decomposing $\uw=ag_0+bg_1$ and taking successively $v=f_0$ and $v=f_1$, we see that $a=b=0$.

The next step is to see that the map $\psi$ is surjective. We know that $dim_{\eC}\uW=dim_{\eC}W^*=2$ and that $\psi(g_0)\neq 0$. Let's prove that $\{\psi(g_0),\psi(g_1)\}$ is a basis of $W^*$. It is clear by linearity that $\{\psi(ag_0):a\in\eC\}=\Span\{\psi(g_0)\}$. The fact that $\psi$ is injective  imposes that $\psi(g_1)$ doesn't belong to $\Span\{\psi(g_0)\}$. So $\{\psi(g_0),\psi(g_1)\}$ is a two-dimensional free subset of $W^*$, and therefore is a basis of $W^*$.
\end{proof}

We turn our attention to the exterior algebra $\Lambda W=\eC\oplus W\oplus(W\wedge W)\oplus\cdots\oplus\wedge^{p+q}W$\nomenclature{$\Lambda W$}{Space of spinor representation} of $W$. 
\nomenclature{$\dpt{\tilde\rho}{(\eR^{1+3})^{\eC}}{\End(\Lambda W )}$}{Spinor representation}

\begin{definition}
\index{endomorphism!of $\Lambda W$}
We define the homomorphism $\dpt{\tilde\rho}{V^{\eC}}{\End(\Lambda W)}$ by 
\begin{equation}
\begin{split}
 \tilde\rho(f_i)\alpha&=f_i\wedge\alpha,\\
 \tilde\rho(g_i)\alpha&=-\iota(g_i)\alpha
\end{split}
\end{equation}
($v\in V^{\eC}$, $\alpha\in\Lambda W$) where $\iota$ denotes the interior product defined in page \pageref{pg_DefProdExt}.\
\label{defrt}
\end{definition}
  More explicitly, for all $z\in\eC$ and for all $w,w'\in W$, we have
\begin{subequations}
\begin{align}
 \tilde\rho(f_i)z&=zf_i,&\tilde\rho(g_i)z&=0,\\
 \tilde\rho(f_i)w&=f_i\wedge w,&\tilde\rho(g_i)w&=-\eta(g_i,w)1,\\
 \tilde\rho(f_i)(w\wedge w')&=0,&\tilde\rho(g_i)(w\wedge w')&=-\eta(g_i,w)w'+\eta(g_i,w')w.
\end{align}
\end{subequations}
We will see that, \emph{via} some manipulations, $\tilde\rho$ provides a faithful representation of the Clifford algebra, the \defe{spinor representation}{spinor!representation}.
\index{representation!of Clifford algebra}

\begin{remark}
By ``endomorphism of $\Lambda W$'', we mean an endomorphism for the \emph{linear} structure of $\Lambda W$. We obviously not have $\tilde\rho(x)(\alpha\wedge\beta)=\tilde\rho(x)\alpha\wedge\tilde\rho(x)\beta$. 
\end{remark}

\begin{proposition}
The map $\tilde\rho$ is injective.
\end{proposition}

\begin{proof}
We have to show that $\tilde\rho(v)=0$ ($v$ in $V^{\eC}$) implies $v=0$. Any $v\in V^{\eC}$ can be written as
$v=a^if_i+b^ig_i$ with a sum over $i$. We first have that
\[
 \tilde\rho(a^if_i+b^ig_i)z=za^if_i=0,
\]
 but the $f_i$ are independents and then $a^i=0$. We can also write
\[
 \tilde\rho(b^0g_0+b^1g_1)f_1=-b^0\eta(g_0,f_1)-b^1\eta(g_1,f_1)=-\frac{b^1}{2}=0,
\]
 then $b^1=0$. The same with $f_0$ proves that $b^0=0$.
 \end{proof}

The homomorphism $\tilde\rho$ extends to the whole the tensor algebra of $V^{\eC}$ by the following definitions:
\begin{subequations}
\begin{align}
 \tilde\rho(1)              &=\id_{\Lambda W},\\
 \tilde\rho(e_k)            &=\tilde\rho(e_k),\\
 \tilde\rho(e_{k_1}\otimes\ldots\otimes e_{k_r})&=
                      \tilde\rho(e_{k_1})\circ\ldots\circ\tilde\rho(e_{k_r}).\label{eq:3101r2}
\end{align}
\end{subequations}
So we get $\dpt{\tilde\rho}{T(V^{\eC})}{\End(\Lambda W)}$.  The following proposition will allow us to descent $\tilde\rho$ to a representation of the Clifford algebra.

\begin{proposition}
The homomorphism $\tilde\rho$ maps $\mI$ to $0$: $\tilde\rho(\mI)=0$.
\end{proposition}

\begin{probleme}
This proposition is wrong: there is a double covering.

Moreover, there is a sign problem in the proof: the sign in the first lines is not the one used in the definition of the Clifford algebra.
\end{probleme}


\begin{proof}
We have to check the following:
\[\tilde\rho(v\otimes w\oplus w\otimes v-2\eta(v,w)1)=0\]
for any choice of
 $v,w$ in $\{e_0,e_1,e_2,e_3\}$.
  Here we will just check it explicitly for $v=e_0$ and $w=e_1$. The computation uses the definition \eqref{eq:3101r2}:
\begin{equation}
\begin{split}
\tilde\rho(e_0\otimes e_1\oplus e_1\otimes
             e_0-2\eta(e_0,e_1)&=\tilde\rho(e_0)\circ\tilde\rho(e_1)+\tilde\rho(e_1)\circ\tilde\rho(e_0)\\
                               &=2\left[\tilde\rho(f_0)^2-\tilde\rho(g_0)^2\right].
\end{split}
\end{equation}
It is easy to see that $\tilde\rho(f_0)^2=0$:
\begin{equation}
 \tilde\rho(f_0)^2\left[z\oplus w\oplus w_1\wedge w_2 \right]=\tilde\rho(f_0)[zf_0\oplus f_0\wedge w]
                                                   =zf_0\wedge f_0,
							=0.
\end{equation}
 The proof that $\tilde\rho(g_0)^2=0$ is almost the same:
\[ 
 \tilde\rho(g_0)^2\left[z\oplus w\oplus w_1\wedge w_2 \right]
 =\tilde\rho(g_0)[-\eta(g_0,w)1\oplus-\eta(g_0,w_1)w_2\oplus\eta(g_0,w_2)w_1].
\]

\end{proof}

We can now see $\tilde\rho$ as a map $\dpt{\tilde\rho}{\Cliff^{\eC}(p,q)}{\End(\Lambda W )}$. By construction, it is a homomorphism and, thus, is a representation of $\Cliff^{\eC}(p,q)$ on $\Lambda W$. For compactness, we use the notation \index{dirac!matrices}\nomenclature{$\gamma_i$}{Abstract definition of Dirac matrices}
\begin{equation}
 \label{defgamma}\gamma_a:=\sqrt{2}\tilde\rho(e_a). 
\end{equation}

\begin{lemma}
The $\gamma$'s operators satisfy the following relation:
\begin{equation}\label{3101r3}
  \gamma_a\gamma_b+\gamma_b\gamma_a=-2\eta_{ab}\mtu.
\end{equation}
\label{3101l1}
\end{lemma}

\begin{proof}
We have to check this equality on any element of $\Lambda W$. If we choose
$w_1=af_0+bf_1$ and $w_2=a'f_0+b'f_1$,
 we find $w_1\wedge w_2=(ab'-ba')f_0\wedge f_1$.

For example, we will explicitly check \eqref{3101r3} with $a=b=0$, i.e. $\tilde\rho(e_0)\circ\tilde\rho(e_0)=\frac{1}{2}\id$, which proves that $\gamma_0\circ\gamma_0=\id$.
 \begin{equation}
\begin{split}
   \tilde\rho(e_0)^2[z\oplus w\oplus(ab'-ba') f_0\wedge f_1]&=\tilde\rho(f_0+g_0)^2[z\oplus w\oplus(ab'-ba') f_0\wedge f_1]\\ 
                                              &=\tilde\rho(f_0+g_0)\Big[zf_0\oplus f_0\wedge w\oplus-\eta(g_0,w)1\\
                                                            &\qquad-(ab'-ba')\eta(g_0,f_0)f_1\\
                                                             &\qquad+(ab'-ba')\eta(g_0,f_1)f_0\Big]\\
                                              &=\frac{1}{2}(z\oplus w\oplus(ab'-ba') f_0\wedge f_1).
\end{split}				      
\end{equation}
\end{proof}

\begin{lemma}
For any sequence $i_0,\ldots i_3$ of $0$ and $1$ (with at least one of them equals to $1$), we have
\begin{equation}
 \tr(\gamma_0^{i_0}\cdots \gamma^{i_{2n-1}}_{2n-1})=0.
\end{equation}
We take the convention that $\gamma_a^0=\mtu$.
\label{3101l2}
\end{lemma}

\begin{proof}
 If the number of nonzero $i_k$ is even (say $2m$), we have:
\[
	\tr(\gamma_{a_1}\ldots\gamma_{a_{2m}})=\tr(\gamma_{a_{2n}}\gamma_{a_1}\ldots\gamma_{a_{2m-1}})
\] 
because the trace is invariant under cyclic permutations. But we can also permute $\gamma_{a_{2m}}$ with the $2m-1$ other $\gamma$'s.  $\tr(\gamma_{a_1}\ldots\gamma_{a_{2m}})=(-1)^{2n-1}\tr(\gamma_{a_{2m}}\gamma_{a_1}\ldots\gamma_{a_{2m-1}})$ because each permutation gives an extra minus sign (\hbox{lemma \ref{3101l1}}). Then the trace is zero.

If the number of nonzero $i_k$ is odd (say $2m-1$). Let $i_a=0$ (we restrict ourself to the even dimensional case). We have $\tr(A)=-\eta_{aa}\tr(A\gamma_a\gamma_a)$. Using once again the cyclic invariance of the trace, $\tr(\gamma_{a_1}\ldots\gamma_{a_{2m-1}}\gamma_a\gamma_a)=\tr(\gamma_a\gamma_{a_1}\ldots\gamma_{a_{2m-1}}\gamma_a)$. But, if we permute the \emph{first} $\gamma_a$ with the $2m-1$ first $\gamma$'s, we find \hbox{$\tr(\gamma_{a_1}\ldots\gamma_{a_{2m-1}}\gamma_a\gamma_a)=-\tr(\gamma_a\gamma_{a_1}\ldots\gamma_{a_{2m-1}}\gamma_a)$ }, and the trace is zero again.
\end{proof}

\begin{proposition}
The subset
\[
	\left\{\mtu,\dga{a}{b}\,(a<b),\tga{a}{b}{c}\,(a<b<c), \cdots, \gamma_{0}\cdots\gamma_{2n} \right\}
\]
 is free in $\End(\Lambda W)$.
\end{proposition}

\begin{proof}
We consider a general linear combination of these operators:
\[
 E=\lambda\mtu+\sum_a\lambda_a\gamma_a+\sum_{a<b}\lambda_{ab}\dga{a}{b}+\cdots+
 \sum_{a<b<c<d}\lambda_{abcd}\qga{a}{b}{c}{d}.
\]
The claim is that if $E=0$, then all the coefficients $\lambda_{(\ldots)}$ must be zero. First note that $Tr(E)=0=\lambda$ by lemma \ref{3101l2}. It is also clear that $Tr(\gamma_iE)=0=\lambda_i$. In order to see that $\lambda_{ij}=0$, we compute $Tr(\gamma_j\gamma_iE)=0=\lambda_{ij}$. And so on.
\end{proof}

How many operators does we have in this free system? Any operators in this system can be written as $\gamma_{0}^{i_{0}},\cdots\gamma^{i_{2n-1}}_{2n-1}$ with $i_k$ equal to zero or one. Thus we have $2^{2n}$ operators. On the other hand, we know that $dim_{\eC}\Lambda W=2p+2$, and then that $dim_{\eC}\End(\Lambda W)=4^2=16$. The result is that $\{ \gamma_{0}^{i_{0}},\cdots\gamma^{i_{2n-1}}_{2n-1}  \tq i_k=0\,or\,1\}$ is a basis of $\End(\Lambda W)$. In other words (if we suppose a suitable ordering), the image by $\tilde\rho$ of 
\[ 
B=\{1,e_a,e_a\otimes e_b,e_a\otimes e_b\otimes e_c,e_a\otimes e_b\otimes e_c\otimes e_d\}
\]
 is a basis of $\End(\Lambda W)$.

If $B$ is a basis of $C^{\eC}_{(p,q)}$, then $\tilde\rho$ is bijective and thus isomorphic.  Therefore, we expect $\dpt{\tilde\rho}{C^{\eC}_{(p,q)}}{\End(\Lambda W)}$ to be a faithful representation\index{representation!of Clifford algebra}. It is not difficult to see that $B$ is indeed a basis thanks to the equivalence relation.

\subsection{Explicit representation}
%------------------------------

First, we choose a basis for $\Lambda W$:
\begin{eqnarray}\label{102r2} 1=\left(\begin{matrix}
1 \\
0 \\
0 \\
0
\end{matrix}\right),\quad
f_0=\left(\begin{matrix}
0 \\
1 \\
0 \\
0
\end{matrix}\right),\quad
f_1=\left(\begin{matrix}
0 \\
0 \\
1 \\
0
\end{matrix}\right),\quad
f_0\wedge f_1=\left(\begin{matrix}
0 \\
0 \\
0 \\
1
\end{matrix}\right).
\end{eqnarray}
 Here is the explicit computation for the matrix $\gamma_0$ in this basis. First remark that $\tilde\rho(e_0)1=f_0$, $\tilde\rho(e_0)f_0=\frac{1}{2}$, $\tilde\rho(e_0)f_1=f_0\wedge f_1$, $\tilde\rho(e_0)(f_0\wedge f_1)=\frac{1}{2} f_1$. Then
\begin{equation}
\begin{split}
 \gamma_0\left(\begin{matrix}1 \\0 \\0 \\0\end{matrix}\right)=\sqrt{2}\left(\begin{matrix}0 \\1 \\0  \\0\end{matrix}\right),\quad
 \gamma_0\left(\begin{matrix}0 \\1 \\0 \\0\end{matrix}\right)=\sqrt{2}\left(\begin{matrix}\frac{1}{2} \\0 \\0 \\0\end{matrix}\right),\\
 \gamma_0\left(\begin{matrix}0 \\0 \\1 \\0\end{matrix}\right)=\sqrt{2}\left(\begin{matrix}0 \\0 \\0 \\1\end{matrix}\right),\quad
 \gamma_0\left(\begin{matrix}1 \\0 \\0 \\0\end{matrix}\right)=\sqrt{2}\left(\begin{matrix}0 \\0 \\\frac{1}{2} \\0\end{matrix}\right).
\end{split}
\end{equation}
This allows us to write down $\gamma_0$; the same computation gives the other matrices.\index{dirac!matrices}\nomenclature{$\gamma_i$}{Explicit form of gamma matrices}
\begin{equation}
\begin{split}
\gamma_0=\sqrt{2}\begin{pmatrix}
0 & \frac{1}{2} & 0 & 0 \\
1 & 0 & 0 & 0 \\
0 & 0 & 0 & \frac{1}{2} \\
0 & 0 & 1 & 0
\end{pmatrix}, \qquad
\gamma_1=\sqrt{2}\left(\begin{matrix}
0 & -\frac{1}{2} & 0 & 0 \\
1 & 0 & 0 & 0 \\
0 & 0 & 0 & -\frac{1}{2} \\
0 & 0 & 1 & 0
\end{matrix}\right),\\
\gamma_2=\sqrt{2}\begin{pmatrix}
0 & 0 & -\frac{1}{2} & 0 \\
0 & 0 & 0 & \frac{1}{2} \\
1 & 0 & 0 & 0 \\
0 & -1 & 0 & 0
\end{pmatrix},\qquad
\gamma_3=\sqrt{2}\begin{pmatrix}
0 & 0 & -\frac{i}{2} & 0 \\
0 & 0 & 0 & \frac{i}{2} \\
-i & 0 & 0 & 0 \\
0 & i & 0 & 0
\end{pmatrix}.
\end{split}
\end{equation}
It is easy to check that these matrices satisfies \eqref{3101r3}. 

Notice that, up to a suitable change of basis in $\Lambda W $, these are the usual Dirac matrices\index{dirac!matrices}. Indeed we actually solved the physical problem to find a representation of the algebra \eqref{cliffphys}.  We understand by the way why do physicists work with $4$-components spinors: the $\gamma$'s are operators on the four-dimensional space $\Lambda W$; hence the Dirac operator will naturally acts on four-components objects.

The main result of this section is an explicit faithful representation of $\CCliff(p,q)$. This allows us to write a \defe{Dirac operator}{dirac!operator!on $\protect\eR^{1,3}$} which solve (see the invitation \ref{Secqft} and \cite{Bronn}) the problem  to find a ``square root'' of the d'Alembert operator: the differential operator $\Dir=\gamma\hmu\partial_{\mu}$ satisfies $\Dir^2=\Box$.

\subsection{A remark}  % C'est ce qui arrive quand je ne sais pas quel titre donner.
%---------------------

 Let us compare the two faithful representations
\[ 
\begin{split}
  c\colon \Cliff(V)&\to \End_{\eR}(\wedge V)\\
\tilde\rho\colon \CCliff&\to  \End_{\eR}(\wedge W).
\end{split}  
\]
They obviously comes from the same ideas. One common point is that 
\[ 
  c(e_1)(e_1\wedge e_2)=2\tilde\rho (e_1)(e_1\wedge e_2)=e_2,
\]
but they are different:
\[ 
\begin{split}
  \tilde\rho(e_3)(e_0\wedge e_2)&=0\\
c(e_3)(e_0\wedge e_2)&=e_3\wedge e_0\wedge e_1.
\end{split}  
\]

\subsection{General two dimensional Clifford algebra}
%---------------------------------------------------

The Clifford algebra for the metric
\[ 
  g=\begin{pmatrix}
\alpha&\delta\\\delta&\beta
\end{pmatrix}
\]
is realised by matrices
\[ 
  \gamma_1=
\epsilon\begin{pmatrix}
\sqrt{\alpha}\\ & -\sqrt{\alpha}
\end{pmatrix},\quad
\gamma_2=
\epsilon\begin{pmatrix}
\delta/\sqrt{\alpha}& \beta-\delta^2/| \alpha |\\
1		& -\delta/\sqrt{\alpha}	
\end{pmatrix}
\]
where $\epsilon=\pm 1$ is chosen in such a way that $\epsilon| \alpha |=\alpha$. 
 
\section{Spin group}
%+++++++++++++++++++

We will not immediately go on with Dirac operators on Riemannian manifolds because we still have to build some theory about the Clifford algebra itself. In particular, we have to define the spin group which will play a central role in the definition of the Dirac operator. Almost all --and (too?) much more-- the concepts we will introduce in this section can be found in \cite{Chevalley}; a more physical oriented but useful approach can be found in \cite{Preparation}.

Let define the map $\dpt{\chi}{\Gamma(p,q)}{GL(\eR^{1,3})}$ by 
\begin{equation}
                \chi(x)y=\alpha(x)\cdot y\cdot x^{-1}.
\end{equation}
\nomenclature{$\chi$}{A representation of $\Gamma(p,q)$}\nomenclature[G]{$\Gamma(p,q)$}{Clifford group}
Let
\[
 \Gamma(p,q)=\{x\in\Cliff(p,q)\tq\textrm{$x$ is invertible and }  \chi(x)y  \in V\textrm{ for all $y\in V$}\}.
\]
It should be remarked that this definition comes back to the real Clifford algebra. The Clifford algebra product gives this subset a group structure which is called the \defe{Clifford group}{Clifford!group}. Any $x\in V$ is invertible since $x\cdot x=-\eta(x,x)1$, the inverse of $x$ is given by $x^{-1}=x/\|x\|^2$.

\index{Clifford!algebra!grading of}
The subset $\Cliff(p,q)^+$ (resp. $\Cliff(p,q)^-$) of $\Cliff(p,q)$ is the image of even (resp. odd) tensors of $T(V)$ by the canonical projection $T(V)\to\Cliff(p,q)$. With these definitions, we have a natural grading of $\Cliff$:
\begin{equation}
 \label{directC}\Cliff(p,q)=\Cliff(p,q)^+\oplus\Cliff(p,q)^-,
 \end{equation}
and the subgroups
\begin{align}
\label{defgplus}
\Gamma(p,q)^+&=\Gamma(p,q)\cap\Cliff(p,q)^+,&\Gamma(p,q)^-&=\Gamma(p,q)\cap\Cliff(p,q)^-.
\end{align}\nomenclature[G]{$\Cliff(p,q)^{\pm}$}{Grading of Clifford algebra}

For $x_1,\ldots,x_n\in V$, we have $\tau(x_1\cdots x_n)=x_n\cdots x_1$. \nomenclature[G]{$\Spin(p,q)$}{Spin group of $\eR^{1,3}$} The \defe{spin group}{spin!group!on $\protect\eR^{1,3}$} is
\begin{equation}   \label{defSpinun}
 \Spin(p,q)=\{x\in\Gamma(p,q)^+\vert\tau(x)=x^{-1}\}
\end{equation}
while the \defe{spin norm}{spin!norm}\nomenclature{$\dpt{N}{\Gamma(p,q)}{\Gamma(p,q)}$}{Spin norm} is the map $\dpt{N}{\Gamma(p,q)}{\Gamma(p,q)}$ defined by
\[
 N(x)=x\tau(\alpha(x)).
\]

\begin{proposition} \label{proppourN}
The map $N$ takes values in $\eR$ and the formula
\begin{equation}
             N(x\cdot y)=N(x)N(y),
\end{equation}
holds for all $x$, $y\in\Gamma(p,q)$.
\end{proposition}

\begin{proof}
We write as usual $x\in\Gamma(p,q)$ as $x=cv_1\cdots v_r$. So,
\begin{equation}
 N(x)=cv_1\cdots v_r\tau(\alpha(cv_1\cdots v_r))
     =\me{r}c^2v_1\cdots v_r\cdot v_r\cdots v_1.
\end{equation}
The first equality comes from the fact that $\alpha(cv_1\cdots v_r)=\me{r}cv_1\cdots v_r$. Now $N(x)\in\eR$ because $v_i\cdot v_i=-\brak{v_i}{v_i}\in\eR$ for all $i$. Hence the following hold:
\begin{equation}
\begin{split}
 N(x\cdot y)&=v\cdot y\cdot\tau(\alpha(v\cdot y))\\
            &=v\cdot y\cdot\tau(\alpha(y))\cdot\tau(\alpha(v))\\
            &=v\cdot N(y)\tau(\alpha(v))\\
            &=N(y)N(x).
\end{split}
\end{equation}
This is the claim.
\end{proof}

Therefore  $N\colon \Gamma(p,q)\to \eR$ is an homomorphism.

\begin{remark}
The elements of $\Spin(p,q)$ are spin-normed at $1$. Indeed, take a $s$ in $\Spin(p,q)$. We have $N(s)=s\cdot \tau(s)=1$ because $\alpha(s)=s$ and $\tau(s)=s^{-1}$. In particular $\Spin(p,q)\cap\eR=\eZ_{2}$.
\label{rem:spin_norm_u}
\end{remark}

\subsection{Studying the group structure}
%--------------------------------

\begin{proposition}
The set $\Gamma(p,q)$ admits a Lie group structure.
\end{proposition}
\begin{proof}

During this proof, $\mu$ denotes the Clifford multiplication: $\mu(x,y)=x\cdot y$. We know that $\Cliff^{\eC}(p,q)$ is isomorphic to $\End(\Lambda W)$ in which the multiplication is a continuous map. Thus $\mu$ is continuous on $\Cliff^{\eC}(p,q)$. But $\Cliff(p,q)$ is a closed subset of $\Cliff^{\eC}(p,q)$, so $\mu$ is a continuous map in $\Cliff(p,q)$. This proves that  $\chi$ seen as a map from $\Gamma(p,q)\times V$ to $V$ is a continuous map.

The space $V$ is closed in $\Cliff(p,q)$, thus $\sigma^{-1}(V)$ is also closed. But $\sigma^{-1}(V)=\Gamma(p,q)\times\Cliff(p,q)$. So $\Gamma(p,q)$ is closed in $\Cliff(p,q)$.

Now the result is just a consequence of theorems \ref{Helgason2.3} and \ref{Helgason4.2}. Indeed, let us study the subset $\mI$ which appears in the definitions of the Clifford algebra. It makes no difficult to convince ourself that it is a closed subgroup of $T(V)$. The theorem \ref{Helgason4.2} thus makes $\Cliff(p,q)=T(V)/\mI$ a Lie group. But we just say that $\Gamma(p,q)$ is closed in $\Cliff(p,q)$, and the fact that $\Gamma(p,q)$ is a subgroup of $\Cliff(p,q)$ is clear. By theorem \ref{Helgason2.3} we conclude that there exists a Lie group structure on $\Gamma(p,q)$.
\end{proof}

\begin{lemma}
The map $\chi$ is a homomorphism, in other words $\chi$ is a representation of $\Gamma(p,q)$.
\index{representation!of $\Gamma(p,q)$}
\end{lemma}

\begin{proof}
The following computation uses the fact that $\alpha$ is a homomorphism:
\[
\begin{split}
\chi(a\cdot b)y&=\alpha(a\cdot b)\cdot y\cdot (a\cdot b)^{-1}
               =\alpha(a)\cdot\alpha(b)y\cdot b^{-1}\cdot a^{-1}\\
               &=\alpha(a)\cdot\chi(b)y\cdot a^{-1}
               =\chi(a)\chi(b)y.
\end{split}
\]
\end{proof}
Let $y\in\Gamma(p,q)^-$ and $v\in V$. Where is $y\cdot v$? First note that $(y\cdot v)^{-1}=v^{-1}\cdot y^{-1}$, so that
\begin{equation}
\begin{split}
  \alpha(y\cdot v)\cdot w\cdot(y\cdot v)^{-1}&=-\alpha(y)\cdot v\cdot w\cdot v^{-1}\cdot y^{-1}\\
                                            &=-\alpha(y)\big( 2\eta(v,w)-w\cdot v \big)\cdot v^{-1}\cdot y^{-1}\\
					    &=-2\eta(v,w)\alpha(y)\cdot v^{-1}\cdot y+\alpha(y)\cdot w\cdot y^{-1}
\end{split}
\end{equation}
which belongs to $V$ because $y\in\Gamma(p,q)$. This reasoning shows that (apart for $0$), $y\cdot v\in\Gamma(p,q)^+$ if and only if $y\in\Gamma(p,q)^-$.

\begin{lemma}
If $x\in V$ is non-isotropic (i.e. $\eta(x,x)\neq 0$), the automorphism $\chi(x)$  is the orthogonal symmetry with respect to $x^{\perp}$.
\end{lemma}

We recall that\nomenclature{$x^{\perp}$}{Space orthogonal to $x$}
\[ 
  x^{\perp}=\{ y\in V\tq\eta(x,y)=0  \}.
\]
We will denote by $\sigma^x$ the orthogonal symmetry with respect to $x^{\perp}$.

\begin{proof}
When the operator $\sigma^x$ acts on $y$, it just change the sign of the ``$x$-part''\ of $y$. So we can write $\sigma^x y=y-2\eta(x,y) 1_x$, where $1_x:=x/\|x\|$. It should be checked if
$\chi(x)y=\alpha(x)\cdot y\cdot x^{-1}$ is equal to $y-2\eta(x,y) 1_x$ or not. We know that $x\cdot x=\eta(x,x)1=-\|x\|$. It follows that
\[
  x\cdot y+y\cdot x=2\eta(x,y)\frac{x\cdot x}{\|x\|}.
  \]
If we multiply this at right by $x^{-1}$, using the fact that $\alpha(x)=-x$, we find
\[
-\alpha(x)\cdot y\cdot x^{-1}=-y+2\eta(x,y) 1_x,
\]
which is precisely the identity we wanted to check.
\end{proof}

The following result will help us to identify subgroups of Clifford group as isometry groups.
\begin{theorem}[Cartan-Dieudonné theorem]
\index{Cartan-Dieudonné theorem}\index{theorem!Cartan-Dieudonné}
Each $\sigma$ in $O(1,3)$ can be written as
\hbox{$\sigma=\tau_1\circ\ldots\circ\tau_m$}, where the $\tau$'s are orthogonal symmetries with respect to hyperplanes which are orthogonal to non-isotropic vectors.
\label{CartanDieu}
\end{theorem}

\begin{proposition}
\[
              \chi(\Gamma(p,q))=O(p,q).
\]
\label{prop1001t1}
\end{proposition}

\begin{proof}
In order to show that $\chi(\Gamma(p,q))\subset O(p,q)$ take $z\in V$ and $x\in\Gamma(p,q)$. Since $\alpha(x)\cdot z\cdot x^{-1}$ lies in $V$, we can write:
\[
\alpha(x)\cdot z\cdot x^{-1}=-\alpha\left(\alpha(x)\cdot z\cdot x^{-1}\right)
=-x\cdot\alpha(z)\cdot\alpha(x^{-1})=x\cdot z\cdot\alpha(x^{-1}).
\]
In order to see that $\chi(x)\in O(p,q)$, we have to prove that $\left\|\chi(x)y\right\|_{(p,q)}^2=\|y\|_{(p,q)}^2$. This is achieved by the following computation:
\begin{equation}
\begin{split}
 \left\|\chi(x)y\right\|_{(p,q)}^2&=-\left(\alpha(x)\cdot y\cdot x^{-1}\right)^2
                                  =\left(\alpha(x)\cdot y\cdot x^{-1}\right)\left(x\cdot y\cdot\alpha(x^{-1})\right)\\
                                  &=-\alpha(x)\cdot y^2\cdot\alpha(x^{-1})
                                  =\|y\|^2_{(p,q)}.
\end{split}
\end{equation}
The last step is simply the fact that $y^2\in\eR$ and therefore commutes with anything. We now know that $\chi(x)\in O(p,q)$ for all $x\in\Gamma(p,q)$. Thus $\chi(\Gamma(p,q))\subset O(p,q)$.

For the second part, let $\sigma$ be in $O(p,q)$. The Cartan-Dieudonné theorem\index{Cartan-Dieudonné theorem}\index{theorem!Cartan-Dieudonné}(theorem \ref{CartanDieu}) says that $\sigma=\sigma^{x_1}\circ\ldots\circ\sigma^{x_r}$ for some $x_1,\ldots, x_r$ in $V$. Thus $\sigma=\chi(x_1\cdots x_r)$, \hbox{and $O(p,q)\subset\chi(\Gamma(p,q))$}.
\end{proof}

\begin{proposition}
\begin{equation}
      \ker\chi=\eR\invtible
\end{equation}
where the right hand side is the set of invertible elements of $\eR$.
\label{prop1001p1}
\end{proposition}
\nomenclature{$\cA\invtible$}{The set of invertible elements of the algebra $\cA$; for example $\eR\invtible=\eR\setminus\{ 0 \}$}

\begin{proof}
Before beginning the proof, we want to insist on the fact that $x\in \ker\chi$ does not mean that $\chi(x)y=0$ for all $y$ in $V$. The ``zero''\ of an algebra is the element $e$ which satisfies $e\cdot y=y\cdot e=y$ for all $y$ in the algebra. In other words, $x$ is in the kernel of $\chi$ if and only if $\chi(x)=\id$.

First we show that $\eR_0\subset \ker\chi$. If $x\in \eR$, then $\alpha(x)=x$. Therefore, when $x\neq 0$,
\[
\chi(x)y=\alpha(x)\cdot y\cdot x^{-1}=y,
\]
because the algebra product $\cdot$ between an element of $\Cliff(p,q)$ and a real is commutative. Note that this does not work with $x=0$.

We are now going to show that $\ker\chi\subset\eR$. Let $z\in\ker\chi$. We decompose (definitions \eqref{defgplus}) it into his odd and even part: $z=z^++z^-$, with $z^{\pm}\in\Gamma(p,q)^{\pm}$. These two can be written as $z^+=e_{j_1}\cdots e_{j_{2r}}$ and $z^-=e_{i_1}\cdots e_{i_{2r-1}}$ with no two $i_k$ or $j_k$ equals. This is almost the general form of elements in even and odd part of $\Gamma(p,q)$: the only other possibility is $z$ in $\eR$. Obviously $\alpha(z^{\pm})=\pm z^{\pm}$. Multiplying the condition $\chi(z)y=y$ at right by $(z^++z^-)$, we find \[(z^+-z^-)y=y(z^++z^-).\] Thanks to equation \eqref{directC}, we can split this condition into even and odd parts:
\begin{align}
 z^+y&=yz^+,
 &z^-y&=-yz^-.
\end{align}
The first equation with $y=e_{j_1}$ gives $e_{j_1}\cdots e_{j_{2r}}\cdot e_{j_1}=e_{j_1}e_{j_1}\cdots e_{j_{2r}}$. In the left hand side, permute the last $e_{j_1}$ from last to second position. So we find the right hand side, with an extra minus sign. This means that $z^+=0$. In the same way, the second equation gives $z^-=0$. We are left with the last possibility: $z\in\eR$.
\end{proof}

\begin{corollary}
For any $s\in\Gamma(p,q)$, there exists some non-isotropic vectors $x_1,\ldots,x_r$, and $c\in\eR$ such that $s=cx_1\cdots x_r$.
\label{602c1}
\end{corollary}

\begin{proof}
Let us take a $s\in\Gamma(p,q)$; we just saw (theorem \ref{prop1001t1}) that $\chi(s)$ is an element of $O(p,q)$. It can be written $\chi(s)=\sigma_1\circ\ldots\circ\sigma_m$. But we had shown that $\sigma_i=\chi(x_i)$ for any $x_i$ normal to the hyperplane defining $\sigma_i$. We thus have
\[
     \chi(s)=\chi(x_1\cdots x_m),
\] 
where $s$ belongs to $\Gamma(p,q)$ and is therefore invertible. This leads us to write $\id=\chi(s^{-1}\cdot x_1\cdots x_m)$. But the kernel of $\chi$ is $\eR$ (proposition \ref{prop1001p1}); so one can find a $r\in\eR$ such that $s^{-1}\cdot x_1\cdots x_m=r$. The claim follows.
\end{proof}

\begin{lemma}
If $v\in V$,
\begin{equation}
                 \det\chi(v)=-1.
\end{equation}
\end{lemma}

\begin{proof}
We already know that $det\chi(v)=\pm 1$. To check that the right sign is plus, take the following basis of $V$: $\{v,v_i^{\perp}\}$ where $\{v_i^{\perp}\}$ is a basis of $v^{\perp}$. Calculating the action of $\chi(v)$ on this basis, we find:
\begin{equation}
\begin{split}
 \chi(v)v&=-v\cdot v\cdot v^{-1}=-v,\\
 \chi(v)v_i^{\perp}&=-v\cdot v_i^{\perp}\cdot v^{-1}
                   =v_i^{\perp}\cdot v\cdot v^{-1}
                   =v_i^{\perp}.
\end{split}
\end{equation}
In this computation\nomenclature{$\sQ$}{A subgroup of $\sG$}, we used the relation $v\cdot w=-w\cdot v-2\brak{v}{w}$ which is true for all $v$, $w$ in $V$. The action of $\chi(v)$ on this basis is thus to let unchanged three vectors and to change the sign of the fourth. This proves the claim.
\end{proof}

\begin{theorem}
\begin{equation}
                   \chi(\Gamma(p,q)^+)=\SO(p,q).
\end{equation}
\label{2102p1}
\end{theorem}

\begin{proof}
From corollary \ref{602c1}, and definition \ref{defgplus}, an element $s\in\Gamma(p,q)^+$ reads $s=cv_1\cdots v_{2r}$. Thus
\begin{equation}
 \det\chi(s)=\det\chi(v_1\cdots v_{2r})
            =\det\left[\chi(v_1)\ldots\chi(v_{2r})\right].
\end{equation}
 But we know that, for all $v_i$ in $V$, $det\chi(v_i)=-1$. So $\det\chi(s)=1$ and $\chi(\Gamma(p,q)^+)\subseteq \SO(p,q)$. As set, 
\[
  \Gamma(p,q)=\Gamma(p,q)^+\cup\Gamma(p,q)^-,
\]
but the lemma shows that $\det\chi(\Gamma(p,q)^-)=-1$ so, from theorem \ref{prop1001t1}, $\chi(\Gamma(p,q)^+)$ must be the whole $\SO(p,q)$.
\end{proof}


\begin{theorem}
We have the following isomorphism of groups
\[ 
  \Spin(p,q)=\SO_0(p,q).
\]
provided by the map $\chi$.
\end{theorem}

\begin{probleme}
	This result is wrong because of a double covering issue. The real proposition is the next one. I should try to merge the proofs.
\end{probleme}

\begin{proof}
Let $\{ e_1,\cdots,e_p,f_1,\cdots,f_p \}$ be a basis of $\eR^{p+q}$ where the $e_i$'s are time-like and the $f_j$'s are space-like.
Following the discussion at page \pageref{PgDisGeoConnSO}, we have
\[ 
  \SO(p,q)=\SO_0(p,q)\cup\xi \SO_0(p,q)
\]
where $\xi$ is defined as follows: $\xi e_1=-e_1$, $\xi f_1=-f_1$ and $\xi e_k=e_k$, $\xi f_k=f_k$ for $k\neq 1$. This element can be implemented as $\xi=\chi(g)$ for $g=e_1f_1$. It is easy to see that $g^{-1}=-f_1e_1$ and that $\tau(g)=f_1e_1$, so that $g\notin\Spin(p,q)$.

Is it possible to find another $h\in\Gamma(p,q)$ such that $\chi(h)=\xi$? If $\chi(a)=\chi(b)$ for $a$, $b\in\Gamma(p,q)$, then $a=rb$ for a certain $r\in\eR$. So we find that $h=g^{-1}/r$ is the general form of an element in $\Gamma(p,q)$ such that $\chi(h)=\xi$. This is an element of $\Spin(p,q)$ if and only if $\tau(h)=h^{-1}$, or $-e_1f_1/r=re_1f_1$ which has no solutions. We conclude that no element of $\Spin(p,q)$ is send on $\xi$ by $\chi$. So
\[ 
  \chi\big( \Spin(p,q) \big)\subset SO_0(p,q).
\]

\begin{probleme}
	I still have to prove the surjectivity of $\chi$ from $\Spin(p,q)$ to $\SO(p,q)$.
\end{probleme}

\end{proof}
\begin{theorem}
\begin{equation}	\label{EqchiSpinSO}	
             \chi(\Spin(p,q))=\SO_0(p,q)
\end{equation} 
where the index $0$ means the identity component.
\end{theorem}

\begin{proof}
Proposition \ref{prop1001p1}, theorem \ref{2102p1} and remark \ref{rem:spin_norm_u} show that the map $\dpt{\chi}{\Spin(p,q)}{\SO(p,q)}$ is a homomorphism with $\mathbb{Z}_2$ as kernel. We begin to prove that $\dpt{\chi}{\Spin(p,q)}{\SO_0(p,q)}$ is surjective. On the one hand, elements of $\Spin(p,q)$ satisfy one more condition than the ones of $\Gamma(p,q)^+$. Thus the algebra $\Spin(p,q)$ has codimension one in $\Gamma(p,q)^+$.

On the other hand, we know that $\SO(p,q)=\SO_0(p,q)\cup h\SO_0(p,q)$ where $h$ is the matrix such that $he_i=-e_i$ for $i=0,\ldots,3$. Since $\Spin(p,q)$ has codimension one in $\Gamma(p,q)^+$, there is at most one more generator in $\chi(\Gamma(p,q)^+)$ than in $\chi(\Spin(p,q))$ (because $\chi$ is a homomorphism). In order to prove this theorem, we just need to show that elements of $\chi(\Gamma(p,q)^+)$ which do not belong to $\chi(\Spin(p,q))$ is $h$.

Is is no difficult to see that $\chi(e_0\cdot e_1\cdot e_2\cdot e_3)e_i=-e_i$ for $i=0\ldots 3$: just write
$\chi(e_0\cdot e_1\cdot e_2\cdot e_3)e_i=e_0\cdot e_1\cdot e_2\cdot e_3\cdot e_i\cdot e_3^{-1}\cdot e_2^{-1}\cdot e_1^{-1}\cdot e_0^{-1}$ and use the commutation relations. An easy computation gives
$N(e_0\cdot e_1\cdot e_2\cdot e_3)=-1$; then this is not in $\Spin(p,q)$ by remark \ref{rem:spin_norm_u}.
\end{proof}

We write it by the exact sequence
\begin{equation}
 \xymatrix{
    \eZ_2  \ar@{^{(}->}[r] & \Sppq \ar[r]^{\chi} & \SO_0(p,q)
  } 
\end{equation}
we say that the group $\Spin(p,q)$ is a \defe{double covering}{double covering!of $\SO_{0}(p,q)$} of $\SO_0(p,q)$.

\begin{lemma}
If $\dpt{\pi}{\tX}{X}$ is a covering which satisfies

\begin{enumerate}
\item $X$ is path connected,
\item $\forall x\in X$, $\tX_x:=\pi^{-1}(x)$ is path connected in $\tX$ \emph{i.e.} for all $a$, $b\in \tX$, there exist a path in $\tX$ which joins $a$ and $b$,
\end{enumerate}
then $\tX$ is path connected.
\label{lem_cov_path_con}
\end{lemma}
\begin{proof}
If $\tx$ and $\ty$ are in $\tX$, we can suppose that $\pi(\tx)\neq\pi(\ty)$ (because if $\pi(\tx)=\pi(\ty)$, the second assumption gives the thesis). We define $x$ and $y$ as their projections: $x=\pi(\tx)$ and $y=\pi(\ty)$. Let $\gamma$ be a path such that $\gamma(0)=x$ and $\gamma(1)=y$, and $\tgamma$ be the lift of $\gamma$ in $\tX$ which contains $\tx$: $\tgamma(0)=\tx$ and $\pi(\tgamma(1))=\gamma(1)=y$. Then $\tgamma(1)$ lies in $\tX_y$. Therefore, we can consider $\gamma'$ which joins $\tgamma(1)$ and $\ty$.

So, $\gamma'\circ\tgamma$ is a path which contains $\tx$ and $\ty$.
\end{proof}


\begin{proposition}
 The group $\Spin(p,q)$ is connected.
\end{proposition}

\begin{proof}
We will prove that the covering $\dpt{\chi}{\Spin(p,q)}{\SO_0(p,q)}$ fulfils lemma \ref{lem_cov_path_con}. We just have to show that $\Spin(p,q)$ fulfills the second assumption of the lemma. First note that $\chi(\tx)=\chi(\ty)$ implies $\chi(\tx\ty^{-1})=e$, and then $\tx=\pm\ty$ because of proposition \ref{prop1001p1}. Since the other case is trivial, we can suppose $\tx=-\ty$.

It remains to prove that for every $g\in\Spin(p,q)$, there is a path in $\Spin(p,q)$ which joins $g$ and $-g$. The answer is given by the path $t\mapsto \gamma(t)g$ where
\[
\gamma(t)=\exp(te_1\cdot e_2)=\cos(t)(-1)+\sin(t)e_1\cdot e_2
\]
which satisfies $\gamma(0)=1$ and $\gamma(\pi)=-1$. 
\end{proof}

\begin{proposition}

The homomorphism $\tilde\rho$ restricts to a homomorphism $\tilde\rho\colon \Spin(p,q)\to \GL(\Lambda^+W)$.
\end{proposition}

\begin{proof}
An element in $\Spin(p,q)$ reads $s=cv_1\cdots v_{2r}$ and its image by $\tilde\rho$ is
\[ 
  \tilde\rho(s)=c\tilde\rho(v_1)\circ \cdots \circ\tilde\rho(v_{2r}).
\]
When one applies $\tilde\rho(v_1)$ to an element $\alpha\in\Lambda^kW$, one obtains a linear combination of an element of $\Lambda^{k-1}W$ and one of $\Lambda^{k+1}W$. The element $\tilde\rho(s)$ being an even composition of such maps, its transforms an element of $\Lambda^+W$ into an element of $\Lambda^+W$. 
\end{proof}

Notice that an element of $V$ ---no $V^{\eC}$--- is represented on $\Lambda^+W$ by complex matrices. This is not a problem. In the case of $\eR^{1,3}$, we have $\dim\Lambda^+W=2$ and thus 
\[ 
  \tilde\rho\big( \Spin(1,3) \big)\subset \GL(2,\eC).
\]
The following is the lemma 8.5 (page 57) of \cite{Michelson}.

\begin{lemma}
Let $\rho\colon \Cl(p,q)\to \Hom_{\eC}(E,E)$ be a representation of the Clifford algebra on a vector space $E$. If $p+q\geq 2$, then for all $s\in\Spin(p-1,q)\subset \Cl(p,q) $,
\[ 
  \det{}_{\eC}\big( \rho(s) \big)=\pm 1.
\]

\end{lemma}
\begin{proof}
No proof.
\end{proof}

\begin{theorem}
The representation $\tilde\rho$ provides a group isomorphism 
\[ 
  \Spin(1,3)\simeq \SL(2,\eC)
\]

\end{theorem}

\begin{proof}
In the case $p=2$, $q=3$, the lemma assures us that for each $s$ in the spin group, $\det\tilde\rho(s)=1$. Since $\Spin(1,3)$ is connected and the determinant function is continuous, we deduce that $\det\tilde\rho(s)\equiv 1$. This proves that $\tilde\rho\big( \Spin(1,3) \big)\subset \SL(2,\eC)$. The proposition \ref{PropUssGpGenere} thus implies that
\[ 
  \tilde\rho\big( \Spin(1,3) \big)=\SL(2,\eC),
\]
 but from $\Cl(1,3)$, the representation $\tilde\rho$ is yet injective. \emph{A forciori}, the representation $\tilde\rho$ is injective from $\Spin(1,3)$. This finishes the proof.
\end{proof}
