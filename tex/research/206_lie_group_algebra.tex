% This is part of Giulietta
% Copyright (c) 2013-2015, 2019
%   Laurent Claessens
% See the file fdl-1.3.txt for copying conditions.


Here are the results which relate Lie groups and Lie algebras.

%+++++++++++++++++++++++++++++++++++++++++++++++++++++++++++++++++++++++++++++++++++++++++++++++++++++++++++++++++++++++++++ 
\section{Lie algebra of a Lie group}
%+++++++++++++++++++++++++++++++++++++++++++++++++++++++++++++++++++++++++++++++++++++++++++++++++++++++++++++++++++++++++++

\begin{propositionDef}      \label{DEFooKDCPooZOJsMD}
    If \( G\) is a Lie group, its tangent space on at the identity is a Lie algebra. 
    
    This is the \defe{Lie algebra}{Lie algebra of a Lie group} is is tangent space at identity. From a notational point of view, this is written
    \begin{equation}
        \lG=T_eG.
    \end{equation}
\end{propositionDef}

\begin{proof}
    We know from proposition \ref{PROPooOHLQooCNetuD} that \( T_eG\) is a vector space. We have to define a Lie bracket on it. For that we use the left-invariant vector field. Let \( X\in T_eG\) and \( g\in M\) we define
    \begin{equation}
        X^L_g=dL_gX
    \end{equation}
    where \( L_g\colon G\to G\) is the left translation: \( L_g(h)=gh\). If \( X,Y\in T_eG\) we define
    \begin{equation}
        [X,Y]=[X^L,Y^L]_e
    \end{equation}
    where the bracket on the right hand side is the commutator of vector field defined in \ref{DEFooHOTOooRaPwyo}. It defines a Lie algebra structure by the proposition \ref{PROPooSWQSooSEfTuX}.
\end{proof}

In order to make the notations clear, let us write the formula explicitly. If \( X,Y\in T_eG\) are given by \( X=\alpha'(0)\) and \( Y=\beta'(0)\) we have
\begin{subequations}        \label{SUBEQSooHKWMooQbeStl}
    \begin{align}
        (XY)f&=X(Y(f))\\
        &=\Dsdd{ (Yf)\big( \alpha(t) \big) }{t}{0}\\
        &=\Dsdd{ Y_{\alpha(t)}(f) }{t}{0}\\
        &=\Dsdd{ Y^L_{\alpha(t)}(f) }{t}{0}\\
        &=\DDsdd{ f\big( \alpha(t)\beta(u) \big) }{t}{0}{s}{0}.
    \end{align}
\end{subequations}

Now a great theorem without proof:
\begin{theorem} \label{tho:loc_isom}
Two Lie groups are locally isomorphic if and only if their Lie algebras are isomorphic.
\end{theorem}

\begin{theorem}		\label{ThoSubGpSubAlg}		\label{tho:gp_alg}
If $G$ is a Lie group, then
\begin{enumerate}
\item\label{ThoSubGpSubAlgi} if $\lH$ is the Lie algebra of a Lie subgroup $H$ of $G$, then it is a subalgebra of $\lG$,
\item Any subalgebra of $\lG$ is the Lie algebra of one and only one connected Lie subgroup of $G$.
\end{enumerate}

\begin{probleme}
À mon avis, il faut dire ``connexe et simplement connexe'', et non juste ``connexe''.
\end{probleme}

\end{theorem}
\begin{proof}

\subdem{First item}
Let $\dpt{i}{H}{G}$ be the identity map; it is a homomorphism from $H$ to $G$, thus $di_e$ is a homomorphism from $\lH$ to $\lG$. Conclusion: $\lH$ is a subalgebra of $\lG$.

\subdem{Characterization for $\lH$}
Before to go on with the second point, we derive an important characterization of $\lH$:
\begin{equation}\label{eq:path_alg}
\lH=\{X\in\lG:\text{the map } t\to\exp tX\text{ is a path in $H$}\}.
\end{equation}
For that, consider $\dpt{\exp_H}{\lH}{H}$ and $\dpt{\exp_G}{\lG}{G}$; from unicity of the exponential, for any $X\in\lH$, $\exp_HX=\exp_GX$, so that one can simply write ``$\exp$''\ instead of ``$\exp_h$''\ or ``$\exp_G$''.

Now, if $X\in\lH$, the map $t\to\exp tX$ is a curve in $H$. But it is not immediately clear that such a curve in $H$ is automatically build from a vector in $\lH$ rather than in $\lG$.  More precisely, consider a $X\in\lG$ such that $t\to\exp tX$ is a path (continuous curve) in H. By lemma~\ref{lem:var_cont_diff}, the map $t\to\exp tX$ is differentiable and thus by derivation, $X\in\lH$.
The characterisation \eqref{eq:path_alg} is proved.

Thus $\lH$ is a Lie subalgebra of $\lG$.

\subdem{Second item}
For the second part, we consider $\lH$ any subalgebra of $\lG$ and $H$, the smallest subgroup of $G$ which contains $\exp\lH$. We also consider a basis $\{X_1,\ldots,X_n\}$ of $\lG$ such that $\{X_{r+1},\ldots,X_n\}$ is a basis of $\lH$.

By corollary~\ref{cor:/24}, the set of linear combinations of elements of the form $X(M)$ with $M=(0,\ldots,0,m_{r+1},\ldots,m_r)$ form a subalgebra of $U(\lG)$. If $X=x_1X_1+\cdots+x_nX_n$, we define $|X|=(x_1^2+\cdots+x_n^2)^{1/2}$ ($x_i\in\eR$).

Let us consider a $\delta>0$ such that $\exp$ is a diffeomorphism (normal neighbourhood) from $B_{\delta}=\{X\in\lG:|X|<\delta\}$ to a neighbourhood $N_e$ of $e\in G$ and such that $\forall x,y,xy\in N_e$,
\begin{equation}\label{eq:coord_xy}
   (xy)_k=\sum_{M,N}C^{[k]}_{MN}x^My^N
\end{equation}
holds\footnote{The validity of this second condition is assured during the proof of theorem~\ref{tho:loc_isom} which is not given here.}. We note $V=\exp(\lH\cap B_{\delta})\subset N_e$. The map
\[
   \exp(x_{r+1}X_{r+1}+\cdots+x_nX_n)\to(x_{r+1},\ldots,x_n)
\]
is a coordinate system on $V$ for which $V$ is a connected manifold. But $\lH\cap B_{\delta}$ is a submanifold of $B_{\delta}$, then $V$ is a submanifold of $N_e$ and consequently of~$G$.

Let $x$, $y\in V$ such that $xy\in N_e$ (this exist: $x=y=e$); the canonical coordinates of $xy$ are given by \eqref{eq:coord_xy}. Since $x_k=y_k=0$ for $1\leq k\leq r$, $(xy)_k=0$ for the same $k$ because for $(xy)_k$ to be non zero, one need $m_1=\ldots=m_r=n_1=\ldots=n_r=0$ -- otherwise, $x^M$ or $y^N$ is zero. Now we looks at $C^{[k]}_{MN}$ for such a $k$ (say $k=1$ to fix ideas): $[k]=(\delta_{11},\ldots,\delta_{1k})=(1,0,\ldots,0)$ and by definition of the $C$'s,
\[
   X(M)X(N)=\sum_PC_{MN}^PX(P).
\]
But we had seen that the set of the $X(A)$ with $A=(0,\ldots,0,a_{r+1},\ldots,a_n)$ form a subalgebra of $U(\lG)$. Then, only terms with $P=(0,\ldots,0,p_{r+1},\ldots,p_n)$ are present in the sum; in particular, $C_{MN}^{[k]}=0$ for $k=1,\ldots,r$. Thus $VV\cap N_e\subset V$.

The next step is to consider $\mV$, the set of all the subset of $H$ whose contains a neighbourhood of $e$ in $V$. We can check that this fulfils the six axioms of a topological group\index{topological!group}:

\begin{enumerate}
\item The intersection of two elements of $\mV$ is in $\mV$;
\item the intersection of all the elements of $\mV$ is $\{e\}$;
\item any subset of $H$ which contains a set of $\mV$ is in $\mV$;
\item If $\mU\in\mV$, there exists a $\mU_1\in\mV$ such that $\mU_1\mU_1\subset\mU$ because $VV\cap N_e\subset V$;
\item if $\mU\in\mV$, then $\mU^{-1}\in\mV$ because the inverse map is differentiable and transforms a neighbourhood of $e$ into a neighbourhood of $e$;
\item if $\mU\in\mV$ and $h\in H$, then $h\mU h^{-1}\in\mV$.
\end{enumerate}

To see this last item, we denote by $\log$ the inverse map of $\dpt{\exp}{B_{\delta}}{N_e}$. By definition of $V$, it sends $V$ on $\lH\cap B_{\delta}$. If $X\in\lG$, there exists one and only one $X'\in\lG$ such that $he^{tX}h^{-1}=e^{tX'}$ for any $t\in\eR$. Indeed we know that $he^{X}h^{-1}=e^{\Ad_hX}$, then $X'$ must satisfy $e^{tX'}=e^{\Ad_htX}$. If it is true for any $t$, then, by derivation, $X'=\Ad_hX$.

The map $X\to X'$ is an automorphism of $\lG$ which sent $\lH$ on itself. So one can find a $\delta_1$ with $0<\delta_1<\delta$ such that
\[
   h\exp({B_{\delta_1}\cap\lH})h^{-1}\subset V.
\]
Indeed, $he^{\lH} h^{-1}\subset\lH$, so that taking $\delta_1<\delta$, we get the strict inclusion. We can choose $\delta_1$ even smaller to satisfy $he^{B_{\delta_1}}h^{-1}\subset N_e$. Since the map $X\to\log(he^{X}h^{-1})$ from $B_{\delta_1\cap\lH}$ to $B_{\delta}\cap\lH$ is regular, the image of $B_{\delta_1}\cap\lH$ is a neighbourhood of $0$ in $\lH$. Thus $he^{B_{\delta_1}\cap\lH}h^{-1}$ is a neighbourhood of $e$ in $V$. Finally, $h\mU h^{-1}\in\mV$ and the last axiom of a topological group is checked.

This is important because there exists a topology on $H$ such that $H$ becomes a topological group and $\mV$ is a family of neighbourhood of $e$ in $H$. In particular, $V$ is a neighbourhood of $e$ in $H$.

For any $z\in G$, we define the map $\dpt{\phi_z}{zN_e}{B_{\delta}}$ by
\begin{equation}
  \phi_z(ze^{x_1X_1+\cdots+x_nX_n})=(x_1,\ldots,x_n),
\end{equation}
and we denote by $\varphi_z$ the restriction of $\phi_z$ to $zV$. If $z\in H$, then $\varphi_z$ sends the neighbourhood $zV$ of $z$ in $H$ to the open set $B_{\delta}\cap\lH$ in $\eR^{n-r}$. Indeed, an element of $zV$ is a $ze^Z$ with $Z\in\lH\cap B_{\delta}$ which is sent by $\varphi_z$ to an element of $\lH\cap B_{\delta}$. (we just have to identify $x_1X_1+\cdots+x_nX_n$ with $(x_1,\ldots,x_n)$).

Moreover, if $z_1,z_2\in H$, the map $\varphi_{z_1}\circ\varphi_{z_2}^{-1}$ is the restriction to an open subset of $\lH$ of $\phi_{z_1}\circ\phi_{z_2}$. Then $\varphi_{z_1}\circ\varphi_{z_2}^{-1}$ is differentiable. Conclusion: $(H,\varphi_z: z\in H)$ is a differentiable manifold.

Recall that the definition of $\lH$ was to be a subalgebra of $\lG$; therefore $V=e^{\lH\cap B_{\delta}}$ is a submanifold of $G$. But the left translations are diffeomorphism of $H$ and $H$ is the smallest subgroup of $G$ containing $e^{\lH}$. Thus $H$ is a manifold on which the multiplication is diffeomorphic and consequently, $H$ is a Lie subgroup of $G$.

Rest to prove that the Lie algebra of $H$ is $\lH$ and the unicity part of the theorem.

We know that $\dim H=\dim\lH$ and moreover for $i>r$, the map $t\to\exp tX_i$ is a curve in $H$. Now, the fact that $\lH$ is the set of $X\in\lG$ such that $t\to\exp tX$ is a path in $H$ show that $X_i\in\lH$. Then the Lie algebra of $H$ is $\lH$ and $H$ is a connected group because it is generated by $\exp\lH$ which is a connected neighbourhood of $e$ in $H$.

We turn our attention to the unicity part. Let $H_1$ be a connected Lie subgroup of $G$ such that $T_eH_1=\lH$. Since $\exp_{\lH}X=\exp_{\lH_1}X$, $H=H_1$ as set. But $\exp$ is a differentiable diffeomorphism from a neighbourhood of $0$ in $\lH$ to a neighbourhood of $e$ in $H$ and $H_1$, so as Lie groups, $H$ and $H_1$ are the same.
\end{proof}

Let us consider an element $X\in\lG$ such that $\exp tX\in H$ for every $t\in\eR$, and the map $\dpt{\varphi}{\eR}{G}$, $\varphi(t)=\exp tX$. This is continuous, then there exists a connected neighbourhood $\mU$ of $0$ in $\eR$ such that $\varphi(\mU)\subset V$. Then $\varphi(\mU)\subset H\cap V$ and the connectedness of $\varphi(\mU)$ makes $\varphi(\mU)\subset\exp\mU_h$. But $\exp\mU_h$ is an arbitrary small neighbourhood of $e$ in $H$; the conclusion is that $\varphi$ is a continuous map from $\eR$ into $H$. Indeed, we had chosen $X$ such that $\exp tX\in H$.

Moreover, we know that
\[
  e^{(t_0+\epsilon)X}=e^{t_0X}e^{\epsilon X},
\]
but $\exp \epsilon X$ can be as close to $e$ as we want (this proves the continuity at $t_0$). Then $\varphi$ is a path in $H$.

In definitive, we had shown that $\exp tX\in H$ implies that $t\to\exp tX$ is a path. Now equation \eqref{eq:path_alg} gives the result.

\end{proof}

\begin{corollary}
Let $G$ be a Lie group and $H_1$, $H_2$, two subgroups both having a finite number of connected components (each for his own topology). If $H_1=H_2$ as sets, then $H_1=H_2$ as Lie groups.
\end{corollary}

\begin{proof}
The proposition shows that $H_1$ and $H_2$ have same Lie algebra. But any Lie subalgebra of $\lG$ is the Lie algebra of exactly one connected subgroup of $G$ (theorem~\ref{tho:gp_alg}). Then as Lie groups, ${H_1}_0={H_2}_0$. Since $H_1$ and $H_2$ are topological groups, the equality of they topology on one connected component gives the equality everywhere (because translations are differentiable).
\end{proof}

\begin{lemma}
Let $\lG$ admit a direct sum decomposition (as vector space) $\lG=\lM\oplus\lN$. Then there exists open and bounded neighbourhoods $\mU_m$ and $\mU_n$ of $0$ in $\lM$ and $\lN$ such that the map
		\begin{equation}
		\begin{aligned}
			\phi \colon \mU_m\times\mU_n &\to G\
			(A,B)&\mapsto e^Ae^B
		\end{aligned}
	\end{equation}
is a diffeomorphism between $\mU_m\times\mU_n$ and an open neighbourhood of $e$ in $G$.
 \label{lem:decomp}
\end{lemma}


\begin{proof}
Let $\{X_1,\ldots,X_n\}$ be a basis of $\lG$ such that $X_i\in\lM$ for $1\leq i\leq r$ and $X_j\in\lN$ for $r<j\leq n$. We consider $\{t_1,\ldots,t_n\}$, the canonical coordinates of $\exp(x_1X_1+\cdots+x_rX_r)\exp(x_{r+1}X_{r+1}+\cdots+x_nX_n)$ in this coordinate system. By properties of the exponential, the function $\varphi_j$ defined by $t_j=\varphi_j(x_1,\ldots,x_n)$ is differentiable at $(0,\ldots,0)$. If $x_i=\delta_{ij}s$, then $t_i=\delta_{ij}s$ and the Jacobian of
\[
   \dsd{(\varphi_1,\ldots,\varphi_n)}{(x_1,\ldots,x_n)}
\]
is $1$ for $x_1=\ldots=x_n=0$. Thus $d\varphi_e$ is a diffeomorphism and so $\varphi$ is a locally diffeomorphic.
\end{proof}

\begin{theorem}
Let $G$ be a Lie group whose Lie algebra is $\lG$ and $H$, a closed subgroup (not specially a \emph{Lie} subgroup) of $G$. Then there exists one and only one analytic structure on $H$ for which $H$ is a topological Lie subgroup of $G$.
\label{tho:diff_sur_ferme}
\end{theorem}

\begin{remark}
A \textit{topological} Lie subgroup\index{topological!Lie subgroup} is stronger that a common Lie subgroup because it needs to be a topological subgroup: it must carry \emph{exactly} the induced topology. In our definition of a Lie group, this feature doesn't appears.
\end{remark}

\begin{proof}
   Let $\lH$ be the subspace of $\lG$ defined by
\begin{equation}\label{eq:lH_de_G}
  \lH=\{X\in\lG\tq \forall t\in\eR,\, e^{tX}\in H\}.
\end{equation}
We begin to show that $\lH$ is a subalgebra of $\lG$; i.e. to show that $t(X+Y)\in\lH$ and $t^2[X,Y]\in\lH$ if $X$, $Y\in\lH$. Remark that $X\in\lH$ and $s\in\eR$ implies $sX\in\lH$. Consider now $X$, $Y\in\lH$ and the classical formula:
\begin{subequations}
\begin{align}
\left(  \exp(\frac{t}{n}X)\exp(\frac{t}{n}Y)  \right )^n
                       =\exp( t(X+Y)+\frac{t^2}{2n}[X,Y]+o(\frac{1}{n^2}) ),\\
\left(  \exp(-\frac{t}{n}X)\exp(-\frac{t}{n}Y)\exp(\frac{t}{n}X)\exp(\frac{t}{n}Y)   \right)^{n^2}
                       =\exp\left( t^2[X,Y]+o(\frac{1}{n})\right).
\end{align}
\end{subequations}
The left hand side of these equations are in $H$ for any $n$; but, since $H$ is closed, it keeps in $H$ when $n\to\infty$. The right hand side, at the limit, is just $\exp(t(X+Y))$ and $\exp(t^2[X,Y])$, which keeps in $H$ for any $t$. Thus $X+Y$ and $[X,Y]$ belong to $\lH$. The space $\lH$ is thus a Lie subalgebra of $\lG$.

Let $H^*$ be the connected Lie subgroup of $G$ whose Lie algebra is $\lH$ (existence and unicity from~\ref{tho:gp_alg}). From the proof of theorem~\ref{tho:gp_alg}, we know that $H^*$ is the smallest subgroup of $G$ containing $\exp\lH$, then it is made up from products and inverses of elements of the type $e^X$ with $X\in\lH$, and thus is is included in $H$ by definition of $\lH$. So, $H^*\subset H$.

We will show that if we put on $H^*$ the induced topology from $G$ and if $H_0$ denotes the identity component of $H$, then $H^*=H_0$ as topological groups. For this, we first have to show the equality as set and then prove that if $N$ is a neighbourhood of $e$ in $H^*$, then it is a neighbourhood of $e$ in $H_0$. In facts, the equality as set can be derives from this second fact. Indeed, since $H_0$ is a connected topological group, it is generated by any neighbourhood of $e$, so if one can show that any neighbourhood $N$ of $e$ in $H^*$ is a neighbourhood of $e$ in $H$, then $H^*$ is a neighbourhood of $e$ in $H_0$ and then $H_0$ should be generated by $H^*$, so that $H_0\subset H^*$ (as set). Moreover, the most general element of $H^*$ is product and inverse of $e^X$ with $X\in\lH$ and $e^X$ is connected to $e$ by the path $e^{tX}$ ($\dpt{t}{1}{0}$). Then $H^*\subset H_0$, and $H^*=H_0$ as set. Immediately, $H^*=H_0$ as topological groups from our assertion about neighbourhoods of $e$. Let us now prove it.

We consider a neighbourhood $N$ of $e$ in $H^*$ and suppose that this is not a neighbourhood of $e$ in $H$. Thus there exists a sequence $c_k\in H\setminus N$ such that $c_k\to e$ in the sense of the topology on $G$. Indeed, a neighbourhood of $e$ in the sense of $H$ must contains at least a point which is not in $N$ because if we have an open set of $H$ around $e$ included in $N$, then $N$ is a neighbourhood of $e$ for $H$. So we consider a suitable sequence of such open sets around $e$ and one element not in $N$ in each of them. There is the $c_k$'s\quext{Je crois qu'on utilise l'axiome du choix.}.

Using lemma~\ref{lem:decomp} with a decomposition $\lG=\lH\oplus\lM$ (i.e. $\lM$: a complementary for $\lH$ for $\lG$), one can find sequences $A_k\in\mU_m$ and $B_k\in\mU_n$ such that
\[
   c_k=e^{A_k}e^{B_k}.
\]
Here, $\mU_m$ is an open neighbourhood of $0$ in $\lM$ and $\mU_h$, an open neighbourhood of $0$ in $\lH$.

As $e^{B_k}\in N$ and $c_k\in H\setminus N$, $A_k\neq 0$ and $\lim A_k=\lim B_k=0$ (because $(A,B)\to e^Ae^B$ is a diffeomorphism and $e^0e^0=e$ -- and also because all is continuous and thus has a good behaviour with respect to the limit). The set $\mU_m$ is open and bounded --this is a part of the lemma. Then there exist a sequence of positive reals numbers $r_k\in$ such that $r_kA_k\in\mU_m$ and $(r_k+1)A_k\notin\mU_m$. We know that $\mU_m$ is a bounded open subset of the vector space $\lM$, then the whole sequences $r_kA_k$ and $(r_k+1)A_k$ are in a compact domain of $\lM$. Then --by eventually considering subsequences-- there are no problems to consider limits of these sequences in $\lM$: $r_kA_k\to A\in\lM$ (not necessary in $\mU_m$). Since $A_k\to 0$, the point $A$ is the common limit of $r_kA_k\in\mU_m$ and of $(r_k+1)A_k\notin\mU_m$. Thus $A$ is in the boundary of $\mU_m$; in particular, $A\neq 0$.

On the other hand, consider two integers $p,q$ with $q>0$. One can find sequences $s_k,t_k\in\eN$ and $0\leq t_k<q$ such that $pr_k=qs_k+t_k$. It is clear that
\begin{equation}
  \lim_{k\to\infty}\frac{t_k}{q}A_k=0,
\end{equation}
thus
\[
   \exp \frac{p}{q}A=\lim \exp\frac{pr_k}{a}A_k=\lim (\exp A_k)^{s_k},
\]
which belongs to $H$. By continuity, $\exp tA\in H$ for any $t\in\eR$ and finally $A\in\lH$; this contradict $A\neq 0$ so that $A\in\lM$ (because by definition, $A\in\lM$ and the sum $\lG=\lH\oplus\lM$ is direct).

By its definition, $H^*$ has an analytic structure of Lie subgroup of $G$; but we had just proved that the induced topology from $G$ is the one of $H_0$ which by definition is a submanifold of $G$. So the set $H_0=H^*$ becomes a submanifold of $G$ whose topology is compatible with the analytic structure: thus it is a Lie subgroup of $G$. From analyticity, this structure is extended to the whole $H$.

\begin{probleme}
Est-ce bien vrai, tout \c ca ? En particulier, je n'utilise pas que $H_0$ est ouvert dans $H$ (ce qui est un tho de topo classique : je ne vois pas pourquoi Helgason fait tout un cin\'ema --que je ne comprends pas-- dessus). En prenant $N=H^*$, on a juste d\'emontr\'e que $H_0$ est un voisinage de $e$ dans $H$, mais ça, on le savait bien avant.
\end{probleme}

The unicity part comes from the corollary~\ref{cor:top_subgroup}.
\end{proof}


With the notations and the structure of theorem~\ref{tho:diff_sur_ferme}, the subgroup $H$ is discrete if and only if $\lH=\{0\}$. Indeed, recall the definition \eqref{eq:lH_de_G}:
\[
  \lH=\{X\in\lG: \forall t\in\eR, e^{tX}\in H\},
\]
and the fact that there exists a neighbourhood of $e$ in $H$ on which the exponential map is a diffeomorphism.

\begin{remark}
This fact should not be placed after the following lemma. In fact, we use here just the existence of normal neighbourhood (which is a common result) while the following lemma gives much more than normal neighbourhood.
\end{remark}

The lemma (without proof):

\begin{lemma}
 Let $G$ be a Lie group and $H$, a Lie subgroup of $G$ ($\lG$ and $\lH$ are the corresponding Lie algebras). If $H$ is a topological subspace of $G$, then there exists an open neighbourhood $V$ of $0$ in $\mG$ such that
 \begin{enumerate}
 \item $\exp$ is a diffeomorphism between $V$ and an open neighbourhood of $e$ in~$G$,
 \item $\exp(V\cap\lH)=(\exp V)\cap H$.
 \end{enumerate}
\label{lem:sugroup_normal}
\end{lemma}

\begin{definition}
A \defe{differentiable subgroup}{differentiable!subgroup} is a connected Lie subgroup.
\end{definition}

\begin{corollary}
Let $G$ be a Lie group, and $K$, $H$ two differentiable subgroups of $G$. We suppose $K\subset H$. Then $K$ is a differentiable subgroup of the Lie group $H$.
\end{corollary}

\begin{proof}
The Lie algebras of $K$ and $H$ are respectively denoted by $\lK$ and $\lH$. We denote by $K^*$ the differentiable subgroup of $H$ which has $\lK$ as Lie algebra. The differentiable subgroups $K$ and $K^*$ have same Lie algebra, and then coincide as Lie groups.
\end{proof}

\label{pg:ex_topo_Lie}
Consider the group $T=S^1\times S^1$ and the continuous map $\dpt{\gamma}{\eR}{T}$ given by
\[
  \gamma(t)=(e^{it},e^{i\alpha t})
\]
with a certain irrational $\alpha$ in such a manner that $\gamma$ is injective and $\Gamma=\gamma(\eR)$ is dense in $T$.

The subset $\Gamma$ is not closed because his complementary in $T$ is not open: any neighbourhood of element $p\in T$ which don't lie in $\Gamma$ contains some elements of $\Gamma$. We will show that the inclusion map $\dpt{\iota}{\Gamma}{T}$ is continuous. An open subset of $T$ is somethings like
\[
  \mO=(e^{iU},e^{iV})
\]
where $U,V$ are open subsets of $\eR$. It is clear that
\[
   \iota^{-1}(\mO)=\{ \gamma(t)\tq t\in U+2k\pi,\alpha t\in V+2m\pi \},
\]
but the set of elements $t$ of $\eR$ which satisfies it is clearly open. Then $\Gamma$ has at least the induced topology from $T$ (as shown in proposition~\ref{prop:topo_sub_manif}). In fact, the own topology of $\Gamma$ is \emph{more} than the induced: the open subsets of $\Gamma$ whose are just some small segments clearly doesn't appear in the induced topology. Thus the present case is an example (and not a counter-example) of theorem~\ref{tho:H_ferme}.

This example show the importance of the condition for a topological subspace to have \emph{exactly} the induced topology. If not, any Lie subgroup were a topological Lie subgroup because a submanifold has at least the induced topology. We will go further with this example after the proof.

%+++++++++++++++++++++++++++++++++++++++++++++++++++++++++++++++++++++++++++++++++++++++++++++++++++++++++++++++++++++++++++ 
\section{Matrix Lie group and its algebra}
%+++++++++++++++++++++++++++++++++++++++++++++++++++++++++++++++++++++++++++++++++++++++++++++++++++++++++++++++++++++++++++
\label{SECooTSAJooNtjgMD}

In this section we deal with Lie groups made from matrices, that is subgroups of \( \GL(n, \eC)\) (typically \( \SO(n)\) or \( \SU(n)\)) and their Lie algebra. We will denote the identity either by \( e\) or by \( \mtu\).

\begin{normaltext}      \label{NORMooHZGKooJEiamo}
    It is time to reread the remark \ref{REMooJQFHooQuoZxt}. In this section, when \( \gamma\) is a path in the matrix group \( G\), we denote by \( \gamma'(0)\) the ``usual'' derivative of \( \gamma\): that is the component-wise derivative; not the differential operator.

    We denote by \( D_{\gamma}\) the differential operator
    \begin{equation}
        \begin{aligned}
            D_{\gamma}\colon  C^{\infty}(G)&\to \eR \\
            f&\mapsto \Dsdd{ f\big( \gamma(t) \big) }{t}{0}. 
        \end{aligned}
    \end{equation}

    We aim to study the link between \( D_{\gamma}\) and \( \gamma'(0)\).

    From the Lie group of matrix \( G\) we can build (at least) two Lie algebras\footnote{Definition \ref{DEFooVBPKooGxlDBn}.}:
    \begin{itemize}
        \item The usual Lie algebra of the group: \( T_eG\) with the definition \ref{DEFooKDCPooZOJsMD}. As set, this is
            \begin{equation}
                T_eG=\{ D_{\gamma}\st \gamma(0)=e \}
            \end{equation}
            with the implicit that \( \gamma\) is a smooth path in \( G\).
        \item 
            The set of ``usual'' derivatives of the paths in \( G\):
            \begin{equation}
                G'=\{ \gamma'(0)\tq \gamma(0)=e \}.
            \end{equation}
            This is a set of matrices on which we can use the bracket \( [X,Y]=XY-YX\) (matrix product). We will see the following facts.
            \begin{itemize}
                \item 
                    The set \( G'\) is a Lie algebra in proposition \ref{PROPooUKITooLnEKZW},
                \item
                    The Lie algebras \( G'\) and \( T_eG\) are isomorphic as Lie algebras in theorem \ref{THOooWQGMooHyjRtx} for the case \( G=\GL(n,\eC)\)
                \item
                    When \( H\) is a Lie subgroup of \( \GL(n,\eC)\), the Lie algebras \( H'\) and \( T_eH\) are isomorphic as Lie algebras in proposition \ref{PROPooSQHLooGQAykc} for the Lie subgroups of \( \GL(n,\eC)\).
            \end{itemize}
    \end{itemize}
\end{normaltext}

\begin{lemma}[\cite{MonCerveau}]
    Let \( G\) be a matrix Lie group, et \( g\in G\) and \( X\in G'\). Then \( gXg^{-1}\in G'\).
\end{lemma}

\begin{proof}
    Let \( x\colon \eR\to G\) be a smooth path such that \( X=x'(0)\). Then we the derivative of the path given by the matrix product
    \begin{equation}
        t\mapsto gx(t)g^{-1}
    \end{equation}
    is \( gXg^{-1}\).
\end{proof}

\begin{lemma}[\cite{MonCerveau}]        \label{LEMooHQUYooSoiKbI}
    Let \( G\) be a matrix Lie group. Then \( G'\) is a vector space on \( \eR\).
\end{lemma}

\begin{proof}
    Let \( X,Y\in G'\) be the derivatives of the paths \( x\) and \( y\). If we set \( \varphi_1(t)=x(t)y(t)\) we have
    \begin{equation}
        \varphi_1'(0)=x'(0)y(0)+x(0)y'(0).
    \end{equation}
    Since \( x(0)=y(0)=e\) we have \( \varphi'(0)=X+Y\), so that \( X+Y\in G'\).

    For the product by a scalar, let the path \( \varphi_2(t)=x(\lambda t)\). The component-wise derivative
    \begin{equation}
        \varphi_2'(0)=\lambda x'(0)=\lambda X,
    \end{equation}
    so that \( \lambda X\in G'\).
\end{proof}

\begin{proposition}     \label{PROPooUKITooLnEKZW}
    Let \( G\) be a matrix Lie group. The vector space \( G'\) is a Lie algebra for the matrix commutator.
\end{proposition}

\begin{proof}
    We already know that \( G'\) is a real vector space by lemma \ref{LEMooHQUYooSoiKbI}. The fact that \( (X,Y)\mapsto XY-YX\) satisfies the axioms of a Lie algebra is easy to check. The only point is to show that if \( X,Y\in G'\), then \( [X,Y]=XY-YX\in G'\).

    Let
    \begin{equation}        \label{EQooJDTLooGWsDiq}
        \varphi(t)=x(t)Yx(-t).
    \end{equation}
    This is for sure a path in the full matrix vector space, and this is derivable because \( x\) is derivable while the matrix product is linear. So the derivative \( \varphi'(0)\) is still a matrix. The question is: why \( \varphi'(0)\in G'\) ?

    By lemma \ref{LEMooHQUYooSoiKbI}, for each \( t\) we have
    \begin{equation}
        \frac{ \varphi(t)-\varphi(0) }{ t }\in G'.
    \end{equation}
    Now, \( G'\) is a vector subspace of \( \eM(n,\eC)\) which is finite dimensional; is is thus closed and the limit belongs to \( G'\).

    Is is now a simple computation to show that \( \varphi'(0)=[X,Y]\).
\end{proof}

\begin{normaltext}
The following theorem is a Giulietta's masterpiece in the following sense:
\begin{itemize}
    \item It is fundamental because the Lie algebra isomorphism between \( T_eGL(n,\eR)\) and the matrices is used everywhere one says «The Lie algebra of $\SO(3)$ is the set of skew-symmetric traceless matrices».
    \item
        Either I'm idiot, either I never seen that theorem even stated (let alone being proved)\footnote{There is in fact a third possibility:  this theorem is a classic one but cannot be found \emph{on internet}.}.
    \item
        I think that the fundamental misunderstanding\footnote{Once again, either I'm idiot either everybody is wrong but me\ldots well \ldots} is that in the context of Lie groups, people \emph{define} \( [X,Y]\) as being \( \ad(X)Y\) while \( \ad\) is defined as the ``second differential'' of \( \AD(g)h=ghg^{-1}\). In that case, obviously we get \( [X,Y]=XY-YX\) with the matrix product. This way fails to make the link with the commutator of vector fields as defined by \ref{DEFooHOTOooRaPwyo}.
    \item
        So you must read this proof with much care and write me if you see any mistake or unclear point.
\end{itemize}
\end{normaltext}
So here it is with the notations explained in \ref{NORMooHZGKooJEiamo}.
    

\begin{theorem}     \label{THOooWQGMooHyjRtx}
    Let \( G=\GL(n,\eC)\) be the group of invertible matrices. The map
    \begin{equation}
        \begin{aligned}
            \phi\colon G'&\to T_eG \\
            \gamma'(0)&\mapsto D_{\gamma} 
        \end{aligned}
    \end{equation}
    is 
    \begin{enumerate}
        \item
            well defined,
        \item
            bijective,
        \item
            linear,
        \item
            a Lie algebra isomorphism.
    \end{enumerate}
\end{theorem}

\begin{proof}
    Several points to be proved.
    \begin{subproof}
        \item[\( \phi\) is well defined]
            Let \( \alpha\) and \( \beta\) be paths in \( G\) such that \( \alpha'(0)=\beta'(0)\) and let \( f\colon G\to \eR\) be a smooth function. We have to prove that \( D_{\alpha}(f)=D_{\beta}(f)\).

            We consider a chart \( \varphi\colon \mU\to \mO\) where \( \mU\) is a neighbourhood of \( 0\) in \( \eR^m\) and \( \mO\) is a neighbourhood of \( e\) in \( \GL(n,\eC)\). We suppose that \( \varphi(0)=e\). We set \( \tilde f=f\circ \varphi\), \( \tilde \alpha=\varphi^{-1}\circ \alpha\) and \( \tilde \beta=\varphi^{-1}\circ\beta\). We have
            \begin{subequations}
                \begin{align}
                    D_{\alpha}(f)&=\Dsdd{ f\big( \alpha(t) \big) }{t}{0}\\
                    &=\Dsdd{ \tilde f\big( \tilde \alpha(t) \big) }{t}{0}\\
                    &=\sum_{i=1}^m\frac{ \partial \tilde f }{ \partial x_i }\big( \tilde \alpha(0) \big)\tilde \alpha_i(0).
                \end{align}
            \end{subequations}
            Since \( \tilde \alpha(0)=\tilde \beta(0)\) we still have to prove that \( \tilde \alpha_i'(0)=\tilde \beta_i'(0)\). As you remember, \( \tilde \alpha\) is a map from \( \eR\) to \( \eR^m\), so that the following derivative is quite usual:
            \begin{subequations}
                \begin{align}
                    \tilde \alpha'(0)&=\Dsdd{ (\varphi^{-1}\circ \alpha)(t) }{t}{0}\\
                    &=d\varphi^{-1}_{\alpha(0)}\big( \alpha'(0) \big)\\
                    &=d\varphi^{-1}_{\beta(0)}\big( \beta'(0) \big).
                \end{align}
            \end{subequations}
            Thus the map \( \phi\) is well defined.
        \item[\( \phi\) is linear]
            This is from the linearity of the derivation.
        \item[\( \phi\) is injective]
            If \( \phi(\alpha')=\phi(\beta')\), then \( D_{\alpha}(f)=D_{\beta}(f)\) for every function \( f\). In that case,
            \begin{equation}
                \sum_{i=1}^m\frac{ \partial \tilde f }{ \partial x_i }(e)\tilde \alpha_i'(0)=\sum_{i=1}^m\frac{ \partial \tilde f }{ \partial x_i }(e)\tilde \beta_i'(0).
            \end{equation}
            That equation must be satisfied for every function. Taking the projection on the components, we get \( \tilde \alpha_i'(0)=\tilde b_i'(0)\), which means \( \alpha'(0)=\beta'(0)\) because \( \varphi^{-1}\) is bijective.
        \item[\( \phi\) is surjective]
            Every element of \( T_eG\) is of the form \( D_{\alpha}\) for some path \( \alpha\), so \( \phi\) is surjective.
        \item[\( \phi\) is a Lie algebra isomorphism]
            Let \( X,Y\in G'\) being the derivative of the paths \( \alpha\) and \( \beta\). We have to prove that
            \begin{equation}
                [\phi(X),\phi(Y)]=\phi[X,Y].
            \end{equation}
            If \( t\) is small enough, the paths
            \begin{subequations}
                \begin{align}
                    \alpha(t)=\mtu+tX\\
                    \beta(t)=\mtu+tY\\
                \end{align}
            \end{subequations}
            are good ones because \( \det(\mtu)\neq 0\), so that the determinant of \( \mtu+tX\) remains different from zero when \( t\) is small, whatever \( X\) is. So \( \alpha\) and \( \beta\) are paths in \( \GL(n,\eC)\). Using the general definition in differential geometry,
            \begin{subequations}        \label{SUBEQSooCYRDooFOdLrn}
                \begin{align}
                    [\phi(X),\phi(Y)]f&=[\phi(X)^L,\phi(Y)^L]_ef\\
                    &=\phi(X)^L_e\big( \phi(Y)^L(f) \big)-\phi(Y)^L_e\big( \phi(X)^L(f) \big) \label{SUBEQooOPUAooZYsZlX}.
                \end{align}
            \end{subequations}
            We focus on the first term:
            \begin{subequations}        \label{SUBEQooTUNFooFkDmuP}
                \begin{align}
                    \phi(X)^L\big( \phi(Y)^L(f) \big)&=\Dsdd{ \phi(Y)^L_{\phi(X)^L_e(t)}(f) }{t}{0}\\
                    &=\DDsdd{ f\big( (\mtu+tX)(\mtu+sY) \big) }{t}{0}{s}{0}\\
                    &=\DDsdd{ f(\mtu+tX+sY+tsXY) }{t}{0}{s}{0}\\
                    &=\Dsdd{ df_{\mtu+tX}\big( (\mtu+tX)Y \big) }{t}{0} \label{SUBEQooLHPBooTnXiZd}\\
                    &=\Dsdd{ df_{\mtu+tX}(Y) }{t}{0}+\Dsdd{ df_{\mtu+tX}(tXY) }{t}{0}   \label{SUBEQooMXJJooBFTLsM}
                \end{align}
            \end{subequations}
            where we have used the linearity of \( df_{\mtu+tX}\) and where \( XY\) stands for the matrix product. In the expression \eqref{SUBEQooLHPBooTnXiZd}, the symbol \( df\) stands for the differential of \( f\) as function from \( \eM(n,\eC)\) (as vector space), not for the differential of \( f\) on \( G\) as manifold. This is why we are allowed to put an expression as the matrix \( Y\) as argument of \( df_{\mtu+tX}\) while \( Y\) is not an element of \( T_{\mtu+tX}G\).

            The expression \eqref{SUBEQooMXJJooBFTLsM} is still made of two terms. The second one is
            \begin{equation}
                \Dsdd{ df_{\mtu+tX}(tXY) }{t}{0}=\Dsdd{ tdf_{\mtu+tX}(XY) }{t}{0}=df_{\mtu}(XY)
            \end{equation}
            where we used the Leibnitz rule\footnote{In general, notice that \( \Dsdd{ tf(t) }{t}{0}=f(0)\)}.

            The first term in \eqref{SUBEQooMXJJooBFTLsM} is computed as
            \begin{equation}
                    \Dsdd{ df_{\mtu+tX}(Y) }{t}{0}=\DDsdd{ f(\mtu+tX+sY) }{t}{0}{s}{0}.
            \end{equation}
            We set 
            \begin{equation}
                \begin{aligned}
                    \gamma\colon \eR^2&\to G \\
                    (t,s)&\mapsto \mtu+tX+sY, 
                \end{aligned}
            \end{equation}
            so that
            \begin{subequations}
                \begin{align}
                    \Dsdd{ df_{\mtu+tX}(Y) }{t}{0}&=\DDsdd{ f(\mtu+tX+sY) }{t}{0}{s}{0}\\
                    &=\DDsdd{ (\tilde f\circ\varphi^{-1}\circ\gamma)(t,s) }{t}{0}{s}{0}\\
                    &=\DDsdd{ g(t,s) }{t}{0}{s}{0}
                \end{align}
            \end{subequations}
            where the function \( g=\tilde f\circ\varphi^{-1}\circ \gamma\) is a smooth function from \( \eR^2\) to \( \eR\).        

            The expression \eqref{SUBEQooTUNFooFkDmuP} is now
            \begin{equation}
                \phi(X)^L\big( \phi(Y)^L(f) \big)=\DDsdd{ g(t,s) }{t}{0}{s}{0}+df_{\mtu}(XY).
            \end{equation}
            The commutator we have to compute, with the same computations is
            \begin{equation}
                [\phi(X),\phi(Y)]f=\DDsdd{ g(t,s) }{t}{0}{s}{0}+df_{\mtu}(XY)-\DDsdd{ g(s,t) }{t}{0}{s}{0}-df_{\mtu}(YX).
            \end{equation}
            The function \( g\) being \(  C^{\infty}\), the derivative commute and the corresponding termes annihilate each other and we are left with
            \begin{equation}
                [\phi(X),\phi(Y)]f=df_{\mtu}(XY)-df_{\mtu}(YX)=df_{\mtu}(XY-YX)
            \end{equation}
            where we used the linearity of the differential.

            In the other sense,
            \begin{equation}
                \phi[X,Y]f=\Dsdd{ f(\mtu+tXY-tYX) }{t}{0}=df_{\mtu}\big( [X,Y] \big)
            \end{equation}
            where, once again, \( df\) stands for the ``usual'' differential.
    \end{subproof}
\end{proof}

Ok. This is proved for \( G=\GL(n,\eC)\), the full matrix group. What about subgroups ? Here is the result.

\begin{proposition}[\cite{MonCerveau}]      \label{PROPooSQHLooGQAykc}
    Let \( H\) be a Lie subgroup\footnote{Thanks to the Cartan theorem \ref{THOooDEJHooVKJYBL}, there are plenty of them.} of \( \GL(n,\eC)\). With the same notations as above, the map
    \begin{equation}
        \begin{aligned}
            \phi\colon H'&\to T_eH \\
            \gamma'(0)&\mapsto D_{\gamma} 
        \end{aligned}
    \end{equation}
    is a Lie algebra isomorphism.
\end{proposition}

\begin{proof}
    We have to prove that
    \begin{equation}        \label{EQooRLBBooYgHhtH}
        \phi[X,Y]f=[\phi(X),\phi(Y)]f
    \end{equation}
    for every \( X,Y\in H'\) and \( f\in  C^{\infty}(H)\). For that, we will see the left and right hand sides of \eqref{EQooRLBBooYgHhtH} in \( G=\GL(n,\eC)\), and use the already proved result, theorem \ref{THOooWQGMooHyjRtx}.

    If \( X,Y\in H'\) we know from proposition \ref{PROPooUKITooLnEKZW} that \( [X,Y]\in H'\). Thus there exists a path \( \gamma\colon \eR\to H\) such that \( [X,Y]=\gamma'(0)\). We consider the extension\footnote{The proposition \ref{PROPooOTZQooIfboXV} can be used since \( H\) is a submanifold of \( G\) by \ref{PROPooFXZJooCOFXZX}.} \( \tilde f\colon W\to \eR\) of \( f\) such that \( \tilde f=f\) on \( H\) and \( W\) is an open set around \( e\) in \( \GL(n,\eC)\). For the sake of making things complicated we also define \( \tilde \gamma=\iota\circ \gamma\) where \( \iota\colon H\to \GL(n,\eC)\) is the inclusion. With all that we have
    \begin{equation}
        \phi[X,Y]f=\Dsdd{ f\big( \gamma(t) \big) }{t}{0}=\Dsdd{ \tilde f\big( \tilde \gamma(t) \big) }{t}{0}=\clubsuit.
    \end{equation}
    At this point, notice that \( [X,Y]\in \GL(n,\eC)'\) and \( [X,Y]=\tilde \gamma'(0)\), so that if we consider \( \tilde \phi\colon \GL(n,\eC)\to T_e\GL(n,\eC)\) we also have
    \begin{equation}
        \clubsuit=\Dsdd{ \tilde f\big( \tilde \gamma(t) \big) }{t}{0}=\tilde \phi[X,Y]\tilde f=\big[ \tilde \phi(X),\tilde \phi(Y) \big]\tilde f
    \end{equation}
    where we used the result \ref{THOooWQGMooHyjRtx} on \( \GL(n,\eC)\).

    We still have to prove that \( \tilde \phi(X)\tilde \phi(Y)\tilde f=\phi(X)\phi(Y)f\). Using, among others the formula \ref{SUBEQSooHKWMooQbeStl} adapted to \( \tilde \phi(X)\) instead of \( X\):
    \begin{subequations}
        \begin{align}
            \tilde \phi(X)\tilde \phi(Y)\tilde f&=\Dsdd{ \big( \tilde \phi(Y)^L\tilde f \big)\big( \alpha(t) \big) }{t}{0}\\
            &=\Dsdd{ \tilde \phi(Y)^L_{\alpha(t)}\tilde f }{t}{0}\\
            &=\DDsdd{ \tilde f\big( \alpha(t)\beta(u) \big) }{t}{0}{s}{0}.
        \end{align}
    \end{subequations}
    At this point, notice that \( \alpha(t)\) and \( \beta(u)\) are elements in \( H\) which is a group, so \( \tilde f\big( \alpha(t)\beta(u) \big)=f\big( \alpha(t)\beta(u) \big)\). Thus
    \begin{subequations}
        \begin{align}
            \tilde \phi(X)\tilde \phi(Y)\tilde f&=\DDsdd{ \tilde f\big( \alpha(t)\beta(u) \big) }{t}{0}{s}{0}\\
            &=\DDsdd{ f\big( \alpha(t)\beta(u) \big) }{t}{0}{s}{0}\\
            &=\phi(X)\phi(y)f.
        \end{align}
    \end{subequations}
\end{proof}

\begin{lemma}[\cite{MonCerveau}]
    Let \( G\) be a Lie group of matrices and \( X\in T_eG\) such that 
    \begin{equation}
        df_e(X)=0
    \end{equation}
    for every smooth function \( f\colon G\to \eR\). Then \( X=0\).
\end{lemma}

\begin{proof}
    We consider the functions \( \pr_{ij}\colon G\to \eR\) defined by \( \pr_{ij}(A)=A_{ij}\). If \( g\colon \eR\to G\) is a path, for every \( t\) we have \( \pr_{ij}g(t)=g(t)_{ij}\) and then
    \begin{equation}
        \Dsdd{ \pr_{ij}g(t) }{t}{0}=g'(0)_{ij}.
    \end{equation}
    Then we build
    \begin{equation}
        \begin{aligned}
            f\colon G&\to \eR \\
            A&\mapsto \pr_{11}(A)\pr_{ij}(A). 
        \end{aligned}
    \end{equation}
    If \( g\colon \eR\to G\) is a path such that \( g(0)=e\) and \( g'(0)=X\), then we have
    \begin{subequations}
        \begin{align}
            \Dsdd{ f\big( g(t) \big) }{t}{0}&=\Dsdd{ \pr_{11}\big( g(t) \big)\pr_{ij}\big( g(t) \big) }{t}{0}\\
            &=\pr_{11}g(0)\Dsdd{ \pr_{ij}g(t) }{t}{0}+\Dsdd{ \pr_{11}g(t) }{t}{0}\pr_{ij}g(0)\\
            &=X_{ij}+\delta_{ij}X_{11}\\
            &=X_{ij}+\delta_{ij}X_{11}.
        \end{align}
    \end{subequations}
    We know that this is zero for every choice of \( ij\):
    \begin{equation}
        X_{ij}+\delta_{ij}X_{11}=0
    \end{equation}
    In particular with \( i=j=1\) we have \( 2X_{11}=0\), so that \( X_{11}=0\). Then we are left with \( X_{ij}=0\) for every \( ij\).
\end{proof}

\section{Adjoint group, inner automorphisms}\label{sec:adj_gp}
%--------------------------

Let $\lA$ be a \emph{real} Lie algebra. We denote by $GL(\lA)$\nomenclature[G]{$GL(\lA)$}{The group of nonsingular endomorphisms of $\lA$} the group of all the nonsingular endomorphisms of $\lA$: the linear and nondegenerate operators on $\lA$ as vector space. An element $\sigma\in\GL(\lA)$ does not specially fulfils somethings like $\sigma[X,Y]=[\sigma X,\sigma Y]$. The Lie algebra $\gl(\lA)$\nomenclature[G]{$\protect\gl(\lA)$}{space of endomorphisms with usual bracket} is the vector space of the endomorphisms (without non degeneracy condition) endowed with the usual bracket $(\ad A)B=[A,B]=A\circ B-B\circ A$. The map $X\to\ad X$ is a homomorphism from $\lA$ to the subalgebra $\ad(\lA)$ of $\gl(\lA)$.

The group $\Int(\lA)$\nomenclature[G]{$\Int(\lA)$}{Adjoint group of $\lA$} is the analytic Lie subgroup of $\GL(\lA)$ whose Lie algebra is $\ad(\lA)$ by theorem~\ref{tho:gp_alg}. This is the \defe{adjoint group}{adjoint!group}\index{group!adjoint} of $\lA$.

\begin{proposition}
The group $\Aut(\lA)$\nomenclature[G]{$\Aut\lA$}{Group of automorphisms of $\lA$} of all the automorphisms of $\lA$ is a closed subgroup of $\GL(\lA)$.
\end{proposition}

\begin{proof}
The property which distinguish the elements in $\Aut(\lA)$ from the ``commons'' elements of $\GL(\lA)$ is the preserving of structure: $\varphi[A,B]=[\varphi A,\varphi B]$. These are equalities, and we know that a subset of a manifold which is given by some equalities is closed.
\end{proof}

Now, theorem~\ref{tho:diff_sur_ferme} provides us an unique analytic structure on $\Aut(\lA)$ in which it is a topological Lie subgroup of $\GL(\lA)$. From now we only consider this structure. We denote by $\partial(\lA)$\nomenclature[G]{$\partial\lA$}{The Lie algebra of $\Aut(\lA)$} the Lie algebra of $\Aut(\lA)$: this is the set of the endomorphisms $D$ of $\lA$ such that $\forall t\in\eR$, $e^{tD}\in\Aut(\lA)$. By differencing the equality
\begin{equation}\label{eq:exp_der}
  e^{tD}[X,Y]=[e^{tD}X,e^{tD}Y]
\end{equation}
with respect to $t$, we see\footnote{As usual, if we consider a basis of $\lA$ as vector space, the expression in the right hand side of \[[e^{tD}X,e^{tD}Y]=\ad(e^{tD}X)e^{tD}X\] can be seen as a product matrix times vector, so that Leibnitz works.} that $D$ is a \defe{derivation}{derivation!of a Lie algebra} of $\lA$:
\begin{equation}
  D[X,Y]=[DX,Y]+[X,DY]
\end{equation}
for any $X$, $Y\in\lA$. Conversely, consider $D$, any derivation of $\lA$; by induction,
\begin{equation}
   D^k[X,Y]=\sum_{i+j=k}\frac{k!}{i!j!}[D^iX,D^jY]
\end{equation}
where by convention, $D^0$ is the identity in $\lA$. This relation shows that $D$ fulfils condition \eqref{eq:exp_der}, so that any derivation of $\lA$ lies in $\partial(\lA)$. Then
\[
  \partial(\lA)=\{\text{derivations of }\lA\}.
\]
The Jacobi identities show that
\[
\ad(\lA)\subset\partial(\lA).    \label{pg:ad_subset_der}
\]
From this, we deduce\footnote{See error~\ref{err:Intt_Aut}}: 
\begin{equation}\label{eq:int_sub_aut}
  \Int(\lA)\subset\Aut(\lA).
\end{equation}
Indeed the group $\Int(\lA)$ being connected, it is generated\footnote{See proposition~\ref{PropUssGpGenere}} by any neighbourhood of $e$; note that $\Aut(\lA)$ has not specially this property. We take a neighbourhood of $e$ in $\Int(\lA)$ under the form  $\exp V$  where $V$ is a sufficiently small neighbourhood of $0$ in $\ad(\lA)$ to be a neighbourhood of $0$ in $\partial(\lA)$ on which $\exp$ is a diffeomorphism. In this case, $\exp V\subset\Aut(\lA)$ and then $\Int(\lA)\subset\Aut(\lA)$.

Elements of $\ad(\lA)$ are the \defe{inner derivations}{derivation!inner} while the ones of $\Int(\lA)$ are the \defe{inner automorphisms.}{inner!automorphism}

Let $\mO$ be an open subset of $\Aut(\lA)$; for a certain open subset $U$ of $\GL(\lA)$, $\mO=U\cap\Aut(\lA)$. Then
\begin{equation}
  \iota^{-1}(\mO)=\mO\cap\Int(\lA)
           =U\cap\Aut(\lA)\cap\Int(\lA)
       =U\cap\Int(\lA).
\end{equation}

The subset $U\cap\Int(\lA)$ is open in $\Int(\lA)$ for the topology because $\Int(\lA)$ is a Lie\quext{Is it true??} subgroup of $\GL(\lA)$ and thus has at least the induced topology. This proves that the inclusion map $\dpt{\iota}{\Int(\lA)}{\Aut(\lA)}$ is continuous.

The lemma \ref{lem:var_cont_diff} and the consequence below makes $\Int(\lA)$ a Lie subgroup of $\Aut(\lA)$. Indeed $\Int(\lA)$ and $\Aut(\lA)$ are both submanifolds of $\GL(\lA)$ which satisfy \eqref{eq:int_sub_aut}. 


By definition, $\Aut(\lA)$ has the induced topology from $\GL(\lA)$. Then $\Int(\lA)$ is a submanifold of $\Aut(\lA)$. 
This is also a subgroup and a topological group : $\Int(\lA)$ is not a topological subgroup of $\Aut(\lA)$. Then $\Int(\lA)$ is a Lie subgroup of $\Aut(\lA)$. Schematically, links between $\Int\lG$, $\ad\lG$, $\Aut\lG$ and $\partial\lG$ are
\begin{subequations}\label{eq:schem_ad_int}
\begin{align}
  \Int\lG&\longleftarrow\ad\lG\\
  \Aut\lG&\longrightarrow\partial\lG.
\end{align}
\end{subequations}
Remark that the sense of the arrows is important. By definition $\partial\lG$ is the Lie algebra of $\Aut\lG$, then there exist some algebras $\lG$ and $\lG'$ with $\Aut\lG\neq\Aut\lG'$ but with $\partial\lG=\partial\lG'$, because the equality of two Lie algebras doesn't implies the equality of the groups. The case of $\Int\lG$ and $\ad\lG$ is very different: the group is defined from the algebra, so that $\ad\lG=\ad\lG'$ implies $\Int\lG=\Int\lG'$ and $\Int\lG=\Int\lG'$ if and only if $\ad\lG=\ad\lG'$.

A result about the group of inner automorphism which will be useful later:

\begin{lemma}\label{lem:Int_g_gR}
If $\lG$ is a complex semisimple Lie algebra, then $\Int\lG=\Int\lG\heR$.
\end{lemma}

\begin{proof}
If $\{X_i\}$ is a basis of $\lG$, then $\{X_j,iX_j\}$ is a basis of $\lG\heR$. We define $\dpt{\psi}{\ad\lG}{\ad\lG\heR}$ by
\[
   \psi(\ad(a^jX_j))=\ad(a^jX_j).
\]
It is clearly surjective. On the other hand, if $\ad(a^jX_j)\ad(b^kX_k)$ as elements of $\ad\lG\heR$, then they are equals as elements of $\ad\lG$. The discussion following equations \eqref{eq:schem_ad_int} finishes the proof.
\end{proof}

\begin{corollary}
Any two Cartan involutions of a real semisimple Lie algebra are conjugate by an inner automorphism. \index{inner!automorphism}
\label{cor:Cartan_conj_inner}
\end{corollary}

\begin{proof}
Let $\sigma$ and $\sigma'$ be two Cartan involutions of $\lF$. We can find a $\varphi\in\inf\lF$ such that $[\varphi\sigma\varphi^{-1},\sigma']=0$. Thus it is sufficient to prove that any two Cartan involutions which commute are equals. So let us consider $\theta$ and $\theta'$, two Cartan involutions such that $[\theta,\theta']=0$. By lemma~\ref{lem:invol_compat}, we know that the decompositions into $+1$ and $-1$  eigenspaces with respect to $\theta$ and $\theta'$ are compatibles. If we consider $X\in\lF$ such that $\theta X=X$ and $\theta' X=-1$ (it is always possible if $\theta\neq\theta'$), we have
\[
\begin{split}
  0<B_{\theta}(X,X)=-B(X,\theta X)=-B(X,X)\\
  0<B_{\theta'}(X,X)=-B(X,\theta' X)=B(X,X)
\end{split}
\]
which is impossible.
\end{proof}

\begin{corollary}
Any two real compact form of a complex semisimple Lie algebra are conjugate by an inner automorphism.
\end{corollary}

\begin{proof}
    We know that any real form of $\lG$ induces an involution (the conjugation) and that if the real form is compact, the involution is Cartan on $\lG\heR$. Let $\lU_0$ and $\lU_1$ be two compact real forms of $\lG$ and $\tau_0$, $\tau_1$ the associated involutions of $\lG$ (which are Cartan involutions of $\lG\heR$). For a suitable $\varphi\in\Int\lG\heR$,
    \[
       \tau_0=\varphi\tau_1\varphi^{-1}.
    \]
    The fact that $\Int\lG=\Int\lG\heR$ (lemma~\ref{lem:Int_g_gR}) finishes the proof.
\end{proof}

\begin{proposition}
 The group $\Int(\lA)$ is a normal subgroup of $\Aut(\lA)$.
\end{proposition}

\begin{proof}
Let us consider a $s\in\Aut(\lA)$. The map $\dpt{\sigma_s}{\Aut(\lA)}{\Aut(\lA)}$, $\sigma_s(g)=sgs^{-1}$ is an automorphism of $\Aut(\lA)$. Indeed, consider $g$, $h\in\AutA$; direct computations show that $\sigma_s(gh)=\sigma_s(g)\sigma_s(h)$ and $[\sigma_s(g),\sigma_s(h)]=\sigma_s([g,h])$. From this, $(d\sigma_s)_e$ is an automorphism of $\partial(\lA)$, the Lie algebra of $\AutA$. For any $D\in\partial(\lA)$ we have
\begin{equation}\label{eq:ad_s_2}
 (d\sigma_s)_eD=\Dsdd{ sD(t)s^{-1} }{t}{0}
             =sDs^{-1}.
\end{equation}
Since $s$ is an automorphism of $\lA$ and $\ad(\lA)$, a subalgebra of $\gl(\lA)$,
\begin{equation}\label{eq:ad_s_1}
  s\ad Xs^{-1}=\ad(sX)
\end{equation}
for any $X\in\lA$, $s\in\Aut(\lA)$. Since $\ad(\lA)\subset\partial(\lA)$, we can write \eqref{eq:ad_s_2} with $D=\ad X$ and put it in \eqref{eq:ad_s_1}:
\[
   (d\sigma)_e\ad X=s\ad Xs^{-1}=\ad(s\cdot X).
\]
We know from general theory of linear operators on vector spaces that if $A,B$ are endomorphism of a vector space and if $A^{-1}$ exists, then $Ae^BA^{-1}=e^{ABA^{-1}}$. We write it with $A=s$ and $B=\ad X$:
\[
  \sigma_s\cdot e^{\ad X}=se^{\ad X}s^{-1}=e^{s\ad Xs^{-1}}=e^{\ad(s\cdot X)},
\]
sot that
\begin{equation}\label{eq:sigma_aut_s}
  \sigma_s\cdot e^{\ad X}=e^{\ad(s X)}.
\end{equation}

Ont the other hand, we know that $\IntA$ is connected, so it is generated by elements of the form $e^{\ad X}$ for $X\in\lA$. Then $\IntA$ is a normal subgroup of $\AutA$; the automorphism $s$ of $\lA$ induces the isomorphism $g\to sgs^{-1}$ in $\IntA$ because of equation \eqref{eq:sigma_aut_s}.
\end{proof}

More generally, if $s$ is an isomorphism from a Lie algebra $\lA$ to a Lie algebra $\lB$, then the map $g\to sgs^{-1}$ is an isomorphism between $\AutA$ and $\AutB$ which sends $\IntA$ to $\IntB$. Indeed, consider an isomorphism $\dpt{s}{\lA}{\lB}$ and $g\in\AutA$. If $g\in\IntA$, we have to see that $sgs^{-1}\in\IntB$. By definition, $\IntA$ is the analytic subgroup of $\GL(\lA)$ which has $\ad(\lA)$ as Lie algebra. We have $g=e^{\ad A}$, then $sgs^{-1}=e^{\ad(sA)}$ which lies well in $\IntB$.


