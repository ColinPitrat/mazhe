% This is part of (almost) Everything I know in mathematics
% Copyright (c) 2013-2014
%   Laurent Claessens
% See the file fdl-1.3.txt for copying conditions.



%+++++++++++++++++++++++++++++++++++++++++++++++++++++++++++++++++++++++++++++++++++++++++++++++++++++++++++++++++++++++++++
\section{Non commutative vector bundle}


%---------------------------------------------------------------------------------------------------------------------------
					\subsection{The category of complex vector bundles}
%---------------------------------------------------------------------------------------------------------------------------

In order to define the concept of noncommutative vector bundle, we have to get a more precise comprehension of a commutative vector bundle. Let $M$ be a compact manifold and $\catC$ be the category of complex vector bundle over $M$ in which the arrows are the vector bundle isomorphisms, i.e. maps of the form $\tau\colon E\to E'$ such that $\pi'\circ\tau=\pi$ and $\tau_x\colon E_x\to E'_x$ being a linear map with obvious notations. One usually denote $\Gamma(E)= C^{\infty}(M,E)$.

If $\tau\colon E\to E'$ is an isomorphism of vector bundle, we denote by $\Gamma\tau$ the map
\begin{equation}
\begin{aligned}
 \Gamma\tau\colon \Gamma(E)&\to \Gamma(E) \\ 
   (\Gamma\tau)s&\mapsto =\tau\circ s. 
\end{aligned}
\end{equation}
Let $\cA$ be a (not yet determined) dense subalgebra of $ C^{\infty}(M)$.
For every $a\in\cA$ and $x\in M$, the map $\Gamma\tau$ satisfies $(\Gamma\tau)(sa)=(\Gamma\tau)(s)a$, so that $\Gamma\tau$ is a morphism of right $\cA$-module.

Let $\catC$ be the category of vector bundle over $M$ in which the arrows are isomorphisms of vector bundle, and $\catD$, the category of right $\cA$-modules. We define the functor $\Gamma$ from $\catC$ to $\catD$ which to each vector bundle make correspond the right $\cA$-module of its sections. An arrow $\tau\colon E\to E'$ is transformed into the arrow $(\Gamma\tau)(s)=\tau\circ s$.

That functor respects the tensor product in the sense that $\Gamma(E\otimes E')=\Gamma(E)\otimes_{\cA}\Gamma(E')$ where $\otimes_{\cA}$ denotes the tensor product with the relation $sa\otimes s'=s\otimes as'$ for every $s\in\Gamma(E)$, $s'\in\Gamma(E')$ and $a\in\cA$.

\begin{proposition}
Any $\cA$-linear map from $\Gamma(E)$ to $\Gamma(E')$ is of the form $\Gamma\tau$ for an unique map $\tau\colon E\to E'$.
\end{proposition}

\begin{proof}
No proof.
\end{proof}

Let us now study the image of the functor $\Gamma$: what are the $\cA$-modules of the form $\Gamma(E)$? First, remark that, for the trivial bundle $E=M\times\eC^r$, then the image is the free $\cA$-module $\Gamma(E)=\cA^r$. Let us now take a more general vector bundle $E$. Compactness of $M$ provide us partition of unity $\psi_i\colon  \mU_i\to \eR$ with $\psi_i>0$ and $\psi_1^2+\cdots+\psi_q^2=1$. For sake of notational simplicity, we denote $\mU_{ij}=\mU_i\cap\mU_j$.  We suppose without loss of generality that $E$ is trivial over each of the $\mU_i$. The transition functions $f_{ij}\colon \mU_{ij}\to \GL(r,\eC)$ are subject to the relations $f_{ik}f_{kj}=f_{ij}$ on $\mU_{ijk}$. Let us pose
\begin{equation}
 p_{ij}=
\begin{cases}
	\psi_if_{ij}\psi_j&\text{in }\mU_{ij}\\
	0&\text{outside}
\end{cases}
\end{equation}
One easily finds that
\[ 
  \sum_kp_{ij}p_{kj}=p_{ij},
\]
so that $p$ is an element of $M_{qr}(\cA)$ with the property that $p^2=p$. Now a section of $E$ is given by the local functions $s_j\colon \mU_j\to \eC^r$ given by $s_i=f_{ij}s_j$ on $\mU_{ij}$ and can be seen as a column
\begin{equation}
s=
\begin{pmatrix}
\psi_1s_1\\\vdots\\\psi_qs_q
\end{pmatrix}\in C^{\infty}(M)^{qr}.
\end{equation}
Let us compute $ps$:
\[ 
\begin{split}
(ps)_i	&=\sum_kp_{ik}s_k
	=\sum_kp_{ik}\psi_ks_k\\
	&=\sum_k\psi_if_{ik}\psi_k\psi_ks_k
	=\sum_k\psi_if_{ik}\psi_k^2f_{ki}s_i\\
	&=\psi_is_i\sum_k\psi_k^2
	=\psi_is_i=s_i,
\end{split}  
\]
so that $ps=s$. Thus we identify $\Gamma(E)$ with $p\cA^{qr}$ with $r$ being the dimension of the vector bundle and $q$, the number of open sets needed to have a partition of unity in the same time as a trivialization of $E$. The Serre-Swan\index{Serre-Swan theorem} theorem provides the result in the inverse sense.
\begin{theorem}[Serre-Swan]
Every right $\cA$-module of the form $p\cA^m$ with $p$, an idempotent element of $M_m(\cA)$ is of the form
\[ 
  \Gamma(E)= C^{\infty}\big( M(\cA),E \big)
\]
where the fibre over $\mu\in M(\cA)$ is the vector space $p\cA^m\otimes_{\cA}(\cA/\ker\mu)$ whose dimension is the trace of the matrix $\mu(p)\in M_m(\eC)$.
\end{theorem}
So $\Gamma$ is a functor between the category of complex vector bundle over $M$ and the category of finite projective $C^{\infty}(M)$-modules. The content of the Serre-Swan theorem is that that functor has an inverse, so that the two categories are equivalent.

That motivates the following definition
\begin{definition}
A \defe{noncommutative vector bundle}{noncommutative!vector bundle} is a right finite projective module over an algebra which is not specially commutative.
\end{definition}
Most of time we study modules overs algebras whose are dense subalgebra of a $C^*$-algebra.


\section{Hermitian structure and compatible connection}
%++++++++++++++++++++++++++++++++++++++++++++++++++++++

\subsection{Hermitian structures}
%---------------------------------

The motivation of the following definitions is the fact that when $\cA$ is an involutive algebra of functions on the manifold $M$, any vector bundle is characterised by its  $\cA$-module $\modE$ of sections which have the same regularity as the functions in $\cA$. If $E$ is locally trivial and has finite dimension, $\modE$ is a direct sum of free modules $\cA^N$ and $\modE$ is a finite projective bundle on $\cA$.

The data of an inner product $\langle .,\,.\rangle_x$ on each fibre $E_x$ provides a sesquilinear form
\begin{equation}
\begin{aligned}
 \langle .,\,.\rangle\colon \modE\times\modE&\to \cA \\ 
\langle \xi,\,\eta\rangle(x)&=  \langle \xi(x),\,\eta(x)\rangle_x.
\end{aligned}
\end{equation}
This product fulfils the following relations for every $\xi$, $\eta\in\modE$ and for all $a$, $b\in\cA$: 
\begin{enumerate}
\item $\langle \xi a,\,\eta b\rangle=a^*\langle \xi,\,\eta\rangle b$,
\item $\langle \xi,\,\xi\rangle\geq 0$,
\item $\modE$ is self-dual for the product.
\end{enumerate}
The first comes from the following simple computation:
\[ 
  \langle \xi a,\,\eta b\rangle(x)=\langle \xi(x) a(x),\,\eta(x)b(x)\rangle_x=a(x)^*b(x)\langle \xi(x),\,\eta(x)\rangle_x=\big((a^*b)\langle \xi,\,\eta\rangle\big)(x)
\]
In the present case, the algebra $\cA$ is commutative and $a^*b\langle \xi,\,\eta\rangle=a^*\langle \xi,\,\eta\rangle b$, but since we will soon pass to a more general case, the right hand side form is more natural.

\begin{definition}
Let $\cA$ be an unital involutive algebra and $\modE$, a finite projective module over $\cA$. An \defe{hermitian structure}{hermitian!structure on a module} on $\modE$ is a sesquilinear map $ \langle .,\,.\rangle\colon \modE\times\modE\to \cA$ which fulfils the three conditions above.
\end{definition}
Notice that an hermitian structure is not a map $\modE\times\modE\to \eC$ because, in the noncommutative picture, elements of $\modE$ correspond to sections of $E$, not to \emph{elements} of $E$. But on the sections we have $\langle s, s'\rangle (x)=\langle s(x),s'(x)\rangle \in \eC$, so that the pairing of two sections is a functions, which corresponds to an element of $\cA$.

A \defe{pre-$C^*$-module}{pre-$C^*$-module} structure on a dense subalgebra $\cA$ of the $C^*$-algebra $A$ is a right module $\modE$ over $\cA$ endowed with an Hermitian structure. If $\| . \|_A$ is the $C^*$-norm of $A$, we can put the norm 
\begin{equation}
	\| s \|_{\modE}=\sqrt{\| (s|s) \|_A}
\end{equation}
on $\modE$. The completion of $\modE$ for that norm gives rise to a $C^*$-module.

The typical example of that is a vector bundle $E\to M$ endowed with an hermitian structure, i.e. a product $\langle ., .\rangle_x\colon E_x\times E_x\to \eC$ on each fibre. One consider the $ C^{\infty}(M)$-module $\modE=\Gamma(E,M)$ of sections and the product
\begin{equation}
\begin{aligned}
 \langle ., .\rangle\colon \modE\times\modE&\to C^{\infty}(M)  \\ 
   \langle \eta_1, \eta_2\rangle (x)&= \langle \eta_1(c), \eta_2(c)\rangle_x. 
\end{aligned}
\end{equation}
One immediately checks that it fulfils the three conditions 
\begin{enumerate}
\item $\langle \eta_1a, \eta_2b\rangle= a^*\langle \eta_1, \eta_2\rangle b$,
\item $\langle \eta_1, \eta_2\rangle^*=\langle \eta_2, \eta_1\rangle $,
\item $\langle \eta, \eta\rangle \geq 0$ and $\langle \eta, \eta\rangle =0$ if and only if $\eta=0$.
\end{enumerate}

The \defe{dual module}{dual!of a module} of the right module $\modE$ is the module
\[ 
  \modE'=\{ \phi\colon \modE\to \cA\tq \phi(\eta a)=\phi(\eta)a \}.
\]
The right module structure being defined by $\phi a=a^*\phi\in\modE'$. One says that the Hermitian structure is \defe{nondegenerate}{nondegenerate!hermitian structure} when the map $\eta\mapsto\langle \eta, .\rangle $ is an isomorphism between $\modE$ and $\modE'$. We denote by $U_N(\cA)=U(\cA^N)$ the space of unitary endomorphisms of the free module $\cA^N$.

The free $\cA$-module $\cA^N$ accepts the structure of pre-$C^*$-module by
\begin{equation}	\label{EqHermCanN}
(r|s) = \sum_{j=1}^N r^*_js_j.
\end{equation}

\begin{lemma}		\label{Lemqqstarherm}
If $q\in\eM_N(\cA)$, the product \eqref{EqHermCanN} fulfills 
\[ 
 (q^*\xi|\eta)=(\xi|q\eta)
\]
where $q^*$ is the matrix defined by $(q^*)_{ij}=(q_{ji})^*$ where, in the right hand side, the star denotes the involution on $\cA$.
\end{lemma}

\begin{proof}
The proof is a simple computation:
\begin{align*}
(p^*\xi|\eta)	=\sum_j(p^*\xi)_j^*
		=\sum_{jk}\big( (p^*)_{jk}\xi_k \big)^*\eta_j
		=\sum_{jk}\xi_k^*(p^*)_{jk}^*\eta_j
		=\sum_{jk}\xi^*_k p_{kj}\eta_j
		=\sum_k\xi_k^*(p\eta)_k
		=(\xi|p\eta).
\end{align*}
\end{proof}

We suppose that the $\cA$-module $\modE$ can be written as $\modE=p\cA^N$ for a certain $N\in \eN$ and an idempotent element\footnote{Notice that $p\cA^N\neq\cA^N$ in general because $a\cA\neq\cA$ in general. This situation is different of a group situation where $gG=G$ for any element $g\in G$.} $p\in M_N(\cA)$. Since $\modE$ is self-dual for the hermitian structure, the data of $\langle \xi,\,\eta\rangle$ for all $\eta\in\modE$ defines $\xi$. So we can define an involution $T\mapsto T^*$ on $\End_{\cA}(\modE)$ by the formula
\[ 
  \langle T^*\xi,\,\eta\rangle=\langle \xi,\,T\eta\rangle
\]
for all $\xi$,$\eta\in\modE$.

What we want to do is to put the restriction of the canonical structure \eqref{EqHermCanN} to $p\cA^N$ (with $p^2=p$) as Hermitian structure on $\modE$. 

\begin{proposition}
The canonical structure \eqref{EqHermCanN} restricts to $p\cA^N$ only if $p=p^*$.
\end{proposition}

\begin{proof}
We pose $\modE=p\cA^N$ and 
\[ 
  \modE^{\perp}=\{ u\in\cA^N\tq (u|\eta)=0\text{ for all }\eta\in\modE \}.
\]
If $u\in\modE^{\perp}$, we have $(ua|\eta)=a^*(u|\eta)=0$, so that $\modE^{\perp}$ is a right $\cA$-module. Take $u\in\cA^N$ and $\eta\in\modE$, we have
\[ 
  \big( (1-p^*)u | \eta\big) =\big( u | (1-p)\eta\big) =0,
\]
so that $(1-p^*)\cA^N\subseteq\modE^{\perp}$. Let us now try to express $\modE^{\perp}$ as $q\cA^N$. For every $\xi$ and $\eta$ in $\cA^N$, we must have
\[ 
  0=( q\xi | p\eta) =( p^*q\xi | \eta),
\]
which proves that $p^*q=0$ because we suppose the hermitian structure to be nondegenerate. The image of $q$ is contained in the kernel of $p^*$, so that $q=r(1-p^*)$ for a certain operator $r\colon \ker(p^*)\to \ker(p^*)$. Since $(1-p^*)\cA^N\subseteq q\cA^N$, the operator $r$ must have an empty kenrel. So as set, $r(1-p^*)\cA^N=(1-p^*)\cA^N$. What we proved up to now is that
\[ 
  \modE^{\perp}=(p\cA^N)^{\perp}=(1-p^*)\cA^N.
\]
But the free module $\cA^N$ decomposes as $\cA^N=p\cA^N\oplus (1-p)\cA^N$. The canonical hermitian structire on $\cA^N$ restricts to $p\cA^N$ only if that decomposition is orthogonal with respect to the Hermitian product, so the condition is $(1-p)\cA^N=(p\cA^N)^{\perp}=(1-p^*)\cA^N$, or $p=p^*$.
\end{proof}

From now we suppose the module $\modE=p\cA^N$ to fullfil the condition $p=p^2=p^*$.

%---------------------------------------------------------------------------------------------------------------------------
					\subsection{Morita equivalence}
%---------------------------------------------------------------------------------------------------------------------------

If $\modE$ is a right $\cA$-module , we define its \defe{conjugate module}{conjugate!module} as
\begin{equation}
\bar\modE=\{ \bar s\tq s\in\modE \}
\end{equation}
with a left $\cA$-module structure $(a\bar s)=\overline{  sa^* }$. When $\modE=p\cA^N$, we have $\bar\modE=\overline{ \cA^N }p$ and its elements have to be seen as row-vectors. 

Let $\modE$ be a projective $\cA$-module of finite type. An operator on $\modE$ is a \defe{ket-bra}{ket-bra operator} if it is of the form
\begin{equation}
\begin{aligned}
 \ketbra{\eta_1}{\eta_2} \colon \modE&\to \modE \\ 
   \xi&\mapsto \eta_1( \eta_2 | \xi)  
\end{aligned}
\end{equation}
Since $\eta_1( \eta_2 | \xi a) =\eta_1( \eta_2 | \xi) a$ for every $\xi$, $\eta_1$, $\eta_2\in\modE$ and $a\in\cA$, the ket-bra commutes with the right action of $\cA$ on $\modE$, so that one can write 
\[ 
  \ketbra{\eta_1}{\eta_2} \xi a
\]
without ambiguities. The composition of ket-bra is still a ket-bra:
\begin{equation}
\Big( \ketbra{\xi_1}{\xi_2}  \Big)\big( \ketbra{\eta_1}{\eta_2}  \big)\sigma = \ketbra{\xi_1}{( \eta_1 | \xi_2) \eta_2} \sigma.
\end{equation}
The algebra of finite sums of ket-bras is an algebra that we denote by $\End_{\cA}(\modE)$\nomenclature[D]{$\End_{\cA}(\modE)$}{The algebra of ket-bras on the module $\cA$}.

If $V$ is a vector space, it is a known fact that $\End(V)=V\otimes V^*$. The same kind of identification holds for modules.
\begin{proposition}
The map
\begin{equation}
\begin{aligned}
 \End_{\cA}(\modE)&\to \modE\otimes_{\cA}\bar\modE  \\ 
   \ketbra{\xi_1}{\xi_2} &\mapsto \xi_1\otimes\bar\xi_2 
\end{aligned}
\end{equation}
is an isomorphism.
\end{proposition}

\begin{proof}
Any element of $\modE\otimes_{\cA}\bar\modE$ can be written under the form $(\xi_1 a)\otimes_{\cA}\bar\xi_2$, with eventually $a=1$. We are going to prove that if 
\begin{equation}		\label{Eqhypunxiaotibarxi}
(\xi_1 a)\otimes_{\cA}\bar\xi_2=\eta_1\otimes\bar\eta_2,
\end{equation}
 then $\ketbra{\xi_1 a}{\xi_2} =\ketbra{\eta_1}{\eta_2} $. If \eqref{Eqhypunxiaotibarxi} is true, then $\eta_1=\xi_1$ and $\bar\eta_2=a\bar\xi_2$, or $\xi_1a=\eta_1$ and $\xi_2=\eta_2$. The conclusion is immediate in the second case. In the first case, we have
\[  
  \ketbra{\xi_1 a}{\xi_2} \sigma=\xi_1 a( \xi_1 | \sigma) =\xi_1( \xi_2 a^* | \sigma) =\eta_1( \eta_2 | \sigma) =\ketbra{\eta_1}{\eta_2} \sigma.
\]
\end{proof}

Now we pose $\cB=\End_{\cA}(\modE)$, and $\modE$ becomes a left $\cB$-module, so that one says that $\modE$ is a $\cB$-$\cA$-bimodule. We have $\cB=p\eM_N(\cA)p$, and we denote an element of $\cB$ by $\varphi_A=pAp$. We have $(\varphi_A)^*=\varphi_{A^*}=pA^*p$ because we assume $p=p^*$. As particular case of lemme \ref{Lemqqstarherm}, we have $( \xi | \varphi_A\eta) =( \varphi_A^*\xi | \eta) $.

The algebra $\cB=\End_{\cA}(\modE)$ acts at right on $\bar\modE$ by
\begin{equation}
	\bar\eta\varphi_A=\overline{ \varphi_A^*\eta }.
\end{equation}
We are now able to consider the tensor product $\bar\modE\otimes_{\cB}\modE$ in which
\begin{equation}		\label{EqDeftensoeurB}
	\bar\eta\otimes_{\cB}(\varphi_A\xi)=\overline{ \varphi_A^*\eta }\otimes_{\cB}\eta.
\end{equation}
\begin{proposition}
The map
\begin{equation}
\begin{aligned}
 \psi\colon \bar\modE\otimes_{\cB}\modE&\to \cA \\ 
   \bar\eta\otimes_{\cB}\xi&\mapsto ( \eta | \xi)  
\end{aligned}
\end{equation}
is an isomorphism of $\cA$-bimodule.
\end{proposition}

\begin{proof}
The map $\psi$ is well defined because of relation \eqref{EqDeftensoeurB}. For the surjectivity, we use the assumption of non degenerate Hermitian structure. We have to prove that $\{ ( \xi | \eta) \tq \xi\, \eta\in\modE  \}=\cA$. Pick any $\xi_0\in\modE$ and define a map $\phi\colon \modE\to \cA$ such that $\phi(\xi_0)=a$ and $\phi(\eta b)=\phi(\eta)b$. In this case, there exists a $\eta\in\modE$ such that $\phi(\xi)=( \eta | \xi) $. For this $\eta$ we have $( \eta | \xi_0)=a$.

\begin{probleme}
	I'm not able to prove injectivity.
\end{probleme}

\end{proof}

\subsection{Differential over a module}
%---------------------------------------

We have now to extend the definition of $\delta$ from $\cA$ to $\modE=p\cA^N$. In that purpose we begin to extend $\lambda$ and $p$ of page \pageref{PgdeflambdaMod} as
\begin{equation}
\begin{aligned}
 p\colon \eC^N\otimes_{\eC}\Omega^p\cA&\to \modE\otimes_{\cA}\Omega^p\cA \\ 
   \sum_{i}e_i\otimes_{\eC}\omega_i&\mapsto \sum_ip(f_i)\otimes_{\cA}\omega_i 
\end{aligned}
\end{equation}
and
\begin{equation}
\begin{aligned}
 \lambda\colon \modE\otimes_{\cA}\Omega^p\cA&\to \eC^N\otimes_{\eC}\Omega^p\cA \\ 
   \lambda(\xi\otimes_{\cA}\omega)&=\sum_i e_i\otimes_{\eC}\xi^i\omega
\end{aligned}
\end{equation}
if $\lambda(\xi)=\sum_ie_i\otimes_{\eC}\xi^i$. Now we define $p\delta\colon\modE\otimes_{\cA}\Omega^p\cA\to \modE\otimes_{\cA}\Omega^{p+1}\cA$ by
\[ 
  p\delta=p\circ(\mtu\otimes\delta)\circ\lambda.
\]
In order to see what is going on, let us apply that on a general element of $\modE\otimes_{\cA}\Omega^p\cA$:
\begin{align*}
\big( p\circ(\mtu\otimes\delta)\circ\lambda \big)p\big( \sum_ie_i\otimes_{\eC}\omega_i \big)	&=p\circ(\mtu\otimes\delta)\big( \sum_ie_i\otimes_{\eC}\omega_i \big)\\
			&=p\big( \sum_ie_i\otimes_{\eC}\delta\omega_i \big)\\
			&=\sum_ip(f_i)\otimes_{\cA}\delta\omega_i.
\end{align*}
The definition of $pd\xi$ is in the same way: if $\xi=p\big( \sum_ie_i\otimes_{\eC}\omega_i \big)$, we pose
\begin{equation}
pd\xi=\sum_ip(f_i)\otimes_{\cA}d\omega_i.
\end{equation}



\subsection{Connections over Hermitian modules}
%----------------------------------------------

\begin{definition}
Let $\modE$ be a finite projective hermitian module on $\cA$. A \defe{connection}{connection!on an hermitian module} is a linear map $\nabla\colon \modE\to \modE\otimes_{\cA}\Omega^1_D(\cA)$ such that
\begin{equation}
\nabla(\xi a)=(\nabla\xi)a+\xi\otimes da
\end{equation}
for all $\xi\in\modE$ and $a\in\cA$. Recall that $da$ means $[D,a]$ in the operator representation. The connection is said to be \defe{compatible with the hermitian structure}{compatible!connection} if
\begin{equation}
 \langle \xi,\,\nabla\eta\rangle-\langle \nabla\xi,\,\eta\rangle=d\langle \xi,\,\eta\rangle.
\end{equation}
We denote by $\mC(\modE)$\nomenclature[D]{$\mC(\modE)$}{The space of compatible connections on the module $\modE$} the space of compatible connections.

\end{definition}
The latter deserve some comments. First, $\langle \xi,\,\eta\rangle\in\cA$, so that $d\langle \xi,\,\eta\rangle$ makes sense. Second, we have to define $\langle .,\,.\rangle$ when there are a component in $\Omega^1(\cA)$. We naturally define
\[ 
  \langle \xi,\,\varphi\otimes_{\cA}\omega\rangle=\underbrace{\langle \xi,\, \varphi\rangle}_{\in\cA}\omega.
\]
which is a well defined element in $\Omega^1_D(\cA)$. There are no kind of left Leibnitz rule in the definition of a connection. There always exists a compatible connection. For example the \defe{Grassmann connection}{grassmann connection} defined by
\begin{equation}
\nabla_0\xi=p\eta
\end{equation}
if $\eta_j=d\xi_j$. That connection is compatible because
\begin{equation}
\begin{split}
  d\langle \eta,\,\xi\rangle&=d\big( \sum_{i=1}^{N}\eta_i^*\xi_i \big)=\sum d\eta_i^*\xi_i+\sum \eta_i^*d\xi_i
		=\sum (-)(d\eta_i)^*\xi_i+\sum \eta_i^*d\xi_i\\
		&=-\langle d\eta,\,p\xi\rangle+\langle p\eta,\,d\xi\rangle=-\langle pd\eta,\,\xi\rangle+\langle  \eta,\,pd\xi\rangle
	=-\langle \nabla_0\eta,\,\xi\rangle+\langle \eta,\,\nabla_0\xi\rangle
\end{split}
\end{equation}
where we used the fact that $p=p^*$.


We denote by $C(\modE)$\nomenclature[D]{$C(\modE)$}{the space of compatible connections on the hermitian module $\modE$} the space of compatibles connections on $\modE$, and the \defe{unitary group}{unitary!group!of a module} of $\modE$ is 
\[ 
  \gU(\modE)=\{ u\in\End_{\cA}(\modE)\tq uu^*=u^*u=1 \}.
\]
This group acts by conjugation on $C(\modE)$ following the formula $\gamma_u(\nabla)=u\nabla u^*$, or more explicitly:
\begin{equation}
(\gamma_u\nabla)\xi=u\nabla(u^*\xi).
\end{equation}
One easily checks that $\gamma_u(\nabla)$ is a compatible connection when $\nabla$ is a compatible connection:
\[ 
\begin{split}
(\gamma_u\nabla)(\xi a)&=u\nabla(u^*\xi a)=u\big( \nabla(u^*\xi)a+u^*\xi\otimes da \big)\\
		&=u\nabla(u^*\xi)a+\xi\otimes da=\big( \gamma_u(\nabla)\xi \big)a+\xi\otimes da,
\end{split}  
\]
and, using the sesquilinearity, 
\[ 
\begin{split}
   \langle \xi,\,(\gamma_u\nabla)\eta\rangle-\langle (\gamma_u\nabla)\xi,\,\eta\rangle&=\langle \xi,\,u\nabla(u^*\eta)\rangle-\langle u\nabla(u^*\xi),\,\eta\rangle\\
		&=\langle u^*\xi,\,\nabla(u^*\eta)\rangle-\langle \nabla(u^*\xi),\,u^*\eta\rangle,
\end{split}  
\]
which gives the compatibility using the compatibility of  $\nabla$.

\subsection{Universal compatible connection}
%-------------------------------------------

An \defe{universal compatible connection}{universal!compatible connection} on a finite projective Hermitian module $\modE$ is a map $\nabla\colon \modE\to \modE\otimes_{\cA}\Omega^1(\cA)$ such that
\begin{enumerate}
\item $\nabla(\xi a)=(\nabla \xi)a+\xi\otimes da$,
\item $\langle \xi, \nabla\eta\rangle -\langle \nabla\xi, \eta\rangle =d\langle \xi, \eta\rangle$
\end{enumerate}
for every $\xi$, $\eta\in\modE$ and $a\in\cA$. A map which only fulfils the first condition is an \defe{universal connection}{universal!connection}.

 We denote by $\mCC(\modE)$\nomenclature[D]{$\mCC(\modE)$}{space of universal compatible connections} the space of universal compatible connections. An universal compatible connection extends in an unique way to a map 
\begin{equation}
\begin{aligned}
\tilde\nabla\colon \modE\otimes_{\cA}\Omega^p\cA&\to \modE\otimes_{\cA}\Omega^{p+1}\cA\\
\xi\otimes\omega_0&\mapsto (\nabla\xi)\omega_0+\xi\otimes d\omega_0
\end{aligned}
\end{equation}
with $\xi\in\modE$ and $\omega\in\Omega^*_D$. If $\nabla\xi=\eta\otimes\omega_0$, the expression $(\nabla\xi)\omega$ means $(\nabla\xi)\omega=\eta\otimes(\omega_0\omega)$. The operation $\tilde\nabla$ fulfils the Leibnitz rule
\[ 
  \tilde\nabla(\eta\omega)=(\tilde\nabla\eta)+(-1)^{\eta}\eta d\omega
\]
if $\eta$ is homogeneous and $\omega\in\Omega^*_D$. Indeed the fact that $\eta$ is homogeneous means that it reads $\xi\otimes\omega_0$ with $\omega_0\in\Omega_D^k$     for a certain $k$. Indeed,
\[ 
\begin{split}
 \tilde\nabla(\eta\omega)&=\tilde\nabla(\xi\otimes\omega_0\omega)=(\nabla\xi)\omega_0\omega+\xi\otimes\big( (d\omega_0)\omega+(-1)^{\omega_0}\omega_0 d\omega \big)\\
		&=\big( (\nabla\xi)\omega_0+(d\omega_0) \big)\omega+(-1)^{\omega_0}\xi\otimes\omega_0 d\omega
		=\tilde\nabla(\xi\otimes\omega_0)\omega+(-1)^{\eta}\xi\otimes\omega_0 d\omega\\
		&=(\tilde\nabla\eta)\omega+(-1)^{\eta}\eta d\omega.
\end{split}  
\]
Most of time, we will make no difference between $\nabla$ and $\tilde\nabla$.

\begin{lemma}
Every universal connection $\nabla\in\mCC(\modE)$ reads
\begin{equation}\label{EqDefConnpdapha}
\nabla = p\delta+\alpha
\end{equation}
for a certain $\alpha\in p\eM_{N\times N}(\cA)p\otimes_{\cA}\Omega^1\cA$.
\end{lemma}

\begin{proof}
We can express the module as $\modE=p\cA^N$. If $\nabla_1$ and $\nabla_2$ are two universal connections, we have
\[ 
(\nabla_1-\nabla_2)(\eta a)=(\nabla_1\eta)a+\eta(\delta a)-(\nabla_2\eta)a-\eta(\delta a)=\big( (\nabla_1-\nabla_2)\eta \big)a,
\]
so that $\nabla_1-\nabla_2\in\End_{\cA}(\modE)\otimes_{\cA}\Omega^1\cA\simeq p\eM_{N\times N}(\cA)p\otimes_{\cA}\Omega^1\cA$. Taking in particular the Grassmannian connection $\nabla_0$ as $\nabla_1$, we find that for any universal connection $\nabla$ fulfils $\nabla-p\delta=\alpha$ with $\alpha\in\eM_{N\times N}(\cA)\otimes\Omega^1(\cA)$ satisfying $\alpha=p\alpha=\alpha p=p\alpha p$. 
\end{proof}
When $\nabla$ is given, the corresponding $\alpha$ is the \defe{gauge potential}{gauge!potential!(universal connection)} of $\nabla$. 


\begin{proposition}\label{PropmCCmCsurjun}
The representation $\pi\colon \Omega(\cA)\to \oB(\hH)$ given in equation \eqref{EqDEfpirep} extends to 
\begin{equation}
\begin{aligned}
 (\mtu\otimes\pi)\colon \mCC(\modE)&\to \mC(\modE) \\ 
   \nabla&\mapsto (\cun\otimes\pi)\circ\nabla 
\end{aligned}
\end{equation}
which is surjective.
\end{proposition}
As a particular case of the proposition, let us prove that the Grassmann connection belongs to $\pi\big( \mCC(\modE) \big)$. When one defines the Grassmann connection as $(\nabla_0\xi)=p\eta$, the right hand side has to be understood as $p\eta\otimes_{\cA}1$ with the $1$ of $\Omega^1_D(\cA)$. So the Grassmann connection is obtained by applying $\pi$ to the connection $\nabla_0'(\xi)=p\eta\otimes_{\cA}1$ with the $1$ of $\Omega^1(\cA)$.

In particular existence of that connection shows that every projective module accepts a connection.
\begin{probleme}
	The following proof is finished?
\end{probleme}

\begin{proof}

By construction, the map 
\[ 
  (\cun\otimes\pi)  \colon \End_{\cA}(\modE)\otimes\Omega^1(\cA)\to \End_{\cA}(\modE)\otimes_{\cA}\Omega_D^1(\cA)
\]
is surjective. 


\end{proof}

\begin{proposition}
A right module has a connection if an only if it is projective.
\end{proposition}

\begin{proof}
See proposition 7.4 in \cite{Landi}.
\end{proof}


\subsection{Curvature}
%---------------------

Now we define the \defe{curvature}{curvature} of $\nabla$ by $\theta(\nabla)=\tilde\nabla^2$.

\begin{proposition}
This curvature is an endomorphism of $\tilde\modE$ for its structure of right $\Omega^*_D(\cA)$-module.
\end{proposition}

\begin{proof}
If we consider $\xi\otimes\omega_0\in\tilde\modE$ and $\omega\in\Omega^*_D$ with $\nabla\xi=\eta\otimes\omega'$, it is just a computation to prove that
\[ 
  \tilde\nabla^2(\xi\otimes\omega_0\omega)=\big((\nabla\eta)\omega'\omega_0+\eta\otimes d\omega'\omega_0\big)\omega,
\]
and then to see that $\theta\big( (\xi\otimes\omega_0)\omega \big)=\theta(\xi\otimes\omega_0)\omega$.
\end{proof}

From the formula 
\begin{equation}
\tilde\nabla^2(\xi\otimes\omega_0)=(\nabla\eta)\omega'\omega_0+\eta\otimes d\omega'\omega_0,
\end{equation}
we see that $\theta=\tilde\nabla^2$ is completely determined by its restriction to $\modE$.

Let us compute $\nabla^2(\xi\otimes_{\cA}\omega_0)$ with $\omega_0\in\Omega^p\cA$ and $\xi\in\modE$. If $\nabla\xi=\eta\otimes_{\cA}\omega'$,
\begin{align*}
\nabla^2(\xi\otimes_{\cA}\omega_0)&=\nabla\big( (\nabla\xi)\omega_0+\xi\delta\omega_0 \big)\\
		& = (\nabla\eta)\omega'\omega_0+(-1)^{\eta}\eta\otimes_{\cA}\delta(\omega'\omega_0)+(\nabla\xi)\delta\omega_0+\xi\delta^{2}\omega_0\\
		& = (\nabla\eta)\omega'\omega_0+(-1)^{\eta}\eta(\delta\omega')\omega_0+(-1)^{\eta}\eta(-1)^{\omega'}\delta\omega_0+(\eta\otimes_{\cA}\omega')\delta\omega_0\\
		& = \big( (\nabla\eta)\omega'+\eta\delta\omega' \big)\omega+(\eta\otimes_{\cA}\omega'-\eta\otimes\omega').
\end{align*}
We finally have
\begin{equation}
\nabla^2(\xi\otimes_{\cA}\omega_0)=\big( (\nabla\eta)\omega'+\eta\otimes_{\cA}\delta\omega' \big)\omega_0
\end{equation}
if $\nabla\xi=\eta\otimes\omega'$. That shows that 
\begin{equation}	\label{EqThetaCourbomega}
\theta\big( (\xi\otimes_{\cA}\omega_0)\omega \big)=\theta(\xi\otimes_{\cA}\omega_0)\omega.
\end{equation}

In terms of the gauge potential $\alpha$, the curvature reads
\begin{equation}		\label{EqConnPotalpha}
 \theta(\xi)=(p\delta p\delta p+p(\delta\alpha)+\alpha^2)\xi
\end{equation}
because $\alpha=p\alpha=\alpha p=p\alpha p$ and $p\xi=\xi$.


\subsection{Transformation law under the unitary group}
%------------------------------------------------------

The unitary group (subsection \ref{SubsecUnitGroup}) acts on an element $\nabla\in\mCC(\modE)$ by
\begin{equation}
(u\cdot\nabla)(\xi\otimes_{\cA}\omega_0)=u\nabla\big( u^*(\xi\otimes_{\cA}\omega_0) \big),
\end{equation}
or in a more compact way $u\cdot\nabla=\nabla^u=u\nabla u^*$. We want to know how does the curvature change under such a transformation of the connection: we want to know $\theta(\nabla^u)$ in terms of $\nabla$ and $u$. We immediately have
\[ 
  \theta^u(\Sigma)=u\nabla\Big( u^*\big( u\nabla u^*(\xi\otimes\omega_0) \big) \Big)=u\nabla u^*u\nabla\big( u^*(\xi\otimes_{\cA}\omega_0) \big)=u\theta u^*(\Sigma)
\]
with $\Sigma\in\modE\otimes_{\cA}\Omega\cA$. So we have the covariant transformation law 
\begin{equation}
\theta^u=u\theta u^*. 
\end{equation}
Any connection can be written under the form \eqref{EqDefConnpdapha}: $\nabla=p\delta+\alpha$. It is a simple computation to find $\alpha^u$ defined by $\nabla^u=p\delta+\alpha^u$; using the notation $\Sigma=\xi\otimes_{\cA}\omega_0$, we have
\begin{align*}
\nabla^u(\xi\otimes_{\cA}\omega_0) & = u(p\delta+\alpha)u^*(\xi\otimes_{\cA}\omega_0)\\
		& = up\delta(u^*\Sigma)+u\alpha(u^*\sigma)\\
		& = up(\delta u^*)\Sigma+upu^*\delta\Sigma+u\alpha u^*\Sigma\\
		& = (p\delta+pu\delta u^*+u\alpha u^*)\Sigma,
\end{align*}
so the transformation law of the gauge potential reads
\begin{equation}
\alpha^u=pu\delta u^*+u\alpha u^*.
\end{equation}

\subsection{Connection on bimodule} 
%----------------------------------

Let $\modE$ be a left and right-projective $\cA$-bimodule with a right connection $\nabla\colon \modE\to \modE\otimes_{\cA}\Omega^1\cA$. If $\sigma\colon \Omega^1\otimes_{\cA}\modE\to \modE\otimes_{\cA}\Omega^1\cA$ is a bimodule isomorphism, we say that the couple $(\nabla,\sigma)$ is \defe{compatible}{compatible!isomorphism and connection} if we have the Leibnitz rule
\begin{equation}
\nabla(a\xi)=(\nabla\xi)a+\sigma(\delta a\otimes_{\cA}\xi).
\end{equation}

\subsection{Connection on dual module}
%-----------------------------------------

The pairing $(.,.)\colon \modE'\times\modE\to \cA$ defined by $(\phi,\xi)=\phi(\xi)$ can be extended by two ways:
\begin{enumerate}
\item 
 \begin{equation}
\begin{aligned}
 (.,.)\colon \big( \modE'\otimes_{\cA}\Omega\cA \big)\times\modE&\to \Omega\cA \\ 
   (\phi\otimes_{\cA}\omega,\xi)&=\omega^*(\phi,\xi)
\end{aligned}
\end{equation}
\item
\begin{equation}
\begin{aligned}
 (.,.)\colon \modE'\times\big( \modE\otimes_{\cA}\Omega\cA \big)&\to \Omega\cA \\ 
   (\phi,\xi\otimes_{\cA}\omega)&=(\phi,\xi)\omega. 
\end{aligned}
\end{equation}
\end{enumerate}
When one has a connection over $\modE$, one defines the \defe{dual connection}{dual!connection} $\nabla'\colon \modE'\to \modE'\otimes_{\cA}\Omega^1\cA$ by
\begin{equation}
(\nabla'\phi,\xi)=(\phi,\nabla\xi)-\delta(\phi,\xi).
\end{equation}
The dual connection satisfies the Leibnitz rule
\begin{equation}
\nabla'(\phi a)=\phi\otimes_{\cA}\delta a+(\nabla'\phi)a.
\end{equation}
Indeed,
\begin{align*}
\big( \nabla'(\phi a),\xi \big)&=(\phi a,\nabla\xi)-\delta\underbrace{(\phi a,\xi)}_{=a^*(\phi,\xi)}\\
				&=(\phi a,\nabla\xi)+(\delta a)^*(\phi,\xi)-a^*\big( (\phi,\nabla\xi)-(\nabla'\phi,\xi) \big)\\
				&=(\delta a)^*(\phi,\xi)+a^*(\nabla'\phi,\xi)\\
				&=(\phi\otimes_{\cA}\delta a,\xi)+\big( (\nabla'\phi)a,\xi \big).
\end{align*}

\subsection{Connections and Connes' calculus}
%++++++++++++++++++++++*++++++++++++++++++++++

Let $\modE$ be a projective $\cA$-module and let us consider the Connes' calculus $(\Omega_D\cA,d)$. In this framework, a \defe{connection}{connection!on Connes' calculus} is a $\eC$-linear map $\nabla\colon \modE\otimes_{\cA}\Omega_D^p\cA\to \modE\otimes_{\cA}\Omega^{p+1}_D\cA$ such that
\begin{equation}
\nabla(\Sigma\omega)=(\nabla\Sigma)\omega+(-1)^p\Sigma\,d\omega
\end{equation}
for every $\Sigma\in\modE\otimes_{\cA}\Omega^p_D\cA$ and $\omega\in\Omega_D\cA$. The composition $\nabla^2=\nabla\circ\nabla$ is $\Omega_D\cA$-linear and the restriction to $\modE$ is the \defe{curvature}{curvature!on Connes' calculus} denoted by $F(\nabla)$\nomenclature[D]{$F(\nabla)$}{Curvature of a connection}. An interesting property of the curvature is its $\eC$-linearity
\begin{equation}
  F(\xi a)=F(\xi)a,
\end{equation}
indeed
\begin{align*}
F(\xi a)	&=\nabla\big( (\nabla\xi)a+\xi da \big)=(\nabla^2\xi)a-(\nabla\xi)da+\nabla(\xi da)\\
		&=F(\xi)a-(\nabla\xi)da+(\nabla\xi)\,da+\xi\,d^2a.
\end{align*}
More generally we have
\begin{equation}
\nabla^2(\xi\otimes_{\cA}\rho)=F(\xi)\rho.
\end{equation}
The operator $\nabla^2$ can be seen as an element of $\End_{\cA}(\modE)\otimes_{\cA}\Omega^2_D\cA$ and we have the \defe{Bianchi identities}{Bianchi identities}
\begin{equation}
[\nabla,F]=0
\end{equation}
because $\nabla\circ\nabla^2=\nabla^2\circ\nabla$.

In the same way as equation \eqref{EqConnPotalpha}, we have
\begin{equation}
F=p(dA)+A^2+p\,dp\,p\,dp
\end{equation}
We say that the connection $\nabla\colon \modE\otimes_{\cA}\Omega^p_D\cA\to \modE\otimes_{\cA}\Omega^{p+1}_D\cA$ is \defe{compatible}{compatible!connection} with the hermitian structure if for all $\xi$, $\eta\in\modE$,
\begin{equation}
-\langle \nabla\eta, \xi\rangle +\langle \eta, \nabla\xi\rangle =d\langle \eta, \xi\rangle .
\end{equation}
That definition relies on the following two extensions of the hermitian structure:
\begin{equation}
\begin{aligned}
 \langle ., .\rangle \colon \big( \modE\otimes_{\cA}\Omega^1_D\cA \big)\times \modE&\to \Omega^1_D\cA \\ 
   \langle \xi\otimes_{\cA}\omega, \eta\rangle &=\omega^*\langle \xi, \eta\rangle  
\end{aligned}
\end{equation}
and
\begin{equation}
\begin{aligned}
  \langle ., .\rangle \colon \modE\times\big( \modE\otimes_{\cA}\Omega^1_D\cA \big)&\to \Omega^1_D\cA \\ 
   \langle \eta, \xi\otimes_{\cA}\omega\rangle &=\langle \eta, \xi\rangle \omega. 
\end{aligned}
\end{equation}

\begin{proposition}
The connection $\nabla=pd+A$ is compatible with the canonical hermitian structure $\langle \xi, \eta\rangle =\sum_{j=1}^{N}\xi_j\eta_j$ if and only if $A=A^*$.
\end{proposition}

\subsubsection{The simplest example in the world}	\label{ConnEequalAsime}
%////////////////////////////////////////////////

We will build the connection theory in the case of $\modE=\cA$ with the hermitian structure $\langle a, b\rangle =a^*b$. A connection on that module is a map $\nabla\colon \cA\to \cA\otimes_{\cA}\Omega^1_D\cA=\Omega^1_D\cA$ such that $\nabla(ab)=(\nabla a)b+a\,db$. 

A connection has the form $\nabla=d+A$ with $A\in\eM_{1\times 1}(\cA)\otimes_{\cA}\Omega^1_D\cA=\Omega^1_D\cA$ and the compatibility reads $A=A^*$. The curvature is given by $F=dA+A^2$.

In this particular case, one often write $V$ instead of $A$ and $\theta$ instead of $F$.

\subsubsection{Action of the gauge group}	\label{PgSimplestwordl}
%//////////////////////////////////////////

If $u\in U(\modE)$, we define the action of $u$ on the connection $\nabla$ by 
\begin{equation}
\nabla^u=u\nabla u^*
\end{equation}
in the sense that $\nabla^u(\xi)=(u\circ\nabla)(u^*\xi)$. In order to find the transformation law for the vector potential, we consider $\nabla=pd+A$ and we compute
\[ 
\begin{split}
\nabla^u\xi	=u\nabla(u^*\xi)
		=upd(u^*\xi)+uAu^*\xi
		=up(du^*)\xi+upu^*d\xi+uAu^*\xi
		=(updu^*+pd+uAu^*)\xi,
\end{split}  
\]
so that we have the transformation law
\begin{equation}		\label{EqTransAjauge}
A^u=updu^*+uAu^*.
\end{equation}

\subsection{Scalar product on \texorpdfstring{$\pi(\Omega^p\cA)$}{piOpA} and Hilbert structure}
%---------------------------------------------------------------------------------------------

A spectral triple $(\cA,\hH,D)$ is \defe{tame}{tame spectral triple} if for all $T\in\pi(\Omega\cA)$ and $S\in\oB(\hH)$, we have
\[ 
 \tr_{\omega}(ST|D|^{-n})=\tr_{\omega}(S|D|^{-n} T) 
\]
where $\tr_{\omega}$ is the Dixmier trace. In this case, we have
\begin{equation}
\tr_{\omega}(T_1^*T_2|D|^{-n})=\tr_{\omega}(T_1^*|D|^{-n}T_2)=\tr_{\omega}(T_2|D|^{-n}T_1^*):=\langle T_1,\,T_2\rangle.
\end{equation}
Second equality is the cyclic invariance of trace while the last one is a definition. Tameness assures the symmetry: $\langle T_1,\,T_2\rangle=\langle T_2,\,T_1\rangle$. 

Note that when $T\in\pi(\Omega^p\cA)$, we have $T\in\oB(\hH)$. Indeed the definition of $\pi$ is $\pi(a_0\delta a_1)=a_0[D,a_1]$ and the very definition of a spectral triple says that $[D,a_1]$ and $a_0$ are bounded.

Notice that the definition
\begin{equation}
\langle T_1,\,T_2\rangle_{p}=\tr_{\omega}(T_1^*|D|^{-n}T_2)
\end{equation}
only holds for $T_1$, $T_2\in \pi(\Omega^p\cA)$. When $T_1\in\pi(\Omega^p\cA)$ and $T_2\in\pi(\Omega^q\cA)$ with $p\neq q$, we define $\langle T_1,\,T_2\rangle$.

Let us now take $\tilde{\hH}_{p}$, the completion of $\pi(\Omega^p\cA)$ for $\langle .,\,.\rangle_{p}$. When $a\in\cA$ and $T_1$, $T_2\in\pi(\Omega^p\cA)$, using tameness and cyclic invariance, we find
\begin{subequations}   \label{EqUnitRepHtilde}
\begin{equation}
\langle aT_1,\,aT_2\rangle_{p}=\tr_{\omega}(T_1^*a^*|D|^{-n}aT_2)
		=\tr_{\omega}(aT_2|D|^{-n}T_1^*a^*)
	=\tr_{\omega}(T_2|D|^{-n}T_1^*a^*a),
\end{equation}
and with the same computation
\begin{equation} 
  \langle T_1a,\,T_2a\rangle_{p}=\tr_{\omega}(aa^*T_1^*|D|^{-n}T_2).
\end{equation}
\end{subequations}
We define the \defe{unitary group}{unitary!group!of an algebra}\nomenclature{$U(\cA)$}{Unitary group of the algebra $\cA$} $U(\cA)$ by
\begin{equation}
 U(\cA)=\{ u\in\cA\tq u^*u=uu^*=1 \}.
\end{equation}
This group can be represented on $\tilde{\hH}_{p}$ by
\[ 
  L(a)T=aT\quad\text{and}\quad R(a)T=Ta.
\]
Equations \eqref{EqUnitRepHtilde} show that $R$ and $L$ are unitary representations:
\[ 
  \langle L(a)T_1,\,L(a)T_2\rangle=\langle R(a)T_1,\,R(a)T_2\rangle=\langle T_1,\,T_2\rangle.
\]

\begin{lemma}
The set 
\[ 
  \pi\big( \delta(J_0\cap\Omega^{p-1}\cA) \big)
\]
is a two-sided submodule of $\pi(\Omega^p\cA)$.
\end{lemma}

\begin{proof}
An element of $\pi\delta(J_0\cap\Omega^p\cA)$ has the form $\pi\delta(\omega)$ with $\omega\in\Omega^p\cA$ and $\omega\in J_0$, i.e. $\pi(\omega)=0$. Let $a\in\cA$; from definition of $\pi$, we have $a\pi\delta(\omega)=\pi(a\delta\omega)$. Then
\[ 
  \pi(a\delta\omega)=\pi\big( \delta(a\omega)-\delta(a)\omega \big)
\]
in which the second term is $\pi(\delta a)\pi(\omega)=0$. The conclusion is that $a\pi\delta(\omega)=\pi(\delta (a\omega))$. In order for $\pi\delta(J_0\cap \Omega^p\cA)$ to be a left sub-module of $\pi(\Omega^{p+1}\cA)$, it remains to be proved that $a\omega\in J_0\cap \Omega^p\cA$ when $\omega\in J_0\cap \Omega^p\cA$. It is the case because $\pi(a\omega)=\pi(a)\pi(\omega)=0$.

The right sub-module structure is left as an exercise.

\end{proof}

Let us see that the closure of $\pi\delta(J_0\cap\Omega^{p-1}\cA)$ in $\tilde{\hH}$ is invariant under the representations $L$ and $R$ of $U(\cA)$. Let $u\in U(\cA)$ and $\omega\in \overline{ \pi\delta(J_0\cap \Omega^{p-1}\cA) }$, we want $R(u)\pi\delta\omega\in\overline{ \pi\delta(J_0\cap\Omega^{p-1}\cA) }$ where the bar denotes the closure. We have $R(u)\pi\delta\omega=(\pi\delta\omega)u$, but we just proved that $\pi\delta(J_0\cap\Omega^{p-1}\cA)$ is a right module, so $(\pi\delta\omega)u$ still belongs to this set.

\begin{proposition}
Let $P_{p}$ be the projection parallel to $\pi\delta(J_0\cap \Omega^{p-1}\cA)$ in $\tilde{\hH}_{p}$. Then $[L(a),P_{p}]=0$.
\end{proposition}

\begin{proof}
Let $x=\pi\delta\omega+y\in\tilde{\hH}_{p}$ with $\omega\in J_0\cap\Omega^{p-1}\cA$ and $y$ in the complement. We have 
\[ 
  P_{p}L(a)x=P_{p}(ay)
\]
because $a\pi\delta\omega\in\pi\delta J_0^{p-1}$ from the fact that $\pi\delta J_0^{p-1}$ is a sub-module. On the other hand, $L(a)P_{p}x=ay$. It remains to be proved that $P_{p}(ay)=ay$ when, by construction, $P_{p}y=y$. Indeed if  $ay$ has a component in $J_0^{p-1}$, the $y$ has too because $y=a^{-1}(ay)$ and $J_0^{p-1}$ is a submodule.
\end{proof}
\section{Connections on modules and Yang-Mills functional}
%++++++++++++++++++++++++++++++++++++++++++++++++++++++++++

\subsection{Potential vector and field strength}
%----------------------------------------------

Let $(\cA,\hH,D)$ be a tame spectral triple of dimension $b$ on which we build $\Omega_D\cA=\oplus_{p}\Omega_D^{p}\cA$. A \defe{potential vector}{potential!vector}, or a \defe{gauge potential}{gauge!potential} is a self-adjoint element of $\Omega_D^{1}\cA$. The \defe{field strength}{field!strength} corresponding to the potential vector $V$ is 
\[ 
  \theta=dV+V^{2},
\]
to be compared with the expression \eqref{EaCurvdVVsq} for the curvature. 

\begin{proposition}
The field strength is self-adjoint.
\end{proposition}

\begin{proof}
What does mean, at the class level, the fact to be self-adjoint? Remember that
\[ 
  \Omega_D\cA\simeq \Omega\cA/J\simeq\pi(\Omega\cA)/\pi(\delta J_0).
\]
An element $V\in\Omega_D^1\cA$ is $V=[\omega]$ with $\omega\in\Omega^1\cA$ and the class taken as $[\omega]=[\omega+\eta_1+\delta\eta_2]$ for all $\eta_1\in\Omega^1\cA$ and $\eta_2\in\cA$ such that $\pi(\eta_1)=\pi(\eta_2)=0$. The condition $V=V^*$ means $[\omega]=[\omega^*]$ which in turn \emph{doesn't} implies that $\omega=\omega^*$.

Now let $\omega=a\delta b$, we have 
\[ 
dV-dV^*=[\delta a\delta b]-[d(\delta b^*\, a^*)]
		=[\delta a\delta b]+[\delta b^*\delta a^*]\neq 0.
\]
So we have $[\delta(\omega-\omega^*)]\neq 0$ although $[\omega-\omega^*]=0$. That proves that $\omega-\omega^*$ is junk.

From definition, $\eta$ is junk when $\pi(\eta)=0$ while $\pi(\delta\eta)\neq 0$. The condition $V=V^*$ means $[\omega]=[\omega^*]$, and then $[\omega-\omega^*]=0$. Therefore we can write $\omega-\omega^*=\eta$ with $\pi(\eta)=0$ and $\pi(\delta\eta)\neq 0$. In conclusion we have
\begin{subequations}
\begin{align}
  \pi(\omega-\omega^*)&=0\\
	\pi(\delta\omega-\delta\omega^*)&\neq 0.
\end{align}
\end{subequations}
Since $dV=[\delta\omega]$ and $(dV)^*=[\delta\omega^*]$, 
\[
dV-(dV)^*=[\delta\omega-\delta\omega^*]
		=[\delta(\omega-\omega^*)].
\]
The claim $dV=(dV)^*$ needs to show that $[\delta(\omega-\omega^*)]=0$. It is the case because it is of the form $[\delta\eta]$ with $\pi(\eta)=0$.

\end{proof}


\subsection{Action of the gauge group}
%-------------------------------------

The element $u\in U(\cA)$ acts on the potential vector $V$ by
\begin{equation}	\label{EqDefActGaugePotVec}
  V\mapsto V^{u}:=uVu^*+u[D,u^*],
\end{equation}
to be compared with the expression given in proposition \eqref{EaCurvdVVsq} of the transformation rule of a connection under a gauge transformation.
 
\begin{lemma}
When $V$ is a potential vector, the element $V^{u}$ is still a potential vector.
\end{lemma}

\begin{proof}
Our work is to prove that $V^{u}-(V^{u})^*=0$ in the sense of classes. If $V=[\omega]$, the definition is $[\omega]^{u}=[u\omega u^*]+[u\delta u^*]$, and thus
\[ 
  (V^{u})^*=[u\omega u^*]+[\delta uu^*].
\]
So we have to prove that $u\delta u^*-\delta uu^*\in J$. We have $\delta(1)=\delta(uu^*)=\delta uu^*-u\delta u^*$, but $\pi(\delta(1))=[D,1]=0$, so $\delta(1)\in J$. It proves that $V^{u}=(V^{u})^*$.
\end{proof}

\begin{lemma}   
The transformation law of the field strength is 
\begin{equation}
\theta^{u}=u\theta u^*.
\label{LemTrFieldStrm}
\end{equation}

\end{lemma}
\begin{proof}
No proof.
\end{proof}

\subsection{Yang-Mills}
%----------------------

The \defe{Yang-Mills functional}{Yang-Mills} is defined by
\begin{equation}
YM(V):=\langle dV+V^{2},\,dV+V^{2}\rangle.
\end{equation}

The product is \emph{a priori} only defined on $\Omega^2\cA$, not on the classes. Let us prove that the Yang-Mills functional is well defined. We have to prove that

\begin{theorem}
 The Yang-Mills functional
\begin{enumerate}
\item is positive,
\item is gauge invariant:
\begin{equation}
YM(V)=YM(V^{u}),
\end{equation}
\item and satisfies a minimal principle
\begin{equation}
  YM(V)=\inf\{ I(\alpha)\tq \pi(\alpha)=V \}
\end{equation}
where
\[ 
  I(\alpha)=\tr_{\omega}\big( \pi(\delta\alpha+\alpha^{2})^{2}|D|^{-n} \big)
\]
for $\alpha\in\Omega^1\cA$.
\end{enumerate}

\end{theorem}

\begin{proof}
From property of the Dixmier trace, $\tr_{\omega}(T)\geq 0$ when $T\geq0$. In our case we are looking on $T=\theta^*\theta|D|^{-n}$

\begin{probleme}
	This is for sure positive in some sense, but I have to formalise that concept.
\end{probleme}

The invariance of Yang-Mills under $U(\cA)$ is easy:
\begin{equation}
YM(V^{u})=YM\big( uVu^*+u[D,u^*] \big)
		=\langle u\theta u^*,\,u\theta u^*\rangle
		=\tr_{\omega}(\theta^{2}|D|^{-n})
		=YM(V).
\end{equation}

For the third claim, we begin by proving that $I$ is positive and invariant under the transformation
\[ 
  \alpha\mapsto \alpha^{u}:=u\alpha u^*+u\delta u^*
\]
in the space $\{ \alpha\in\Omega^1\cA \tq\alpha=\alpha^* \}$. Positivity comes from Dixmier trace while for invariance first notice that lemma \ref{LemTrFieldStrm} says that under $\alpha\mapsto u\alpha u^*+u\delta u^*$, we have the transformation $\delta\alpha+\alpha^{2}\mapsto u(\delta\alpha+\alpha^{2})u^*$. Thus
\begin{align*}
  I(\alpha^{u})&=\tr_{\omega}\Big[ \pi\big( u(\delta\alpha+\alpha^{2})u^* \big)^{2}|D|^{-n}  \Big]\\
		&=\tr_{\omega}\Big[ \pi\big( u(\delta\alpha+\alpha^{2})^{2}u^* \big)|D|^{-n}  \Big]\\
		&=\tr_{\omega}\big[ \pi(\delta\alpha+\alpha^{2})^{2}|D|^{-n} \big]\\
		&=I(\alpha).
\end{align*}
Now we see $\Omega_D\cA\simeq\pi(\Omega\cA)/\pi(\delta J_0)$, and we take a $\alpha\in\Omega^1\cA$ such that $[\pi(\alpha)]=V$ where the class is taken modulo $\pi(\delta J_0)$. We are going to prove that 
\begin{equation} \label{EqDvVppi}
  dV+V^{2}=[P\pi(\delta\alpha+\alpha^{2})].
\end{equation}
First, $dV=[\pi(\delta\alpha)]$, so $dV+V^{2}=[\pi(\delta\alpha+\alpha^{2})]$. From definition, $P$ projects $\pi(\delta J_0)$ on zero, so from definition of the classes that precisely are modulo $\delta J_0$, we have for any $\omega$, $[\pi(\omega)]=[P\pi(\omega)]$. In particular, for all $\alpha\in\Omega^1\cA$ such that $[\pi(\alpha)]=V$, we have equation \eqref{EqDvVppi}. 

\end{proof}

\subsection{Fermionic action}
%-----------------------------

Let us first work out the case of the $\cA$-module $\modE=\cA$ described in page \pageref{PgSimplestwordl}.

\begin{theorem}
If $V\in\Omega^1_D\cA$ is a gauge potential (i.e. if $V=V^*$), the quantity
\begin{equation}
I_{Dir}(V,\psi)=\langle \psi, (D+\pi(V))\psi\rangle 
\end{equation}
with $\psi\in\dom(D)\subset\hH$ is invariant under the action of $U(\cA)$. That quantity is the \defe{fermionic action}{fermionic action}.
\end{theorem}

\begin{proof}
First we have 
\[ 
  \pi(V^u)(u\psi)=(u[D,u^*]+uVu^*)(u\psi)=uD(u^*u\psi)-uu^*D(u\psi)+u\pi(V)u^*u\psi;
\]
using the fact that $uu^*=u^*u=1$, we find
\[ 
  \big( D+\pi(V^u) \big)(u\psi)=D(u\psi)+uD\psi-D(u\psi)+u\pi(V)\psi=u\big( D\psi+\pi(V)\psi \big).
\]
The claim now results from the fact that $u$ is an isometry for the inner product.
\end{proof}

Let $\cA$ and $\cB$ be two \hyperref[PgMoritaEq]{Morita equivalent} algebras. If $\cA$ acts on the Hilbert space $\hH$, then we define
\[ 
  \hH'=\modE\otimes_{\cA}\hH
\]
where $\modE$ is the module that implement the Morita equivalence. The algebra naturally acts on $\hH'$. If $\modE$ has an hermitian structure with values in $\cA$ which satisfies the conditions of an inner product (including positivity), then
\begin{equation}
\big( \xi_1\otimes\psi_1,\xi_2\otimes\psi_2 \big)=\langle \psi_1, \langle \xi_1, \xi_2\rangle \psi_2\rangle 
\end{equation}
defines an Hilbert space structure on $\hH'$. Since $\cB$ acts on $\hH'$ in the same way as $\cA$ acts on $\hH$, we want to define an associated Dirac operator. The most immediate choice is $D'(\xi\otimes_{\cA}\psi)=\xi\otimes_{\cA}D\psi$, but it is not well defined because of the tensor product:
\[ 
  \xi a\otimes_{\cA}\psi=\xi\otimes_{\cA}\pi(a)\psi,
\]
so that $D'(\xi a\otimes_{\cA}\psi)=\xi\otimes_{\cA}aD\psi$ while $D'(\xi\otimes_{\cA}a\psi)=\xi\otimes_{\cA}D(a\psi)$, but in general $aD\psi\neq D(a\psi)$. In order to build a suitable extension of $D$ to $\hH'$, we consider a connection $\nabla\colon \modE\to \modE\otimes_{\cA}\Omega^1_D(\cA)$ and we define
\[ 
  D_{\nabla}(\xi\otimes_{\cA}\psi)=\xi\otimes_{\cA}D\psi+(\nabla\xi)\psi
\]
where the second terms has to be understood as $(\nabla\xi)\psi=(\eta\otimes_{\cA}\omega)\psi=\eta\times\pi(\omega)\psi$. In the case where $\modE=\cA$, we know from \emph{the simplest example in the world} (page \pageref{ConnEequalAsime}) that a connection reads $d+A$, so that the fluctuation of the Dirac operator reads $D\mapsto D+V$ where $V$ is any potential vector.


Let us now pass to a general hermitian projective finitely generated module $\modE$ on $\cA$. For that, we consider the space of \defe{gauged spinors}{spinor!gauged} $\modE\otimes_{\cA}\hH$. That space is endowed with the inner product
\begin{equation}
( \xi_1\otimes_{\cA}\psi_1,\xi_2\otimes_{\cA}\psi_2  )=(\psi_1,\langle \xi_1, \xi_2\rangle \psi_2).
\end{equation}
The action of an element of $\End_{\cA}(\modE)$ extends to gauges spinors by
\[ 
  \phi(\xi\otimes_{\cA}\psi)=\phi(\xi)\otimes_{\cA}\psi
\]
for $\phi\in\End_{\cA}(\modE)$. In particular the action of $U(\modE)$ is unitary: $\big( u(\xi_1\otimes\psi_1),u(\xi_2\otimes\psi_2) \big)=(\xi_2\otimes\psi_1,\xi_2\otimes\psi_2)$.

If $\nabla\colon \modE\to \modE\otimes_{\cA}\Omega^1_D\cA$ is a compatible connection, we define the \defe{gauged Dirac operator}{gauged!Dirac operator}
\begin{equation}
\begin{aligned}
 D_{\nabla}\colon \modE\otimes_{\cA}\hH&\to \modE\otimes_{\cA}\hH \\ 
   D_{\nabla}(\xi\otimes\psi)&=	\xi\otimes D\psi+\big( (\mtu\otimes\pi)\nabla_{un}\xi \big)\psi 
\end{aligned}
\end{equation}
where $\nabla_{un}$ is any universal connection projecting on $\nabla$, $\pi(\nabla_{un})=\nabla$. Let $\nabla_{un}=p\delta+\alpha$ with $\alpha\in\End_{\cA}(\modE)\otimes_{\cA}\Omega^1\cA$. Using the action \eqref{EqActallphaEOAp}, the second term in the gauged Dirac operator reads
\begin{align*}
(\mtu\otimes\pi)\nabla_{un}p\big(  \sum_ie_i\otimes\xi^i \big)&=(\mtu\otimes\pi)(p\delta+\alpha)\sum_ip(f_i)\xi^i\\
		&=(\mtu\otimes\pi)\big( \sum_ip(f_i) \otimes_{\cA} \delta \xi^i +\alpha(\xi) \big)\\
		&=\sum_ip(f_i)\otimes_{\cA}[D,\xi^i]+\pi(\alpha)\xi\\
		&=[D,\xi]+\pi(\alpha)\xi\\
		&=d\xi+\pi(\alpha)\xi
\end{align*}
in which the three last lines are notations. If we apply $D_{\nabla}$ on $\Psi=\sum_j\xi_j\otimes\psi_j$, we have
\[ 
  \sum_j\sum_ip(f_i)\otimes_{\cA}[D,\xi^i]\psi_j=\sum_j[D,\xi]\psi_j\in\modE\otimes_{\cA}\hH
\]
and
\begin{align*}
\pi(\alpha)\Psi&=\sum_k(\mtu\otimes\pi)\big( \alpha(\xi) \big)\psi_k\\
		&=\sum_{ijk}(\mtu\otimes\pi)\big( A_i\xi_j\otimes_{\cA}a^i_0\delta a^i_1 \big)\psi_k\\
		&=\sum_{ijk}A_i\xi_j\otimes_{\cA}a^i_0[D,a^i_0]\psi_k\in\modE\otimes\hH.
\end{align*}
The whole gives
\[ 
\begin{split}
  D_{\nabla}\Psi&=\xi\otimes D\psi+\big( [D,\xi]+\pi(\alpha)\xi \big)\psi\\
		&=\xi\otimes D\psi+(d\xi)\psi+\pi(\alpha)\psi\\
		&=\big( pD+\pi(\alpha) \big)\Psi
\end{split}
\]
where $D(\xi\otimes\psi)=\xi\otimes D\psi+d\xi\psi$. Remark that $D_{\nabla}$ only depend on $\pi(\alpha)$ and not of the whole choice of $\nabla_{un}$. Since $\pi(\nabla^1_{un})=\pi(\nabla^2_{un})$ only when $\pi(\alpha^1)=\pi(\alpha^2)$, the gauged operator $D_{\nabla}$ in fact does not depend on the choice of $\nabla_{un}$ in $\pi^{-1}(\nabla)$.

\begin{proposition}
The \defe{gauged Dirac action}{gauged!Dirac action} is
\begin{equation}
I_{Dir}(\nabla,\Psi)=\langle \Psi, D_{\nabla}\Psi\rangle.
\end{equation}
That action is invariant under the unitary group:
\[ 
  I_{Dir}(\nabla^u,u\Psi)=I_{Dir}(\nabla,\Psi)
\]
for every $\Psi\in\modE\otimes_{\cA}\dom D$ and $\nabla\in \mC(\modE)$. 
\end{proposition}

\begin{proof}
We know that $\nabla=pd+A$ and $\nabla^u=pd+A^u$. Using the fact that $(1\otimes\pi)$ commutes with $u$ and $(1\otimes\pi)\alpha=A$,
\[ 
\begin{split}
(1\otimes\pi)\alpha^u&=(1\otimes\pi)(up\delta u^*+u\alpha u^*)\\
		&=u(\mtu\otimes\pi)(p\delta)u^*+u(\mtu\otimes\pi)\alpha u^*\\
		&=updu^*+uAu^*\\
		&=A^u.
\end{split}  
\]
Thus, using $up:pu$,  we have
\begin{align}
\big( pD+\pi(\alpha^u) \big)(u\Psi)&=\big( pD+\pi(up\delta u^*+u\alpha u^*) \big)(u\Psi)\\
		&=(pD)(u\Psi)+updu^*(u\Psi)+u\pi(\alpha)u^*(u\Psi)\\
		&=(pD)(u\Psi)+up[D,u^*](u\Psi)+u\pi(\alpha)(\Psi)\\
		&=upD(u^*u\Psi)+u\pi(\alpha)\Psi\\
		&=u\big( pD+\pi(\alpha) \big)\Psi.
\end{align}

\end{proof}




%%%%%%%%%%%%%%%%%%%%%%%%%%
%
   \section{Forms and physics formalism}
%
%%%%%%%%%%%%%%%%%%%%%%%%

\subsection{Unitary group}
%--------------------------

We consider a K-cycle $(\hH,D)$ over an involutive unital algebra $\cA$. Its \defe{unitary group}{unitary!group!of an algebra} is
\[ 
  U(\cA)=\{ u\in\cA\tq u^*u=uu^*=1 \}.
\]
For such a $u$, we have $uDu^*=D+u[D,u^*]$. This shows that $D$ is not invariant under $U(\cA)$, but that its variation is of the form $u[D,u^*]$. When one looks at equation \eqref{eq_jaugeA_em}, we want the variation of $D$ to be a vector potential. Hence we take the following definitions. A \defe{$k$-form}{form of degree $k$} on $\cA$ is an operator on $\hH$ of the form
\[ 
  \omega^k=\sum_{j}a_0^j[D,a_1^j]\cdots[D,a_k^j],
\]
and a \defe{vector potential}{vector!potential} is a self-adjoint $1$-form of the form
\begin{equation}   \label{EqPotVectV}
  V=\sum_{j}a_0^j[D,a_1^j]
\end{equation}
with $a_k^j\in\cA$. The action of $u\in U(\cA)$ on the operator $D+V$ is
\[ 
  D+V\mapsto u(D+V)u^*=uDu^*+uVu^*=D+u[D,u^*]+uVu^*.
\]
For that reason, we define the action of $u$ on a potential vector $V$ by
\[ 
  V\mapsto\gamma_u(V)=u[D,u^*]+uVu^*,
\]
in such a way that $D+V\mapsto D+\gamma_u(V)$. The definition of $V$ is tricky. Because of the ambiguity of the way to write $V$ under the form \eqref{EqPotVectV}. From this form, in order to make the same as in the Fredholm modules associated with a cycle (see equation \eqref{EqFreddDefbel}), we want to define $dV=\sum_{j}[D,a^j_0][D,a^j_1]$. We defer a few the discussion about the ill definiteness of this definition. Let us go on and study the consequence of that definition on the curvature $\theta(V)=dV/V^2$.

If we replace $V$ by $\gamma_u(V)$, we have to compute $\theta\big( \gamma_u(V) \big)=d\big( \gamma_u(V) \big)+\gamma_u(V)^2$. Since $[D,.]$ is a derivation we find
\[ 
\gamma_u(V)=u[D,u^*]+\sum_{j}ua^j_0[D,a^j_1]u^*
		=u[D,u^*]+\sum_{j}ua^j_0\big( [D,a^j_1u^*]-a^j_1[D,u^*] \big),
\]
so
\[ 
  d\big( \gamma_u(V) \big)=[D,u][D,u^*]+\sum_{j}[D,ua^j_0][D,a^j_1u^*]-\sum_{j}[D,ua^j_0a^j_1][D,u^*],
\]

\begin{proposition}
 We have the equality
\[ 
  d\big( \gamma_u(V) \big)+\gamma_u(V)^2=u(dV+V^2)u^*
\]
as operator on $\hH$.
\end{proposition}
\begin{proof}
No proof.
\end{proof}

\begin{probleme}
	What is a \emph{reduced} algebra?
\end{probleme}

As far as we are considering a K-cycle over $\cA$, we can consider the graded universal reduced algebra $\Omega^*(\cA)$. In this algebra, we have $\Omega^0(\cA)=\cA$ while $\Omega^1(\cA)$ is generated by the symbols $da$ (with $a\in\cA$) and the rules
\begin{itemize}
\item $d(ab)=(da)b+a(db)$,
\item $d1=0$.
\end{itemize}
If $\alpha\in\eC$, these two requirements imply that $d(\alpha a)=(d\alpha)a+\alpha(da)=\alpha(da)$. Hence if we suppose the linearity, we find
\[ 
  d(\alpha a+\beta b)=\alpha da+\beta db.
\]

Recall that the involution of $\cA$ extends to $\Omega^*(\cA)$ by $(da)^*=-da^*$, and the differential is defined by $d(a^0da^1\cdots da^{n})=da^0da^1\cdots da^{n}$. It fulfills $d^2=0$ and $d(\omega_1\omega_2)=(d\omega_1)\omega_2+(-1)^{\omega_1}\omega_1d\omega_2$.

\section{Yang-Mills on the two point space}
%+++++++++++++++++++++++++++++++++++++++++++

Let $X=\{ a,b \}$ be a two point space; the space of functions on $X$ is identified with $\eC\oplus\eC$. An element $f\in\cA$ is given by $(f_a,f_b)$. We are going to define a K-cycle on $X$. First we consider a finite dimensional Hilbert space $\hH$ and we consider the following representation of $\cA$ on $\hH$:
\[ 
  \pi(f)=\begin{pmatrix}
f_a\mtu_a\\&f_b\mtu_b
\end{pmatrix}
\]
with respect to a direct space decomposition $\hH=\hH_a\oplus\hH_b$. The operator $D$ is constrained by commutators with elements of $\cA$. Since a diagonal part of $D$ would vanish in commutators with $\cA$, we can choose an off-diagonal operator $D$ under the form
\begin{equation}
D=\begin{pmatrix}
0&M^*\\
M&0
\end{pmatrix}
\end{equation}
with $M\colon \hH_1\to \hH_2$. For the graduation, one can choose 
\[ 
  \Gamma=\begin{pmatrix}
\mtu_{\hH_1}\\
&-\mtu_{\hH_2}
\end{pmatrix}.
\]
The commutator $[D,\cA]$ reads
\begin{equation}
[D,f]=(f_b-f_a)\begin{pmatrix}
0&M^*\\
-M&0
\end{pmatrix}
\end{equation}
The fact that $D$ is self-adjoint on a finite dimensional space makes that there exists a basis of eigenvectors and that the biggest eigenvalue of $D$ is the square root of the one of $D^2$. So if $\lambda$ is the biggest eigenvalue of $| M |=\sqrt{M^*M}$,
\[ 
  \| [D,f] \|=\lambda|f_b-f_a|.
\]
Therefore the noncommutative distance between $a$ and $b$ is given by
\begin{equation}
d(a,b)=\sup\{ | f_b-f_a |\text{ such that } \lambda| f_b-f_a |\leq 1 \}=1/\lambda.
\end{equation}


One can show that the map $\pi\colon \Omega^1\cA\to \mL(\hH)$ is injective, so that $\Omega^1_D\cA=\Omega^1\cA$ by lemma \ref{LemOmpmdDp}. A general element of $\Omega^1_D\cA$ has the form $\lambda ede+\mu(1-e)de$. Under the representation $\pi$ we have
\begin{align*}
\pi(e)&=\begin{pmatrix}
\mtu\\&0
\end{pmatrix}&
\pi(de)=[D,e]&=\begin{pmatrix}
0	&-M^*\\
M	&0
\end{pmatrix}\\
\pi(ede)&=\begin{pmatrix}
0&-M^*\\
0&0
\end{pmatrix}&
\pi\big( (1-e)de \big)&=
\begin{pmatrix}
0&0\\M&0
\end{pmatrix},
\end{align*}
so that
\begin{equation}		\label{EqElOmAD}
\pi\big( \lambda e de+\mu(1-e)de \big)=\begin{pmatrix}
0	&-\lambda M^*\\
\mu M&0
\end{pmatrix}
\end{equation}



We are now going to study two examples of modules on $\cA$ and write down the corresponding Yang-Mills theory.

\subsection{The simplest module}
%--------------------------------

As first example of module, we consider $\modE=\cA$. We know a spectral triple and the space of $1$-forms $\Omega^1\cA$ on the two point space from \ref{SubSecTripleDeuxPoints}. We know from \ref{ConnEequalAsime} that a potential vector is a self-adjoint element of $\Omega^1_D\cA$; using expression \ref{EqElOmAD} we find
\begin{align}
	V&=-\overline{ \Phi }ede+\Phi(1-e)de\\
	\pi(V)&=\begin{pmatrix}
0&\overline{ \Phi }M^*\\
\Phi M
\end{pmatrix},
\end{align}
where $\Phi$ is any complex number. The curvature is given by $\theta=dV+V^2$ and is computed using the relations
\begin{align*}
ede(1-e)&=ede		&e(de)e&=0	&(1-e)de(1-e)&=0\\
e^2&=e			&d(1-e)&=-de.	&(de)e&=(1-e)de\\
e(1-e)&=0		
\end{align*}
We find the result
\begin{equation}
\theta(V)=-(\overline{ \Phi }+\Phi)dede-\overline{ \Phi }\Phi dede
\end{equation}
The general formula $YM(\nabla)=\tr_{\omega}\big( \pi(\theta)^2| D |^{-d} \big)$ reduces to
\[ 
  YM(V)=\tr\big( \pi(\theta)^{2} \big)
\]
because the spectral triple is zero-dimensional and the Dixmier trace is the usual trace.

\begin{probleme}
	The usual trace is used as Dixmier trace because the Hilbert space is finite dimensional?
\end{probleme}
We have
\[ 
  \pi(dede)^2=\begin{pmatrix}
-M^*M&0\\0&-MM^*
\end{pmatrix},
\]
so that $\tr\big( \pi(dede)^2 \big)=2\tr(MM^*)^2$, and
\[ 
  \pi(\theta)=-\big( (\overline{ \Phi }+\Phi)+\Phi\overline{ \Phi } \big)\begin{pmatrix}
0&M^*\\M&0
\end{pmatrix}.
\]
What is interesting is the dependence in $\Phi$ in the Yang-Mills action:
\begin{equation}
YM(V)=2\big( | \Phi+1 |^2-1 \big)\tr\big( (M^*M)^2 \big).
\end{equation}
If one sees that as a field equation for $\Phi$, one concludes that the right field is $\Phi+1$ instead of $\Phi$. For that reason we perform the change of basis $\phi=1+\Phi$.

In our simple case $\modE=\cA$, an element of $U(\modE)$ is an element $u\in\cA$ such that $uu^*=u^*u=1$. If we write $u=u_ae+u_b(1-e)$ with $u_i\in\eC$, it is easy to see that the condition is $u_au_a^*=u_a^*u_a= u_bu_b^*=u_b^*u_b=1$, so that $U(\cA)=U(1)\times U(A)$.

Let us now apply formula \eqref{EqTransAjauge} on the gauge potential $A=V=(1-\overline{ \phi })ede+(\phi-1)(1-e)de$ and $u=u_ae+u_b(1-e)$. A simple computation shows that 
\[ 
  V^u=ede-(1-e)de-u_a\overline{ u }_b\overline{ \phi }ede+u_b\overline{ u }_a\phi(1-e)de.
\]
Comparing the expression of $V$ with the one of $V^u$ we see that the action of $u$ is to replace $\phi\to u_b\overline{ u }_a\phi$.
