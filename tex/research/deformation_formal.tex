% This is part of (almost) Everything I know in mathematics
% Copyright (c) 2013-2014
%   Laurent Claessens
% See the file fdl-1.3.txt for copying conditions.

%+++++++++++++++++++++++++++++++++++++++++++++++++++++++++++++++++++++++++++++++++++++++++++++++++++++++++++++++++++++++++++
\section{Twists of module (co)algebras}
%+++++++++++++++++++++++++++++++++++++++++++++++++++++++++++++++++++++++++++++++++++++++++++++++++++++++++++++++++++++++++++
\label{SecTheoryTwist}

We are going to follow \cite{GiaquintoZhangTwist} and use the notions of subsection \ref{subSecModulebialgebra}. 
\begin{definition}	\label{DefTwist}
	An element $F\in B\otimes B$ is a \defe{twisting element}{twist} based on $B$ if it satisfies the two following conditions
	\begin{enumerate}

		\item\label{ItemTwistUn}
			$(\epsilon_B\otimes \id)F=(\id\otimes \epsilon_B)F=1\otimes 1$
		\item\label{ItemTwistDeux}
			$\big[ (\Delta_B\otimes\id)(F) \big](F\otimes 1)=\big[ (\id\otimes\Delta_B)(F) \big](1\otimes F)$.
	
	\end{enumerate}
\end{definition}
If $F$ is invertible, we have
\begin{equation}		\label{EqDelidFinve}
	\big[ (\Delta_B\otimes\id)F\big]^{-1}=(\Delta_B\otimes\id)F^{-1}.
\end{equation}
In order to see that, first notice that the third component in the tensor product is $F_2^{-1}$ in both sides of \eqref{EqDelidFinve} while the two first components are given by $\Delta_B(F)^{-1}$ in the left hand side and by $\Delta_B(F^{-1})$ in the right hand side. Using the coalgebra properties, we have
\begin{equation}
	\Delta_B(F)\Delta_B(F^{-1})=\Delta_B(FF^{-1})=\Delta_B(1)=1\otimes 1.
\end{equation}
Thus if we take the inverse of the property \ref{ItemTwistDeux} in the definition of a twist, we get
\begin{equation}		\label{EqFemuVaLeLem}
	(F^{-1}\otimes 1)\big[ (\Delta_B\otimes\id)F^{-1} \big]=(1\otimes F^{-1})\big[ (\id\otimes\Delta_B)F^{-1} \big]^{-1}
\end{equation}

%---------------------------------------------------------------------------------------------------------------------------
\subsection{Twisting the module algebras}
%---------------------------------------------------------------------------------------------------------------------------

Let $\eA$ be a $B$-module algebra with its multiplication $\mu_{\eA}$. If $F$ is a twist based on $B$, we can define the new multiplication
\begin{equation}
	\begin{aligned}
		\mu_F=\mu_{\eA}\circ F_l\colon \eA\otimes\eA&\to \eA \\
		a\otimes a'&\mapsto \mu_{\eA}(ba\otimes b'a') 
	\end{aligned}
a\end{equation}
if $F=b\otimes b'$. In the same way, if $C$ is a $B$-module coalgebra with comultiplication $\Delta_C\colon C\to C\otimes C$, we can deform it by
\begin{equation}
	\Delta_F=F_r\circ\Delta_C\colon C\to C\otimes C.
\end{equation}

\begin{theorem}		\label{ThoTwistAlgEtCoalg}
	Let $B$ be a bialgebra.
	\begin{enumerate}

		\item
			If $\eA$ is a left $B$-module algebra, then $\eA_F=(\mu_{\eA}\circ F_l,1_{\eA})$ is an associative algebra on $\eK$.
		\item
			If $C$ is a $B$-module coalgebra, then $C_F=(F_r\circ\Delta_C,\epsilon_C)$ is a coassociative $\eK$-coalgebra.

	\end{enumerate}
\end{theorem}

\begin{proof}
	Coassociativity of $\eA_F$ means that for every $a,b,c$ in $\eA$,
	\begin{equation}
		(\mu_{\eA}\circ F_l)\big( a\otimes(\mu_{\eA}\circ F_l)(b\otimes c) \big)=(\mu_{\eA}\circ F_l)\big( (\mu_{\eA}\circ F_l)(a\otimes b)\otimes c \big).
	\end{equation}
	In other terms,
	\begin{equation}
		(\mu_{\eA}\circ F_l)\circ\big( \id\otimes(\mu_{\eA}\circ F_l) \big)=(\mu_{\eA}\circ F_l)\circ(\mu_{\eA}\circ F_l)\otimes\id
	\end{equation}
	as map from $\eA\otimes\eA\otimes\eA$ to $\eA$. If we use the decomposition
	\begin{equation}
		\id\otimes(\mu_{\eA}\otimes F_l)=(\id\otimes\mu_{\eA})\circ(\id\otimes F_l),
	\end{equation}
	we see that we have to prove the commutativity of the diagram
	\begin{equation}
		\xymatrix{%
		\eA\otimes\eA\otimes\eA \ar[r]^{F_l\otimes\id}\ar[d]_{\id\otimes F_l}	&	\eA\otimes\eA\otimes\eA \ar[r]^{\mu_{\eA}\circ\id}	&	\eA\otimes\eA\ar[d]^{F_l}\\
		\eA\otimes\eA\otimes\eA	\ar[d]_{\id\otimes F_l}			&								& \eA\otimes\eA\ar[d]^{\mu_{\eA}}\\
		\eA\otimes\eA\ar[r]_{F_l}	& \eA\otimes\eA\ar[r]_{\mu_{\eA}}							&				 \eA
		   }
	\end{equation}
	We refer to \cite{GiaquintoZhangTwist} for the remaining of the proof.
\end{proof}

\begin{lemma}		\label{LemRemakAecP}
	If $P\in B\otimes B$ satisfies
	\begin{equation}
		(P\otimes 1)\big[ (\Delta_B\otimes\id)P \big]=(1\otimes P)\big[ (\id\otimes\Delta_B)(P) \big],
	\end{equation}
	then $P$ twists the left $B$-modules coalgebras.
\end{lemma}
\begin{proof}
		No proof.
\end{proof}
Notice that, by equation \eqref{EqFemuVaLeLem}, $F^{-1}$ satisfies lemma \ref{LemRemakAecP} when it is invertible.
	

%---------------------------------------------------------------------------------------------------------------------------
\subsection{Twisting the bialgebra itself}
%---------------------------------------------------------------------------------------------------------------------------

Let $F$ be an invertible twist base on the bialgebra $B$. We can deform $B$ by
\begin{equation}
	 \Delta_B'=F^{-1}_l\circ F_r\circ\Delta_B.
\end{equation}

\begin{theorem}
	The structure
	\begin{equation}
		B_F=(\mu_B,\Delta_B',1_B,\epsilon_B)
	\end{equation}
	is a $\eK$-bialgebra.
\end{theorem}

\begin{proof}
	The second point of theorem \ref{ThoTwistAlgEtCoalg} makes $F_r\circ\Delta_B$ coassociative since $F$ is a twist. In order to check that this is still a $B$-module coalgebra, we rewrite the diagram \eqref{EqDiagModCoAlg} with $B$ instead of $C$ and $\Delta_B(b)$ by $\Delta_B(b)F$:
\begin{equation}	
	\xymatrix{%
	B\otimes B 		&	B\ar[l]_{F_r\circ\Delta}\\
	B\otimes B \ar[u]^{F_r\circ\Delta(b)_r}	&	B\ar[l]_{F_r\circ\Delta}\ar[u]_{b_r}
	   }
\end{equation}
	We see that the result of this diagram is the one of the non twisted one multiplied by $F$. For example, the vertical left arrow is the mapping
	\begin{equation}
		c\otimes c'\mapsto(c\otimes c')\cdot\Delta(b)F.
	\end{equation}
	Thus the coalgebra $B$ endowed with $\Delta=F_r\circ\Delta_B$ is still a $B$-module coalgebra. Now, using the lemma \ref{LemRemakAecP} and the fact that $F^{-1}$ satisfies that lemma, the structure $\Delta_{B}'=F_l^{-1}\circ F_r\circ\Delta_B$ is coassociative. We check that this structure is compatible with the multiplication:
	\begin{equation}
		\begin{aligned}[]
			\Delta'_B(bb')&=F^{-1}\Delta_B(bb')F\\
					&=F^{-1}\Delta_B(b)\Delta_B(b')F\\
					&=F^{-1}\Delta_B(b)FF^{-1}\Delta_B(b')F\\
					&=\Delta_B'(b)\Delta_B'(b').
		\end{aligned}
	\end{equation}
\end{proof}

%---------------------------------------------------------------------------------------------------------------------------
\subsection{Twisting the left module algebra}
%---------------------------------------------------------------------------------------------------------------------------

\begin{theorem}
	Let $\eA$ be a left $B$-module algebra and $F$, an invertible twist based on $B$. Then
	\begin{equation}
		A_F=(\eA,\mu_{\eA}\circ F_l,1_{\eA})
	\end{equation}
	is a left $B$-module algebra.
\end{theorem}

\begin{proof}
	We already proved that the product in $\eA_F$ is associative. Let us now prove that $\eA_F$ is a $B_F$-module. The action of $B_F$ on $\eA_F$ is unchanged: $b\cdot a$ if $b\in B_F$ and $a\in\eA_F$. The first axiom of definition \ref{DefBModuleAlgebra}, $b\cdot A_{\eA}=\epsilon(b)\cdot 1_{\eA}$, remains because we didn't change $1_{\eA}$ neither $\epsilon$. In the present case, the second condition reads
	\begin{equation}		\label{EqDiagCommdeThobmodalgtwist}
		\xymatrix{%
		\eA\otimes\eA \ar[r]^{\mu_{\eA}\circ F_l}\ar[d]_{\big( F^{-1}\Delta_B(b)F \big)_l}		&	\eA\ar[d]^{b_l}\\
		\eA\otimes\eA \ar[r]_{\mu_{\eA}\circ F_l}	&	\eA
		   }
	\end{equation}
	Let us consider $a\otimes a'\in\eA\otimes\eA$ and follow its evolution when we apply $b_l\circ F_l\circ\mu_{\eA}$.
	\begin{equation}
		\begin{aligned}[]
		a\otimes a'\stackrel{F_l}{\to} F_1a\otimes F_2a'\stackrel{\Delta_B(b)}{\to}\sum b_{(1)}F_1a\otimes b_{(2)}F_2a'\stackrel{F^{-1}}{\to}\sum F^{-1}_1b_{(1)}F_1a\otimes F_2^{-1} b_{(2)}F_2a'\\
		\stackrel{F_l}{\to} \sum b_{(1)}F_1a\otimes b_{(2)}F_1a'\stackrel{\mu_{\eA}}{\to}\mu_{\eA}\big( \stackrel{F_l}{\to} \sum b_{(1)}F_1a\otimes b_{(2)}F_1a' \big).
		\end{aligned}
	\end{equation}
	If we follow the other arrows, we find
	\begin{equation}
			a\otimes a'\stackrel{F_l}{\to}F_1a\otimes F_2a'\stackrel{\mu_{\eA}}{\to}\mu_{\eA}\big( F_1a\otimes F_2 a' \big)\stackrel{b_l}{\to}b\mu_{\eA}\big( F_1a\otimes F_2a' \big).
	\end{equation}
	Now, the commutativity of the untwisted diagram states that
	\begin{equation}
		b\mu_{\eA}(a\otimes a)=\mu_{\eA}\big( \sum b_{(1)}a\otimes b_{(2)}a' \big).
	\end{equation}
	Thus the commutativity of diagram \eqref{EqDiagCommdeThobmodalgtwist} is nothing else than the commutativity of the untwisted diagram taken with $F_1a\otimes F_2a'$ instead of $a\otimes a'$.
\end{proof}

%+++++++++++++++++++++++++++++++++++++++++++++++++++++++++++++++++++++++++++++++++++++++++++++++++++++++++++++++++++++++++++
\section{Twist and star product}
%+++++++++++++++++++++++++++++++++++++++++++++++++++++++++++++++++++++++++++++++++++++++++++++++++++++++++++++++++++++++++++

Let $\eB$ be a connected Lie group with Lie algebra $\lB$ and universal enveloping algebra $\mU(\lB)$.

\begin{definition}
	If $\Delta$ and $\epsilon$ are the usual co-product and co-unit\footnote{See definition \ref{DefHopfsurCG}.} on $\mU(\lB)$, then a \defe{twist based on}{twist!based on $\mU(\lB)$} based on $\mU(\lB)$ is an element $F\in\mU(\lB)\dcr{h}\otimes\mU(\lB)\dcr{h}$ such that
	\begin{subequations}
		\begin{align}
			(\epsilon\otimes\id)F=1\otimes 1=(\id\otimes \epsilon)F					\label{subEqHpfun}	\\
			\big[ (\Delta\otimes\id)(F) \big](F\otimes 1)=\big[ (\id\otimes\Delta)(F) \big](1\otimes F).	\label{subEqHpfcocs}
		\end{align}
	\end{subequations}
\end{definition}
Remark in that definition the difference between ``$\id$'' which is the identity on $\mU(\lB)$ and ``$1$'' which is the constant function, and then the identity on $ C^{\infty}(\eB)$.

\begin{definition}
	A formal \defe{universal deformation formula}{universal!deformation formula} based on $\mU(\lG)$ is a twisting element $F$ based on $\mU(\lB)\dcr{h}$ which reads
	\begin{equation}
		F=1\otimes 1+hF_1+h^2F_2+\cdots+h^nF_n+\ldots
	\end{equation}
	where $F_i\in\big( \mU(\lB)\otimes\mU(\lB) \big)\dcr{h}$.
\end{definition}

\begin{theorem}
	The data of a formal universal deformation formula on $\mU(\lB)\dcr{h}$ is equivalent to the data of a left invariant star product $\star$ on $\eB$.

	The correspondence is as follows. Let $\star=\sum_k h^kC_k$ where $C_k$ are left invariant bidifferential operators on $\eB$. So
	\begin{equation}
		F=\sum_kh^kC_k
	\end{equation}
	is an element of $\mU(\lB)\dcr{h}\otimes \mU(\lB)\dcr{h}$.

	In that setting, $F$ is a Drinfel'd twist and every Drinfel'd twists are produced in that way\index{Drinfel'd twist}.
\end{theorem}

\begin{proof}
	Let us make the correspondence more explicit. From proposition \ref{PropbidiffUU}, we have $\biDiff^{\eB}(\eB)\simeq\mU(\lB)\otimes\mU(\lB)$, the elements $C_k$ can be seen in $\mU(\lB)\otimes\mU(\lB)$ and we define the $F_k$ by $C_k=F_k^L$ where $T^L\in\biDiff^{\eB}(\eB)$ is the left invariant operator associated with $T\in\mU(\lB)\otimes\mU(\lB)$. Then one defines
	\begin{equation}
		F=\sum_kt^kF_k\in\big( \mU(\lB)\otimes\mU(\lB) \big)\dcr{t}.
	\end{equation}
	What we have to proof is that the so defined $F$ is a twist based on $\mU(\lB)$. We are going to prove that the associativity of $\star$ is equivalent to the condition \ref{ItemTwistDeux} while the condition $f\star 1=1\star f=f$ is equivalent to the condition \ref{ItemTwistUn} in definition \ref{DefTwist}.

	Consider $T\in\mU(\lB)\otimes\mU(\lB)$ as $T=\sum_i X_i\otimes Y_i$ where $X_i,Y_i\in\mU(\lB)$ and the sum is finite. Associativity of the product means that
	\begin{equation}
		\tilde T\circ(\tilde T\otimes \id)=\tilde T\circ(\id\otimes \tilde T).
	\end{equation}
	In our computations we are going to use the following rules:
	\begin{enumerate}

		\item
			$(X\otimes Y)(f\otimes g)=Xf\otimes Yg$,
		\item	 \label{ItemOptimRuleDeux}
			$\widetilde{(X\otimes Y)}(f\otimes g)=(\tilde Xg)(\tilde Yg)\in  C^{\infty}(\eB)$,
		\item
			$(\tilde X\otimes \tilde Y)(f\otimes g)=\tilde Xf\otimes \tilde Yg\in C^{\infty}(\eB)\otimes C^{\infty}(\eB)$
		\item
			$\widetilde{(X\cdot Y)}f=\tilde X(\tilde Yf)$ where the dot denotes the product in $\mU(\lB)$.

	\end{enumerate}
	with, as usual, the definition $(\tilde Xf)(x)=\tilde X_xf\in \eR$. As far as the product in $\mU(\lB)$ is concerned, we have
	\begin{equation}
		\begin{aligned}[]
			\big[ (X_1\otimes Y_1\otimes Z_1)\cdot (X_2\otimes Y_2\otimes Z_2) \big]^{\expotilde}(f\otimes g\otimes h)&=
			\big[ X_1\cdot X_2\otimes Y_1\cdot Y_2\otimes Z_1\cdot Z_2 \big]^{\expotilde}(f\otimes g\otimes h)\\
			&=(X_1X_2)^{\expotilde}(f)(Y_1Y_2)^{\expotilde}(g)(Z_1Z_2)^{\expotilde}(h)
		\end{aligned}
	\end{equation}
	where the dot $\cdot$ denotes the product in $\mU(\lB)$. 
	
	Using these rules we have
	\begin{equation}		\label{EqUneCyclTcondass}
		\begin{aligned}[]
			\tilde T\circ(\tilde T\otimes 1)(f\otimes g\otimes h)&=\sum_{ij}(\tilde X_i\otimes\tilde Y_j)\circ
							\Big( (\tilde X_i\otimes\tilde Y_j)\otimes 1 \Big)(f\otimes g\otimes h)\\
			&=\sum_{ij}(\tilde X_i\otimes\tilde Y_i)\Big( (\tilde X_j\otimes\tilde Y_j)(f\otimes g)\otimes h \Big)\\
			&=\sum_{ij}(\tilde X_i\otimes\tilde Y_i)\Big( (\tilde X_jf)(\tilde Y_jg)\otimes h \Big)\\
			&=\sum_{ij}\tilde X_i\Big( (\tilde X_jf)(\tilde Y_jg) \Big)\tilde Y_ih.
		\end{aligned}
	\end{equation}
	Using the fact that $\Delta$ is the usual coproduct on $\mU(\lB)$ and the formula \eqref{EqXfgDeltaUnif}, the last line equals
	\begin{equation}	\label{EqExpijPqsDex}
		\begin{aligned}[]
			\sum_{ij}\widetilde{\Delta(x_i)}\big( (\tilde X_jf)\otimes(\tilde Y_jg) \big)\tilde Y_ih&=\sum_{ij}\big( \widetilde{\Delta(X_i)}\otimes \tilde Y_i \big)\Big( (\tilde X_jf\otimes\tilde Y_jg)\otimes h \Big)\\
			&=\sum_{ij}\Big[ (\Delta\otimes \id)(X_i\otimes Y_i)\Big]^{\expotilde}   \Big( (\tilde X_jf\otimes\tilde Y_jg)\otimes h \Big)\\
			&=\sum_j\big[ (\Delta\otimes \id)T \big]^{\expotilde}\circ\big[ X_j\otimes Y_j\otimes 1 \big]^{\expotilde}(f\otimes g\otimes h)\\
			&=\big[ (\Delta\otimes \id)T \big]^{\expotilde}\circ\big[ T\otimes 1 \big]^{\expotilde}(f\otimes g\otimes h)\\
			&=\big[ (\Delta\otimes\id)(T)\cdot(T\otimes 1) \big]^{\expotilde}(f\otimes g\otimes h).
		\end{aligned}
	\end{equation}
	Equating the last line with the left hand side of \eqref{EqUneCyclTcondass}, we get
	\begin{equation}
		\tilde T\circ(\tilde T\otimes 1)=\big[ (\Delta\otimes I)\cdot (T\otimes 1) \big]^{\expotilde}
	\end{equation}
	Doing the same for each power of $t$, we get the same equation for $F$ instead of $T$. The same way, we also get
	\begin{equation}
		F^L\circ(I\otimes F^L)=\big[ (I\otimes\Delta)(F)\cdot(1\otimes F) \big]^L,
	\end{equation}
	so that the associativity of $\star$ is equivalent to the Hopf cocycle condition \eqref{subEqHpfcocs}.

	Now, we prove that the fact that $1$ is an unit for the star product is equivalent to the condition \eqref{subEqHpfun}. Associativity means that for every function $f$ we have
	\begin{equation}
		F^L(f\otimes 1)=F^L(1\otimes f)=f.
	\end{equation}
	Using the counit on $\mU(\lB)$ (given by item \ref{ItemCounitUg} on page \pageref{ItemCounitUg}), we have
	\begin{equation}
		\begin{aligned}[]
			T^L(f\otimes 1)&=\sum_i(X_i^Lf)(Y_i^L1)\\
			&=\sum_i(X_i^Lf)\epsilon(Y_i)\\
			&=(I\otimes \epsilon)(T)^Lf.
		\end{aligned}
	\end{equation}
	By the same computation,
	\begin{equation}
		T^L(1\otimes f)=(\epsilon\otimes I)(T)^Lf.
	\end{equation}
	Thus the equality $1\star f=f\star 1$ is equivalent to $(I\otimes\epsilon)T=(\epsilon\otimes I)(T)$. Now if we want $\sum_i\epsilon(X_i)Y_i^Lf$ to be equal to $f$ for every $f$, we need $Y_i=1$ whenever $X_i=1$. Thus $T$ has to be of the form
	\begin{equation}
		T=1\otimes 1+\sum_iX_i\otimes Y_i
	\end{equation}
	where none of the $X_i$ and $Y_i$ are $1$. In other words, the only term containing $1$ is the term $1\otimes 1$.
\end{proof}

%+++++++++++++++++++++++++++++++++++++++++++++++++++++++++++++++++++++++++++++++++++++++++++++++++++++++++++++++++++++++++++
\section{Formal Extension lemma}		\label{SecExtenLemK}
%+++++++++++++++++++++++++++++++++++++++++++++++++++++++++++++++++++++++++++++++++++++++++++++++++++++++++++++++++++++++++++

The \emph{extension lemme} was already presented in \cite{These,articleBVCS}. A non-formal version is given in section \ref{SecExtLem}; here we follow the presentation of \cite{QuantifKhalerian} and we give here more details from the formal twist point of view.

Let $\eB_1$ and $\eB_2$ be two Lie groups and let us consider the direct product $\eB=\eB_1\times_R\eB_2$. In particular, $\eB\simeq\eB_1\times\eB_2$ as manifold, $\eB_1$ is normal in $\eB$ and $\eB_1\cap\eB_2=\{ e \}$. The extension map $R$ is given by
\begin{equation}
	R_x(y)=xyx^{-1}
\end{equation}
for every $x\in\eB_1$ and $y\in\eB_2$.

Let now $\lB$, $\lB_1$ and $\lB_2$ be the respective Lie algebras and consider
\begin{equation}
	\rho\colon \lB_2\to \Der(\lB_1)
\end{equation}
be the differential of $R$. At the Lie algebra level we have $\rho_X(Y)=[X,Y]$, but it can be extended to the universal enveloping algebra by the formula
\begin{equation}
	\rho_X(Y)=X^{(1)}\cdot Y\cdot S(X^{(2)})
\end{equation}
where $S$ is the antipode in $\mU(\lB_2)$ and $\Delta(X)=X^{(1)}\otimes X^{(2)}$.
\begin{probleme}
	Could be great to have a confirmation of that.
\end{probleme}


We have an action
\begin{equation}
	\begin{aligned}
		r\colon \eB_2&\to \Aut(\lB_1) \\
		r_h(X)&\mapsto \Dsdd{ R_h( e^{tX}) }{t}{0} 
	\end{aligned}
\end{equation}
for every $h\in\eB_2$ and $X\in\lB_1$. This action extends to $\mU(\lB_1)$.

\begin{lemma}
	The action $R$ can be retrieved from $r$ by the formula
	\begin{equation}
		R_h( e^{X})= e^{r_h(X)}.
	\end{equation}
\end{lemma}

\begin{proof}
	By definition, $ e^{r_h(X)}= e^{(dR_h)_eX}=R_h( e^{X})$.
\end{proof}

The map $R_h$ can be extended to $\mU(\lB_1)$ by
\begin{equation}
	r_h(X\cdot Y)=\DDsdd{ R_h( e^{tX})R_h( e^{sY}) }{t}{0}{s}{0}=\DDsdd{ h e^{tX} e^{sY}h^{-1} }{t}{0}{s}{0}
\end{equation}

Let now denote by $\mU(\lB_1)^{\eB_2}$\nomenclature[G]{$\mU(\lB_1)^{\eB_2}$}{The elements of $\mU(\lB_1)$ that are fixed by the action of $\eB_2$.} the set of the elements of $\mU(\lB_2)$ that are fixed by the action $r$, that is the elements $X\in\mU(\lB_1)$ such that $r_h(X)=X$ for every $h\in\eB_2$.

\begin{lemma}
	Let $X\in\mU(\lB_1)$ such that $r_h(X)=X$ and $Y\in\lB_2$. Thus $X\cdot Y=Y\cdot X$.
\end{lemma}

\begin{proof}
	The condition $r_h(X)=X$ means that $\Dsdd{ h e^{tX}h^{-1} }{t}{0}=X$ for every $h\in\eB_2$. Let us take a path $Y(s)$ in $\eB_2$ and derive the equation $r_{Y(s)}X=X$ with respect to $s$:
	\begin{equation}
			0=\DDsdd{ Y(s) e^{tX}Y(s)^{-1} }{t}{0}{s}{0}=\DDsdd{ \AD\big( Y(s) \big) e^{tX} }{t}{0}{s}{0}=[Y,X]
	\end{equation}
\end{proof}

\begin{lemma}
	We have $\Diff^{\eB_1\times\eB_1,\eB_2\times\eB_2}(\eB_1\times\eB_1)\simeq\biDiff^{\eB_1,\eB_2}(\eB_1)$.
\end{lemma}

\begin{proof}
	For the proof we show that both of the two sets can be identified with $\mU(\lB_1)^{\lB_2}\otimes\mU(\lB_1)^{\lB_2}$. 

	The $\eB_2\times\eB_2$-invariance of an operator $P\in\Diff^{\eB_1\times\eB_1}(\eB_1\times\eB_1)$ means that for every $u\in C^{\infty}(\eB_1\times\eB_1)$ and $(b_2,b'_2)\in\eB_2\times\eB_2$ we have
	\begin{equation}
		(Pu)(b_2b_1,b'_2b'_1)=P\big( L_{(b_2,b'_2)}u \big)(b_1,b'_1).
	\end{equation}
	Thus for each $t,s\in\eR$, $Z_2,Z'_2\in\lB_2$ we have
	\begin{equation}
		(Pu)\big(  e^{tZ_2}b_1, e^{sZ'_2}b'_1 \big)=P\big( L_{( e^{tZ_2}, e^{sZ'_2})}u \big)(b_1,b'_1)
	\end{equation}
	If $P$ corresponds to $X\otimes Y\in\mU(\lB_1)\otimes\mU(\lB_1)$, the derivative with respect to $t$ and $s$ yields
	\begin{equation}
		\big[ (Z_2\otimes Z'_2)\cdot(X\otimes Y)u \big](b_1,b_2)=\big[ (X\otimes Y)\cdot (Z_2\otimes Z'_2)u \big](b_1,b_2),
	\end{equation}
	that is the fact that $Z_2$ and $Z'_2$ respectively commute with $X$ and $Y$. Thus the elements $X$ and $Y$ have to belong to $\mU(\lB_1)^{\lB_2}$.

	Let now $c\in\biDiff^{\eB_1}(\eB_1)$ be given by
	\begin{equation}
		c(u\otimes v)(b_1)=\sum_{ab}\tilde X_{b_1}^a(u)\tilde X_{b_1}^b(v)
	\end{equation}
	with $X^a,X^b\in\mU(\lB_1)$. Now we want to impose it to be $\eB_2$-invariant for the action $R_y(x)=yxy^{-1}$. So
	\begin{equation}
		(R_y^*c)(u\otimes v)=c\big( R^*_yu\otimes R^*_yv \big),
	\end{equation}
	or more explicitly
	\begin{equation}		\label{EqCuvexplinv}
		c(u\otimes v)(yxy^{-1})=\sum_{ab}\big( \tilde X^a_{yxy^{-1}}u \big)\big( \tilde X^b_{yxy^{-1}}v \big)
		\stackrel{!}{=}\sum_{ab}\tilde X^a_x(R^*_yu)\tilde X^b_x(R^*_yv).
	\end{equation}
	On the one hand
	\begin{equation}
		\begin{aligned}[]
			\tilde X_x^a(R^*_yu)&=\Dsdd{ (R^*_yu)(x e^{tX}) }{t}{0}\\
			&=\Dsdd{ u\big( yx e^{tX}y^{-1} \big) }{t}{0},
		\end{aligned}
	\end{equation}
	and on the other hand
	\begin{equation}
		\tilde X^a_{yxy^{-1}}u=\Dsdd{ u\big( yxy^{-1} e^{tX} \big) }{t}{0}.
	\end{equation}
	If we consider the relation \eqref{EqCuvexplinv} at $x=e$ we find the condition $ e^{tX}y=y e^{tX}$ for every $y$. Taking $y= e^{sY}$ and taking the derivative with respect to $t$ and $s$ we find the relation $[X,Y]=0$.
\end{proof}

Consider the extension
\begin{equation}
	\lB=\lB_2\times_{\rho}\lB_1
\end{equation}
and the associated group extension
\begin{equation}
	\eB=\eB_2\times_R\eR_1
\end{equation}
with $R\colon \eB_2\to \Aut(\eB_1)$. 

When $\cA$ and $\cB$ are $C^*$-algebra, the elements of $\cA\otimes \cB$ are limits of sums of the form $\sum_ia_i\otimes b_i$ with $a_i\in\cA$ and $b_i\in\cB$. In the case with $\cA= C^{\infty}(A)$ and $\cB= C^{\infty}(B)$, we have
\begin{equation}		\label{EqCABsimeqCACB}
	C^{\infty}(A)\otimes C^{\infty}(B)\simeq C^{\infty}(A\times B)
\end{equation}
from section \ref{SecTensProdCSA}.

The product on $\eB$ is given by
\begin{equation}
	(b_2,b_1)\cdot (b_2',b_1')=\big( b_2b_2',b_1\cdot R(b_2)b'1 \big).
\end{equation}
At the level of the regular left representation, we have
\begin{equation}	\label{Eq1507Lbdb1LoL}
	L^*_{(b_2,b_1)}=L^*_{b_2}\otimes L^*_{b_1}\circ R(b_2).
\end{equation}
For sake of compactness we write
\begin{equation}
	A=L_{b_1}\circ R(b_2).
\end{equation}
The identification \eqref{EqCABsimeqCACB} allows us to give a sense to the action of $L^*_{(b_2,b_1)}$ on elements like $(a\otimes u)\otimes(b\otimes v)$. First, the expression $(a\otimes u)\otimes(b\otimes v)$ is associated to $(a\cdot b)\otimes(u\cdot v)$ and \eqref{Eq1507Lbdb1LoL} allows us to write
\begin{equation}
	\begin{aligned}[]
		L^*_{(b_2,b_1)}\big( a\cdot b\otimes u\cdot v \big)&=L^*_{b_2}(a\cdot b)\otimes \big( L_{b_1}\circ R(b_2) \big)^*(u\cdot v)\\
		&=(L^*_{b_2}a)\cdot(L^*_{b_2}b)\otimes (A^*u)\cdot (A^*v)\\
		&=\big( L^*_{b_2}a\otimes A^*u \big)\otimes\big( L^*_{b_2}b\otimes A^*v \big),
	\end{aligned}
\end{equation}
where the last equality is the identification \eqref{EqCABsimeqCACB} in the reverse sense. Thus we \emph{define} that $L^*_{(b_2,b_1)}$ acts on $\Big(  C^{\infty}(\eB_2)\otimes C^{\infty}(\eB_1) \Big)\otimes\Big(  C^{\infty}(\eB_2)\otimes C^{\infty}(\eB_1) \Big)$ as
\begin{equation}		\label{Eq1507LsurBBBB}
	L^*_{(b_2,b_1)}(a\otimes u)\otimes(b\otimes v)=\big( L^*_{b_2}a\otimes A^*u \big)\otimes\big( L^*_{b_2}b\otimes A^*v \big)
\end{equation}
where $A=L_{b_1}\circ R(b_2)$.

Consider the multiplications $\mu^1$ on $ C^{\infty}(\eB_1)$, $\mu^2$ in $ C^{\infty}(\eB_2)$ and $\mu^{12}$ on $ C^{\infty}(\eB_2\times \eB_1)$. 

\begin{lemma}		\label{Lem1607mualalmu}
	We have $\mu^{21}\circ(L_{b_2}\otimes A)^*\otimes(L_{b_2}\otimes A)^*=(L_{b_2}\otimes A)^*\circ\mu^{21}$.
\end{lemma}

\begin{proof}
	The result comes essentially that $A$ and $L^*_{b_2}$ commute with the pointwise product of functions:
	\begin{equation}
		\begin{aligned}[]
			A^*(u\cdot v)&=A^*u\cdot A^*v\\
			L^*_{b_2}(a\cdot b)&=(L^*_{b_2}a)\cdot(L^*_{b_2}b).
		\end{aligned}
	\end{equation}
	We have
	\begin{equation}
		\begin{aligned}[]
			(L_{b_2}\otimes A)^*\mu^{21}(a\otimes u)\otimes(b\otimes v)
			&=(L_{b_2}\otimes A)^*(a\cdot b\otimes u\cdot v)\\
			&=\mu^{21}(L^*_{b_2}a\otimes A^*u)\otimes(L^*_{b_2}b\otimes A^*v)\\
			&=\mu^{21}\circ\Big( (L_{b_2j\otimes A})^*\otimes(L_{b_2}\otimes A)^* \Big)(a\otimes u)\otimes(b\otimes v).
		\end{aligned}
	\end{equation}
\end{proof}

The same kind of result.
\begin{lemma}		\label{Lem1607LmumuLotimes}
	We have
	\begin{equation}
		L^*_{(b_2,b_1)}\circ\mu^{21}=\mu^{21}\circ\big( L^*_{(b_2,b_1)}\otimes L^*_{(b_2,b_1)} \big).
	\end{equation}
\end{lemma}

\begin{proof}
	If $f\in C^{\infty}(\eB_2)$ and $g\in C^{\infty}(\eB_1)$, seen as functions on $€B_2\times\eB_1$ which depend of only one variable, we have
	\begin{equation}
		\begin{aligned}[]
			L^*_{(b_2,b_1)}\circ\mu^{21}(f\otimes g)(b'_2,b'_1)&=L^*_{(b_2,b_1)}(f\cdot g)(b'_2,b'_1)\\
			&=(L^*_{(b_2,b_1)}f)(b'_2,b'_1)(L^*_{(b_2,b_1)}g)(b_2',b_1')\\
			&=\mu^{21}\big( L^*_{(b_2,b_1)}\otimes L^*_{(b_2,b_1)} g \big)(b'_2,b'_1)\\
			&=\mu^{21}\circ(L^*_{(b_2,b_1)}\otimes L^*_{(b_2,b_1)})(f\otimes g)(b'_2,b'_1).
		\end{aligned}
	\end{equation}
\end{proof}

A bidifferential operator on $ C^{\infty}(\eB_1)$ reads
\begin{equation}
	\tilde P=\mu^1\circ\sum_i \tilde P'_i\otimes \tilde P''_i
\end{equation}
with $P_i',P''_i\in\mU(\lB_1)$. From $\tilde P$ we define a bidifferential operator on $ C^{\infty}(\eB_2)\otimes C^{\infty}(\eB_1)$ by
\begin{equation}
	\tilde P^{(1)}=\mu^{21}\circ(\id\otimes \tilde P')\otimes(\id\otimes \tilde P'').
\end{equation}
That operator acts on (sums of) elements of the type $(u_1\otimes v_1)\otimes(u_2\otimes v_2)$ with $u_i\in C^{\infty}(\eB_2)$ and $v_i\in C^{\infty}(\eB_1)$ with
\begin{equation}
	(\id\otimes \tilde P'')(u_2\otimes v_2)=u_2\otimes \tilde P''v_2\in C^{\infty}(\eB_2)\otimes C^{\infty}(\eB_1).
\end{equation}
Thus
\begin{equation}
	P^{(1)}(u_1\otimes v_1)\otimes(u_2\otimes v_2)=\mu^{21}(u_1\otimes \tilde Pv_1)\otimes(u_2\otimes \tilde P''v_2)=u_1u_2\otimes \tilde P'v_1\tilde P''v_2.
\end{equation}

In the same way, to every bidifferential operator $\tilde L$ we associate the bidifferential operator $\tilde L^{(2)}$ on $ C^{\infty}(\eB_2\otimes\eB_1)$ defined by
\begin{equation}
	\tilde L^{(2)}=\mu^{21}\circ(\tilde L'\otimes\id)\otimes(\tilde L''\otimes\id).
\end{equation}
where a summation is understood.

The following proposition (as everything here) comes from \cite{QuantifKhalerian}.

\begin{proposition}		\label{PropPBBPtildeComm}
	Let $P\in\mU(\lB_1)\otimes\mU(\lB_1)$ and $\tilde P$ the associated left invariant bidifferential operator on $\eB_1$.
	\begin{enumerate}

		\item\label{ItemPropPBBPtildeCommi}
			For every $g\in\eB_2$ we have $R(g)\circ\tilde P=\tilde P\circ R(g)\otimes R(g)$ if and only if $P\in\big( \mU(\lB_1)\otimes\mU(\lB_1) \big)^{\lB_2}$. The latter subalgebra of $\mU(\lB_1)\otimes\mU(\lB_1)$ being formed by the elements $X_1\otimes Y_1$ such that $\rho(Z)(X_1\otimes Y_1)=0$ for every $Z\in\lB_2$.

		\item\label{ItemPropPBBPtildeCommii}
			For every $P\in\mU(\lB_1)^{\lB_2}\otimes\mU(\lB_1)^{\lB_2}$, the associated left invariant operator $P^{(1)}$ on $ C^{\infty}(\eB_2\times \eB_1)$ is $\eB$-invariant. In other words we have
			\begin{equation}
				L^*_{b_2,b_1}\circ\tilde P^{(1)}=\tilde P^{(1)}\circ L^*_{b_2,b_1}
			\end{equation}
			in the sense of the identification \eqref{EqCABsimeqCACB}.
		\item\label{ItemPropPBBPtildeCommiii}
			If $\tilde L$ is a left invariant bidifferential operator on $ C^{\infty}(\eB_2)$, the associated bidifferential operator $\tilde L^{(2)}$ on $ C^{\infty}(\eB)$ is $\eB$-invariant, that is
			\begin{equation}
				L^*_{(b_2,b_1)}\circ \tilde L^{(2)}=\tilde L^{(2)}\circ L^*_{(b_2,b_1)}.
			\end{equation}
	\end{enumerate}
	
\end{proposition}

\begin{proof}
	Let $P=X\otimes Y$ with $X,Y\in\lB_1$. First we have $R(g^{-1})\otimes R(g^{-1})=R(g^{-1})X\otimes R(g^{-1})Y$, but by definition we have
	\begin{equation}
		R(g)X= e^{\rho(Z)}X
	\end{equation}
	and $ e^{\rho(Z)}X=X$ if and only if $\rho(Z)X=0$.
	\begin{probleme}
		Because $ e^{tZ}X$ is a polynomial in $t$ which can only be equal to $X$ when the coefficients of all the powers of $t$ are zero ??
	\end{probleme}

	Let us now prove that 
	\begin{equation}
		R(g)\circ\tilde P=\big[ R(g^{-1})\otimes R(g^{-1})P \big]^{\expotilde}\circ R(g)\otimes R(g),
	\end{equation}
	which is equivalent to the property \ref{ItemPropPBBPtildeCommi} because of the property we just mentioned. We begin with $P=X_1\otimes X_2\in\lB_1\otimes \lB_2$. We write $X=X_1\otimes X_2$.

	Using the definitions,
	\begin{equation}		\label{EqRgxtXunRgpT}
		\begin{aligned}[]
			\big( R(g)\circ \tilde P \big)(u\otimes v)(x)&=\tilde X(u\otimes v)\big( R(g)x \big)\\
			&=\Dsdd{ u\big( R(g)(x) e^{tX_1} \big) }{t}{0}\Dsdd{ v\big( R(g)(x) e^{tX_2} \big) }{t}{0}.
		\end{aligned}
	\end{equation}
	Since $R_g\in\Aut(\eB_1)$, we have $(R_gx)y=R_g(xR_{g^{-1}}y)$, so that the right hand side of \eqref{EqRgxtXunRgpT} becomes
	\begin{equation}	\label{EqDsdduRgXungu}
		\begin{aligned}[]
			\Dsdd{ u\big( R_g(xR_{g^{-1}} e^{tX_1}) \big) }{t}{0}&\Dsdd{ \big(   R_g(xR_{g^{-1}} e^{tX_2})    \big) }{t}{0}\\
			&=\Dsdd{ (R^*_gu)\big( xR_{g^{-1}} e^{tX_1}\big)}{t}{0}\Dsdd{ (R^*_gv)\big( xR_{g^{-1}} e^{tX_2} \big) }{t}{0}.
		\end{aligned}
	\end{equation}
	For sake of shortness, we write $R_gY=\Dsdd{ R_g e^{tY} }{t}{0}$, so that we have
	\begin{equation}
			\Dsdd{ w\big( xR_{g^{-1}} e^{tX} \big) }{t}{0}=(\widetilde{ R_{g^{-1}}X })_x(w)
			=\widetilde{ (R_{g^{-1}}X) }(w)(x),
	\end{equation}
	and \eqref{EqDsdduRgXungu} becomes
	\begin{equation}
		\begin{aligned}[]
			(\widetilde{ R_{g^{-1}}X_1 })(R_g^*u)(x)(\widetilde{ R_{g^{-1}}X_2 })(R_g^*v)(x)&=\Big[ (R_{g^{-1}}\otimes R_{g^{-1}})(X_1\otimes X_2) \Big]^{\expotilde}\big( R^*_gu\otimes R^*_gv \big)(x)\\
			&=\big[ (R_{g^{-1}}\otimes R_{g^{-1}})X \big]^{\expotilde}\circ(R_g\otimes R_g)(u\otimes v)(x).
		\end{aligned}
	\end{equation}
	This concludes the proof of the point \ref{ItemPropPBBPtildeCommi}.

	For point \ref{ItemPropPBBPtildeCommii}, we take $a_i,b_j\in C^{\infty}(\eB_2)$ and $u_i,v_j\in C^{\infty}(\eB_1)$ and we apply $L^*_{(b_2,b_1)}\circ \tilde P^{(1)}$ to the combination $\big( \sum_ia_i\otimes u_i \big)\otimes\big( \sum_j b_j\otimes v_j \big)$. By linearity, we restrict ourself to only one term and we denote by a single dot the pointwise function product:
	\begin{equation}	\label{EqLnnmuLBPP1507}
		\begin{aligned}[]
			\big( L^*_{(b_2,b_1)}\circ\tilde P^{(1)} \big)\big( (a\otimes u)\otimes (b\otimes v) \big)&=
			L^*_{(b_2,b_1)}\circ\mu^{21}\big( (a\otimes \tilde P'u)\otimes(b\otimes \tilde P''v) \big)\\
			&=L^*_{(b_2,b_1)}\big( a\cdot b\otimes \tilde P'u\cdot\tilde P''v \big)\\
			&=L^*_{b_2}(a\cdot b)\otimes \big( L_{b_1}\circ R(b_2) \big)^*(\tilde P'u\cdot \tilde P''v)\\
			&=(L^*_{b_2}a)\cdot(L^*_{b_2}b)\otimes \big( L_{b_1}\circ R(b_2) \big)^*\tilde P'u\cdot \big( L_{b_1}\circ R(b_2) \big)^*\tilde P''v\\
			&=\mu^{12}\Big[ (L^*_{b_2}a)\otimes\big( L_{b_1}\circ R(b_2) \big)^*\tilde P'u \Big]\otimes\Big[ L^*_{b_2}b\otimes(L_{b_1}\circ R(b_2))^*\tilde P''v \Big].
		\end{aligned}
	\end{equation}
	What lies in the first bracket is
	\begin{equation}
		\Big( L^*_{b_2}\otimes\big( L_{b_1}\circ R(b_2) \big)^* \Big)(a\otimes \tilde P'u)=L_{(b_2,b_1)}^*(a\otimes \tilde P'u).
	\end{equation}
	Thus the last line of \eqref{EqLnnmuLBPP1507} provides
	\begin{equation}
		\begin{aligned}[]
			\big( L^*_{(b_2,b_1)}\circ\tilde P^{(1)} \big)&\big( (a\otimes u)\otimes (b\otimes v) \big)\\
			&=\mu^{12}\circ\big( L^*_{(b_2,b_1)}\otimes L^*_{(b_2,b_1)} \big)(a\otimes \tilde P'u)\otimes(b\otimes \tilde P''v)\\
			&=\mu^{12}\circ\big( L^*_{(b_2,b_1)}\otimes L^*_{(b_2,b_1)} \big)\circ\big( (\id\otimes\tilde P')\otimes(\id\otimes\tilde P'') \big)(a\otimes u)\otimes(b\otimes v)
		\end{aligned}
	\end{equation}
	Dropping the explicit reference to the functions $a$, $b$, $u$ and $v$ and using the formula \eqref{Eq1507Lbdb1LoL}, we have
	\begin{equation}
		\begin{aligned}[]
			\big( L^*_{(b_2,b_1)}\circ\tilde P^{(1)} \big)&=
			\mu^{21}\circ\Big[ \id\circ L^*_{b_2}\otimes\tilde P'\circ A \Big]\otimes\Big[  \id\circ L^*_{b_2}\otimes\tilde P'\circ A  \Big]\\
			&=\mu^{21}\circ\Big[(\id\otimes\tilde \tilde P')\circ\big( L^*_{b_2}\otimes A^* \big)]\otimes\Big[  (\id\otimes\tilde P'')\circ\big( L^*_{b_2}\otimes A^* \big)\Big]\\
			&=\mu^{21}\circ(\id\otimes \tilde P')\otimes(\id\otimes\tilde P'')\circ(L^*_{b_2}\otimes A)^*\otimes(L^*_{b_2}\otimes A)^*\\
			&=\tilde P^{(1)}\circ L^*_{(b_2,b_1)}
		\end{aligned}
	\end{equation}
	where we used the expression \eqref{Eq1507LsurBBBB}. This concludes the proof of \ref{ItemPropPBBPtildeCommii}.

	For \ref{ItemPropPBBPtildeCommiii}, the hypothesis of invariance is that for every $a,b\in C^{\infty}(\eB_2)$,
	\begin{equation}
		\tilde L(L^*_{b_2}a\otimes L^*_{b_2}b)=(L^*_{b_2}\circ\tilde L)(a\otimes b).
	\end{equation}
	From linearity and the density of $ C^{\infty}(\eB_2)\otimes C^{\infty}(\eB_1)$ in $ C^{\infty}(\eB_2\times\eB_1)$, it is sufficient to study 
	\begin{equation}
		L^*_{(b_2,b_1)}\tilde L^{(2)}\Big( (a\otimes u)\otimes(b\otimes v) \Big).
	\end{equation}
	We have
	\begin{equation}
		\begin{aligned}[]
			L^*_{(b_2,b_1)}\tilde L^{(2)}\Big( (a\otimes u)\otimes(b\otimes v) \Big)&=L^*_{(b_2,b_1)}\circ\mu^{21}(\tilde L'a\otimes u)\otimes (\tilde L''b\otimes v)\\
			&=L^*_{(b_2,b_1)}(\tilde L'a\cdot\tilde L''b)\otimes u\cdot v\\
			&=L^*_{b_2}(\tilde P'a\cdot\tilde L''b)\otimes A^*(u\cdot v)\\
			&=\mu^{21}\Big[ L^*_{b_2}\tilde P'a\otimes A^*u \Big]\otimes\Big[ L^*_{b_2}\tilde L''b\otimes A^*v \Big]\\
			&=\mu^{21}\circ\big( (L_{b_2}\otimes A)^*\otimes(L_{b_2}\otimes A)^* \big)\\
			&\quad	\circ\big( (\tilde L'\otimes\id)\otimes(\tilde L''\otimes\id) \big)(a\otimes u)\otimes(b\otimes v)
		\end{aligned}
	\end{equation}
	The result now comes from the fact that
	\begin{equation}
		\mu^{21}\circ(L_{b_2}\otimes A)^*\otimes(L_{b_2}\otimes A)^*=(L_{b_2}\otimes A)^*\circ\mu^{21}.
	\end{equation}
	This is lemma \ref{Lem1607mualalmu}.
\end{proof}

\begin{theorem}[\defe{formal extension lemma}{formal!extension lemma}]
	Let $\star^j=\sum_kh^kC_k^j$ be left invariant products on $ C^{\infty}(\eB_j)\dcr{h}$, and suppose that $\star^2$ is invariant under $R(g)$ for every $g\in \eB_2$. Then we define the product $\star$ on $ C^{\infty}(\eB)\dcr{h}$ by
	\begin{equation}
		\star=\star^2\otimes\star^2=\sum_kh^kC_k
	\end{equation}
	with
	\begin{equation}
		C_k(a\otimes u,b\otimes v)=\sum_{k_1+k_2=k}C_{k_1}^2(a,b)C^1_{k_2}(u,v).
	\end{equation}
	This product is left invariant under the action of $\eB$.
\end{theorem}

\begin{proof}
	What we have to study is
	\begin{equation}
		L^*_{(b_2,b_1)}\circ C_k\big( (a\otimes u),(b\otimes v) \big)=\sum_{k_1+k_2=k}L^*_{(b_2,b_1)}C^2_{k_1}(a,b)C^1_{k_2}(u,v),
	\end{equation}
	and by linearity, we can restrict ourself to study only one term in the sum:
	\begin{equation}		\label{Eq1607LCCLCLC}
		\begin{aligned}[]
			L^*_{(b_2,b_1)}C_k^2(a,b)C_l^1(u,v)&=L^*_{(b_2,b_1)}\circ\mu^{21}C^2_k(a,b)\otimes C^1_l(u,v)\\
			&=\mu^{21}\Big[ L^*_{(b_2,b_1)}C^2_k(a,b) \Big]\otimes\Big[ L^*_{(b_2,b_1)}C_l^1(u,v) \Big].
		\end{aligned}
	\end{equation}
	because here, $\mu^{21}$ has to be understood as in lemma \ref{Lem1607LmumuLotimes}, in particular it commutes with $L^*_{(b_2,b_1)}$. We are going to study separately the content of the two brackets. For the first one we have
	\begin{equation}
		L^*_{(b_2,b_1)}C^2_k(a,b)(b'_2,b'_1)=C^2_k(a,b)(b_2b'_2,b_1R_{b_2}b'_1),
	\end{equation}
	but $a$ and $b$ do not depend on the $\eB_1$ component of the variable while $C^2$ is $\eB_2$-invariant, thus
	\begin{equation}	\label{Eq1607BrackUnMu}
		C^2_k(a,b)(b_2b'_2,b_1R_{b_2}b'_1)=C^2_k(a,b)(b_2b'_2)
		=C^2_k(L^*_{b_2}a,L^*_{b_2}b).
	\end{equation}
	For the second bracket of \eqref{Eq1607LCCLCLC} we have
	\begin{equation}		\label{1607BrackDeuxMu}
		\begin{aligned}[]
			L^*_{(b_2,b_1)}C^1_l(u,v)(b'_2,b'_2)&=C^1_l(u,v)\big( b_2b'_2,b_1R(b_2)b'_1 \big)\\
			&=(L^*_{b_1}C^1_l)(u,v)\big( .,R(b_2)b'_1 \big)\\
			&=C^1_l\big( L^*_{b_1}u,L^*_{b_1}v \big)(.,R(b_2)b'_1)\\
			&=C^1_l\big( R(b_2)^*\circ L_{b_1}^*u,A^*v \big)
		\end{aligned}
	\end{equation}
	Putting \eqref{Eq1607BrackUnMu} and \eqref{1607BrackDeuxMu} into the brackets of \eqref{Eq1607LCCLCLC} we find
	\begin{equation}
		\begin{aligned}[]
			L^*_{(b_2,b_1)}C^2_k(a,b)C^1_l(u,v)&=\mu^{21} C^2(L^*_{b_2}a,L^*_{b_2}b)\otimes C^1_l(A^*u,A^*v)\\
			&=\mu^{21}\circ(C^2_k\otimes C^1_l)(L^*_{b_2}a\otimes L^*_{b_2}b)\otimes(A^*u\otimes A^*v)\\
		\end{aligned}
	\end{equation}
	If we make the sum $\sum_{k_1+k_2=k}$ of the latter equation, we find
	\begin{equation}
		\sum_{k_1+k_2=k}C_k\Big( (L^*_{b_2}a\otimes A^*u),(L^*_{b_2}b\otimes A^*v) \Big)=\sum_{k_1+k_2=k}C_k\Big( L^*_{(b_2,b_1)}(a\otimes u),L^*_{(b_2,b_1)}(b\otimes v) \Big).
	\end{equation}
\end{proof}

This was the extension lemma from the point of view of the formal star product. From the point of view of the Drinfel'd twist, we have the following.

\begin{proposition}
	Let $F^{(j)}$ be formal twists based on $\mU(\lB_j)\dcr{h}$ with
	\begin{equation}
		F^{(1)}\in\Big( \mU(\lB_1)\dcr{h}\otimes\mU(\lB_1)\dcr{h} \Big)^{\lB_2}.
	\end{equation}
	Then we define $F\in\mU(\lB)\dcr{h}\otimes\mU(\lB)\dcr{h}$ by
	\begin{equation}
		F=F^{(2)}\otimes F^{(1)}.
	\end{equation}
	This is a Drinfel'd twist based on $\mU(\lB)\dcr{h}$.
\end{proposition}
The article about deformation philosophy by Flato is \cite{FlatoDeforView}.
