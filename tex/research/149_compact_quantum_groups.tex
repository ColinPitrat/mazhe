% This is part of (almost) Everything I know in mathematics and physics
% Copyright (c) 2013-2014
%   Laurent Claessens
% See the file fdl-1.3.txt for copying conditions.

%+++++++++++++++++++++++++++++++++++++++++++++++++++++++++++++++++++++++++++++++++++++++++++++++++++++++++++++++++++++++++++
\section{Definitions}
%+++++++++++++++++++++++++++++++++++++++++++++++++++++++++++++++++++++++++++++++++++++++++++++++++++++++++++++++++++++++++++

Literature : \cite{CompactQuantumGpWoro,Ritter,Kustermans,Koelink} and Wikipedia: \wikipedia{en}{Quantum_group}{quantum group} and \wikipedia{en}{Compact_quantum_group}{compact quantum group}. For introductions about Hopf algebra and some related topics such as quantum groups and deformation quantization of (co)-Poisson structures, see \cite{Tjin,MaximeRey}.

\begin{definition}
    A \defe{compact quantum group}{quantum!group!compact}\index{compact!quantum group} is a pair $G=(A,\Phi)$ where $A$ is an unital separable $C^*$-algebra and $\Phi\colon A\to A\otimes A$ is a $*$-homomorphism such that
    \begin{enumerate}
        \item
            $(\id\otimes\id)\Phi=(\id\otimes\mtu)\Phi$;
        \item
            The sets
            \begin{equation}
                \begin{aligned}[]
                    \{ (b\otimes\mtu)\Phi(c)\tq b,c\in A \},\\
                    \{ (\mtu\otimes b)\Phi(c)\tq b,c\in A \}
                \end{aligned}
            \end{equation}
            are linearly dense in $A\otimes A$.
    \end{enumerate}
\end{definition}

%---------------------------------------------------------------------------------------------------------------------------
\subsection{Example: representation of groups}
%---------------------------------------------------------------------------------------------------------------------------

If $u$ is a finite-dimensional representation of a group $G$, we can build the $*$-subalgebra of $C(G)$ generated by the elements $u_{ij}$ and $\alpha(u_{ij})$ where $\alpha$ is defined by
\begin{equation}
    \alpha(u_{ij})(g)=u_{ij}(g^{-1}).
\end{equation}
Using the definition \ref{DefHopfsurCG}, we turn this algebra into a Hopf algebra\footnote{This is not trivial since nothing guarantee \emph{a priori} that the right hand sides of the definitions are elements of our small algebra.}. Let us particularize the definitions. For the coproduct we have
\begin{equation}
    \Delta(u_{ij})=\sum_k u_{ik}\otimes u_{kj}
\end{equation}
because if $g$ and $h$ are elements of $G$, the coproduct on $C(G)$ is equal to
\begin{equation}
    \Delta(u_{ij})(g,h)=u_{ij}(gh)=\big( u(gh) \big)_{ij}=\big( u(g)u(h) \big)_{ij}=\sum_k u(g)_{ik}u(h)_{kj}=\Big( \sum_k u_{ik}\otimes u_{kj}\Big)(g,h)
\end{equation}
where we used the fact that $u$ is a representation and the fact that, by definition, $u_{ij}(g)=u(g)_{ij}$. The counit is given by
\begin{equation}
    \epsilon(u_{ij})=\delta_{ij}.
\end{equation}
since the neutral element $e$ is always represented by the unit matrix. The antipode is 
\begin{equation}
    \alpha(u_{ij})(g)=u_{ij}(g^{-1}).
\end{equation}

The unit is given by the constant function $1$. Fortunately, it turns out that this function is a combination of the functions $u_{ij}$ :
\begin{equation}
    1=\sum_k u_{1k}\alpha(u_{k1})=\sum_k\alpha(u_{1k})u_{k1}.
\end{equation}
Indeed,
\begin{equation}
    \begin{aligned}[]
        \sum_ku_{1k}(g)\alpha(u_{k1})(g)&=\sum_ku_{1k}(g)u_{k1}(g^{-1})  \\
        &=\Big( u(g)u(g^{-1}) \Big)_{11}\\
        &=u(e)_{11}\\
        &=1.
    \end{aligned}
\end{equation}

Let us check one of the diagrams \eqref{EqDiagUnitHopf}:
\begin{equation}
    \xymatrix{%
    \eC\otimes C(G)     &   C(G)\otimes C(G)\ar[l]_{\epsilon\otimes\id}\\
    &   C(G)\ar[ul]^{\psi}\ar[u]_{\Delta}
       }
\end{equation}
We have
\begin{equation}
    \begin{aligned}[]
        (\epsilon\otimes\id)\Delta(u_{ij})&=(\epsilon\otimes\id)\sum_k u_{ik}\otimes u_{kj}\\
        &=\sum_k\delta_{ik}\otimes u_{kj}\\
        &=\sum_k1\otimes \delta_{ik}u_{kj}\\
        &=1\otimes u_{ij}\\
        &=\psi(u_{ij}).
    \end{aligned}
\end{equation}

%---------------------------------------------------------------------------------------------------------------------------
\subsection{Matrix quantum group}
%---------------------------------------------------------------------------------------------------------------------------

\begin{definition}      \label{DefQuantumMatrixGroup}
    A \defe{matrix quantum group}{matrix!quantum group}\index{quantum!group!matrix} is a pair $(C,u)$ where $C$ is a $C^*$-algebra and $u=(u_{ij})$ is a matrix with entries in $C$ such that
    \begin{enumerate}
        \item
            The $*$-subalgebra $C_0$ generated by the elements $u_{ij}$ is dense in $C$.
        \item
            There exists a $C^*$-algebra homomorphism $\Delta\colon C\to C\otimes C$ such that
            \begin{equation}
                \Delta(u_{ij})=\sum_k u_{ik}\otimes u_{kj}.
            \end{equation}
        \item\label{DefQuantumMatrixGroupItemiii}
            There exists an antimultiplicative map $\alpha\colon C_0\to C_0$ called \emph{coinverse} such that
            \begin{enumerate}
                \item
                    $\alpha\big( \alpha(v^*)^* \big)=v$ for any $v\in C_0$,
                \item
                    $\sum_k\alpha(u_{ik})u_{kj}=\sum_ku_{ik}\alpha(u_{kj})=\delta_{ij}\mtu$ where $\mtu$ is the unit in $C$.
            \end{enumerate}
    \end{enumerate}
\end{definition}

%///////////////////////////////////////////////////////////////////////////////////////////////////////////////////////////
\subsubsection{Example: \texorpdfstring{$\SU_q(2)$}{SU2}}
%///////////////////////////////////////////////////////////////////////////////////////////////////////////////////////////

As an example, we give the quantum matrix group $\SU_q(2)$. Let $q$ be a positive real number. As $C^*$-algebra, $\SU_q(2)$ is generated by the elements $a$ and $b$ and the relations\footnote{We follow the notations of \cite{DiracSUq}.}
\begin{equation}        \label{EqDefAlfSUab}
    \begin{aligned}[]
        ba&=qab     \\  b^*a&=qab^*\\
        bb^*&=b^*b  \\  a^*a+q^2b^*b&=1\\
        aa^*+bb^*&=1
    \end{aligned}
\end{equation}
We build a compact matrix quantum group $\Big( C\big( \SU_q(2) \big),u \Big)$ where
\begin{equation}
    u=\begin{pmatrix}
        a   &   qb  \\ 
        -b^*    &   a^* 
    \end{pmatrix}.
\end{equation}
The Hopf-$*$-algebra structure on $C\big( \SU_q(2) \big)$ is given by the relations in the definition \ref{DefQuantumMatrixGroup}. We have to check that these are compatible. 

We have
\begin{equation}        \label{EqDelaSU}
    \Delta(a)=\Delta(u_{11})=\sum_k u_{1k}\otimes u_{k1}=u_{11}\otimes u_{11}+u_{12}\otimes u_{21}=a\otimes a-qb\otimes b^*,
\end{equation}
and
\begin{equation}
    \Delta(a^*)=\Delta(u_{22})=\sum_k u_{2k}\otimes u_{k2}=u_{21}\otimes u_{12}+u_{22}\otimes u_{22}=-qb^*\otimes b+a^*\otimes a^*,
\end{equation}
which is compatible with \eqref{EqDelaSU}. Doing the same with $b=\frac{1}{ q }u_{12}$ (and checking the compatibility with $b^*$), we find the coproduct
\begin{equation}
    \begin{aligned}[]
        \Delta(a)&=a\otimes a-qb\otimes b^*\\
        \Delta(b)&=b\otimes a^*+a\otimes b.
    \end{aligned}
\end{equation}
The counit $\epsilon(u_{ij})=\delta_{ij}$ provides
\begin{equation}
    \begin{aligned}[]
        \epsilon(a)&=1\\
        \epsilon(b)&=0.
    \end{aligned}
\end{equation}

Using the relations \eqref{EqDefAlfSUab}, we can check that the condition \ref{DefQuantumMatrixGroupItemiii} of definition \ref{DefQuantumMatrixGroup} are satisfied by the antipode
\begin{equation}
    \begin{aligned}[]
        \alpha(a)&=a^*,     &   \alpha(a^*)&=a,\\
        \alpha(b)&=-qb,     &   \alpha(b^*)&=-q^{-1}b^*.
    \end{aligned}
\end{equation}
As an example,
\begin{equation}
    \alpha\big( \alpha(b^*)^* \big)=\alpha\big( (-\frac{1}{ q }b^*)^* \big)=-\frac{1}{ q }\alpha(b)=b
\end{equation}
and $\sum_k\alpha(u_{ik})u_{k2}=\delta_{12}=0$ since
\begin{equation}
    \alpha(u_{11})u_{12}+\alpha(u_{12})u_{22}=\alpha(a)qb+q\alpha(b)a^*=qa^*b-q^2ba^*=0
\end{equation}
because $a^*b=(b^*a)^*=(qab^*)^*=qba^*$.

All this structure provides a Hopf $*$-algebra named $C\big(\SU_q(2))$\nomenclature[A]{$C\big( \SU_q(2) \big)$}{the quantum matrix group}.

%///////////////////////////////////////////////////////////////////////////////////////////////////////////////////////////
\subsubsection{Corepresentation}
%///////////////////////////////////////////////////////////////////////////////////////////////////////////////////////////

A \defe{corepresentation}{corepresentation} of the quantum group $(A,\Delta)$ is an element $u\in\eM_n(A)$ such that
\begin{equation}
    \Delta(u_{kl})=\sum_{j=1}^nu_{kj}\otimes u_{jl}
\end{equation}
for every $k,l\in\{ 1,\ldots n \}$.

%---------------------------------------------------------------------------------------------------------------------------
\subsection{Haar measure on compact quantum groups}
%---------------------------------------------------------------------------------------------------------------------------

Let $A$ be the $C^*$-algebra $C(G)$ where $G$ is a compact group. The \defe{Haar measure}{Haar!measure} $\mu$ on $G$ is the unique regular Borel measure on $G$ such that $\mu(G)=1$ and
\begin{equation}
    \int_G f(st)d\mu(s)=\int_G f(ts)d\mu(s)=\int_Gf(s)d\mu(s).
\end{equation}
We can see the Haar measure $\mu$ as a positive functional $\varphi\colon A\to \eC$,
\begin{equation}
    \varphi(f)=\int_G f(s)d\mu(s).
\end{equation}
If $h\in A$ is such that $\Delta(h)=h_{(1)}\otimes h_{(2)}$ (sum implied), then we have
\begin{equation}
    \Delta(h)(s,t)=h(st)=\big( h_{(1)}\otimes h_{(2)} \big)(s,t)=h_{(1)}(s)h_{(2)}(t).
\end{equation}
Thus we have
\begin{equation}
    \int_G\Delta(h)(s,t)d\mu(s)=\int_Gh_{(1)}(s)h_{(2)}(t)d\mu(s)=(\varphi\otimes\id)\big( h_{(1)}\otimes h_{(2)} \big)(t)=(\varphi\otimes\id)\Delta(h).
\end{equation}
Using the invariance of the Haar measure, this is also equal to
\begin{equation}
    \begin{aligned}[]
        \int_G\Delta(h)(s,t)d\mu(s)&=\int_Gh(st)d\mu(s)\\
        &=\int_Gh(ts)d\mu(s)\\
        &=\int_Gh_{(1)}(t)h_{(2)}(s)d\mu(s)\\
        &=(\id\otimes\varphi)\big( h_{(1)}\otimes h_{(2)} \big)(t)\\
        &=(\id\otimes\varphi)\Delta(h).
    \end{aligned}
\end{equation}
Since $C(G)\otimes CC(G)$ is dense in $C(G\times G)$, we have
\begin{equation}
    (\id\otimes\varphi)\Delta(h)=(\varphi\otimes\id)\Delta(h)=\varphi(h)1
\end{equation}
for every $h\in A$.

The generalisation of that result to an arbitrary compact quantum group is the following theorem.
\begin{theorem}
    If $(A,\Phi)$ is a compact quantum group, there exists one and only one state $\phi\colon A\to \eC$ such that
    \begin{equation}
        (\id\otimes\phi)\Phi(a)=\phi(a)1=(\phi\otimes\id)\Phi(a)
    \end{equation}
    for every $a\in A$.
\end{theorem}
The unique state guaranteed by that theorem is called the \defe{Haar state}{Haar!state} of $A$. If $\omega$ and $\omega'$ are linear functionals on $A$, we define the \defe{convolution}{convolution!on compact quantum groups} by
\begin{equation}        \label{DefProdConvCQG}
    \begin{aligned}[]
        \omega * a&=(\id\otimes\omega)\Phi(a)\\
        a*\omega &=(\omega\otimes\id)\Phi(a)\\
        \omega * \omega'&=(\omega\otimes\omega')\Phi(a).
    \end{aligned}
\end{equation}
The first two are elements of $A$ when the lase one is a linear functional on $A$. Remember that here the map $\Phi$ plays the role of a coproduct; when the context is about a coalgebra, we need to adapt the notation and write $\Delta$ instead of $\Phi$ in the definitions \ref{DefProdConvCQG}.

The Haar state is the state $\phi$ such that
\begin{equation}
    \phi*a=a*\phi=\phi(a)1.
\end{equation}


%+++++++++++++++++++++++++++++++++++++++++++++++++++++++++++++++++++++++++++++++++++++++++++++++++++++++++++++++++++++++++++
\section{Construction of \texorpdfstring{$\SU_q(n)$}{SUqn}}
%+++++++++++++++++++++++++++++++++++++++++++++++++++++++++++++++++++++++++++++++++++++++++++++++++++++++++++++++++++++++++++
\label{SecGeneratorsonSUQn}

In order to build the quantum group $\SU_q(n)$, we will proceed the following step\cite{Koelink}
\begin{enumerate}
    \item
        we consider the $n^2$ elements $U=(u_{ij})$ and we name $\suqA_q$ the algebra generated by these elements;
    \item
        we impose some relations;
    \item
        we define the determinant $D$;
    \item
        we impose $D=1$;
    \item\label{ItemQGSUqnC}
        we define the involution $u_{ij}\mapsto u_{ij}^*$;
    \item
        we impose $UU^*=U^*U=1$.
        
\end{enumerate}
Notice that the point \ref{ItemQGSUqnC} will not introduce new elements $u_{ij}^*$. It will define $u_{ij}^*$ as combination of the $u_{ij}$'s.

The first relations that we impose are
\begin{equation}        \label{EqRelsSUqnAvecR}
    \sum_{kl}R^{kl}_{ij}u_{km}u_{lp}=\sum_{kl}R^{mp}_{kl}u_{ik}u_{jl}
\end{equation}
where $R\in\eM_{n^2}(\eR)$ is the matrix given by
\begin{equation}
    \begin{aligned}[]
        R_{ii}^{ii}&=q^{-1}&R_{ij}^{ji}&=1\text{ if }i\neq j\\
        R_{ij}^{ij}&=q^{-1}-q\text{ if }i>j\\
        R_{ij}^{kl}&=0\text{ otherwise}.
    \end{aligned}
\end{equation}

\begin{remark}
    In \cite{Bragiel}, there is an alternative compact form for the same constraints.
\end{remark}

\begin{proposition}
    \begin{subequations}        \label{SUBEquuijcondiv}
        \begin{align}
            u_{ij}u_{il}&=qu_{il}u_{ij}&\text{if }j<l         \label{subEquuijcondi} \\
            u_{ij}u_{kj}&=qu_{kj}u_{ij}&\text{if }i<k\label{subEquuijcondii}\\
            u_{ij}u_{kl}&=u_{kl}u_{ij}&\text{if }i>k,j<l\label{subEquuijcondiii}\\
            u_{ij}u_{kl}-u_{kl}u_{ij}&=(q-q^{-1})u_{il}u_{kj}&\text{if }i<k,j<l\label{subEquuijcondiv}.
        \end{align}
    \end{subequations}
\end{proposition}

\begin{proof}
    Let us check the relation \eqref{subEquuijcondiii}. If we consider \( i,j,m,p\) with \( i<j\) and \( m<p\), the only non vanishing \( R^{kl}_{ij}\) is \( k=j\), \( l=i\) and the only non vanishing \( R^{mp}_{kl}\) is \( k=p\), \( l=m\). Thus we immediately get \( u_{jm}u_{ip}=u_{ip}u_{jm}\).
\end{proof}


A particular consequence of \eqref{subEquuijcondiii} is that
\begin{equation}
    u_{ij}u_{ji}=u_{ji}u_{ij};
\end{equation}
an element commutes with its ``transposed''.

\begin{lemma}       \label{lestmunijdiff}
    If $l=\min(i,j)$ with $i\neq j$ then
    \begin{equation}
        u_{ij}u_{ll}=q^{-1}u_{ll}u_{ij}.
    \end{equation}    
\end{lemma}

\begin{proof}
    If $l=i$ then $l<j$ and the relation \eqref{subEquuijcondi} provides $u_{lj}u_{ll}=q^{-1}u_{ll}u_{lj}$. Now if $l=j$, then $l<i$ and the relation \eqref{subEquuijcondii} provides $u_{il}u_{ll}=q^{-1}u_{ll}u_{il}$.
\end{proof}


\begin{lemma}       \label{lestmaxjdiff}
    If $l=\max(i,j)$ with $i\neq j$ then
    \begin{equation}
        u_{ij}u_{ll}=qu_{kk}u_{ij}.
    \end{equation}
\end{lemma}

\begin{proof}
    If $l=i$, we have $j<i$ and the relation \eqref{subEquuijcondi} leads to $u_{li}u_{ll}=qu_{ll}u_{li}$. If $l=j$, then $i<j$ and the relation \eqref{subEquuijcondii} shows $u_{il}u_{ll}=qu_{ll}u_{il}$.
\end{proof}


The $R$ matrix is a Yang-Baxter\index{Yang-Baxter} operator:
\begin{equation}
    R_{(12)}R_{(23)}R_{(12)}=R_{(23)}R_{(12)}R_{(23)}
\end{equation}
where the operators $R_{(12)},R_{(23)}\colon \eC\otimes\eC\otimes\eC\to \eC\otimes\eC\otimes\eC$ are defined by
\begin{equation}
    \begin{aligned}[]
        R_{(12)}&=R\otimes \id\\
        R_{(23)}&=\id\otimes R.
    \end{aligned}
\end{equation}

Now we define the homomorphisms
\begin{equation}
    \begin{aligned}
        \Delta\colon \mA_q&\to \mA_q\otimes\mA_q \\
        \Delta(u_{ij})&\mapsto \sum_{k=1}^nu_{ik}\otimes u_{kj}
    \end{aligned}
\end{equation}
and
\begin{equation}
    \begin{aligned}
        \epsilon\colon \mA_q&\to \eC \\
        \epsilon(u_{ij})&=\delta_{ij}. 
    \end{aligned}
\end{equation}
We also set \( \epsilon(1)=1\) as required for a counit (definition \ref{DefBialgebra}). At this point $\mA_q$ is a bialgebra.


%---------------------------------------------------------------------------------------------------------------------------
\subsection{Determinant}
%---------------------------------------------------------------------------------------------------------------------------

We introduce the \defe{quantum determinant}{quantum!determinant}
\begin{equation}
    D=\sum_{\sigma\in S_n}(-q)^{| \sigma |}u_{1\sigma(1)}\ldots u_{n\sigma(n)}
\end{equation}
where $| \sigma |$ is the length of the permutation. We also introduce the \defe{quantum minor}{quantum!minor} $D^{ij}$
\begin{equation}        \label{Eqdefdijmineurquan}
    D^{ij}=\sum_{\sigma\colon \{ 1,\ldots,\hat\imath,\ldots,n \}\to \{ 1,\ldots,\hat\jmath,\ldots,n \}}(-q)^{| \sigma |}u_{1\sigma(1)}\ldots\widehat{u_{i\sigma(i)}}\ldots u_{n\sigma(n)}.
\end{equation}

Let $I$ and $J$ be subsets of size $t$ of $\{ 1,\ldots,n \}$. We introduce the \defe{quantum minor}{quantum!minor}
\begin{equation}
    [I|J]=\sum_{\sigma\in S_t}(-q)^{| \sigma |} u_{i_{\sigma(1),j_1}}\ldots u_{i_{\sigma(t),j_t}}.
\end{equation}

The coproduct of that minor has the simple expression\cite{Goodearl}
\begin{proposition}     \label{LemMineurQuantique}
    We have
    \begin{equation}
        \Delta[I|J]=\sum_{| K |=| I |}[I|K]\otimes[K|I]
    \end{equation}
    where the sum is over all the subset $K\subseteq\{ 1,\ldots,n \}$ of size equal to the size of $I$.
\end{proposition}

\begin{proof}[Sketch of the proof]
    
    We prove by induction on the size of $I$. Using the fact that $\Delta$ is an homomorphism,
    \begin{equation}
        \begin{aligned}[]
            \Delta[I|J]&=\sum_{\sigma\in S_t}(-q)^{| \sigma |}\sum_{k=1}^nu_{i_{\sigma(1)}k}\otimes u_{kj_1}\Delta\big( u_{i_{\sigma(2)}j_2}\ldots u_{i_{\sigma(n)}j_n} \big)\\
            &=\sum_{l=1}^t\sum_{\sigma\in S_t(l)}(-q)^{| \sigma |}\sum_{k=1}^nu_{lk}\otimes u_{kj_1}\Delta\big( u_{i_{\sigma(2)}j_2}\ldots u_{i_{\sigma(n)}j_n} \big)\\
            &=\sum_{l=1}^t\sum_{\sigma\in S_t(l)}(-q)^{| \sigma |}\sum_{k=1}^nu_{lk}\otimes u_{kj_1}\sum_{| K |=| I |-1}[I\setminus\{ l \}|K][K|J\setminus\{ j_1 \}]
        \end{aligned}
    \end{equation}
    where $S_t(l)$ is the subset of $S_t$ of permutations such that $\sigma(1)=l$.

    \begin{probleme}
        This proof is not finished. I think that there remain a few combinatorial work in order to get the result. At least it looks very like everything is going to sum up correctly.
    \end{probleme}
\end{proof}

Using lemma \ref{LemMineurQuantique}, we prove that
\begin{equation}
    \Delta(D)=D\otimes D.
\end{equation}
We also have $\epsilon(D)=1$.

Now we add to $\mA_q$ an element called $D^{-1}$ on which we impose the conditions
\begin{equation}
    DD^{-1}=D^{-1}D=1.
\end{equation}
We extend $\Delta$ and $\epsilon$ by
\begin{equation}
    \begin{aligned}[]
        \Delta(D^{-1})&=D^{-1}\otimes D^{-1}\\
        \epsilon(D^{-1})&=1.
    \end{aligned}
\end{equation}

The extension of $\mA_q$ by $D^{-1}$ is $\Pol\big( \gU_q(n) \big)$, or $\Pol\big( \SU_q(n) \big)$ if we impose the extra relation $D=1$.

For the involution we define $\kappa$ by
\begin{equation}        \label{EqDefInvolutionSSUqn}
    \begin{aligned}[]
        \kappa(u_{ij})=(-q)^{i-j}D^{ji}D^{-1}\\
        \kappa(D^{-1})=D.
    \end{aligned}
\end{equation}
This extends in an unique way to $\kappa\colon \Pol\big( \gU_q(n) \big)\to \Pol\big( \gU_q(n) \big)$ in such a way that $\Pol\big( \gU_q(n) \big)$ becomes a Hopf algebra with $\kappa$ as antipode.

We introduce the involution by
\begin{equation}
    \begin{aligned}[]
        u_{ij}^*&=\kappa(u_{ji})\\
        (D^{-1})^*&=D.
    \end{aligned}
\end{equation}
We have $DD^*=1=D^*D$.

\begin{lemma}
    We have \( \epsilon(u_{ij}^*)=\delta_{ij}\).
\end{lemma}

\begin{proof}
    If we apply \( \epsilon\) on the definition \eqref{Eqdefdijmineurquan} we get a non vanishing contribution only from the term \( \sigma=\id\). Thus we see that \( \epsilon(D^{ij})=\delta_{ij}\) and the result follows immediately.
\end{proof}
Notice that this result is part of the fact that \( \Pol(U_q(n))\) is an Hopf algebra with involution (see lemma \ref{LemcounitstarHopfalg}).


Using the formula
\begin{equation}
    \delta_{ij}D=\sum_{k=1}^n(-q)^{k-j}u_{ik}D^{jk},
\end{equation}
one proves that
\begin{equation}        \label{Equustrunsuqn}
    \sum_{k=1}^nu_{ik}u^*_{jk}=\delta_{ij}1=\sum_{k=1}^nu_{ki}^*u_{kj}.
\end{equation}

\begin{lemma}
    In \( \SU_q(n)\) we have the relations\cite{Koelink,Bragiel}
    \begin{subequations}
        \begin{align}
        u^*_{kl}u_{ij}&=u_{ij}u^*_{kl}&\text{if }k\neq i\label{eqREflsuusikl}\\
        u_{ij}^*u_{kj}&=q^{-1}u_{kj}u^*_{ij}+(q^{-1}-q)\sum_{s<j}u_{ks}u^*_{is}&\text{if }i\neq k     \label{subequkluijeqkneqiuust}\\
        u^*_{ij}u_{ik}&=qu_{ik}u_{ij}^*+(q^2-1)\sum_{s>i}u^*_{sj}u_{sk}&\text{if }j\neq k\\
        u^*_{ij}u_{ij}&=u_{ij}u_{ij}^*+(q^2-1)\sum_{s>i}u^*_{sj}u_{sj}+(1-q^2)\sum_{s<j}u_{is}u^*_{is}\label{EqLemsumsumstarpar}
        \end{align}
    \end{subequations}
    where \( i,j,k,l\in\{ 1,2,\ldots,n \}\) and the sums are running up to \( n\).
\end{lemma}

\begin{proof}
    We are starting from equation \eqref{EqRelsSUqnAvecR} that we multiply on the left by \( u^*_{ir}\), on the right by \( u^*_{sp}\) and sum over \( i\) and \( p\):
    \begin{equation}
        \sum_{pikl}R^{kl}_{ij}u^*_{ir}u_{km}u_{lp}u^*_{sp}=\sum_{irkl}R^{mp}_{kl}u^*_{ir}u_{ik}u_{jl}u^*_{sp}.
    \end{equation}
    Using \eqref{Equustrunsuqn}, the sum over \( p\) is easy to perform on the left hand side while the sum over \( i\) is easy to perform on the right hand side:
    \begin{equation}
        \sum_{ikl}R^{kl}_{ij}u^*_{ir}u_{km}\delta_{ls}=\sum_{pkl}R^{mp}_{kl}\delta_{rk}u_{jl}u^*_{sp}.
    \end{equation}
    For the sake of unifying the notations, we rename the summation variable \( p\) to \( i\) in the right hand side:
    \begin{equation}
        \sum_{ik}R^{ks}_{ij}u^*_{ir}u_{km}=\sum_{il}R^{mi}_{rl}u_{jl}u^*_{si}.
    \end{equation}
    At this point we have four possibilities following \( s=j\) or \( s\neq j\) and \( m=r\) or not. We look at the case \( s=j\), \( m=r\). In the expression
    \begin{equation}
        \sum_{il}R^{mi}_{ml}u_{sl}u^*_{si},
    \end{equation}
    only the term \( i=l\) is not zero. Thus the equality reduces to
    \begin{equation}
        \sum_kR_{ks}^{ks}u^*_{km}u_{km}=\sum_iR^{mi}_{mi}u_{si}u^*_{si}.
    \end{equation}
    We divide the sum over \( k\) into \( k=s\) and \( k\neq s\) and the sum over \( i\) into \( i=m\) and \( i\neq m\):
    \begin{equation}
        R_{ss}^{ss}u^*_{sm}u_{sm}+\sum_{j>s}R^{ks}_{ks}u^*_{km}u_{km}=R_{mm}^{mm}u_{sm}u^*_{sm}+\sum_{i<m}R_{mi}^{mi}u_{si}u^*_{si}.
    \end{equation}
    Changing the names \( s\to i\) and \( m\to j\) we have
    \begin{equation}
        q^{-1}u^*_{ij}u_{ij}=u_{ij}u_{ij}^*+(q^2-1)\sum_{k>i}u^*_{kj}u_{kj}+(1-q^2)\sum_{k<j}u_{ik}u^*u_{ik}.
    \end{equation}
    This proves the equality \eqref{EqLemsumsumstarpar}.
\end{proof}

\begin{probleme}
    For the other ones, I think that one has to work with the other possibilities about \( s=j\) and \( m=r\). To be checked.
\end{probleme}

The quantum group $\SU_q(n)$ is the unital $C^*$-algebra obtained from the completion of $\mA_q$ on which the map $\Delta$ extends to a $C^*$-homomorphism
\begin{equation}
    \Delta\colon \SU_q(n)\to \SU_q(n)\otimes\SU_q(n).
\end{equation}

\begin{probleme}
    Is the completion with that requirement unique ?
\end{probleme}

%---------------------------------------------------------------------------------------------------------------------------
\subsection{Norm}
%---------------------------------------------------------------------------------------------------------------------------

\begin{proposition}     \label{Propqpiideinve}
    If $\pi$ is a representation of $\mA_q$ on $D$, then $\| \pi(u_{ij}) \|^2\leq 1$.
\end{proposition}

\begin{proof}
    Using relation \eqref{Equustrunsuqn} we have for every $f\in D$ that
    \begin{equation}
        \begin{aligned}[]
            \| f \|^2&=\langle f, f\rangle \\
            &=\langle \sum_k\pi(u_{ik})^*\pi(u_{ik})f, f\rangle \\
            &=\sum_k\langle \pi(u_{ik})f, \pi(u_{ik})f\rangle \\
            &=\sum_k\| \pi(u_{ik})f \|^2.
        \end{aligned}
    \end{equation}
    If $\| f \|=1$, we thus have $\| \pi(u_{ik}) \|\leq 1$ and
    \begin{equation}
        \| \pi(u_{ik}) \|^2=\sup_{\| f \|=1}\| \pi(u_{ik})f \|^2\leq 1.
    \end{equation}
\end{proof}

\begin{corollary}       \label{CorOpOdquijInverti}
    The operator $\id-q\pi(u_{ij})$ is invertible.
\end{corollary}

\begin{proof}
    Since $q<1$ and $\| \pi(u_{ij}) \|\leq 1$, it is impossible to have $\big[ \id-q\pi(u_{ij}) \big]f=0$ without violating $\| \pi(u_{ij}) \|\leq 1$.
\end{proof}

%+++++++++++++++++++++++++++++++++++++++++++++++++++++++++++++++++++++++++++++++++++++++++++++++++++++++++++++++++++++++++++
\section{Representations of $\SU_q(3)$}
%+++++++++++++++++++++++++++++++++++++++++++++++++++++++++++++++++++++++++++++++++++++++++++++++++++++++++++++++++++++++++++

We are following \cite{Bragiel}.

\begin{lemma}       \label{Lemxijxstijsuqt}
    In \( \SU_q(3)\) we have the relations
    \begin{subequations}
        \begin{align}
            u_{11}u^*_{11}&=(1-q^2)+q^2u_{11}^*u_{11}       \label{subequustuuuutroisi}\\
            u_{12}u_{12}^*&=q^2(1-q^2)+q^2u_{12}^*u_{12}-q^2(1-q^2)u_{11}^*u_{11}   \label{subequustuuuutroisii}\\
            u_{21}u_{21}^*&=(1-q^2)+q^2u_{21}^*u_{21}-(1-q^2)u_{11}^*u_{11}\\
            u_{22}u_{22}^*&=(1-q^2)+q^2u_{22}^*u_{22}-(1-q^2)(u_{21}^*u_{21}+u_{12}^*u_{12})-(1-q^2)^2u_{31}^*u_{31}.
        \end{align}
    \end{subequations}
\end{lemma}

\begin{proof}
    Let us set \( i=j=1\) in \eqref{EqLemsumsumstarpar}. The second sum vanishes while we reform the first sum by adding and substracting \( (q^2-1)u_{11}^*u_{11}\). What we obtain is
    \begin{equation}
        u_{11}^*u_{11}=u_{11}u_{11}^*+(q^2-1)\underbrace{\sum_ku_{k1}^*u_{k1}}_{=1}-(q^2-1)u_{11}^*u_{11}.
    \end{equation}
    So
    \begin{equation}
        q^2u_{11}^*u_{11}=u_{11}u_{11}^*+(q^2-1)
    \end{equation}
    which is the relation \eqref{subequustuuuutroisi}.

    In order to prove the relation \eqref{subequustuuuutroisii} we start from
    \begin{equation}
        u_{12}u_{12}^*=u_{12}^*u_{12}-(q^2-1)(\underbrace{u_{22}^*u_{22}+u_{32}^*u_{32}}_{=1-u_{12}^*u_{12}})-(1-q^2)u_{11}u_{11}^*
    \end{equation}
    in which we substitute the value of \( u_{11}u_{11}^*\) given by \eqref{subequustuuuutroisi}.

    \begin{probleme}
        I guess this is the same for the other relations, setting other values for \( i\) and \( j\).
    \end{probleme}
\end{proof}

\begin{lemma}       \label{Lemuijustijnstarnsubeqijuk}
    In \( \SU_q(3)\) we have the relations
    \begin{subequations}
        \begin{align}
            u_{11}(u_{11}^*)^n&=(1-q^{2n})(u_{11}^*)^{n-1}+q^{2n}(u_{11}^*)^nu_{11}     \label{subeqlemuijuklstarni}\\
            u_{12}(u_{12}^*)^n&=q^2(1-q^{2n})(u_{12}^*)^{n-1}+q^{2n}(u_{12}^*)^nu_{12}-q^2(1-q^{2n})(u_{12}^*)^{n-1}u_{11}^*u_{11}\\
            u_{21}(u_{21}^*)^n&=(1-q^{2n})(u_{21}^*)^{n-1}+q^{2n}(u_{21}^*)^nu_{21}-(1-q^{2n})(u_{21}^*)^{n-1}u_{11}^*u_{11}\\
            u_{22}(u_{12}^*)^n&=q^n(u_{12}^*)^nu_{22}-q^2\frac{ 1-q^{2n} }{ q^n }(u_{12}^*)^{n-1}u_{11}^*u_{21}     \label{subeqlemuijuklstarnivd}\\
            u_{22}(u_{21}^*)^n&=q^n(u_{21}^*)^nu_{22}-\frac{ 1-q^{2n} }{ q^n }(u_{21}^*)^{n-1}u_{11}^*u_{12}\\
            u_{23}(u_{21}^*)^n&=q^n(u_{21}^*)^nu_{23}-\frac{ 1-q^{2n} }{ q^n }(u_{21}^*)^{n-1}u_{11}^*u_{13}.
        \end{align}
    \end{subequations}
    
\end{lemma}

\begin{proof}
    Equation \eqref{subeqlemuijuklstarni} is checked by setting \( i=j=1\) in equation \eqref{EqLemsumsumstarpar}. 

    Let us now check the equation \eqref{subeqlemuijuklstarnivd}. First, writing \eqref{subequkluijeqkneqiuust} with \( i=j=l=1\), \( k=2\)  and taking the adjoint produces
    \begin{equation}        \label{Equduuuuqstarlem}
        u_{21}u_{11}^*=qu_{11}^*u_{21}.
    \end{equation}
    Now if we write equation \eqref{subequkluijeqkneqiuust} with \( k=1\) and \( l=i=j=2\) we have
    \begin{equation}        \label{equdduusstarlem}
        u_{22}u_{12}^*=qu_{12}^*u_{22}-(1-q^2)u_{21}u_{11}^*.
    \end{equation}
    Substituting \eqref{Equduuuuqstarlem} into \eqref{equdduusstarlem} we get
    \begin{equation}
        u_{22}u_{12}^*=qu_{12}^*u_{22}-q(1-q^2)u_{11}^*u_{21}
    \end{equation}
    which is \eqref{subeqlemuijuklstarnivd} with \( n=1\). We proceed now by induction over \( n\). Using the relations
    \begin{subequations}
        \begin{align}
            u_{22}u_{12}^*&=qu_{12}^*u_{22}-(1-q^2)u_{21}u_{11}^*\\
            u_{21}u_{12}^*&=u_{12}^*u_{21}
        \end{align}
    \end{subequations}
    we find
    \begin{equation}
        \begin{aligned}[]
            u_{22}(u_{12}^*)^{n+1}&=\big[ q^n(u_{12}^*)^nu_{22}-q^{2-n}(1-q^{2n})(u_{12}^*)^{n-1}u_{11}^*u_{21} \big]u_{12}^*\\
            &=q^{n+1}(u_{12}^*)^{n+1}u_{22}-q^n(1-q^2)(u_{12}^*)^nu_{21}u_{11}^*-q^{2-n}(1-q^{2n})(u_{12}^*)^{n-1}u_{11}^*u_{12}u_{21}.
        \end{aligned}
    \end{equation}
    In the second term we substitute \( u_{21}u_{11}^*=qu_{11}^*u_{21}\) and in the third one we substitute \( u_{11}^*u_{12}^*=q^{-1}u_{12}^*u_{11}^*\). What we get is
    \begin{subequations}
        \begin{align}
            u_{22}(u_{12}^*)^{n+1}&=q^{n+1}(u_{12}^*)^{n+1}u_{22}-\big[ q^{n+1}(1-q^2)+q^{1-n}(1-q^{2n}) \big](u_{12}^*)^nu_{11}^*u_{21}\\
            &=q^{n+1}(u_{12}^*)^{n+1}u_{22}-(q^{3+n}-q^{1-n})(u_{12}^*)^nu_{11}^*u_{21}\\
            &=q^{n+1}(u_{12}^*)^{n+1}u_{22}-q^2\frac{ 1-q^{2(n+1)} }{ q^{n+1} }(u_{12}^*)^nu_{11}^*u_{21}.
        \end{align}
    \end{subequations}
    This is equation \eqref{subeqlemuijuklstarnivd}.

    \begin{probleme}
        I guess the other ones are checked similarly.
    \end{probleme}
\end{proof}

By equation \eqref{subEquuijcondiii} we have
\begin{equation}
    u_{13}u_{31}=u_{31}u_{13}.
\end{equation}

\begin{lemma}
    The elements \( u_{13}\) and \( u_{31}\) are normal.
\end{lemma}

\begin{proof}
    What we have to show is \( u_{13}^*u_{13}=u_{13}u_{13}^*\) (definition \ref{DefElemNormal}). The relations \eqref{EqLemsumsumstarpar} and \eqref{Equustrunsuqn} provide
    \begin{subequations}
        \begin{align}
            u_{13}^*u_{13}&=u_{13}u_{13}^*+(q^2-1)(\underbrace{u_{23}^*u_{23}+u_{33}^*u_{33}}_{=1-u_{13}^*u_{13}})+(1-q^2)(\underbrace{u_{11}u_{11}^*+u_{12}u_{12}^*}_{=1-u_{13}u_{13}^*})\\
            &=u_{13}u_{13}^*+(q^2-1)(u_{13}u_{13}^*-u_{13}^*u_{13}).
        \end{align}
    \end{subequations}
    Thus
    \begin{equation}
        u_{13}^*u_{13}-u_{13}u_{13}^*=(q^2-1)(u_{13}u_{13}^*-u_{13}^*u_{13})
    \end{equation}
    and the combination \( u_{13}^*u_{13}-u_{13}u_{13}^*\) vanishes.
\end{proof}

Let \( \pi\colon SU_q(3)\to \oB(H)\) be a representation. We write \( x_{ij}=\pi(u_{ij})\).

\begin{probleme}
    Check if \( x_{1n}\) and \( x_{n1}\) are normal for every \( n\) and that \( x_{1n}x_{n1}=x_{n1}x_{1n}\).
\end{probleme}

\begin{probleme}
    I think that looking very hard to \cite{AndrewGreen} can help to justify the fact that \( x_{13}\) and \( x_{31}\) have a basis of eigenvectors. Since they are commuting, we have a common basis of eigenvectors.

    In the remaining I suppose that \( x_{13}\) and \( x_{31}\) have an unique basis of common eigenvectors.

    In \cite{Bragiel}, they ask to see at \cite{PuszWoronowicz}.
\end{probleme}

\begin{proposition}     \label{Propxuteigenuud}
    Let \( h\in H\setminus\{ 0 \}\) be a common eigenvector of \( x_{13}\) and \( x_{31}\), namely suppose \( x_{13}h=\lambda h\) and \( x_{31}h=\rho h\). Then
    \begin{subequations}
        \begin{align}
            x_{13}(x_{11}h)&=\frac{ \lambda }{ q }(x_{11}h)&x_{13}(x_{12}h)&=\frac{ \lambda }{ q }(x_{12}h)\\
            x_{31}(x_{11}h)&=\frac{\rho}{ q }(x_{11}h)&x_{31}(x_{21}h)&=\frac{\rho}{ q }(x_{21}h).
        \end{align}
    \end{subequations}
    
\end{proposition}

\begin{proof}
    This is nothing else than the relations
    \begin{equation}
        \begin{aligned}[]
            qu_{13}u_{11}&=u_{11}u_{13}\\
            u_{12}u_{13}&=qu_{13}u_{12}\\
            u_{11}u_{31}&=qu_{31}u_{11}.
        \end{aligned}
    \end{equation}
\end{proof}

Let us suppose that \( \rho\neq 0\neq\lambda\) and iterate the equations of proposition \eqref{Propxuteigenuud}:
\begin{equation}
    x_{31}(x_{21}^n)h=\frac{ \rho }{ q^n }x_{21}^nh.
\end{equation}
So if not vanishing, the vector \( x_{21}^nh\) is an eigenvector for \( x_{31}\) with eigenvalue \( \rho/q^n\). Since \( q<1\), we must have \( k\in\eN\) such that \( x_{21}^kh\neq 0\) and \( x_{21}^{k+1}h=0\), if not we would have arbitrary large eigenvalues for \( x_{31}\), which is impossible because \( \| x_{13} \|\leq 1\) by proposition \ref{Propqpiideinve}.

Doing the same with \( x_{11}\) and the eigenvector \( x_{21}^kh\) of \( x_{31}\) we find \( l\in\eN\) such that
\begin{equation}
    x_{11}^lx_{12}^k\neq 0
\end{equation}
and 
\begin{equation}
    x_{11}^{l+1}x_{12}^k=0.
\end{equation}
We name \( h_1\) that vector: $h_1=x_{11}^lx_{21}^kh$. It satisfies
\begin{subequations}
    \begin{align}
        x_{11}h_1&=0\\
        x_{21}h_1&=0\\
        x_{31}h_1&=\frac{ \rho }{ q^{l+k} }h_1      \label{subeqxtuhurhoqlk}.
    \end{align}
\end{subequations}
Indeed by construction \( x_{11}h_1=x_{11}^{l+1}x_{21}^kh=0\) and
\begin{equation}
    x_{21}h_1=x_{21}x_{11}^{l}x_{21}^kh=\frac{1}{ q^l }x_{11}^lx_{21}^{k+1}h=0
\end{equation}
because \eqref{subEquuijcondii} implies \( qx_{21}x_{11}=x_{11}x_{21}\).

\begin{proposition}
    We have \( \| x_{31} \|=1\) and every eigenvalue of \( x_{31}\) is of the form \( q^n\) for some \( n\in\eN\).
\end{proposition}

\begin{proof}
    Using the same way as in the proof of proposition \ref{Propqpiideinve}, we have, for every \( f\in\hH\), 
    \begin{equation}
        \| f \|^2=\langle \sum_kx_{ki}^*x_{ki}f, f\rangle=\sum_k\| x_{ki}f \|^2.
    \end{equation}
    Taking \( f=h_1\) and \( i=1\) we find
    \begin{equation}    \label{Eqhunxtunormsqcomp}
        \| h_1 \|^2=\| x_{11}h_1 \|^2+\| x_{21}h_1 \|^2+\| x_{31}h_1 \|^2=\| x_{31}h_1 \|^2.
    \end{equation}
    Thus
    \begin{equation}
        \| x_{31} \|=\sup_{v\in\hH}\frac{ x_{13}v }{ \| v \| }\geq\frac{ \| x_{31}h_1 \| }{ h_1 }=1.
    \end{equation}
    From proposition \ref{Propqpiideinve} we have \( \| x_{31} \|\leq 1\) and thus \( \| x_{31} \|=1\).

    Comparing \eqref{Eqhunxtunormsqcomp} with \eqref{subeqxtuhurhoqlk} we have
    \begin{equation}
        \| h_1 \|=\left| \frac{ \rho }{ q^{l+k} } \right| \| h_1 \|,
    \end{equation}
    so that \( | \rho/q^{l+k} |=1\).
\end{proof}

Let \( h_2\) be an eigenvector of \( x_{31}\) with eigenvalue \( \rho_2\) with \( | \rho_2 |=1\). We have
\begin{equation}
    x_{31}x_{11}h_2=\frac{ \rho_2 }{ q }x_{11}h_2,
\end{equation}
so that \( x_{11}h_2\) has to be an eigenvector of \( x_{13}\) with eigenvalue \( \rho/q\). Since \( | \rho/q |>1\) we have \( x_{11}h_2=0\). In the same way we find \( x_{21}h_2=0\).

Let \( m\in\eN\) be such that \( x_{12}^mh_2\neq 0\) and \( x_{12}^{m+1}h_2=0\) and set
\begin{equation}
    h_0=x_{12}^mh_2.
\end{equation}
Using the relations
\begin{subequations}
    \begin{align}
        u_{11}u_{12}&=qu_{12}u_{11}\\
        u_{21}u_{12}&=u_{12}u_{21}\\
        u_{31}u_{12}&=u_{12}u_{31}
    \end{align}
\end{subequations}
we have
\begin{equation}
        x_{11}h_0=x_{21}h_0=x_{12}h_0=0
\end{equation}
and, if we set \( \rho_0=\rho_2\),
\begin{equation}
    x_{31}h_0=\rho_0h_0
\end{equation}
with \( | \rho_0 |=1\). 

In addition, if \( x_{13}h_0=\lambda_0h_0\), we have \( | \lambda_0 |=q^2\). In order to prove this claim, we need some more relations. Applying  \eqref{subequustuuuutroisi} to \( h_0\) we have
\begin{equation}
    x_{11}x_{11}^*h_0=(1-q^2)h_0.
\end{equation}
Now we apply to \( h_0\) the equality
\begin{equation}
    x_{11}x_{11}^*+x_{12}x_{12}^*+x_{13}x_{13}^*=1.
\end{equation}
If \( x_{13}h_0=\lambda_0h_0\) we have \( x_{13}x_{13}^*h_0=\overline{ \lambda_0 }\lambda_0h_0=| \lambda_0 |^2h_0\). Thus
\begin{equation}
    x_{12}x_{12}^*h_0=(q^2-| \lambda_0 |^2)h_0.
\end{equation}
On the other hand applying \eqref{subequustuuuutroisii} to \( h_0\) we have
\begin{equation}
    x_{12}x_{12}^*h_0=q^2(1-q^2)h_0.
\end{equation}
Thus we have \( | \lambda_0 |=q^2\).

What we have proved up to here is the following.
\begin{proposition}     \label{Propeignxutxtunonzeroeigvvap}
    If there exists \( h\in\hH\) such that \( x_{13}h=\lambda h\) and \( x_{31}h=\rho h\) with \( \lambda\neq 0\neq \rho\), then there exists \( h_0\in\hH\) and \( \psi,\varphi\in S^1\) such that \( \| h_0\|=1 \),
    \begin{equation}
        x_{11}h_0=x_{12}h_0=x_{21}h_0=0
    \end{equation}
    and
    \begin{subequations}
        \begin{align}
            x_{13}h_0&=\lambda_0h_0\\
            x_{31}h_0&=\rho_0 h_0,
        \end{align}
    \end{subequations}
    with \( \lambda_0=q^2\psi\) and \( \rho_0=\varphi\). The eigenvalues of \( x_{31}\) are of the form \( q^n\).
\end{proposition}

\begin{probleme}
    Question : do we have \( \| x_{13} \|=q^2\) ?
\end{probleme}

We already know that the representation space \( \hH\) has a basis of common eigenvectors of \( x_{13}=\pi(u_{13})\) and \( x_{31}=\pi(u_{31})\). The representations of \( \SU_q(3)\) are now divided in families that depend on the existence of such eigenvectors with simultaneously non vanishing eigenvalues. The proposition \ref{Propeignxutxtunonzeroeigvvap} describes the situation where one has a common eigenvector with non vanishing eigenvalue of \( x_{13}\) and \( x_{31}\). The following theorem describes the representations in that case.

\begin{theorem}     \label{ThoRepresSUqtnz}
    Let \( \pi\) be a representation of \( \SU_q(3)\) on \( \hH\). Suppose that it contains a common eigenvector \( h_0\) of \( x_{13}\) and \( x_{31}\) with simultaneously non vanishing eigenvalues. Let us write \( x_{13}h_0=\lambda_0h_0\), \( x_{31}h_0=\rho_0h_0\) with \( \lambda_0\neq 0\neq\rho_0\). Then there exist \( \psi, \varphi\in S^1\) such that \( \lambda_0=q^2\psi\) and \( \rho_0=\varphi\).

    Moreover the representation is described in the following way: let 
    \begin{subequations}
        \begin{align}
            \ket{0,0,0}&=h_0\\
            \ket{N,0,0}&=\frac{1}{ A_N }(x_{11}^*)^N\ket{0,0,0}\\
            \ket{N,M,0}&=\frac{1}{ q^{M(N+1)} }(x_{12}^*)^M\ket{N,0,0}\\
            \ket{N,M,L}&=\frac{1}{ q^{NL}A_L }(x_{21}^*)^L\ket{N,M,0}
        \end{align}
    \end{subequations}
    where \( N,M,L\in\eN\) and \( A_N=\left( \prod_{j=1}^N(1-q^{2j})\right)^{1/2}\). Then the action of \( \SU_q(3)\) on \( \hH\) is given by
    \begin{subequations}
        \begin{align}
            x_{11}\ket{N,M,L}&=(1-q^{2N})^{1/2}\ket{N-1,M,L}        \label{subeqxijNMLa}\\
            x_{12}\ket{N,M,L}&=q^{N+1}(1-q^{2M})^{1/2}\ket{N,M-1,L}\\
            x_{13}\ket{N,M,L}&=q^{2+N+M}\psi\ket{N,M-1,L}\\
            x_{21}\ket{N,M,L}&=q^N(1-q^{2L})^{1/2}\ket{N,M,L-1}\\
            x_{22}\ket{N,M,L}&=-q^{L+M+1}\bar\psi\bar\varphi\ket{N,M,L}\\
                &\quad-\Big( (1-q^{2L})(1-q^{2M})(1-q^{2(N+1)}) \Big)^{1/2}\ket{N+1,M-1,L-1}\\
            x_{23}\ket{N,M,L}&=q^{L+1}(1-q^{2(M+1)})^{1/2}\bar\varphi\ket{N,M+1,L}\\
            &\quad-q^{M+1}\Big( (1-q^{2L})(1-q^{2(N+1)}) \Big)^{1/2}\psi\ket{N+1,M,L-1}\\
            x_{31}\ket{N,M,L}&=q^{N+L}\varphi\ket{N,M,L}\\
            x_{32}\ket{N,M,L}&=q^M(1-q^{2(L+1)})^{1/2}\bar\psi\ket{N,M,L+1}\\&\quad-q^L\Big( 1-q^{2(N+1)}(1-q^{2M}) \Big)\varphi\ket{N+1,M-1,L}\\
            x_{33}\ket{N,M,L}&=-q^{L+M+1}(1-q^{2(N+1)})\psi\varphi\ket{N+1,M,L}\\&\quad-\Big( (1-q^{2(L+1)})(1-q^{2(M+1)}) \Big)^{1/2}\ket{N,M+1,L+1},
        \end{align}
    \end{subequations}
    and
    \begin{subequations}
        \begin{align}
            x_{11}^*\ket{N,M,L}&=(1-q^{2(N+1)})^{1/2}\ket{N+1,M,L}\\
            x_{12}^*\ket{N,M,L}&=q^{N+1}(1-q^{2(M+1)})^{1/2}\ket{N,M+1,L}\\
            x_{13}^*\ket{N,M,L}&=q^{2+M+N}\bar\psi\ket{N,M,L}\\
            x_{21}^*\ket{N,M,L}&=q^N(1-q^{2(L+1)})^{1/2}\ket{N,M,L+1}\\
            x_{22}^*\ket{N,M,L}&=-q^{L+M+1}\varphi| \psi |^2\ket{N,M,L}\\&\quad -\Big( (1-q^{2(L+1)})(1-q^{2(M+1)})(1-q^{2N}) \Big)^{1/2}\ket{N-1,M+1,L+1}\\
            x_{23}^*\ket{N,M,L}&=q^{L+1}(1-q^{2M})^{1/2}\varphi\ket{N,M-1,L}\\&\quad -q^{M+1}\Big( (1-q^{2N})(1-q^{2(M+1)}) \Big)^{1/2}\bar\psi\ket{N-1,M+1,L}\\
            x_{31}^*\ket{N,M,L}&=q^{N+L}(1-q^{2L})^{1/2}\psi\ket{N,M,L-1}\\&\quad q^L\Big( (1-q^{2N})(1-q^{2(M+1)}) \Big)^{1/2}\bar\varphi\ket{N-1,M+1,L}\\
            x_{33}^*\ket{N,M,L}&=q^{L+M+1}(1-q^{2N})^{1/2}\bar\psi\bar\varphi\ket{N-1,M,L}\\&\quad -\Big( (1-q^{2L})(1-q^{2M}) \Big)^{1/2}\ket{N,M-1,L-1}.
        \end{align}
    \end{subequations}
    The basis \(\{ \ket{N,M,L} \}\) of \( \hH\) is orthonormal, that is
    \begin{equation}
        \langle L,M,N|A,B,C\rangle = \delta_{AN}\delta_{BM}\delta_{CL}.
    \end{equation}
\end{theorem}

Before to pass to the proof, we give some formulas that are used in the proof.
\begin{lemma}       \label{LemTechrepresSUqtnez}
    Under the assumptions of theorem \ref{ThoRepresSUqtnz} we have
    \begin{enumerate}
        \item   \label{ItemLemTechrepresSUqtnez}
            \( u_{11}(u_{12}^*)^Mh_0=0\).
    \end{enumerate}
\end{lemma}

\begin{proof}
    We proof \ref{ItemLemTechrepresSUqtnez} by induction over \( M\). For \( M=1\) we apply the relation
    \begin{equation}
        u_{11}u_{12}^*=\frac{1}{ q }u_{12}^*u_{11}+(q^{-1}-q)(u_{22}^*u_{21}+u_{32}u_{31})
    \end{equation}
    to \( h_0\). By construction \( x_{11}h_0=x_{21}h_0=0\). Since \( qu_{31}u_{32}^*=u_{32}^*u_{31}\), the vector \( u_{32}^*h_0\) is an eigenvector of \( x_{31}\) with eigenvalue \( \rho_0/q\). Thus
    \begin{equation}
        u_{32}^*h_0=0
    \end{equation}
    because \( | \rho_0/q |>1\). Item \ref{ItemLemTechrepresSUqtnez} is proved.

\end{proof}


\begin{proof}[proof of theorem \ref{ThoRepresSUqtnz}]
    Existence of \( h_0\), \( \psi\) and \( \varphi\) are part of proposition \ref{Propeignxutxtunonzeroeigvvap}. The action of \( \SU_q(3)\) on the basis \( \ket{N,M,L}\) is computed using the commutation relations and the lemmas \ref{Lemxijxstijsuqt} and \ref{Lemuijustijnstarnsubeqijuk}. That computation proves in the same time that the vectors \( \ket{N,M,L}\) actually form a basis.

    For sake of simplicity in the computations we introduce the vectors
    \begin{equation}
        h'_{N,M,L}=(x_{21}^*)^L(x_{12}^*)^M(x_{11}^*)^Nh_0
    \end{equation}
    or
    \begin{equation}        \label{EqhpcoeffndeketNML}
        h'_{N,M,L}=A_NA_Lq^{M(N+1)}q^NL\ket{N,M,L}.
    \end{equation}

    Let us perform the computation for \eqref{subeqxijNMLa}. First we compute \( x_{11}h'_{N,M,0}\):
    \begin{equation}
        x_{11}h'_{N,M,0}=x_{11}(X_{12}^*)^M(x_{11}^*)^N=q^{NM}x_{11}(x_{11}^*)^N(x_{12}^*)^Mh_0.
    \end{equation}
    We used \( x_{12}^*x_{11}^*=qx_{11}^*x_{12}^*\). Using lemma \ref{Lemuijustijnstarnsubeqijuk} we get
    \begin{equation}
        x_{11}h'_{N,M,0}=q^{MN}\Big( (1-q^{2N})(u_{11}^*)^{N-1}+q^{2N}(u_{11}^*)^{N}u_{11} \Big)(u_{12}^*)^Mh_0
    \end{equation}
    Using lemma \ref{LemTechrepresSUqtnez} it remains
    \begin{subequations}
        \begin{align}
            x_{11}h'_{N,M,0}&=q^{MN}(1-q^{2N})(x_{11}^*)^{N-1}(x_{12}^*)^Mh_0\\
            &=q^{MN}(1-q^{2N})\frac{1}{ q^{M(N-1)} }(u_{12}^*)^M(u_{11}^*)^{N-1}\\
            &=q^M(1-q^{2N})h'_{N-1,M,0}.
        \end{align}
    \end{subequations}
    Using now the relation \( u_{11}u_{21}^*=qu_{21}^*x_{11}\) we have
    \begin{subequations}
        \begin{align}
            x_{11}h'_{N,M,L}&=x_{11}(x_{21}^*)^Lh'_{N,M,0}\\
            &=q^L(x_{21}^*)^Lx_{11}h'_{N,L,0}\\
            &=q^{L+M}(1-q^2N)h'_{N-1,M,N}.
        \end{align}
    \end{subequations}
    Using the conversion factor \eqref{EqhpcoeffndeketNML} we get the desired action:
    \begin{equation}
        x_{11}\ket{N,M,L}=(1-q^{2N})^{1/2}\ket{N-1,M,L}.
    \end{equation}
    
    
\end{proof}


%+++++++++++++++++++++++++++++++++++++++++++++++++++++++++++++++++++++++++++++++++++++++++++++++++++++++++++++++++++++++++++
\section{Quantized universal algebras}
%+++++++++++++++++++++++++++++++++++++++++++++++++++++++++++++++++++++++++++++++++++++++++++++++++++++++++++++++++++++++++++

We follow \cite{SoibelmanI}.

Let \( \lG\) be a complex simple Lie algebra of rank \( n\) with its standard Lie bialgebra structure (proposition \ref{PropStandardBialgStruct}). Let \( q\in\eC\setminus\{ 0 \}\) a complex number which is not a root of unity\footnote{There are no \( n\in\eN\) such that \( q^n=1\).}.

We denote by \( \lH\) a Cartan subalgebra of \( \lG \), by \( \alpha_1,\ldots,\alpha_n\) the simple roots and \( A_{ij}=\frac{ 2(\alpha_i,\alpha_j) }{ (\alpha_i,\alpha_i) }\) the Cartan matrix\footnote{This is a different convention from \cite{SoibelmanI} in which \( A_{ij}=2(\alpha_i,\alpha_i)/(\alpha_j,\alpha_j)\).}. We also introduce the following notations:
\begin{equation}
    \begin{aligned}[]
        q_i&=q^{(\alpha_i,\alpha_i)/2}\\
        (a;t)_k&=(1-a)(1-at)\ldots (1-at^{k-1})\\
        \binom{m}{n}_t&=\frac{ (t;t)_m }{ (t;t)_n(t;t)_{m-n} }.
    \end{aligned}
\end{equation}
The latter are the \( t\)-binomial coefficients also called \defe{Gauss polynomial}{Gauss!polynomial}.

\begin{definition}
    Let \( \lG\) be a simple complex Lie algebra with its Cartan matrix 
    \begin{equation}
        A_{ij}=2\frac{ (\alpha_i,\alpha_j) }{ (\alpha_i,\alpha_i) }.
    \end{equation}
    The Hopf algebra \( U_h\lG\)\nomenclature[A]{\( U_h\lG\)}{a Hopf algebra} on \( \eC\ldbrack h\rdbrack \) is the algebra generated by \( X_i\), \( Y_i\), \( H_i\) (\( i=1,\ldots,n\)) which is complete for the \( h\)-adic topology and subject to the relations
    \begin{equation}
        \begin{aligned}[]
            [H_i,H_j]&=0\\
            [H_i,X^{\pm}_j]&=\pm(\alpha_i,\alpha_j)X^{\pm}_j\\
            [X_i^+,X_j^-]&=\delta_{ij}\frac{ \sinh\left( \frac{ h }{2}H_i \right) }{ \sinh\left( \frac{ h }{2} \right) }\\
            \sum_{k=0}^{1-A_{ij}}(-1)^k\binom{1-A_{ij}}{k}(X_i^{\pm})^kX_j^{\pm}(X_i^{\pm})^{1-A_{ij}-k}&=0 &   \text{if } i\neq j.
        \end{aligned}
    \end{equation}
    
\end{definition}

\begin{proposition}
    The following defined a structure of Hopf algebra on \( U_h\lG\):
    \begin{equation}
        \begin{aligned}[]
            \Delta(H_i)&=H_i\otimes 1+1\otimes H_i,  &\Delta(X_i^{\pm})&=q^{-H_i/2}\otimes X_i^{\pm}+X_i^{\pm}\otimes q^{H_i/2},\\
            S(H_i)&=-H_i,        &   S(X_i^{\pm})&=-q_i^{\pm 1}X_i^{\pm},\\
            \epsilon(H_i)&=0,    &   \epsilon(X_i^{\pm})&=0.
        \end{aligned}
    \end{equation}
    
\end{proposition}

\begin{lemma}
    The algebra  \( U_h\lG\) accepts the set of generators \( \{ H_i,E_i,F_i \}\) with
    \begin{subequations}
        \begin{numcases}{}
            E_i=X_i^+q^{-H_i/2}\\
            F_i=X_i^-q^{H_i/2}.
        \end{numcases}
    \end{subequations}
    The coproduct is given by
    \begin{subequations}
        \begin{align}
            \Delta(E_i)&=E_i\otimes 1+q^{-H_i}\otimes E_i\\
            \Delta(F_i)&=1\otimes F_i+F_i\otimes q^{H_i},
        \end{align}
    \end{subequations}
    the antipode is
    \begin{subequations}
        \begin{align}
            S(E_i)&=-q^{H_i}E_i\\
            S(f_i)&=-F_iq^{-H_i},
        \end{align}
    \end{subequations}
    and the counit is
    \begin{subequations}
        \begin{align}
            \epsilon(E_i)&=0\\
            \epsilon(F_i)&=0.
        \end{align}
    \end{subequations}
\end{lemma}

\begin{proof}
    In order to see how it works, we show the formula for \( \Delta(E_i)\).
    \begin{probleme}
        One has to check the others.
    \end{probleme}
    We consider the expansion \( q^{x}=\sum_ka_kx^k\). We have
    \begin{subequations}
        \begin{align}
            (a\otimes b)q^{X\otimes 1}&=\sum_ka_k(a\otimes b)(X\otimes 1)^k\\
            &=\sum_k a_k aX^k\otimes b\\
            &=aq^X\otimes b.
        \end{align}
    \end{subequations}
    In the same way \( (a\otimes b)q^{1\otimes X}=a\otimes bq^{X}\). Using these formulas we find
    \begin{subequations}
        \begin{align}
            \Delta(E_i)=\Delta(X_i)\Delta(q^{-H_i/2})&=\big( q^{-H_i/2}\otimes X_i^{\pm}+X_i^{\pm}\otimes q^{H_i/2} \big)q^{-\frac{ 1 }{2}(H_i\otimes 1)}q^{-\frac{ 1 }{2}(1\otimes H_i)}\\
            &=q^{-H_i/2}q^{-H_i/2}\otimes X_i^{\pm}q^{-H_i/2}+X_i^{\pm}q^{-H_i/2}\otimes q^{H_i/2}q^{-H_i/2}\\
            &=q^{-H_i}\otimes E_i+E_i\otimes 1.
        \end{align}
    \end{subequations}
\end{proof}

If \( V_1\) and \( V_2\) are \( \eK\ldbrack h\rdbrack\)-modules we define \( V_1\hat\otimes V_2\)\nomenclature[A]{$ V_1\hat\otimes V_2$}{product of $ \eK\ldbrack h\rdbrack$-modules } as the limit
\begin{equation}
    V_1\hat\otimes V_2=\lim_{\leftarrow n}V_1/h^n V_1\otimes V_2/h^nV_2.
\end{equation}
This is the completion of \( V_1\otimes V_2\) for the \( h\)-adic topology.

\begin{probleme}
    I've to know if that limit is the one defined in section \ref{SecDirectLimit}.
\end{probleme}

\begin{theorem}     \label{ThoIsomUhGUg}
    Let \( \lG\) be a simple complex finite dimensional Lie algebra with its standard Lie bialgebra structure.
    \begin{enumerate}
        \item
        
            There exists an algebra isomorphism 
            \begin{equation}
                U_h\lG\to(U\lG)\ldbrack h\rdbrack
            \end{equation}
            which is the identity modulo \( h\eC\ldbrack h\rdbrack\).
        \item
            That isomorphism can be chosen in such a way to be the identity on \( \lH\).
        \item
            The center of \( U_h\lG\) is isomorphic to \( Z\ldbrack h\rdbrack\) if \( Z\) is the center of \( U\lG\).
    \end{enumerate}
    
\end{theorem}

For a proof, see \cite{SoibelmanI}, page 60.

\begin{probleme}
    If you understand the proof, please write me an email.
\end{probleme}

A \( U_h\lG\)-module \( V\) is \defe{finite dimensional}{module!over $U_h\lG$!finite dimensional} if it is  a \( \eK\)-module of finite type. The corresponding homomorphism \( \pi\colon U_h\lG\to \End(V)\) is a finite dimensional \defe{representation}{representation!of $U_h\lG$}.

This theorem allows to bring the representation theory of the classical case (\( U\lG\)) to the quantum case. More precisely if \( \mM_h\) is the category of finite dimensional \( U_h\lG\)-modules and if \( \mM_{\lG}\) is the one of finite dimensional \( \lG\)-modules, we can define a functor
\begin{equation}
    \begin{aligned}
        F\colon \mM_{\lG}&\to \mM_h \\
        V&\mapsto V\ldbrack h\rdbrack 
    \end{aligned}
\end{equation}
and the action of \( U_h\lG\) on \( V\ldbrack h\rdbrack\) is the composition of the action of \( (U\lG)\ldbrack h\rdbrack \) and \( \psi\) where 
\begin{equation}
    \psi\colon U_h\lG\to (U\lG)\ldbrack h\rdbrack
\end{equation}
is the isomorphism of theorem \ref{ThoIsomUhGUg}.

We also consider the functor
\begin{equation}
    \begin{aligned}
        G\colon \mM_h&\to \mM_{\lG} \\
        G(W)&= W/hW. 
    \end{aligned}
\end{equation}
The action of \( U\lG\) on \( W/hW\) is given by
\begin{equation}
    X\cdot [v]=\big[ \psi^{-1}(X)\cdot v \big]
\end{equation}
where \( v\in W\), \( [v]\) is the class modulo \( \ldbrack h\rdbrack\) and \( \psi\) is the isomorphism of theorem \ref{ThoIsomUhGUg}. Notice that the action is well defined since
\begin{equation}
    \psi^{-1}(X)\cdot(v+hw)=\psi^{-1}(X)\cdot v+h\psi^{-1}(X)\cdot w.
\end{equation}
The last term belongs to \( hW\).

\begin{proposition}
    The functors \( F\) and \( G\) are a bijections between the isomorphisms classes of objects of \( \mM_{\lG}\) and \( \mM_h\). The simple modules in \( \mM_{\lG}\) correspond to the indecomposable modules in \( \mM_h\).
\end{proposition}
The reference \cite{SoibelmanI} says that the proof is straightforward. I didn't tried\quext{Let me know if it is straightforward ;)}.

Let \( P_+\) be the set of dominant weights of \( \lG\). We denote by \( L_0(\Lambda)\) the \( U\lG\)-module of highest weight \( \Lambda\in P_+\). Then we denote by \( L(\Lambda)\)\nomenclature[Q]{\( L(\Lambda)\)}{Module of highest weight \( \Lambda\) on \( U_h\lG\)} such that
\begin{equation}
    L_0(\Lambda)=L(\Lambda)/hL(\Lambda).
\end{equation}
The corresponding representation is written \( \pi_{\Lambda}\). If \( \Lambda=\omega_i\) is the \( i\)th fundamental weight (i.e. \( \omega_i(H_j)=\delta_{ij}\)) then we say that \( \pi_{\omega_i}\) is the \( i\)th \defe{fundamental representation}{representation!fundamental}\index{fundamental!representation} of \( U_h\lG\).

\begin{lemma}
    The module \( L(\Lambda)\) defined by \( L_0(\Lambda)=L(\Lambda)/hL(\Lambda) \) is equivalently given by
    \begin{equation}
        L(\Lambda)=L_0(\Lambda)\ldbrack h\rdbrack.
    \end{equation}
\end{lemma}

\begin{proof}
    The quotient \( L(\Lambda)/hL(\Lambda)\) consist in removing all the terms of nonzero order in \( h\). Thus
    \begin{equation}
        L(\Lambda)=\frac{ L(\Lambda) }{ hL(\Lambda) }\ldbrack h\rdbrack.
    \end{equation}
    
\end{proof}

\begin{proposition}
    Let \( \Lambda\) be a dominant weight and \( L_0(\Lambda)\) be the \( U\lG\)-module of highest weight \( \Lambda\). We consider the \( U_h\lG\)-module \( L(\Lambda)=L_0(\Lambda)\ldbrack h\rdbrack\). We have the weight decomposition
    \begin{equation}
        L_0(\Lambda)=\bigoplus_{\lambda\in P(\lambda)}L_0(\Lambda)_{\lambda}
    \end{equation}
    where \( P(\Lambda)\) is the set of weights of \( L_0(\Lambda)\) and
    \begin{equation}
        L_0(\Lambda)_{\lambda}=\{ v\in L_0(\Lambda)\tq av=\lambda(a)v\,\forall a\in\lH \}.
    \end{equation}
    Correspondingly we have the decomposition
    \begin{equation}
        L(\Lambda)=\bigoplus_{\lambda\in P(\Lambda)}L(\Lambda)_{\lambda}
    \end{equation}
    where
    \begin{equation}
        L(\Lambda)_{\lambda}=L_0(\Lambda)_{\lambda}\ldbrack h\rdbrack=\{ v\in L(\Lambda)\tq av=\lambda(a)v\,\forall a\in\lH\subset U_h\lG \}.
    \end{equation}
    
\end{proposition}

\begin{proposition}
    The finite dimensional indecomposable \( U_h\lG\)-module \( L(\Lambda)\) is generated by a vector \( v_{\Lambda}\) such that
    \begin{subequations}
        \begin{numcases}{}
            X_i^{+}v_{\Lambda}=0\\
            H_i v_{\Lambda}=\Lambda(H_i)v_{\Lambda}.
        \end{numcases}
    \end{subequations}
    This vector is the \defe{highest weight vector}{highest weight!in $ U_h\lG$-modules} and \( \Lambda\) is the highest weight of \( L(\Lambda)\).

    In the same way, the module \( L(\Lambda)\) is generated by a vector \( v_{\tilde \Lambda}\) such that
    \begin{subequations}
        \begin{numcases}{}
            X_i^-v_{\tilde \Lambda}=0\\
            H_i v_{\tilde \Lambda}=\tilde \Lambda(H_i)v_{\tilde \Lambda}
        \end{numcases}
    \end{subequations}
    where \( \tilde \Lambda=w_0\Lambda\) and \( w_0\) is an element of the Weyl group of longest length. The vector \( v_{\tilde \Lambda}\) is a \defe{lowest weight vector}{lowest weight!in $ U_h\lG$-module} of \( L(\Lambda)\).

    The vectors \( v_{\Lambda}\) and \( v_{\tilde \Lambda}\) are unique up to scalar multiple.

\end{proposition}

%---------------------------------------------------------------------------------------------------------------------------
\subsection{Example with $ \gsl(2,\eC)$}
%---------------------------------------------------------------------------------------------------------------------------

The classical representations of \( \gsl(2,\eC)\) are given in section \ref{SecsldeuxCandrepres} and equations \eqref{EqReprezgsldeuxC}. In the case of \( U_h\gsl(2,\eC)\), the representation space is given by \( V_m\ldbrack h\rdbrack\) and if \( v\in V_m\ldbrack h\rdbrack\), \( Z\in U_h\gsl(2,\eC)\) we define the corresponding representation
\begin{equation}
    Z\cdot v=\psi(Z)\cdot v   
\end{equation}
where \( \psi(Z)\ U\gsl(2,\eC)\ldbrack h\rdbrack\) has a well defined action on \( v\in V_m\ldbrack h\rdbrack\). In order to determine the representation, we have to write the isomorphism \( \psi\).



%+++++++++++++++++++++++++++++++++++++++++++++++++++++++++++++++++++++++++++++++++++++++++++++++++++++++++++++++++++++++++++
\section{Quantum universal enveloping algebra}
%+++++++++++++++++++++++++++++++++++++++++++++++++++++++++++++++++++++++++++++++++++++++++++++++++++++++++++++++++++++++++++
One speak about the representations of \( U_q\lG\) in the chapter 10 of \cite{GuideToQuantumGroups}.

\begin{definition}      \label{DefUqlG}
    The \defe{quantum universal enveloping algebra}{quantum!universal enveloping algebra} \( U_q\lG\)\nomenclature[A]{\( U_q\lG\)}{quantum universal enveloping algebra} is the complex unital algebra with generators \( X_i^+\), \( X_i^-\), \( K_i\) and \( K_i^{-1}\) (\( i=1,\ldots,n\)) and the relations
    \begin{enumerate}
        \item
            \( K_iK_i^{-1}=K_i^{-1}K_i=1\);
        \item
            \( [K_i,K_j]=0\);
        \item
            \( [X_i^+,X_j^-]=\delta_{ij}\frac{ K_i^2-K_i^{-2} }{ q_i-q_i^{-1} }\);
        \item       \label{EqUqlGdefiv}
            \( K_iX_j^{\pm}=q^{\pm(\alpha_i,\alpha_j)}X_j^{\pm}K_i\);
        \item
            \begin{equation}
                \sum_{k=0}^{1-A_{ij}}(-1)^k\binom{1-A_{ij}}{k}_{q_i}(X_i^{\pm})^kX_j^{\pm}(X_i^{\pm})^{1-A_{ij}-k}=0
            \end{equation}
    \end{enumerate}
    where \( q_i=q^{(\alpha_i,\alpha_i)/2}\).
\end{definition}
We will sometimes write \( X_i\) for \( X_i^+\) and \( Y_i\) for \( X_i^-\).

\begin{remark}
    In the literature we find other conventions. In \cite{Kassel} the algebra \( U_q\lG\) has the relations
    \begin{equation}
        [E_i,F_j]=\delta_{ij}\frac{ K_i-K_i^{-1} }{ q-q^{-1} } 
    \end{equation}
    and
    \begin{equation}
        K_iE_jK_i^{-1}=q^{A_{ij}}E_j.
    \end{equation}
    This correspond to an other choice of \( q\) and \( K\to K^2\).
\end{remark}

\begin{proposition}
    The algebra \( U_q\lG\) becomes a Hopf algebra with the definitions
    \begin{subequations}
        \begin{align}
            \Delta(K_i)&=K_i\otimes K_i &\Delta(X_i^{\pm})&=X_i^{\pm}\otimes K_i+K_i^{-1}\otimes X_i^{\pm}\\
            S(K_i)&=K_i^{-1}        &S(X_i^{\pm})&=-q_i^{\pm 1}X_i^{\pm}        \label{subEqantpUqGl}\\
            \epsilon(K_i)&=1        &\epsilon(X_i^{\pm})&=0
        \end{align}
    \end{subequations}
    where \( q_i=q^{(\alpha_i,\alpha_i)/2}\).
\end{proposition}

\begin{proof}
    We have to check the relations of definition \ref{DefHopfAlgebra} on the generators. Let us begin by \( (\id\otimes\epsilon)\Delta=\id \):
    \begin{subequations}
        \begin{align}
            (\id\otimes\epsilon)\Delta K_i&=(\id\otimes\epsilon )(K_i\otimes K_i)=K_i\otimes 1=K_i\\
            (\id\otimes\epsilon)\Delta X_i^+&=(\id\otimes\epsilon)(X_i^+\otimes K+K^{-1}\otimes X_i^+)\\
            &=X_i^+\otimes 1+K^{-1}\otimes 0\\
            &=X_i^+.
        \end{align}
    \end{subequations}
    We also have
    \begin{subequations}
        \begin{align}
            \mu(\id\otimes S)\Delta X_i^+&=\mu(X_i^+\otimes SK+K^{-1}\otimes SX_i^+)\\
            &=X_i^+K^{-1}-q_iK^{-1}X_i^+\\
            &=0\\
            &=\eta\epsilon X_i^+.
        \end{align}
    \end{subequations}
    We used the relation \eqref{EqUqlGdefiv} of definition \ref{DefUqlG}.
\end{proof}

We denote by \( U_q\lB_+\)\nomenclature[A]{$U_q\lB_{\pm}q$}{Hopf subalgebra of $U_q\lG$} the Hopf subalgebra of \( U_q\lG\) generated by \( \{ X_i^+,K_i,K_i^{-1} \}_{i=1,\ldots,n}\) and by \( U_q\lB_-\) the one generated by \( \{ X^-,K_i,K_i^{-1} \}_{i=1,\ldots,n}\). The Hopf subalgebra generated by \( \{ K_i,K_i^{-1} \}_i\) will be denoted by \( U_q\lH\).\nomenclature[A]{$U_q\lH$}{Hopf subalgebra of $U_q\lG$}

We denote by \( U_q\lN_+\) the subalgebra generated by \( \{ X^+ _i\}_i\) and by \( U_q\lN^-\) the one generated by \( \{ X_i^- \}\). These two subalgebras are not Hopf subalgebras since the coproduct of \( X^{\pm}_i\) involves \( K_i\). For each \( \nu\in\eN\) we will denote by \( (U_q\lN_+)_{\nu}\) the subspace of \( U_q\lN_+\) generated by the monomials of length \( \nu\):
\begin{equation}
    X_{i_1}^{+}X_{i_2}^+\ldots X_{i_{\nu}}^+.
\end{equation}


%---------------------------------------------------------------------------------------------------------------------------
\subsection{Admissible modules}
%---------------------------------------------------------------------------------------------------------------------------

If \( V\) is a finite dimensional \( U_q\lG\)-module and if \( \lambda\in\lH^*\) we define
\begin{equation}
    V_{\lambda}=\{ v\in V\tq K_iv=q^{(\lambda,\alpha_i)}v \}.
\end{equation}
\begin{definition}
    A finite dimensional \( U_q\lG\)-module is \defe{admissible}{admissible! $U_q\lG$-module} if it accepts the decomposition
    \begin{equation}
        V=\bigoplus_{\lambda\in\lH^*}V_{\lambda}.
    \end{equation}
\end{definition}

If \( V\) and \( W\) are admissible \( U_q\lG\)-modules we define 
\begin{equation}
    \begin{aligned}
        \psi_{V,W}\colon V\otimes W&\to V\otimes W \\
        v\otimes w&\mapsto q^{(\lambda,\mu)}v\otimes w 
    \end{aligned}
\end{equation}
if \( v\in V_{\lambda}\) and \( w\in W_{\mu}\).

\begin{theorem}
    There exists an element \( \Theta\in U_q\lG\hat\otimes U_q\lG\) such that
    \begin{enumerate}
        \item
            \( \Theta  \) reads as a sum \( \Theta=\sum_{\nu=0}^{\infty}\Theta_{\nu}\) with \( \Theta_0=1\otimes 1\) and
            \begin{equation}
                \Theta_{\nu}\in (U_q\lN_+)_{\nu}(U_q\lH)\otimes(U_q\lN_-)_{\nu}(U_q\lH);
            \end{equation}
        \item
            for every pair of modules \( V\) and \( W\) we have
            \begin{equation}
                \Theta\psi_{V,W}\Delta(a)=\Delta'(a)\Theta\psi_{V,W}
            \end{equation}
            where \( \Delta'=\sigma\circ\Delta\).
    \end{enumerate}
\end{theorem}

\begin{probleme}
    Write me if you know a proof of that.
\end{probleme}

Let an admissible module
\begin{equation}
    V=\bigoplus_{\lambda\in\lH^*}V_{\lambda}
\end{equation}
with
\begin{equation}
    V_{\lambda}=\{ v\tq K_iv=q^{(\lambda,\alpha_i)}v \}.
\end{equation}
Let \( v\in V_{\lambda}\), using the commutation relations we have
\begin{subequations}
    \begin{align}
        K_i X_j^+v&=q^{(\alpha_i,\alpha_j)}X_j^+K_iv\\
        &=q^{(\alpha_i,\alpha_j)}q^{(\lambda,\alpha_i)}X_j^+v\\
        &=q^{(\alpha_j+\lambda,\alpha_i)}X_j^+v.
    \end{align}
\end{subequations}
Thus we have 
\begin{equation}
    X_j^+ V_{\lambda}\subset V_{\alpha_j+\lambda}.
\end{equation}

%---------------------------------------------------------------------------------------------------------------------------
\subsection{Example on $U_q\gsl(2,\eC)$}
%---------------------------------------------------------------------------------------------------------------------------

For notational simplicity we write
\begin{equation}
    [n]=\frac{ q^{n}-q^{-n} }{ q-q^{-1} }
\end{equation}
and 
\begin{equation}
    [n]!=[1][2]\ldots [n].
\end{equation}

We follow \cite{Kassel} and we consider the algebra \( U_q\gsl(2,\eC)\) defined by the generators \( E\), \( F\), \( K\), \( K^{-1}\) and the relations
\begin{subequations}        \label{SubEqsDefUrsldc}
    \begin{align}
        KK^{-1}&=K^{-1}K=1      \\
        KEK^{-1}&=q^2E          \label{EqUqslKEK}\\
        KFK^{-1}&=q^{-2}F       \label{EqsiKFKqdGF}\\
        [E,F]&=\frac{ K-K^{-1} }{ q-q^{-1} }.
    \end{align}
\end{subequations}
The first point is to show that this family of algebra is the same as the one defined in definition \ref{DefUqlG} in which we pose \( (\alpha,\alpha)=\frac{ 1 }{2}\) because of equation \eqref{Eqinnerhstarsldc}. For this purpose we temporally rewrite the equations \eqref{SubEqsDefUrsldc} as
\begin{subequations}
    \begin{align}
        HEH^{-1}&=r^2E\\
        HFH^{-1}&=r^{-2}F\\
        [E,F]&=\frac{ H-H^{-1} }{ r-r^{-1} }
    \end{align}
\end{subequations}
and we consider the map \( \psi(H)=K^2\), \( \psi(E)=aX\), \( \psi(F)=bY\). We have
\begin{equation}
    \psi(HEH^{-1})=aK^2XK^{-1}=aKq^{1/2}XK^{-1}=aqX,
\end{equation}
and on the other hand \( r^2\psi(E)=ar^2X\), so that we must have \( q=r^2\). As for the commutator we have
\begin{equation}
    [\psi(E),\psi(F)]=ab[X,Y]=ab\frac{ K^{2}-K^{-2} }{ q^{1/4}-q^{-1/4} },
\end{equation}
so we need to fix \( a\) and \( b\) in such a way that 
\begin{equation}
    \frac{ ab }{ q^{1/4}-q^{-1/4} }=\frac{1}{ r-r^{-1} }.
\end{equation}
The algebras \eqref{SubEqsDefUrsldc} is then the same as the algebra \eqref{DefUqlG} with \( \lG=\gsl(2,\eC)\).

\begin{lemma}
    There is an unique automorphism of \( U_q\gsl(2,\eC)\) such that \( \omega(E)=F\), \( \omega(F)=E\) and \( \omega(K)=K^{-1}\).
\end{lemma}

\begin{proof}
    Unicity is automatic since \( \omega\) is defined on the generators. The point is only to see that it extends as an automorphism. As an example we have \( \omega(KEK^{-1})=K^{-1}FK\) while \( q^2\omega(E)=q^2F\), but equation \eqref{EqsiKFKqdGF} says that \( K^{-1}FK=q^2F\).

    The other relations are checked in the same way.
\end{proof}

\begin{lemma}
    Let \( m\geq 0\) and \( n\in\eZ\). We have the following relations:
    \begin{subequations}
        \begin{align}
            E^mK^n&=q^{-2mn}K^nE^m     \label{EqComUqslEmKn}\\
            F^mK^n&=q^{2mn}K^nF^m,
        \end{align}
    \end{subequations}
    and
    \begin{subequations}
        \begin{align}
            [E,F^m]&=[m]F^{m-1}\frac{ q^{-(m-1)}K-q^{m-1}K^{-1} }{ q-q^{-1} }       \label{EqComUqslEFm}\\
            &=[m]\frac{ q^{m-1}K-q^{-(m-1)}K^{-1} }{ q-q^{-1} }F^{m-1}      \label{EqComUqslEFmb}
        \end{align}
    \end{subequations}
    and
    \begin{subequations}
        \begin{align}
            [E^m,F]&=[m]\frac{ q^{-(m-1)}K-q^{m-1}K^{-1} }{ q-q^{-1} }E^{m-1}\\
            &=[m]E^{m-1}\frac{ q^{m-1}K-q^{-(m-1)}K^{-1} }{ q-q^{-1} }.
        \end{align}
    \end{subequations}
\end{lemma}

\begin{proof}
    We check the relations \eqref{EqComUqslEmKn} and \eqref{EqComUqslEFm}. The other can be deduced applying \( \omega\). For the first one, we use the commutator \eqref{EqUqslKEK} under the form \( EK=q^{-2}KE\). We have \( E^mK=q^{-2m}KE^m\).

    With \( m=1\) the relation \eqref{EqComUqslEFm} reduces to the definition of \( U_q\gsl(2,\eC)\). We proceed by induction on \( m\): assuming that the result is true with \( m-1\) we have
    \begin{equation}
        [E,F^{m}]=[E,F^{m-1}]F+F^{m-1}[E,F]
    \end{equation}
    in which we substitute
    \begin{equation}
        [E,F^{m-1}]F=[m-1]F^{m-2}\frac{ q^{-(m-2)}K-q^{m-2}K^{-1} }{ q-q^{-1} }F.
    \end{equation}
    Grouping the terms we find
    \begin{equation}
        [E,F^{m}]=\frac{ F^{m-1} }{ q-q^{-1} }\Big( [m-1](q^{-m}K-q^mK^{-1})+K+K^{-1} \Big)
    \end{equation}
    The coefficient with \( K\) in the parenthesis is
    \begin{subequations}
        \begin{align}
            [m-1]q^{-m}+1=\frac{ q^{m-1}-q^{-m+1} }{ q-q^{-1} }q^{-m}+1.
        \end{align}
    \end{subequations}
    Taking the common denominator we find
    \begin{equation}
        q^{-m+1}\frac{ q^m-q^{-m} }{ q-q^{-1} }=q^{-m+1}[m].
    \end{equation}
    Putting all together we find the result.
\end{proof}

\begin{corollary}       \label{CorvKEnvKvaep}
    If \( v\) is an eigenvector of \( K\) with eigenvalue \( \alpha\), and if \( v_n=E^nv\) is non vanishing, then it is an eigenvector with eigenvalue \( q^{2n}\alpha\). In particular the vectors \( v_n\) are distinct.
\end{corollary}
    
\begin{proof}
    This is a computations using the relation \eqref{EqComUqslEmKn}:
    \begin{equation}
        KE^nv=q^{2n}E^nKv=q^{2n}\alpha E^nv.
    \end{equation}
\end{proof}

Since \( \lG=\gsl(2,\eC)\), the Cartan algebra is one dimensional and the roots are just numbers. Let \( V\) be a \( U_q\gsl(2,\eC)\)-module. We consider, for \( \lambda\neq 0\), 
\begin{equation}
    V_{\lambda}=\{ v\in V\tq Kv=\lambda v \}.
\end{equation}
We say that \( \lambda\) is a \defe{weight}{weight!in $ U_q\gsl(2,\eC)$} is \( V_{\lambda}\neq\{ 0 \}\).

Let \( V\) be a \( U_q\gsl(2,\eC)\)-module and \( \lambda\in\eC\). One say that \( v\neq 0\) is a \defe{height weight vector}{highest weigh vector!for $U_q\gsl(2,\eC)$} of weight \( \lambda\) if \( Ev=0\) and \( Kv=\lambda v\). The module \( V\) is said to be a \defe{highest weight}{highest weight!module} if it is generated by a highest weight vector.

\begin{lemma}
    We have \( E V_{\lambda}\subset V_{q^2\lambda}\) and \( FV_{\lambda}\subset V_{q^{-2}\lambda}\).
\end{lemma}

\begin{proof}
    If \( v\in V_{\lambda}\) the relations \eqref{EqUqslKEK} and \eqref{EqsiKFKqdGF} imply
    \begin{equation}
        K(Ev)=q^2EKv=q^2\lambda(Ev)
    \end{equation}
    and
    \begin{equation}
        K(Fv)=q^{-2}FKv=q^{-2}\lambda(Fv).
    \end{equation}
\end{proof}

\begin{proposition}     \label{PropFDmodulehashiweightvec}
    Every finite dimensional \( U_q\gsl(2,\eC)\)-module has a highest weight vector.
\end{proposition}

\begin{proof}
    Since \( \eC\) is algebraically closed, the operator \( K\) has an eigenvector. Let \( Kw=\alpha v\) with \( \alpha\neq 0\). If \( Ew=0\), this is a highest weight vector. If not we consider \( w_n=E^nw\) (\( n\in\eN\)). By corollary \ref{CorvKEnvKvaep}, the vectors \( w_n\) are eigenvectors of \( K\) with distinct eigenvalues; thus there exists a \( n\) such that \( w_n\neq 0\) and \( w_{n+1}=0\).
\end{proof}

\begin{lemma}
    If \( V\) is a finite dimensional \( U_q\gsl(2,\eC)\)-module, the elements \( E\) and \( F\) are nilpotent as endomorphisms of \( V\).
\end{lemma}

\begin{proof}
    Let \( U\) be an unitary matrix such that \( U^*EU\) is upper diagonal (proposition \ref{PropMtrDiagablaUnit}). The eigenvalue of \( E\) and \( U^*EU\) are the same and are on the diagonal of \( U^*EU\). If we prove that the eigenvalue of \( E\) are zero, then \( U^*EU\) will be nilpotent and thus \( E\) itself will be nilpotent.

    Let \( Ev=\lambda v\). Since \( EKv=\lambda q^{-2}Kv\), the vectors \( K^nv\) are eigenvectors of \( E\) with distinct eigenvalues \( \lambda q^{-2n}\). This is impossible in a finite dimensional space, so \( \lambda=0\).
\end{proof}

\begin{proposition}
    Let \( V\) be a \( U_q\gsl(2,\eC)\)-module and \( v\) a highest weight vector of weight \( \lambda\). We consider the sequence \( v_0=v\)
    \begin{equation}
        v_p=\frac{1}{  [p]! }F^pv.
    \end{equation}
    Then each time \( v_p\) is nonzero we have
    \begin{subequations}        \label{subeqsActionUqsldcVp}
        \begin{align}
            K v_p&=\lambda q^{-2p}v_p       \label{EqKvpqdpvp}\\
            K^{-1} v_{p}&=\lambda^{-1}q^{2p}v_p\\
            E v_p&=\frac{ q^{-(p-1)}\lambda-q^{p-1}\lambda^{-1} }{ q-q^{-1} }v_{p-1}\\
            F v_{p-1}&=[p]v_p.
        \end{align}
    \end{subequations}
\end{proposition}

\begin{proof}
    For the first one, using the commutation relations,
    \begin{subequations}
        \begin{align}
            Kv_p&=\frac{1}{ [p]! }KF^pv\\
            &=\frac{1}{ [p]! }q^{2p}F^pKv\\
            &=\frac{ 1 }{ [p]! }\lambda q^{2p}F^pv\\
            &=\lambda q^{2p}v_p.
        \end{align}
    \end{subequations}
    The action of \( K^{-1}\) is immediately deduced from that one applying \( K\) on both sides. For the last one,
    \begin{equation}
        Fv_{p-1}=\frac{1}{ [p-1]! }F^pv=\frac{1}{ [p-1]! }[p]!v_p=[p]v_p.
    \end{equation}
    And for the second one we write \( EF^pv= [E,F^p]v+\underbrace{F^pEv}_{=0}\) in which we substitute \eqref{EqComUqslEFmb}.  We find
    \begin{equation}
        Ev_p=\frac{ q^{p-1}K-q^{-(p-1)}K^{-1} }{ q-q^{-1} }v_{p-1}.
    \end{equation}
    We already know the action of \( K\) and \( K^{-1}\) on \( v_{p-1}\).
\end{proof}

\begin{corollary}
    The vectors \( v_p\) are eigenvectors of \( K\) with distinct eigenvalues.
\end{corollary}

\begin{proof}
    This is equation \eqref{EqKvpqdpvp} and the fact that \( q\) is not a root of unity.
\end{proof}

The structure of the simple \( U_q\gsl(2,\eC)\)-modules is described by the theorems \ref{ThoVfintemofdsldcun} and \ref{ThoVfintemofdslddeux}.

\begin{theorem}     \label{ThoVfintemofdsldcun}
    Let \( V\) be a finite dimensional \( U_q\gsl(2,\eC)\)-module generated by a highest weight vector \( v\) of weight \( \lambda\). Then
    \begin{enumerate}
        \item       \label{ItemThoVintmoddcuni}
            The weight \( \lambda\) reads \( \lambda=\epsilon q^n\) where \( \epsilon=\pm 1\) and \( n=\dim(V)-1\).
        \item       \label{ItemThoVintmoddcunii}
            Set \( v_0=v\) and \( v_p=\frac{1}{ [p]! }F^pv\). We have \( v_p=0\) with \( p>n\) and the set
            \begin{equation}
                \{ v_0,v_1,\ldots,v_n \}
            \end{equation}
            is a basis of \( V\).
        \item       \label{ItemThoVintmoddcuniii}
            The action of \( K\) on \( V\) is diagonalisable and has the \( m+1\) distinct eigenvalues
            \begin{equation}
                \{ \epsilon q^n,\epsilon q^{n-2},\cdots,\epsilon q^{-n+2},\epsilon q^{-n} \}.
            \end{equation}
        \item       \label{ItemThoVintmoddcuniv}
            Any highest weight vector in \( V\) is a multiple of \( v\).
    \end{enumerate}
\end{theorem}

\begin{theorem}     \label{ThoVfintemofdslddeux}
    A \( U_q\gsl(2,\eC)\)-module is simple if and only if it is generated by an highest weight. Moreover if two modules are generated by a highest weight vector of same weight, they are isomorphic as \( U_q\gsl(2,\eC)\)-modules.
\end{theorem}

\begin{proof}[Proof of theorem \ref{ThoVfintemofdsldcun}]
    We know that the vectors \( v_p\) are eigenvector of \( K\) with distinct eigenvalues. Thus there exits \( n\in\eN\) such that \( v_n=0\) and \( v_{n+1}=0\). For that value of \( n\) we have \( v_m=0\) \( \forall m>n\) and \( v_m\neq 0\) when \( m\leq n\). We also have
    \begin{equation}
        0=Ev_{n+1}=\frac{ q^{-n}\lambda-q^n\lambda^{-1} }{ q-q^{-1} }v_n,
    \end{equation}
    so that \( q^{-n}\lambda- q^n\lambda^{-1}=0\). This implies \( \lambda=\pm q^n\).
    
    Let us now prove that \( \{ v_0,\ldots,v_n \}\) is a basis of \( V\). This will show that \( \dim(V)=n+1\). First we know that the vectors \( v_i\) are eigenvectors of \( K\) for distinct eigenvalues, so that the set \( \{ v_i \}\) is free. By hypothesis, the vector space \( V\) is generated by \( v_0\). From the relations \eqref{subeqsActionUqsldcVp}, we see that the action of \( U_q\gsl(2,\eC)\) on the vectors \( v_i\) will only generate linear combinations of the vectors \( v_i\). Thus the set of \( v_i\)'s is generating. The points \ref{ItemThoVintmoddcuni} and \ref{ItemThoVintmoddcunii} are proved.

    For item \ref{ItemThoVintmoddcuniii}, the operator \( K\) is diagonalisable because the vectors \( v_p\) form a basis of eigenvectors of \( K\). The eigenvalues are given by the relation \( Kv_p=\lambda q^{-2p}v_p\). Since \( \lambda=\epsilon q^n\) we have the eigenvalues
    \begin{equation}
        \{ \epsilon q^n,\epsilon q^{n-2},\ldots,\epsilon q^{-n} \}
    \end{equation}
    corresponding to \( p=0,\ldots,n\).

    In order to prove point \ref{ItemThoVintmoddcuniv} let \( v'\) be an other highest weight vector. By definition \( Ev'=0\) and there exists an \( \lambda'\in\eC\) such that \( Kv'=\lambda'v'\). Since \( v'\) is eigenvector of \( V\), this is up to scalar multiple one of the vectors \( v_i\). From the constraint \( Ev_i=0\) we see that \( i=0\) so that \( v'\) is multiple of \( v_0=v\).

\end{proof}

\begin{proof}[Proof of theorem \ref{ThoVfintemofdslddeux}]
    First we suppose that \( V\) is generated by an highest weight \( v\) and we prove that it is simple. Let\( V'\) be a submodule of \( V\). By proposition \ref{PropFDmodulehashiweightvec} \( V'\) has an highest weight vector \( v'\). The vector \( v'\) is highest weight for \( V\) also, thus \( v'\) is a multiple of \( v\). Since \( v\in V'\) we have \( V\subset V'\) and consequently \( V=V'\).

    Let \( V\) be a simple module, $v$ a highest weight vector in \( V\) and \( V'\) the subspace of \( V\) generated by \( v\). The space \( V'\) is a submodule of \( V\) while \( V\) is simple, so \( V'=V\) and \( v\) generated \( V\).


    Finally let \( V\) be generated by \( v\) of weight \( \lambda\) and \( V'\) be generated by \( v'\) of same weight \( \lambda\). Since
    \begin{equation}
        \lambda=\epsilon q^n=\epsilon'q^{n'}
    \end{equation}
    we have \( \epsilon=\epsilon'\) and\footnote{Once again we use the fact that \( q\) is not a root of unity.} \( n=n'\), so the modules \( V\) and \( V'\) have same dimension \( n+1\). The set \( \{ v_p=F^pv/[p]! \}\) is a basis of \( V\) while the set \( \{ v'_p=F^pv'/[p]! \}\) is a basis of \( V'\). One checks that the map \( \psi\colon V\to V'\), \( \psi(v_i)=v'_i\) is an isomorphism of \( U_q\gsl(2,\eC)\)-modules.
\end{proof}

The conclusion is that the \( U_q\gsl(2,\eC)\)-modules are classified up to isomorphisms by their dimension \( n+1\) and \( \epsilon=\pm 1\). We denote by \( V_{\epsilon,n}\) the module of dimension \( n+1\) with highest weight \( \lambda=\epsilon q^n\). The corresponding representation \( \rho_{\epsilon,n}\colon U_q\gsl(2,\eC)\to \End(V_{\epsilon,n})\) is given by
\begin{equation}
    \begin{aligned}[]
        Kv_p&=\epsilon q^{n-2p}v_p\\
        Ev_p&=\epsilon[n-p+1]v_{p-1}\\
        Fv_{p-1}&=[p]v_p
    \end{aligned}
\end{equation}
with \( p=0,\ldots,n\).

%+++++++++++++++++++++++++++++++++++++++++++++++++++++++++++++++++++++++++++++++++++++++++++++++++++++++++++++++++++++++++++
\section{Quantized function algebra}
%+++++++++++++++++++++++++++++++++++++++++++++++++++++++++++++++++++++++++++++++++++++++++++++++++++++++++++++++++++++++++++

\begin{definition}
    Let \( G\) be the connected simply connected Lie group with Lie algebra \( \lG\). The \defe{quantized algebra of regular functions}{quantized!algebra of regular functions} on the group \( G\) is the Hopf \( *\)-subalgebra of matrix elements of the unitarizable finite dimensional \( U_q\lG\)-modules. We denote it by \( \eC[G]_q\)\nomenclature[A]{\( \eC[G]_q\)}{quantized algebra of regular functions}
\end{definition}

The multiplication law in \( \eC[G]_q\) is given by formula \eqref{subeqDualHopfmult}:
\begin{equation}
    c^{\Lambda'}_{l',v'}c^{\Lambda}_{l,v}(x)=(l'\otimes l)\rho_{\Lambda'\otimes \Lambda}(x)(v'\otimes v)=(l'\otimes l)(\Delta x)(v'\otimes v),
\end{equation}
so
\begin{equation}
    c^{\Lambda'}_{l',v'}c^{\Lambda}_{l,v}=c^{\Lambda'\otimes\Lambda}_{l'\otimes l,v'\otimes v}
\end{equation}
where \( \Lambda'\otimes\Lambda\) stands for the representation of \(U_q\lG\otimes U_q\lG\) on \( L(\Lambda')\otimes L(\Lambda)\). We can be more explicit if we write \( \Delta x=\sum_i a_i\otimes b_i\), we have
\begin{equation}
    c^{\Lambda}_{l,v}c^{\Lambda'}_{l',v'}(x)=\sum_i c^{\Lambda}_{l,v}(a_i)c^{\Lambda'}_{l',v'}(b_i)=\sum_il\big( \rho_{\Lambda}(a_i)v \big)l'\big( \rho_{\Lambda'}(b_i)v' \big)
\end{equation}
where \( \rho_{\Lambda}\) stands for the representation on \( L(\Lambda)\). At the end the product \( c^{\Lambda'}_{l',v'}c^{\Lambda}_{l,v}\) is still an element in \( \eC[G]_q\) and so a linear form on \( U_q\lG\).

\begin{remark}
    In the setting of Hopf algebra we consider the notion of involution of the definition \ref{DefInvolutionHopf} in which we do not require \( a^{**}=a\).
\end{remark}

\begin{proposition}
    If \( q\in\eR_0\) the map \( \omega\colon U_q\lG\to U_q\lG\) defined on the generators by
    \begin{subequations}
        \begin{align}
            \omega(X^{\pm}_{i})&=X^{\mp}_i\\
            \omega(K_i)&=K_i.
        \end{align}
    \end{subequations}
\end{proposition}
One also speaks about that structure in \cite{SoilSchubert}.

Since \( U_q\lG\) is a Hopf \( *\)-algebra, the dual also becomes a Hopf \( *\)-algebra by proposition \ref{PropAstarstaralg}. The algebra \( (U_q\lG)^*\) contains in particular the matrix elements of \( U_q\lG\)-modules, so the space \( \eC[G]_q\) is a Hopf \( *\)-algebra.

Let \( V=L(\Lambda)\) be the simple admissible \( U_q\lG\)-module of highest weight \( \Lambda\in P_+\) where \( P_+\) is the set of dominant weights in \( \lH^*\). Since it is admissible we have the decomposition
\begin{equation}
    L(\Lambda)=\bigoplus_{\lambda\in\lH^*}L(\Lambda)_{\lambda}
\end{equation}
with
\begin{equation}
    L(\Lambda)_{\lambda}=\{ v\in L(\Lambda)\tq K_iv=q^{(\lambda,\alpha_i)}v \}.
\end{equation}
By general theory (see subsection \ref{SubSecMOdulREepe}), the dual vector space \( L(\Lambda)^*\) is also a \( U_q\lG\)-module. We chose on \( L(\Lambda)^*\) the left \( U_q\lG\)-module structure \( (a\cdot l)=\mL(a)l\) in the sense of definition \eqref{EqDefTroisleftmodAstar}, that means
\begin{equation}
    (a\cdot l)(v)=l\big( S^{-1}(a)v \big)
\end{equation}
for any \( a\in U_q\lG\), \( l\in L(\Lambda)^*\) and \( v\in L(\Lambda)\).

\begin{lemma}
    The module \( L(\Lambda)^*\) accepts the decomposition
    \begin{equation}        \label{EqLLamstdecompllsma}
        L(\Lambda)^*=\bigoplus_{\lambda\in\lH^*}L(\Lambda)_{\lambda}^*
    \end{equation}
    where \( L(\Lambda)_{\lambda}^*=\big( L(\Lambda)_{-\lambda} \big)^*\).
\end{lemma}

\begin{proof}
    As vector space the decomposition \eqref{EqLLamstdecompllsma} is nothing else that \( (A\oplus B)^*=A^*\oplus B^*\). Let \( \alpha\in\big( L(\Lambda)_{\lambda} \big)^*\). By definition \eqref{EqDefTroisleftmodAstar} and using the antipode \eqref{subEqantpUqGl} if \( v\in L(\Lambda)_{\lambda}\) we have
    \begin{equation}
        (K_i\alpha)v=\alpha\big( S^{-1}(K_i)v \big)=\alpha(K_i^{-1}v) =\alpha\big( q^{-(\lambda,\alpha_i)}v \big) =q^{-(\lambda,\alpha_i)}\alpha(v).
    \end{equation}
    Thus
    \begin{equation}
        K_i\alpha=q^{-(\lambda,\alpha_i)}\alpha.
    \end{equation}
    This proves that 
    \begin{equation}
        \big( L(\Lambda)_{\lambda} \big)^*=\big( L(\Lambda)^* \big)_{-\lambda}.
    \end{equation}
\end{proof}

From a notational point of view, what we write \( L(\Lambda)^*_{\lambda}\) is
\begin{equation}
    L(\Lambda)^*_{\lambda}=\big( L(\Lambda)^* \big)_{\lambda}=\big( L(\Lambda)_{-\lambda} \big)^*   
\end{equation}

We can consider the regular left and right representations of \( U_q\lG\) on \( \eC[G]_q\), since the latter is a part of the dual of \( U_q\lG\). The space \( \eC[G]_q\) becomes a \( U_q\lG\otimes U_q\lG\)-module with the representation \( \mL\otimes \mR\) described around equation \eqref{EqmRmLAAsurAstar}.
\begin{proposition}
    The map
    \begin{equation}        \label{EqhomoCGqLLstar}
        \begin{aligned}
            \psi\colon \eC[G]_q&\to \bigoplus_{\Lambda\in P_+}L(\Lambda)^*\otimes L(\Lambda) \\
            c^{L(\Lambda)}_{l,v}&\mapsto l\otimes v 
        \end{aligned}
    \end{equation}
    is an isomorphism of \( U_q\lG\otimes U_q\lG\)-modules. Here \( P_+\) is the set of dominant weights of \( \lG\). 
\end{proposition}

\begin{proof}
    The fact that \( \psi\) is bijective is contained in the definition of \( \eC[G]_q\). The point is to check that this is a morphism. Let \( a\otimes b\in U_q\lG\otimes U_q\lG\). The same computation as in \eqref{EqmRmLabf} shows that
    \begin{equation}
        \big( (a\otimes b)\cdot c^{\Lambda}_{l,v} \big)(x)=\big( \mL(a)\mR(b)c_{l,v}^{\Lambda} \big)(x)=c^{\Lambda}_{l,v}\big( S^{-1}(a)xb \big)=l\big( S^{-1}(a)xbv \big)=c^{\Lambda}_{\mL(a)l,bv}(x),
    \end{equation}
    so we have
    \begin{equation}
        \psi(a\otimes b)\cdot c_{l,v}^{\Lambda}=\mL(a)l\otimes bv.
    \end{equation}
    On the other hand we have
    \begin{equation}
        (a\otimes b)\psi c_{l,v}^{\Lambda}=(a\otimes b)l\otimes v=\mL(a)l\otimes bv.
    \end{equation}
    The map \( \psi\) is then an homomorphism.

    The fact that the sum in \eqref{EqhomoCGqLLstar} only runs over the dominant weight comes from subsection \ref{SubSecDomiunSei}.
\end{proof}

\begin{proposition}
    Let \( w_0\) be the element of the Weyl group which sends positive roots to negative ones\footnote{See theorem \ref{ThoLOngestlowestrepres}}. Then we have
    \begin{equation}        \label{EqPropcclvlvppp}
        c^{\Lambda}_{-\Lambda,\lambda}c^{\Lambda'}_{-w_0\Lambda',\Lambda'}=q^{-2\langle \Lambda, \lambda\rangle }c^{\Lambda'}_{-w_0\Lambda',\Lambda'}c^{\Lambda}_{-\Lambda,\lambda}.
    \end{equation}
\end{proposition}

\begin{probleme}
    Formula \eqref{EqPropcclvlvppp} is not the one given in \cite{SoibelmanI}, equation (2.1.4) page 98. I don't know where I got wrong in the computation of the proof.
\end{probleme}

\begin{proof}
    Consider the elements \( c^{\Lambda}_{l,v}\) and \( c_{l',v'}^{\Lambda'}\) in \( c^{\Lambda}_{-\Lambda,\lambda} \) and \( c^{\Lambda'}_{-w_0\Lambda',\Lambda'}\) with
    \begin{equation}
        \begin{aligned}[]
            l&\in L(\Lambda)^*_{-\Lambda},&v&\in L(\Lambda)_{\lambda}\\
            l'&\in L(\Lambda')^*_{-w_0\Lambda'},&v'&\in L(\Lambda')_{\Lambda'}.
        \end{aligned}
    \end{equation}
    First notice that \( l'(v')=0\) because \( l'\in \big( L(\Lambda')_{w_0\Lambda'} \big)^*\) while \( v'\in L(\Lambda')_{\Lambda'}\) and \( w_0\Lambda'\neq \Lambda'\). In the same way \( \rho_{\Lambda'}(X_i^+)v'=0\) since \( v'\) is a highest weight vector in \( L(\Lambda')\).

    The product \( c^{\Lambda}_{l,v}c^{\Lambda'}_{l',v'}\) is defined by the equation \eqref{subeqDualHopfmult}. If we apply it to \( K_i\) we find
    \begin{equation}
        \begin{aligned}[]
            (c^{\Lambda}_{l,v}c^{\Lambda'}_{l',v'})(K_i)&=(c^{\Lambda}_{l,v}\otimes c^{\Lambda'}_{l',v'})(\Delta K_i)\\
            &=l\big( \rho_{\Lambda}(K_i)v \big)l'\big( \rho_{\Lambda'}(K_i)v' \big)\\
            &=0
        \end{aligned}
    \end{equation}
    because \( \rho_{\Lambda'}(K_i)v'\) is a multiple of \( v'\). The same shows that \( (c^{\Lambda'}_{l',v'}c^{\Lambda}_{l,v})(K_i)=0\).

    We still have to check the equality on \( X_i^{\pm}\). On the left hand side of \eqref{EqPropcclvlvppp} we get
    \begin{subequations}
        \begin{align}
            (c_{l,v}^{\Lambda}c^{\Lambda'}_{l',v'})(X_i^{\pm})&=(c_{l,v}^{\Lambda}\otimes c^{\Lambda'}_{l',v'})(X_i^{\pm}\otimes K_i+K_i^{-1}\otimes X_i^{\pm})\\
            &=l\big( \rho_{\Lambda}(X_i^{\pm})v \big)\underbrace{l'\big( \rho_{\Lambda'}(K_i)v' \big)}_{=0}+l\big( \rho_{\Lambda}(K_i^{-1})v \big)l\big( \rho_{\Lambda'}(X_i)^{\pm}v' \big)\\
            &=q^{-\langle \Lambda, \lambda\rangle }l(v)l'\big( \rho_{\Lambda'}(X_i^{\pm})v' \big).
        \end{align}
    \end{subequations}
    On the right hand side we have
    \begin{subequations}
        \begin{align}
            (c_{l',v'}^{\Lambda'}c_{l,v}^{\Lambda})(X_i^{\pm})&=l'\big( \rho_{\Lambda'}(X_i^{\pm})v' \big)l\big( \rho_{K_i}v \big)+
            \underbrace{l'\big( \rho_{\Lambda'}(K_i^{-1})v' \big)}_{=0} l\big( \rho_{\Lambda}(X_i^{\pm})v \big)\\
            &=q^{\langle \Lambda, \lambda\rangle }l'\big( \rho_{\Lambda'}(X_i^{\pm})v' \big)l(v).j
        \end{align}
    \end{subequations}
\end{proof}
