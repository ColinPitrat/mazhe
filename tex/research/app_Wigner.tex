\section{Alternative formalism for the quantum mechanics}\label{app:Wigner}
%++++++++++++++++++++++++++++++++++++++++++++++++++++++++

We can a little reformulate the axioms of the quantum mechanics. Since we are in a Hilbert space $\hH$ we can speak about orthogonal projections; if $\phi\in\mR$, we can consider the projection on the space spanned by $\phi$:
\[
   P_{\phi}e_k=\frac{\scal{\phi}{e_k}}{\|\phi\|}\phi
\]
where $\{e_i\}$ is a basis of $\hH$. It is pretty clear that 
\begin{equation}
\tr(P_{\phi} P_{\psi})=\frac{| \scal{\psi}{\phi} |^2}{\|\psi\|\|\phi\|}.
\end{equation}
If $\psi\in\mR$ and $\phi\in\mR'$ are unimodular, then
\begin{equation}
P(\mR\to\mR')=\tr(P_{\phi}P_{\psi}),
\end{equation}
so we can express the axioms in terms of projections instead of rays. For notational convenience, we put
\begin{equation}
\hH_1=\{\psi\in\hH\tq\|\psi\|=1\}.
\end{equation}

We denote by $\hS$ the space of the projections into one dimensional subspaces of $\hH$ (in other words $\hS$ is the space of physical states) and for $P$, $Q\in\hS$, the transition probability is $P\cdot Q=\tr(PQ)$.
Now a \defe{quantum symmetry}{symmetry!of quantum system} is a map $\dpt{T}{\hS}{\hS'}$ such that $(TP)\cdot(TQ)=P\cdot Q$.

One can prove the following :

\begin{theorem}
If $\dpt{T}{\hS}{\hS'}$ is a quantum symmetry, then there exists an operator $\dpt{U}{\hH}{\hH'}$ such that
\begin{enumerate}
\item $P_{U\phi}=TP_{\phi}$,
\item $U(\xi+\eta)=U(\xi)+U(\eta)$,
\item $\scal{U\xi}{U\eta}=\kappa(\scal{\xi}{\eta})$ \label{item:cond_3}
\end{enumerate}
where $P_{\psi}$ is the projection onto the one dimensional space spanned by $\psi$ and $\dpt{\kappa}{\eC}{\eC}$ fulfils $\kappa(\lambda)=\lambda$ or $\kappa(\lambda)=\overline{\lambda}$ and 

\begin{enumerate}
\item $U(\lambda\xi)=\kappa(\lambda)\xi$.
\end{enumerate}
\label{tho:pre_Wigner}
\end{theorem}
Here is why this implies Wigner's theorem as given by theorem \ref{tho:Wigner}. Let us consider some $\varphi_i\in\hH$ such that $\|\varphi_i\|=1$ and $P_{\varphi_i}$, the corresponding projections. Let
\[
   \Delta(P_1,P_2,P_3)=\scal{\varphi_1}{\varphi_2}\scal{\varphi_2}{\varphi_3}\scal{\varphi_3}{\varphi_1}.
\]
It is clear that this expression doesn't depend on the choice of $\varphi_i$ in its ray. We have
\begin{equation}\label{eq:Delta_T}
\begin{split}
\Delta(TP_1;TP_2,TP_3)&=\Delta(P_{U\varphi_1},P_{U\varphi_2},P_{U\varphi_3})\\
                      &=\scal{U\varphi_1}{U\varphi_2}\scal{U\varphi_2}{U\varphi_3}\scal{U\varphi_3}{U\varphi_1}\\
		      &=\kappa(\scal{U\varphi_1}{U\varphi_2})\kappa(\scal{U\varphi_2}{U\varphi_3})\kappa(\scal{U\varphi_3}{U\varphi_1})\\
		      &=\kappa\big(\Delta(P_1,P_2,P_3)\big).
\end{split}
\end{equation}
We can see from this that the choice of $\kappa(\lambda)=\lambda$ or $\kappa(\lambda)=\overline{\lambda}$ is determined by the data of $T$ if $\dim\hH\geq 2$. In the case where $\dim\hH=1$, $\Delta$ is always equals to $1$ and the equality \eqref{eq:Delta_T} don't give any informations. In the case $\dim\hH\geq 2$, we can choice $\varphi_1$ and $\varphi_2$ such that $\scal{\varphi_1}{\varphi_2}$ takes any value $z\in\eC$ with $\|z\|\leq 1$. Taking $\varphi_3=\varphi_1+\varphi_2$, we find
\[
  \Delta(P_1,P_2,P_3)=z(1+\overline{z})^2.
\]
which is easily non real for a suitable choice of $z\in\eC$. Let us suppose that we have an operator $U$ which satisfies the theorem \ref{tho:pre_Wigner}. If $\kappa(\lambda)=\lambda$, then
\begin{equation}
U(z\psi+z'\phi)=U(z\psi)+U(z'\phi)
               =zU(\psi)+zU(\phi)
\end{equation}
and 
\begin{equation}
\scal{U\psi}{U\phi}=\kappa(\scal{\psi}{\phi})
	=\scal{\psi}{\phi},
\end{equation}
so that $U$ is linear. If $\kappa(\lambda)=\overline{z}$, then 
\begin{equation}
 U(z\psi)=\overline{z}U\psi
\end{equation}
and
\begin{equation}
\scal{U\xi}{U\eta}=\kappa(\scal{\xi}{\eta})
                  =\overline{\scal{\xi}{\eta}}.
\end{equation}


% \begin{proof}[Proof of theorem \ref{tho:pre_Wigner}]
% We begin with $\dim\hH=1$. We have $\dpt{T}{\hS}{\hS'}$; for a given $\varphi\in\hH_1$, we can finf $\varphi'\in\hH_1'$ such that $P_{\varphi'}=TP_{\varphi}$ and define $U\varphi=\varphi'$. We define $U$ on the other vectors of $\hH$ by $U(z\varphi)=\kappa(z)U\varphi$ and $U(\xi+\eta)=U(\xi)+U(\eta)$. In order to see that $\scal{U\xi}{U\eta}$, remark that there exists $z,z'\in\eC$ such that $\eta=z\xi$ and $\xi=z'\varphi$. Then
% 
% \begin{equation}
% \begin{split}
% \scal{U\xi}{U\eta}&=\scal{U\xi}{U\xi}\kappa(z)\\
%                   &=\overline{\kappa(z')}\scal{U\varphi}{U\varphi}\kappa(z)\kappa(z')\\
% 		  &=\overline{\kappa(z')}\kappa(z)\kappa(z'),
% \end{split}
% \end{equation}
% but 
% \begin{equation}
% \begin{split}
% \scal{\xi}{eta}&=\scal{z'\varphi}{zz'\varphi}\\
%                &=\overline{z'}zz'.
% \end{split}
% \end{equation}
% 
% Now we turn our attention to the case $\dim\hH\geq2$. The condition $P_{U\varphi}=TP_{\varphi}$ fix $U$ up to a complex number: if $U(\xi)$ works, then $U'(\xi)=z(\xi)U(\xi)$ equaly works. The condiction \ref{item:cond_3} gives
% 
% \begin{equation}
% \begin{split}
%   \kappa(\scal{\xi}{\eta})&=\scal{U'\xi}{U'\eta}\\
%                           &=\overline{z(\xi)}z(\eta)\kappa(\scal{\xi}{\eta})
% \end{split}
% \end{equation}
% for any $\eta,\xi\in\hH$. Then $z$ is a constant of module $1$. Let $\xi'$ be a candidate to be $U\xi$. We can define $\varphi$ by $\xi=\|\xi\|\varphi$ (then $\|\varphi\|=1$). Since $P_{U\varphi}=TP_{\varphi}$ and $U\varphi=\xi'$ up to a multiplicative constant, then $\xi'$ is an arbitrary vector of nomr $\|\xi\|$ in the space given by $TP_{\varphi}$. Then
% 
% \[
%   \|\scal{\xi'}{\eta'}\|=\|\scal{\xi}{\eta}\|
% \]
% beause $\kappa$ preserves the norms. If $\{\varphi_1\ldots,\varphi_m\}$ is an orthonormal set of vectors in $\hH$, then $\{\varphi_1,\ldots,\varphi_m\}$ is a set of orthonormal vectors in $\hH'$ because $\|\scal{\varphi_i'}{\varphi'_j}\|=\|\scal{\varphi_i}{\varphi_j}|=\delta_{ij}$. Note that $\scal{\varphi_i}{\varphi_i}=1$ because is must be real and positive.
% 
% \end{proof}
