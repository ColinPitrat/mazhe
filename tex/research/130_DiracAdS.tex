% This is part of (almost) Everything I know in mathematics and physics
% Copyright (c) 2013-2014
%   Laurent Claessens
% See the file fdl-1.3.txt for copying conditions.

\section{Dirac operator on \texorpdfstring{$AdS_{3}$}{AdS3}}  \label{PgDiracAdSTrois}
%--------------------------------------------------------------

The definition is 
\[ 
  AdS_{3}=\frac{ \SO(2,2) }{ \SO(1,2) },
\]
and the group which acts is the $AN$ of $\SO(2,2)$. The Lie algebra is given by
\[ 
\begin{split}
  \sA&=\{ J_{1},J_{2} \}\\
  \sN&=\{ M,L \}
\end{split}  
\]
which has dimension $4$. So there is a stabiliser. One can prove that for the open orbit of $u=\begin{pmatrix}
0&1\\-1&0
\end{pmatrix}$, the stabiliser is $\{  e^{aJ_{2}} \}$, i.e.
\begin{equation}
  [ e^{aJ_{2}}u]=[u].
\end{equation}
For the spin group, we find 
\[ 
  \Spin(2,1)\simeq SL_2^*(\eR),
\]
the group of $2\times 2$ matrices with determinant equals to $\pm 1$ (cf \cite{Michelson}). Let us recall that the isomorphism $AdS_{3}\simeq SL(2,\eR)$ is given by
\[ 
  SL(2,\eR)=\begin{pmatrix}
t+x&y-u\\
y+u&t-x
\end{pmatrix}
\]
with $u^{2}+t^{2}-x^{2}-y^{2}=1$. For sake of simplicity, we denote $SL(2,\eR)$ by $G$. It is explained in \cite{Clement} that the map
\begin{equation}
\begin{aligned}
 \psi\colon (G\times G)\times AdS_{3}&\to AdS_3 \\ 
(g_{1},g_{2})x&= g_{1}xg_{2}^{-1} 
\end{aligned}
\end{equation}
provides a local isomorphism $G\times G\simeq O(2,2)$. Moreover we have locally:
\[ 
  \frac{ G\times G }{ \eZ_{2} }\simeq \SO(2,2).
\]
At each point $x\in AdS_3$, we have an isomorphism
\[ 
  \SO(2,2)_{x}\simeq \SO(2,1)
\]
where $\SO(2,2)_{x}$ is the stabiliser of $x$ in $\SO(2,2)$. So we define the isomorphism
\[ 
  \chi_{x}\colon \Spin(2,1)\to \SO(2,2)_{x}
\]
which is a double covering. If $d\psi\colon \mG\oplus\mG\to \mathfrak{so}(2,2)$ is the isomorphism of \cite{Clement}, we define $\psi\colon G\times G\to \SO(2,2)$ by
\[ 
  \psi( e^{X})= e^{d\psi X},
\]
which is a good definition because the exponential is surjective on $G\times G$. For each $x\in AdS_3$, we consider the isomorphism
\[ 
  \phi_{x}\colon \SO(2,1)\to \SO(2,2)_{x}
\]
such that $\phi_{x}\big( \SO(2,1) \big)=\SO(2,2)_{x}$.

We define $\chi(s)_i\colon \Spin(2,1)\to \SO(2,2)$ by
\[ 
  \chi(s)=\chi(s)_1v\chi(s)_2.
\]
The choice of $\chi(s)_i$ is not unique. So we define the action of $\Spin(2,1)$ on $G\times G$ by
\begin{equation}
(g,h)\cdot s=\big( \chi(s)_1g,\chi(s)_2^{-1}h \big).
\end{equation}
Therefore we have
\[ 
\begin{split}
  \psi\big( (g,h)\cdot s \big)x&=\chi(s)_1gxh^{-1}\chi(s)_2\\
        &=\chi(s)\big( gxh^{-1} \big)\\
        &=\chi(s)\big(\psi(g,h)x\big).
\end{split}  
\]
\subsection{Spin structure on \texorpdfstring{$AdS_3$}{AdS3}}
%+++++++++++++++++++++++++++++++++++

From previous considerations, the first choice should be
\[ 
  P=\frac{ AN }{ S }\times\Spin(2,1),
\]
but it is easy to remark that $\sR'=\{ J_{1},M,L \}$ is a Lie algebra. So we use the corresponding Lie group $R$ instead of the homogeneous space $AN/S$ (these two are isomorphic). Thus the choice is
\begin{equation}
P=R'\times\Spin(2,1),
\end{equation}
with the projection $\pi\colon P\to AdS_3$,
\[ 
  \pi\big( r',s \big)=\left[ r'\begin{pmatrix}
0&1\\-1&0
\end{pmatrix} \right]
\]
 We consider
\begin{equation}
\begin{aligned}
 \theta\colon R'&\to \mU=Ro \\ 
r'&\mapsto ro=\left[ r'\begin{pmatrix}
0&1\\-1&0
\end{pmatrix} \right].
\end{aligned}
\end{equation}
The projection $\pi\colon P\to \mU$ reads $\pi=\theta\circ\pr_{1}$,
\[ 
  \pi\big( r',s \big)=[ro].
\]
This definition works because for all $a$, there exists a $h\in H$ such that
\[ 
   e^{aJ_{2}}\begin{pmatrix}
0&1\\-1&0
\end{pmatrix}=
\begin{pmatrix}
0&1\\-1&0
\end{pmatrix}h,
\]
from construction of $S$. Then we look at the following:
\[ 
\xymatrix{%
   \Spin(2,1) \ar@{~>}[r]&R'\times\Spin(2,1)\ar[rr]^{\displaystyle\varphi}\ar[rd]_{\displaystyle\pi}&&\SO\big( \mU \big)\ar[ld]&\SO(2,1)\ar@{~>}[l]   \\
  &&\mU
}
\]
The action of $\Spin(2,1)$ on $P$ is
\[ 
  \big( r',s' \big)\cdot s=\big( r',s's \big).
\]
In order to define $\varphi$, we consider the isomorphism
\[ 
  \phi_{x}\colon \SO(2,1)\to \SO(2,2)_{x}
\]
between $\SO(2,1)$ and the stabiliser of $x$ in $\SO(2,2)$. This extends to an automorphism
\[ 
  \phi_{x}\colon \SO(2,2)\to \SO(2,2),
\]

\begin{probleme}
    I'm not sure of that extension, but we do not use it here.
\end{probleme}


and we define the action of $\SO(2,1)$ on $\SO\big( AdS_3 \big)$ by
\begin{equation}
\{ b_{i} \}_{x}\cdot g=\{ \phi_{x}(g)b_{i} \}_{x}.
\end{equation}
Then we define
\begin{equation}
  \varphi\big( r',s \big)=\{ \phi_{\pi r'}\big( \chi(s) \big)b_{i} \}_{\pi r'}
\end{equation}
 if $\{ b_{i} \}$ is a reference basis at $\pi[r]$. So this construction implies the choice of a section of $\SO\big( AdS_3 \big)$. Now, using the fact that both $\phi_{x}$ and $\chi$ are morphisms, we find 
\begin{equation}
\begin{split}
\varphi\big( (r',s)\cdot s' \big)&=\left\{ \phi_{\pi r'}\big( \chi(ss') \big) \right\}_{\pi r'}\\
            &=\left\{ \phi_{\pi r'}\big( \chi(s) \big)b_{i} \right\}\cdot\chi(s)\\
            &=\varphi(r',s)\cdot\chi(s').
\end{split}
\end{equation}
This proves that the construction gives a spin structure.

\subsection{Connection on the spinor bundle}
%-----------------------------------------------

A left invariant vector on $\mU$ is of the form
\[ 
  \tilde X_{xo}=\Dsdd{ x e^{tX}o }{t}{0}
        =\Dsdd{ \pi\big( x e^{tX},s \big) }{t}{0}
\]
for any $s\in\Spin(2,1)$. On $AdS_3$ (in fact on $\mU$) we consider the left invariant vector field
\begin{equation}
X^{\sharp}_{[x]}=\Dsdd{ [x e^{tX}] }{t}{0}
\end{equation}
which leads us to consider the following field on $P$:
\begin{equation}
\xi_{X}\big( r',s \big)=\Dsdd{ r' e^{tX},s }{t}{0}\in T_{( r',s )}P.
\end{equation}
This defines a field which projects to the left invariant field on $\mU$:
\begin{equation}   \label{eq_xiXprojXsharp}
d\pi\xi_{X}(r',s)=X^{\sharp}_{r'}.
\end{equation}  


\begin{lemma}
On the general vector 
\begin{equation}   \label{eq_gebevectSig}
  \Sigma=\Dsdd{ r'(t),s(t) }{t}{0},
\end{equation}
the formula
\begin{equation}
\alpha_{(r',s_{0})}\Sigma=-\Dsdd{ s_{0}^{-1}s(t) }{t}{0}\in\spin(2,1)
\end{equation}
where $s_{0}=s(0)$ defines a connection form.

\end{lemma}

\begin{proof}
First let $A\in\spin(2,1)$ and
\[ 
  A^*_{\xi}=\Dsdd{ \xi\cdot e^{-tA} }{t}{0}.
\]
We have
\[ 
   \alpha\big( A^*_{(r'),s_{0}} \big)=\alpha \Dsdd{ (r',s_{0})\cdot e^{-tA}  }{t}{0}
        =\alpha\Dsdd{ (r',s_{0} e^{-tA}) }{t}{0}
        =-\Dsdd{ s_{0}^{-1}s_{0} e^{-tA}} {t}{0}
        =A.
\]
Now we take back the vector $\Sigma$ of equation \eqref{eq_gebevectSig}, an element $a\in\Spin(2,1)$ and we compute
\[ 
\begin{split}
  (dR_{a}\alpha)\Sigma&=\alpha\Dsdd{ \big( r'(t),s(t) \big)\cdot a }{t}{0}\\
        &=\alpha\Dsdd{ \big( r'(t),s(t)a \big) }{t}{0}\\
        &=-\Dsdd{ a^{-1}s(0)^{-1}s(t)a }{t}{0}\\
        &=-\Ad(a^{-1})\Dsdd{ s(0)^{-1}s(t) }{t}{0}\\
        &=\Ad(a^{-1})\alpha(\Sigma).
\end{split}  
\]

\end{proof}
Thus that is a connection. This is however not the spin connection. Let $\beta$ be the Levi-Civita connection on the frame bundle $\SO(AdS_3)$. If 
\[ 
  \Sigma=\Dsdd{ r'(t),s(t) }{t}{0},
\]
we have
\begin{equation} \label{eQbetadphiSigma}
  \beta d\phi\Sigma=\left. \phi_{r'}\big( \chi(s_{0}) \big)^{-1}\Dsdd{ \phi_{r'(t)}\big( \chi(s_{t}) \big) }{t}{0}\right|_{\sH}.
\end{equation}
If we note $\phi_{r'(t)}\big( \chi(s_{t}) \big)=\phi\big( r'(t),\chi(s_{f}) \big)$, the derivative in \eqref{eQbetadphiSigma} with respect to $t$ reads
\begin{equation}
\Dsdd{ \phi\big( r'(t),\chi(s_{0}) \big) }{t}{0}+\Dsdd{ \phi\big( r',\chi(s_{t}) \big) }{t}{0}.
\end{equation}
The second term of $\beta d\varphi\Sigma$ is
\begin{align*}
\left. \Dsdd{ \phi_{r'}(\chi(s_{0}))^{-1}\phi_{r'}(\chi(s_{t})) }{t}{0}\right|_{\sH}
        &=\left. \Dsdd{ \phi_{r'}\big( \chi(s_{0}^{-1}s_{t}) \big) }{t}{0}\right|_{\sH}\\
        &=\left. d\phi_{r'}d\chi(s_{0}^{-1}s'(0))\right|_{\sH}.
\end{align*}
From all that we want to define
\begin{equation}
   \alpha^{S}_{(r',s_{0})}\Sigma=\left.d\phi d\chi\big(s_{0}^{-1}s'(0)\big)\right|_{\sH}+\left.\phi_{r'}\big( \chi(s_{0}) \big)^{-1}\Dsdd{ \phi_{r'(t)}\chi(s_{0}) }{t}{0}\right|_{\sH},
\end{equation}
and we would not have $\alpha^{S}(\xi_{X})=0$.


\subsection{Horizontal lift}
%------------------------------

Since the spin component of the path of $\xi_{X}$ is constant, we have $\alpha(\xi_{X})=0$, so equation  \eqref{eq_xiXprojXsharp} says that
\begin{equation}
\overline{ X^{\sharp} }=\xi_{X}.
\end{equation}
Let us recall that an equivariant function (which defined a section of an associated bundle) is
\begin{equation}
\begin{aligned}
 \hat{\psi}\colon P&\to V \\ 
\hat{\psi}(\xi\cdot g)&= \rho(g^{-1})\hat{\psi}(\xi). 
\end{aligned}
\end{equation}
General definition of an equivariant derivative (theorem \ref{tho_dercovassoequiv}) leads to
\[ 
  \widehat{    \nabla_{X^{\sharp}}\psi    }=\overline{ X^{\sharp} }\cdot\hat{\psi}=\xi_{X} \cdot \hat{\psi}.
\]
In our setting, the equivariance of $\hat{\psi}$ reads, for all $a\in\Spin(2,1)$, 
\[ 
  \hat{\psi}\big( ([r],s)\cdot a \big)=\hat{\psi}\big( [r],sa \big)\stackrel{!}{=}\rho(a^{-1})\hat{\psi}\big( [r],s \big).
\]
We check the equivariance of $\widehat{\nabla_{X^{\sharp}}\psi}$ by the following computation:
\[ 
\begin{split}
\widehat{\nabla_{X^{\sharp}}\psi  }\big( ([r],s)\cdot a \big)&=\widehat{\nabla_{X^{\sharp}}\psi}( [r],sa )\\
        &=(\xi_{X}\cdot \hat{\psi})([r],sa)\\
        &=\Dsdd{ \hat{\psi}\big( [r e^{tX}],sa \big) }{t}{0}\\
        &=\Dsdd{ \rho(a^{-1})\hat{\psi}\big( [r e^{tX}],s \big) }{t}{0}\\
        &=\rho(a^{-1})(\xi_{X}\cdot \hat{\psi})\big( [r],s \big)\\
        &=\rho(a^{-1})\widehat{  \nabla_{X^{\sharp}}\psi  }\big( [r],s \big).
\end{split}  
\]

We define $\tilde{\psi}\colon AN/S\to \Lambda W$ by 
\[ 
  \tilde{\psi}([r])=\hat{\psi}( [r],e ),
\]
so that
\begin{equation}
\hat{\psi}([r],s)=\rho(s^{-1})\tilde{\psi}([r]).
\end{equation}
We can conclude
\[ 
\begin{split}
\widetilde{ \nabla_{X^{\sharp}}\psi  }([r])&=\widehat{\nabla_{X^{\sharp}}\psi}([r],e)\\
        &=\xi_{X}\hat{\psi}([r],e)\\
        &=\Dsdd{ \hat{\psi}\big( [r e^{tX}],e \big) }{t}{0}\\
        &=\Dsdd{ \tilde{\psi}\big( [r e^{tX}] \big) }{t}{0}\\
        &=\tilde X_{[r]}\tilde{\psi}([r]).      
\end{split}  
\]
So
\begin{equation}
\widetilde{\nabla_{X^{\sharp}}\psi}=\tilde X_{[r]}\tilde{\psi}.
\end{equation}

\subsection{Spin structure on \texorpdfstring{$AdS_3$}{AdS3} }
%+++++++++++++++++++++++++++++++++++++++++++++++++++++++++

\subsubsection{Spin structure on the whole \texorpdfstring{$AdS_3$}{AdS3} }

\begin{probleme}
    The following seems to contradict what I find in Michelson-Donaldson
\end{probleme}
The central fact is that
\[ 
  \Spin(2,1)\simeq\Delta\simeq\SL(2,\eR)
\]
where $\Delta=\{ (g,g)\tq g\in\SL(2,\eR) \}\subset G_0$. We take as notations: $G_{0}=\SL(2,\eR)$ and $\overline{G}=G_0\times G_{0}$. 

\begin{lemma}
We have the following homogeneous space isomorphism:
\[ 
  \overline{G}/\Delta\simeq\SL(2,\eR).
\]
\end{lemma}

\begin{proof} 

We have an action $\overline{G}\times AdS_3\to AdS_3$,
\begin{equation} \label{EqActghgxh}
  (g,h)x=gxh^{-1}
\end{equation}
where $x\in\SL(2,\eR)$ is seen as in $AdS_3$ by the usual isomorphism. Moreover we consider the isomorphism
\begin{align}
\overline{G}/\Delta&\simeq\SL(2,\eR)\\
[x_1,x_2]&\mapsto x_1x_2^{-1}
\end{align}
which is well defined because $[x_1g,x_2g]\mapsto x_1gg^{-1}x_2^{-1}=x_1x_2^{-1}$. In particular, $[g,g]\mapsto e\in\SL(2,\eR)$. So $\overline{G}$ acts on $\SL(2,\eR)$ and the elements which fix $e$ are the one of $\Delta$. It proves the lemma.
\end{proof}

We are going to take the following structure:
\begin{equation}  \label{EqScSpinAdS}
  \xymatrix{%
   \overline{G} \ar[rr]^{\displaystyle\varphi}\ar[dr]_{\displaystyle\pi}    &   &   \overline{G}/\eZ_{2}\ar[ld]\\
                        & M 
}
\end{equation}
where $M$ is $G_0$ seen as $M=\overline{G}/\Delta\simeq \SL(2,\eR)\simeq AdS_3$, and the projection $\pi\colon \overline{G}\to M$ is given by $\pi(g,h)=gh^{-1}$. The action of $\Delta\simeq \Spin(2,1)$ on $\overline{G}$ is given by formula $(xg,g)\cdot (a,a)=(xga,ga)$. First, let us prove the following.
\begin{proposition}
The frame bundle over $AdS_3$ can be seen as 
\[ 
  \SO(AdS_3)\simeq \overline{G}/\eZ_{2}
\]
where $\overline{G}=\SL(2,\eR)\times\SL(2,\eR)$.
\end{proposition}

\begin{proof}
In the fiber bundle $\pi\colon \overline{G}\to M$, the fibre over $x\in\SL(2,\eR)$ is the set of $(g,h)$ such that $gh^{-1}=x$, or
\[ 
  \overline{G}_{x}=\{ (xg,g) \}\subset \overline{G}.
\]
We will give a surjective map $\overline{G}_{x}\to\SO(M)_{x}$, the fibre of the frame bundle over $x\in AdS_3$. For this, we see a basis of $AdS_3$ as an isometric map $b\colon \sG_0\to T_{x}M$ where $\sG_{0}=\mathfrak{sl}(2,\eR)$, and we define
\begin{equation}
\begin{aligned}
 \psi_{x}\colon \overline{G}_{x}&\to \SO(M)_{x} \\ 
\psi_{x}(xg,g)(X)&=(dL_{x})_{e}\big( \Ad(g^{-1})X \big)
\end{aligned}
\end{equation}
for all $X\in\sG_0$. Let us study the kernel of this map, i.e. elements such that $\psi(xg_1,g_1)=\psi(xg_2,g_2)$. It needs, for all $X\in\mathfrak{sl}(2,\eR)$,
\[ 
  \Ad(g_1^{-1})X=\Ad(g_2^{-1})X,
\]
but we know that the requirement $\Ad(g)X=X$ is the fact the $g$ is in the center of the group. In our case, it results that $g_2^{-1}g_1=\pm\id$, so
\[ 
  \psi(xg_1,g_1)=\psi(\pm xg_1,\pm g_1)
\]
where the same $\pm$ has to be taken in both appearances of the right hand side. Now we put all the $\psi_{x}$ together to get $\psi\colon \overline{G}\to \SO(M)$. Once again we look in which cases $\psi(g_1,h_1)=\psi(g_2,h_2)$. We put this condition under the form
\[ 
  \psi(g_1h_1^{-1}h_1,h_1)=\psi(g_2h_2^{-1}h_2,h_2)
\]
which immediately gives $h_1=\pm h_2$. But on the other hand the base point of $\psi(g_{i}h_{i}^{-1},h_{i})$ is $g_{i}h_{i}^{-1}$, so that the condition also ask $g_1h_1^{-1}=g_2h_2^{-1}$ which in turn gives $g_1=\pm g_2$ with the same $\pm$ as in $h_1=\pm h_2$. We conclude that $\eZ_{2}$ is the problem for the inverse of $\psi$. This proves the proposition.
\end{proof}
We will usually use the same notation, $\psi$, to denote the map from $\overline{G}$ and the one from $\overline{G}/\eZ_{2}$. The following lemma will prove useful to study the actions of the structure groups in the picture \eqref{EqScSpinAdS}.

\begin{lemma}
The map
\begin{equation}
\begin{aligned}
 \SL(2,\eR)&\to \SO_{0}(1,2)\\
   g&\mapsto\Ad(g).
\end{aligned}
\end{equation}
is a double covering.
\end{lemma}
\begin{proof}
No proof.
\end{proof}
The action of $a\in\SO_{0}(1,2)$ on $(xg,g)\in\overline{G}/\eZ_{2}$ is defined by
\begin{equation}
\psi\big( (xg,g)\cdot a \big)=(dL_{x})_{e}\Ad(a^{-1}g^{-1}).
\end{equation}
On the other hand, let us see how does $(a,a)\in\Delta\simeq \Spin(2,1)$ acts on $\overline{G}$ and how does it reflects on the $\psi$ level. Since $(xg,g)\cdot (a,a)=(xga,ga)$, we have
\[ 
  \psi\big( [xg,g]\cdot a \big)=\psi\big( (xg,g)\cdot(a,a) \big),
\]
and then
\[ 
  \varphi\big( (xg,g)\cdot a \big)=(xg,g)\cdot(a,a).
\]
This proves that our structure is a spin structure.

\subsubsection{Reduction to one open orbit}
%----------------------------------------

We will use this isomorphism between $AdS_3$ and $\SL(2,\eR)$:
\[ 
  \begin{pmatrix}
u\\t\\x\\y
\end{pmatrix}\mapsto
\begin{pmatrix}
u+x&y-t\\y+t&u-x
\end{pmatrix}.
\]
Then the famous point $[u]=\begin{pmatrix}
0&1\\-1&0
\end{pmatrix}\in AdS_3$ corresponds to the element $J:=\begin{pmatrix}
0&1\\-1&0
\end{pmatrix}\in \SL(2,\eR)$. This is our base point of the open orbit. We could also take
\[ 
  k_{0}=\frac{ \sqrt 2 }{ 2 }\begin{pmatrix}
1&1\\-1&1
\end{pmatrix}\in K_{0}
\]
where $K_{0}$ is the ``$K$'' of $\SL(2,\eR)$. 
\begin{probleme}
    I think that $J$ is also a complex structure. To be checked.
\end{probleme}
We have $J=k_{0}^{2}$ and following the action \eqref{EqActghgxh}, we have $J=(k_{0},k_{0}^{-1})e$. The subgroup $\overline{R}\subset\overline{G}$ acts on $AdS_3$, and we want to know the stabilizer of $J$. The condition is $(r,r')\cdot J=J$, or
\[ 
  r=\AD(J)r',
\]
but $\AD(J)=\theta$ (the Cartan involution). So an element $(r,r')\in\overline{R}$ stabilises $J$ if it is of the form $(r,\theta r)$, thus
\[ 
  \mfs=\text{Lie algebra of the stabiliser of }J=\{ (X,\theta X)\tq X\in\sR_{0} \}\cap\sR,
\]
where the intersection with $\sR$ is important because $\theta$ can send out of $\sR_{0}$. Note that when $X$ has a $\sN$ component, then $\theta X$ has a $\overline{ \sN }$ component, so $(X,\theta X)\in(\sA\oplus\sN,-\sA\oplus\overline{ \sN })$ where the minus sign comes from the fact that $\theta(\sA)=-\sA$. Then $X$ cannot have a $\sN$ component and finally,
\[ 
  \mfs=\eR (H,-H)\in\sQ.
\]
The group $R'$ is
\begin{equation}
R'= e^{\sR'}=\{ (an,an')\tq n,n'\in N_{0} \}
\end{equation}
because $\sR'$ is $\sR$ minus the stabiliser, i.e. $\sR'=\eR(H,H)\oplus\sN$. We have the identification $r'\mapsto r'\cdot J$ between $R'$ and the open orbit $\mU$. As usual, the action is $(g,h)\cdot x=gxh^{-1}$ if $r'=(g,h)$. Notice in particular that $R'\neq R_{0}'\times R_{0}'$.

Up to now we studied the fiber $\overline{G}\to M$; we are now able to restrict it to $\overline{G}|_{\mU}\to\mU$ and to establish an isomorphism with the trivial bundle $R'\times G_0\to R'$. The fiber over $x\in\mU$ is
\[ 
  \overline{G}_{x}=\{ (xg,g) \}.
\]
We define the isomorphism as follows:
\begin{equation}
\begin{aligned}
 \tau\colon R'\times G_0&\to \overline{G}|_{\mU} \\ 
(r',g)&\mapsto (r'\cdot Jg,g) 
\end{aligned}
\end{equation}
and we have the following picture: 
\[ 
  \xymatrix{%
   R'\times G_0 \ar[r]^-{\displaystyle\tau}\ar[d]   &   \overline{G}|_{\mU}\ar[d]^{\displaystyle\pi}\\
   R' \ar@{.>}[r]^{\displaystyle\tau}       &   \mU
}
\]
in which are defines by
\[ 
  \xymatrix{%
   (r',g) \ar[r]^-{\displaystyle\tau}\ar[d] &   (r'\cdot Jg,g)\ar[d]^{\displaystyle\pi}\\
   r' \ar@{.>}[r]^{\displaystyle\tau}       &   r'\cdot J
}
\]
where the dotted line denotes the induced map from $\tau$, which is denoted by the same symbol. The map $\tau\colon R'\to \mU$ is just the restriction of the original $\tau$ to $g=e$. Notice that this $\tau$ provides a diffeomorphism of the basis spaces $R'$ and $\mU$.

\subsubsection{Spin connection}
%---------------------------

The spin connection on $\overline{G}|_{\mU}$ is given by
\begin{equation}    \label{EqDefConnAdS3}
  \alpha^{S}_{(g,h)}\Sigma=\left[ dL_{(g,h)^{-1}}\Sigma \right]_{\sH},
\end{equation}
or
\begin{equation}
\alpha^{S}_{(g,h)}=\pr_{\sH}\circ\big( dL_{(g,h)^{-1}} \big)_{(g,h)}.
\end{equation}
Notice that when we write $\sH$, we think about $\Delta$: the group by which quotient  $\overline{G}$ in order to get $\SL(2,\eR)\simeq AdS_3$.
Our task now is to transfer this connection to $R'\times G_0$ by defining $\alpha'=\tau^*\alpha^{S}$. If $\Sigma\in T_{(r',g)}(R'\times G_0)$, we define
\begin{equation}
\alpha'_{(r',g)}\Sigma=\alpha^{S}(d\tau\Sigma).
\end{equation}
Let us take $X\in\sG_0$ and $0\in\sR'$ and let us compute $d\tau(0,X)$. More precisely, we consider
\begin{align*}
d\tau(0\oplus \tilde X_{g})_{(r',g)}&=d\tau\Dsdd{ r',g e^{tX} }{t}{0}\\
        &=\Dsdd{ r'\cdot Jg e^{tX},g e^{tX} }{t}{0}\\
        &=\big( \tilde X_{(r'\cdot Jg)},\tilde X_{g} \big).
\end{align*}
The next step is to compute $d\tau\Sigma$ in the case where $\Sigma=(\utilde Y\oplus -1)_{(r',g)}$ with $Y\in\sR'\subset\sR_{0}\oplus\sR_{0}$. We have
\begin{align}
d\tau\Sigma&=\Dsdd{ \tau\big(  e^{tY}r',g \big) }{t}{0}\\
    &=\Dsdd{ ( e^{tY}r'\cdot J)g,g }{t}{0}
\end{align}
where, if $r'=(r_{1},r_{2})$, we consider $Y=\big((\utilde Y_{1})_{r_{1}},(\utilde Y_{2})_{r_{2}}\big)  $. This appears to be difficult to be computed. This reflects the fact that the connection should be complicated in the trivial bundle $R'\times G_0$.

But there are no fate. We remember that $\tau$ furnish a diffeomorphism between the basis spaces, so one can consider the bundle 
\[ 
  \xymatrix{%
   \overline{G}|_{\mU} \ar[d]^{\displaystyle\tau^{-1}\circ\pi}\\        
   R'
}
\]
Vectors of $\sH$ are of the form $(X,X)$ with $X\in\mathfrak{sl}(2,\eR)$, thus $A\in T_{(xg,g)}\overline{G}|_{\mU}$ fulfils $\alpha^{S}(A)=0$ if and only if
\[ 
  dL_{(xg,g)^{-1}}(A)=(X,-X)
\]
for a certain $X\in \mathfrak{sl}(2,\eR)$. All this makes that the horizontal space over $(xg,g)$ is given by
\begin{equation}
\horsp(xg,g)=\big\{ (\tilde X_{xg},-\tilde X_{g})\tq X\in \sG_0=\mathfrak{sl}(2,\eR) \big\}.
\end{equation}
The strategy now is to project that on $R'$ and express Dirac operator in terms of the result. Let us make this simple computation:
\begin{align*}
d\pi(\tilde X_{xg},\tilde X_{g})&=\Dsdd{ \pi\big( xg e^{tX},g e^{-tX} \big) }{t}{0}\\
        &=\Dsdd{ xg e^{tX} e^{tX}g^{-1} }{t}{0}\\
        &=\Dsdd{ x e^{2t\Ad(g)X} }{t}{0}\\
        &=2(dL_{x})_{e}\Ad(g)X.
\end{align*}
This result has to be brought from $\mU$ to $R'$ by $\tau^{-1}$. Now we take a $\tilde Y\in\cvec(R')$ and we want to know which is the corresponding $X$, i.e. the $X\in\mathfrak{sl}(2,\eR)$ such that
\[ 
  d\tau^{-1}d\pi(\tilde X_{xg},-\tilde X_{g})=\tilde Y.
\]
From the previous computation, $\tilde Y=2d\tau^{-1}dL_{x}\Ad(g)X$, so
\begin{equation}  \label{EqXfracAdY}
  X=\frac{ 1 }{2}\Ad(g^{-1})dL_{x^{-1}}d\tau\tilde Y. 
\end{equation}
We now precise our idea: 
\begin{equation}   \label{EqtildeYrunrdeux}
  \tilde Y_{(r_{1},r_2)}=\big(    (\tilde Y_{1})_{r_1},(\tilde Y_{2})_{r_2}   \big)=\Dsdd{ r_1 e^{tY_{1}},r_2 e^{tY_{2}} }{t}{0}
\end{equation}
for $Y_{i}\in\sR'_{0}$ and $r_1$, $r_2\in R_{0}$. In this case, the ``$x$'' in equation \eqref{EqXfracAdY} is $(r'\cdot J)^{-1}$. Let us begin by taking $s'\in R'$ and compute $L_{(r'\cdot J)^{-1}}\tau(s')$. Remember that $r'\cdot J=r_1Jr_2^{-1}$ from the general action \eqref{EqActghgxh}, so if $r'=(r_1,r_2)$,
\begin{align*}
  dL_{(r'\cdot J)^{-1}}\tau(s')&=(r'\cdot J)^{-1}s_1 Js_2^{-1}\\
        &=(r_1Jr_2^{-1})^{-1}s_1Js_2^{-1}\\
        &=-r_2Jr_2^{-1}s_1Js_2^{-1}.
\end{align*}
Now, we apply that result on computation of \eqref{EqXfracAdY} with \eqref{EqtildeYrunrdeux}:
\begin{align*}
dL_{(r'\cdot J)^{-1}}d\tau\tilde Y&=\Dsdd{ -r_2Jr_1^{-1}r_1 e^{tY_{1}}J e^{-tY_{2}}r_2^{-1} }{t}{0}\\
        &=\Dsdd{ \AD(r_2)\big( -J e^{tY_{1}}J e^{-tY_{2}} \big) }{t}{0}\\
        &=\Dsdd{ \AD(r_2) e^{-tY_{2}} }{t}{0}+\Ad(r_2)\Ad(J)Y_{1}\\
        &=-\Ad(r_2)Y_{2}+\Ad(r_2)\theta(Y_{1}),
\end{align*}
and finally,
\begin{equation}  \label{EqValeurXAdtheta}
\begin{aligned}
X&=\frac{ 1 }{2}\Ad(g^{-1})dL_{(r'\cdot J)^{-1}}d\tau\tilde Y\\
        &=\frac{ 1 }{2}\Ad(g^{-1})\big[ \Ad(r_2)\theta(Y_{1})-\Ad(r_2)Y_{2} \big].
\end{aligned}
\end{equation}
For this $X$, the horizontal lift of $\tilde Y\in\cvec(R')$ is $(X,-X)\in T\overline{G}|_{\mU}$.

\subsection{Left invariance of Dirac}
%------------------------------------

Sections of the spin bundle over the open orbit $\mU$ are given by equivariant functions $\hat{\psi}\colon \overline{G}|_{\mU}\to \eR^{2}$. The action of $\Delta\simeq\Spin(2,1)$ on $\overline{G}$ is 
\[ 
  (g,h)\cdot(a,a)=(ga,ha).
\]
We define $\tilde{\psi}$ by
\begin{equation}
\tilde{\psi}(g)=\hat{\psi}(g,e)
\end{equation}
for $g\in\mU$. We get back the original $\hat{\psi}$ by formula
\begin{equation}
\hat{\psi}(g,h)=\rho(h,h)^{-1}\tilde{\psi}(gh^{-1}).
\end{equation}
Our intention is now to compute $\widehat{\nabla_{Z}\psi}(\xi)=\overline{ Z }_{\xi}(\hat{\psi})$ with $\xi=(xg,g)\in\overline{G}|_{\mU}$ (hence $x\in\mU$) and $Z\in\cvec(R')$. For instance we choose a left invariant $Z=\tilde Y=(\tilde Y_{1},\tilde Y_{2})$ for $Y_{1}$, $Y_2\in\sR_{0}'$. Recall that $\tilde Y$ is given by equation \eqref{EqtildeYrunrdeux}. From definition of the covariant derivative associated with the connection,
\[ 
  \widehat{\nabla_{\tilde Y}\psi}(\xi)=\overline{ \tilde Y }_{\xi}(\hat{\psi})=\overline{ \tilde Y }_{(xg,g)}(\hat{\psi})
\]
where $\overline{ \tilde Y }_{(xg,g)}$ is an horizontal vector at $(xg,g)$ whose projection is $\tilde Y$. From our previous work,
\[ 
  \overline{ \tilde Y }_{xg,g}=(\tilde X_{xg},-\tilde X_{g})
\]
with $X=\frac{ 1 }{2}\Ad(g^{-1})\big( \Ad(r_2)\theta Y_{1}-\Ad(r_2)Y_{2} \big)$. Let us understand the link between $(r_{1},r_2)$ and $g,x$. The vector $(\tilde X_{xg},\tilde X_{g})$ actually projects to a vector at $\tau^{-1}\circ\pi(xg,g)=\tau^{-1}(x)$. The fact that $x\in\mU$ guarantees existence and uniqueness of $(r_1,r_2)\in R'$ such that $r_1 Jr_2^{-1}=x$. We have
\[ 
\begin{split}
\protect\widetilde{\nabla_{\protect\tilde Y}\psi}(x)&=\widehat{\nabla_{\tilde{Y}}\psi}(x,e)\\
        &=(\tilde X_{x},-\tilde X_{e})\hat{\psi}\\
        &=\dsdd{ \hat{\psi}\Big( x e^{tX}, e^{-tX} \Big) }{t}{0}\\
        &=\dsdd{ \rho( e^{tX}, e^{tX})\tilde{\psi}(x e^{2tX}) }{t}{0}.
\end{split}  
\]
The first term of the derivation (the one with $t=0$ in the $\rho$) gives $2\tilde X_{x}\tilde{\psi}$. This is left invariant.
The second is 
\[
     \dsdd{ \rho( e^{tX}, e^{tX})\tilde{\psi}(x) }{t}{0}.
\]
We want to test the condition \eqref{EqDefLxinvarop} on this term. Let us pose
\[ 
  (E\tilde{\psi})(x)=(\tilde X_{x},\tilde X_{e})\hat{\psi}=\dsdd{ \rho( e^{tX}, e^{tX})\tilde{\psi}(x) }{t}{0}
\]
with $X$ given by equation \eqref{EqValeurXAdtheta}. On the one hand,
\begin{subequations}
\begin{equation}
L_{y}(E\tilde{\psi})(x)=(E\tilde{\psi})(yx)
        =\dsdd{ \rho( e^{tX_{a}}, e^{tX_{a}})\tilde{\psi}(yx) }{t}{0}
\end{equation}
with 
\begin{equation}
X_{a}=\frac{ 1 }{2}\Big( \Ad(r_2)\theta Y_{1}-\Ad(r_2) \Big)Y_{2}
\end{equation}
\end{subequations}
where $(r_1,r_2)$ is given by $yx$. On the other hand,
\begin{subequations}
\begin{equation}
E(L_{y}\tilde{\psi})(x)=\dsdd{ \rho( e^{tX_{b}}, e^{tX_{b}})\tilde{\psi}(yx) }{t}{0}
\end{equation}
with 
\begin{equation}
X_{b}=\frac{ 1 }{2}\big( \Ad(s_2)\theta Y_{1}-\Ad(s_2)Y_{2} \big)
\end{equation}
\end{subequations}
where $(s_1,s_2)$ is given by $x$.

The problem is that the choice of $y$ is arbitrary, so that $X_{a}$ and $X_{b}$ could be too different. Ok. That's the proof that Dirac is not invariant. Here is the proof that Dirac is invariant.

Following equation \eqref{EqDefConnAdS3}, the spin connection form is
\[ 
  \alpha^{S}_{(g,h)}\Sigma=\left( dL_{(g,h)^{-1}}\Sigma \right)_{\sH}.
\]
If $L_{(x,y)}$ is the left translation by $(x,y)$ we have
\[ 
  \big( L^{*}_{(x,y)}\alpha \big)_{(g,h)}\Sigma=\alpha_{(xg,yh)}\big( dL_{(x,y)}\Sigma \big)
        =\left( dL_{(g,h)^{-1}}\Sigma \right)_{\sH}.
\]
Thus we have $L^*_{(x,y)}\alpha^{S}=\alpha^{S}$. Now we consider the formula $\widehat{\nabla_{\tilde Y}\psi}(\xi)=(\tilde X_{xg},-\tilde X_{g})\hat{\psi}$, and we will check that 
\begin{equation}
 \big( L_{\eta}\widehat{\nabla_{Z}\psi} \big)(\xi)=\widehat{  \nabla_{Z}(L_{\eta}\psi)     }(\xi).
\end{equation}
with $\xi=(xg,g)$ and $\eta=(a,b)$. On the one hand,
\begin{align*}
  \big( L_{(a,b)}\widehat{\nabla_{Z}\psi} \big)(xg,g)&=\widehat{\nabla_{Z}\psi}(axg,bg)\\
        &=\widehat{\nabla_{Z}\psi}(axgg^{-1}b^{-1}bg,bg)\\
        &=\big( \tilde X_{(axb^{-1})bg},-\tilde X_{bg} \big)\hat{\psi}.
\end{align*}
On the other hand,
\begin{align*}
  \widehat{  \nabla_{Z}(L_{(a,b)}\psi)   }(xg,g)&=(\tilde X_{xg},-\tilde X_{g})\widehat{L_{(a,b)}\psi}\\
        &=\dsdd{ \widehat{L_{(a,b)}\psi}\big( xg e^{tX},g e^{-tX} \big) }{t}{0}\\
        &=\dsdd{ \hat{\psi}\big( axg e^{tX},bg e^{-tX} \big) }{t}{0}\\
        &= (\tilde X_{axg},-\tilde X_{bg}) \hat{\psi}\\
        &=( \tilde X_{(axb^{-1})bg},-\tilde X_{bg} )\hat{\psi}. 
\end{align*}

\section{Spin structure and Dirac operator on \texorpdfstring{$AdS_{l}$}{AdSl}}\label{SecDirADs}
%+++++++++++++++++++++++++++++++++++++++++++++++++++++++++++

Construction of the frame bundle is a straightforward adaptation of theorem 2.2 (chapter II) in \cite{AnnikFranc}, while connection issues are adapted from proposition 1.3 (chapter III).  According proposition \ref{PropGHconn}, notations $G$ and $H$ stand for the identity components of $\SO(2,l-1)$ and $\SO(1,l-1)$.

\subsection{Frame bundle and spin structure}
%--------------------------------------------

An element of the frame bundle is a map from $\sQ$ to $T(G/H)$ of the form\footnote{See \ref{SubSechoappahomsp} for notations.} $d\mu_g\circ A$ where $g\in G$ and $A\in \SO_0(\sQ)$. By proposition \ref{PropSOADHequal}, there exists a $h\in H$ for which $A=\Ad(h)$ for every $A\in\SO_0(\sQ)$ so we have
\[ 
 d\mu_g\circ A=d\pi\circ dL_g\circ\Ad(h)=d\pi\circ dL_g \circ dL_h\circ dR_h=d\pi\circ dL_g \circ dL_h=d\mu_{gh}
\]
hence in fact every element in the frame bundle reads $d\mu_g$ for some $g\in G$. We conclude that the fibre $B_{[g]}$ over $[g]$ is made of maps of the form $d\tau_k$ with $k\in[g]$. The action of $H$ on the frame bundle is given by 
\[ 
  (d\mu_g)\cdot h=d\mu_g\circ\Ad(h).
\]

\begin{proposition}
The map
\begin{equation}
\begin{aligned}
 \beta\colon G&\to B \\ 
g&\mapsto d\mu_g 
\end{aligned}
\end{equation}
is a principal bundle isomorphism between the frame bundle and the principal bundle
\begin{equation}        \label{PrincHGGH}
\xymatrix{%
    G\ar[d]^{\pi}& H\ar@{~>}[l]     \\
                    G/H
 }
\end{equation}
where $\pi$ is the natural projection, the action of $H$ is the right one and the wavy line means ``acts on''.
\end{proposition}

\begin{proof}
Surjectivity of $\beta$ is clear. For injectivity, suppose $d\mu_g=d\mu_{g'}$. In order for the two target spaces to be equal, one needs $g'=gh$ for a $h\in H$. Now we have, for all $q_j\in\sQ$, 
\begin{equation}
  d\mu_gq_j=d\mu_{gh}q_j=d\pi dR_{h^{-1}}dL_gdL_hq_j
        =d\pi dL_g\big( \Ad(h)q_j \big),
\end{equation}
but $d\pi$ is an isomorphism from $\sQ_g$, so we deduce that $q_j=\Ad(h)q_j$. Since we are using the connected component of $\SO(\sQ)$, that implies that $h=e$, and thus that $g=g'$. The following proves that $\beta$ is a morphism:
\[ 
  \beta(gh)=d\pi dL_g dL_h=d\pi dL_g dL_h dR_{h^{-1}}=d\pi dL_g \Ad(h)=\beta(g)\cdot h.
\]
\end{proof}

The following lemma provides  a convenient way to express the tangent bundle over $G/H$ as an associated bundle to the principal bundle \eqref{PrincHGGH}. We denote by $G\times_{\rho} \sQ$ the quotient of $G\times\sQ$ by the equivalence relation $(g,X)\sim(gh,\Ad(h^{-1})X)$ for all $h\in H$.

\begin{lemma}  
The map 
\begin{equation} 
\begin{aligned}
 \beta\colon G\times_{\rho}\sQ&\to TM \\ 
[g,X]&\mapsto d\tau_gd\pi X 
\end{aligned}
\end{equation}
with $\rho(h)X=\Ad(h^{-1})X$ is diffeomorphic. 
\label{LemBazHGGH}
\end{lemma}

\begin{proof}

 In order to check that $\beta$ is well defined, first compute
\[ 
  \beta[gh,\Ad(h^{-1})X]=d\tau_{gh}d\pi\Ad(h^{-1})X=d\pi dL_{gh}\Ad(h^{-1})X,
\]
and then using the fact that $d\pi dR_h=d\pi$, the latter line reduces to $d\pi dL_gX=\beta(g,X)$. For injectivity, let $\beta[g,X]=\beta[g',X']$. In order for these two to be vectors on the same point, there must exists a $h\in H$ such that $g'=gh$. The equality becomes $d\pi dL_g dL_h X'=d\pi dL_gX$. Commuting $d\pi$ with $dL_g$ and using the fact that $d\tau_g$ is an isomorphism, we are left with the condition $d\pi dL_h X'=d\pi X$.

An element of $\sG/\sH$ is an equivalence class which contains exactly one element of $\sQ$. In the right hand side of the condition, this element is $X$ while the element of $\sQ$ in the class $d\pi dL_h X$ is $\Ad(h)X'$. Equating these two elements, we find the condition $X'=\Ad(h^{-1})X$, which proves that $[g,X]=[g',X']$ and concludes the proof of the injectivity of $\beta$.
\end{proof}

The following proposition will prove useful in order to identity the spin structure over $AdS_4$.

\begin{proposition}
If $G$ is a connected Lie group and if $Z$ is the center of $G$, then
\begin{enumerate}
\item $\Ad_G$ is an analytic homomorphism from $G$ to $\Int(G)$, with kernel $Z$,
\item the map $[g]\to\Ad_G(g)$ is an analytic isomorphism from $G/Z$ to $\Int(\lG)$ (the class $[g]$ is taken with respect to $Z$).
\end{enumerate}
\end{proposition}
% cette proposition est en fait le corollaire \ref{cor:Ad_homom} que j'ai recopié ici pour l'utilité de la thèse. Ça peut disparaître pour mazhe.
On the one hand that proposition together with the fact that $Z\big( \SP(2,\eR) \big)=\eZ_2$ proves that the quotient $\SP(2,\eR)/\eZ_2$ is isomorphic to $\Int\big(\gsp(2,\eR)\big)$. On the other hand one knows that $\SO_0(2,3)$ has no center, so that $\SO_0(2,3)\simeq\Int(\so(2,3))$. But the subsection \ref{SubSecIsosp} provides an isomorphism between $\so(2,3)$ and $\gsp(2,\eR)$. Thus we have
\begin{equation}
\SP(2,\eR)/\eZ_2\simeq \SO_0(2,3).
\end{equation}
We denote by $\varphi\colon \SP(2,\eR)\to \SO_0(2,3)$ the corresponding homomorphism with kernel $\eZ_2$. In particular the restriction $\varphi|_{\SL(2,\eC)}$ is a double covering of $\SO_0(1,3)$. But $\chi$ is the same kind of double covering, so universality of $\SL(2,\eR)$ on $\SO_0(1,3)$ provides an automorphism $f\colon \SL(2,\eC)\to \SL(2,\eC)$ such that $\varphi=\chi\circ f$. The spin structure to be considered on $AdS_4$ is
\[ 
\xymatrix{%
   \Spin(1,3) \ar@{~>}[r]       &\SP(2,\eR)  \ar[rd] \ar[rr]^{\displaystyle\varphi} &&  \SO_0(2,3)\ar[ld]&\SO_0(1,3)\ar@{~>}[l]\\
   &&   AdS_4 
}
\]
where the action of $\Spin(1,3)$ on $\SP(2,\eR)$ is given by $a\cdot s=af^{-1}(s)$ where we identified $\Spin(1,3)$ with $\SL(2,\eC)$ as subgroup of $\SP(2,\eR)$. One immediately has $\varphi(a\cdot s)=\varphi(a)\chi(s)$.


\subsection{Connection}
%----------------------

There are a lot of ways to express a vector field $X\colon G/H\to T(G/H)$. From the identification $T(G/H)=G\times_{\rho}\sQ$, one has $X\colon G/H\to G\times_{\rho}\sQ$. As section of an associated bundle, $X$ can be expressed by an equivariant function $\hat{X}\colon G\to \sQ$ such that $X_{[g]}=[g,\hat{X}(g)]$. The $H$-equivariance of $\hat X$ means that $\hat{X}(gh)=\Ad(h^{-1})\hat{X}(g)$.  Let $X\in\sG$ and consider the function
\begin{equation}        \label{EqDefhatAcol}
\begin{aligned}
 \hat A_X\colon G&\to \sQ \\ 
g&\mapsto \big( \Ad(g^{-1})X \big)_{\sQ} 
\end{aligned}
\end{equation} 
which is equivariant because the decomposition $\sG=\sH\oplus\sQ$ is reductive. The corresponding vector field is
\[ 
  A_X[g]=\big[ g,\big( \Ad(g^{-1})X \big)_{\sQ} \big];
\]
or
\[ 
  A_X[g]=d\tau_gd\pi\big( \Ad(g^{-1})X \big)_{\sQ}=d\pi dL_g\big( \Ad(g^{-1})X \big)
\]
because $d\pi X_{\sQ}=d\pi X$. It is easy to check that the form
\[ 
  \omega_g(X)=-\big( dL_{g^{-1}}X \big)_{\sH}
\]
is a connection form on the principal bundle \eqref{PrincHGGH}.  We are going to determine the associated covariant derivative of this connection on the tangent space, and prove that it is torsion free. The horizontal lift of $A_X[g]$ is 
\begin{equation}    \label{EqovlAprQhor}
  \overline{ A }_X(g)=dL_g\big( \Ad(g^{-1})X \big)_{\sQ}=\Dsdd{ g e^{t\pr_{\sQ}\Ad(g^{-1})X} }{t}{0}.
\end{equation}
The equivariant function associated with the covariant derivative of $A_Y$ in the direction of $A_X$ is given by $(\overline{ A }_X)_g\hat A_Y$. Using expressions  \eqref{EqDefhatAcol} and  \eqref{EqovlAprQhor} of $\hat A_Y(g)$ and $\overline{ A }_X(g)$, we have
\[ 
\begin{split}
  (\bar A_X)_g\hat A_Y  &=\Dsdd{ \hat A_Y\big( g e^{t\pr_{\sQ}\Ad(g^{-1})X} \big)_{\sQ} }{t}{0}\\
            &=\Dsdd{ \left( \Ad\big(  e^{-t\pr_{\sQ}\Ad(g^{-1})X}g^{-1}  \big)Y \right)_{\sQ} }{t}{0}\\
            &=\Big( \ad\big( -\pr_{\sQ}\Ad(g^{-1})X \big)\Ad(g^{-1})Y \Big)_{\sQ}\\
            &=-\left[ \big( \Ad(g^{-1})X \big)_{\sQ},\Ad(g^{-1})Y  \right]_{\sQ}.
\end{split}  
\]
This commutator is an expression of the form $[Z_{\sQ},Z'_{\sQ}+Z'_{\sH}]_{\sQ}$. Using reducibility we find
\begin{equation}
  (\overline{ A }_X)_g\hat A_Y=-\Big[ \big( \Ad(g^{-1})X \big)_{\sQ},\big( \Ad(g^{-1})Y \big)_{\sH} \Big].
\end{equation}
The commutator produces
\[ 
  (\overline{ A }_X)_g\hat A_Y-(\overline{ A }_Y)g\hat A_X=-\hat A_{[X,Y]}(g),
\]
which by construction the equivariant function associated with the vector field $\nabla_{A_X}A_Y-\nabla_{A_Y}A_X$; so on the one hand we have
\[ 
(\nabla_{A_X}A_Y-\nabla_{A_Y}A_X)[g]=-d\tau_gd\pi\hat A_{[X,Y]}(g)
        =-d\tau_gd\pi\big( \Ad(g^{-1})[X,Y] \big)_{\sQ}
        =-d\pi dR_g[X,Y].
\]
On the other hand,
\[ 
  [A_X,A_Y][g]=d\pi[dR_g X,dR_gY]=-d\pi dR_g[X,Y],
\]
which proves that the connection is torsion free.

We are now going to study the horizontal vector fields on $\SP(2,\eR)$ with this connection and the homomorphism $h^{-1}$ of equation \eqref{Eqdefhspsl}. We have to study for which elements $\Sigma_a\in \SP(2,\eR)$ the expression
\begin{equation}        \label{EqAtrouverdhemu}
  \omega_a(\Sigma_a)=\omega_{h^{-1}(a)}\big( (dh^{-1})_a\Sigma_a \big)=-\Big( dL_{h^{-1}(a)^{-1}}dh^{-1}\Sigma_a \Big)_{\sH}
\end{equation}
vanishes. Every such element can of course be written under the form $\Sigma_a=dL_a\psi X$ for some $X\in \so(2,3)$. So we are lead to consider the expression
\begin{equation}        \label{Eqdhemuconide}
  (dh^{-1})_a(dL_a)_e\psi X.
\end{equation}
It is easy to deal with that expression in the case of $a=e$:
\[ 
  (dh^{-1})_e\psi(X)=\psi^{-1}\psi X=X.
\]
In particular, if $\Sigma\in\mT$, then $dh^{-1}\Sigma\in\sQ$ and when $\Sigma\in\mI$, we have $dh^{-1}\Sigma\in\sH$. This result propagates to other elements $a\in \SP(2,\eR)$ using the general result
\[ 
  df\circ dL_g=\big( dl_{f(g)} \big)\circ df
\]
which holds for any group homomorphism $f$. Using that property with $h^{-1}$ on the point $a\in \SP(2,\eR)$, we find $(dh^{-1})_a\circ(dL_a)_e=\big( dL_{h^{-1}(a)} \big)\circ (dh^{-1})_e$, and the expression \eqref{EqAtrouverdhemu} becomes
\[ 
  \omega_a(dL_a\psi X)=\Big( dL_{\big(h^{-1}(a)\big)^{-1}}dh^{-1} dL_a\psi X \Big)_{\sH}=X_{\sH}.
\]
It is zero if and only if $X\in\sQ$, so that the horizontal vectors on $a$ are exactly the ones of $dL_a\psi\sQ=\mT_a$.

\subsection{Dirac operator}
%--------------------------

When $\hat{s}\colon \SP(2,\eR) \to \Lambda W$ is the equivariant function associated with a spinor, the Dirac operator reads
\begin{equation}        \label{EqDiracAdsquatre}
\widehat{Ds}(a)=g_{ij}\gamma^j\widehat{\nabla_{t_i}s}(a)
        =g_{ij}\gamma^j\overline{ t }_i(a)\hat{s}
        =g_{ij}\gamma^j\tilde t_i(a)\hat{s}
\end{equation}
where the metric $g$ is the usual four-dimensional Minkowskian metric and the matrices $\gamma$ are the associated $4\times 4$ Dirac matrices. The elements $\tilde t_i(a)=dL_at_i=dL_a\psi(q_i)$ span the natural basis of $\mT_a$, see appendix \ref{SubSecRedspT}. The matrices $\gamma^i$ are the usual $4\times 4$ Dirac matrices for the $4$-dimensional Minkowskian metric.

\begin{probleme}
Et il serait aussi pas mal de préciser un peu une fois tout de suite ce qu'est l'espace $\Lambda W$.
\end{probleme}

One can find a change of basis which express the Dirac operator in terms of vectors on $R_1$. For that, let $\{ X_i \}$ be a basis of $\sR_1$. We have 
\[ 
  X^*_i[u]=\Dsdd{ [ e^{-tX_i}u] }{t}{0}=-d\pi dR_uX_i
\]
that is necessarily decomposable by corollary \ref{Cordpiietwii} as combinations of vectors of the form $d\pi dL_uq_i$ because $[u]$ belongs to an open orbit of the action of $R_1$. That defines a matrix $B$ by
\[ 
  d\pi dL_uq_i=B_{ij}d\pi dR_uX_j,
\]
and then a vector $Y\in \sH$ by
\begin{equation}
q_i=B_{ij}\Ad(u^{-1})X_j+Y.
\end{equation}
Now we have $\tilde t_i(a)=dL_a\psi\big( \Ad(u^{-1})B_{ij}X_j+Y \big)$. We can go further using the fact that $\psi\Big( \Ad\big( h^{-1}(a) \big)X \Big)=\Ad(a)\psi(X)$ for every $a\in\SP(2,\eR)$ and $X\in\sG$. Defining the vectors $s_i=\Ad\big( h(^{-1}) \big)\psi X_i$ we find
\begin{equation}
\tilde t_i(a)=B_{ij}\tilde s_i(a)+\widetilde{\psi(Y)}(a).
\end{equation}

\subsection{Frame bundle}\index{frame!bundle!on $AdS_{l}$}
%------------------------

Construction of the frame bundle and the spin structure is a straightforward adaptation of theorem 2.2 (chapter???) in \cite{AnnikFranc}, while Dirac operator and connection issues are adapted from proposition 1.3 (chapter III)

A \defe{basis}{basis} of a $m$ dimensional vector space $V$ is a free and generating part; it only has the structure of a set. A frame of the vector space $V$ is a nondegenerate map $b\colon \eR^{m}\to V$. Let us give an example in three dimensions the difference. If $\{ v_{1},v_{2},v_{3} \}$ is a basis of $V$, of course $\{ v_{2},v_{1},v_{3} \}$ is the same basis. Order has no importance. But if $\{ e_{1},e_{2},e_{3} \}$ is the canonical basis of $\eR^{3}$, the \emph{frames} $b(e_{1})=v_1$, $b(e_{2})=v_2$, $b(e_{3})=v_{3}$ and $c(e_{1})=v_2$, $c(e_{2})=v_1$, $c(e_{3})=v_{3}$ are not the same.

Now we consider $AdS_l=G/H=\SO(2,l-1)/\SO(1,l-1)$, the Lie algebra $\sG$ has a reductive homogeneous space decomposition $\sG=\sQ\oplus\sH$ and we consider the canonical projection $\pi\colon G\to AdS_l$.

Let the map (see relation \eqref{EqInclAdHSOq})
\begin{equation}
\begin{aligned}
 \alpha\colon H&\to \SO(\sQ) \\ 
h&\mapsto \Ad(h)|_{\sQ}.
\end{aligned}
\end{equation}
We consider, on $G\times\SO(\sQ)$, the equivalence relation $(g,A)\sim(g',A')$ if and only if there exists $h\in H$ such that $g'=gh$ and $A'=\alpha(h^{-1})A$. We denote by $G\times_{\alpha}\SO(\sQ)$ the set of equivalence classes. Now we have a principal bundle
\begin{equation}   \label{EqPrincPreB}
\xymatrix{%
   \SO(\sQ) \ar@{~>}[r]     &   G\times_{\alpha}\SO(\sQ)\ar[d]^{p}\\
                &      G/H
}
\end{equation}
where $p[g,A]=[g]$ and the action is given by $[g,A]\cdot B=[g,A]$. The fact that the projection fulfils $p\big( [g,A]\cdot B \big)=p[g,A]$ is evident, and the fact that the action is well defined is a simple computation:
if $[g',A']=[g,A]$, we have a $h\in H$ such that
\[ 
  [g',A']\cdot B=[g',A'B]
        =[gh,\alpha(h^{-1})AB]
        =[g,AB]
        =[g,A]\cdot B.
\]

\begin{proposition}
Let $\tau(g)\colon AdS_l\to AdS_l$ be the action of $g\in G$ on $AdS_l$: $\tau(g)[g']=[gg']$, and $B$ be the frame bundle. We also consider the map $\sigma\colon \eR^{1,l-1}\to \sQ$ the isometry which sends the canonical basis of $\eR^{1,l-1}$ to the usual basis $\{ q_0,q_{1},\ldots,q_{l-1} \}$ of $\sQ$. The map
\begin{equation}
\begin{aligned}
 \beta\colon G\times_{\alpha} \SO(\sQ)&\to B \\ 
[g,A]&\mapsto d\tau(g)_{\mfo}A\circ\sigma 
\end{aligned}
\end{equation}
provides a principal bundle isomorphism between the principal bundle  \eqref{EqPrincPreB} and the frame bundle over $AdS_l$. 

\end{proposition}

By abuse of notation, we will not always write the $\sigma$.
\begin{proof}
We have to prove first that the map $\beta\colon G\times\SO(\sQ)\to B$ respects the classes. For that, consider $(g,A)\sim(g',A')$ and remark that
\[ 
\begin{split}
\beta(gh,\alpha(h^{-1}))&=d\tau(gh)_{\mfo}\alpha(h^{-1})A
        =d\tau(g)d\tau(h)d\pi \Ad(h^{-1})d\pi^{-1}A\\
        &=d\tau(g)d\tau(h)d\pi\Ad(h^{-1})d\pi^{-1} A
        =d\tau(g)d\pi dR_{h}d\pi^{-1}A\\
        &=d\tau(g)_{\mfo} A
        =\beta(g,A).
\end{split}  
\]
where we used equation \eqref{EqdpiAdpi} and the fact that $\pi\circ L_{g}=\tau(g)\circ\pi$. The frame bundle is
\begin{equation}   \label{EqPrincB}
\xymatrix{%
   \SO(1,l-1) \ar@{~>}[r]   &   B \ar[d]^{p}\\
                &      G/H
}
\end{equation}
where the fibre $B_{[g]}$ in $B$ over $[g]$ is the set of isometric maps $\eR^{1,l-1}\to T_{[g]}(AdS_l)$. So an element of $B$ is of the form $\big( [g],\tilde f\circ\sigma \big)$ where $g\in G$ and $\tilde f\colon \sQ\to T_{[g]}(AdS_l)$ contains the main information while $\sigma$ is the previously explained isometry. The action of $h\in\SO(1,l-1)$ on $\big( [g],\tilde f\circ\sigma \big)$ is defined by means of any fixed isomorphism $\varphi_{0}\colon \SO(1,l-1)\to \SO(\sQ)$ by
\begin{equation}
  \big( [g],\tilde f\circ\sigma \big)\cdot h=\big( [g],\tilde f\circ\varphi_{0}(h)\circ\sigma \big).
\end{equation}
The map $\beta$ is a morphism of principal bundle because
\[ 
\beta[g,A]\cdot\varphi_{0}^{-1}(B)  =\big( [g],d\tau(g)A\circ\sigma \big)\cdot \varphi_{0}^{-1}(B)
                    =\big( [g],d\tau(g)A\circ B\circ \sigma \big)
                    =\beta\big( [g,A]\cdot B \big).
\]
It remains to be proved that $\beta$ is a bijection. Surjectivity is natural: since $d\tau(g)$ is an isometry, $d\tau(g)A$ runs over the whole $\SO\big(T_{[g]}(AdS_l)\big)$ when $A$ runs over $\SO(\sQ)$. Injectivity is as follows; let's suppose $\beta[g,A]=\beta[g',A']$. It is immediate that in this case, $\exists h\in H$ such that $g'=gh$. Using the fact that $d\pi\circ dR_{h^{-1}}\circ d\pi^{-1}=\id$ and $d\tau(h)d\pi=d\pi dL_{h}$, we have
\[ 
d\tau(g)_{\mfo}A=d\tau(g)d\tau(h)A'
        =d\tau(g)d\tau(h)d\pi dR_{h^{-1}}d\pi^{-1}A'
        =d\pi\Ad(h)d\pi^{-1}A'
        =\alpha(h)A'.
\]

\end{proof}
From now on, we identify $G\times_{\alpha}\SO(\sQ)$ with the frame bundle over $AdS_l$. 

\subsection{Spin structure}
%--------------------------

We consider the principal bundle
\begin{equation}   
\xymatrix{%
   \Spin(1,l-1) \ar@{~>}[r]     &   G\times_{\tilde\alpha}\Spin(1,l-1)\ar[d]^{p}\\
                &      G/H
}
\end{equation}
where $\times_{\tilde\alpha}$ is the following equivalence relation on $G\times\Spin(1,l-1)$. We say that $(g,s)\sim(g',s')$ if and only if there exists a $h\in H$ such that
\begin{enumerate}
\item $g'=gh$,
\item $\chi(s')=\Ad(h^{-1})\chi(s)$.
\end{enumerate}
Notice that the second condition implies that $\Ad(h)\in \SO_0(\sQ)$. It is easy to prove that the given structure is well defined and is a principal bundle.  Now we consider the spin structure as follows:
\begin{equation}   
\xymatrix{%
   \Spin(1,l-1) \ar@{~>}[r]&G\times_{\tilde\alpha}\Spin(1,l-1)\ar[dr]\ar[rr]^-{\displaystyle\varphi}&&G\times_{\alpha} \SO(\sQ)\ar[dl]&\SO(\sQ)\ar@{~>}[l]  \\
                &      &G/H
}
\end{equation}
where $\varphi[g,s]=[g,\chi(s)]$. It is well defined since when $[g,s]=[g',s']$, there exists a $h\in H$ with $\chi(s')=\Ad(h^{-1})\chi(s)$ such that  $\varphi[g',s']=\varphi[gh,s']=[gh,\chi(s')]=[gh,\Ad(h^{-1})\chi(s)]=[g,\chi(s)]=\varphi[g,s]$.

\subsection{Reduction of the structural group}
%---------------------------------------------

The case of $AdS_l$ can be seen in the setting of subsection \ref{subsecCanConCovDer}. Let us show now that the bundle
\begin{equation}   \label{EqPrincHzGM}
\xymatrix{%
   H_0 \ar@{~>}[r]      &   G\ar[d]^{\pi}\\
                &   G/H
 }
\end{equation}
is a reduction to $H_0$ (the identity component of $\SO(\sQ)$) of
\begin{equation}
\xymatrix{%
   G \ar@{~>}[r]        &   r(G)\ar[d]^{\pi}\\
                &   G/H.
 }
\end{equation}
 Indeed, $u\colon G\to r(G)$ given by $u(g)=r(g)$ provides the reduction homomorphism: $r(gh)X=d\pi dL_{gh}X$ while $\big( r(g)\cdot h \big)X$ is the same.
 
\begin{lemma}
The tangent space $T(G/H)$ is an associated bundle of $r(G)$ trough the identification
\begin{equation}
\begin{aligned}
 \beta'\colon r(G)\times_{\rho}\sQ&\to T(G/H) \\ 
  [r(g),X]&\mapsto r(g)X 
\end{aligned}
\end{equation}
where $\rho(h)X=\Ad(h)X$, so that the quotient is given by $[g,X]=[gh,\Ad(h^{-1})X]$.
\end{lemma}

\begin{proof}
The proof is entirely similar to the one of lemma \ref{LemBazHGGH}.
\end{proof}
