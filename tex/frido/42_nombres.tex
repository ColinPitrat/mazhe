% This is part of Mes notes de mathématique
% Copyright (c) 2011-2018
%   Laurent Claessens
% See the file fdl-1.3.txt for copying conditions.

%+++++++++++++++++++++++++++++++++++++++++++++++++++++++++++++++++++++++++++++++++++++++++++++++++++++++++++++++++++++++++++
\section{Quelques éléments sur les ensembles}
%+++++++++++++++++++++++++++++++++++++++++++++++++++++++++++++++++++++++++++++++++++++++++++++++++++++++++++++++++++++++++++

%---------------------------------------------------------------------------------------------------------------------------
\subsection{Petit mot d'introduction}
%---------------------------------------------------------------------------------------------------------------------------

\begin{normaltext}

Le Frido n'est pas supposé être lu dans l'ordre de la première à la dernière page; les matières y sont présentées dans l'ordre logique mathématique, et non dans l'ordre logique pédagogique, et encore moins par ordre de difficulté croissante.

En mathématique, si on lit une démonstration et que l'on veut vraiment tout justifier, et justifier toutes les étapes de tous les résultats utilisés, on tombe forcément un jour sur les axiomes.

Or l'axiomatique est un sujet particulièrement difficile. Nous n'allons donc pas «tout justifier» jusque là. Nous n'allons même pas préciser quel système d'axiome est utilisé. En particulier nous n'allons pas donner l'axiomatique des ensembles : nous allons supposer connus les ensembles et leurs principales propriétés.

Bref. Nous supposons avoir une théorie des ensembles qui tient la route. En particulier nous supposons connues les notions suivantes :
\begin{enumerate}
    \item
        ensemble, appartenance, intersection, union,
    \item
        application entre deux ensembles, notation \( f(x)\) pour désigner l'image de \( x\) par \( f\),
    \item
        produit cartésien de plusieurs ensembles.
\end{enumerate}
Ce sont toutes des choses dont la construction à partir des axiomes n'est en aucun cas évidente. En particulier, des «définitions» comme «l'intersection de deux ensembles est l'ensemble contenant exactement les éléments communs aux deux ensembles» ne sont pas correctes parce qu'elles passent à côté de l'existence et de l'unicité d'un tel ensemble.
\end{normaltext}

\begin{normaltext}
    Remarquez par exemple que la première phrase de l'article de Wikipédia sur la construction de \( \eN\) est «Partant de la théorie des ensembles, on identifie 0 à l'ensemble vide, puis on construit \ldots». Il est bien précisé que l'on part d'une théorie des ensembles.
\end{normaltext}

\begin{normaltext}
    La suite de ce chapitre sera essentiellement sans exemples parce qu'avant d'avoir construit les ensembles de nombres, je ne sais pas très bien quels exemples on peut donner de quoi que ce soit.
\end{normaltext}

%---------------------------------------------------------------------------------------------------------------------------
\subsection{Ensemble ordonné}
%---------------------------------------------------------------------------------------------------------------------------

\begin{definition}
    Soient deux ensembles \( E\) et \( F\). Une application \( f\colon E\to F\) est
    \begin{enumerate}
        \item
            \defe{surjective}{surjection} si pour tout \( y\in F\), il existe \( x\in E\) tel que \( y=f(x)\);
        \item
            \defe{injective}{injection} si pour tout \( y\in F\), il existe au plus un \(x\in E \) tel que \( y=f(x)\);
        \item
            \defe{bijective}{bijection} si elle est à la fois injective et surjective.
    \end{enumerate}
\end{definition}
La méthode la plus courante pour démontrer qu'une application \( f\colon E\to F\) est injective est de considérer \( x,y\in E\) tels que \( f(x)=f(y)\), et de prouver à partir de là que \( x=y\). Ou alors de supposer \( x\neq y\) et obtenir une contradiction.

La technique de la contradiction est évidemment la plus courante lorsque l'égalité \( f(x)=g(x)\) implique une équation faisant intervenir \( 1/(x-y)\).

\begin{normaltext}\label{NORooLMBYooYjUoju}
L'\defe{axiome du choix}{axiome!du choix} que nous acceptons peut s'énoncer comme ceci\cite{ooKLIXooHbpufL} : Étant donné un ensemble X d'ensembles non vides, il existe une fonction définie sur X, appelée fonction de choix, qui à chacun d'entre eux associe un de ses éléments.
\end{normaltext}

\begin{definition}      \label{DefooFLYOooRaGYRk}
    Une \defe{relation d'ordre}{ordre} sur un ensemble \( E\) est une relation binaire (notée \( \leq\)) sur \( E\) telle que pour tous \( x,y,z\in E\),
    \begin{description}
        \item[réflexivité] : \( x\leq x\)
         \item[antisymétrie] : \( x\leq y\) et \( y\leq x\) implique \( x=y\)
         \item[transitivité] : \( x\leq y\) et \( y\leq z\) implique \( x\leq z\).
    \end{description}
\end{definition}

\begin{definition}      \label{DEFooVGYQooUhUZGr}
    Un ensemble ordonné est \defe{totalement ordonné}{ordre!total} si deux éléments sont toujours comparables : si \( x,y\in E\) alors nous avons soit \( x\leq y\) soit \( y\leq x\). Si les éléments ne sont pas tous comparables, nous disons que l'ordre est \defe{partiel}{ordre!partiel}.
\end{definition}

\begin{definition}
    Soit un ensemble ordonné \( (E,\leq)\) et une partie \( A\) de \( E\). Nous disons que \( m\in A\) est un \defe{minimum}{minimum!ensemble ordonné} de \( A\) si pour tout \( x\in A\), l'élément \( m\) est comparable à \( x\) et \( m\leq x\).

    Un élément \( p\in E\) est un \defe{minorant}{minorant} de \( A\) si pour tout \( a\in A\), l'élément \( p\) et \( a\) sont comparables et \( p\leq a\).

    Les notions de \defe{maximum}{maximum} et de \defe{majorant}{majorant} sont définis de façon analogue.
\end{definition}

Lorsqu'une partie possède un minimum, ce dernier est nommé le «plus petit élément» de la partie. Attention : il n'en existe pas toujours. D'innombrables exemples pourront être vus lorsque nous aurons construits \( \eQ\) et \( \eR\). Typiquement les intervalles du type \( \mathopen] a , b \mathclose[\).

\begin{definition}   \label{DEFooLJEAooBLGsiS}
    Un ensemble ordonné est \defe{bien ordonné}{bon!ordre}\index{ordre!bon ordre} si toute partie non vide possède un plus petit élément : si \( A\) est une partie de \( E\), alors \( \exists x\in A,\forall y\in A, x\leq y\).
\end{definition}

\begin{normaltext}
    Quelques remarques.
    \begin{enumerate}
        \item
            L'inégalité stricte (définie par: \( x<y\) si et seulement si \( x\leq y\) et \( x\neq y\)) n'est pas une relation d'ordre parce qu'elle n'est pas réflexive.
        \item
            Nous verrons dans la remarque~\ref{REMooXOIOooHjwMvA} que l'intervalle \( \mathopen[ -1 , 1 \mathclose]\) dans \( \eR\) n'est pas bien ordonné.
        \item
            Un ensemble bien ordonné est forcément totalement ordonné parce que toutes les parties de la forme \( \{ x,y \}\) possèdent un minimum. Par conséquent \( x\) et \( y\) doivent être comparables : \( x\leq y\) ou \( y\leq x\).
    \end{enumerate}
\end{normaltext}

\begin{example}
    Si \( E\) est un ensemble, l'inclusion est un ordre sur l'ensemble des parties de \( E\), mais pas un ordre total parce que si \( X,Y\) sont des parties de \( E\), alors nous n'avons pas automatiquement soit \( X\subset Y\) ou \( Y\subset X\).
\end{example}

La notion d'ordre permet d'introduire la notion d'intervalle.

\begin{definition}  \label{DefEYAooMYYTz}
    Soit un ensemble totalement ordonné \( (E,\leq)\). Un \defe{intervalle}{intervalle} de \( E\) est une partie \( I\) telle que tout élément compris entre deux éléments de \( I \) soit dans \( I \). En formule, la partie \( I \) de \( E\) est un intervalle si
    \[
      \forall a,b\in I,(a\leq x\leq b)\Rightarrow x\in I.
    \]
\end{definition}

%---------------------------------------------------------------------------------------------------------------------------
\subsection{Lemme de Zorn}
%---------------------------------------------------------------------------------------------------------------------------

\begin{definition}[Ensemble inductif\cite{MathAgreg}]  \label{DefGHDfyyz}
    Un ensemble est \defe{inductif}{inductif} si tout sous-ensemble totalement ordonné admet un majorant.
\end{definition}


\begin{lemma}[Lemme de Zorn]    \label{LemUEGjJBc}
    Tout ensemble ordonné inductif non vide admet un maximum.
\end{lemma}
\index{lemme!de Zorn}
%TODO : une preuve.
Le point intéressant de ce lemme est que le majorant soit un maximum, c'est à dire qu'il appartienne à l'ensemble.

\begin{definition}      \label{DefEOZLooUMCzZR}
    Un ensemble est \defe{infini}{ensemble!infini} s'il peut être mis en bijection avec un de ses sous-ensembles propres (c'est à dire différent de lui-même).
\end{definition}

La proposition suivante sera utilisée en théorie de la mesure, dans l'exemple~\ref{ExOIXoosScTC}. Ça utilise l'axiome du choix sous la forme du lemme de Zorn.
\begin{proposition}[\cite{ooFAOQooACUugI,KZIoofzFLV}] \label{PropVCSooMzmIX}
    Si \( S\) est un ensemble infini alors il existe une bijection \( \varphi\colon \{ 1,2 \}\times S\to S\).
\end{proposition}

%---------------------------------------------------------------------------------------------------------------------------
\subsection{Complémentaire}
%---------------------------------------------------------------------------------------------------------------------------
\label{AppComplement}

\begin{definition}
    Soit $E$, un ensemble et $A$, une partie de $E$ (c'est à dire un sous-ensemble de $E$). Le \defe{complémentaire}{complémentaire} de l'ensemble $A$, dans $E$, noté $\complement A$\nomenclature[T]{$\complement A$}{Le complémentaire de l'ensemble $A$} est l'ensemble des éléments de $E$ qui ne font pas partie de $A$ :
    \begin{equation}
	    \complement A=E\setminus A=\{ x\in E\tq x\notin A \}.
    \end{equation}
\end{definition}

Nous allons aussi régulièrement noter le complémentaire de \( A\) par \( A^c\)\nomenclature[T]{\( A^c\)}{complémentaire de \( A\)}.

\begin{lemma}		\label{LemPropsComplement}
	Quelques propriétés à propos des complémentaires. Si $E$ est un ensemble et si $A$ et $B$ sont des sous-ensembles de $E$, nous avons
	\begin{enumerate}
		\item
			$\complement \complement A =A $, en d'autres termes, $E\setminus(E\setminus A)=A$,
		\item
			$\complement(A\cap B)=\complement A\cup\complement B$,
		\item
			$\complement(A\cup B)=\complement A\cap\complement B$,
		\item	\label{ItemLemPropComplementiii}
			$A\setminus B=A\cap\complement B$.
	\end{enumerate}
\end{lemma}

\begin{definition}[différence symétrique]    \label{DefBMLooVjlSG}
    Si \( A\) et \( B\) sont des ensembles, l'ensemble \( A\Delta B\)\nomenclature[T]{\( A\Delta B\)}{différence symétrique} est la \defe{différence symétrique}{ensemble!différence symétrique} d'ensembles :
    \begin{equation}
        A\Delta B=(A\cup B)\setminus(A\cap B).
    \end{equation}
\end{definition}
C'est l'ensemble des éléments étant soit dans \( A\) soit dans \( B\) mais pas dans les deux.

\begin{lemma}   \label{LemCUVoohKpWB}
    Si \( A\) et \( B\) sont des ensembles nous avons
    \begin{enumerate}
        \item\label{ItemVUCooHAztC}
            \( A^c\Delta B^c=A\Delta B\).
        \item\label{ItemVUCooHAztCii}
            \( (A\Delta B)\Delta B=A\).
    \end{enumerate}
\end{lemma}

\begin{proof}
    D'abord nous avons l'égalité \( X^c\setminus Y^c=Y\setminus X\). Cela se prouve de façon classique en séparant deux cas selon que \( B\) soit inclus dans \( A\) ou non.

    De là nous avons la première assertion :
    \begin{equation}
        A^c\Delta B^c=(A^c\cup B^c)\setminus(A^c\cap B^c)=(A\cap B)^c\setminus(A\cup B)^c=(A\cup B)\setminus (A\cap B)=A\Delta B.
    \end{equation}

    Pour la seconde assertion, il faut remarquer que \( (A\Delta B)\cup B=A\cup B\) et que \( (A\Delta B)\cap B=B\setminus A\), donc
    \begin{equation}
        (A\Delta B)\Delta B=(A\cup B)\setminus (B\setminus A)=A.
    \end{equation}
\end{proof}

%---------------------------------------------------------------------------------------------------------------------------
\subsection{Relations d'équivalence}
%---------------------------------------------------------------------------------------------------------------------------
\label{appEquivalence}

\begin{definition}  \label{DefHoJzMp}
Si $E$ est un ensemble, une \defe{relation d'équivalence}{equivalence@équivalence!relation} sur $E$ est une relation $\sim$ qui est à la fois
\begin{description}
	\item[réflexive] $x\sim x$ pour tout $x\in E$,
	\item[symétrique] $x\sim y$ si et seulement si $y\sim x$;
	\item[transitive] si $x\sim y$ et $y\sim z$, alors $x\sim z$.
\end{description}
\end{definition}

Par exemple, sur l'ensemble de tous les polygones du plan, la relation «a le même nombre de côtés» est une relation d'équivalence. Plus précisément, si $P$ et $Q$ sont deux polygones, nous disons que $P\sim Q$ si et seulement si $P$ et $Q$ ont le même nombre de côtés. Cela est une relation d'équivalence :
\begin{itemize}
	\item
		un polygone $P$ a toujours le même nombre de côtés que lui-même : $P\sim P$;
	\item
		si $P$ a le même nombre de côtés que $Q$ ($P\sim Q$), alors $Q$ a le même nombre de côtés que $P$ ($Q\sim P$);
	\item
		si $P$ a le même nombre de côtés que $Q$ ($P\sim Q$) et que $Q$ a le même nombre de côtés que $R$ ($Q\sim R$), alors $P$ a le même nombre de côtés que $R$ ($P\sim R$).
\end{itemize}

\begin{example}
Soit \( f\) une application entre deux ensembles \( E\) et \( F\). Nous définissons une relation d'équivalence sur \( E\) par
\begin{equation}
    x\sim y\Leftrightarrow f(x)=f(y).
\end{equation}
Nous notons par \( \pi\colon E\to E/\sim\) la projection canonique. L'application
\begin{equation}
    \begin{aligned}
        g\colon E/\sim&\to F \\
        [x]&\mapsto f(x)
    \end{aligned}
\end{equation}
est bien définie et injective. Elle n'est pas surjective tant que \( f\) ne l'est pas. La \defe{décomposition canonique}{canonique!décomposition}\index{décomposition!canonique} de \( f\) est
\begin{equation}
    f=g\circ\pi.
\end{equation}
\end{example}

%+++++++++++++++++++++++++++++++++++++++++++++++++++++++++++++++++++++++++++++++++++++++++++++++++++++++++++++++++++++++++++
\section{Les naturels}
%+++++++++++++++++++++++++++++++++++++++++++++++++++++++++++++++++++++++++++++++++++++++++++++++++++++++++++++++++++++++++++
\label{SECooPJSYooNYaIaq}
% Lorsque ce chapitre est fait, changer la phrase qui le référentie dans la partie sur la constante de Weiner.

Toutes les constructions sont faites dans \cite{RWWJooJdjxEK}. Les résultats énoncés ici sont utilisés plus bas et servent de guide à un éventuel contributeur qui voudrait écrire une partie sur la construction de \( \eN\) et \( \eZ\). Nous espérons que des preuves se trouvent dans \cite{RWWJooJdjxEK}. En tout cas, le lecteur est invité à ne pas les prendre pour évidents.

\nomenclature{$\eN_0$}{les naturels non nuls : $\eN_0=\eN\setminus\{ 0 \}$}
\begin{definition}
    Un ensemble est \defe{dénombrable}{dénombrable} s'il peut être mis en bijection avec \( \eN\). Il est \defe{non dénombrable}{non dénombrable} s'il est infini et ne peut pas être mis en bijection avec \( \eN\).
\end{definition}
Une chose vraiment amusante avec cette définition que l'on met en rapport avec la définition~\ref{DefEOZLooUMCzZR}, c'est qu'un ensemble fini n'est ni dénombrable ni non dénombrable.

Les ensembles dénombrables sont les plus petits ensembles infinis possibles, comme en témoigne la proposition suivante.
\begin{proposition}[\cite{MonCerveau}]      \label{PROPooUIPAooCUEFme}
    Tout ensemble infini contient une partie en bijection avec \( \eN\).
\end{proposition}

\begin{proof}
    Soient un ensemble infini \( E_0\) et une partie propre \( E_1\) en bijection avec \( E_0\). Nous notons \( \varphi\colon E_0\to E_1\) une bijection.

    Soit \( x_0\in E_0\setminus E_1\) (axiome du choix et tout ça). Nous définissons
    \begin{equation}
        \begin{aligned}
            \psi\colon \eN&\to \{\varphi^k(x_0)\} \\
            n&\mapsto \varphi^n(x_0)
        \end{aligned}
    \end{equation}
    et nous allons prouver que c'est une bijection. Que ce soit surjectif est immédiat. Pour l'injectivité, soit \( \varphi^k(x_0)=\varphi^l(x_0)\) avec \( k\neq l\). Supposons pour fixer les notations que \( k>l\). Alors, vu que \( \varphi\) est inversible nous pouvons écrire
    \begin{equation}
        x_0=\varphi^{k-l}(x_0)=\varphi\big( \varphi^{k-l-1}(x_0) \big)
    \end{equation}
    où il est entendu que \( \varphi^0(x_0)=x_0\). Cela signifie que \( x_0\) est dans l'image de \( \varphi\), c'est à dire dans $E_1$, ce que nous avons exclu par choix de \( x_0\) dans \( E_0\setminus E_1\). Donc en réalité \( \varphi^k(x_0)\neq \varphi^l(x_0)\) dès que \( k\neq l\).
\end{proof}

\begin{proposition} \label{PropQEPoozLqOQ}
    Toute partie d'un ensemble fini est finie, et toute partie d'un ensemble dénombrable est finie ou dénombrable.
\end{proposition}
%TODO : la preuve
% Attention : cette proposition est citée au moins une fois à un endroit où il faudrait quelque chose de plus précis.

%---------------------------------------------------------------------------------------------------------------------------
\subsection{La construction}
%---------------------------------------------------------------------------------------------------------------------------

Cette section est vide, insuffisamment détaillée ou incomplète. Votre aide est la bienvenue ! \hyperref[SecooCKWWooBFgnea]{Comment faire ?}

%---------------------------------------------------------------------------------------------------------------------------
\subsection{Quelques résultats}
%---------------------------------------------------------------------------------------------------------------------------

\begin{proposition}     \label{PROPooBYKCooGDkfWy}
    Deux résultats à propos de la cardinalité de \( \eN\).
    \begin{enumerate}
        \item
            L'ensemble \( \eN\) est infini (définition~\ref{DefEOZLooUMCzZR}).
        \item
            Tout ensemble infini contient une partie en bijection avec \( \eN\).
    \end{enumerate}
\end{proposition}

%+++++++++++++++++++++++++++++++++++++++++++++++++++++++++++++++++++++++++++++++++++++++++++++++++++++++++++++++++++++++++++
\section{Groupes, anneaux}
%+++++++++++++++++++++++++++++++++++++++++++++++++++++++++++++++++++++++++++++++++++++++++++++++++++++++++++++++++++++++++++

%---------------------------------------------------------------------------------------------------------------------------
\subsection{Groupes}
%---------------------------------------------------------------------------------------------------------------------------

\begin{definition}[Groupe]
    Un \defe{groupe}{groupe} est un ensemble \( G\) muni d'une opération interne \( \cdot\colon G\times G\to G\) telle que
    \begin{enumerate}
        \item
            pour tous \( g\), \( h\), \( k\in G\), \( g\cdot(h\cdot k)=(g\cdot h)\cdot k\),
        \item
            il existe un élément \( e\in G\) tel que \( e\cdot g=g\cdot e=g\) pour tout \( g\in G\),
        \item
            pour tout \( g\in G\), il existe un élément \( h\in  G\) tel que \(g\cdot h=h\cdot g=e \).
    \end{enumerate}
\end{definition}

    Notons que nous avons écrit \( g\cdot h\) et non \( \cdot(g,h)\) comme une notation purement fonctionnelle nous l'aurait suggéré. Dans les exemples concrets, selon les cas, le loi de groupe appliquée à \( g\) et \( h\) sera notée tantôt \( g+h\), tantôt \( g\cdot h\) ou, le plus souvent pour un groupe générique, simplement \( gh\).

\begin{lemmaDef}[Unicités]  \label{LEMooECDMooCkWxXf}
    L'inverse et le neutre sont uniques, c'est-à-dire :
    \begin{enumerate}
        \item
            il existe un unique élément \( e\in G\) tel que \( e g=g e=g\) pour tout \( g\in G\),
        \item       \label{ITEMooOIWTooYqmMPP}
            pour tout \( g\in G\), il existe un unique élément \( h\in  G\) tel que \(g h=h g=e \).
    \end{enumerate}
    Le \( e\) ainsi défini est nommé \defe{neutre}{neutre!dans un groupe} de \( G\). Le \( h\) tel que \( g h=h g=e\) est nommé l'\defe{inverse}{inverse!dans un groupe} de \( g\) et est noté \( g^{-1}\).
\end{lemmaDef}

\begin{proof}
    Chaque point séparément.
    \begin{enumerate}
        \item
            Supposons que \( e_1\) et \( e_2\) vérifient la propriété. Nous avons pour tout \( g\in G\) : \( e_1g=ge_1=g\). En particulier pour \( g=e_2\) nous écrivons \( e_1e_2=e_2e_1=e_2\). Mais en partant dans l'autre sens : \( e_2g=ge_2=g\) avec \( g=e_1\) nous avons \( e_2e_1=e_1e_2=e_2\). En égalant ces deux valeurs de \( e_2e_1\) nous avons \( e_1=e_2\).

            Pour la suite de la preuve nous écrivons \( e\) l'unique neutre de \( G\).

        \item
            Supposons que \( k_1\) et \( k_2\) soient deux inverses de \( g\). On considère alors le produit \( k_1 g k_2 \). Puisque \(k_1 g = e \), on a \( k_1 g k_2 = e k_2 = k_2 \); mais, comme \(g k_2 = e \), on a aussi \( k_1 g k_2 = k_1 e = k_1 \). Le produit est donc à la fois égal à \( k_1 \) et à \( k_2 \), et donc \( k_1 = k_2 \).
    \end{enumerate}
\end{proof}

\begin{definition}
    Un groupe \( G\) est \defe{abélien}{abélien!groupe} ou \defe{commutatif}{commutatif!groupe} si pour tous \( g\) et \( h\) dans \( G\), \( gh=hg\).
\end{definition}

%---------------------------------------------------------------------------------------------------------------------------
\subsection{Anneaux}
%---------------------------------------------------------------------------------------------------------------------------

\begin{definition}[Anneau\cite{Tauvel}]     \label{DefHXJUooKoovob}
    Un \defe{anneau}{anneau}\footnote{Nous faisons le choix qu'un anneau admet toujours un neutre pour la multiplication. Certains ouvrages parlent dans ce cas d'anneau unitaire.} est un triplet \( (A,+,\cdot)\) avec les conditions
    \begin{enumerate}
        \item
            \( (A,+)\) est un groupe abélien. Nous notons \( 0\) le neutre.
        \item
            La multiplication est associative et nous notons \( 1\) le neutre.
        \item
            La multiplication est distributive par rapport à l'addition.
    \end{enumerate}
\end{definition}

\begin{lemma}       \label{LEMooVUSMooWisQpD}
    Pour tout élément \( a\) d'un anneau nous avons \( a\cdot 0=0\).
\end{lemma}

\begin{proof}
    L'élément \( 0\) est le neutre de l'addition. Il peut être écrit \( 1-1\), et en utilisant la distributivité,
    \begin{equation}
        a\cdot 0=a\cdot (1-1)=a-a=0.
    \end{equation}
    Notons que la dernière égalité s'écrit en détail \( a-a=a+(-a)\) qui donne le neutre de l'addition.
\end{proof}

\begin{proposition}     \label{PROPooNCCGooXjVyVt}
    Dans un anneau non nul, le neutre pour l'addition est distinct du neutre pour la multiplication.
\end{proposition}
\begin{proof}
    Supposons par contraposée que dans un anneau $A$, \( 1 = 0 \). Alors, pour tout \( a \in A \), on a \( a = 1a = 0a = (1 - 1)a = a - a=0 \), d'où l'on déduit \( -a = 0  \) et par suite, \( a = 0. \)
\end{proof}

Soit \( X\) un ensemble et un anneau $(A, +, \times)$. Nous considérons \( \Fun(X,A)\)\nomenclature[A]{\( \Fun(X,Y)\)}{les applications de \( X\) vers \( Y\)} l'ensemble des applications \( X\to A\). Cet ensemble devient un anneau avec les définitions
\begin{subequations}
    \begin{align}
        (f+g)(x)=f(x)+g(x)\\
        (fg)(x)=f(x)g(x).
    \end{align}
\end{subequations}
Cela est la \defe{structure canonique}{structure d'anneau canonique} d'anneau sur \( \Fun(X,A)\).

Le \defe{centralisateur}{centralisateur} de \( x\in A\) dans \( A\) est l'ensemble
\begin{equation}
    \{ y\in A\tq xy=yx \},
\end{equation}
le \defe{centre}{centre!d'un anneau} de \( A\) est
\begin{equation}
    \{ y\in A\tq xy=yx,\forall x\in A \}.
\end{equation}

\begin{definition}[Morphisme d'anneaux]      \label{DEFooQBGJooKJqHXr}
    Si \( (A,+,\cdot)\) et \( (B,+,\cdot)\) sont des anneaux, un \defe{morphisme}{morphisme!d'anneaux} est une application \( f\colon A\to B\) telle que
    \begin{enumerate}
        \item \( f(a+b)=f(a)+f(b)\)
        \item
            \( f(a\cdot b)=f(a)\cdot f(b)\)
        \item
            \( f(1)=1\).
    \end{enumerate}
    Étant bien entendu que les significations de \( 1\), $+$ et \( \cdot\) sont différentes à gauche et à droite.

    Un \defe{isomorphisme}{isomorphisme!d'anneaux} est un morphisme bijectif.
\end{definition}


\begin{definition}[Idéal dans un anneau]  \label{DefooQULAooREUIU}
    Un sous-ensemble \( I\subset A\) est un \defe{idéal à gauche}{idéal!dans un anneau} à gauche si
    \begin{enumerate}
        \item
            \( I\) est un sous-groupe pour l'addition,
        \item
            pour tout \( a\in A\), \( aI\subset I\).
    \end{enumerate}

    Lorsqu'un ensemble est idéal à gauche et à droite, nous disons que c'est un \defe{idéal bilatère}{idéal!bilatère}. Lorsque nous parlons d'idéal sans précisions, nous parlons d'idéal bilatère.
\end{definition}


%+++++++++++++++++++++++++++++++++++++++++++++++++++++++++++++++++++++++++++++++++++++++++++++++++++++++++++++++++++++++++++
\section{Les entiers}
%+++++++++++++++++++++++++++++++++++++++++++++++++++++++++++++++++++++++++++++++++++++++++++++++++++++++++++++++++++++++++++

\begin{lemma}       \label{LEMooMYEIooNFwNVI}
    Toute partie bornée de \( \eZ\) possède un plus grand élément.
\end{lemma}

%---------------------------------------------------------------------------------------------------------------------------
\subsection{Quelques autres résultats}
%---------------------------------------------------------------------------------------------------------------------------

Nous supposons ici connaître, et avoir démontré les rudiments du calcul dans \( \eN\) et \( \eZ\). En particulier les produits remarquables, et les implications comme \( a>b\) implique qu'il n'existe pas de \( x\) tel que \( a+x=b\).

La proposition suivante donne une bijection explicite entre \( \eN\) et \( \eN\times \eN\). Elle n'a rien de transcendante, mais je ne résiste pas à la donner ici parce qu'elle est utilisée dans l'article \emph{Un peu de programmation transfinie} de David Madore\footnote{Et comme j'aime beaucoup cet article, il me fallait une excuse pour le placer ici.\\ \url{http://www.madore.org/~david/weblog/d.2017-08-18.2460.html}.}. Son utilité est de pouvoir créer un langage de programmation pouvant traiter des paires d'entiers rien qu'en traitant des entiers.
\begin{proposition}[Une bijection \( \eN\times \eN\to \eN\)]        \label{PROPooLPKUooAlsYJg}
    La fonction
    \begin{equation}
        \begin{aligned}
            f\colon \eN\times \eN&\to \eN \\
            (x,y)&\mapsto \begin{cases}
                y^2+x    &   \text{si } x<y\\
                x^2+x+y    &    \text{si } y\leq x.
            \end{cases}
        \end{aligned}
    \end{equation}
    est une bijection.
\end{proposition}

\begin{proof}
    Il s'agit de prouver qu'elle est injective et surjective. Dans la suite, tous les nombres sont des entiers positifs.
    \begin{subproof}
        \item[\( f\) est injective]

            Pour \( k\in \eN\) donné, nous allons prouver que
            \begin{enumerate}
                \item
                    l'équation \( f(x,y)=k\) possède au maximum une solution avec \( x<y\),
                \item
                    l'équation \( f(x,y)=k\) possède au maximum une solution avec \( y\leq x\),
                \item
                    si \(   k=y^2+x \) avec \( x<y\) alors il est impossible que \( k=x'^2+y'\) avec \( y\leq x\).
            \end{enumerate}
            Supposons \( k=y^2+x\) avec \( x<y\). Alors aucun \( y'\) de la forme \( y+s\) ne peut être solution parce que
            \begin{equation}
                y^2+x= k=(y+s)^2+x'=y^2+2sy+s^2+x',
            \end{equation}
            ce qui impliquerait \( x'=x-2sy-s^2<0\). Il ne peut donc pas y avoir deux solutions.

            Supposons de la même manière que \( k=x^2+x+y\). Alors il n'existe pas de solutions avec \( x'=x+s\) parce que cela donnerait
            \begin{equation}
                (x+s)^2+(x+s)+y'=x^2+x+y,
            \end{equation}
            et donc
            \begin{equation}
                y'=y-2sx-s^2-s<0.
            \end{equation}

        \item[\( f\) est surjective]

            Nous devons prouver que tous les éléments de \( \eN\) sont dans l'image de \( \eN\times \eN\) par \( f\). En premier lieu, \( 0=f(0,0)\). C'est un bon début. Soit \( a\in \eN\) non nul; nous montrons que tous les nombres de \( a^2\) à \( (a+1)^2\) sont des images de \( f\). D'abord \( a^2=f(0,a)\), ensuite les nombres
            \begin{equation}
                f(1,a),f(2,a),\ldots, f(a-1,a)
            \end{equation}
            prennent les valeurs \( a^2+1\), \ldots, \( a^2+a-1\). Enfin nous avons \( f(a,0)=a^2+a\) et les nombres \( f(a,1),\ldots, f(a,a)\) prennent les valeurs de \( a^2+a+1\) à \( a^2+2a=(a+1)^2-1\).
    \end{subproof}
\end{proof}
Sachez que cette fonction s'étend aux ordinaux (mais là ce n'est plus pour rigoler).

%---------------------------------------------------------------------------------------------------------------------------
\subsection{Anneau intègre}
%---------------------------------------------------------------------------------------------------------------------------

\begin{definition}[Diviseurs dans un anneau]\label{DiviseursAnneau}
	Soient \( a, b \in A \). On dit que $a$ divise $b$, ou que $a$ est un \defe{diviseur (à gauche)}{diviseur!dans un anneau} de $b$ s'il existe \( c \in A \) tel que \( ac = b \). On dit que c'est un diviseur de $b$ à droite si \( ca = b \) pour un certain \( c \in A \).
\end{definition}
Un cas particulier est le cas des diviseurs de zéro. L'absence de tels diviseurs dans un anneau est une propriété intéressante: on dit dans ce cas que l'anneau est intègre. Nous étudions ces anneaux plus en détail en section~\ref{SECAnneauxIntegres}.

Un élément \( a\in A\) est \defe{régulier à droite}{régulier à droite} si \( ba=0\) implique \( b=0\). Il est régulier à gauche si \( ab=0\) implique \( b=0\).

\begin{definition}[Éléments nilpotents, unipotents et inversibles]
	On dit que \( a \in A \) est \defe{nilpotent}{nilpotent} s'il existe \( n \in \eN \) tel que \( a^n = 0 \). Il est dit \defe{unipotent}{unipotent} si \( a-1\) est nilpotent, c'est à dire si \( (a-1)^n =0\) pour un certain \( n \in \eN \).

	Un élément \( a \in A \) est dit \defe{inversible}{élément!inversible!dans un anneau} s'il existe \( b \in A \) tel que \( ab = 1 \).
\end{definition}

L'ensemble \( U(A)\)\nomenclature[A]{\( U(A)\)}{ensemble des inversibles} des éléments inversibles de \( A\) est un groupe pour la multiplication. Nous notons \( A^*=A\setminus\{ 0 \}\).

Rappelons la notion de diviseurs que nous avons fixée par la définition~\ref{DiviseursAnneau}.  Un élément \( a\neq 0\) est un \defe{diviseur de zéro à gauche}{diviseur!de zéro} s'il existe \( x\neq 0\) tel que $xa=0$. L'élément \( a\) est un diviseur de zéro \defe{à droite}{diviseur!de zéro à droite} s'il existe \( b\) tel que \( ab=0\).
\begin{definition}      \label{DEFooTAOPooWDPYmd}
    Un anneau est \defe{intègre}{intègre!anneau}\index{anneau!intègre} s'il est non nul et ne possède pas de diviseurs de zéro.
\end{definition}

\begin{proposition}     \label{PROPooZWUEooUHTtUI}
    Soit $A$ un anneau. Les assertions suivantes sont équivalentes:
    \begin{enumerate}
        \item
            $A$ est un anneau intègre.
        \item
            La règle du produit nul s'applique dans $A$: pour tous \( a, b \in A \), si \( ab=0\), alors \( a = 0\) ou \( b = 0\).
            \index{règle!du produit nul}
        \item       \label{ITEMooQNTFooSRrVPK}
            On peut simplifier par un même élément non-nul, deux expressions produit dans $A$ qui sont égales: pour tous \( a, b, c \in A \) avec \( a \neq 0 \), si \( ab = ac \), alors \( b = c \).
    \end{enumerate}
\end{proposition}

%+++++++++++++++++++++++++++++++++++++++++++++++++++++++++++++++++++++++++++++++++++++++++++++++++++++++++++++++++++++++++++
\section{Corps}
%+++++++++++++++++++++++++++++++++++++++++++++++++++++++++++++++++++++++++++++++++++++++++++++++++++++++++++++++++++++++++++

%---------------------------------------------------------------------------------------------------------------------------
\subsection{Définitions, morphismes}
%---------------------------------------------------------------------------------------------------------------------------

\begin{definition}  \label{DefTMNooKXHUd}
    Un \defe{corps}{corps} est un anneau non réduit à \( \{ 0 \}\) dans lequel tout élément non nul est inversible.
\end{definition}

\begin{remark}      \label{REMooYRNUooYgBBKF}
    Un anneau est ce qu'on appelle «\emph{ring}» en anglais. Un corps est en anglais «\emph{field}». De plus le mot «\emph{field}» comprend la commutativité. Donc certains utilisent le mot «corps» pour dire «corps commutatif» et parlent alors d'anneau \emph{à division} pour parler de corps non commutatifs.
\end{remark}

La proposition suivante donne une caractérisation d'un corps, en disant un tout petit peu plus que la définition~\ref{DefTMNooKXHUd}.
\begin{proposition}
    L'anneau $A$ est un corps si et seulement si \( U(A) = A^* \).
\end{proposition}

\begin{proof}
    L'inclusion \( A^*\subset U(A)\) est la définition.

    Pour l'inclusion inverse, il faut montrer que les inversibles ne peuvent pas être zéro, c'est à dire que zéro n'est pas inversible. Cela n'est autre que le lemme~\ref{LEMooVUSMooWisQpD} couplé à la proposition~\ref{PROPooNCCGooXjVyVt} : \( a\cdot 0=0\neq 1\) pour tout \( a\).
\end{proof}

\begin{lemma}       \label{LemAnnCorpsnonInterdivzer}
    En tant que anneau, un corps est un anneau intègre : il n'a pas de diviseurs de zéro.
\end{lemma}

\begin{proof}
    En effet si \( a\) est un diviseur de zéro, alors \( ax=0\) pour un certain \( x\neq 0\). Si \( a\) était inversible, nous aurions \( x=a^{-1}ax=0\), ce qui est impossible.
\end{proof}
Conséquence : dans un corps nous avons toujours la règle du produit nul, et l'élément nul n'est jamais inversible.

\begin{definition}[Morphisme de corps]
    Un corps étant un anneau sans plus de structure, un \defe{morphisme de corps}{morphisme!de corps}\index{isomorphisme!de corps} n'est qu'un morphisme des anneaux.
\end{definition}

Le lemme suivant montre que définir un morphisme de corps comme étant simplement un morphisme des anneaux est une bonne idée.
\begin{lemma}       \label{LEMooWBOPooZnsZgQ}
    Si \( \phi\colon \eK\to \eK'\) est un morphisme de corps, alors
    \begin{enumerate}
        \item
            pour tout \( a\in \eK\) nous avons \( \varphi(a^{-1})=\varphi(a)^{-1}\);
        \item
            le morphisme \( \varphi\) est injectif.
    \end{enumerate}
\end{lemma}

\begin{proof}
    Vu que \( \varphi(1)=1\), nous avons aussi
    \begin{equation}
        1=\varphi(aa^{-1})=\varphi(a)\varphi(a^{-1}).
    \end{equation}
    Donc, par unicité de l'inverse\footnote{Lemme~\ref{LEMooECDMooCkWxXf}\,\ref{ITEMooOIWTooYqmMPP}.}, \( \varphi(a^{-1})=\varphi(a)^{-1}\).

    Pour l'injectivité nous supposons \( \varphi(a)=\varphi(b)\). Étant donné que \( \eK'\) est un corps, nous pouvons multiplier par \( \varphi(b)^{-1}\) :
    \begin{equation}
        \varphi(a)\varphi(b)^{-1}=1.
    \end{equation}
    En utilisant le premier point nous avons \( 1=\varphi(a)\varphi(b^{-1})\), puis le morphisme d'anneaux : \( 1=\varphi(ab^{-1})\), et encore le morphisme d'anneaux nous permet de déduire \( ab^{-1}=1\) et donc \(a=b\).
\end{proof}

%---------------------------------------------------------------------------------------------------------------------------
\subsection{Corps des fractions}
%---------------------------------------------------------------------------------------------------------------------------

\begin{definition}[\cite{ooGSDHooLgtHCb}]       \label{DEFooGJYXooOiJQvP}
    Soit un anneau commutatif et intègre\footnote{Définition~\ref{DEFooTAOPooWDPYmd}.} \( A\) et \( E=A\times A\setminus\{ 0 \}\). Nous y définissons les deux opérations suivantes :
    \begin{enumerate}
        \item
            \( (a,b)+(c,d)=(ad+cb,bd)\);
        \item
            \( (a,b)(c,d)=(ac,bd)\).
    \end{enumerate}
    Et aussi la relation d'équivalence \( (a,b)\sim(c,d)\) si et seulement si \( ad=bc\).

    Le \defe{corps des fractions}{corps!des fractions} de \( A\) est le quotient
    \begin{equation}
        \Frac(A)=\big( A\times A\setminus\{ 0 \} \big)/\sim.
    \end{equation}
    Nous notons \( a/b\) la classe de \( (a,b)\).

    Les éléments de \( \Frac(A)\) sont des \defe{fractions rationnelles}{fractions!rationnelles} de \( A\).
\end{definition}
Le fait que \( A\) soit intègre est important pour être certain que \( bd\neq 0\) sous l'hypothèse que \( b,d\neq 0\).

La proposition suivante dit que \( \Frac(A)\) est le plus petit corps contenant \( A\).

\begin{proposition}[\cite{MonCerveau}]      \label{PROPooGSHDooJOnDsp}
    Si \( \eL\) est un corps contenant \( A\) en tant que sous-anneau\footnote{Bien entendu, nous pouvons trouver plein de corps contenant \( A\) en tant que sous-ensemble sans pour autant étendre \( A\); il suffit d'y mettre une loi de composition farfelue.}, alors il existe un morphisme de corps injectif \( \epsilon\colon \Frac(A)\to \eL\).
\end{proposition}

\begin{proof}
    Il suffit de vérifier que la formule
    \begin{equation}        \label{EQooCVOBooQEQJBM}
        \epsilon(a/b)=ab^{-1}
    \end{equation}
    vérifie toutes les conditions. Notons que dans le membre de droite de \eqref{EQooCVOBooQEQJBM}, l'inverse et le produit sont calculés dans \( \eL\).

    Le fait que \( \epsilon\) soit bien défini provient du fait que \( A\) soit commutatif :
    \begin{equation}
        \epsilon(ax/bx)=(ax)(bx)^{-1}=ab^{-1}xx^{-1}=ab^{-1}=\epsilon(a/b).
    \end{equation}

    Le fait que \( \epsilon\) soit un morphisme est une vérification de routine, par exemple ceci pour la somme :
    \begin{equation}
        \epsilon\big( a/b+c/d \big)=\epsilon\big( (ad+cb)/bd)\big)=(ad+cb)(bd^{-1})=ab^{-1}+cd^{-1},
    \end{equation}
    tandis que
    \begin{equation}
        \epsilon(a,b)+\epsilon(c,d)=ab^{-1}+cd^{-1},
    \end{equation}
    qui est égal (il faut aussi vérifier pour le produit).

    Enfin \( \epsilon\) est injective parce que si \( \epsilon(a/b)=\epsilon(c/d)\), alors \( ab^{-1}=cd^{-1}\), d'où il est facilement vu que \( ad=cb\), c'est à dire \( a/b=c/d\).
\end{proof}

Notons que si \( A\) est un anneau qui n'est pas un corps, le corps \( \Frac(A)\) existe, mais si \( R\in\Frac(A)\), il n'a pas de sens de vouloir calculer \( R(\alpha)\) pour \( \alpha\in A\).

\begin{definition}[Évaluation d'une fraction rationnelle]       \label{DEFooLBIWooCPCaSY}
    Soit un corps \( \eK\) contenant l'anneau \( A\). Si \( R=P/Q\in \Frac(A)\) et si \( \alpha\in \eK\) nous définissons
    \begin{equation}
        R(\alpha)=(P/Q)(\alpha)=P(\alpha)Q^{-1}(\alpha).
    \end{equation}
    Dans cette formule, les polynômes, l'inverse et le produit sont calculés dans \( \eK\) et non dans \( A\).
\end{definition}

\begin{theoremDef}     \label{ThogbhWgo}
    Soit \( \eA\) un anneau commutatif intègre.

    \begin{enumerate}
        \item
    Il existe un corps commutatif \( \eK\) et un morphisme d'anneaux injectif \( \epsilon\colon \eA\to \eK\) tels que pour tout \( \lambda\in\eK\), il existe \( (a,b)\in \eA\times \eA^*\) tels que
    \begin{equation}
        \lambda=\epsilon(a)\big( \epsilon(b) \big)^{-1}
    \end{equation}
\item
    Si \( (\eK',\epsilon')\) est un autre couple qui vérifie la propriété, les corps \( \eK\) et \( \eK'\) sont isomorphes.
    \end{enumerate}

    Le corps \( \eK\) associé à l'anneau \( \eA\) est le \defe{corps des fractions}{corps!des fractions}\index{fractions (corps)} de \( \eA\), et sera noté \( \Frac(\eA)\).\nomenclature[A]{\( \Frac(\eA)\)}{Le corps des fractions de l'anneau \( \eA\)}
\end{theoremDef}

\begin{lemma}[Simplification de fraction]
    L'application \( \eA\times \eA^*\to \eK\) donnée par \( (a,b)\mapsto \epsilon(a)\big( \epsilon(b) \big)^{-1}\) envoie \( (xa,xb)\) sur le même que \( (a,b)\).
\end{lemma}

La proposition suivante montre encore que le corps des fractions est le plus petit corps que l'on puisse imaginer à partir d'un anneau.
\begin{proposition}
    Soit un anneau \( A\). Tout corps contenant un sous-anneau isomorphe à \( A\) contient un sous-corps isomorphe à \( \Frac(A)\).
\end{proposition}

\begin{proof}
    Soit un corps \( \eK\) contenant un sous-anneau \( A'\) isomorphe à \( A\). Nous considérons la partie suivante de \( \eK\) :
    \begin{equation}
        S=\{ ab^{-1}\tq a,b\in A \}.
    \end{equation}
    Ensuite nous montrons que
    \begin{equation}
        \begin{aligned}
            \varphi\colon S&\to \Frac(A) \\
            ab^{-1}&\mapsto a/b
        \end{aligned}
    \end{equation}
    est un isomorphisme de corps.

    \begin{subproof}
        \item[Bien définie]

            Si \( ab^{-1}=xy^{-1}\) alors \( ay=xb\) et donc \( a/b=x/y\) par définition des classes de \( \Frac(A)\).

        \item[Surjectif]

            Tout élément de \( \Frac(A)\) est de la forme \( a/b\) avec \( a,b\in A\). Un tel élément est l'image par \( \varphi\) de \( ab^{-1}\in S\).

        \item[Injectif]

            Si \( \varphi(ab^{-1})=\varphi(xy^{-1})\) alors \( a/b=x/y\), et par définition des classes nous avons \( ay=bx\) qui donne immédiatement \( ab^{-1}=xy^{-1}\).
    \end{subproof}
\end{proof}

%---------------------------------------------------------------------------------------------------------------------------
\subsection{Suites de Cauchy dans un corps totalement ordonné}
%---------------------------------------------------------------------------------------------------------------------------

\begin{definition}      \label{DefKCGBooLRNdJf}
    Ordre et choses reliées dans un corps.
    \begin{enumerate}
        \item \label{ITEMooOOOVooJWwIQr}
            Un corps \( \eK\) est \defe{totalement ordonné}{ordre!dans un corps}\index{corps!ordonné} s'il existe une relation d'ordre total\footnote{Définition~\ref{DEFooVGYQooUhUZGr}.} tel que
            \begin{enumerate}
                \item
                    \( x\leq y\) implique \( x+z\leq y+z\) pour tout \( x,y,z\in \eK\)
                \item   \label{CONDooBYYDooElXgPO}
                    \( x\geq 0\) et \( y\geq 0\) implique \( xy\geq 0\).
            \end{enumerate}
        \item       \label{ItemooWUGSooRSRvYC}
            Si \( \eK\) est un corps totalement ordonné, nous y définissons la valeur absolue par
            \begin{equation}
                | x |=\begin{cases}
                    x    &   \text{si }x\geq 0\\
                    -x    &    \text{si } x\leq 0.
                \end{cases}
            \end{equation}
        \item       \label{ItemVXOZooTYpcYN}
    La suite \( (x_n)\) dans le corps totalement ordonné \( \eK\) est \defe{de Cauchy}{suite!de Cauchy!dans un corps} si pour tout \( \epsilon\in \eK^+\), il existe \( N\in \eN\) tel que si \( p,q\geq N\) alors \( | x_p-x_q |\leq \epsilon\).
\item       \label{ITEMooDERQooLmJwFR}
    La suite \( (x_n)\) dans le corps totalement ordonné \( \eK\) est \defe{convergente}{convergence!suite!dans un corps} s'il existe \( q\in \eK\) tel que pour tout \( \epsilon\in \eK^+\), il existe \( N\) tel que si \( k\geq N\) alors \( | x_k-q |\leq \epsilon\).
\item   \label{ItemooDZQKooPsqeRf}
            Un corps \( \eK\) est \defe{archimédien}{corps!archimédien}\index{archimédien} s'il est totalement ordonné et si pour tout \( x,y\in \eK\) avec \( x>0\), il existe \( n\in \eN\) tel que \( nx\geq y\).
        \item       \label{ITEMooKZZYooDaidGU}
            Un corps totalement ordonné est \defe{complet}{corps!complet}\index{complet!corps} si toute suite de Cauchy y est convergente.
        \item       \label{ITEMooMWASooEzhVyh}
            Si \( x,\epsilon\in \eK\) alors nous définissons la \defe{boule ouverte}{boule dans un corps} de centre \( a\) et de rayon \( \epsilon\) par
            \begin{equation}
                B(x,\epsilon)=\{ y\in \eK\tq | x-y |<\epsilon \},
            \end{equation}
            et la \defe{boule fermée}{boule dans un corps} par
            \begin{equation}
                \overline{ B(x,\epsilon) }=\{ y\in \eK\tq | x-y |\leq \epsilon \}.
            \end{equation}

    \end{enumerate}
\end{definition}

\begin{remark}
    Nous étudierons plus tard la notion de caractéristique d'un anneau\footnote{définition~\ref{LEMDEFooVEWZooUrPaDw}} ou d'un corps. Tout anneau totalement ordonné est nécessairement de caractéristique nulle, lemme~\ref{LEMooJQIKooQgukqn}.
\end{remark}

\begin{remark}
    En mettant côte à côte les points~\ref{ITEMooDERQooLmJwFR} et~\ref{ITEMooMWASooEzhVyh} nous pouvons dire que \( (x_k)\) converge vers \( q\) si et seulement si pour tout \( \epsilon>0\), nous avons \( x_k\in B(q,\epsilon)\) à partir d'un certain indice \( N\).

    Ces boules prendront une nouvelle force avec le super-théorème~\ref{ThoORdLYUu}.
\end{remark}

Parmi ces définitions, celles de suite convergente, de Cauchy et de corps complet seront utilisées dans le cas de \( \eQ\) (et de \( \eR\) pour la complétude). Elles seront prouvées être équivalentes aux définitions topologiques dans le cas particulier de \( \eR\) et \( \eQ\) lorsque la topologie métrique sera définie. Dans cet état d'esprit nous n'allons pas démontrer tout de suite que \( \eR\) est un corps complet. Nous allons directement démontrer que c'est un espace topologique complet.

\begin{lemma}[Propriétés de la valeur absolue]  \label{LemooANTJooYxQZDw}
    Soit \( \eK\) un corps totalement ordonné. Si \( x,y\in \eK\) alors
    \begin{enumerate}
        \item       \label{ItemooNVDIooSuiSoB}
            Si \( x\geq 0\) alors \( -x\leq 0\).
        \item
            \( | x |\geq 0\)
        \item
            \( | x |=0\) si et seulement si \( x=0\)
        \item\label{ItemooOMKNooRlanvk}
            \( | x+y |\leq | x |+| y |\).
    \end{enumerate}
\end{lemma}

\begin{proof}
    Point par point
    \begin{enumerate}
        \item
            Nous partons de \( x\geq 0\) et nous ajoutons \( -x\) des deux côtés en profitant de la définition d'un corps totalement ordonné : \( x-x\geq -x\) et donc \( 0\geq-x\), c'est à dire \( -x\leq 0\).
        \item
            Si \( x\geq 0\) alors c'est vrai. Sinon, \( x\leq 0\) et \( | x |=-x\geq 0\) par le point~\ref{ItemooNVDIooSuiSoB}.
        \item
            Si \( x=0\) alors \( x=-x\) et \( | x |=0\). Au contraire si \(x\neq 0\) alors \( -x\neq 0\) et que \( x\) soit positif ou négatif, nous aurons toujours \( \pm x\neq 0\).
        \item
            Nous supposons que \( x\leq y\) et nous distinguons divers cas suivant la positivité de \( x\) et \( y\).
            \begin{enumerate}
                \item
                    Si \( x,y\geq 0\). Dans ce cas, \( x+y\geq y\geq 0\), donc \( | x+y |=x+y=| x |+| y |\).
                \item
                    Si \( x,y\leq 0\). Dans ce cas, \( x+y\leq 0\) et nous avons \( | x+y |=-x-y=| x |+| y |\).
                \item
                    Si \( x\leq 0\) et \( y\geq 0\). Nous subdivisons encore en deux cas suivant que \( x+y\) est positif ou négatif. Si \( x+y\geq 0\), alors nous écrivons successivement
                    \begin{subequations}
                        \begin{align}
                            x&\leq 0\\
                            x+y&\leq y\leq y+| x |=| x |+| y |
                        \end{align}
                    \end{subequations}
                    et donc \( | x+y |=x+y\leq | x |+| y |\).

                    Nous supposons à présent que \( x\leq 0\), \( y\geq 0\) et \( x+y\leq 0\). Dans ce cas il suffit d'écrire \( | x+y |=| (-x)+(-y) |\) pour retomber dans le cas précédent à inversion près de \( x\) et \( y\).
            \end{enumerate}
    \end{enumerate}
\end{proof}

\begin{remark}      \label{RemooJCAUooKkuglX}
    La partie~\ref{ItemooOMKNooRlanvk} est très importante parce que c'est elle qui fera presque toutes les majorations dont nous aurons besoin en analyse. En effet elle donne l'inégalité triangulaire de la façon suivante : si \( x,y,z\in \eK\) nous avons
    \begin{equation}
        | x-y |= |  (x-z)+(z-y) |\leq | x-z |+| z-y |.
    \end{equation}
\end{remark}

\begin{lemma}[À propos de boules]
    Soient un corps totalement ordonné \( \eK\) et des éléments \( x,y,\epsilon\in \eK\).
    \begin{enumerate}
        \item       \label{ITEMooXJGVooSebiip}
            Nous avons \( y\in B(x,\epsilon)\) si et seulement si \( x-\epsilon<y<x+\epsilon\).
        \item       \label{ITEMooRUBBooRayiMs}
            Si \( y\in  \overline{ B(x,\epsilon) }  \) alors \( y\in B(x,\epsilon')\) pour tout \( \epsilon'<\epsilon\).
    \end{enumerate}
\end{lemma}

\begin{proof}
    Pour rappel,
    \begin{equation}
        | x-y |=\begin{cases}
               x-y    &     \text{si } x-y\geq 0 \\
                    y-x    &    \text{si } x-y\leq 0.
               \end{cases}
    \end{equation}
    Nous pouvons maintenant démontrer nos choses.
    \begin{subproof}
        \item[\ref{ITEMooXJGVooSebiip}]
            Des inégalités \( x-\epsilon<y\) et \( y<x+\epsilon\) nous tirons \( x-y<\epsilon\) et \( y-x<\epsilon\). Donc quel que soit le signe de \( x-y\) nous avons toujours \( | x-y |<\epsilon\).

            Dans l'autre sens, nous supposons que \( | x-y |<\epsilon\).

            Si \( x-y\geq 0\) alors l'hypothèse signifie \( x-y<\epsilon\), ce qui donne \( y>x-\epsilon\). Mais l'inégalité \( x-y\geq 0\) donne également \( x\geq y\) et donc \( x+\epsilon\geq y+\epsilon>y\). Notez le jeu de l'inégalité non stricte qui se change en inégalité stricte.

            Si \( x-y\leq 0\) nous pouvons faire le même raisonnement.

        \item[\ref{ITEMooRUBBooRayiMs}]

            C'est immédiat parce que
            \begin{equation}
                | x-y |\leq \epsilon<\epsilon'.
            \end{equation}
    \end{subproof}
\end{proof}

\begin{proposition}     \label{PROPooTFVOooFoSHPg}
    Toute suite convergente dans un corps totalement ordonné est de Cauchy.
\end{proposition}

\begin{proof}
    Soit un corps totalement ordonné \( \eK\) et une suite \( x_n\stackrel{\eK}{\longrightarrow}x\). Soit \( \epsilon>0\). Il est important de se rendre compte que \( \epsilon\in \eK\) et que l'inégalité est au sens de l'ordre dans \( \eK\); en particulier ce n'est pas \( \epsilon\in \eR\) ni \( \epsilon\in \eQ\). D'ailleurs nous n'avons encore pas défini ni \( \eR\) ni \( \eQ\).

    Vu que \( (x_n)\) converge vers \( x\), il existe \( N\in \eN\) tel que pour tout \( k>N\),
    \begin{equation}
        | x_k-x |<\epsilon.
    \end{equation}

    Soient \( p,q>N\). Alors en utilisant la majoration du lemme~\ref{LemooANTJooYxQZDw}\ref{ItemooOMKNooRlanvk},
    \begin{equation}
        | x_p-x_q |=\big| (x_p-x)+(x-x_q) \big|\leq | x_p-x |+| x-x_q |\leq 2\epsilon.
    \end{equation}
    Donc la suite \( (x_n)\) est de Cauchy.
\end{proof}

%+++++++++++++++++++++++++++++++++++++++++++++++++++++++++++++++++++++++++++++++++++++++++++++++++++++++++++++++++++++++++++
\section{Les rationnels}
%+++++++++++++++++++++++++++++++++++++++++++++++++++++++++++++++++++++++++++++++++++++++++++++++++++++++++++++++++++++++++++

Une construction très explicite est faite dans \cite{RWWJooJdjxEK}. Ici nous allons prendre plus court :
\begin{definition}
    Le corps des fractions de \( \eZ\) (définition~\ref{DEFooGJYXooOiJQvP}) est noté \( \eQ\) et ses éléments sont les \defe{rationnels}{rationnels}.
\end{definition}

Les résultats énoncés ici sont utilisés plus bas et servent de guide à un éventuel contributeur qui voudrait écrire une partie dédiée à \( \eQ\) et ses propriétés de base\quext{Par exemple, définir une relation d'ordre sur \( \eQ\) et expliciter l'inclusion de \( \eZ\) dans \( \eQ\).}. Nous espérons que des preuves se trouvent dans \cite{RWWJooJdjxEK}. En tout cas, le lecteur est invité à ne rien prendre comme évident.

\begin{lemma} \label{LEMooEBTIooGMoHsj}
    Tout rationnel est majoré par un naturel.
\end{lemma}

%---------------------------------------------------------------------------------------------------------------------------
\subsection{Suites de Cauchy dans les rationnels}
%---------------------------------------------------------------------------------------------------------------------------

\begin{proposition}[\cite{RWWJooJdjxEK}]        \label{PropFFDJooAapQlP}
    Principales propriétés des suites de Cauchy dans \( \eQ\).
    \begin{enumerate}
        \item       \label{ItemRKCIooJguHdji}
            Toute suite convergente est de Cauchy\footnote{Et non la réciproque, qui sera justement la grande innovation des nombres réels.}.
        \item       \label{ItemRKCIooJguHdjii}
            Toute suite de Cauchy est bornée.
        \item       \label{ItemRKCIooJguHdjiii}
            Si \( x_n\to 0\) et si \( (y_n)\) est bornée, alors \( x_ny_n\to 0\)
        \item
            Si \( (x_n)\) et \( (y_n)\) sont de Cauchy alors \( (x_n+y_n)\), \( (x_n-y_n)\) et \( (x_ny_n)\) sont également de Cauchy.
        \item       \label{ITEMooIAFSooAIUpAN}
            Si \( x_n\to a \) et \( y_n\to b \) alors \( x_n+y_n\to a+b\), \( x_n-y_n\to a-b\) et \(  x_ny_n\to ab  \).
        \item   \label{ItemRKCIooJguHdjvi}
            Soit \( (x_n)\) une suite de Cauchy qui ne converge pas vers zéro. Alors il existe \( n_0\) tel que la suite \( \left( \frac{1}{ x_n } \right)_{n\geq n_0}\) soit de Cauchy.
    \end{enumerate}
\end{proposition}

\begin{proof}
    Point par point.
    \begin{enumerate}
        \item

            C'est la proposition~\ref{PROPooTFVOooFoSHPg}.

        \item
            Soit \( (x_n)\) une suite de Cauchy dans \( \eQ\). Avec \( \epsilon=1\) dans la définition, si \( q>N_1\), nous avons
            \begin{equation}
                | x_q-x_{N_1} |\leq 1.
            \end{equation}
            Et donc pour tout \( q\) plus grand que \( N_1\), \( x_N-1\leq x_q\leq x_N+1\), ou encore, pour tout \( n\) :
            \begin{equation}
                | x_n |\leq\max\{ | x_1 |,| x_2 |,\ldots,| x_N |,| x_N+1 | \}.
            \end{equation}
            La suite est donc bornée.
        \item
            Soit \(\epsilon>0\). Les hypothèses disent qu'il existe un \( N\) tel que \( | x_n |\leq \epsilon\) dès que \( n\geq N\). Et il existe aussi \( M\geq 0\) tel que \( | y_n |\leq M\) pour tout \( n\). Du coup, lorsque \( n\geq N\) nous avons \( | x_ny_n |\leq M\epsilon\).
        \item
            En ce qui concerne la somme,
            \begin{equation}        \label{EqDCNBooAzrrBi}
                | x_p+y_p-x_q-y_q |\leq | x_p-x_q |+| y_p-y_q |.
            \end{equation}
            Soit \( N_1\) tel que si \( p,q\geq N_1\) alors \( | x_p-x_q |\leq \epsilon\) et \( N_2\) de même pour la suite \( (y_n)\). En prenant \( N=\max\{ N_1,N_2 \}\), la somme \eqref{EqDCNBooAzrrBi} est plus petite que \( 2\epsilon\) dès que \( p,q\geq N\).

            Passons à la démonstration du fait que le produit de deux suites de Cauchy est de Cauchy. Les suites \( (x_n)\) et \( (y_n)\) sont bornées et quitte à prendre le maximum, nous disons qu'elles sont toutes les deux bornées par le nombre \( M\) : pour tout \( n\) nous avons \( | x_n |\leq M\) et \( | y_n |\leq M\). Nous avons :
            \begin{equation}
                | x_py_p-x_qy_q |\leq | x_py_p-x_qy_p |+| x_qy_p-x_qy_q |\leq | y_p | |x_p-x_q |+| x_q | |y_p-y_q |.
            \end{equation}
            Vu que \( (x_n)\) et \( (y_n)\) sont de Cauchy, si \( p\) et \( q\) sont assez grands, les deux différences sont majorées par \( \epsilon\) et nous avons
            \begin{equation}
                | x_py_p-x_qy_q |\leq M\epsilon+M\epsilon=2M\epsilon,
            \end{equation}
            ce qui prouve que \( (x_ny_n)\) est de Cauchy.
        \item
            En ce qui concerne la somme, nous pouvons tout de suite calculer
            \begin{equation}
                | x_n+y_n-(a+b) |\leq | x_n-a |+| y_n-b |.
            \end{equation}
            Il existe une valeur de \( n\) à partir de laquelle le premier terme est plus petit que \( \epsilon\) et une à partir de laquelle le second terme est plus petit que \( \epsilon\). En prenant le maximum des deux, la somme est plus petite que \( 2\epsilon\).

            En ce qui concerne le produit,
            \begin{equation}
                | x_ny_n-ab |\leq | x_ny_n-ay_n |+| ay_n-ab |\leq | y_n || x_n-a |+| a || y_n-b |.
            \end{equation}
            Les suites \( | x_n-a |\) et \( | y_n-b |\) convergent vers zéro; la suite \( (y_n)\) est bornée parce que convergente (combinaison des points~\ref{ItemRKCIooJguHdji} et~\ref{ItemRKCIooJguHdjii})  et \( a\) (la suite constante) est également bornée. Donc par le point~\ref{ItemRKCIooJguHdjiii}, nous avons
            \begin{equation}
                y_n| x_n-a |+a| y_n-b |\to 0.
            \end{equation}
            Au passage nous avons également utilisé la propriété de la somme que nous venons de démontrer.
        \item Soit \( (x_n)\) une suite de Cauchy dans \( \eQ\) ne convergeant pas vers zéro : il existe \( \alpha>0\) tel que pour tout \( N\in \eN\), il existe \( n\geq N\) tel que \( | x_n |>\alpha\). Mais notre suite est de Cauchy, donc il existe \( n_0\in \eN\) tel que si \( p,q\geq n_0\) alors
            \begin{equation}
                | x_p-x_q |\leq \frac{ \alpha }{2}.
            \end{equation}
            En fixant \( N = n_0\), on obtient un naturel \( n\geq n_0\) tel que \( | x_n |\geq \alpha\). De plus, comme la suite est de Cauchy, si \( p>n\) nous avons aussi \( | x_n-x_p |\leq \frac{ \alpha }{2}\). Cela implique \( | x_p |\geq \frac{ \alpha }{2}\) et en particulier \( x_p\neq 0\).

            Nous venons de prouver que la suite ne s'annule plus à partir de l'indice \( n\), et même que \( | x_k |\geq\alpha/2\) pour tout \( k\geq n\). La suite \( (1/x_k)_{k\geq n}\) est donc bien définie.

            Soit \( \epsilon>0\). Soit \( n_0\) tel que \( | x_p-x_q |<\epsilon\) pour tout \( p,q>n_0\). Soit \( K\) plus grand que \( n_0\) et que \( n\). En prenant \( p,q\geq K\), nous avons \( |  x_p|>\frac{ \alpha }{2}\) et \( | x_q |>\frac{ \alpha }{2}\). Nous en déduisons que
            \begin{equation}
                \left| \frac{1}{ x_p }-\frac{1}{ x_q } \right| \leq \frac{ | x_q-x_p | }{ | x_px_q | }\leq \frac{ 4 }{ \alpha^2 }| x_q-x_p |\leq \frac{ 4 }{ \alpha^2 }\epsilon.
            \end{equation}
            Donc \( \left( \frac{1}{ x_n } \right)\) est de Cauchy.
    \end{enumerate}
\end{proof}

%---------------------------------------------------------------------------------------------------------------------------
\subsection{Insuffisance des rationnels}
%---------------------------------------------------------------------------------------------------------------------------

Nous allons voir qu'il n'existe pas de nombres rationnels \( x\) tels que \( x^2=2\), mais que pourtant il existe une infinité de suites de rationnels \( (x_n)\) tels que \(  x_n^2\to 2  \).

\begin{lemma}       \label{LemJPIUooWFHaFM}
    Un entier \( x\) est pair si et seulement si l'entier \( x^2\) est pair.
\end{lemma}

\begin{proof}
    Si \( x\) est un nombre pair, alors il existe un entier \( a\) tel que \( x=2a\) alors \( x^2=4a^2\) est pair.

    Inversement, si \( x\) est impair alors il existe un entier \( a\) tel que \( x=2a+1\) et alors \( x^2=4a^2+4a+1=2(2a^2+2a)+1\) est impair.
\end{proof}

\begin{proposition}[Irrationalité de \( \sqrt{2}\)]     \label{PropooRJMSooPrdeJb}
    Il n'existe pas de fractions d'entiers dont le carré soit égal à \( 2\).
\end{proposition}
\index{irrationalité!\( \sqrt{2}\)}
Le théorème~\ref{THOooYXJIooWcbnbm} nous dira que tous les \( \sqrt{n}\) sont irrationnels dès que \( n\) n'est pas un carré parfait.

\begin{proof}
    Nous supposons que la fraction d'entiers \( a/b\) est telle que \( a^2/b^2=2\), et nous allons construire une suite d'entiers strictement décroissante et strictement positive, ce qui est impossible.

    Grâce au lemme~\ref{LemJPIUooWFHaFM} nous avons successivement les affirmations suivantes :
    \begin{itemize}
        \item
        \(\frac{ a^2 }{ b^2 }=2 \)  avec \( a\neq 0\) et \( b\neq 0\).
    \item
        \( a^2=2b^2\), donc \( a^2\) est pair.
    \item
        \( a\) est alors pair et \( a^2\) est divisible par \( 4\). Soit \( a^2=4k\).
    \item
        \( 4k/b^2=2\), donc \( 4k=2b^2\), donc \( b^2=2k\) et \( b^2\) est pair.
    \item
        Nous déduisons que \( b\) est pair.
    \end{itemize}
    La fraction \( \frac{ a/2 }{ b/2 }\) est alors une nouvelle fraction d'entiers dont le carré vaut $2$. En procédant de la même façon, en remplaçant \( a\) par \( a/2\) et \( b\) par \( b/2\), on obtient que la fraction d'entiers \( \frac{ a/4 }{ b/4 }\) a la même propriété.

    En particulier, tous les nombres de la forme \( a/2^n\) sont des entiers.  Ils forment une suite strictement décroissante d'entiers strictement positifs. Impossible, me diriez vous ? Et vous auriez bien raison : toute partie non vide de \( \eN\) admet un plus petit élément\footnote{Voir \cite{RWWJooJdjxEK}, et attention : ce n'est pas tout à fait évident.}. Il n'y a donc pas de fractions d'entiers dont le carré vaut \( 2\).
\end{proof}

\begin{proposition}[\cite{MonCerveau}]      \label{PROPooAHMIooVpunrF}
    Soit la suite de rationnels \( (x_n)\) définie par \( x_0\in \eQ^+\) et
    \begin{equation}
        x_{n+1}=x_n+\frac{ x_n^2-2 }{ 2x_n }.
    \end{equation}
    Alors :
    \begin{enumerate}
        \item
            En posant \( y_n=x_n^2\) nous avons \( y_n\to 2\).
        \item
            La suite \( (x_n)\) est de Cauchy.
        \item
            La suite \( (x_n)\) ne converge pas dans \( \eQ\).
    \end{enumerate}
\end{proposition}

\begin{proof}
    Tout d'abord un petit calcul montre que
    \begin{equation}
        x_{n+1}^2=2+\frac{ (y_n-2)^2 }{ 4y_n },
    \end{equation}
    c'est à dire que quelle que soit la valeur de \( x_0\), dès le \( x_1\), les valeurs de \( y_n\) sont plus grandes que \( 2\). Notons qu'il n'est pas possible d'avoir \( y_n=2\). Nous pouvons donc supposer \( y_n>2\) pour tout \( n\). Alors en posant \( y_n=2+s\) nous avons
    \begin{equation}
        y_{n+1}=2+\frac{ s^2 }{ 8+4s }.
    \end{equation}
    Si \( s<1\) alors \( \frac{ s^2 }{ 8+4s }<s^2\) et en réalité le processus \( s\mapsto s^2/(8+4s)\) tend très vite vers zéro. Nous devons donc montrer qu'il existe un \( n\) tel que \( y_n=2+s\) avec \( s<1\).

    Montrons pour cela que si \( n<s<2n\) alors\footnote{Voir la remarque~\ref{RemUZCAooWNogzI} pour comprendre d'où vient l'idée de cette majoration.}
    \begin{equation}\label{EqYNKQooUBfhgz}
        \frac{ s^2 }{ 8+4s }<n
    \end{equation}
    En effet si \( n<s<2n\) nous pouvons majorer le numérateur par \( 4n^2\) et minorer le dénominateur par \( 4n\).

    Prouvons à présent le résultat. Pour un \( n>1\) nous avons
    \begin{equation}
        y_{n+1}=2+\frac{ s }{ 8+4s }
    \end{equation}
    avec \( s>0\). Nous considérons \( n_0\in \eN\) tel que \( n_0<s<2n_0\). Alors
    \begin{equation}
        y_{n+2}=2+s_1
    \end{equation}
    avec \( s_1<n\). Il existe donc \( n_1\) tel que \( n_1<s_1<2n_1\) et \( n_1<n_0\). Nous avons alors \( y_{n+3}=2+s_2\) avec \( s_2<n_1<n_0\).

    Nous construisons ainsi des suites \( (s_i)\) et \( (n_i)\) telles que
    \begin{equation}
        y_{n+k}=2+s_k
    \end{equation}
    avec \( s_k<n_{k-1}<n_{k-2}<\cdots<n_0\). En procédant ainsi au maximum \( n-1\) fois nous avons \( s_k<1\). À partir du moment où \( y_{n+k}=2+s_k\) avec \( s_k<1\), nous avons déjà vu qu'il est certain que \( y_n\to 2\).

    Prouvons que la suite \( (x_n)\) est de Cauchy. Vu que \( x_n^2\to 2\) nous avons \( x_n\geq 1\) à partir d'une certaine valeur de \( n\). Soit \( \epsilon > 0 \); comme \( (x_n^2)\) converge, par la proposition~\ref{PropFFDJooAapQlP}\ref{ItemRKCIooJguHdji}, elle est de Cauchy, et il existe \( n_0\) tel que \( | x_p^2-x_q^2 | < \epsilon \) pour \( p,q\geq n_0\). Alors, pour \( p,q \) plus grands que \( n_0\) et \( n\), nous avons
    \begin{equation}
        |x_p-x_q | = \frac{| x_p^2-x_q^2 |}{| x_p+x_q |} \leq \frac \epsilon 2.
    \end{equation}
    Cela prouve que la suite \( (x_n)\) est de Cauchy dans \( \eQ\).

    Enfin supposons que \( x_n\to a\in \eQ\). Dans ce cas nous aurions \( x_n^2\to a^2=2\) (proposition~\ref{PropFFDJooAapQlP}\ref{ITEMooIAFSooAIUpAN}). Mais nous savons par la proposition~\ref{PropooRJMSooPrdeJb} que \( a^2=2\) est impossible dans \( \eQ\).
\end{proof}

Bref, la suite \( (x_n)\) est de Cauchy, ne converge pas et a son carré qui converge vers \( 2\).

\begin{remark}\label{RemUZCAooWNogzI}
    Vu que nous n'avons pas encore défini les réels, ce qui suit n'est que informel. Ce qui a motivé la majoration \eqref{EqYNKQooUBfhgz} est la résolution de l'inéquation
    \begin{equation}
        \frac{ s^2 }{ 8+4s }<n
    \end{equation}
    qui donne les bornes \( 2n\pm\sqrt{n^2+2n}\) dont une est toujours négative, et l'autre plus grande que \( 2n\).
\end{remark}

Vous en voulez encore une ?

\begin{lemma}[Série géométrique, voir aussi l'exemple~\ref{ExZMhWtJS}]      \label{LEMooOTVUooImvusn}
    Si \( q\in \eQ\) et \( p\in \eN\) nous avons
    \begin{equation}
        \sum_{k=0}^pq^k=\frac{ 1-q^{p+1} }{ 1-q }.
    \end{equation}
\end{lemma}

\begin{proof}
    En posant \( S_p=1+q+q^2+\cdots +q^{p}\), nous avons $S_p-qS_p=1-q^{p+1}$ et donc
    \begin{equation}
        S_p=\sum_{k=0}^pq^k=\frac{ 1-q^{p+1} }{ 1-q }.
    \end{equation}
\end{proof}

\begin{proposition}
    La suite donnée par
    \begin{equation}
        x_n=1+\frac{ 1 }{ 1! }+\cdots +\frac{1}{ n! }
    \end{equation}
    est de Cauchy et ne converge pas dans \( \eQ\).
\end{proposition}

\begin{proof}
    Si \( p>q>0\) nous avons
    \begin{subequations}
        \begin{align}
            x_p-x_q&=\sum_{k=q+1}^p\frac{1}{ k! }\\
            &\leq \sum_{k=q+1}^p\frac{1}{ (q+1)! }\frac{1}{ (q+1)^{k-q-1} }  \label{SUBEQooAXILooEAcpVB}\\
            &\leq \frac{1}{ (q+1)! }\lim_{p\to \infty} \sum_{k=0}^{p}\frac{1}{ (q+1)^k }  \label{SUBEQooNDPTooDSEYEJ}\\
            &=\frac{1}{ (q+1)! }\frac{1}{ 1-\frac{1}{ q+1 } } \label{SUBEQooEMHJooSnCUiK}  \\
            &=\frac{1}{ (q+1)! }\frac{q+1}{q}\\
            &=\frac{1}{ q!q }.
        \end{align}
    \end{subequations}
    Justifications :
    \begin{itemize}
        \item Pour \eqref{SUBEQooAXILooEAcpVB}, il s'agit de remplacer dans \( k!\) tous les facteurs plus grands que \( (q+1)\) par \( q+1\). Cela rend le dénominateur plus petit.
        \item Pour \eqref{SUBEQooNDPTooDSEYEJ}, il y a une inégalité parce que la suite \( p\mapsto \sum_{k=0}^p1/(q+1)^k\) est une suite strictement croissante.

        \item Pour \eqref{SUBEQooEMHJooSnCUiK}, le lemme~\ref{LEMooOTVUooImvusn} donne la valeur de la somme finie. En ce qui concerne la limite, nous avons demandé \( p>q>0\) et donc \( q+1>1\). Dans ce cas la limite fonctionne.
    \end{itemize}

    Cette inégalité une fois établie nous permet de prouver les assertions. La suite \( (x_n) \) est de Cauchy car, pour tout \( \epsilon\in\eQ\) s'écrivant \( \epsilon=\frac{ a }{ b }\) avec \( a,b\in \eN\), en prenant \( p,q>b\), nous avons
    \begin{equation}
        x_p-x_q\leq \frac{1}{ b!b }<\frac{1}{ b }<\frac{ a }{ b }=\epsilon.
    \end{equation}

    Montrons par l'absurde que cette suite ne converge pas dans \( \eQ\). Pour cela, nous supposons que \( \lim_{n\to \infty} x_n=\frac{ a }{b }\in \eQ\). Pour tout \( p>q\) nous avons établi
    \begin{equation}
        0<x_p-x_q<\frac{1}{ qq! }.
    \end{equation}
    Prenons la limite \( p\to \infty\); par stricte croissance de la suite, les inégalités restent strictes :
    \begin{equation}        \label{EqQLCTooOgQOdh}
        0<\frac{ a }{ b }-x_q<\frac{1}{ qq! }.
    \end{equation}
    Si \( n>b\) alors nous pouvons écrire
    \begin{equation}
        \frac{ a }{ b }-x_n=\frac{ \alpha }{ n! }
    \end{equation}
    avec \( \alpha\in \eZ\) parce que le dénominateur commun entre \( \frac{ a }{ b }\) et \( x_n\) est dans \( n!\). En prenant donc \( q>n\) dans \eqref{EqQLCTooOgQOdh} nous pouvons écrire
    \begin{equation}
        0<\frac{ \alpha }{ q! }<\frac{1}{ qq! },
    \end{equation}
    c'est à dire \( 0<\alpha<\frac{1}{ q }\), ce qui est impossible pour \( \alpha\in \eZ\).
\end{proof}

%+++++++++++++++++++++++++++++++++++++++++++++++++++++++++++++++++++++++++++++++++++++++++++++++++++++++++++++++++++++++++++
\section{Les réels}
%+++++++++++++++++++++++++++++++++++++++++++++++++++++++++++++++++++++++++++++++++++++++++++++++++++++++++++++++++++++++++++

Une construction des réels via les coupures de Dedekind est donnée dans \cite{PaulinTopGmVegN}.

\begin{normaltext}      \label{NormooHRDZooRGGtCd}
    La construction des réels va nécessiter un petit «\wikipedia{fr}{bootstrap}{bootstrap}» au niveau de la topologie. En effet la notion de suite de Cauchy est une notion topologique (définition~\ref{DefZSnlbPc}) alors que la topologie métrique (celle entre autres de \( \eQ\)) ne sera définie que par le théorème~\ref{ThoORdLYUu}. Nous avons donc dû définir en la définition~\ref{DefKCGBooLRNdJf} \emph{ex nihilo} les notions de
\begin{itemize}
    \item
        suite de Cauchy
    \item
        suite convergente
    \item
        complétude
\end{itemize}
Nous allons ensuite construire \( \eR\) comme ensemble de classes d'équivalence de suites de Cauchy dans \( \eQ\). Ce ne sera que plus tard, après avoir défini la notion d'espace métrique que nous allons voir que sur \( \eR\), ces trois notions coïncident avec celles topologiques\footnote{Proposition~\ref{PropooUEEOooLeIImr}.}. Et par conséquent que \( \eR\) sera un espace métrique complet\footnote{Théorème~\ref{THOooUFVJooYJlieh} pour la complétude en tant que corps et théorème~\ref{PROPooTFVOooFoSHPg} pour la complétude en tant que espace métrique.}.
% position 11144-30436
% position 13984-18006

Dans cette optique, il est intéressant de lire ce que dit Wikipédia à propos des suites de Cauchy dans \wikipedia{fr}{Construction_des_nombres_réels}{l'article consacré} à la construction des nombres réels.
\end{normaltext}

%---------------------------------------------------------------------------------------------------------------------------
\subsection{L'ensemble}
%---------------------------------------------------------------------------------------------------------------------------

Soit \( \modE\) l'ensemble des suites de Cauchy\footnote{Définition~\ref{DefKCGBooLRNdJf}\ref{ItemVXOZooTYpcYN}} dans \( \eQ\). Soit aussi l'ensemble \( \modE_0\) constituée des suites qui convergent vers zéro\footnote{Nous rappelons qu'à ce niveau nous n'avons pas encore prouvé que toutes les suites de Cauchy convergent.}.

En posant
\begin{equation}
    x+y=(x_n+y_n)
\end{equation}
et
\begin{equation}
    xy=(x_ny_n),
\end{equation}
l'ensemble \( \modE\) devient un anneau\footnote{Définition~\ref{DefHXJUooKoovob}.} commutatif dont le neutre de l'addition est la suite constante \( x_n=0\) et le neutre pour la multiplication est la suite constante \( x_n=1\).

\begin{proposition}
    La partie \( \modE_0\) est un idéal\footnote{Définition~\ref{DefooQULAooREUIU}.} de l'anneau \( \modE\).
\end{proposition}

\begin{proof}
    Nous savons par la proposition~\ref{PropFFDJooAapQlP}\ref{ItemRKCIooJguHdji} que les suites convergentes sont de Cauchy; par conséquent \( \modE_0\subset\modE\).

    L'ensemble structuré \( (\modE_0,+)\) est un sous-groupe de \( \modE\) par les propriétés de la proposition~\ref{PropFFDJooAapQlP} (il s'agit du fait que la somme de deux suites convergent vers zéro est une suite convergente vers zéro).

    En ce qui concerne la propriété fondamentale des idéaux, si \( x\in\modE_0\) et \( y\in\modE\) nous devons prouver que \( xy\in \modE_0\). Vu que \( (\modE_0,\cdot)\) est commutatif, cela suffira pour être un idéal bilatère. Vu que \( y\) est une suite de Cauchy, elle est bornée; et étant donné que \( x\to 0\) nous avons alors \( xy\to 0\) (par la proposition~\ref{PropFFDJooAapQlP}\ref{ItemRKCIooJguHdjiii}).
\end{proof}

\begin{theoremDef}[L'anneau des réels\cite{RWWJooJdjxEK}]       \label{DefooFKYKooOngSCB}
    Sur l'ensemble quotient \( \modE/\modE_0\), les opérations
    \begin{enumerate}
        \item
            \( \bar u+\bar v=\overline{ u+v }\)
        \item
            \( \bar u\cdot \bar v=\overline{ uv }\)
    \end{enumerate}
    sont bien définies et donnent à \( \modE/\modE_0\) une structure de corps commutatif appelé \defe{corps des réels}{réel} et noté \( \eR\)\nomenclature[Y]{\( \eR\)}{l'ensemble des réels}
\end{theoremDef}
\index{construction!des réels}

\begin{proof}
    Nous divisons la preuve en plusieurs parties.
    \begin{subproof}
    \item[Les opérations sont bien définies]
    Si \( u,v\in \modE\) et \( h,k\in \modE_0\) alors \( h+k\in\modE_0\) et nous avons
    \begin{equation}
        \overline{ u+h }+\overline{ v+k }=\overline{ u+v+h+k }=\overline{ u+v }.
    \end{equation}
    ainsi que
    \begin{equation}
        \overline{ u+h }\cdot \overline{ v+k }=\overline{ (u+h)(v+k) }=\overline{ uv+uk+hv+hk }=\overline{ uv }
    \end{equation}
    parce que les suites des Cauchy \( u\) et \( v\) sont bornées, ce qui donne que \( uk\), \( hv\) et \( hk\) sont des éléments de \( \modE_0\) par la proposition~\ref{PropFFDJooAapQlP}\ref{ItemRKCIooJguHdjiii}. Cela prouve que les définitions proposées sont bonnes.

    \item[Caractérisation des classes]
        Soit \( q\in \eQ\) et une suite \( x\) convergente vers \( q\). Cette suite est de Cauchy comme toute suite convergente. Montrons que
        \begin{equation}
            \bar x=\{ \text{suites qui convergent vers } q \}.
        \end{equation}
        Si \( y\in\bar x\) alors \( y=x+h\) avec \( h\in \modE_0\), et comme \( h_n\to 0\), on a \( y_n\to q\). Réciproquement, si \( y_n\to q\) alors pour chaque \( n\) nous avons
        \begin{equation}
            y_n=x_n+(y_n-x_n),
        \end{equation}
        mais \( y_n-x_n\to 0\). Donc la suite \( y-x\in\modE_0\) ce qui signifie que \( y\in\bar x\).
    \item[Neutre et unité]
        Il est vite vérifié que \( \bar 0\), la classe de la suite constante égale à zéro est neutre pour l'addition. De même, \( \bar 1\), est un neutre pour la multiplication.
    \item[Corps]
        Nous devons prouver que tout élément non nul est inversible. C'est à dire que si \( x\in\modE\) ne converge pas vers zéro\footnote{\( x\in\modE\) peut soit ne pas converger du tout, soit converger vers autre chose que zéro.} alors il existe \( y\in \modE\) tel que \( xy\in\bar 1\).

        Nous savons par la proposition~\ref{PropFFDJooAapQlP}\ref{ItemRKCIooJguHdjvi} que \( x\) étant une suite de Cauchy dans \( \eQ\), il existe \( n_0\in \eN\) tel que \( \left( \frac{1}{ x_n } \right)_{n\geq n_0}\) est une suite de Cauchy. Nous posons alors
        \begin{equation}
            y_n=\begin{cases}
                0    &   \text{si } n\leq n_0\\
                \frac{1}{ x_n }    &    \text{si } n>n_0.
            \end{cases}
        \end{equation}
        Nous avons alors
        \begin{equation}
            (xy)_n=\begin{cases}
                0    &   \text{si } n\leq n_0\\
                1    &    \text{si } n>n_0
            \end{cases}
        \end{equation}
        et donc \( xy\in\bar 1\).
    \end{subproof}
\end{proof}

\begin{proposition}     \label{PropooEPFCooMtDOfP}
    Soit l'application
    \begin{equation}
        \begin{aligned}
            \varphi\colon \eQ&\to \eR \\
            q&\mapsto \bar q .
        \end{aligned}
    \end{equation}
    où par \( \bar q\) nous entendons la classe de la suite constante égale à \( q\) (qui est de Cauchy).
    \begin{enumerate}
        \item
            C'est un homomorphisme injectif.
        \item
            \( \Image(\varphi)\) est un sous-corps de \( \eR\)
        \item
            \( \varphi\colon \eQ\to \Image(\varphi)\) est un isomorphisme de corps.
    \end{enumerate}
\end{proposition}

\begin{proof}
    Le fait que ce soit un homomorphisme est simplement
    \begin{itemize}
        \item \( \varphi(q+q')=\overline{ q+q' }=\bar q+\overline{ q' }\)
        \item \( \varphi(qq')=\overline{ qq' }=\overline{ q }\overline{ q' }\).
    \end{itemize}
    En ce qui concerne l'injectivité, si \( q\) est tel que \( \varphi(q)=\bar 0=\modE_0\), c'est que
    \begin{equation}
        \varphi(q)=\{ \text{suites de Cauchy qui convergent vers zéro} \}
    \end{equation}
    Mais nous savons aussi que\footnote{Voir dans la démonstration du théorème~\ref{DefooFKYKooOngSCB}.}
    \begin{equation}
        \varphi(q)=\bar q=\{ \text{suites de Cauchy qui convergent vers } q \}
    \end{equation}
    Nous en déduisons que \( q=0\).
\end{proof}
Lorsque dans la suite nous parlerons d'un élément de \( \eQ\) comme étant un réel, nous aurons en tête l'image de cet élément par \( \varphi\).

%---------------------------------------------------------------------------------------------------------------------------
\subsection{Relation d'ordre}
%---------------------------------------------------------------------------------------------------------------------------

Nous définissons les parties \( \modE^+\) et \( \modE^-\) de \( \modE\) par
\begin{enumerate}
    \item
        \( x\in  \modE^+\) si et seulement si pour tout \( \epsilon>0\), il existe \( N_{\epsilon}\) tel que \( n>N_{\epsilon}\) implique \( x_n>-\epsilon\).
    \item
        \( x\in  \modE^-\) si et seulement si pour tout \( \epsilon>0\), il existe \( N_{\epsilon}\) tel que \( n>N_{\epsilon}\) implique \( x_n<\epsilon\).
\end{enumerate}
Nous notons aussi \( \modE^{++}=\modE^+\setminus\modE_0\).

\begin{lemma}
    Les parties \( \modE^+\) et \( \modE^-\) partitionnent \( \modE\) de la façon suivante :
    \begin{enumerate}
        \item
            \( \modE^+\cap\modE^-=\modE_0\)
        \item
            \( \modE^+\cup\modE^-=\modE\)
    \end{enumerate}
\end{lemma}

\begin{proof}
    On prouve d'abord que \( \modE^+\cap\modE^-\subset\modE_0\), l'inclusion inverse est évidente. Soit \( \epsilon>0\) et \( x\in \modE^+\cap\modE^-\). Il existe un \( N\in \eN\) tel que \( x_n>-\epsilon\) et \( x_n<\epsilon\) pour tout \( n\geq N\). Par conséquent, \( | x_n |\leq \epsilon\) pour tout \( n\geq N\) et la suite \( x\) converge vers zéro, c'est à dire \( x\in\modE_0\).

    Pour prouver le second point, soit \( x\in \modE\setminus\modE^-\), et prouvons que \( x\in\modE^+\). La condition \( x\notin \modE^-\) donne qu'il existe un \( \alpha>0\) (dans \( \eQ\)) tel que pour tout \( n\), il existe \( p>n\) avec \( x_p>\alpha\). Mais \( x\) est une suite de Cauchy, donc nous avons un \( n_0\) tel que si \( n,p\geq n_0\) alors \( | x_n-x_p |\leq \frac{ \alpha }{2}\). En particulier, si \( n\geq n_0\), et si \( p>n\) est tel que \( x_p>\alpha\), on obtient
    \begin{equation}
        x_n>\frac{ \alpha }{2}>0
    \end{equation}
    Par conséquent \( x\in\modE^+\) parce que \( x\in\modE\) et les \( x_n \) sont tous positifs à partir d'un certain rang.
\end{proof}

\begin{lemma}[\cite{RWWJooJdjxEK}]
    Quelques propriétés du partitionnement.
    \begin{enumerate}
        \item
            \( x\in\modE^-\) si et seulement si \( (-x)\in\modE^+\)
        \item
            \( x\in\modE^+\) et \( y\in\modE^+\) implique \( x+y\in\modE^+\)
        \item
            \( x\in\modE^+\) et \( y\in\modE^+\) implique \( xy\in\modE^+\)
        \item
            Si \( x,y\in\modE\) sont tels que \( x-y\in\modE_0\) alors soit \( x,y\in\modE^+\) soit \( x,y\in\modE^-\).
    \end{enumerate}
\end{lemma}

\begin{proof}
    Point par point.
    \begin{enumerate}
        \item
            Définition de \( \modE^+\) et \( \modE^-\).
        \item
            Pour \( n\geq N_{\epsilon/2}\) nous avons \( x_n>-\epsilon/2\) et \( y_n>-\epsilon/2\). Donc \( x_n+y_n>-\epsilon\).
        \item
            Si \( x\) ou \( y\) est dans \( \modE_0\) alors \( xy\in\modE_0\) et c'est bon. Si par contre \( x,y\in\modE^{++}\) alors pour \( n\) suffisamment grand, \( x_n>0\) et \( y_n>0\). Et dans ce cas, \( (xy)_n> 0\), c'est à dire \( xy\in\modE^+\).
        \item
            Supposons que \( x-y\in\modE_0\) avec \( x\in\modE^+\) et prouvons qu'alors \( y\in\modE^+\). Soit donc \( \epsilon>0\); il existe \( n_1\) tel que \( x_n>-\frac{ \epsilon }{2}\) dès que \( n\geq n_1\). Mais \( x-y\in\modE_0\), donc il existe \( n_2\) tel que \( | x_n-y_n |<\frac{ \epsilon }{2}\) dès que \( n\geq n_2\). En prenant \( n\) plus grand que \( n_1\) et \( n_2\), nous avons en même temps
            \begin{subequations}
                \begin{numcases}{}
                    x_n>-\frac{ \epsilon }{2}\\
                    | x_n-y_n |<\frac{ \epsilon }{2}.
                \end{numcases}
            \end{subequations}
            Cela implique que \( y_n>-\epsilon\) et donc que \( y\in\modE^+\).

            Nous pouvons de même prouver que si \( x\in\modE^-\) alors \( y\in\modE^-\).
    \end{enumerate}
\end{proof}

\begin{definition}[Positivité dans \( \eR\)]        \label{DefooLMQIooTgzZXd}
    Vocabulaire et notations.
    \begin{enumerate}
        \item
            Nous notons \( \eR=\modE/\modE_0\).
        \item
            Nous notons \( \eR^+=\modE^+\).\nomenclature[Y]{\( \eR^+\)}{les réels positifs ou nuls}
        \item
            Nous notons \( \eR^-=\modE^-\).
        \item
            Un élément de \( \eR\) est \defe{positif}{positif} s'il est la classe d'une suite de Cauchy appartenant à \( \modE^+\).
        \item
            Un élément de \( \eR\) est \defe{négatif}{négatif} s'il est la classe d'une suite de Cauchy appartenant à \( \modE^-\).
        \item
            Lorsque nous parlons de nombres réels, le symbole «\( 0\)» signifie \( \modE_0\) ou plus précisément la classe d'un élément de \( \modE_0\) modulo \( \modE_0\).
    \end{enumerate}
\end{definition}

\begin{normaltext}\label{REMooOCXLooKQrDoq}
    Avec les conventions de la définition~\ref{DefooLMQIooTgzZXd}, et en anticipant sur nos connaissances à propos des réels,
    \begin{enumerate}
        \item
            zéro est positif et négatif.
        \item
            L'intersection entre \( \eR^+\) et \( \eR^-\) est le singleton \( \{ 0 \}\).
        \item
            L'ensemble des nombres \emph{strictement} positifs est noté \( (\eR^+)^*\) ou \( \eR^+\setminus\{ 0 \}\).
        \item
            Le mot «positif» signifie «positif ou nul»; le mot «négatif» signifie «négatif ou nul».
    \end{enumerate}

    Cela vient des conventions de la remarque \ref{REMooOCXLooKQrDoq} qui sont également celles de Wikipédia\cite{ooSBSSooTlnuKi}.
\end{normaltext}

\begin{definition}[Ordre dans \( \eR\)]     \label{DefooYALBooHSXZqB}
    Si \( a,b\in \eR\) nous notons \( a\leq b\) si et seulement si \( b-a\) est positif. Nous notons aussi \( a>b\) si et seulement si \( b-a\in \eR^+\setminus\{ 0 \}\), etc.
\end{definition}

\begin{lemma}       \label{LemooRordonne}
    Les premières propriétés de l'ordre.
    \begin{enumerate}
        \item
            L'ensemble \( (\eR,\leq)\) est un corps totalement ordonné (définitions~\ref{DEFooVGYQooUhUZGr} et~\ref{DefKCGBooLRNdJf}).
        \item
            L'application
            \begin{equation}
                \begin{aligned}
                    \varphi\colon \eQ&\to \eR \\
                    q&\mapsto \bar q
                \end{aligned}
            \end{equation}
            dont nous avons déjà parlé dans la proposition~\ref{PropooEPFCooMtDOfP} est strictement croissante.
    \end{enumerate}
\end{lemma}

\begin{proof}
    Nous prouvons la stricte croissance de \( \varphi\). Si \( q< l\) alors \( \varphi(q)-\varphi(l)=\overline{ q-l }\) est la classe de la suite constante \( q-l\) qui est un élément strictement positif de \( \eQ\). Nous avons donc \( \overline{ q-l }\in \eR^+\), et donc \( \varphi(q)<\varphi(l)\).
\end{proof}

\begin{remark}
    Comme déjà mentionné plus haut, à chaque fois que nous parlerons d'un élément de \( \eQ\) comme étant un élément de \( \eR\), nous considérons la classe de la suite constante.
\end{remark}

\begin{lemma}       \label{LemooYNOVooOwoRwD}
    Si \( x,y,z\in \eR\) avec \( x>0\) sont tels que \( z>y/x\) alors \( zx>y\).
\end{lemma}

\begin{proof}
    Nous savons que
    \begin{equation}
        z-\frac{ y }{ x }\in \modE^+\setminus\{ 0 \}=\modE^{++}.
    \end{equation}
    Vu que \( x\in\modE^{++}\), multiplier par \( x\) fait rester dans \( \modE^{++}\) :
    \begin{equation}
        zx-x\frac{ y }{ x }\in \modE^{++}.
    \end{equation}
    Un représentant de \( x\frac{ y }{ x }\) est la suite \( n\mapsto x_n\frac{ y_n }{ x_n }=y_n\). Donc \( x\frac{ y }{ x }=y\). Cela signifie que \( zx-y\in\modE^{++}\) et donc que \( zx>y\).
\end{proof}

\begin{lemma}       \label{LemooMWOUooVWgaEi}
    Pour tout \( a\in \eR\), il existe \( p\in \eN\) tel que \( p>a\).
\end{lemma}

\begin{proof}
    Nous allons donner deux preuves différentes de ce lemme.
    \begin{subproof}
    \item[Première façon]

        L'élément \( a\) de \( \eR\) admet un représentant \( (a_n)\) qui est une suite de Cauchy dans \( \eQ\). C'est donc une suite bornée, c'est-à-dire qu'il existe \( m,q\in \eN\) tels que \( | a_n |\leq m/q\) pour tout \( n\) (proposition~\ref{PropFFDJooAapQlP}\ref{ItemRKCIooJguHdjii}). Soit \( M\) un naturel strictement plus grand que \( m/q\)\footnote{Lemme~\ref{LEMooEBTIooGMoHsj}.}.

    La suite de Cauchy \( (M-a_n)_{n\in \eN}\) est constituée de rationnels positifs et est donc dans \( \modE^+\). La classe de \( M-a\) est donc un réel positif\footnote{Et nous allons d'ailleurs arrêter de toujours préciser «la classe de» lorsque ce n'est pas nécessaire.}. Par définition de la relation d'ordre, \( M\geq a\).
\item[Seconde façon]

    La suite \( (a_n)\) est majorée par \( \frac{ m }{ q }\), donc on a dans \( \eQ\) et pour tout \( n\) :
    \begin{equation}
        a_n\leq \frac{ m }{ q }=M\leq qM.
    \end{equation}
    L'application \( \varphi\colon \eQ\to \eR\) est croissante, donc
    \begin{equation}
        \varphi\big( (a_n) \big)\leq \varphi(qM).
    \end{equation}
    \end{subproof}
\end{proof}

En corollaire, nous avons
\begin{lemma}      \label{LEMooMWOUooVWgbFi}
    Pour tout \( x\in \eR\), il existe \( q\in \eZ\) tel que \( q < x\).
\end{lemma}
\begin{proof}
    Utilisation du lemme précédent avec \( a = -x \): on prend \( q = -p \).
\end{proof}

\begin{theorem}[\cite{RWWJooJdjxEK}]        \label{ThoooKJTTooCaxEny}
    Le corps \( \eR\) est archimédien\footnote{Définition~\ref{DefKCGBooLRNdJf}\ref{ItemooDZQKooPsqeRf}.}.
\end{theorem}

\begin{proof}
    Le lemme~\ref{LemooRordonne} dit que \( \eR\) est totalement ordonné. Soient \( x,y\in \eR\) avec \( x>0\); posons \( a=\frac{ y }{ x }\). Le lemme~\ref{LemooMWOUooVWgaEi} nous donne un \( p \in \eN\) tel que \(p > a\).Nous concluons alors avec le lemme~\ref{LemooYNOVooOwoRwD} :
    \begin{equation}
        px>ax=\frac{ y }{ x }x=y.
    \end{equation}
\end{proof}

Le lemme suivant n'est pas loin de dire que \( \eQ\) est dense dans \( \eR\), à part que nous n'avons pas encore donné de topologie sur \( \eR\).
\begin{lemma}       \label{LemooHLHTooTyCZYL}
    Si \( x,y\in \eR\) sont tels que \( x<y\), alors il existe \( s\in \eQ\) tel que \( x<s<y\).
\end{lemma}

\begin{proof}
    Nous avons par hypothèse que \( y-x>0\) et donc le fait que \( \eR\) soit archimédien (théorème~\ref{ThoooKJTTooCaxEny}) nous donne \( q\in \eN\) tel que \( q(y-x)>1\). Soit
    \begin{equation}
        E=\{ n\in \eZ\tq \frac{ n }{ q }\leq x \}.
    \end{equation}
    Cet ensemble n'est pas vide à cause du lemme~\ref{LEMooMWOUooVWgbFi}; de plus, comme \( |x|q \leq n_0\) pour un certain \( n_0 \) (à cause du lemme~\ref{LemooMWOUooVWgaEi}), l'ensemble \( E\) est majoré par \( n_0\). Donc \( E\) possède un plus grand élément\footnote{Lemme~\ref{LEMooMYEIooNFwNVI}.} \( p\) qui vérifie
    \begin{equation}
        \frac{ p }{ q }\leq x<\frac{ p+1 }{ q }.
    \end{equation}
    De plus \( (p+1)/q<y\). En effet nous avons
    \begin{equation}
        \frac{ p+1 }{ q }=\frac{ p }{ q }+\frac{1}{ q }\leq x+\frac{1}{ q }<x+y-x=y
    \end{equation}
    où nous avons utilisé l'inégalité stricte \( y-x>\frac{1}{ q }\).

    Nous avons donc
    \begin{equation}
        x<\frac{ p+1 }{ q }<y,
    \end{equation}
    et le nombre \( (p+1)/q\) convient comme \( s\).
\end{proof}

\begin{remark}      \label{REMooXOIOooHjwMvA}
    Le lemme~\ref{LemooHLHTooTyCZYL} a également pour conséquence que des ensembles comme \( \mathopen[ -1 , 1 \mathclose]\) ne sont pas bien ordonnés (définition~\ref{DEFooVGYQooUhUZGr}). En effet la partie \( \mathopen] 0 , 1 \mathclose[\) ne possède pas de minimum parce que si \( x\in \mathopen] 0 , 1 \mathclose[\) alors \( 0<x\) et il existe \( s\in \eQ\) (a fortiori \( s\in \eR\)) tel que \( 0<s<x\), c'est à dire que \( x\) n'est pas un minimum de \( \mathopen] 0 , 1 \mathclose[\).
\end{remark}

%---------------------------------------------------------------------------------------------------------------------------
\subsection{Complétude}
%---------------------------------------------------------------------------------------------------------------------------

\begin{lemma}[\cite{RWWJooJdjxEK}]      \label{LemooRTGFooYVstwS}
    Toute suite de Cauchy dans \( \eQ\) converge dans \( \eR\) vers le réel qu'elle représente.
\end{lemma}

\begin{proof}
    Soit \( (x_n)\) une suite de Cauchy de \( \eQ\), c'est à dire que \( x_k\in \eQ\) pour tout \( k\) et qu'elle est de Cauchy. Elle représente un réel \( \bar x\in \eR\), et nous voulons prouver que pour la topologie de \( \eR\) nous avons \( \lim_{n\to \infty} x_n=\bar x\). Dans cette dernière limite, chacun des \( x_n\) est vu dans \( \eR\).


    Si \( \epsilon\in \eQ\) est donné, il existe \( N_{\epsilon}\) tel que si \( p,q\geq N_{\epsilon}\) alors \( | x_p-x_q |< \epsilon\), c'est à dire
    \begin{equation}
        x_p-\epsilon<x_q<x_p+\epsilon.
    \end{equation}
    Soit \( p\geq N_{\epsilon}\) fixé. Pour tout \( q\geq N_{\epsilon}\) nous avons \(  x_p-\epsilon<x_q<x_p+\epsilon \). Par conséquent, la suite \( n\mapsto (x_p-\epsilon)-q_n\) est un élément de \( \modE^-\) et au niveau des classes nous pouvons écrire
    \begin{equation}
        \overline{ n\mapsto (x_p-\epsilon)-x_n }\leq 0.
    \end{equation}
    Vu que \( x_p-\epsilon\) représente la suite constante nous avons l'inégalité suivante dans \( \eR\) :
    \begin{equation}
        x_p-\epsilon-\bar x\leq 0
    \end{equation}
    ou encore : \( \bar x\geq x_p-\epsilon\). En faisant de même avec l'autre partie de l'inégalité, \( x_p-\epsilon\leq \bar x\leq x_p+\epsilon\), ce qui implique que
    \begin{equation}
        x_p\in B(\bar x,\epsilon)
    \end{equation}
    dès que \( p\geq N_{\epsilon}\). Cela signifie que \( x_p\to \bar x\) dans \( \eR\).
\end{proof}

\begin{proposition}     \label{PROPooFGBOooHiZqbs}
    Soit un réel \( x\). Il existe une suite de rationnels strictement croissante qui converge vers \( x\).

    Si de plus \( x>0\), alors la suite peut être choisie parmi les rationnels strictement positifs.
\end{proposition}

\begin{theorem}[Complétude de \( \eR\), critère de Cauchy\cite{RWWJooJdjxEK}] \label{THOooUFVJooYJlieh}
    Nous avons :
    \begin{enumerate}
        \item
            Le corps \( \eR\) est un corps complet (définition~\ref{DefKCGBooLRNdJf}\ref{ITEMooKZZYooDaidGU})
        \item
            Une suite dans \( \eR\) est convergente (définition~\ref{DefKCGBooLRNdJf}\ref{ITEMooDERQooLmJwFR}) si et seulement si elle est de Cauchy (définition~\ref{DefKCGBooLRNdJf}\ref{ItemVXOZooTYpcYN}).
    \end{enumerate}
\end{theorem}
\index{complet!$\eR$!corps}
\index{critère!de Cauchy}
Notez la grande similitude entre ce théorème et le théorème~\ref{THOooNULFooYUqQYo}. Ils ne sont pas équivalents, ne parlent pas exactement du même objet «\( \eR\)», ni des mêmes notions de suites de Cauchy et de complétude.

\begin{proof}
    Soit \( (x_n)\) une suite de Cauchy dans \( \eR\). Pour chaque \( n\), il existe par le lemme~\ref{LemooHLHTooTyCZYL} un \( y_n\in \eQ\) tel que
    \begin{equation}
        x_n-\frac{1}{ n }<y_n<x_n+\frac{1}{ n }.
    \end{equation}
    \begin{subproof}
        \item[\( (y_n)\) est une suite de Cauchy dans \( \eQ\)]
            Nous prouvons que \( (y_n)\) est une suite de Cauchy dans \( \eQ\) (définition~\ref{DefKCGBooLRNdJf}\ref{ItemVXOZooTYpcYN}). Vu que \( (x_n)\) est de Cauchy pour le corps \( \eR\), si \( \epsilon>0\) dans \( \eR\) est donné, il existe \( n_{\epsilon}\) tel que si \( p,q\geq n_{\epsilon}\), alors \( | x_p-x_q |<\epsilon\).

        Nous avons :
        \begin{equation}
            | y_p-y_q |\leq | y_p-x_p |+| x_p-x_q |+| x_q-y_q |<\frac{1}{ p }+\epsilon+\frac{1}{ q }.
        \end{equation}
        En choisissant \( N_{\epsilon}>\max\{ n_{\epsilon},\frac{1}{ \epsilon } \}\) (ce qui est possible par le lemme~\ref{LemooMWOUooVWgaEi}), et en prenant \( p,q>N_{\epsilon}\), nous avons
        \begin{equation}
            | y_p-y_q |\leq 3\epsilon,
        \end{equation}
        ce qui prouve que \( (y_p)\) est une suite de Cauchy dans \( \eQ\), pour la notion de suite de Cauchy dans \( \eQ\).

    \item[Le réel représenté]

        Vu que \( (y_p)\) est de Cauchy dans \( \eQ\), elle représente un réel que nous notons \( \bar y\).

    \item[Convergence de \( (x_n)\)]

        Nous prouvons que \(     x_n\stackrel{\eR}{\longrightarrow}\bar y \).

        Nous savons qu'une suite de Cauchy de rationnels converge dans \( \eR\) vers le réel qu'elle représente, c'est à dire : \( y_n\stackrel{\eR}{\longrightarrow}\bar y\) où chaque \( y_n\in \eQ\) est vu comme la suite constante (cela est le lemme~\ref{LemooRTGFooYVstwS}). Autrement dit, pour \( \epsilon>0\), il existe un \( N_{\epsilon}\in \eN\) tel que si \( p>N_{\epsilon}\) alors \( | \bar y-y_p |<\epsilon\). Pour un tel \( p\) nous avons
        \begin{equation}
            | \bar y-x_p |\leq| \bar y-y_p |+| y_p-x_p |\leq \epsilon+\frac{1}{ p }.
        \end{equation}
        Donc dès que \( p\) est plus grand que \( \max\{ N_{\epsilon},\frac{1}{ \epsilon } \}\), nous avons \( | \bar y-x_p |<2\epsilon\), ce qui signifie que la suite \( (x_n) \) converge vers \( \bar y\) dans \( \eR\).

        Ceci achève de prouver que \( \eR\) est un corps complet.
        \end{subproof}

        En ce qui concerne l'équivalence entre les suites convergentes et de Cauchy, nous venons de prouver que toute suite de Cauchy dans \( \eR\) est convergente. La réciproque est la proposition~\ref{PROPooTFVOooFoSHPg}.

\end{proof}

Nous avons terminé avec la construction des réels. Les propriétés topologiques arrivent en la section~\ref{SECooGKHYooMwHQaD}. En particulier le théorème~\ref{THOooNULFooYUqQYo} pour la complétude de \( \eR\) en tant qu'espace métrique.

%---------------------------------------------------------------------------------------------------------------------------
\subsection{Maximum, supremum et compagnie}
%---------------------------------------------------------------------------------------------------------------------------

Ce n'est un secret pour personne que $\eR$ est un \href{http://fr.wikipedia.org/wiki/Relation_d'ordre}{ensemble totalement ordonné}\footnote{Définition~\ref{DefooYALBooHSXZqB}.} : il y a des éléments plus grands que d'autres, et mieux : à chaque fois que je prends deux éléments différents dans $\eR$, il y en a un des deux qui est plus grand que l'autre. Il n'y a pas d'\emph{ex æquo} dans $\eR$.

  Si je regarde l'intervalle $I=[0,1]$, je peux même dire que $10$ est plus grand que tous les éléments de $I$. Nous disons que $10$ est un \emph{majorant} de $[0,1]$. La définition est la suivante.
\begin{definition}
    Soit \( A\), une partie de \( \eR\). 
    \begin{enumerate}
        \item
            Un nombre \( M\) est un \defe{majorant}{majorant} de \( A\) si \( M\) est plus grand que tous les éléments de \( A\) : pour tout \( x\in A\), \( M\geq x\).
        \item
            Un nombre \( m\) est un \defe{minorant}{minorant} de \( A\) si \( m\) est plus petit que tous les éléments de \( A\) : pour tout \( x\in A\), \( m\leq x\).
    \end{enumerate}
    Nous parons de majorant ou de minorants \emph{stricts} lorsque les inégalités sont strictes.
\end{definition}

Nous insistons sur le fait que l'inégalité n'est pas stricte. Ainsi, $1$ est un majorant de $[0,1]$. Dès qu'un ensemble a un majorant, il en a plein. Si $s$ majore l'ensemble $A$, alors $s+1$, $s+4$ et \( s+\frac{ 3 }{ 7 }\) majorent également $A$.

\begin{example}
Une petite galerie d'exemples de majorants.
\begin{itemize}
\item L'intervalle fermé $[4,8]$ admet entre autres $8$ et $130$ comme majorants,
\item l'intervalle ouvert $]4,8[$ admet également $8$ et $130$ comme majorants,
\item $7$ n'est pas un majorant de $[1,5]\cup]8,32]$,
\item $10/10$ majore les notes qu'on peut obtenir à un devoir.
\item l'intervalle $[4,\infty[$ n'a pas de majorants.
\end{itemize}
\end{example}

\begin{propositionDef}[\wikipedia{en}{http://en.wikipedia.org/wiki/Least_upper_bound_principle}{Least upper bound principle}.]		\label{DefSupeA}
    Soit $A$ une partie majorée de $\eR$. Il existe un unique élément \( M\in \eR\) tel que
    \begin{enumerate}
        \item
            $M\geq x$ pour tout $x\in A$,
        \item
            pour tout $\varepsilon$, le nombre $M-\varepsilon$ n'est pas un majorant de $a$, c'est à dire qu'il existe un élément $x\in A$ tel que $x>M-\varepsilon$.
    \end{enumerate}

    Cet élément est nommé \defe{supremum}{supremum} de $A$ et est noté \( \sup(A)\). De la même façon, \defe{l'infimum}{infimum} de $A$, noté $\inf A$, est le plus grand de ses minorants.
\end{propositionDef}

Par convention, si la partie n'est pas bornée vers le haut, nous dirons que son supremum n'existe pas, ou bien qu'il est égal à $+\infty$, suivant les contextes. Pour votre culture générale, sachez toutefois que $\infty\notin\eR$.

\begin{proof}
    Nous faisons la preuve pour l'infimum.

    \begin{subproof}
    \item[Unicité]

    En ce qui concerne l'unicité, soient \( m_1\) et \( m_2\), deux infimums de \( A\). Supposons \( m_1>m_2\). Alors il existe \( \epsilon>0\) tel que \( m_2<m_2+\epsilon<m_1\) (c'est le lemme~\ref{LemooHLHTooTyCZYL}). Cela prouve que \( m_2+\epsilon\) est un minorant de \( A\) et donc que \( m_2\) n'est pas un infimum.

\item[Existence]

	Soit $A$, une partie de $\eR$. Nous allons trouver son infimum en suivant une méthode de dichotomie. Pour cela nous allons construire trois suites en même temps de la façon suivante. D'abord nous choisissons un point $x_0$ de $A$ et un point $x_1$ qui minore $A$ (qui existe par hypothèse) :
	\begin{equation}
		\begin{aligned}[]
			x_0&\text{ est un élément de }A,\\
			x_1&\text{ est un minorant de }A,\\
			a_0&=x_0\\
			b_0&=x_1\\
			b_1&=x_1.
		\end{aligned}
	\end{equation}
	Ensuite, nous faisons la récurrence suivante :
	\begin{equation}
		\begin{aligned}[]
			x_{n+1}&=\frac{ a_n+b_n }{2},\\
			a_{n+1}&=\begin{cases}
                a_{n}	&	\text{si }x_{n+1} \text{ minore } A \\
				x_{n+1}	&	 \text{sinon},
			\end{cases}\\
			b_{n+1}&=\begin{cases}
                x_{n+1}	&	\text{si }x_{n+1} \text{ minore } A\\
				b_n	&	 \text{sinon}.
			\end{cases}
		\end{aligned}
	\end{equation}
    Nous allons montrer que \( (a_n)\) et \( (b_n)\) sont des suites convergentes de même limite et que cette limite est l'infimum de \( A\).

	Soit $n\in\eN$; il y a deux possibilités. Soit $a_n=a_{n-1}$ et $b_n=x_n$, soit $a_n=x_n$ et $b_n=b_{n-1}$. Supposons que nous soyons dans le premier cas (le second se traite de façon similaire). Alors nous avons
	\begin{equation}
		\begin{aligned}[]
			| a_n-b_n |&=| a_{n-1}-x_n |\\
			&=\left| a_{n-1}-\frac{ a_{n-1}+b_{n-1} }{2} \right| \\
			&=\frac{ 1 }{2}| a_{n-1}-b_{n-1} |,
		\end{aligned}
	\end{equation}
	ce qui prouve que $| a_n-b_n |\to 0$. Nous montrons maintenant que la suite \( (a_n)\) est de Cauchy. En effet nous avons
    \begin{equation}
        | a_n-a_{n-1} |=\begin{cases}
          0\\
          \left| \frac{ a_n -b_n}{ 2} \right|
      \end{cases}\leq \frac{1}{ 2n }.
    \end{equation}
    Il en est de même pour la suite \( (b_n)\). Ce sont deux suites de Cauchy (donc convergentes par la proposition~\ref{PROPooTFVOooFoSHPg}) qui convergent vers la même limite. Soit \( \ell\) cette limite.

	Le nombre $\ell$ minore $A$. En effet si $a\in A$ est plus petit que $\ell$, les éléments $b_n$ tels que $| b_n-\ell |<| a-\ell |$ ne peuvent pas minorer $A$. D'autre part, pour tout $\epsilon$, le nombre $\ell+\epsilon$ ne peut pas minorer $A$. En effet, $\ell$ est la limite de la suite décroissante $(a_n)$, donc il existe $a_n$ entre $\ell$ et $\ell+\epsilon$. Mais $a_n$ ne minore pas $A$, donc $\ell+\epsilon$ ne minore pas non plus $A$.

	Nous avons prouvé que toute partie minorée de $\eR$ possède un infimum.
    \end{subproof}

    La preuve que toute partie majorée possède un supremum se fait de la même façon.
\end{proof}

\subsubsection{\ldots et quelques exemples}
%//////////////////////

En matières de notations, le maximum de l'ensemble $A$ est noté $\max A$, le supremum est noté $\sup A$. Le minimum et l'infimum sont notés $\min A$ et $\inf A$.

\begin{example}
Exemples de différence entre majorant, supremum et maximum.
\begin{itemize}
\item Le nombre $10$ est un supremum, majorant et maximum de l'intervalle fermé $[0,10]$,
\item Le nombre $10$ est un majorant et un supremum, mais pas un maximum de l'intervalle ouvert $]0,10[$,
\item Le nombre $136$ est un majorant, mais ni un maximum ni un supremum de l'intervalle $[0,10]$.
\end{itemize}
\end{example}

En utilisant les notations concises, ces différents cas s'écrivent ainsi :
\begin{align*}
10&=\max[0,10]=\sup[0,10]	& 10&=\sup[0,10[
\end{align*}


\begin{example}
Si on dit que un pont s'effondre à partir d'une charge de $10$ tonnes, alors $10$ tonnes est un \emph{supremum} des charges que le pont peut supporter : si on met $9,999999$ tonnes dessus, il tient encore le coup, mais si on ajoute un gramme, alors il s'effondre (on sort de l'ensemble des charges acceptables).
\end{example}

\begin{example}
Si on dit qu'un pont résiste jusqu'à $10$ tonnes, alors $10$ tonnes est un \emph{maximum} de la charge acceptable. Sur ce pont-ci, on peut ajouter le dernier gramme. Mais à partir de là, le moindre truc qu'on ajoute, il s'effondre.
\end{example}



\begin{example}
	Pour les intervalles, ces notions sont simples : les bornes de l'intervalle sont les supremum et infimum, et ce sont des minima et maxima si l'intervalle est fermé.
	\begin{enumerate}
		\item
			$A=\mathopen[ 1 , 2 \mathclose]$. Tous les nombres plus petits ou égaux à $1$ sont minorants, $1$ est infimum et minimum. Le nombre $2$ est un majorant, le maximum et le supremum.
		\item
			$B=\mathopen] 3 , \pi \mathclose[$. Le nombre $\pi$ est le supremum et est un majorant, mais n'est pas le maximum (parce que $\pi\notin B$). L'ensemble $B$ n'a pas de maximum. Bien entendu, $-1000$ est un minorant.
	\end{enumerate}
    Dans les deux cas, le nombre $53$ est un majorant.
\end{example}

Il existe évidemment de nombreux exemples plus vicieux.

\begin{example}
	Prenons $E=\{ \frac{1}{ n }\tq n\in\eN_0 \}$, dont les premiers points sont indiqués sur la figure~\ref{LabelFigSuiteUnSurn}. Cet ensemble est constitué des nombres $1$, $\frac{ 1 }{2}$, $\frac{1}{ 3 }$, \ldots Le plus grand d'entre eux est $1$ parce que tous les nombres de la forme $\frac{1}{ n }$ avec $n\geq 1$ sont plus petits ou égaux à $1$. Le nombre $1$ est donc maximum de $E$.

	L'ensemble $E$ n'a par contre pas de minimum parce que tout élément de $E$ s'écrit $\frac{1}{ n }$ pour un certain $n$ et est plus grand que $\frac{1}{ n+1 }$ qui est également dans $E$.

	Prouvons que zéro est l'infimum de $E$. D'abord, tous les éléments de $E$ sont strictement positifs, donc zéro est certainement un minorant de $E$. Ensuite, nous savons que pour tout $\varepsilon>0$, il existe un $n$ tel que $\frac{1}{ n }$ est plus petit que $\varepsilon$. L'ensemble $E$ possède donc un élément plus petit que $0+\varepsilon$, et zéro est bien l'infimum.
\end{example}

\newcommand{\CaptionFigSuiteUnSurn}{Les premiers points du type $x_n=1/n$.}
\input{auto/pictures_tex/Fig_SuiteUnSurn.pstricks}

L'exemple suivant est une source classique d'erreurs en ce qui concerne l'infimum. Il sera à relire après avoir vu la définition de limite (définition~\ref{PropLimiteSuiteNum}).

\begin{example}
	Les premiers points de l'ensemble $F=\{ \frac{ (-1)^n }{ n }\tq n\in\eN_0 \}$ sont représentés à la figure~\ref{LabelFigSuiteInverseAlterne}. Bien que (comme nous le verrons plus tard) la limite de la suite $x_n=(-1)^n/n$ soit zéro, il n'est pas correct de dire que zéro est l'infimum de l'ensemble $F$. Le dessin, au contraire, montre bien que $-1$ est le minium (aucun point est plus bas que $-1$), tandis que le maximum est $1/2$.

	Nous reviendrons avec cet exemple dans la suite. Pour l'instant, ayez bien en tête que zéro n'est rien de spécial pour l'ensemble $F$ en ce qui concerne les notions de maximum, minimum et compagnie.
\end{example}
\newcommand{\CaptionFigSuiteInverseAlterne}{Les quelques premiers points du type $(-1)^n/n$.}
\input{auto/pictures_tex/Fig_SuiteInverseAlterne.pstricks}

%+++++++++++++++++++++++++++++++++++++++++++++++++++++++++++++++++++++++++++++++++++++++++++++++++++++++++++++++++++++++++++
\section{Les complexes}
%+++++++++++++++++++++++++++++++++++++++++++++++++++++++++++++++++++++++++++++++++++++++++++++++++++++++++++++++++++++++++++

\begin{probleme}
    Encore une fois, cette section n'est pas du tout faite. Le but de cette section serait de
    \begin{itemize}
        \item Construire \( \eC\) en tant qu'ensemble
        \item Construire les opérations courantes.
        \item Démontrer les propriétés de base.
    \end{itemize}
    Attention : il est sans espoir de parler de forme trigonométrique ici parce que les exponentielles et fonctions trigonométriques ne sont définies qu'avec les séries.

    Nous donnons ici en vrac quelques propriétés et définitions que se doivent d'être présentes dans la version finale de cette section, si elle existe un jour.

    Comme partout dans ce chapitre sur la construction des ensembles de nombres, certains propriétés qui ont l'air toute simples peuvent, en fonction des définitions prises, s'avérer pas du tout évidentes.
\end{probleme}


 \subsection{Définitions}
 Un nombre complexe s'écrit sous la forme $z = a + b i$, où $a$ et $b$
 sont des nombres réels appelés (et notés) respectivement partie réelle
 ($a = \Re(z)$) et partie imaginaire ($b = \Im(z)$) de $z$. L'ensemble
 des nombres de cette forme s'appelle l'ensemble des nombres complexes
 ; cet ensemble porte une structure de corps et est noté $\eC$. Le
 nombre complexe $i = 0 + 1 i$ est un nombre imaginaire qui a la
 particularité que $i^2 = -1$.

 Deux nombres complexes $a + bi$ et $c + di$ sont égaux si et seulement
 si $a = c$ et $b = d$, c'est-à-dire leurs parties réelles sont égales,
 et leurs parties imaginaires sont égales.

 Pour $z = a + bi$ un nombre complexe, on note $\bar z = a - bi$ le
 \Defn{complexe conjugué} de $z$. Dans le plan de Gauss, il s'agit du
 symétrique de $z$ par rapport à la droite réelle (généralement
 dessinée horizontalement).

 On définit le module du complexe $z$ par $\module z = \sqrt{z\bar z} =
 \sqrt{a^2 + b^2}$. Dans le plan de Gauss, il s'agit de la distance
 entre $0$ et $z$.

\begin{proposition}
Pour tous $z = a+bi$ et $z^\prime$ nombres complexes, on a
   \begin{enumerate}
   \item $z \bar z = a^2 + b^2$;
   \item $\bar{\bar{z}} = z$;
   \item $\module z = \module {\bar z}$;
   \item $\module{zz^\prime} = \module z \module{z^\prime}$;
   \item $\module{z+z^\prime} \leq \module z + \module{z^\prime}$.
   \end{enumerate}
\end{proposition}

\begin{lemma}   \label{LEMooONLNooXLNbtB}
    Pour tout \( z\in \eC\) nous avons \( z\bar z=\bar z z=| z |^2\).
\end{lemma}
