% This is part of Mes notes de mathématique
% Copyright (c) 2011-2017
%   Laurent Claessens
% See the file fdl-1.3.txt for copying conditions.

%+++++++++++++++++++++++++++++++++++++++++++++++++++++++++++++++++++++++++++++++++++++++++++++++++++++++++++++++++++++++++++ 
\section{Quelques éléments sur les ensembles}
%+++++++++++++++++++++++++++++++++++++++++++++++++++++++++++++++++++++++++++++++++++++++++++++++++++++++++++++++++++++++++++

\begin{definition}      \label{DefEOZLooUMCzZR}
    Un ensemble est \defe{infini}{ensemble!infini} s'il peut être mis en bijection avec un de ses sous-ensembles propres.
\end{definition}

%--------------------------------------------------------------------------------------------------------------------------- 
\subsection{Lemme de Zorn}
%---------------------------------------------------------------------------------------------------------------------------

\begin{normaltext}\label{NORooLMBYooYjUoju}
L'\defe{axiome du choix}{axiome!du choix} que nous acceptons peut s'énoncer comme ceci\cite{ooKLIXooHbpufL} : Étant donné un ensemble X d'ensembles non vides, il existe une fonction définie sur X, appelée fonction de choix, qui à chacun d'entre eux associe un de ses éléments. 
\end{normaltext}

\begin{definition}      \label{DefooFLYOooRaGYRk}
    Une \defe{relation d'ordre}{ordre} sur un ensemble \( E\) est une relation binaire (notée \( \leq\)) sur \( E\) telle que pour tout \( x,y,z\in E\),
    \begin{enumerate}
        \item
            Réflexivité : \( x\leq x\)
        \item
            antisymétrie : \( x\leq y\) et \( y\leq x\) implique \( x=y\)
        \item
            transitivité : \( x\leq y\) et \( y\leq z\) implique \( x\leq z\).
    \end{enumerate}

    Un ensemble ordonné est \defe{totalement ordonné}{ordre!total} si deux éléments sont toujours comparables : si \( x,y\in E\) alors nous avons soit \( x\leq y\) soit \( y\leq x\).

    Un ensemble ordonné est \defe{bien ordonné}{bon!ordre}\index{ordre!bon ordre} si toute partie non vide possède un plus petit élément : si \( A\) est une partie de \( E\), alors \( \exists x\in A,\forall y\in A, x\leq y\).
\end{definition}
Notons que l'inégalité stricte (nous définissons \( x<y\) si et seulement si \( x\leq y\) et \( x\neq y\)) n'est pas une relation d'ordre parce qu'elle n'est pas réflexive.

\begin{definition}[Ensemble inductif\cite{MathAgreg}]  \label{DefGHDfyyz}
    Un ensemble est \defe{inductif}{inductif} si tout sous-ensemble totalement ordonné admet un majorant.
\end{definition}

Si \( E\) est un ensemble, l'inclusion est un ordre sur l'ensemble des partie de \( E\), mais pas un ordre total parce que si \( X,Y\) sont des parties de \( E\), alors nous n'avons pas automatiquement soit \( X\subset Y\) ou \( Y\subset X\).

\begin{lemma}[Lemme de Zorn]    \label{LemUEGjJBc}
    Tout ensemble ordonné inductif non vide admet un maximum.
\end{lemma}
\index{lemme!de Zorn}
%TODO : une preuve.
Le point intéressant de ce lemme est que le majorant soit un maximum, c'est à dire qu'il appartienne à l'ensemble.

\begin{proposition}[\cite{KZIoofzFLV}, lemme 8.1] \label{PropVCSooMzmIX}
    Si \( S\) est un ensemble infini alors il existe une bijection \( \varphi\colon \{ 1,2 \}\times S\to S\).
\end{proposition}
%TODO : la preuve

%--------------------------------------------------------------------------------------------------------------------------- 
\subsection{Complémentaire}
%---------------------------------------------------------------------------------------------------------------------------
\label{AppComplement}

Soit $E$, un ensemble et $A$, une partie de $E$ (c'est à dire un sous-ensemble de $E$). Nous désignons par $\complement A$\nomenclature[T]{$\complement A$}{Le complémentaire de l'ensemble $A$} désigne le \defe{complémentaire}{complémentaire} de l'ensemble $A$ dans $E$. Il s'agit de l'ensemble des points de $E$ qui ne font pas partie de $A$ :
\begin{equation}
	\complement A=E\setminus A=\{ x\in E\tq x\notin A \}.
\end{equation}
Nous allons aussi régulièrement noter le complémentaire de \( A\) par \( A^c\)\nomenclature[T]{\( A^c\)}{complémentaire de \( A\)}.

\begin{lemma}		\label{LemPropsComplement}
	Quelques propriétés à propos des complémentaires. Si $E$ est un ensemble et si $A$ et $B$ sont des sous-ensembles de $E$, nous avons
	\begin{enumerate}
		\item
			$\complement \complement A =A $, en d'autres termes, $E\setminus(E\setminus A)=A$,
		\item
			$\complement(A\cap B)=\complement A\cup\complement B$,
		\item
			$\complement(A\cup B)=\complement A\cap\complement B$,
		\item	\label{ItemLemPropComplementiii}
			$A\setminus B=A\cap\complement B$.
	\end{enumerate}
\end{lemma}

\begin{definition}[différence symétrique]    \label{DefBMLooVjlSG}
    Si \( A\) et \( B\) sont des ensembles, l'ensemble \( A\Delta B\)\nomenclature[T]{\( A\Delta B\)}{différence symétrique} est la \defe{différence symétrique}{ensemble!différence symétrique} d'ensembles : 
    \begin{equation}
        A\Delta B=(A\cup B)\setminus(A\cap B).
    \end{equation}
\end{definition}
C'est l'ensemble des éléments étant soit dans \( A\) soit dans \( B\) mais pas dans les deux.

\begin{lemma}   \label{LemCUVoohKpWB}
    Si \( A\) et \( B\) sont des ensembles nous avons
    \begin{enumerate}
        \item\label{ItemVUCooHAztC}
            \( A^c\Delta B^c=A\Delta B\).
        \item\label{ItemVUCooHAztCii}
            \( (A\Delta B)\Delta B=A\).
    \end{enumerate}
\end{lemma}

\begin{proof}
    La première assertion provient de l'égalité \( X^c\setminus Y^c=Y\setminus X\) (qu'on montre de façon classique, éventuellement en séparant les cas suivant que \( B\) est inclus à \( A\) ou non) :
    \begin{equation}
        A^c\Delta B^c=(A^c\cup B^c)\setminus(A^c\cap B^c)=(A\cap B)^c\setminus(A\cup B)^c=(A\cup B)\setminus (A\cap B)=A\Delta B.
    \end{equation}

    Pour la seconde assertion, il faut remarquer que \( (A\Delta B)\cup B=A\cup B\) et que \( (A\Delta B)\cap B=B\setminus A\), donc
    \begin{equation}
        (A\Delta B)\Delta B=(A\cup B)\setminus (B\setminus A)=A.
    \end{equation}
\end{proof}

%--------------------------------------------------------------------------------------------------------------------------- 
\subsection{Relations d'équivalence}
%---------------------------------------------------------------------------------------------------------------------------
\label{appEquivalence}

\begin{definition}  \label{DefHoJzMp}
Si $E$ est un ensemble, une \defe{relation d'équivalence}{equivalence@équivalence!relation} sur $E$ est une relation $\sim$ telle qui est à la fois
\begin{description}
	\item[réflexive] $x\sim x$ pour tout $x\in E$,
	\item[symétrique] $x\sim y$ si et seulement si $y\sim x$;
	\item[transitive] si $x\sim y$ et $y\sim z$, alors $x\sim z$.
\end{description}
\end{definition}

Par exemple, sur l'ensemble de tous les polygones du plan, la relation «a le même nombre de côté» est une relation d'équivalence. Plus précisément, si $P$ et $Q$ sont deux polygones, nous disons que $P\sim Q$ si et seulement si $P$ et $Q$ ont le même nombre de côté. Cela est une relation d'équivalence :
\begin{itemize}
	\item 
		un polygone $P$ a toujours le même nombre de côtés que lui-même : $P\sim P$;
	\item
		si $P$ a le même nombre de côtés que $Q$ ($P\sim Q$), alors $Q$ a le même nombre de côtés que $P$ ($Q\sim P$);
	\item
		si $P$ a le même nombre de côtés que $Q$ ($P\sim Q$) et que $Q$ a le même nombre de côtés que $R$ ($Q\sim R$), alors $P$ a le même nombre de côtés que $R$ ($P\sim R$).
\end{itemize}

Soit \( f\) une application entre deux ensembles \( E\) et \( F\). Nous définissons une relation d'équivalence sur \( E\) par
\begin{equation}
    x\sim y\Leftrightarrow f(x)=f(y).
\end{equation}
Nous notons par \( \pi\colon E\to E/\sim\) la projection canonique. L'application
\begin{equation}
    \begin{aligned}
        g\colon E/\sim&\to F \\
        [x]&\mapsto f(x) 
    \end{aligned}
\end{equation}
est bien définie et injective. Elle n'est pas surjective tant que \( f\) ne l'est pas. La \defe{décomposition canonique}{canonique!décomposition}\index{décomposition!canonique} de \( f\) est 
\begin{equation}
    f=g\circ\pi.
\end{equation}

%+++++++++++++++++++++++++++++++++++++++++++++++++++++++++++++++++++++++++++++++++++++++++++++++++++++++++++++++++++++++++++ 
\section{Les naturels}
%+++++++++++++++++++++++++++++++++++++++++++++++++++++++++++++++++++++++++++++++++++++++++++++++++++++++++++++++++++++++++++

\nomenclature{$\eN_0$}{les naturels non nuls : $\eN_0=\eN\setminus\{ 0 \}$}
\begin{definition}
    Un ensemble est \defe{dénombrable}{dénombrable} s'il peut être mis en bijection avec \( \eN\). Il est \defe{non dénombrable}{non dénombrable} s'il est infini et ne peut pas être mis en bijection avec \( \eN\).
\end{definition}
Une chose vraiment amusante avec dette définition que l'on met en rapport avec la définition \ref{DefEOZLooUMCzZR}, c'est qu'un ensemble fini n'est ni dénombrable ni non dénombrable.

\begin{proposition} \label{PropQEPoozLqOQ}
    Toute partie d'un ensemble fini est finie, et toute partie d'un ensemble dénombrable est finie ou dénombrable.
\end{proposition}
%TODO : la preuve

%+++++++++++++++++++++++++++++++++++++++++++++++++++++++++++++++++++++++++++++++++++++++++++++++++++++++++++++++++++++++++++ 
\section{Les entiers}
%+++++++++++++++++++++++++++++++++++++++++++++++++++++++++++++++++++++++++++++++++++++++++++++++++++++++++++++++++++++++++++

Toutes les constructions sont faites dans \cite{RWWJooJdjxEK}.

%+++++++++++++++++++++++++++++++++++++++++++++++++++++++++++++++++++++++++++++++++++++++++++++++++++++++++++++++++++++++++++ 
\section{Quelques structures algébriques}
%+++++++++++++++++++++++++++++++++++++++++++++++++++++++++++++++++++++++++++++++++++++++++++++++++++++++++++++++++++++++++++

\begin{definition}[Anneau\cite{Tauvel}]     \label{DefHXJUooKoovob}
    Un \defe{anneau}{anneau} est un triple \( (A,+,\cdot)\) avec les conditions
    \begin{enumerate}
        \item
            \( (A,+)\) est un groupe abélien. Nous notons \( 0\) le neutre.
        \item
            La multiplication est associative et nous notons \( 1\) le neutre
        \item
            La multiplication est distributive par rapport à l'addition.
    \end{enumerate}
\end{definition}

\begin{definition}  \label{DefTMNooKXHUd}
    Un \defe{corps}{corps} est un anneau non nul dans lequel tout élément non nul est inversible.
\end{definition}

\begin{remark}      \label{REMooYRNUooYgBBKF}
    Un anneau est ce qu'on appelle «\emph{ring}» en anglais. Un corps est en anglais «\emph{field}». De plus le mot «\emph{field}» comprend la commutativité. Donc certains utilisent le mot «corps» pour dire «corps commutatif» et parlent alors d'anneau \emph{à division} pour parler de corps non commutatifs.
\end{remark}

\begin{definition}[Morphisme d'anneau]      \label{DEFooQBGJooKJqHXr}
    Si \( (A,+,\cdot)\) et \( (B,+,\cdot)\) sont des anneaux, un \defe{morphisme}{morphisme!d'anneaux} est une application \( f\colon A\to B\) telle que
    \begin{enumerate}
        \item \( f(a+b)=f(a)+f(b)\)
        \item
            \( f(a\cdot b)=f(a)\cdot f(b)\)
        \item
            \( f(1)=1\).
    \end{enumerate}
    Étant bien entendu que les significations de \( 1\), $+$ et \( \cdot\) sont différentes à gauche et à droite. 

    Un \defe{isomorphisme}{isomorphisme!d'anneaux} est un morphisme bijectif.

    Un corps étant un anneau sans plus de structure, un \defe{morphisme de corps}{morphisme!de corps}\index{isomorphisme!de corps} n'est qu'un morphisme des anneaux.
\end{definition}


\begin{definition}  \label{DefooQULAooREUIU}
    Un sous ensemble \( I\subset A\) est un \defe{idéal}{idéal!dans un anneau} à gauche si
    \begin{enumerate}
        \item
            \( I\) est un sous-groupe pour l'addition,
        \item
            pour tout \( a\in A\), \( aI\subset I\).
    \end{enumerate}
\end{definition}


\begin{definition}      \label{DefKCGBooLRNdJf}
    Ordre et choses reliées dans un corps.
    \begin{enumerate}
        \item
            Un corps \( \eK\) est \defe{totalement ordonné}{ordre!dans un corps}\index{corps!ordonné} s'il existe une relation d'ordre total\footnote{Définition \ref{DefooFLYOooRaGYRk}.} tel que
            \begin{enumerate}
                \item
                    \( x\leq y\) implique \( x+z\leq y+z\) pour tout \( x,y,z\in \eK\)
                \item
                    \( x\geq 0\) et \( y\geq 0\) implique \( xy\geq 0\).
            \end{enumerate}
        \item       \label{ItemooWUGSooRSRvYC}
            Si \( \eK\) est un corps totalement ordonné, nous y définissons la valeur absolue par
            \begin{equation}
                | x |=\begin{cases}
                    x    &   \text{si }x\geq 0\\
                    -x    &    \text{si } x\leq 0.
                \end{cases}
            \end{equation}
        \item       \label{ItemVXOZooTYpcYN}
    La suite \( (x_n)\) dans le corps totalement ordonné \( \eK\) est \defe{de Cauchy}{suite!de Cauchy!dans un corps} si pour tout \( \epsilon\in \eK^+\), il existe \( N\in \eN\) tel que si \( p,q\geq N\) alors \( | x_p-x_q |\leq \epsilon\).
        \item
    La suite \( (x_n)\) dans le corps totalement ordonné \( \eK\) est \defe{convergente}{convergence!suite!dans un corps} s'il existe \( q\in \eK\) tel que pour tout \( \epsilon\in \eK^+\), il existe \( N\) tel que si \( k\geq N\) alors \( | x_k-q |\leq \epsilon\).
\item   \label{ItemooDZQKooPsqeRf}
            Un corps \( \eK\) est \defe{archimédien}{corps!archimédien}\index{archimédien} s'il est totalement ordonné et si pour tout \( x,y\in \eK\) avec \( x>0\), il existe \( n\in \eN\) tel que \( nx\geq y\).
        \item
            Un corps totalement ordonné est \defe{complet}{corps!complet}\index{complet!corps} si toute suite de Cauchy dans \( \eK\) y est convergente.
    \end{enumerate}
\end{definition}

Parmi ces définition, celles de suite convergente, de Cauchy et de corps complet seront utilisées dans le cas de \( \eQ\) (et de \( \eR\) pour la complétude). Elles seront prouvées être équivalentes aux définitions topologiques dans le cas particulier de \( \eR\) et \( \eQ\) lorsque la topologie métrique sera définie. Dans cet état d'esprit nous n'allons pas démontrer tout de suite que \( \eR\) est un corps complet. Nous allons directement démontrer que c'est un espace topologique complet.

\begin{lemma}[Propriétés de la valeur absolue]  \label{LemooANTJooYxQZDw}
    Soit \( \eK\) un corps totalement ordonné. Si \( x,y\in \eK\) alors
    \begin{enumerate}
        \item       \label{ItemooNVDIooSuiSoB}
            Si \( x\geq 0\) alors \( -x\leq 0\).
        \item
            \( | x |\geq 0\)
        \item
            \( | x |=0\) si et seulement si \( x=0\)
        \item\label{ItemooOMKNooRlanvk}
            \( | x+y |\leq | x |+| y |\).
    \end{enumerate}
\end{lemma}

\begin{proof}
    Point par point
    \begin{enumerate}
        \item
            Nous partons de \( x\geq 0\) et nous ajoutons \( -x\) des deux côtés en profitant de la définition d'un corps totalement ordonné : \( x-x\geq -x\) et donc \( 0\geq-x\), c'est à dire \( -x\leq 0\).
        \item
            Si \( x\geq 0\) alors c'est vrai. Sinon, \( x\leq 0\) et \( | x |=-x\geq 0\) par le point \ref{ItemooNVDIooSuiSoB}.
        \item
            Si \( x=0\) alors \( x=-x\) et \( | x |=0\). Au contraire si \(x\neq 0\) alors \( -x\neq 0\) et que \( x\) soit positif ou négatif, nous aurons toujours \( \pm x\neq 0\).
        \item
            Nous supposons que \( x\leq y\) et nous subdivisons suivante la positivité de \( x\) et \( y\).
            \begin{enumerate}
                \item
                    Si \( x,y\geq 0\). Dans ce cas, \( x+y\geq y\geq 0\), donc \( | x+y |=x+y=| x |+| y |\).
                \item
                    Si \( x,y\leq 0\). Dans ce cas, \( x+y\leq 0\) et nous avons \( | x+y |=-x-y=| x |+| y |\).
                \item
                    Si \( x\leq 0\) et \( y\geq 0\). Nous subdivisons encore en deux cas suivant que \( x+y\) est positif ou négatif. Si \( x+y\geq 0\), alors nous écrivons successivement
                    \begin{subequations}
                        \begin{align}
                            x&\leq 0\\
                            x+y&\leq y\leq y+| x |=| x |+| y |
                        \end{align}
                    \end{subequations}
                    et donc \( | x+y |=x+y\leq | x |+| y |\).

                    Nous supposons à présent que \( x\leq 0\), \( y\geq 0\) et \( x+y\leq 0\). Dans ce cas il suffit d'écrire \( | x+y |=| (-x)+(-y) |\) pour retomber dans le cas précédent à inversion près de \( x\) et \( y\).
            \end{enumerate}
    \end{enumerate}
\end{proof}

\begin{remark}      \label{RemooJCAUooKkuglX}
    La partie \ref{ItemooOMKNooRlanvk} est très importante parce que c'est elle qui fera presque toutes les majorations dont nous aurons besoin en analyse. En effet elle donne l'inégalité triangulaire de la façon suivante : si \( x,y,z\in \eK\) nous avons
    \begin{equation}
        | x-y |= |  (x-z)+(z-y) |\leq | x-z |+| z-y |.
    \end{equation}
\end{remark}

%+++++++++++++++++++++++++++++++++++++++++++++++++++++++++++++++++++++++++++++++++++++++++++++++++++++++++++++++++++++++++++ 
\section{Les rationnels}
%+++++++++++++++++++++++++++++++++++++++++++++++++++++++++++++++++++++++++++++++++++++++++++++++++++++++++++++++++++++++++++

%--------------------------------------------------------------------------------------------------------------------------- 
\subsection{Suites de Cauchy dans les rationnels}
%---------------------------------------------------------------------------------------------------------------------------

\begin{proposition}[\cite{RWWJooJdjxEK}]        \label{PropFFDJooAapQlP}
    Principales propriétés des suites de Cauchy dans \( \eQ\).
    \begin{enumerate}
        \item       \label{ItemRKCIooJguHdji}
            Toute suite convergente est de Cauchy\footnote{Et non la réciproque, qui sera justement la grande innovation des nombres réels.}.
        \item       \label{ItemRKCIooJguHdjii}
            Toute suite de Cauchy est bornée.
        \item       \label{ItemRKCIooJguHdjiii}
            Si \( x_n\to 0\) et si \( (y_n)\) est bornée, alors \( x_ny_n\to 0\)
        \item
            Si \( (x_n)\) et \( (y_n)\) sont de Cauchy alors \( (x_n+y_n)\), \( (x_n-y_n)\) et \( (x_ny_n)\) sont également de Cauchy.
        \item
            Si \( x_n\to a \) et \( y_n\to b \) alors \( x_n+y_n\to a+b\), \( x_n-y_n\to a-b\) et \(  x_ny_n\to ab  \).
        \item   \label{ItemRKCIooJguHdjvi}
            Soit \( (x_n)\) une suite de Cauchy qui ne converge pas vers zéro. Alors il existe \( n_0\) tel que la suite \( \left( \frac{1}{ x_n } \right)_{n\geq n_0}\) soit de Cauchy.
    \end{enumerate}
\end{proposition}

\begin{proof}
    Point par point.
    \begin{enumerate}
        \item
            Soit \( (x_n)\) une suite dans \( \eQ\) qui converge vers \( x\in \eQ\). Soit aussi\footnote{Le \( \epsilon\) que nous prenons maintenant est dans \( \eQ\), et ce sera toujours ainsi dans les prochaines pages; nous ne le répéterons pas à chaque fois.} \( \epsilon>0\). Il existe \( N_{\epsilon}\in \eN\) tel que \( n>N_{\epsilon}\) implique \( | x_n-x |\leq \epsilon\). Si \( p,q\geq N_{\epsilon/2}\) alors
            \begin{equation}
                | x_p-x_q |\leq | x_p-x |+| x-x_q |\leq \epsilon.
            \end{equation}
            Donc la suite \( (x_n)\) est de Cauchy.
        \item
            Soit \( (x_n)\) une suite de Cauchy dans \( \eQ\). Avec \( \epsilon=1\) dans la définition, si \( q>N_1\), nous avons
            \begin{equation}
                | x_q-x_{N_1} |\leq 1.
            \end{equation}
            Et donc pour tout \( q\) plus grand que \( N_1\), \( x_N-1\leq x_q\leq x_N+1\), ou encore, pour tout \( n\) :
            \begin{equation}
                | x_n |\leq\max\{ | x_1 |,| x_2 |,\ldots,| x_N |,| x_N+1 | \}.
            \end{equation}
            La suite est donc bornée.
        \item
            Soit \(\epsilon>0\). Les hypothèses disent qu'il existe un \( N\) tel que \( | x_n |\leq \epsilon\) dès que \( n\geq N\). Et il existe aussi \( M\geq 0\) tel que \( | y_n |\leq M\) pour tout \( n\). Du coup, lorsque \( n\geq N\) nous avons \( | x_ny_n |\leq M\epsilon\).
        \item
            En ce qui concerne la somme,
            \begin{equation}        \label{EqDCNBooAzrrBi}
                | x_p+y_p-x_q-y_q |\leq | x_p-x_q |+| y_p-y_q |.
            \end{equation}
            Soit \( N_1\) tel que si \( p,q\geq N_1\) alors \( | x_p-x_q |\leq \epsilon\) et \( N_2\) de même pour la suite \( (y_n)\). En prenant \( N=\max\{ N_1,N_2 \}\), la somme \eqref{EqDCNBooAzrrBi} est plus petite que \( 2\epsilon\) dès que \( p,q\geq N\).

            Passons à la démonstration du fait que le produit de deux suites de Cauchy est de Cauchy. Les suites \( (x_n)\) et \( (y_n)\) sont bornées et quitte à prendre le maximum, nous disons qu'elles sont toutes les deux bornées par le nombre \( M\) : pour tout \( n\) nous avons \( | x_n |\leq M\) et \( | y_n |\leq M\). Nous avons :
            \begin{equation}
                | x_py_p-x_qy_q |\leq | x_py_p-x_qy_p |+| x_qy_p-x_qy_q |\leq | y_p | |x_p-x_q |+| x_q | |y_p-y_q |.
            \end{equation}
            Vu que \( (x_n)\) et \( (y_n)\) sont de Cauchy, si \( p\) et \( q\) sont assez grand, les deux différences sont majorées par \( \epsilon\) et nous avons
            \begin{equation}
                | x_py_p-x_qy_q |\leq M\epsilon+M\epsilon=2M\epsilon,
            \end{equation}
            ce qui prouve que \( (x_ny_n)\) est de Cauchy.
        \item
            En ce qui concerne la somme, nous pouvons tout de suite calculer
            \begin{equation}        
                | x_n+y_n-(a+b) |\leq | x_n-a |+| y_n-b |.
            \end{equation}
            Il existe une valeur de \( n\) à partir de laquelle le premier terme est plus petit que \( \epsilon\) et une à partir de laquelle le second terme est plus petit que \( \epsilon\). En prenant le maximum des deux, la somme est plus petite que \( 2\epsilon\).

            En ce qui concerne le produit,
            \begin{equation}
                | x_ny_n-ab |\leq | x_ny_n-ay_n |+| ay_n-ab |\leq | y_n || x_n-a |+| a || y_n-b |.
            \end{equation}
            Les suites \( | x_n-a |\) et \( | y_n-b |\) convergent vers zéro; la suite \( (y_n)\) est bornée parce que convergente (combinaison des points \ref{ItemRKCIooJguHdji} et \ref{ItemRKCIooJguHdjii})  et \( a\) (la suite constante) est également bornée. Donc par le point \ref{ItemRKCIooJguHdjiii}, nous avons
            \begin{equation}
                y_n| x_n-a |+a| y_n-b |\to 0.
            \end{equation}
            Au passage nous avons également utilisé la propriété de la somme que nous venons de démontrer.
        \item 
            La négation de «converge vers zéro» est : il existe \( \alpha>0\) tel que pour tout \( N\in \eN\), il existe \( n\geq N\) tel que \( | x_n |>\alpha\). Mais notre suite est de Cauchy, donc il existe \( n_0\in \eN\) tel que si \( p,q\geq n_0\) alors 
            \begin{equation}
                | x_p-x_q |\leq \frac{ \alpha }{2}.
            \end{equation}
            Soit aussi \( n\geq n_0\) tel que \( | x_n |\geq \alpha\). Si \( p>n\) nous avons aussi \( | x_n-x_p |\leq \frac{ \alpha }{2}\). Cela implique \( | x_p |\geq \frac{ \alpha }{2}\) et en particulier \( x_p\neq 0\).

            En prenant \( p,q\geq n\) nous avons donc \( |  x_p|>\frac{ \alpha }{2}\) et \( | x_q |>\frac{ \alpha }{2}\). Avec cela,
            \begin{equation}
                \left| \frac{1}{ x_p }-\frac{1}{ x_q } \right| \leq \frac{ | x_q-x_p | }{ | x_px_q | }\leq \frac{ 4 }{ \alpha^2 }| x_q-x_p |.
            \end{equation}
            Mais comme \( (x_n)\) est de Cauchy, il suffit de choisir \( p\) et \( q\) à la fois plus grand que \( n\) et plus grand que le \( N\) qui fait \( | x_p-x_q |\leq \epsilon\) pour \( p,q\geq N\) pour obtenir
            \begin{equation}
                \left| \frac{1}{ x_p }-\frac{1}{ x_q } \right| \leq \frac{ 4 }{ \alpha^2 }\epsilon.
            \end{equation}
            Donc \( \left( \frac{1}{ x_n } \right)\) est de Cauchy.
    \end{enumerate}
\end{proof}

%--------------------------------------------------------------------------------------------------------------------------- 
\subsection{Insuffisance des rationnels}
%---------------------------------------------------------------------------------------------------------------------------

Nous allons voir qu'il n'existe pas de nombres rationnels \( x\) tels que \( x^2=2\), mais que pourtant il existe une infinité de suites de rationnels \( (x_n)\) tels que \(  x_n^2\to 2  \).

\begin{lemma}       \label{LemJPIUooWFHaFM}
    Un entier \( x\) est pair si et seulement si l'entier \( x^2\) est pair.
\end{lemma}

\begin{proof}
    Si \( x\) est un nombre pair, alors il existe un entier \( a\) tel que \( x=2a\) alors \( x^2=4a^2\) est pair.

    Inversement, si \( x\) est impair alors il existe un entier \( a\) tel que \( x=2a+1\) et alors \( x^2=4a^2+4a+1=2(2a^2+2a)+1\) est impair.
\end{proof}

\begin{proposition}[Irrationalité de \( \sqrt{2}\)]     \label{PropooRJMSooPrdeJb}
    Il n'existe pas de fractions d'entiers dont le carré soit égal à \( 2\).
\end{proposition}
\index{irrationalité!\( \sqrt{2}\)}
Le théorème \ref{THOooYXJIooWcbnbm} nous dira que tous les \( \sqrt{n}\) sont irrationnels dès que \( n\) n'est pas un carré parfait.

\begin{proof}
    Nous supposons que la fraction d'entiers \( a/b\) est telle que \( a^2/b^2=2\), et nous allons construire une suite d'entiers strictement décroissante et strictement positive, ce qui est impossible.

    Grâce au lemme \ref{LemJPIUooWFHaFM} nous avons successivement les affirmations suivantes :
    \begin{itemize}
        \item 
        $\frac{ a^2 }{ b^2 }=2$  avec \( a\neq 0\) et \( b\neq 0\).
    \item
        \( a^2=2b^2\), donc \( a^2\) est pair.
    \item
        \( a\) est alors pair et \( a^2\) est divisible en \( 4\). Soit \( a^2=4k\).
    \item
        \( 4k/b^2=2\), donc \( 4k=2b^2\), donc \( b^2=2k\) et \( b^2\) est pair.
    \item
        Nous déduisons que \( b\) est pair.
    \end{itemize}
    La fraction \( \frac{ a/2 }{ b/2 }\) est alors une nouvelle fraction d'entiers dont le carré vaut $2$. En procédant comme à nouveau, la fraction d'entiers \( \frac{ a/4 }{ b/4 }\) a la même propriété.

    En particulier, tous les nombres de la forme \( a/2^n\) sont des entiers.  Ils forment une suite strictement décroissante d'entiers positifs.  Impossible, me diriez vous ?!? Et vous auriez bien raison : il n'y a pas de fractions d'entiers dont le carré vaut \( 2\).
\end{proof}

\begin{proposition}[\cite{MonCerveau}]      \label{PROPooAHMIooVpunrF}
    Soit la suite de rationnels \( (x_n)\) définie par \( x_0\in \eQ^+\) et 
    \begin{equation}
        x_{n+1}=x_n+\frac{ x_n^2-2 }{ 2x_n }.
    \end{equation}
    Alors :
    \begin{enumerate}
        \item
            En posant \( y_n=x_n^2\) nous avons \( y_n\to 2\).
        \item
            La suite \( (x_n)\) est de Cauchy.
        \item
            La suite \( (x_n)\) ne converge pas dans \( \eQ\).
    \end{enumerate}
\end{proposition}

\begin{proof}
    Tout d'abord un petit calcul montre que
    \begin{equation}
        x_{n+1}^2=2+\frac{ (y_n-2)^2 }{ 4y_n },
    \end{equation}
    c'est à dire que quelle que soit la valeur de \( x_0\), dès le \( x_1\), les valeurs de \( y_n\) sont plus grandes que \( 2\). Notons qu'il n'est pas possible d'avoir \( y_n=2\). Nous pouvons donc supposer \( y_n>2\) pour tout \( n\). Alors en posant \( y_n=2+s\) nous avons
    \begin{equation}
        y_{n+1}=2+\frac{ s^2 }{ 8+4s }.
    \end{equation}
    Si \( s<1\) alors \( \frac{ s^2 }{ 8+4s }<s^2\) et en réalité le processus \( s\mapsto s^2/(8+4s)\) tend très vite vers zéro. Nous devons donc montrer qu'il existe un \( n\) tel que \( y_n=2+s\) avec \( s<1\).

    Montrons pour cela que si \( n<s<2n\) alors\footnote{Voir la remarque \ref{RemUZCAooWNogzI} pour comprendre d'où vient l'idée de cette majoration.}
    \begin{equation}\label{EqYNKQooUBfhgz}
        \frac{ s^2 }{ 8+4s }<n
    \end{equation}
    En effet si \( n<s<2n\) nous pouvons majorer le numérateur par \( 4n^2\) et minorer le dénominateur par \( 4n\).

    Prouvons à présent le résultat. Pour un \( n>1\) nous avons
    \begin{equation}
        y_{n+1}=2+\frac{ s }{ 8+4s }
    \end{equation}
    avec \( s>0\). Nous considérons \( n_0\in \eN\) tel que \( n_0<s<2n_0\). Alors
    \begin{equation}
        y_{n+2}=2+s_1
    \end{equation}
    avec \( s_1<n\). Il existe donc \( n_1\) tel que \( n_1<s_1<2n_1\) et \( n_1<n_0\). Nous avons alors \( y_{n+3}=2+s_2\) avec \( s_2<n_1<n_0\).

    Nous construisons ainsi des suites \( (s_i)\) et \( (n_i)\) telles que 
    \begin{equation}
        y_{n+k}=2+s_k
    \end{equation}
    avec \( s_k<n_{k-1}<n_{k-2}<\cdots<n_0\). En procédant ainsi au maximum \( n-1\) fois nous avons \( s_k<1\). À partir du moment où \( y_{n+k}=2+s_k\) avec \( s_k<1\), nous avons déjà vu qu'il est certain que \( y_n\to 2\).

    Prouvons que la suite \( (x_n)\) est de Cauchy. Vu que \( x_n^2\to 2\) nous avons \( x_n\geq 1\) à partir d'une certaine valeur de \( n\). Si de plus elle n'est pas de Cauchy, alors pour tout \( \epsilon\), et tout \( N\) donnés, il existe \( p,q\geq N\) tels que \( | x_p-x_q |\geq \epsilon\). Cela donne
    \begin{equation}
        | x_p^2-x_q^2 |=| x_p+x_q | |x_p-x_q |\geq 2\epsilon.
    \end{equation}
    Par conséquent la suite \( (x_n^2)\) n'est pas de Cauchy, alors qu'elle l'est parce qu'elle est convergente (par la proposition \ref{PropFFDJooAapQlP}\ref{ItemRKCIooJguHdji}).

    Enfin supposons que \( x_n\to a\in \eQ\). Dans ce cas nous aurions \( x_n^2\to a^2=2\). Mais nous savons que \( a^2=2\) est impossible dans \( \eQ\).
\end{proof}

Bref, la suite \( (x_n)\) est de Cauchy, ne converge pas et a son carré qui converge vers \( 2\).

\begin{remark}\label{RemUZCAooWNogzI}
    Vu que nous n'avons pas encore défini les réels, ce qui suit n'est que informel. Ce qui a motivé la majoration \eqref{EqYNKQooUBfhgz} est la résolution de l'inéquation
    \begin{equation}
        \frac{ s^2 }{ 8+4s }<n
    \end{equation}
    qui donne les bornes \( 2n\pm\sqrt{n^2+2n}\) dont une est toujours négative, et l'autre plus grande que \( 2n\).
\end{remark}    

Vous en voulez encore une ?

\begin{proposition}
    La suite donnée par
    \begin{equation}
        x_n=1+\frac{ 1 }{ 1! }+\cdots +\frac{1}{ n! }
    \end{equation}
    est de Cauchy et ne converge pas dans \( \eQ\).
\end{proposition}

\begin{proof}
    Si \( p>q\) nous avons
    \begin{subequations}
        \begin{align}
            x_p-x_q&=\sum_{k=q+^p}\frac{1}{ k! }\\
            &=\sum_{k=q+1}^p\frac{1}{ (q+1)! }\frac{1}{ (q+1)^{k-q-1} }\\
            &\leq \frac{1}{ (q+1)! }\sum_{k=0}^{\infty}\frac{1}{ (q+1)^k }\\
            &=\frac{1}{ (q+1)! }\frac{1}{ 1-\frac{1}{ q+1 } }\\
            &=\frac{1}{ q!q }.
        \end{align}
    \end{subequations}
    Soit \( \epsilon\in\eQ\) s'écrivant \( \epsilon=\frac{ a }{ b }\) avec \( a,b\in \eN\). En prenant \( p,q>b\), nous avons
    \begin{equation}
        x_p-x_q\leq \frac{1}{ b!b }<\frac{1}{ b }<\frac{ a }{ b }=\epsilon.
    \end{equation}
    Donc oui, cette suite est de Cauchy. 

    Montrons que cette suite ne converge pas dans \( \eQ\) en supposant qu'elle y converge : \( \lim_{n\to \infty} x_n=\frac{ a }{b }\in \eQ\). Pour tout \( p>q\) nous avons déjà vu avoir
    \begin{equation}
        0<x_p-x_q<\frac{1}{ qq! }.
    \end{equation}
    Prenons la limite \( p\to \infty\); par stricte croissance de la suite, les inégalités restent strictes :
    \begin{equation}        \label{EqQLCTooOgQOdh}
        0<\frac{ a }{ b }-x_q<\frac{1}{ qq! }.
    \end{equation}
    Si \( n>b\) alors nous pouvons écrire
    \begin{equation}
        \frac{ a }{ b }-x_n=\frac{ \alpha }{ n! }
    \end{equation}
    avec \( \alpha\in \eZ\) parce que le dénominateur commun entre \( \frac{ a }{ b }\) et \( x_n\) est dans \( n!\). En prenant donc \( q>n\) dans \eqref{EqQLCTooOgQOdh} nous pouvons écrire
    \begin{equation}
        0<\frac{ \alpha }{ q! }<\frac{1}{ qq! },
    \end{equation}
    c'est à dire \( 0<\alpha<\frac{1}{ q }\), ce qui est impossible pour \( \alpha\in \eZ\).
\end{proof}

%+++++++++++++++++++++++++++++++++++++++++++++++++++++++++++++++++++++++++++++++++++++++++++++++++++++++++++++++++++++++++++ 
\section{Les réels}
%+++++++++++++++++++++++++++++++++++++++++++++++++++++++++++++++++++++++++++++++++++++++++++++++++++++++++++++++++++++++++++

Une construction des réels via les coupures de Dedekind est donnée dans \cite{PaulinTopGmVegN}.

\begin{normaltext}      \label{NormooHRDZooRGGtCd}
    La construction des réels va nécessiter un petit «\wikipedia{fr}{bootstrap}{bootstrap}» au niveau de la topologie. En effet la notion de suite de Cauchy est une notion topologique (définition \ref{DefZSnlbPc}) alors que la topologie métrique (celle entre autres de \( \eQ\)) ne sera définie que par le théorème \ref{ThoORdLYUu}. Nous avons donc dû définir en la définition \ref{DefKCGBooLRNdJf} \emph{ex nihilo} les notions de
\begin{itemize}
    \item
        suite de Cauchy 
    \item
        suite convergente
    \item
        complétude
\end{itemize}
Nous allons ensuite construire \( \eR\) comme ensemble de classes d'équivalence de suites de Cauchy dans \( \eQ\). Ce ne sera que plus tard, après avoir définit la topologie métrique que nous allons voir que sur \( \eR\), ces trois notions coïncident avec celles topologiques\footnote{Proposition \ref{PropooUEEOooLeIImr}.}. Et par conséquent que \( \eR\) sera un espace topologique complet\footnote{Théorème \ref{ThoTFGioqS}.}.
% position 11144-30436
% position 13984-18006

Dans cette optique, il est intéressant de lire ce que dit Wikipédia à propos des suites de Cauchy dans \wikipedia{fr}{Construction_des_nombres_réels}{l'article consacré} à la construction des nombres réels.
\end{normaltext}

%--------------------------------------------------------------------------------------------------------------------------- 
\subsection{L'ensemble}
%---------------------------------------------------------------------------------------------------------------------------

Soit \( \modE\) l'ensemble des suites de Cauchy\footnote{Définition \ref{DefKCGBooLRNdJf}\ref{ItemVXOZooTYpcYN}} dans \( \eQ\). Soit aussi l'ensemble \( \modE_0\) constituée des suites qui convergent vers zéro\footnote{Nous rappelons qu'à ce niveau nous n'avons pas encore prouvé que toutes les suites de Cauchy convergent.}.

En posant 
\begin{equation}
    x+y=(x_n+y_n)
\end{equation}
et
\begin{equation}
    xy=(x_ny_n),
\end{equation}
l'ensemble \( \modE\) devient un anneau\footnote{Définition \ref{DefHXJUooKoovob}.} commutatif dont le neutre de l'addition est la suite constante \( x_n=0\) et le neutre pour la multiplication est la suite constante \( x_n=1\).

\begin{proposition}
    La partie \( \modE_0\) est un idéal\footnote{Définition \ref{DefooQULAooREUIU}.} de l'anneau \( \modE\).
\end{proposition}

\begin{proof}
    Nous savons par la proposition \ref{PropFFDJooAapQlP}\ref{ItemRKCIooJguHdji} que les suites convergentes sont de Cauchy; par conséquent \( \modE_0\subset\modE\).

    L'ensemble structuré \( (\modE_0,+)\) est un sous-groupe de \( \modE\) par les propriétés de la proposition \ref{PropFFDJooAapQlP} (il s'agit du fait que la somme de deux suites convergent vers zéro est une suite convergente vers zéro).

    En ce qui concerne la propriété fondamentale des idéaux, si \( x\in\modE_0\) et \( y\in\modE\) nous devons prouver que \( xy\in \modE_0\). Vu que \( (\modE_0,\cdot)\) est commutatif, cela suffira pour être un idéal bilatère. Vu que \( y\) est une suite de Cauchy, elle est bornée; et étant donné que \( x\to 0\) nous avons alors \( xy\to 0\) (par la proposition \ref{PropFFDJooAapQlP}\ref{ItemRKCIooJguHdjiii}).
\end{proof}

\begin{theoremDef}[L'anneau des réels\cite{RWWJooJdjxEK}]       \label{DefooFKYKooOngSCB}
    Sur l'ensemble quotient \( \modE/\modE_0\), les opérations
    \begin{enumerate}
        \item
            \( \bar u+\bar v=\overline{ u+v }\)
        \item
            \( \bar u\cdot \bar v=\overline{ uv }\)
    \end{enumerate}
    sont bien définies et donnent à \( \modE/\modE_0\) une structure de corps commutatif appelé \defe{corps des réels}{réel} et noté \( \eR\)\nomenclature[Y]{\( \eR\)}{l'ensemble des réels}
\end{theoremDef}
\index{construction!des réels}

\begin{proof}
    Nous divisons la preuve en plusieurs parties.
    \begin{subproof}
    \item[Les opérations sont bien définies]
    Si \( u,v\in \modE\) et \( h,k\in \modE_0\) alors \( h+k\in\modE_0\) et nous avons 
    \begin{equation}
        \overline{ u+h }+\overline{ v+k }=\overline{ u+v+h+k }=\overline{ u+v }.
    \end{equation}
    ainsi que
    \begin{equation}
        \overline{ u+h }\cdot \overline{ v+k }=\overline{ (u+h)(v+k) }=\overline{ uv+uk+hv+hk }=\overline{ uv }
    \end{equation}
    parce que les suites des Cauchy \( u\) et \( v\) sont bornées, ce qui donne que \( uk\), \( hv\) et \( hk\) sont des éléments de \( \modE_0\) par la proposition \ref{PropFFDJooAapQlP}\ref{ItemRKCIooJguHdjiii}. Cela prouver que les définitions proposées sont bonnes.

    \item[Caractérisation des classes]
        Soit \( q\in \eQ\) et une suite \( x\) convergente vers \( q\). Cette suite est de Cauchy comme toute suite convergente. Montrons que
        \begin{equation}
            \bar x=\{ \text{suites qui convergent vers } q \}.
        \end{equation}
        Si \( y\in\bar x\) alors \( y=x+h\) avec \( h\in \modE_0\) alors \( y_n\to q\) parce que \( h_n\to 0\). Réciproquement si \( y_n\to q\) alors pour chaque \( n\) nous avons
        \begin{equation}
            y_n=x_n+(y_n-x_n),
        \end{equation}
        mais \( y-x\to 0\). Donc \( y-x\in\modE_0\) ce qui signifie que \( y\in\bar x\).
    \item[Neutre et unité]
        Il est vite vérifié que \( \bar 0\) et \( \bar 1\), les classes des suites constantes sont neutre pour l'addition et unité pour la multiplication.
    \item[Corps]
        Nous devons prouver que tout élément non nul est inversible. C'est à dire que si \( x\in\modE\) ne converge pas vers zéro\footnote{\( x\in\modE\) peut soit ne pas converger du tout, soit converger vers autre chose que zéro.} alors il existe \( y\in \modE\) tel que \( xy\in\bar 1\). 

        Nous savons par la proposition \ref{PropFFDJooAapQlP}\ref{ItemRKCIooJguHdjvi} que \( x\) étant une suite de Cauchy dans \( \eQ\), il existe \( n_0\in \eN\) tel que \( \left( \frac{1}{ x_n } \right)_{n\geq n_0}\) est une suite de Cauchy. Nous posons alors
        \begin{equation}
            y_n=\begin{cases}
                0    &   \text{si } n\leq n_0\\
                \frac{1}{ x_n }    &    \text{si } n>n_0.
            \end{cases}
        \end{equation}
        Nous avons alors
        \begin{equation}
            (xy)_n=\begin{cases}
                0    &   \text{si } n\leq n_0\\
                1    &    \text{si } n>n_0
            \end{cases}
        \end{equation}
        et donc \( xy\in\bar 1\).
    \end{subproof}
\end{proof}

\begin{proposition}     \label{PropooEPFCooMtDOfP}
    Soit l'application
    \begin{equation}
        \begin{aligned}
            \varphi\colon \eQ&\to \eR \\
            q&\mapsto \bar q .
        \end{aligned}
    \end{equation}
    où par \( \bar q\) nous entendons la classe de la suite constante égale à \( q\) (qui est de Cauchy).
    \begin{enumerate}
        \item
            C'est un homomorphisme injectif.
        \item
            \( \Image(\varphi)\) est un sous-corps de \( \eR\)
        \item
            \( \varphi\colon \eQ\to \Image(\varphi)\) est un isomorphisme de corps.
    \end{enumerate}
\end{proposition}

\begin{proof}
    Le fait que ce soit un homomorphisme est simplement 
    \begin{itemize}
        \item \( \varphi(q+q')=\overline{ q+q' }=\bar q+\overline{ q' }\)
        \item \( \varphi(qq')=\overline{ qq' }=\overline{ q }\overline{ q' }\).
    \end{itemize}
    En ce qui concerne l'injectivité, si \( q\) est tel que \( \varphi(q)=\bar 0=\modE_0\), c'est que
    \begin{equation}
        \varphi(q)=\{ \text{suites de Cauchy qui convergent vers zéro} \}
    \end{equation}
    Mais nous savons aussi que\footnote{Voir dans la démonstration du théorème \ref{DefooFKYKooOngSCB}.}
    \begin{equation}
        \varphi(q)=\bar q=\{ \text{suites de Cauchy qui convergent vers } q \}
    \end{equation}
    Nous en déduisons que \( q=0\).
\end{proof}
Lorsque dans la suite nous parlerons d'un élément de \( \eQ\) comme étant un réel, nous aurons en tête l'image de cet élément par \( \varphi\).

%--------------------------------------------------------------------------------------------------------------------------- 
\subsection{Relation d'ordre}
%---------------------------------------------------------------------------------------------------------------------------

Nous définissons les parties \( \modE^+\) et \( \modE^-\) de \( \modE\) par
\begin{enumerate}
    \item
        \( x\in  \modE^+\) si et seulement si pour tout \( \epsilon>0\), il existe \( N_{\epsilon}\) tel que \( n>N_{\epsilon}\) implique \( x_n>-\epsilon\).
    \item
        \( x\in  \modE^-\) si et seulement si pour tout \( \epsilon>0\), il existe \( N_{\epsilon}\) tel que \( n>N_{\epsilon}\) implique \( x_n<\epsilon\).
\end{enumerate}
Nous notons aussi \( \modE^{++}=\modE^+\setminus\modE_0\).

\begin{lemma}
    Les parties \( \modE^+\) et \( \modE^-\) partitionnent \( \modE\) de la façon suivante :
    \begin{enumerate}
        \item
            \( \modE^+\cap\modE^-=\modE_0\)
        \item
            \( \modE^+\cup\modE^-=\modE\)
    \end{enumerate}
\end{lemma}

\begin{proof}
    Soit \( \epsilon>0\) et \( x\in \modE^+\cap\modE^-\). Il existe un \( N\in \eN\) tel que \( x_n>-\epsilon\) et \( x_n<\epsilon\) pour tout \( n\geq N\). Par conséquent, \( | x_n |\leq \epsilon\) pour tout \( n\geq N\) et la suite \( x\) converge vers zéro, c'est à dire \( x\in\modE_0\). Cela prouve que \( \modE^+\cap\modE^-\subset\modE_0\). L'inclusion inverse est évidente.

    Soit \( x\in \modE\setminus\modE^-\) et prouvons que \( x\in\modE^+\). La condition \( x\notin \modE^-\) donne qu'il existe un \( \alpha>0\) (dans \( \eQ\)) tel que pour tout \( n\), il existe \( p>n\) avec \( x_p>\alpha\). Mais \( x\) est une suite de Cauchy, donc nous avons un \( n_0\) tel que si \( n,p\geq n_0\) alors \( | x_n-x_p |\leq \frac{ \alpha }{2}\). Écrivons cette dernière inégalité avec \( p>n_0\) tel que \( x_p>\alpha\) et \( n>n_0\) :
    \begin{equation}
        x_n>\frac{ \alpha }{2}>0
    \end{equation}
    Par conséquent \( x\in\modE^+\) parce que \( x\in\modE\) et est positif à partir d'un certain rang.
\end{proof}

\begin{lemma}[\cite{RWWJooJdjxEK}]
    Quelques propriétés du partitionnement.
    \begin{enumerate}
        \item
            \( x\in\modE^-\) si et seulement si \( (-x)\in\modE^+\)
        \item
            \( x\in\modE^+\) et \( y\in\modE^+\) implique \( x+y\in\modE^+\)
        \item
            \( x\in\modE^+\) et \( y\in\modE^+\) implique \( xy\in\modE^+\)
        \item
            Si \( x,y\in\modE\) sont tels que \( x-y\in\modE_0\) alors soit \( x,y\in\modE^+\) soit \( x,y\in\modE^-\).
    \end{enumerate}
\end{lemma}

\begin{proof}
    Point par point.
    \begin{enumerate}
        \item
            Definition de \( \modE^+\) et \( \modE^-\).
        \item
            Pour \( n\geq N_{\epsilon/2}\) nous avons \( x_n>-\epsilon/2\) et \( y_n>-\epsilon/2\). Donc \( x_n+y_n>-\epsilon\).
        \item
            Si \( x\) ou \( y\) est dans \( \modE_0\) alors \( xy\in\modE_0\) et c'est bon. Si par contre \( x,t\in\modE^{++}\) alors pour \( n\) suffisamment grand, \( x_n>0\) et \( y_n>0\). Et dans ce cas, \( (xy)_n> 0\), c'est à dire \( xy\in\modE^+\).
        \item
            Supposons que \( x-y\in\modE_0\) avec \( x\in\modE^+\) et prouvons qu'alors \( y\in\modE^+\). Soit donc \( \epsilon>0\); il existe \( n_1\) tel que \( x_n>-\frac{ \epsilon }{2}\) dès que \( n\geq n_1\). Mais \( x-y\in\modE_0\), donc il existe \( n_2\) tel que \( | x_n-y_n |<\frac{ \epsilon }{2}\) dès que \( n\geq n_2\). En prenant \( n\) plus grand que \( n_1\) et \( n_2\), nous avons en même temps
            \begin{subequations}
                \begin{numcases}{}
                    x_n>-\frac{ \epsilon }{2}\\
                    | x_n-y_n |<\frac{ \epsilon }{2}.
                \end{numcases}
            \end{subequations}
            Cela implique que \( y_n>-\epsilon\) et donc que \( y\in\modE^+\).

            Nous pouvons de même prouver que si \( x\in\modE^-\) alors \( y\in\modE^-\).
    \end{enumerate}
\end{proof}

\begin{definition}[Positivité dans \( \eR\)]        \label{DefooLMQIooTgzZXd}
    Vocabulaire et notations.
    \begin{enumerate}
        \item
            Nous notons \( \eR^+=\modE^+\).\nomenclature[Y]{\( \eR^+\)}{les réels positifs ou nuls}
        \item
            Nous notons \( \eR^-=\modE^-\).
        \item
            Un élément de \( \eR\) est \defe{positif}{positif} s'il est la classe d'une suite de Cauchy appartenant à \( \modE^+\).
        \item
            Un élément de \( \eR\) est \defe{négatif}{négatif} s'il est la classe d'une suite de Cauchy appartenant à \( \modE^-\).
    \end{enumerate}
\end{definition}

\begin{remark}      \label{REMooOCXLooKQrDoq}
    Avec les conventions de la définition \ref{DefooLMQIooTgzZXd}, et en anticipant sur nos connaissances à propos des réels,
    \begin{enumerate}
        \item
            zéro est positif et négatif.
        \item
            L'intersection entre \( \eR^+\) et \( \eR^-\) est le singleton \( \{ 0 \}\).
        \item
            L'ensemble des nombres \emph{strictement} positifs est noté \( (\eR^+)^*\) ou \( \eR^+\setminus\{ 0 \}\).
        \item
            Le mot «positif» signifie «positif ou nul»; le mot «négatif» signifie «négatif ou nul».
    \end{enumerate}
\end{remark}

\begin{probleme}
    Une inconsistance dans les notations et le vocabulaire en ce qui concerne \( \eR^+\), \( \eR^-\) et le fait que zéro soit dedans ou non  m'a été signalée. Nous avons tranché pour les conventions données dans la remarque \ref{REMooOCXLooKQrDoq} qui sont aussi celles de \href{https://fr.wikipedia.org/wiki/Nombre_positif}{  Wikipédia}, mais il n'est pas exclu qu'il y ait des incohérences par-ci par-là dans le texte. Merci de me les signaler. 
\end{probleme}

\begin{definition}[Ordre dans \( \eR\)]     \label{DefooYALBooHSXZqB}
    Si \( a,b\in \eR\) nous notons \( a\leq b\) si et seulement si \( b-a\) est positif. Nous notons aussi \( a>b\) si et seulement si \( b-a\in \eR^+\setminus\{ 0 \}\), etc.
\end{definition}

\begin{lemma}
    Les premières propriétés de l'ordre.
    \begin{enumerate}
        \item
            L'ensemble \( (\eR,\leq)\) est un corps totalement ordonné (définitions \ref{DefooFLYOooRaGYRk} et \ref{DefKCGBooLRNdJf}).
        \item
            L'application 
            \begin{equation}
                \begin{aligned}
                    \varphi\colon \eQ&\to \eR \\
                    q&\mapsto \bar q 
                \end{aligned}
            \end{equation}
            dont nous avons déjà parlé dans la proposition \ref{PropooEPFCooMtDOfP} est strictement croissante.
    \end{enumerate}
\end{lemma}

\begin{proof}
    Nous prouvons la stricte croissance de \( \varphi\). Si \( q< l\) alors \( \varphi(q)-\varphi(l)=\overline{ q-l }\) est la classe de la suite constante \( q-l\) qui est un élément strictement positif de \( \eQ\). Nous avons donc \( \overline{ q-l }\in \eR^+\), et donc \( \varphi(q)<\varphi(l)\).
\end{proof}

\begin{remark}
    Comme déjà mentionné plus haut, à chaque fois que nous parlerons d'un élément de \( \eQ\) comme étant un élément de \( \eR\), nous considérons la classe de la suite constante.
\end{remark}

\begin{lemma}       \label{LemooYNOVooOwoRwD}
    Si \( x,y,z\in \eR\) avec \( y>0\) sont tels que \( z>x/y\) alors \( zy>x\).
\end{lemma}

\begin{proof}
    Nous savons que 
    \begin{equation}
        z-\frac{ x }{ y }\in \modE^+\setminus\{ 0 \}. 
    \end{equation}
    Vu que \( y\in\modE^+\), multiplier par \( y\) fait rester dans \( \modE^{++}\) :
    \begin{equation}
        zy=y\frac{ x }{ y }\in \modE^{++}.
    \end{equation}
    Un représentant de \( y\frac{ x }{ y }\) est la suite \( n\mapsto y_n\frac{ x_n }{ y_n }=x_n\). Donc \( y\frac{ x }{ y }=x\). Cela signifie que \( zy-x\in\modE^{++}\) et donc que \( zy>x\).
\end{proof}

\begin{lemma}       \label{LemooMWOUooVWgaEi}
    Pour tout \( a\in \eR\), il existe \( p\in \eN\) tel que \( p>a\). 
\end{lemma}

\begin{proof}
    Nous allons donner deux preuves différentes de ce lemme.
    \begin{subproof}
    \item[Première façon]
        
    L'élément \( a\) de \( \eR\) admet un représentant \( (a_n)\) qui est une suite de Cauchy dans \( \eQ\). C'est donc une suite bornée supérieurement, c'est à dire qu'il existe \( m,q\in \eN\) tels que \( | a_n |\leq m/q\) pour tout \( n\) (proposition \ref{PropFFDJooAapQlP}\ref{ItemRKCIooJguHdjii}). Soit \( M\) un naturel strictement plus grand que \( m/q\). 

    La suite de Cauchy \( (M-a_n)_{n\in \eN}\) est constituée de rationnels positifs et est donc dans \( \modE^+\). La classe de \( M-a\) est donc un réel positif\footnote{Et nous allons d'ailleurs arrêter de toujours préciser «la classe de» lorsque ce n'est pas nécessaire.}. Par définition de la relation d'ordre, \( M\geq a\).
\item[Seconde façon]

    La suite \( (a_n)\) est majorée par \( \frac{ m }{ q }\), donc on a dans \( \eQ\) et pour tout \( n\) :
    \begin{equation}
        a_n\leq \frac{ m }{ q }=M\leq qM.
    \end{equation}
    L'application \( \varphi\colon \eQ\to \eR\) est croissante, donc
    \begin{equation}
        \varphi\big( (a_n) \big)\leq \varphi(qM).
    \end{equation}
    \end{subproof}
\end{proof}

\begin{theorem}[\cite{RWWJooJdjxEK}]        \label{ThoooKJTTooCaxEny}
    Le corps \( \eR\) est archimédien\footnote{Définition \ref{DefKCGBooLRNdJf}\ref{ItemooDZQKooPsqeRf}.}.
\end{theorem}

\begin{proof}
    Soient \( x,y\in \eR\) avec \( x>0\) et trouvons \( s\in \eN\) tel que \( px>y\). Nous posons \( a=\frac{ y }{ x }\) et le lemme \ref{LemooMWOUooVWgaEi} nous donne un \( p>a\) dans \( \eN\). Par le lemme \ref{LemooYNOVooOwoRwD}.
\end{proof}

Le lemme suivant n'est pas loin de dire que \( \eQ\) est dense dans \( \eR\), à part que nous n'avons pas encore donné de topologie sur \( \eR\).
\begin{lemma}       \label{LemooHLHTooTyCZYL}
    Si \( x,y\in \eR\) sont tels que \( x<y\), il existe \( s\in \eQ\) tel que \( x<s<y\)
\end{lemma}

\begin{proof}
    Nous avons par hypothèse que \( y-x>0\) et donc le fait que \( \eR\) soit archimédien (théorème \ref{ThoooKJTTooCaxEny}) nous donne \( q\in \eN\) tel que \( q(y-x)>1\). Soit
    \begin{equation}
        E=\{ n\in \eZ\tq \frac{ n }{ q }\leq x \}.
    \end{equation}
    Cet ensemble n'est pas vide parce que \( 1/q>0\), ce qui donne \( n_0\in \eN\) tel que \( n_0/q\geq | x |\). Dans ce cas, \( -n_0\in E\). De plus \( E\) est majoré par \( n_0\). Donc \( E\) possède un plus grand élément \( p\) qui vérifie
    \begin{equation}
        \frac{ p }{ q }\leq x<\frac{ p+1 }{ q }.
    \end{equation}
    De plus \( (p+1)/q<y\). En effet nous avons
    \begin{equation}
        \frac{ p+1 }{ q }=\frac{ p }{ q }+\frac{1}{ q }\leq x+\frac{1}{ q }<x+y-x=y
    \end{equation}
    où nous avons utilisé l'inégalité stricte \( y-x>\frac{1}{ q }\).

    Le nombre \( s\in \eQ\) que nous recherchions est \( s=\frac{ p+1 }{ q }\).
\end{proof}

\begin{remark}
    Le lemme \ref{LemooHLHTooTyCZYL} a également pour conséquence que des ensembles comme \( \mathopen[ -1 , 1 \mathclose]\) ne sont pas bien ordonnés. En effet la partir \( \mathopen] 0 , 1 \mathclose[\) ne possède pas de minimum parce que si \( x\in \mathopen] 0 , 1 \mathclose[\) alors \( 0<x\) et il existe \( s\in \eQ\) (a fortiori \( s\in \eR\)) tel que \( 0<s<x\), c'est à dire que \( x\) n'est pas un minimum de \( \mathopen] 0 , 1 \mathclose[\).
\end{remark}

Nous avons terminé avec la construction des réels. Les propriétés topologiques arrivent en \ref{SECooGKHYooMwHQaD}. En particulier, le fait que \( \eR\) soit complet est le théorème \ref{ThoTFGioqS}.

%+++++++++++++++++++++++++++++++++++++++++++++++++++++++++++++++++++++++++++++++++++++++++++++++++++++++++++++++++++++++++++ 
\section{Les complexes}
%+++++++++++++++++++++++++++++++++++++++++++++++++++++++++++++++++++++++++++++++++++++++++++++++++++++++++++++++++++++++++++

 \subsection{Définitions}
 Un nombre complexe s'écrit sous la forme $z = a + b i$, où $a$ et $b$
 sont des nombres réels appelés (et notés) respectivement partie réelle
 ($a = \Re(z)$) et partie imaginaire ($b = \Im(z)$) de $z$. L'ensemble
 des nombres de cette forme s'appelle l'ensemble des nombres complexes
 ; cet ensemble porte une structure de corps et est noté $\eC$. Le
 nombre complexe $i = 0 + 1 i$ est un nombre imaginaire qui a la
 particularité que $i^2 = -1$.

 Deux nombres complexes $a + bi$ et $c + di$ sont égaux si et seulement
 si $a = c$ et $b = d$, c'est-à-dire leurs parties réelles sont égales,
 et leurs parties imaginaires sont égales.

 Un nombre complexe étant représenté par deux nombres, on peut le
 représenter dans un plan appelé « plan de Gauss ». La plupart des
 opérations sur les nombres complexes ont leur interprétation
 géométrique dans ce plan.

 Pour $z = a + bi$ un nombre complexe, on note $\bar z = a - bi$ le
 \Defn{complexe conjugué} de $z$. Dans le plan de Gauss, il s'agit du
 symétrique de $z$ par rapport à la droite réelle (généralement
 dessinée horizontalement).

 On définit le module du complexe $z$ par $\module z = \sqrt{z\bar z} =
 \sqrt{a^2 + b^2}$. Dans le plan de Gauss, il s'agit de la distance
 entre $0$ et $z$.

 \begin{proposition}
Pour tout $z = a+bi$ et $z^\prime$ nombres complexes, on a
   \begin{enumerate}
   \item $z \bar z = a^2 + b^2$;
   \item $\bar{\bar{z}} = z$;
   \item $\module z = \module {\bar z}$;
   \item $\module{zz^\prime} = \module z \module{z^\prime}$;
   \item $\module{z+z^\prime} \leq \module z + \module{z^\prime}$.
   \end{enumerate}
 \end{proposition}
