% This is part of Mes notes de mathématique
% Copyright (c) 2008-2018
%   Laurent Claessens, Carlotta Donadello
% See the file fdl-1.3.txt for copying conditions.

%+++++++++++++++++++++++++++++++++++++++++++++++++++++++++++++++++++++++++++++++++++++++++++++++++++++++++++++++++++++++++++
\section{Topologie en général}
%+++++++++++++++++++++++++++++++++++++++++++++++++++++++++++++++++++++++++++++++++++++++++++++++++++++++++++++++++++++++++++

%---------------------------------------------------------------------------------------------------------------------------
\subsection{Définitions basiques}
%---------------------------------------------------------------------------------------------------------------------------

\begin{definition}		\label{DefTopologieGene}
Soit $E$, un ensemble et $\mT$, une partie de l'ensemble de ses parties qui vérifie les propriétés suivantes
\begin{enumerate}
\item
les ensembles $\emptyset$ et $E$ sont dans $\mT$,
\item
    Une union quelconque\footnote{Par «quelconque» nous entendons vraiment quelconque : c'est à dire indicée par un ensemble qui peut autant être \( \eN\) que \( \eR\)
    qu'un ensemble encore considérablement plus grand.} d'éléments de \( \mT\) est dans \( \mT\).
\item
    Une intersection \emph{finie} d'éléments de \( \mT\) est dans \( \mT\).

\end{enumerate}
Un tel choix $\mT$ de sous-ensembles de $E$ est une  \defe{\href{http://fr.wikipedia.org/wiki/Espace_topologique}{topologie}}{topologie} sur $E$, et les éléments de $\mT$ sont appelés des \defe{ouverts}{ouvert}. Nous disons que un sous-ensemble $A$ de $E$ est \defe{fermé}{fermé} si son complémentaire, $A^c$ est ouvert.
\end{definition}

\begin{lemma}   \label{LemQYUJwPC}
    Union et intersection de fermés.
    \begin{enumerate}
        \item
            Une intersection quelconque de fermés est fermée.
        \item       \label{ItemKJYVooMBmMbG}
            Une union finie de fermés est fermée.
    \end{enumerate}
\end{lemma}

\begin{proof}
    Soit \( \{ F_i \}_{i\in I} \) est un ensemble de fermés; nous avons
    \begin{equation}
        \left( \bigcap_{i\in I}F_i \right)^c=\bigcup_{i\in I}F_i^c.
    \end{equation}
    L'union de droite est un ouvert (union d'ouvert), et donc le terme de gauche est le complémentaire d'un ouvert. Donc fermé.

    Le complémentaire de l'union finie est une intersection finie de complémentaires et est donc ouvert.
\end{proof}

Dans un espace topologique, nous avons une caractérisation très importante des ouverts.
\begin{theorem}		\label{ThoPartieOUvpartouv}
    Une partie d'un espace topologique est ouverte si et seulement si elle contient un voisinage ouvert de chacun de ses éléments.
\end{theorem}

\begin{proof}
    Soit \( X\) un espace topologique et \( A\subset X\). Le sens direct est évident : $A$ lui-même est un ouvert autour de $x\in A$, qui est inclus dans $A$.

Pour le sens inverse, nous supposons que \( A\) contienne un ouvert autour de chacun de ses points. Pour chaque $x\in A$, nous considérons l'ensemble $\mO_x\subset A$, un ouvert autour de $x$. Nous avons que
\begin{equation}	\label{EqAUniondesOx}
	A=\bigcup_{x\in A}\mO_x.
\end{equation}
En effet $A\subset\bigcup_{x\in A}\mO_x$ parce que tous les éléments de $A$ sont dans un des $\mO_x$, par construction. D'autre part, $\bigcup_{x\in A}\mO_x\subset A$ parce que chacun des $\mO_x$ est compris dans $A$.

L'union du membre de droite de \eqref{EqAUniondesOx} est une union d'ouverts et est donc un ouvert. Cela prouve que $A$ est un ouvert.

\end{proof}
Une utilisation typique de ce théorème est faite dans le lemme~\ref{LemMESSExh}.

%---------------------------------------------------------------------------------------------------------------------------
\subsection{Adhérence, fermeture}
%---------------------------------------------------------------------------------------------------------------------------

Disons-le tout de suite : «adhérence» et «fermeture» sont synonymes.

\begin{definition}      \label{DEFooSVWMooLpAVZR}
    Soient un espace topologique \( X\) et une partie \( A\) de \( X\).
    \begin{enumerate}
        \item
            Un point \( x\in X\) est \defe{adhérent}{point adhérent} à \( A\) si tout ouvert de \( X\) contenant \( x\) a une intersection non vide avec \( A\). L'ensemble des points d'adhérence de \( A\) est noté $\Adh(A)$.\nomenclature[T]{$\Adh(A)$}{adhérence de \( A\)}
        \item
            L'\defe{adhérence}{adhérence} de \( A\), notée \( \bar A\), est l'intersection de tous les fermés de \( X\) contenant \( A\).
    \end{enumerate}
\end{definition}

\begin{lemma}       \label{LEMooILNCooOFZaTe}
    L'adhérence de \( A\) est l'ensemble des points adhérents :
    \begin{equation}
        \Adh(A)=\bar A.
    \end{equation}
\end{lemma}

\begin{proof}
    Deux inclusions à prouver, que nous allons toutes deux démontrer par contraposée.
    \begin{subproof}
        \item[Si \( x\in \bar A\) alors \( x\in\Adh(A)\)]
            Si \( x\) n'est pas dans \( \bar A\) alors nous avons un fermé \( F\) contenant \( A\) et pas \( x\). Le complémentaire \( F^c\) est un ouvert qui contient \( x\) et dont l'intersection avec \( A\) est vide. Donc \( x\) n'est pas dans \( \Adh(A)\).

        \item[Si \( x\in\bar A\) alors \( x\in \Adh(A)\)]

            Si \( x\) n'est pas dans \( \Adh(A)\) alors il existe un ouvert \( \mO\) contenant \( x\) et n'intersectant pas \( A\). Le complémentaire \( \mO^c\) est un fermé qui contient \( A\) et qui ne contient pas \( x\).

            Vu que \( \bar A\) est l'intersection de tous les fermés contenant \( A\), nous avons \( \bar A\subset\mO^c\) et donc \( x\) n'est pas dans \( \bar A\).
    \end{subproof}
\end{proof}

%---------------------------------------------------------------------------------------------------------------------------
\subsection{Convergence de suite}
%---------------------------------------------------------------------------------------------------------------------------

Dès que nous avons une topologie nous avons une notion de convergence.
\begin{definition}[Convergence de suite] \label{DefXSnbhZX}
    Une suite $(x_n)$ d'éléments de $E$ \defe{converge}{convergence!de suite} vers l'élément $y$ de $E$ si pour tout ouvert $\mO$ contenant $y$, il existe un $K\in \eN$ tel que $k>K$ implique $x_k\in\mO$.
\end{definition}
\index{limite!de suite!espace topologique}

%---------------------------------------------------------------------------------------------------------------------------
\subsection{Topologie produit}
%---------------------------------------------------------------------------------------------------------------------------

\begin{definition}[Produit d'espaces topologiques, thème~\ref{THEMEooYRIWooDXZnhX}]      \label{DefIINHooAAjTdY}
    Soient \( X_1\),\ldots, \( X_n\) des espaces topologiques. Leur \defe{produit}{produit!espaces topologiques}\index{topologie!produit} est l'ensemble
    \begin{equation}
        X=\prod_{i=1}^nX_i
    \end{equation}
    muni de la topologie suivante. L'ensemble \( U\) est ouvert si et seulement si pour tout \( x\in U\) il existe des ouverts \( A_i\in X_i \) tels que \( x\in A_1\times \cdots\times A_n\subset U\).
\end{definition}

\begin{lemma}[Application partielle\cite{MonCerveau}]       \label{LEMooHAODooYSPmvH}
    Soient trois espaces topologiques \( X_1\), \( X_2\) et \( Y\). Nous considérons une fonction continue \( f\colon X_1\times X_2\to Y\) ainsi que \( x_1\in X_1\). Alors l'application
    \begin{equation}
        \begin{aligned}
            g\colon X_2&\to Y \\
            x_2&\mapsto f(x_1,x_2)
        \end{aligned}
    \end{equation}
    est continue.
\end{lemma}

\begin{proof}
    Soit un ouvert \( \mO\) de \( Y\); par hypothèse sur \( f\), la partie \( f^{-1}(\mO)\) est ouverte dans \( X_1\times X_2\). Notre but est de prouver que \( g^{-1}(\mO)\) est un ouvert de \( X_2\). Nous avons :
    \begin{equation}
        g^{-1}(\mO)=\{ x_2\in X_2\tq (x_1,x_2)\in f^{-1}(\mO) \}.
    \end{equation}
    Nous considérons \( x_2\in g^{-1}(\mO)\) et nous prouvons qu'il existe dans \( X_2\) un voisinage de \( x_2\) entièrement contenu dans \( g^{-1}(\mO)\).

    Étant donné que \( (x_1,x_2)\) est dans \( f^{-1}(\mO)\) qui est ouvert, la définition~\ref{DefIINHooAAjTdY} de la topologie sur \( X_1\times X_2\) nous donne des ouverts \( A_1\) dans \( X_1\) et \( A_2\) dans \( X_2\) tels que
    \begin{equation}
        (x_1,x_2)\in A_1\times A_2\subset f^{-1}(\mO).
    \end{equation}

    Nous montrons à présent que \( A_2\subset g^{-1}(\mO)\). Soit \( y_2\in A_2\). Par construction \( (x_1,y_2)\in A_1\times A_2\subset f^{-1}(\mO)\), donc
    \begin{equation}
        g(y_2)=f(x_1,x_2)\in \mO.
    \end{equation}
    Cela termine la démonstration.
\end{proof}

\begin{proposition}[\cite{MonCerveau}]      \label{PROPooNRRIooCPesgO}
    La convergence d'une suite pour la topologie de l'espace produit implique la convergence des suites «composante par composante».
\end{proposition}

\begin{proof}
    Pour simplifier les notations, nous allons considérer le produit de deux espaces. Soit donc \( (x_k,y_k)\stackrel{X\times Y}{\longrightarrow}(x,y)\) et des ouverts \( \mO_1\) dans \( X\) autour de \( x\) et \( \mO_2\) autour de \( y\) dans \( Y\). La partie \( \mO_1\times \mO_2\) est ouverte dans \( X\times Y\). Donc il existe \( K\) tel que \( k>K\) implique \( (x_k,y_k)\in \mO_1\times \mO_2\).

    Nous avons prouvé que pour tout ouvert \( \mO_1\) autour de \( x\) il existe \( K\) tel que \( k>K\) implique \( x_k\in \mO_1\). Donc \( x_k\stackrel{X}{\longrightarrow}x\). Idem pour \( y\).
\end{proof}

%---------------------------------------------------------------------------------------------------------------------------
\subsection{Séparabilité}
%---------------------------------------------------------------------------------------------------------------------------

\begin{definition}[Espace séparé]  \label{DefYFmfjjm}
    Si deux points distincts ont toujours deux voisinages distincts, nous disons que l'espace est \defe{séparé}{séparé} ou \defe{Hausdorff}{Hausdorff}.
\end{definition}
La notion d'espace séparé est particulièrement importante parce qu'elle assure l'unicité de la limite d'une fonction en un point par la proposition~\ref{PropFObayrf}.

\begin{definition}  \label{DefUADooqilFK}
    Un espace topologique est \defe{séparable}{séparable!espace topologique} s'il possède une partie dénombrable dense.
\end{definition}
À ne pas confondre avec :
\begin{definition}  \label{DefWEOTrVl}
    Un espace topologique est \defe{séparé}{séparé!espace topologique} ou \defe{Hausdorff}{Hausdorff} si deux points distincts possèdent des voisinages disjoints.
\end{definition}

\begin{definition}  \label{DefJJVsEqs}
  Une partie $A$ d'un espace topologique est \defe{compacte}{compact} s'il vérifie la propriété de Borel-Lebesgue : pour tout recouvrement de $A$ par des ouverts (c'est-à-dire une collection d'ouverts dont la réunion contient $A$) on peut tirer un recouvrement fini.
\end{definition}
\begin{remark}
    Certaines sources (dont \wikipedia{fr}{Compacité_(mathématiques)}{wikipédia}) disent que pour être compact il faut aussi être séparé\footnote{Définition~\ref{DefWEOTrVl}.}. Pour ces sources, un espace qui ne vérifie que la propriété de Borel-Lebesgue est alors dit \defe{quasi-compact}{quasi-compact}\index{compact!quasi}.
\end{remark}

\begin{definition}
    Une partie d'un espace topologique est \defe{relativement compact}{compact!relativement}\index{relativement!compact} si son adhérence est compacte.
\end{definition}

\begin{definition}  \label{DefEIBYooAWoESf}
    Un espace topologique est \defe{localement compact}{compact!localement} si tout élément possède un voisinage compact.
\end{definition}

\begin{definition}[Séquentiellement compact]
    Nous disons qu'un espace topologique est \defe{séquentiellement compact}{compact!séquentiellement} si toute suite admet une sous-suite convergente.
\end{definition}

\begin{definition}      \label{DefFCGBooLpnSAK}
    Un espace topologique est \defe{dénombrable à l'infini}{dénombrable!à l'infini} s'il est réunion dénombrable de compacts.
\end{definition}

\begin{definition}[Base de topologie]   \label{DefQELfbBEyiB}
    Une famille \( \mB\) d'ouverts de \( X\) est une \defe{base de la topologie}{base!de topologie} de \( X\) si pour tout \( x\in X\) et pour tout voisinage \( V\) de \( x\), il existe \( A\in \mB\) tel que \( x\in A\subset V\).
\end{definition}

\begin{proposition} \label{PropMMKBjgY}
    Si \( \mB\) est une base de la topologie de \( X\) alors tout ouvert de \( X\) est une union d'éléments de \( \mB\).
\end{proposition}

\begin{proof}
    Soit \( \mO\) un ouvert de \( X\); pour chaque \( x\in\mO\) nous considérons un ouvert \( U(x)\) tel que \( x\in U(x)\subset \mO\) (possible par le théorème~\ref{ThoPartieOUvpartouv}). Nous prenons alors \( B(x)\in\mB\) tel que
    \begin{equation}
        x\in B(x)\subset U(x)\subset \mO.
    \end{equation}
    Alors nous avons \( \mO=\bigcup_{x\in \mO}B(x)\).
\end{proof}
Notons toutefois que nous sommes loin d'avoir une union dénombrable en général.

%---------------------------------------------------------------------------------------------------------------------------
\subsection{Topologie métrique}
%---------------------------------------------------------------------------------------------------------------------------

\begin{definition}  \label{DefMVNVFsX}
    Si $E$ est un ensemble, une \defe{distance}{distance} sur $E$ est une application $d\colon E\times E\to \eR$ telle que pour tout $x,y\in E$,
    \begin{enumerate}

    \item
    $d(x,y)\geq 0$

    \item
    $d(x,y)=0$ si et seulement si $x=y$,

    \item
    $d(x,y)=d(y,x)$

    \item
    $d(x,y)\leq d(x,z)+d(z,y)$.

    \end{enumerate}
    La dernière condition est l'\defe{inégalité triangulaire}{inégalité!triangulaire}.

    Un couple $(E,d)$ formé d'un ensemble et d'une distance est un \defe{espace métrique}{espace!métrique}.
\end{definition}

La définition-théorème suivante donne une topologie sur les espaces métriques en partant des boules. Par construction, les boules sont une base de la topologie\footnote{Définition~\ref{DefQELfbBEyiB}.} d'un espace métrique.

\begin{theoremDef}     \label{ThoORdLYUu}
    Soit \( (E,d)\) un espace métrique. Nous définissons les \defe{boules ouvertes}{boule!ouverte} par
    \begin{equation}        \label{EQooYCWSooIhibvd}
        B(x,r)=\{ y\in E\tq d(x,y)<r \}.
    \end{equation}
    Alors en posant
    \begin{equation}        \label{EqGDVVooDZfwSf}
        \mT=\big\{  \mO\subset E  \tq\forall x\in A,\exists r>0\tq B(x,r)\subset \mO \big\}
    \end{equation}
    nous définissons une topologie sur \( E\).

    Cette topologie sur \( E\) est la \defe{topologie métrique}{topologie!métrique} de \( (E,d)\). En présence d'une distance, sauf mention explicite du contraire, c'est toujours cette topologie-là que nous utiliserons.
\end{theoremDef}

\begin{proof}
    D'abord \( \emptyset\in\mT\) parce que tout élément de l'ensemble vide \ldots heu \ldots enfin parce que d'accord hein\footnote{Pour qui ne serait pas d'accord, allez ajouter \( \emptyset\) dans la définition des ouverts et puis c'est tout.}.

    Ensuite si \( (A_i)_{i\in I}\) sont des éléments de \( \mT\) et si \( x\in\bigcup_{i\in I}A_i\) alors il existe \( k\in I\) tel que \( x\in A_k\). Par hypothèse il existe une boule \( B(x,r)\subset A_k\subset\bigcup_{i\in I}A_i\).

    Enfin si \( (A_i)_{i\in\{ 1,\ldots, n \}}\) sont des éléments de \( \mT\) alors pour tout \( i\) il existe \( r_i>0\) tel que \( B(x,r_i)\subset A_i\). En prenant \( r=\min\{ r_i \}_{i=1,\ldots, n}\) nous avons $B(x,r)\subset\bigcap_{i=1}^nA_i.$
\end{proof}

\begin{remark}  \label{RemQDRooKnwKk}
    Deux remarques à propos de cette définition.
    \begin{enumerate}
        \item
    Cette définition est faite exprès pour respecter le théorème~\ref{ThoPartieOUvpartouv}.
\item
    Par construction, les boules ouvertes sont une base de la topologie (définition~\ref{DefQELfbBEyiB}) des espaces métriques.
\item       \label{ITEMooUIHJooXAFaJa}
    Si \( V\) est un voisinage de \( x\) (c'est à dire que \( V\) contient un ouvert contenant \( x\)) alors il existe\( r\) tel que \( B(x,r)\subset V\).
    \end{enumerate}
\end{remark}

\begin{normaltext}
    Si vous avez un peu de temps, vous pouvez vérifier que si \( \eK\) est un corps totalement ordonné, alors avec toutes les définitions de~\ref{DefKCGBooLRNdJf}, en posant \( d(x,y)=| x-y |\) nous avons une distance sur \( \eK\).

    De plus, les boules définies en~\ref{DefKCGBooLRNdJf} sont alors les mêmes que celles définies en \eqref{EQooYCWSooIhibvd}, ce qui donne à tout corps totalement ordonné une structure d'espace topologique.
\end{normaltext}

\begin{definition}
  Un sous-ensemble $A \subset \eR^n$ est \defe{borné}{borné} s'il existe une boule de $\eR^n$ contenant $A$.
\end{definition}

\begin{proposition}
  Toute réunion finie d'ensembles bornés est un ensemble borné. Toute partie d'un ensemble borné est un ensemble borné.
\end{proposition}

\begin{definition}
    Si \( (X,d_X)\) et \( (Y,d_Y)\) sont des espaces métriques, une \defe{isométrie}{isométrie d'espaces métriques} est une application bijective \( f\colon X\to Y\) telle que pour tout \( x,y\in X\) nous ayons
    \begin{equation}        \label{EQooVUOXooKJntMN}
        d_Y\big( f(x),f(y) \big)=d_X(x,y).
    \end{equation}
\end{definition}

\begin{remark}
    Une application vérifiant \eqref{EQooVUOXooKJntMN} est automatiquement injective. En pratique, il ne faut donc vérifier que la surjectivité.
\end{remark}

\begin{example}[Manque de surjectivité]
    Si \( X=\mathopen[ 0 , \infty \mathclose[\) et \( f(x)=x+1\) alors \( f\) vérifie \eqref{EQooVUOXooKJntMN} pour la distance \( d(x,y)=| x-y |\), mais n'est pas surjective.
\end{example}

\begin{example}
    Si \( X\) est un ensemble, nous pouvons écrire la \defe{distance discrète}{distance discrète} :
    \begin{equation}
        d(x,y)=\begin{cases}
            0    &   \text{si } x=y\\
            1    &    \text{si } x\neq y\text{.}
        \end{cases}
    \end{equation}
    La topologie résultante sera nommée la \defe{topologie discrète}{topologie discrète}.

    Pour cette métrique, le groupe des isométries est le groupe symétrique de \( X\), c'est à dire le groupe de toutes les bijections de \( X\) sur lui-même.
\end{example}

\begin{propositionDef}[Groupe des isométries]
    Si \( (X,d)\) est un espace métrique,
    \begin{enumerate}
        \item
            l'ensemble des isométries de \( X\), noté \( \Isom(X)\)\nomenclature[Y]{$\Isom(X)$}{Le groupe des isométries de \( X\)} est un groupe pour la composition\index{isométrie!groupe}\index{groupe!des isométries!espace métrique}.
        \item
            Ce groupe agit fidèlement sur \( X\).
    \end{enumerate}
\end{propositionDef}
Nous verrons plus loin (proposition~\ref{PropLYMgVMJ}) qu'une isométrie est toujours continue.

%---------------------------------------------------------------------------------------------------------------------------
\subsection{Espace vectoriel topologique}
%---------------------------------------------------------------------------------------------------------------------------

\begin{definition}
    Un espace vectoriel \( V\) sur le corps \( \eK\) muni d'une topologie est un \defe{espace vectoriel topologique}{espace vectoriel!topologique} si
    \begin{enumerate}
        \item
            La somme de deux vecteurs est une application continue \( V\times V\to V\)
        \item
            La multiplication par un scalaire est une application continue \( \eK\times V\to V\).
    \end{enumerate}
    Note : le corps\footnote{Définition~\ref{DefTMNooKXHUd}} lui-même doit avoir sa topologie; typiquement c'est \( \eR\) ou \( \eC\) muni de la topologie usuelle.
\end{definition}

\begin{definition}      \label{DEFooGTOZooRcvGHg}
    Une distance \( d\) sur un espace vectoriel topologique \( V\) est dite \defe{compatible}{distance!compatible} avec la topologie si la topologie induite\footnote{Définition~\ref{ThoORdLYUu}.} de \( d\) est celle de \( V\).

    Une distance \( d\) sur un espace vectoriel \( V\) est dite \defe{invariante}{distance!invariante} si pour tout \( x,y,u\in V\) nous avons
    \begin{equation}
        d(x+u,y+u)=d(x,y).
    \end{equation}
\end{definition}
Notons que lorsque nous parlons d'une distance compatible avec un espace vectoriel topologique, nous parlons de compatibilité avec la topologie, pas avec la structure vectorielle.

%+++++++++++++++++++++++++++++++++++++++++++++++++++++++++++++++++++++++++++++++++++++++++++++++++++++++++++++++++++++++++++
\section{Espace vectoriel normé}
%+++++++++++++++++++++++++++++++++++++++++++++++++++++++++++++++++++++++++++++++++++++++++++++++++++++++++++++++++++++++++++

%---------------------------------------------------------------------------------------------------------------------------
\subsection{Introduction : norme et valeur absolue}
%---------------------------------------------------------------------------------------------------------------------------

La valeur absolue est essentielle pour introduire les notions de limite et de continuité pour les fonctions d'une variable. En fait nous disons que la fonction $f\colon \eR\to \eR$ est continue au point $a$ lorsque pour tout $\varepsilon$, il existe un $\delta$ tel que
\begin{equation}
	| x-a |\leq\delta \Rightarrow | f(x)-f(a) |\leq \varepsilon.
\end{equation}
La quantité $| x-a |$ donne la «distance» entre $x$ et $a$; la définition de la continuité signifie que pour tout $\varepsilon$, il existe un $\delta$ tel que si $a$ et $x$ sont au plus à la distance $\delta$ l'un de l'autre, alors $f(x)$ et $f(a)$ ne seront éloigné au plus d'une distance $\varepsilon$.

La valeur absolue, dans $\eR$, nous sert donc à mesurer des distances entre les nombres. Les principales propriétés de la valeur absolue sont :
\begin{enumerate}

	\item
		$| x |=0$ implique $x=0$,
	\item
		$| \lambda x |=| \lambda | |x |$,
	\item
		$| x+y |\leq | x |+| y |$

\end{enumerate}
pour tout $x,y\in\eR$ et $\lambda\in\eR$.

Afin de donner une notion de limite pour les fonctions de plusieurs variables, nous devons trouver un moyen de définir les notions de <<taille>> d'un vecteur et de distance entre deux points de $\eR^n$, avec $n>1$. La notion de <<taille>> doit satisfaire propriétés analogues à celles de la valeur absolue.

La premier notion de «taille» pour un vecteur de $\eR^2$ que nous vient à l'esprit est la longueur du segment entre l'origine et l'extrémité libre du vecteur. Cela peut être calculée à l'aide du théorème de Pythagore :
\begin{equation}
  \textrm{taille de } (a,b) = \sqrt{a^2+b^2}.
\end{equation}
Nous pouvons introduire une notion de distance entre les éléments de $\eR^2$ de façon similaire :
\begin{equation}
	d\big((a_x,a_y),(b_x,b_y)\big)=\sqrt{  (a_x-b_x)^2+(a_y-b_y)^2  }.
\end{equation}
Cette définition a l'air raisonnable; est-elle mathématiquement correcte ? Peut-elle jouer le rôle de la valeur absolue dans $\eR^2$ ? Est-elle la seule définition possibles de «taille» et distance en $\eR^2$ ?

%---------------------------------------------------------------------------------------------------------------------------
\subsection{Norme}
%---------------------------------------------------------------------------------------------------------------------------

Nous voulons formaliser les notions de «taille» et de distance dans $\eR^n$, et plus généralement dans un espace vectoriel $V$ de dimension finie. Pour cela nous nous inspirons des propriétés de la valeur absolue.

\begin{definition}[\cite{BrunelleMatricielle}, thème~\ref{THEMEooUJVXooZdlmHj}]  \label{DefNorme}
    Soit \( E\) un espace vectoriel (pas spécialement de dimension finie) sur le corps \( \eK\) (\( =\eR\) ou \( \eC\)). Une  \defe{norme}{norme} sur $E$ est une application $N\colon E\to \eR^+$ telle que
	\begin{enumerate}
		\item
            \( N(x)=0\) si et seulement si \( x=0\);
		\item\label{ItemDefNormeii}
			$N(\lambda x)=| \lambda |N(x)$ pour tout $\lambda\in\eR$ et $x\in E$;
		\item\label{ItemDefNormeiii}
			$N(x+y)\leq N(x)+N(y)$
	\end{enumerate}
    pour tout $x,y\in E$ et pour tout $\lambda\in\eK$.

    La propriété~\ref{ItemDefNormeiii} est appelée \defe{inégalité triangulaire}{inégalité!triangulaire}.
\end{definition}
En prenant $\lambda=-1$ dans la propriété~\ref{ItemDefNormeii}, nous trouvons immédiatement que $N(-x)=N(x)$.

\begin{proposition}		\label{PropNmNNm}
	Toute norme $N$ sur l'espace vectoriel $E$ vérifie l'inégalité
	\begin{equation}
		\big| N(x)-N(y) \big|\leq N(x-y)
	\end{equation}
	pour tout $x,y\in E$.
\end{proposition}

\begin{proof}
	Nous avons, en utilisant le point~\ref{ItemDefNormeiii} de la définition~\ref{DefNorme},
	\begin{subequations}
		\begin{align}
			N(x)&=N(x-y+y)\leq N(x-y)+N(y),	\label{subEqNNNxxyyya}\\
			N(y)&=N(y-x+x)\leq N(y-x)+N(x).	\label{subEqNNNxxyyyb}
		\end{align}
	\end{subequations}
	Supposons d'abord que $N(x)\geq N(y)$. Dans ce cas, en utilisant \eqref{subEqNNNxxyyya},
	\begin{equation}
		\big| N(x)-N(y) \big|=N(x)-N(y)\leq N(x-y)+N(y)-N(y)=N(x-y).
	\end{equation}
	Si par contre $N(x)\leq N(y)$, alors nous utilisons \eqref{subEqNNNxxyyyb} et nous trouvons
	\begin{equation}
		\big| N(x)-N(y) \big|=N(y)-N(x)\leq N(y-x)+N(x)-N(x)=N(y-x).
	\end{equation}
	Dans les deux cas, nous avons retrouvé l'inégalité annoncée.
\end{proof}
Cette proposition signifie aussi que
\begin{equation}	\label{EqNleqNNleqNvqlqbs}
	-N(x-y)\leq N(x)-N(y)\leq N(x-y).
\end{equation}

Afin de suivre une notation proche de celle de la valeur absolue, à partir de maintenant, la norme d'un vecteur $v$ sera notée $\| v\|$ au lieu de $N(v)$. La proposition~\ref{PropNmNNm} s'énoncera donc
\begin{equation}
\big| \| x \|-\| y \| \big|\leq \| x-y \|.
\end{equation}
\begin{definition}		\label{DefEVNetDistance}
	Un espace vectoriel $E$ muni d'une norme est une \defe{espace vectoriel normé}{normé!espace vectoriel}, et on écrit $(E,\| . \|)$.
\end{definition}

\begin{lemmaDef}[Distance induite par une norme]        \label{LEMooWGBJooYTDYIK}
    Soit un espace vectoriel normé \( (E,\| . \|)\). Nous posons
    \begin{equation}        \label{EQooZYJRooAHnsIG}
        d(x,y)=\| x-y \| .
    \end{equation}
    Alors
    \begin{enumerate}
        \item       \label{ITEMooLITDooPeReOk}
            \( d\) est invariante par translations : $d(a,b)=d(a+u,b+u)$
        \item
            \( d\) est une distance\footnote{Définition~\ref{DefMVNVFsX}.} sur \( E\).
    \end{enumerate}
    C'est la \defe{distance induite}{distance!associée à une norme} par la norme.
\end{lemmaDef}

\begin{proof}
    Le fait que la formule \eqref{EQooZYJRooAHnsIG} soit invariante par translations est immédiat. En ce qui concerne le fait que ce soit une distance, le seul point délicat à vérifier est l'inégalité triangulaire. Nous avons \( \| x \|=d(x,0)\), donc nous pouvons écrire
    \begin{equation}
            d(x,y)=\| x-y \| \leq\| x \|+\| y \| =d(x,0)+d(y,0) =d(x,z)+d(y,z)
    \end{equation}
    parce que la distance associée à la norme est invariante par translation.
\end{proof}

Un espace vectoriel normé est alors immédiatement un espace vectoriel topologique

%---------------------------------------------------------------------------------------------------------------------------
\subsection{Quelques exemples}
%---------------------------------------------------------------------------------------------------------------------------

Il est possible de définir de nombreuses normes sur $\eR^n$. Citons en quelques unes.

\begin{propositionDef}      \label{PROPooCLZRooIRxCnZ}
    Les formules suivantes définissent des normes sur \( \eR^n\).
    \begin{enumerate}
        \item
    Les normes $\| . \|_{L^p}$ ($p\in\eN$) sont définies de la façon suivante :
    \begin{equation}		\label{EqDeformeLp}
        \| x \|_{L^p}=\Big( \sum_{i=1}^n| x_i |^p\Big)^{1/p},
    \end{equation}
    pour tout $x=(x_1,\ldots,x_n)\in\eR^n$.
\item
    La norme $L^2$ est la \defe{norme euclidienne}{norme!euclidienne}.
\item
    Nous définissons également la \defe{norme supremum}{norme!supremum} par
    \begin{equation}
	    \| x \|_{\infty}=\max_i| x_i |.
    \end{equation}
    \end{enumerate}
\end{propositionDef}

\begin{proof}
    Point par point\quext{Preuve non terminée}.
    \begin{enumerate}
        \item
        \item
    Le fait que \( x\mapsto\| x \|_{L^2}\) soit une norme provient de la propriété suivante :
    \begin{equation}
        \sqrt{ (a+b)^2 }\leq \sqrt{ a^2 }+\sqrt{ b^2 },
    \end{equation}
    laquelle se démontre en passant au carré :
    \begin{equation}        \label{EQooRYNYooTzZpPz}
        (a+b)^2=a^2+b^2+2ab\leq a^2+b^2+2| ab |=\big( \sqrt{ a^2 }+\sqrt{ b^2 } \big)^2.
    \end{equation}
\item
    \end{enumerate}
\end{proof}

Parmi ces normes, celles qui seront le plus souvent utilisées dans ces notes sont
\begin{equation}
	\begin{aligned}[]
		\| x \|_{L^1}&=\sum_{i=1}^n| x_i |,\\
		\| x \|_{L^2}&=\Big( \sum_{i=1}^n| x_i |^2 \Big)^{1/2}.
	\end{aligned}
\end{equation}

\newcommand{\CaptionFigDistanceEuclide}{La \emph{norme} euclidienne induit la \emph{distance} euclidienne. D'où son nom. Le point $C$ est construit aux coordonnées $(A_x,B_y)$.}
\input{auto/pictures_tex/Fig_DistanceEuclide.pstricks}

Soient $A=(A_x,A_y)$ et $B=(B_x,B_y)$ deux éléments de $\eR^2$. La distance\footnote{Ne pas confondre «distance» et «norme».} euclidienne entre $A$ et $B$ est donnée par $\| A-B \|_2$. En effet, sur la figure~\ref{LabelFigDistanceEuclide}, la distance entre les points $A$ et $B$ est donnée par
\begin{equation}
	| AB |^2=| AC |^2+| CB |^2=| A_x-B_x |^2+| A_y-B_y |^2,
\end{equation}
par conséquent,
\begin{equation}
	| AB |=\sqrt{| A_x-B_x |^2+| A_y-B_y |^2}=\| A-B \|_2.
\end{equation}

\begin{remark}
	Si $A$, $B$ et $C$ sont trois points dans le plan $\eR^2$, alors l'inégalité triangulaire $| AB |\leq| AC |+| CB |$ est précisément la propriété~\ref{ItemDefNormeiii} de la norme (définition~\ref{DefNorme}). En effet l'inégalité triangulaire s'exprime de la façon suivante en terme de la norme $\| . \|_2$ :
	\begin{equation}	\label{EqNDeuxAmBNNdd}
		\| A-B \|_2\leq \| A-C \|_2+\| C-B \|_2.
	\end{equation}
	En notant $u=A-C$ et $v=C-B$, l'équation \eqref{EqNDeuxAmBNNdd} devient exactement la propriété de définition de la norme :
	\begin{equation}
		\| u+v \|_2\leq \| u \|_2+\| v \|_2.
	\end{equation}
	Ceci explique pourquoi cette propriété des normes est appelée «inégalité triangulaire».
\end{remark}

Les distances que nous avons vues jusqu'à présent sont des distances définies à partir d'une norme. La définition générale d'une distance est la définition~\ref{DefMVNVFsX}.

%+++++++++++++++++++++++++++++++++++++++++++++++++++++++++++++++++++++++++++++++++++++++++++++++++++++++++++++++++++++++++++
\section{Suites de Cauchy, métrique et espaces complets}
%+++++++++++++++++++++++++++++++++++++++++++++++++++++++++++++++++++++++++++++++++++++++++++++++++++++++++++++++++++++++++++

%---------------------------------------------------------------------------------------------------------------------------
\subsection{Généralités}
%---------------------------------------------------------------------------------------------------------------------------

\begin{definition}[Suite de \( \tau\)-Cauchy, espace vectoriel topologique\cite{TQSWRiz,ooMKWJooLSkGfh}]   \label{DefZSnlbPc}
    Soit \( E\) un espace vectoriel topologique. Une suite \( (x_k)\) dans \( E\) est une \defe{suite \( \tau\)-Cauchy}{suite!de Cauchy} si pour tout voisinage \( \mU\) de \( 0\) il existe \( N\in \eN\) tel que \( x_k-x_l\in\mU\) pour tout \( k,l\geq N\).
\end{definition}

\begin{definition}[Espace \( \tau\)-complet]      \label{DEFooVQDBooNxprFU}
    Nous disons qu'une partie \( A\) d'un espace vectoriel topologique est \defe{\( \tau\)-complet}{complet!espace topologique} si toute suite \(  \tau\)-Cauchy d'éléments de \( A\) converge vers un élément de \( A\).
\end{definition}

\begin{definition}[Suite de Cauchy, espace métrique]      \label{DEFooXOYSooSPTRTn}
    Une suite \( (a_k)\) dans un espace métrique \( (V,d)\) est \defe{de Cauchy}{suite!de Cauchy} si pour tout \( \epsilon\in \eR\), il existe \( N\) tel que si \( n,m\geq N\) alors \( d(a_n,a_m)<\epsilon\).
\end{definition}

Notons qu'ici, même si l'espace \( V\) n'a rien à voir avec \( \eR\), nous prenons \( \epsilon\) dans \( \eR\) et la distance à valeurs dans \( \eR\). Cela semble une évidence, mais il faut se rendre compte que \( \eR\) commence à prendre une place centrale dans nos constructions. Ce n'était pas le cas du temps où nous parlions de suites de Cauchy et de complétude dans des corps totalement ordonnés (définitions~\ref{DefKCGBooLRNdJf}). Dans ce contexte, le \( \epsilon\) était pris dans le corps lui-même.

\begin{definition}[Métrique complete]       \label{DEFooHBAVooKmqerL}
    Soit \( (E,d)\) un espace métrique. Nous disons que la métrique \( d\) est \defe{complète}{complet!métrique} si toute suite de Cauchy dans \( (E,d)\) converge dans \( E\).
\end{definition}

\begin{normaltext}
    Ces définitions méritent quelques remarques.
    \begin{enumerate}
        \item
            Dans le cas des espaces vectoriels topologiques, nous définissons les notions de suite \( \tau\)-Cauchy et d'espace topologique \( \tau\)-complet. Nous ajoutons le préfixe \( \tau\) pour indiquer que ce sont des notions topologiques.
        \item
            Dans le cas des espaces métriques, nous définissons la notion de \emph{métrique} complète. C'est bien la métrique qui est complète, et non l'espace. En effet nous allons voir dans l'exemple~\ref{EXooNMNVooXyJSDm} que le même espace topologique peut accepter plusieurs distances différentes (donnant la même topologie) donnant lieu à des suites de Cauchy différentes.
        \item
            Si un espace vectoriel a une topologie issue d'une distance, rien ne dit que ses suites \( \tau\)-Cauchy et ses suites de Cauchy sont les mêmes. Ce sont deux notions a priori séparées. Si \( V\) est un espace vectoriel topologique que l'on peut munir de deux distances \( d_1, d_2\) donnant toutes deux la topologie, dire que \( V\) est \( \tau\)-complet, dire que \( d_1\) est complète et dire que \( d_2\) est complète sont trois choses différentes. Même si les trois topologies sont identiques.
        \item
            Nous allons bien entendu voir que dans de larges gammes d'exemples, les notions de suite de Cauchy et \( \tau\)-Cauchy coincident.
    \end{enumerate}
\end{normaltext}

\begin{example}[La complétude n'est pas une propriété topologique\cite{ooSCDYooWutzzr}]     \label{EXooNMNVooXyJSDm}
    Le fait pour un espace d'être complet n'est pas une propriété topologique, mais une propriété métrique. Plus exactement, il existe des espaces topologiques isomorphes, mais dont l'un est complet et l'autre non.

    Nous considérons la distance suivante sur \( \eN\) :
    \begin{equation}
        d_1(x,y)=\big| \frac{1}{ x }-\frac{1}{ y } \big|.
    \end{equation}
    Pour vérifier que cette formule définit bien une distance (définition~\ref{DefMVNVFsX}), le seul point non immédiat est l'inégalité triangulaire :
    \begin{equation}
        d_1(x,y)=\big| \frac{1}{ x }-\frac{1}{ y } \big|\leq\big| \frac{1}{ x }-\frac{1}{ z } \big|+\big| \frac{1}{ z }-\frac{1}{ y } \big|=d_1(x,z)+d_1(z,y).
    \end{equation}

    Au niveau de la topologie induite par cette distance, c'est la topologie discrète. En effet, soit \( x\in \eN\) et \( \epsilon>0\); nous voulons déterminer la boule \( B(x,\epsilon)\) en résolvant l'équation
    \begin{equation}
        \big| \frac{1}{ x }-\frac{1}{ y } \big|<\epsilon
    \end{equation}
    pour \( y\in \eN\). Nous trouvons
    \begin{subequations}
        \begin{numcases}{}
            y>\frac{ x }{ \epsilon x+1 }\\
            y<\frac{ x }{ 1-\epsilon x }.
        \end{numcases}
    \end{subequations}
    Si \( x\) est assez petit, la seule solution entière est \( y=x\). Les ouverts sont donc toutes les parties parce que tous les singletons sont ouverts.

    Si \( d\) est la distance usuelle sur \( \eN\) (\( d(x,y)=| x-y |\)), nous avons donc un isomorphisme d'espaces topologiques
    \begin{equation}
        (\eN,d)\simeq (\eN,d_1).
    \end{equation}
    Nous pouvons même donner un isomorphisme explicite : \( f(n)=n\).

    La suite \( (x_n)=n\) est une suite de Cauchy dans \( (\eN,d_1)\) parce que si \( \epsilon>0\) est donné, il suffit de prendre \( N\) assez grand pour avoir \( \frac{1}{ N }<\epsilon\) (possible par le lemme~\ref{LemooHLHTooTyCZYL}) nous avons, pour \( n,m>N\) :
    \begin{equation}
        | \frac{1}{ n }-\frac{1}{ m } |<\frac{1}{ n }<\frac{1}{ N }<\epsilon.
    \end{equation}
    Or cette suite ne converge pas. Soit en effet un candidat limite \( k\). Calculons
    \begin{equation}
        d_1(x_n,k)=| \frac{1}{ n }-\frac{1}{ k } |\to \frac{1}{ k }\neq 0.
    \end{equation}
    L'espace \( (\eN,d_1)\) n'est pas complet.

    Notons que cette suite n'est pas de Cauchy dans \( (\eN,d)\).

    En résumé :
    \begin{enumerate}
        \item
            Les espaces topologiques \( (\eN,d)\) et \( (\eN,d_1)\) sont isomorphes.
        \item
            Ils ont les mêmes notions de suites convergentes : une suite convergente pour l'un est convergente pour l'autre.
        \item
            Ils n'ont pas les mêmes notions de suites de Cauchy.
        \item
            Dans \(  (\eN,d_1)  \), il existe des suites de Cauchy qui ne convergent pas (pas complet).
        \item
            L'espace \( (\eN,d)\) est complet, mais \( (\eN,d_1)\) n'est pas complet.
        \item
            Le fait pour un espace topologique métrique d'être complet n'est pas intrinsèque à sa topologie : la complétude est une propriété de la distance. La complétude est une propriété de la métrique, et non de la topologie qui s'en suit.
    \end{enumerate}
\end{example}

%---------------------------------------------------------------------------------------------------------------------------
\subsection{Cas d'équivalence}
%---------------------------------------------------------------------------------------------------------------------------

\begin{lemma}       \label{LEMooIAHSooFkXjvr}
    Soit un espace vectoriel topologique \( V\) et une distance \( d\colon V\times V\to \eR^+\) compatible avec la topologie de \( V\). Si \( d\) est invariante\footnote{Définition~\ref{DEFooGTOZooRcvGHg}.}, alors les suites \( d\)-Cauchy et les suites \( \tau\)-Cauchy sont les mêmes.
\end{lemma}

\begin{proof}
    Nous avons deux implications à prouver.
    \begin{subproof}
    \item[\( d\)-Cauchy implique \( \tau\)-Cauchy]
        Soit \( (x_n)\), une suite \( d\)-Cauchy dans \( V\), et un voisinage \( U\) de \( 0\). Vu que \( d\) est compatible avec la topologie de \( V\), il existe une boule ouverte \( B(0,\epsilon)\) inclue à \( U\). Soit \( N>0\) tel que \( m,n>N\) implique \( d(x_n,x_m)<\epsilon\). Par invariance de la métrique, nous avons aussi
        \begin{equation}
            d(0,x_m-x_n)<\epsilon,
        \end{equation}
        c'est à dire \( x_m-x_n\in B(0,\epsilon)\subset U\). La suite \( (x_n)\) est donc \( \tau\)-Cauchy.
    \item[\( \tau\)-Cauchy implique \( d\)-Cauchy]
        Soit $(x_n)$, une suite \( \tau\)-Cauchy dans \( V\) et \( \epsilon>0\). Vu que \( B(0,\epsilon)\) est un voisinage de \( 0\) dans \( V\), il existe \( N\) tel que \( m,n>N\) implique \( x_n-x_m\in B(0,\epsilon)\). Cela signifie que \( d(0,x_n-x_m)<\epsilon\) et toujours par invariance, que \( d(x_n,x_m)<\epsilon\).
    \end{subproof}
\end{proof}

\begin{definition}[Espace vectoriel topologique métrisable\cite{ooOFEPooVFgTXm}]
    Un espace vectoriel topologique est \defe{métrisable}{métrisable} si il existe une distance compatible avec la topologie.
\end{definition}

\begin{proposition}[\cite{ooCGEHooVTyTuY}]      \label{PROPooXWBTooCvGLOj}
    Soit un espace topologique métrisable \( X\).
    \begin{enumerate}
        \item
            Tout fermé de \( X\) est une intersection dénombrable d'ouverts.
        \item
            Tout ouvert de \( X\) est une union dénombrable de fermés.
    \end{enumerate}
\end{proposition}

\begin{proof}
    Soit une métrique \( d\) compatible avec la topologie de \( X\) et un fermé \( A\). Nous posons
    \begin{equation}
        V_n=\{ x\in X\tq d(x,A)<\frac{1}{ n } \}.
    \end{equation}
    Et juste pour faire simple nous notons \( V_0=X\).
    \begin{subproof}
        \item[Les parties \( V_n\) sont ouvertes]
            Soit \( x\in V_n\). Trouvons un voisinage de \( x\) contenu dans \( V_n\) -- encore le théorème~\ref{ThoPartieOUvpartouv}. Si \( y\in B(x,\epsilon)\) alors il existe \( a\in A\) tel que \( d(x,a)<\frac{1}{ n }\) (ici les inégalités strictes sont importantes) et donc
            \begin{equation}
                d(y,a)\leq d(y,x)+d(x,a)<\epsilon+\frac{1}{ n }.
            \end{equation}
            Nous pouvons donc choisir \( \epsilon\) de telle sorte que \( d(y,a)<\epsilon\) pour tout \( a\in B(x,\epsilon)\). Cela prouve que \( V_n\) est ouvert.
        \item[\( A\) est l'intersection des \( V_n\)]
            Nous avons évidemment \( A\subset V_n\) pour tout \( n\). Et d'autre part, si \( a\in\bigcap_{n\in \eN} V_n\) alors \( d(a,A)<\frac{1}{ n }\) pour tout \( n\). Cela implique \( d(a,A)=0\), c'est à dire \( a\in A\).
        \end{subproof}
    En ce qui concerne la seconde partie, nous passons au complémentaire. Si \( \mO\) est ouvert, \( \mO^c\) est fermé et
    \begin{equation}
        \mO^c=\bigcap_{n\in \eN}V_n,
    \end{equation}
    ce qui donne immédiatement
    \begin{equation}
        \mO=\bigcup_{n\in \eN}V_n^c
    \end{equation}
    où les \( V_n^c\) sont fermés.
\end{proof}

\begin{corollary}       \label{CORooTWFYooCNMieM}
    Si \( X\) est un espace topologique métrisable, alors \( X\) accepte une base dénombrable de topologie autour de chaque point.
\end{corollary}

\begin{proof}
    Il s'agit seulement de remarquer que les singletons sont fermés et d'appliquer la proposition~\ref{PROPooXWBTooCvGLOj}.
\end{proof}

\begin{theorem}[\cite{ooMKWJooLSkGfh}]      \label{THOooAGBXooZnvQLK}
    Si $V$ est un espace vectoriel topologique possédant en tout point une base de topologie dénombrable, alors il existe une distance \( d\) sur \( V\) telle que
    \begin{enumerate}
        \item
            \( d\) est compatible avec la topologie de \( V\),
        \item
            \( d\) est invariante par translation.
    \end{enumerate}
\end{theorem}
Une preuve est donnée dans \cite{ooMKWJooLSkGfh} et je vous préviens : c'est pas simple.

\begin{proposition}     \label{PROPooPRLBooGtsRjr}
    Un espace vectoriel topologique est métrisable si et seulement si il possède en tout point une base dénombrable de topologie.
\end{proposition}

\begin{proof}
    Il s'agit seulement de mettre bout à bout les corollaires~\ref{CORooTWFYooCNMieM} et théorème~\ref{THOooAGBXooZnvQLK}.
\end{proof}

Tout ceci nous mène à donner une large classe d'espaces vectoriels topologiques sur lesquelles les notions de suites \( d\)-Cauchy et \( \tau\)-Cauchy coïncident.

\begin{theoremDef}     \label{THOooGQZSooAmQolf}
    Soit \( V\) un espace vectoriel topologique métrisable\footnote{i.e. admet une base dénombrable de topologique, voir la proposition~\ref{PROPooPRLBooGtsRjr}}, alors il admet une métrique \( d\) compatible avec la topologie telle que une suite dans \( V\) est \( d\)-Cauchy si et seulement si elle est \( \tau\)-Cauchy.

    Une \defe{suite de Cauchy}{Cauchy!suite} dans un espace vectoriel métrique \( (E,d)\) est une suite \( d\)-Cauchy ou \( \tau\)-Cauchy.
\end{theoremDef}

\begin{proof}
    Soit \( d\) une métrique sur \( V\) satisfaisant au théorème~\ref{THOooAGBXooZnvQLK}. Vu qu'elle est invariante par translation, les suites \( d\)-Cauchy sont exactement les suites \( \tau\)-Cauchy par le lemme~\ref{LEMooIAHSooFkXjvr}.
\end{proof}

\begin{remark}  \label{REMooUFQYooUVCCjs}
    Même si \( V\) est métrisable, si on choisit la métrique n'importe comment, on ne peut rien espérer.
\end{remark}

\begin{normaltext}
    Sur les espaces vectoriels topologiques métrisables, nous pouvons donc parler de suite de Cauchy sans préciser si nous parlons de \( \tau\)-Cauchy ou de \( d\)-Cauchy, parce que nous sous-entendons avoir choisi une métrique non seulement compatible avec la topologie, mais également invariante par translation.

    Il reste cependant à traiter le cas d'un espace vectoriel topologique non métrisable. Dans ce cas, il n'y a pas de métrique, et la question de l'équivalence des définitions ne se pose pas.
\end{normaltext}

Le théorème suivant donne la complétude de \( \eR\) et le critère de Cauchy pour les définitions métriques et topologiques usuelles. Lorsqu'on dit que \( \eR\) est complet, le plus souvent nous parlons de ce théorème, et non de~\ref{THOooUFVJooYJlieh} qui en est un lemme indispensable mais qui parle de notions différentes, bien que très liées.
\begin{theorem}[Complétude de \( \eR\), critère de Cauchy\cite{RWWJooJdjxEK}]       \label{THOooNULFooYUqQYo}
    Nous avons :
    \begin{enumerate}
        \item
            L'espace métrique \( (\eR,d)\) est complet (définition~\ref{DEFooHBAVooKmqerL}).
        \item
            Une suite dans \( \eR\) est convergente (définition~\ref{DefXSnbhZX}) si et seulement si elle est de Cauchy (définition~\ref{THOooGQZSooAmQolf}).
    \end{enumerate}
\end{theorem}
\index{complet!$\eR$!espace métrique}
\index{critère!de Cauchy}

\begin{proof}
    Tout ce théorème se base sur le fait que la définition de suite de Cauchy dans \( (\eR,d)\) et de suite convergente dans \( (\eR,d)\) coïncident avec les définitions correspondantes dans \( \eR\) vu comme simple corps ordonné (définitions~\ref{DefKCGBooLRNdJf}).

    Donc si \( (x_n)\) est de Cauchy dans \( (\eR,d)\), elle est de Cauchy dans le corps ordonné \( (\eR,\leq)\). Donc le théorème~\ref{THOooUFVJooYJlieh} nous dit que \( (x_n)\) est convergente dans \( (\eR,\leq)\). Et donc convergente dans \( (\eR,d)\).

    Toutes les autres affirmations se prouvent de la même manière.
\end{proof}

Si vous n'êtes pas sûr ou si vous ne voulez pas étudier les notations de convergence et de suites de Cauchy dans les corps, vous pouvez simplement recopier la démonstration du théorème~\ref{THOooUFVJooYJlieh} en remplaçant partout \( \eQ\) par \( \eR\), et aussi en remplaçant les \( | x-y |\) par \( d(x,y)\).

\begin{normaltext}
    Nous pouvons également mettre une structure d'espace métrique sur \( \eC\) en posant
    \begin{equation}
        d(z,z')=| z-z' |.
    \end{equation}
\end{normaltext}

\begin{proposition}
    L'espace métrique \( (\eC,d)\) est complet.
\end{proposition}

\begin{proof}
    Commençons par nous rendre compte que pour tout \( z\in \eC\) nous avons \( | \real(z) |\leq | z |\). C'est bon ? Vous vous en êtes rendu compte ? Ok. Continuons.

    Soit une suite de Cauchy \( (z_k)\) dans \( \eC\) et \( \epsilon>0\). Si \( x_k=\real(z_j)\), nous avons
    \begin{equation}
        | x_k-x_l |=| \real(z_k-z_l) |\leq | z_k-z_l |.
    \end{equation}
    Vu que \( (z_k)\) est de Cauchy, il existe un \( N\) tel que si \( k,l\geq N\),
    \begin{equation}
        | x_k-x_l |\leq | z_k-z_l |\leq \epsilon.
    \end{equation}

    Donc la suite des parties réelles converge par la complétude de \( (\eR,d)\) du théorème~\ref{THOooNULFooYUqQYo}. Notez que le \( d\) ici n'est pas tout à fait le même, et que la démonstration fonctionne parce que la distance prise sur \( \eR\) est la restriction à \( \eR\) de la distance prise sur \( \eC\). Notons \( x\) la limite de \( (x_k)\).

    De la même manière la suite des parties imaginaires \( y_k=\imag(z_k)\) converge vers un réel que nous notons \( y\). Avec tout cela, la suite \( z_k\) converge dans \( \eC\) vers \( x+iy\). En effet pour \( \epsilon\) donné et pour un \( k\) suffisament grand,
    \begin{equation}
        | z_k-(x+iy) |=\big| \real(z_k)-x+i(\imag(z_k)-y) \big|\leq | x_k-x |+| y_k-y |\leq \epsilon.
    \end{equation}
\end{proof}

%---------------------------------------------------------------------------------------------------------------------------
\subsection{Espace topologique métrique}
%---------------------------------------------------------------------------------------------------------------------------

Dans les espaces vectoriels topologiques métriques, il n'y a pas d'ambiguïté.
\begin{proposition}[Suite de Cauchy]     \label{PropooUEEOooLeIImr}
    Soit \( (E,d)\) un espace vectoriel topologique métrique.
    \begin{enumerate}
        \item   \label{ItemooROYMooAQCXnj}
            Une suite \( (x_n)\) dans \( E\) est convergente\footnote{Définition~\ref{DefXSnbhZX}.} vers \( x\) si et seulement si pour tout \( \epsilon\in \eR\) il existe \( N_{\epsilon}\) tel que pour tout \( n\geq N_{\epsilon}\) nous avons \( d(x_n,x)\leq \epsilon\).
        \item
            Une suite \( (x_n)\) dans \( E\) est de Cauchy\footnote{Définition~\ref{DefZSnlbPc}.} si pour tout \( \epsilon\in \eR\), il existe un \( N_{\epsilon}\) tel que si \( p,q\geq N_{\epsilon}\), nous avons \( d(x_p,x_q)\leq \epsilon\).
    \end{enumerate}
\end{proposition}

\begin{proof}
   En ce qui concerne la convergence :
    \begin{subproof}
        \item[Sens direct]

            Nous supposons que \( x_k\to x\) dans \( E\). Soit \( \epsilon>0\); vu que \( B(x,\epsilon)\) est un ouvert contenant \( x\), il existe un \( N_{\epsilon}>0 \) tel que \( k>N_{\epsilon}\) implique \( x_k\in B(x,\epsilon)\). Cela signifie \( d(x,x_k)\leq \epsilon\).

        \item[Réciproque]

            Nous supposons que pour tout \( \epsilon>0\), il existe \( N_{\epsilon}>0\) tel que si \( k>N_{\epsilon}\) alors \( x_k\in B(x,\epsilon)\). Soit un ouvert \( \mO\) autour de \( x\). Nous sommes dans un espace métrique; ergo la topologie est donné par le théorème~\ref{ThoORdLYUu} et en particulier la liste des ouverts est donnée par \eqref{EqGDVVooDZfwSf}. Il existe donc une boule \( B(x,\epsilon)\) inclue à \( \mO\). Pour tout \( k>N_{\epsilon}\) nous avons alors \( x_k\in B(x,\epsilon)\subset\mO\).
    \end{subproof}
    En ce qui concerne les suites de Cauchy :
    \begin{subproof}
    \item[Sens direct]
        Si \( (x_n)\) est une suite de Cauchy et si \( \epsilon>0\) est donné, alors \( B(0,\epsilon)\) est un voisinage de \( 0\) et il existe \( N_{\epsilon}\) tel que si \( p,q\geq N_{\epsilon}\) alors \( x_p-x_q\in B(0,\epsilon)\). Posons \( u=x_p-x_q\); en utilisant l'invariance par translation (lemme~\ref{LEMooWGBJooYTDYIK}\ref{ITEMooLITDooPeReOk}) nous avons
        \begin{equation}
            d(u,0)=d(x_p-x_q,0)=d(x_p,x_q).
        \end{equation}
        Par conséquent \( d(x_p,x_q)\leq \epsilon\).
    \item[Réciproque]
        Soit \( \mO\) un voisinage de \( 0\). Il existe \( \epsilon\) tel que \( B(0,\epsilon)\subset \mO\). Par hypothèse il existe \( N_{\epsilon}\) tel que \( d(x_p,x_q)\leq \epsilon\) dès que \( p,q\geq N_{\epsilon}\). En utilisant encore l'invariance par translation nous avons
        \begin{equation}
            d(x_p,x_q)=d(x_p-x_q,0),
        \end{equation}
        et comme cela est plus petit que \( \epsilon\), nous avons \( x_p-x_q\in B(0,\epsilon)\subset\mO\).
    \end{subproof}
\end{proof}

%---------------------------------------------------------------------------------------------------------------------------
\subsection{Espace métrique}
%---------------------------------------------------------------------------------------------------------------------------

\begin{proposition}[\cite{IRWFPQB}]
    Toute suite convergente dans un espace métrique est de Cauchy.
\end{proposition}

\begin{proof}
    Nous utilisons les caractérisations de la proposition~\ref{PropooUEEOooLeIImr} des suites convergentes et de Cauchy.

    Soit un espace métrique \( (X,d)\) et \( x_n\to\ell\) une suite convergente. Si \( \epsilon>0\), la proposition~\ref{PropooUEEOooLeIImr}\ref{ItemooROYMooAQCXnj}, dit qu'il existe \( N\) tel que pour tout \( n>N\) nous ayons \( d(x_n,\ell)<\epsilon\). Par conséquent si \( n,m>N\) alors
    \begin{equation}
        d(x_n,x_m)\leq d(x_m,\ell)+d(l,x_m)\leq 2\epsilon.
    \end{equation}
    Cela prouve que \( (x_n)\) est de Cauchy.
\end{proof}

%+++++++++++++++++++++++++++++++++++++++++++++++++++++++++++++++++++++++++++++++++++++++++++++++++++++++++++++++++++++++++++
\section{Topologie sur l'ensemble des réels}
%+++++++++++++++++++++++++++++++++++++++++++++++++++++++++++++++++++++++++++++++++++++++++++++++++++++++++++++++++++++++++++
\label{SECooGKHYooMwHQaD}

Nous allons à présent donner la topologie sur \( \eR\) et ainsi résoudre les questions laissées en suspend lors de la construction des réels, voir~\ref{NormooHRDZooRGGtCd}.

La valeur absolue de la définition~\ref{DefKCGBooLRNdJf}\ref{ItemooWUGSooRSRvYC} permet de définir une norme sur \( \eR\).
\begin{lemma}
    L'application
    \begin{equation}
         x\mapsto | x |
    \end{equation}
     est une norme sur $\eR$.
\end{lemma}

\begin{proof}
    Les conditions de la définition~\ref{DefNorme} sont immédiatement vérifiées grâce au lemme~\ref{LemooANTJooYxQZDw} et à la remarque~\ref{RemooJCAUooKkuglX}.
\end{proof}

\begin{normaltext}      \label{ooLCMFooQjMaxV}
Tant sur \( \eQ\) que sur \( \eR\), nous considérons la topologie métrique correspondant à cette norme (hors cas rarissimes qui seront signalés). Nous utiliserons toujours les caractérisations de la proposition~\ref{PropooUEEOooLeIImr} pour parler de suites convergentes et de suites de Cauchy.
\end{normaltext}

\begin{proposition}     \label{PropooUHNZooOUYIkn}
    Les rationnels sont denses dans les réels.
\end{proposition}
\index{densité!de \( \eQ\) dans \( \eR\)}

\begin{proof}
    Soient \( r\in \eR\) et \( \epsilon\in \eR^+\). Nous devons prouver l'existence d'un rationnel dans \( B(x,\epsilon)\). Le lemme~\ref{LemooHLHTooTyCZYL} dit qu'il existe un rationnel dans \( \mathopen] x-\epsilon/2 , x+\epsilon/2 \mathclose[\) et donc dans \( B(x,\epsilon)\).
\end{proof}


\begin{proposition}[\cite{MonCerveau}] \label{PropSLCUooUFgiSR}
    Quel que soit le réel \( r\), il existe une suite croissante de rationnels convergente vers \( r\).
\end{proposition}

\begin{proof}
    Soient \( x\in \eR\) et \( \delta\in \eR\); vu que \( x-\delta\) et \( x\) sont des réels, le lemme~\ref{LemooHLHTooTyCZYL} donne un élément \( x_{\delta}\) tel que
    \begin{equation}
        x-2\delta<x_{\delta}<x.
    \end{equation}
    Il ne reste plus qu'à pêcher parmi ces \( x_{\delta}\) pour trouver une suite croissante. Soit \( x_0\) un rationnel plus petit que \( x\). Nous posons \( \delta_0=x-x_0\) et ensuite :
    \begin{subequations}
        \begin{numcases}{}
            \delta_i=x-x_i\\
            x_{i+1}=x_{\delta_i/2}.
        \end{numcases}
    \end{subequations}
    Ainsi nous avons pour tout \( i\) les inégalités
    \begin{equation}
        x_i=x-\delta_i<x-\frac{ \delta_i }{ 2 }<x_{i+1}<x.
    \end{equation}
    La suite est donc croissante et toujours plus petite que \( x\). Mais nous avons à chaque étape \( \delta_{i+1}<\frac{ \delta_i }{ 2 }\), ce qui donne \( \delta_i\to 0\). Soit \( \epsilon>0\) et \( k\) tel que \( \delta_k<\epsilon\). Nous avons alors
    \begin{equation}
        x_{k+1}\in B(x,\frac{ \delta_k }{ 2 })\subset B(x,\epsilon).
    \end{equation}
\end{proof}

%+++++++++++++++++++++++++++++++++++++++++++++++++++++++++++++++++++++++++++++++++++++++++++++++++++++++++++++++++++++++++++
\section{Base de topologie}
%+++++++++++++++++++++++++++++++++++++++++++++++++++++++++++++++++++++++++++++++++++++++++++++++++++++++++++++++++++++++++++

\begin{proposition} \label{PropNBSooraAFr}
    Un espace métrique séparable\footnote{Qui possède une partie dense dénombrable, définition~\ref{DefUADooqilFK}.} accepte une base de topologie dénombrable.

     Soit \( A\) dense et dénombrable dans l'espace métrique séparable \( (E,d)\). Si \( \{ a_i \}_{i\in \eN}\) est une énumération de \( A\) et \( \{ r_i \}_{i\in \eN}\) une énumération de \( \eQ\), alors
    \begin{equation}
        \mB=\{ B(a_i,r_j) \}_{i,j\in \eN}
    \end{equation}
    est une base de la topologie de \( E\).
\end{proposition}
\index{base!de topologie!espace métrique}
\index{espace!métrique!base de topologie}
\index{base!de topologie!dénombrable}

\begin{proof}
    Soient \( x\in E\) et \( V\) un voisinage de \( x\). Ce dernier contient une boule \( B(x,r)\) et quitte à prendre \( r\) un peu plus petit nous supposons que \( r\in \eQ\) (densité de \( \eQ\) dans \( \eR\), proposition~\ref{PropooUHNZooOUYIkn}).

    Soit \( a\in A\) avec \( \| a-x \|<\frac{ r }{ 3 }\) (existe par densité de \( A\) dans \( E\)); nous avons \( B(a,\frac{ 2r }{ 3 })\subset B(x,r)\) parce que si \( y\in B( a,\frac{ 2r }{ 3 } )\) alors
    \begin{equation}
        \| y-x \|\leq \| y-a \|+\| a-x \|<\frac{ 2 }{ 3 }r+\frac{ 1 }{ 3 }r=r.
    \end{equation}
    La seconde inégalité est stricte parce que les boules sont ouvertes. Le tout montre que \( y\in B(x,r)\). Par ailleurs \( x\in B(a,\frac{ 2 }{ 3 }r)\) et nous avons trouvé un élément de \( \mB\) contenant \( x\) tout en étant inclus dans \( V\). Cela prouve que \( \mB\) est bien une base de la topologie de \( E\).
\end{proof}


\begin{remark}      \label{RemIPVLooHUXyeW}
    Il est vite vu que les cubes ouverts forment aussi une base de la topologie de \( \eR^n\). Cela est à mettre en rapport avec le fait que toutes les normes sont équivalentes sur \( \eR^n\) (proposition~\ref{ThoNormesEquiv}).

    % position 13268

    Voir aussi le corollaire~\ref{CorTHDQooWMSbJe} qui donnera tout ouvert comme union de pavés presque disjoints.
\end{remark}

\begin{lemma}   \label{LemDUJXooWsnmpL}
    Soient \( (X_1,d_1)\) et \( (X_2,d_2)\) des espaces métriques séparables. Alors \( X_1\times X_2\) admet une base dénombrable de topologie constituée de produits de boule de \( X_1\) par des boules de \( X_2\). Plus précisément si $A_i$ est dénombrable et dense dans \( X_i\) alors l'ensemble des produits
    \begin{equation}
        \big\{ B(y_1,r_1)\times B(y_2,r_2)\big\}_{\substack{y_i\in A_i\\r_i\in \eQ^+}}
    \end{equation}
    est une base de topologie pour \( X_1\times X_2\).
\end{lemma}

\begin{proof}
    Soit \( \mO\) un ouvert de \( X_1\times X_2\) et \( (x_1,x_2)\in \mO\). Par définition de la topologie produit\footnote{Définition~\ref{DefIINHooAAjTdY}.}, il existe \( r_1,r_2\in \eQ^+\) tels que \( B(x_1,r_1)\times B(x_2,r_2)\subset\mO\). Les parties \( A_i\) étant denses, il existe \( y_i\in B(x_i,r_i/2)\cap A_i\). Avec ces choix nous avons $x_i\in B(y_i,\frac{ r_i }{2})$. Nous avons donc
    \begin{equation}
        (x_1,x_2)\in B(y_1,\frac{ r_1 }{ 2 })\times B(y_2,\frac{ r_2 }{2}).
    \end{equation}
    Il est facile de voir que \( B(y_i,r_i/2)\subset B(x_i,r_i)\). En effet si \( z_i\in B(y_i,r_i/2)\) alors
    \begin{equation}
        d_i(z_i,x_i)\leq d(z_i,y_i)+d(y_i,x_i)\leq \frac{ r_i }{2}+\frac{ r_i }{2}=r_i.
    \end{equation}
    Au final,
    \begin{equation}
        (x_1,x_2)\in B(y_1,\frac{ r_1 }{ 2 })\times B(y_2,\frac{ r_2 }{2})\subset \mO.
    \end{equation}
\end{proof}

\begin{lemma}   \label{LemOWVooZKndbI}
    Une partie \( K\) d'un espace topologique est compacte si et seulement si de tout recouvrement par des ouverts d'une base de topologie nous pouvons extraire un sous-recouvrement fini.
\end{lemma}

\begin{proof}
    Soit \( K\) une partie d'un espace topologique et \( \{ \mO_i \}_{i\in I}\) un recouvrement de \( K\) par des ouverts. Chacun des \( \mO_i\) est une union d'éléments de la base de topologie par la proposition~\ref{PropMMKBjgY}. Soient \( \{ A_j \}_{j\in J}\) un ensemble d'éléments de la base de topologie tel que chacun des \( \mO_i\) est une union de certains des \( A_j\). Nous avons \( \bigcup_{j\in J}A_j=\bigcup_{i\in I}\mO_i\).

    Par hypothèse nous pouvons extraire un ensemble fini \( J_0\subset J\) tel que \( K\subset\bigcup_{j\in J_0}A_j\). Par construction chacun des \( A_j\) est inclus dans (au moins) un des \( \mO_i\). Le choix d'un élément de \( I\) pour chacun des éléments de \( J_0\) donne une partie finie \( I_0\) de \( I\) telle que \( K\subset\bigcup_{j\in J_0}A_j\subset\bigcup_{i\in I_0}\mO_i\).
\end{proof}
