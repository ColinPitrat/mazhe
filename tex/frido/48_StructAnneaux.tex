% This is part of Mes notes de mathématique
% Copyright (c) 2011-2018
%   Laurent Claessens
% See the file fdl-1.3.txt for copying conditions.


Attention aux conventions. Dans le Frido, un corps peut être réduit à \( \{ 0 \}\) et un idéal premier ne peut pas être \( \{ 0 \}\). Ces conventions ont une série de conséquences un peu partout, par exemple dans la proposition \ref{PROPooRUQKooIfbnQX} où nous parlons d'idéal maximum propre. Comparez par exemple avec \cite{ooWEUDooQybvIx}. Soyez attentifs.

En cas de doutes, nous suivons les conventions de Wikipédia.

%+++++++++++++++++++++++++++++++++++++++++++++++++++++++++++++++++++++++++++++++++++++++++++++++++++++++++++++++++++++++++++ 
\section{Inversible et niplolens}
%+++++++++++++++++++++++++++++++++++++++++++++++++++++++++++++++++++++++++++++++++++++++++++++++++++++++++++++++++++++++++++

Le concept d'anneau est la définition \ref{DefHXJUooKoovob}.

\begin{lemma}
    Si \( a\) et \( b\) commutent, nous avons, pour tout \( r \in \eN \) et \( r > 0\), la formule
    \begin{equation}        \label{Eqarpurmkbk}
        a^{r+1}-b^{r+1}=(a-b)\left(\sum_{k=0}^ra^{r-k}b^k\right).
    \end{equation}
\end{lemma}

\begin{proof}
  Démontrons cela par récurrence. Le cas \( r = 0 \) est évident. Pour
  un \( r \) donné, si \eqref{Eqarpurmkbk} est vraie, alors
  \begin{align*}
    a^{r+2}-b^{r+2}&= a^{r+1}a - a^{r+1}b +a^{r+1}b - b^{r+1}b\\
    &= a^{r+1}(a - b) + (a^{r+1} - b^{r+1})b\\
    &= a^{r+1}(a - b) + (a-b)\left(\sum_{k=0}^ra^{r-k}b^k\right)b\\
    &= (a - b) \left(a^{r+1} + \left(\sum_{k=0}^ra^{r-k}b^k\right)b\right)\\
    &= (a - b) \left(a^{r+1} + \sum_{k=0}^ra^{r-k}b^{k + 1}\right)\\
    &= (a - b) \left(a^{r+1} + \sum_{k'=1}^{r+1}a^{(r+1)-k'}b^{k'}\right)\\
    &= (a - b) \left(\sum_{k'=0}^{r+1}a^{(r+1)-k'}b^{k'}\right).
  \end{align*}
\end{proof}

\begin{proposition}
    Si \( a\) est un élément nilpotent de l'anneau \( A\), alors \( 1-a\) est inversible. Si \( a\) est nilpotent non nul, alors il est diviseur de zéro.
\end{proposition}

\begin{proof}
    Soit \( n\) le minimum tel que \( a^n=0\). En vertu de la formule \eqref{Eqarpurmkbk} nous avons
    \begin{equation}
        1=1-a^n=(1-a)(1+a+\cdots+a^{n-1})=(1+a+\cdots+a^{n-1})(1-a).
    \end{equation}
    La somme \( 1+a+\cdots+a^{n-1}\) est donc un inverse de \( (1-a)\).
\end{proof}

\begin{definition}      \label{DEFooTCUOooWHlbee}
    Soient \( A\) un anneau et \( a,b\in A\). Nous disons que \( d\) est un \( \pgcd\)\index{pgcd} de \( a\) et \( b\) si
    \begin{itemize}
        \item \( d\) divise \( a\) et \( b\),
        \item
    tout diviseur commun de \( a\) et \( b\) divise \( d\).
    \end{itemize}
\end{definition}

Nous rappelons également la définition~\ref{DEFooQBGJooKJqHXr} de morphisme d'anneaux. Remarquons que si \( f\) est un morphisme, nous avons \( f(0)=0\) et \( f(x)^{-1}=f(x^{-1})\).

\begin{lemma}[\cite{ooLIOMooBuCPUS}]
    Les éléments inversibles d'un anneau sont diviseurs de tous les éléments.
\end{lemma}

\begin{proof}
    Soit \( k\) inversible d'inverse \( k'\) : \( kk'=1\); soit aussi \( a\in A\). Alors \( a=k(k'a)\), ce qui montre que \( k\) divise \( a\).
\end{proof}

\begin{lemma}[\cite{ooLIOMooBuCPUS}]
    Dans un anneau, \( 1\) est un pgcd de \( a\) et \( b\) si et seulement si les seuls diviseurs communs sont les inversibles.
\end{lemma}

\begin{proof}
    Supposons pour commencer que \( 1\) est un pgcd de \( a\) et \( b\). Un diviseur commun de \( a\) et \( b\) doit donc diviser \( 1\). Or un diviseur de \( 1\) est forcément inversible.

    Dans l'autre sens, les diviseurs communs de \( a\) et \( b\) sont tous inversibles et donc diviseurs de \( 1\). Donc \( 1\) est un pgcd de \( a\) et \( b\).
\end{proof}

%+++++++++++++++++++++++++++++++++++++++++++++++++++++++++++++++++++++++++++++++++++++++++++++++++++++++++++++++++++++++++++
\section{Binôme de Newton et morphisme de Frobenius}
%+++++++++++++++++++++++++++++++++++++++++++++++++++++++++++++++++++++++++++++++++++++++++++++++++++++++++++++++++++++++++++

\begin{proposition}[\cite{ooPTQCooIWykWP}]     \label{PropBinomFExOiL}
Pour tout $x$, $y\in\eR$ et $n\in\eN$, nous avons
\begin{equation}        \label{EqNewtonB}
    (x+y)^n=\sum_{k=0}^n{n\choose k}x^{n-k}y^k
\end{equation}
où
\begin{equation}
    {n\choose k}=\frac{ n! }{ k!(n-k)! }
\end{equation}
sont les \defe{coefficients binomiaux}{coefficients binomiaux}.
\end{proposition}

\begin{proof}
    La preuve se fait par récurrence. La vérification pour $n=0$ et $n=1$ se fait aisément pour peu que l'on se rappelle que \( x^0=1\) et que \( 0!=1\), ce qui donne entre autres \( {0\choose 0}=1\).

    Supposons que la formule \eqref{EqNewtonB} soit vraie pour $n\geq1$, et prouvons la pour $n+1$. Nous avons
\begin{equation}        \label{EqBinTrav}
    \begin{aligned}[]
        (x+y)^{n+1} &=(x+y)\cdot  \sum_{k=0}^n{n\choose k}x^{n-k}y^k\\
                &= \sum_{k=0}^n{n\choose k}x^{n-k+1}y^k+\sum_{k=0}^n{n\choose k}x^{n-k}y^{k+1}\\
                &=x^{n+1}+ \sum_{k=1}^n{n\choose k}x^{n-k+1}y^k+\sum_{k=0}^{n-1}{n\choose k}x^{n-k}y^{k+1}+y^{n+1}.
    \end{aligned}
\end{equation}
La seconde grande somme peut être transformée en posant $i=k+1$ :
\begin{equation}
    \sum_{k=0}^{n-1}{n\choose k}x^{n-k}y^{k+1}  =\sum_{i=1}^n{n\choose i-1}x^{n-(i-1)}y^{i-1+1},
\end{equation}
dans lequel nous pouvons immédiatement renommer $i$ par $k$. En remplaçant dans la dernière expression de \eqref{EqBinTrav}, nous trouvons
\begin{equation}
    (x+y)^{n+1}=x^{n+1}+y^{n+1}+\sum_{k=1}^n\left[ {n\choose k}+{n\choose k-1} \right]x^{n-k+1}y^k.
\end{equation}
La thèse découle maintenant de la formule
\begin{equation}
    {n\choose k}+{n\choose k-1}={n+1\choose k}
\end{equation}
qui est vraie parce que
\begin{equation}
    \frac{ n! }{ k!(n-k)! }+\frac{ n! }{ (k-1)(n-k+1)! }=\frac{ n!(n-k+1)+n!k }{ k!(n-k+1)! }=\frac{ n!(n+1) }{  k!(n-k+1)!  },
\end{equation}
par simple mise au même dénominateur.
\end{proof}

%+++++++++++++++++++++++++++++++++++++++++++++++++++++++++++++++++++++++++++++++++++++++++++++++++++++++++++++++++++++++++++
\section{Idéal dans un anneau}
%+++++++++++++++++++++++++++++++++++++++++++++++++++++++++++++++++++++++++++++++++++++++++++++++++++++++++++++++++++++++++++

La définition d'un idéal dans un anneau est la définition~\ref{DefooQULAooREUIU}.


\begin{definition}[Idéal engendré par un élément]  \label{DefSKTooOTauAR}
    Si \( p\) est un élément d'un anneau \( A\) alors nous notons \( (p)\)\nomenclature[A]{\( (p)\)}{idéal engendré par \( p\)}\index{engendré!idéal dans un anneau} l'idéal dans \( A\) \defe{engendré}{engendré} par \( p\), c'est à dire \( pA\).
\end{definition}

\begin{definition}  \label{DefAJVTPxb}
    Un sous-ensemble \( B\subset A\) d'un anneau est un \defe{sous anneau}{sous anneau} si
    \begin{enumerate}
        \item
            \( 1\in B\)
        \item
            \( B\) est un sous-groupe pour l'addition
        \item
            \( B\) est stable pour la multiplication.
    \end{enumerate}
\end{definition}

\begin{remark}
    Un idéal n'est pas toujours un anneau parce que l'identité pourrait manquer. Un idéal qui contient l'identité est l'anneau complet.
\end{remark}

\begin{example}
    L'ensemble \( 2\eZ\) est un idéal de \( \eZ\). On peut aussi le noter \( (2) \).
\end{example}

\begin{propositionDef}      \label{PROPooGXMRooTcUGbi}
    Soit \( A\), un anneau, \( I\) un idéal bilatère\footnote{Définition~\ref{DefooQULAooREUIU}.} de \( A\). Nous considérons la relation d'équivalence \( x\sim y\) si et seulement si \( x-y\in I\). Sur le quotient
    \begin{equation}
        A/\sim=A/I,
    \end{equation}
    nous mettons les opérations
    \begin{enumerate}
        \item
            \( \bar x+\bar y=\overline{ x+y }\);
        \item
            \( \bar x\bar y=\overline{ xy }\).
    \end{enumerate}
    Nous avons alors les résultats suivants :
    \begin{enumerate}
        \item
            Les opérations sont bien définies,
        \item
            l'ensemble \( A/I\) est un anneau
        \item
            la surjection canonique \( A\to A/I\) est un morphisme.
    \end{enumerate}
    Cet anneau est appelé \defe{anneau quotient}{anneau!quotient par un idéal}.
\end{propositionDef}

\begin{proof}
    Nous montrons que le produit est bien défini\quext{Si vous le voulez, n'hésitez pas à m'envoyer un patch pour le reste de la démonstration.}. Nous savons que, par définition,
    \begin{equation}
        \bar x=\{ x+i\tq i\in I \}.
    \end{equation}
    Calculons le produit de représentants génériques de \( \bar x\) et de \( \bar y\) :
    \begin{equation}
        (x+i_1)(y+i_2)=xy+xi_2+yi_1+i_1i_2.
    \end{equation}
    Vu que \( I\) est un idéal nous avons \( xi_2+yi_1+i_1i_2\in I\) et donc bien
    \begin{equation}
        (x+i_1)(y+i_2)\in \overline{ xy }.
    \end{equation}
\end{proof}

\begin{proposition}[Premier théorème d'isomorphisme pour les anneaux]
    Soient \( A\) et \( B\) des anneaux et un homomorphisme \( f\colon A\to B\). Nous considérons l'injection canonique \( j\colon f(A)\to B\) et la surjection canonique \( \phi\colon A\to A/\ker f\). Alors il existe un unique isomorphisme
    \begin{equation}
        \tilde f \colon A/\ker f\to f(A)
    \end{equation}
    tel que \( f=j\circ\tilde f\circ\phi\).

    \begin{equation}
        \xymatrix{%
        A \ar[r]^{f}\ar[d]_{\phi}        &   B\ar[d]^{j}\\
           A/\ker f \ar[r]_{\tilde f}   &   f(A)\subset B
           }
    \end{equation}
\end{proposition}
\index{théorème!isomorphisme!premier!pour les anneaux}

\begin{proposition}     \label{PropIJJIdsousphi}
    Soient \( I\) un idéal de \( A\) et la projection canonique
    \begin{equation}
        \phi\colon A\to A/I.
    \end{equation}
    Elle est une bijection entre les idéaux de \( A\) contenant \( I\) et les idéaux de \( A/I\).

    Dit de façon imagée :
    \begin{equation}        \label{EqKbrizu}
        \{ \text{idéaux de } A\text{ contenant } I\}\simeq\{ \text{idéaux de } A/I \}.
    \end{equation}
\end{proposition}

\begin{proof}
    Si \( I\subset J\) et si \( J \) est un idéal de \( A\), alors \( \phi(J)\) est un idéal dans \( A/I\). En effet un élément de \( \phi(J)\) est de la forme \( \phi(j)\) et un élément de \( A/I\) est de la forme \( \phi(i)\). Leur produit vaut
    \begin{equation}
        \phi(i)\phi(j)=\phi(ij)\in\phi(J).
    \end{equation}

    Soit maintenant \( K\) un idéal dans \( A/I\) et soit \( J=\phi^{-1}(K)\). Étant donné qu'un idéal doit contenir \( 0\) (parce qu'un idéal est un groupe pour l'addition), \( [0]\in K\) et par conséquent \( I\subset\phi^{-1}(K)\).
\end{proof}
% TODO : il faudrait dire à peu près ici qu'une des utilités de Z_2 est le groupe modulaire PSL(2,Z)=SL(2,Z)/Z_2

\begin{proposition}[\cite{MonCerveau}]     \label{AnnCorpsIdeal}\label{PROPooUOCVooZGAVVk}
    Si \( A\) est un anneau, nous avons les équivalences
    \begin{enumerate}
        \item       \label{ITEMooLAAVooXhTcMe}
            \( A\) est un corps.
        \item       \label{ITEMooDGZIooRopYGx}
            \( A\) est non nul et ses seuls\footnote{Je vous laisse vous poser de grandes questions sur le fait que le vide est un idéal ou non.} seuls idéaux à gauche sont \( \{ 0 \}\) et \( A\).
        \item       \label{ITEMooLPWHooDJpTbR}
            \( A\) est non nul et ses seuls idéaux à droite sont \( \{ 0 \}\) et \( A\).
    \end{enumerate}
\end{proposition}

\begin{proof}
    Nous allons montrer que le point \ref{ITEMooLAAVooXhTcMe} est équivalent aux deux autres.
    \begin{subproof}
        \item[\ref{ITEMooLPWHooDJpTbR} implique \ref{ITEMooDGZIooRopYGx}]
            Si \( I\) est un idéal à gauche différent de \( \{ 0 \}\), alors il contient un certain \( a\neq 0\). Vu que \( A\) est un corps, il contient un inverse \( a^{-1}\), et comme \( I\) est un idéal, \( a^{-1} I\subset I\). En particulier \( a^{-1}a\in I\). Donc \( 1\in I\) et \( I=A\).
        \item[\ref{ITEMooDGZIooRopYGx} implique \ref{ITEMooLAAVooXhTcMe}]

            Supposons que les seuls idéaux de \( A\) soient \( \{ 0 \}\) et \( A\). Soit \( a\in A\). Si \( a\) est non nul, alors \( aA=A\), en particulier, \( 1\in aA\), c'est à dire qu'il existe \( b\in A\) tel que \( ab=1\). L'élément \( a\) est donc inversible.
    \end{subproof}
\end{proof}

\begin{definition}\label{DEFIdealMax}
Un idéal \( I\) dans un anneau \( A \) est dit \defe{idéal maximal}{idéal!maximal}\index{idéal!maximal} ou idéal maximum si tout idéal \( J \) vérifiant \( I \subset J \subset A \) est soit \( I \), soit \( A \).
\end{definition}

\begin{proposition}[Thème~\ref{THEMEooZYKFooQXhiPD}]     \label{PROPooSHHWooCyZPPw}
    Un idéal \( I\) dans un anneau \( A \) est maximum si et seulement si \( A/I\) est un corps.
\end{proposition}

\begin{proof}
    Soit un idéal maximum \( I\subset A\). Alors les idéaux contenant \( I\) sont \( A\) et \( I\). L'application \( \phi\) de la proposition~\ref{PropIJJIdsousphi} est une bijection, donc l'ensemble des idéaux de \( A/I\) ne contient que deux éléments. Les seuls idéaux de \( A/I\) sont donc \( \{ 0 \}\) et \( A/I\); donc \( A/I\) est un corps par la proposition~\ref{PROPooUOCVooZGAVVk}.

    Dans l'autre sens, c'est la même chose : si \( A/I\) est un corps, il possède exactement deux idéaux, donc \( A\) ne contient que deux idéaux contenant $I$. Donc \( I\) est un idéal maximum.
\end{proof}

%---------------------------------------------------------------------------------------------------------------------------
\subsection{Résultats supplémentaires sur l'anneau des entiers}
%---------------------------------------------------------------------------------------------------------------------------

\begin{corollary}       \label{CORooLINXooBlUKPG}
    Les quotients de \( \eZ\) sont \( \eZ/n\eZ\).
\end{corollary}

\begin{proof}
    Tous les idéaux de \( \eZ\) sont de la forme \( n\eZ\). En effet en vertu de la proposition~\ref{PropSsgpZestnZ}, les seuls sous-groupes de \( \eZ\) (en tant que groupe additif) sont les \( n\eZ\). Tous les idéaux sont donc de cette forme. De plus les \( n\eZ\) sont effectivement tous des idéaux : si \( a\in n\eZ\) et si \( k\in \eZ\) alors \( ak\in n\eZ\). Cela est donc un idéal.
\end{proof}

\begin{proposition}     \label{PropZpintssiprempUzn}
    Soient \( n\geq 2\) un entier et \( \phi\colon \eZ\to \eZ/n\eZ\) la surjection canonique. Nous noterons \( \tilde a=\phi(a)\). Alors l'ensemble des inversibles de \( \eZ/n\eZ\) est donné par
    \begin{equation}
        U(\eZ/n\eZ)=\phi(P_n)=\{ \tilde x\tq 0\leq x\leq n\tq\pgcd(x,n)=1 \}.
    \end{equation}
    où \( P_n\) est l'ensemble $P_n=\{ x\in\{ 0,\ldots,n-1 \}\tq\pgcd(x,n)=1 \}$.

    De plus,
    \begin{equation}
        \Card\big( U(\eZ/n\eZ) \big)=\varphi(n).
    \end{equation}
\end{proposition}

\begin{proof}
    Soit \( 0\leq x\leq n\) tel que \( \pgcd(x,n)=1\). Il existe donc\footnote{Théorème de Bézout~\ref{ThoBuNjam}} \( u,v\in\eZ\) tels que \( ux+vn=1\). En passant aux classes,
    \begin{equation}
        \tilde u\tilde x=\tilde 1,
    \end{equation}
    donc \( \tilde u\) est l'inverse de \( \tilde x\). Cela prouve que \( \phi(P_n)\subset U(\eZ/n\eZ)\).

    Nous prouvons maintenant l'inclusion inverse. Soient \( \tilde x\) et \( \tilde y\) inverses l'un de l'autre : $\tilde x\tilde y=\tilde 1$. Il existe donc \( q\in\eZ\) tel que \( xy-qn=1\), ce qui prouve\footnote{À nouveau avec le Théorème de Bézout.} que \( \pgcd(x,n)=1\).
\end{proof}

%+++++++++++++++++++++++++++++++++++++++++++++++++++++++++++++++++++++++++++++++++++++++++++++++++++++++++++++++++++++++++++
\section{Caractéristique}
%+++++++++++++++++++++++++++++++++++++++++++++++++++++++++++++++++++++++++++++++++++++++++++++++++++++++++++++++++++++++++++

\begin{lemmaDef}        \label{LEMDEFooVEWZooUrPaDw}
    Soit l'application
    \begin{equation}
        \begin{aligned}
            \mu\colon \eZ&\to A \\
            n&\mapsto n\cdot 1_A .
        \end{aligned}
    \end{equation}
    \begin{enumerate}
        \item
            C'est un morphisme d'anneaux.
        \item
            Le noyau est un sous-groupe de \( \eZ\)
        \item
            Il existe un unique \( p\in \eZ\) tel que \( \ker(\mu)=p\eZ\).
    \end{enumerate}
    Ce \( p\) est la \defe{caractéristique}{caractéristique!d'un anneau} de \( A\).
\end{lemmaDef}

Par exemple la caractéristique que \( \eQ\) est zéro parce qu'aucun multiple de l'unité n'est nul.

À propos de diagonalisation en caractéristique \( 2\), voir l'exemple~\ref{ExewINgYo}.

\begin{lemma}
    Si \( A\) est de caractéristique nulle, alors \( A\) est infini.
\end{lemma}

\begin{proof}
    En effet, \( \ker\mu=\{0\} \) implique que \( n1_A \neq  m1_A\) dès que \(n \neq m \) et par conséquent \( A\) contient \(\eZ 1_A \), et  est infini.
\end{proof}

\begin{lemma}       \label{LemHmDaYH}
    Si \( p\) est la caractéristique de l'anneau \( A\), alors nous avons l'isomorphisme d'anneaux
    \begin{equation}
         \eZ 1_A\simeq\eZ/p\eZ.
    \end{equation}
\end{lemma}

\begin{proof}
    L'isomorphisme est donné par l'application \( n1_A\mapsto \phi(n)\) si \( \phi\) est la projection canonique \( \eZ\to \eZ/p\eZ\).
\end{proof}

\begin{proposition}     \label{PropGExaUK}
    La caractéristique d'un anneau fini divise son cardinal.
\end{proposition}

\begin{proof}
    Si \( A\) est un anneau, le groupe \( \eZ\) agit sur \( A\) par
    \begin{equation}
        n\cdot a=a+n1_A.
    \end{equation}
    Chaque orbite de cette action est de la forme
    \begin{equation}
        \mO_a=\{ a+n1_A\tq n=0,\ldots, p-1 \}
    \end{equation}
    où \( p\) est la caractéristique de \( A\). Les orbites ont \( p\) éléments et forment une partition de \( A\), donc le cardinal de \( A\) est un multiple de \( p\).
\end{proof}

\begin{lemma}[\cite{ooIBWOooSjOvXd}]        \label{LEMooJQIKooQgukqn}
    Un anneau totalement ordonné est de caractéristique nulle.
\end{lemma}

\begin{proof}
    Le morphisme \( \mu\colon \eZ\to A\), \( n\mapsto n 1_A\) est strictement croissant, en particulier \( \mu(x)\neq \mu(y)\) dès que \( x\neq y\). Donc \( \ker(\mu)=\{ 0 \}\).
\end{proof}

L'ensemble typique de caractéristique \( p\) est \( \eF_p=\eZ/p\eZ\).



\begin{proposition}     \label{Propqrrdem}
    Soit \( A\) un anneau commutatif de caractéristique première \( p\). Alors \( \sigma(x)=x^p\) est un automorphisme de l'anneau \( A\). Nous avons la formule
    \begin{equation}
        (a+b)^p=a^p+b^p
    \end{equation}
    pour tout \( a,b\in A\).
\end{proposition}

\begin{proof}
    Nous utilisons la formule du binôme de la proposition~\ref{PropBinomFExOiL} et le fait que les coefficients binomiaux non extrêmes sont divisibles par \( p\) et donc nuls.
\end{proof}

\begin{proposition} \label{PropFrobHAMkTY}
    Soit \( A\) un anneau commutatif unitaire de caractéristique \( p\). L'application
    \begin{equation}
        \begin{aligned}
            \Frob_A\colon A&\to A \\
            x&\mapsto x^p
        \end{aligned}
    \end{equation}
    est un automorphisme d'anneau unitaire.
\end{proposition}
Nous le nommons le \defe{morphisme de Frobenius}{morphisme!Frobenius}\index{Frobenius!morphisme}. Nous utiliserons aussi les itérés du morphisme de Frobenius : \( \Frob^k\colon x\mapsto x^{p^k}\).

\begin{example}
    Soit à factoriser \( X^p-1\) dans \( \eF_p\). Grâce au morphisme de Frobenius, nous avons immédiatement
    \begin{equation}
        X^p-1=(X-1)^p.
    \end{equation}
\end{example}

%+++++++++++++++++++++++++++++++++++++++++++++++++++++++++++++++++++++++++++++++++++++++++++++++++++++++++++++++++++++++++++
\section{Module sur un anneau}
%+++++++++++++++++++++++++++++++++++++++++++++++++++++++++++++++++++++++++++++++++++++++++++++++++++++++++++++++++++++++++++

\begin{definition}[module sur un anneau\cite{ooJGVOooSjQBVh}]       \label{DEFooHXITooBFvzrR}
    Soit un anneau \( A\). Un \defe{module à gauche}{module!à gauche} sur \( A\) est la donnée d'un triple \( (M,+,\cdot)\) où
    \begin{enumerate}
        \item
            \( +\) est une loi de composition interne à \( M\), c'est à dire \( +\colon M\times M\to M\),
        \item
            \( \cdot\) est une loi de composition externe, c'est à dire \( \cdot\colon A\times M\to M\)
    \end{enumerate}
    telles que
    \begin{enumerate}
        \item
            \( (M,+)\) est un groupe\footnote{Nous verrons dans la proposition~\ref{PROPooGARGooDiMqtN} qu'il est forcément commutatif.}.
        \item
            \( a\cdot(x+y)=a\cdot x+a\cdot y\),
        \item
            \( (a+b)\cdot x=a\cdot x+b\cdot x\),
        \item
            \( (ab)\cdot x=a\cdot(b\cdot x)\)
        \item
            \( 1\cdot x=x\).
    \end{enumerate}
    pour tout \( a,b\in A\) et \( x,y\in M\).
\end{definition}

\begin{proposition}\label{PROPooGARGooDiMqtN}
    Si \( M\) est un module sur un anneau, alors \( (M,+)\) est un groupe commutatif.
\end{proposition}

\begin{proof}
    Il suffit de calculer \( (1+1)\cdot (x+y)\) de deux façons différentes :
    \begin{equation}
        (1+1)\cdot (x+y)=1\cdot (x+y)+1\cdot (x+y)=x+y+x+y
    \end{equation}
    d'une part et
    \begin{equation}
        (1+1)\cdot (x+y)=(1+1)\cdot x+(1+1)\cdot y=x+x+y+y,
    \end{equation}
    d'autre part. En égalant les deux expressions, il vient
    \begin{equation}
        x+y+x+y=x+x+y+y,
    \end{equation}
    qui se simplifie (nous sommes dans un groupe) en \( y+x=x+y\).
\end{proof}

\begin{definition}\label{DEFooKHWZooIfxdNc}
    Un \defe{espace vectoriel}{espace!vectoriel} est un module sur un corps commutatif\footnote{La condition de commutativité n'est pas indispensable, mais comme nous ne parlerons que de corps commutatifs \ldots}.
\end{definition}

Soient \( M\) un \( A\)-module et \( x=(x_i)_{i\in I}\) une famille d'éléments de \( M\) paramétrée par l'ensemble \( I\). Nous considérons l'application
\begin{equation}
    \begin{aligned}
        \mu_x\colon A^{(I)}&\to M \\
        (a_i)_{i\in I}&\mapsto \sum_{i\in I}a_ix_i.
    \end{aligned}
\end{equation}
Ici \( A^{(I)}\) désigne l'ensemble de toutes les applications \( I\to A\) de support fini.

\begin{definition}      \label{DefBasePouyKj}
    À l'instar des espaces vectoriels, les modules ont une notion de partie libre, génératrice et de bases :
    \begin{enumerate}
        \item
            Si \( \mu_x\) est surjective, nous disons que \( x\) est une partie \defe{génératrice}{génératrice!partie d'un module}.
        \item
            Si \( \mu_x\) est injective, nous disons que la partie \( x\) est \defe{libre}{libre!partie d'un module}.
        \item
            Si \( \mu_x\) est bijective, nous disons que la partie \( x\) est une \defe{base}{base!d'un module}.
    \end{enumerate}
\end{definition}

\begin{definition}
  Un sous-ensemble \( N\subset M\) est un \defe{sous-module}{sous-module} si \( (N,+)\) est un sous-groupe de \( (M,+)\) et si \( a\cdot x\in N\) pour tout \( x\in N\) et pour tout \( a\in A\).
\end{definition}

\begin{example}
    Un anneau \( A\) est lui-même un \( A\)-module et ses sous-modules sont les idéaux.
\end{example}

\begin{definition}
    Soit \( M\) un module sur un anneau commutatif \( A\). Un \defe{projecteur}{projecteur!dans un module} est une application linéaire \( p\colon M\to M\) telle que \( p^2=p\).

    Une famille \( (p_i)_{i\in I}\) sur \( M\) est \defe{orthogonale}{orthogonal!famille de projecteurs} si \( p_i\circ p_j=0\) pour tout \( i\neq j\). La famille est \defe{complète}{complète!famille de projecteurs} si \( \sum_{i\in I}p_i=\mtu\).
\end{definition}

\begin{theorem}     \label{ThoProjModpAlsUR}
    Soient des sous modules \( M_1,\ldots,M_n\) du module \( M \) tels que \( M=M_1\oplus\ldots\oplus M_n\). Les applications \( p_i\) définies par
    \begin{equation}
        p_i(x_1+\ldots+x_n)=x_i
    \end{equation}
    forment une famille orthogonale de projecteurs et \( p_1+\cdots +p_n=\id\).

    Inversement, si \( (p_1,\ldots, p_n)\) est une famille orthogonale de projecteurs dans un module \( \modE\) tel que \( \sum_{i=1}^np_i=\id\), alors
    \begin{equation}
        M=\bigoplus_{i=1}^np_i(M).
    \end{equation}
\end{theorem}

\begin{definition}
    Un module est \defe{simple}{simple!module}\index{module!simple} ou \defe{irréductible}{irréductible!module}\index{module!irréductible} s'il n'a pas d'autre sous-modules que \( \{ 0 \}\) et lui-même. Un module est \defe{indécomposable}{indécomposable!module}\index{module!indécomposable} s'il ne peut pas être écrit comme somme directe de sous-modules.
\end{definition}

Un module simple est a fortiori indécomposable. L'inverse n'est pas vrai comme le montre l'exemple suivant.

\begin{example}
    Soit \( \modE=\eC[X]/(X^2)\) vu comme \( \eC[X]\)-module. C'est le \( \eC[X]\)-module des polynômes de la forme \( aX+b\) avec \( a,b\in \eC\). L'ensemble des polynômes de la forme \( aX\) est un sous-module. Le module \( \modE\) n'est donc pas simple. Il est cependant indécomposable parce que \( \{ aX \}\) est le seul sous-module non trivial. En effet si \( \modF\) est un sous-module de \( \modE\) contenant \( aX+b\) avec \( b\neq 0\), alors \( \modF\) contient \( X(aX+b)=bX\) et donc contient tout \( \modE\).
\end{example}

\begin{definition}[Algèbre\cite{ZSyHmiy}]   \label{DefAEbnJqI}
    Si \( \eK\) est un corps commutatif\footnote{Définition~\ref{DefTMNooKXHUd}}, une \( \eK\)-\defe{algèbre}{algèbre} \( A\) est un espace vectoriel\footnote{Définition~\ref{DEFooKHWZooIfxdNc}.} muni d'une opération bilinéaire \( \times\colon A\times A\to A\), c'est à dire telle que pour tout \( x,y,z\in A\) et pour tout \( \alpha,\beta\in\eK\),
    \begin{enumerate}
        \item
            \( (x+y)\times z=x\times z+y\times z\)
        \item
            \( x\times (y+z)=x\times y+x\times z\)
        \item
            \( (\alpha x)\times (\beta y)=(\alpha\beta)(x\times y)\).
    \end{enumerate}
    Si \( A\) et \( B\) sont deux \( \eK\)-algèbres, une application \( f\colon A\to B\) est un \defe{morphisme d'algèbres}{morphisme!d'algèbres} entre \( A\) et \( B\) si pour tout \( x,y\in A\) et pour tout \( \alpha\in \eK\),
    \begin{enumerate}
        \item
            \( f(xy)=f(x)f(y)\)
        \item
            \( f(x+\alpha y)=f(x)+\alpha f(y)\)
    \end{enumerate}
    où nous avons noté \( xy\) pour \( x\times y\).
\end{definition}

\begin{lemma}[\cite{MonCerveau}]   \label{LEMooVKLKooSAHmpZ}
    Soient une algèbre \( A\) et une famille \( (X_i)_{i\in I}\) de sous-algèbres de \( A\) (ici \( I\) est un ensemble quelconque). Alors la partie \( X=\bigcap_{i\in I}X_i\) est une sous-algèbre de \( A\).
\end{lemma}

\begin{proof}
    Nous devons prouver que si \( x\) et \( y\) sont dans \( X\) et \( \lambda\in \eK\), alors \( xy\), \( x+y\) et \( \lambda x\) sont dans \( X\). Pour tout \( i\in I\) nous avons \( x,y\in X_i\) et donc \( xy\in X_i\), \( x+y\in X_i\) et \( \lambda x\in X_i\) (parce que \( X_i\) est une algèbre). Donc \( xy\),\( x+y\) et \( \lambda x\) sont dans \( X_i\) pour tout \( I\), et donc dans \( X\).
\end{proof}

\begin{definition}\label{DefkAXaWY}
    L'\defe{algèbre engendrée}{algèbre!engendrée} par \( X\) est l'intersection de toutes les sous-algèbres de \( A\) contenant \( X\) (qui est une algèbre par le lemme~\ref{LEMooVKLKooSAHmpZ}).
\end{definition}

%+++++++++++++++++++++++++++++++++++++++++++++++++++++++++++++++++++++++++++++++++++++++++++++++++++++++++++++++++++++++++++
\section{Anneau intègre}
%+++++++++++++++++++++++++++++++++++++++++++++++++++++++++++++++++++++++++++++++++++++++++++++++++++++++++++++++++++++++++++
\label{SECAnneauxIntegres}

La définition d'un anneau intègre est la définition~\ref{DEFooTAOPooWDPYmd}.

\begin{example}     \label{EXooMXNTooZaRPPi}
    Un corps\footnote{Définition~\ref{DefTMNooKXHUd}.} est toujours un anneau intègre. En effet, soient un corps \( \eK\) et deux éléments \( x,y\in \eK\) tels que \( xy=0\). Si \( y\) est inversible, alors nous pouvons multiplier par \( y^{-1}\) pour trouver \( x=0\). Cela prouve que \( \eK\) est un anneau intègre.
\end{example}

\begin{example}     \label{EXooLDXRooSxUAXs}
    L'ensemble \( \eZ\) avec les opérations usuelles est un anneau intègre.
\end{example}

\begin{example}
    L'anneau \( \eZ/6\eZ\) n'est pas intègre parce que \( 3\cdot 2=0\) alors que ni \( 3\) ni \( 2\) ne sont nuls.
\end{example}

Nous verrons au théorème~\ref{ThoBUEDrJ} que l'anneau \( A\) est intègre si et seulement si \( A[X]\) est intègre.

\begin{corollary}   \label{CorZnInternprem}
    L'anneau \( \eZ/n\eZ\) est intègre si et seulement si \( n\) est premier.
\end{corollary}

\begin{proof}
    Supposons que \( n\) soit premier. La proposition \ref{PropZpintssiprempUzn} donne les inversibles de \( \eZ/n\eZ\) par
    \begin{equation}
        U(\eZ/n\eZ)=\{ \tilde x\tq 0\leq x\leq n\tq\pgcd(x,n)=1 \}.
    \end{equation}
    Mais comme \( n\) est premier, \( \pgcd(x,n)=1\) pour tout \( x\), et donc tous les éléments de \( \eZ/n\eZ\) sont inversibles. Donc \( \eZ/n\eZ\) est intègre.

    Si \( n\) n'est pas premier, alors \( n=pq\) avec \( 1<p\leq q<n\). Alors
    \begin{equation}
        [p]_n[q]_n=[pq]_n=[0]_n.
    \end{equation}
    Donc lorsque \( n\) n'est pas premier,  l'anneau \( \eZ/n\eZ\) possède des diviseurs de zéro et n'est alors pas intègre.
\end{proof}

%---------------------------------------------------------------------------------------------------------------------------
\subsection{Caractéristique d'un anneau intègre}
%---------------------------------------------------------------------------------------------------------------------------

\begin{lemma}       \label{LemCaractIntergernbrcartpre}
    La caractéristique\footnote{Définition~\ref{LEMDEFooVEWZooUrPaDw}.} d'un anneau intègre est zéro ou un nombre premier.
\end{lemma}

\begin{proof}
    Si \( A\) est intègre, alors \( \eZ 1_A\) est a fortiori intègre. Notons \( p \) la caractéristique de \( A \). Si \( p = 0 \), la preuve est finie; supposons donc que \( p \neq 0 \). Alors, l'anneau \( \eZ/p\eZ\) est isomorphe à \( \eZ 1_A\), et est donc intègre. Or, la proposition~\ref{CorZnInternprem} dit que \( \eZ/p\eZ\) est intègre si et seulement si \( p\) est premier, ce qui conclut la preuve.
\end{proof}

\begin{example}
    Il existe des corps dont la caractéristique n'est pas égale au cardinal (contrairement à ce que laisserait penser l'exemple des \( \eZ/p\eZ\)). En effet les matrices \( n\times n\) inversibles sur \( \eF_{3}\) forment un corps qui n'est pas de cardinal trois alors que la caractéristique est \( 3\) :
    \begin{equation}
        \begin{pmatrix}
            1    &       \\
                &   1
            \end{pmatrix}+\begin{pmatrix}
                1    &       \\
                    &   1
                \end{pmatrix}+\begin{pmatrix}
                    1    &       \\
                        &   1
                \end{pmatrix}=0.
    \end{equation}
\end{example}

\begin{example}
    Si \( \eK\) est un corps de caractéristique \( 2\), alors l'égalité \( x=-x\) n'implique pas \( x=0\), vu que \( 2x=0\) est vérifiée pour tout \( x\). Cela se répercute sur un certain nombre de résultats. Par exemple, en caractéristique deux, une forme antisymétrique n'est pas toujours alternée: voir le lemme~\ref{LemHiHNey}.
\end{example}

%---------------------------------------------------------------------------------------------------------------------------
\subsection{Divisibilité et classes d'association}
%---------------------------------------------------------------------------------------------------------------------------
\label{DivisibiliteAnneauxIntegres}

\begin{lemma}\label{LemRmVTRq}
    Si \( A\) est un anneau intègre et si \( a,b\in A\) sont tels que \( a\divides b\) et \( b\divides a\), alors il existe un inversible \( u\in A\) tel que \( a=ub\).
\end{lemma}

\begin{proof}
    Les hypothèses à propos de la divisibilité nous indiquent que \( a=xb\) et \( b=ya\) pour certains \( x,y\in A\). Du coup,
    \begin{equation}
        b(1-yx)=0.
    \end{equation}
    Étant donné que \( \eA\) est intègre, cela montre que \( b=0\) ou \( 1-yx=0\). Si \( b=0\) nous avons immédiatement \( a=0\) et le lemme est prouvé. Si au contraire \( yx=1\), c'est que \( y\) et \( x\) sont inversibles et inverses l'un de l'autre.
\end{proof}

\begin{definition}\label{DefrXUixs}
    On dit de deux éléments \( a,b\in A\) qu'ils sont \defe{associés}{associés!éléments d'un anneau} si ils vérifient les hypothèses du lemme~\ref{LemRmVTRq}; en d'autres termes, $a$ et $b$ sont associés s'il existe un inversible \( u\in A\) tel que \( a=ub\).

    La \defe{classe d'association}{classe d'association}\index{classe!d'association} d'un élément \( a \in A \) est l'ensemble des éléments qui lui sont associés; en d'autres termes, c'est \( a  U(A) \).
\end{definition}

\begin{example}
    Dans \( \eZ[i]\), les inversibles sont \( \pm 1\) et \( \pm i\). Donc les éléments associés à \( z\) sont \( z\), \( -z\), \( iz\) et \( -iz\).

    Notons au passage que la notion de divisibilité dans \( \eZ[i]\) n'est pas immédiatement intuitive. En effet bien que \( 7\) ne soit pas divisible par \( 2\) (ni dans \( \eZ\) ni dans \( \eZ[i]\)), le nombre \( 7+6i\) est divisible par \( 2+i\) dans \( \eZ[i]\). En effet :
    \begin{equation}
        (2+i)(4+i)=7+6i.
    \end{equation}
\end{example}

\begin{probleme}
    Est-ce que quelqu'un connaît un anneau contenant \( \eZ\) dans lequel \( 7\) est divisible en \( 2\) ?

    Peut-être \( \eZ\) étendu par tous les \( 1/2^n\) ?
\end{probleme}

%---------------------------------------------------------------------------------------------------------------------------
\subsection{PGCD et PPCM}
%---------------------------------------------------------------------------------------------------------------------------

Pour le théorème de Bézout et autres opérations avec des modulo, voir le thème~\ref{THEMEooNRZHooYuuHyt}.

Commençons par donner une autre vision de la divisibilité dans les anneaux intègres.
\begin{proposition}\label{PropDiviseurIdeaux}
    Dans un anneau intègre\footnote{Définition \ref{DEFooTAOPooWDPYmd}.} $A$, on a l'équivalence suivante concernant deux éléments \( a, b \in A \):
\begin{equation}
    a\divides b\Leftrightarrow (b)\subset (a).
\end{equation}
\end{proposition}

Donc la divisibilité devient en réalité une relation d'ordre dont nous pouvons chercher un maximum et un minimum. Si \( S\) est une partie de \( A\), nous notons \( a\divides S\) pour exprimer que \( a\divides x\) pour tout \( x\in S\); de la même façon, \( S\divides b\) signifie que \( x\divides b\) pour tout \( x\in S\)

\begin{definition}\label{DefrYwbct}
    Soient \( A\) un anneau intègre et \( S\subset A\). Nous disons que \( \delta\in A\) est un \defe{PGCD}{PGCD!dans un anneau intègre} de \( S\) si
    \begin{enumerate}
        \item
            \( \delta\divides S\)
        \item
            si \( d\divides S\) alors\footnote{Il me semble qu'à ce niveau il y a une faute de frappe dans \cite{XPXxPl}.} \( d\divides \delta\).
    \end{enumerate}
    Nous disons que \( \mu\in A\) est un \defe{PPCM}{PPCM!dans un anneau intègre} de \( S\) si
    \begin{enumerate}
        \item
            \( S\divides \mu\),
        \item
            si \( S\divides m\), alors \( \mu\divides m\).
    \end{enumerate}
\end{definition}
Notons qu'il n'y a en général pas unicité du PGCD ou du PPCM d'un ensemble.

\begin{lemma}
    Soent \( A\) un anneau intègre et \( S\subset A\). Si \( \delta\) est un PGCD de \( S\), alors l'ensemble des PGCD de \( S\) est la classe d'association de \( \delta\).

    De la même façon si \( \mu\) est un PPCM de \( S\), alors l'ensemble des PPCM de \( S\) est la classe d'association de \( \mu\).
\end{lemma}

\begin{proof}
    Soient \( \delta\) un PGCD de \( S\) et \( u\) un inversible dans \( A\). Si \( x\in S\) nous avons \( \delta\divides x\) et donc \( x=a\delta\). Par conséquent \( x=au^{-1}u\delta\) et donc \( u\delta\) divise \( x\). De la même manière, si \( d\) divise \( x\) pour tout \( x\in S\), alors \( d\) divise \( \delta\) et donc \( \delta=ad\) et \( u\delta=aud\), ce qui signifie que \( d\) divise \( u\delta\).

    Dans l'autre sens nous devons prouver que si \( \delta'\) est un autre PGCD de \( S\), alors il existe un inversible \( u\in \eA\) tel que \( \delta'=u\delta\). Vu que \( \delta'\) divise \( x\) pour tout \( x\in S\), nous avons \( \delta'\divides \delta\), et symétriquement nous trouvons \( \delta\divides\delta'\). Par conséquent (lemme~\ref{LemRmVTRq}), il existe un inversible \( u\) tel que \( \delta=u\delta'\).

    Le même type de raisonnement tient pour le PPCM.
\end{proof}

Si \( \delta\) est un PGCD de \( S\), nous dirons \emph{par abus de langage} que \( \delta\) est \emph{le} PGCD de \( S\), gardant en tête qu'en réalité toute sa classe d'association est PGCD. Nous noterons aussi, toujours par abus que \( \delta=\pgcd(S)\).

\begin{remark}
    La classe d'association d'un élément n'est pas toujours très grande. Les inversibles dans \( \eZ\) étant seulement \( \pm 1\), nous pouvons obtenir l'unicité du PGCD et du PPCM en imposant qu'ils soient positifs.

    Pour les polynômes, nous obtenons l'unicité en demandant que le PGCD soit unitaire.

    Dans les cas pratiques, il y a donc en réalité peu d'ambiguïté à parler du PGCD ou du PPCM d'un ensemble.
\end{remark}


%---------------------------------------------------------------------------------------------------------------------------
\subsection{Anneaux intègres et corps}
%---------------------------------------------------------------------------------------------------------------------------


Le fait d'être intègre pour un anneau n'assure pas le fait d'être un corps. Nous avons cependant ce résultat pour les anneaux finis.

\begin{proposition}     \label{PropanfinintimpCorp}
    Un anneau fini intègre est un corps.
\end{proposition}

\begin{proof}
    Soit \( A\) un tel anneau. Soit \( a\neq 0\). Les applications
    \begin{subequations}
        \begin{align}
            l_a\colon x\to ax\\
            r_a\colon x\to xa
        \end{align}
    \end{subequations}
    sont injectives. En tant que applications injectives entre ensembles finis, elles sont surjectives. Il existe donc \( b\) et \( c\) tels que \( 1=ba=ac\). Il se fait que \( b\) et \( c\) sont égaux parce que
    \footnote{Il faut être un peu souple avec les notations communément employées dans les ouvrages mathématiques, et que nous reprenons telles quelles. Dans l'équation qui suit, \( b(ac)\) est le produit de \( b\) par l'élément \( ac\), et non quelque chose comme le produit de \( b\) avec l'idéal \( (ac)\) par exemple.}
    \begin{equation}
        b=b(ac)=(ba)c=c.
    \end{equation}
    Par conséquent \( b\) est un inverse de \( a\).
\end{proof}



\begin{proposition}     \label{PropzhFgNJ}
    Soit \( n\in\eN^*\). Les conditions suivantes sont équivalentes :
    \begin{enumerate}
        \item
            \( n\) est premier.
        \item
            \( \eZ/n\eZ\) est un anneau intègre.
        \item
            \( \eZ/n\eZ\) est un corps.
    \end{enumerate}
\end{proposition}

\begin{proof}
    L'équivalence entre les deux premiers points est le contenu du corollaire~\ref{CorZnInternprem}. Le fait que \( \eZ/n\eZ\) soit un corps lorsque \( \eZ/n\eZ\) est intègre est la proposition~\ref{PropanfinintimpCorp}. Le fait que \( \eZ/n\eZ\) soit intègre lorsque \( \eZ/n\eZ\) est un corps est une propriété générale des corps : ce sont en particulier des anneaux intègres (lemme~\ref{LemAnnCorpsnonInterdivzer}).
\end{proof}

%--------------------------------------------------------------------------------------------------------------------------- 
\subsection{Élément irréductible}
%---------------------------------------------------------------------------------------------------------------------------

\begin{definition}[Élément irréductible\cite{ooWUNIooXKxRya}]  \label{DeirredBDhQfA}
    Un élément d'un anneau commutatif est \defe{irréductible}{irréductible!dans un anneau} si il n'est ni inversible, ni le produit de deux éléments non inversibles.
\end{definition}

\begin{normaltext}
    Nous allons voir dans la section \ref{SECooSWGKooEeOZTO} que le concept d'élément irréductible n'est vraiment utile que dans le cas des anneaux intègres.
\end{normaltext}

\begin{example}
    Un corps n'a pas d'éléments irréductibles parce qu'à part zéro tous les éléments sont inversibles. Mais \( 0\) n'est pas irréductible parce qu'il peut être écrit comme produit d'éléments non inversibles : \( 0=0\cdot 0\).
\end{example}

\begin{example}
    Les éléments irréductibles de l'anneau \( \eZ\) sont les nombres premiers. En effet les seuls inversibles de \( \eZ\) sont \( \pm 1\). Si \( p\) est premier et \( p=ab\) avec \( a,b\in \eZ\), alors nous avons soit \( a=\pm 1\) soit \( b=\pm 1\).
\end{example}

%+++++++++++++++++++++++++++++++++++++++++++++++++++++++++++++++++++++++++++++++++++++++++++++++++++++++++++++++++++++++++++
\section{Anneau factoriel}
%+++++++++++++++++++++++++++++++++++++++++++++++++++++++++++++++++++++++++++++++++++++++++++++++++++++++++++++++++++++++++++

\begin{definition}[Anneau factoriel]        \label{DEFooVCATooPJGWKq}
    Un anneau commutatif \( A\) est \defe{factoriel}{factoriel!anneau}\index{anneau!factoriel} s'il vérifie les propriétés suivantes.
    \begin{enumerate}
        \item
            L'anneau \( A\) est intègre (pas de diviseurs de zéro).
        \item
            Si \( a\in A\) est non nul et non inversible, alors il admet une décomposition en facteurs irréductibles: \( a=p_1\ldots p_k\) où les \( p_i\) sont irréductibles.
        \item
            Si \( a=q_1\ldots q_m\) est une autre décomposition de \( a\) en irréductibles, alors \( m=k\) et il existe une permutation\footnote{Définition~\ref{DEFooJNPIooMuzIXd}.} \( \sigma\in S_k\) telle que \( p_i\) et \( q_{\sigma(i)}\) soient associés\footnote{Définition~\ref{DefrXUixs}.}.
    \end{enumerate}
\end{definition}

Un anneau factoriel permet de caractériser le \( \pgcd\) et le \( \ppcm\) de nombres.

\begin{proposition}
Soit une famille \( \{ a_n \}\) d'éléments de \( A\) qui se décomposent en irréductibles comme
\begin{equation}
    a_i=\prod_k p_k^{\alpha_{k,i}}.
\end{equation}
Alors
\begin{equation}
    \pgcd\{ a_n \}=\prod_k p_k^{\min_i\{ \alpha_{k,i} \}}.
\end{equation}

De plus le PGCD est :
\begin{enumerate}
    \item
        Un multiple de tous les diviseurs communs des \( a_i\).
    \item
        Unique pour cette propriété à multiple près par un inversible\quext{Soyez prudent avec cette affirmation : je n'en n'ai pas de démonstrations sous la main et ne suis pas certain que ce soit vrai.}.
\end{enumerate}

\end{proposition}

De la même manière,
\begin{equation}
    \ppcm\{ a_n \}=\prod_kp_k^{\max_i\{ \alpha_{k,i} \}}.
\end{equation}
Un anneau factoriel a une relation de préordre partiel\index{ordre!sur un anneau factoriel} donnée par \( a<b\) si \( a\) divise \( b\). En termes d'idéaux, cela donne l'ordre inverse de celui de l'inclusion\footnote{Voir proposition~\ref{PropDiviseurIdeaux}.} : \( a<b\) si et seulement si \( (b)\subset (a)\).

\begin{example} \label{EXooCWJUooCDJqkr}
    L'anneau \( \eZ[i\sqrt{3}]\) n'est pas factoriel parce que \( 4\) a au moins deux décompositions distinctes en irréductibles :
    \begin{equation}
        4=2\cdot 2,
    \end{equation}
    et
    \begin{equation}
        4=(1+i\sqrt{3})(1-i\sqrt{3}).
    \end{equation}
\end{example}

Nous allons voir dans l'exemple~\ref{ExeDufyZI} que \( \eZ[i\sqrt{2}]\) est factoriel parce qu'il sera euclidien.

%+++++++++++++++++++++++++++++++++++++++++++++++++++++++++++++++++++++++++++++++++++++++++++++++++++++++++++++++++++++++++++
\section{Anneau principal et idéal premier}
%+++++++++++++++++++++++++++++++++++++++++++++++++++++++++++++++++++++++++++++++++++++++++++++++++++++++++++++++++++++++++++

\begin{definition}      \label{DEFooMZRKooBPLAWH}
    Un idéal \( I\) dans \( A\) est \defe{principal à gauche}{idéal!principal!à gauche} s'il existe \( a\in I\) tel que \( I= A a\). Il est \defe{principal à droite}{idéal!principal!à droite} s'il existe \( a\in I\) tel que \( I=a A\). Nous disons qu'il est \defe{principal}{principal!idéal} s'il est principal à gauche et à droite.
\end{definition}

\begin{definition}          \label{DEFooGWOZooXzUlhK}
    Un anneau est \defe{principal}{principal!anneau} si
    \begin{enumerate}
        \item
            il est commutatif et intègre
        \item
            tous ses idéaux sont principaux.
    \end{enumerate}
\end{definition}

Souvent pour prouver qu'un anneau est principal, nous prouvons qu'il est euclidien (définition~\ref{DefAXitWRL}) et nous utilisons la proposition~\ref{Propkllxnv} qui dit qu'un anneau euclidien est principal.

Une manière de prouver qu'un anneau n'est pas principal est de prouver qu'il n'est pas factoriel, théorème~\ref{THOooANCAooBChmwp}.

\begin{definition}      \label{DEFooAQSZooVhvQWv}
    Nous disons qu'un idéal \( I\) dans \( A\) est \defe{premier}{premier!idéal} si \( I\) est strictement inclus dans \( A\) et si pour tout \( a,b\in A\) tels que \( ab\in I\) nous avons \( a\in I\) ou \( b\in I\).
\end{definition}

\begin{lemma}       \label{LEMooYRPBooYxXXsi}
    L'idéal nul (réduit à \( \{ 0 \}\)) est premier si et seulement si \( A\) est intègre.
\end{lemma}

\begin{proof}
    En deux sens.
    \begin{subproof}
    \item[Si \( \{ 0 \}\) est premier]
        Alors \( A\neq \{ 0 \}\) parce que \( I=\{ 0 \}\) est propre (définition d'idéal premier).
        
        De plus, si \( ab=0\), alors \( ab\in I\) qui est un idéal premier. Donc soit \( a\) soit \( b\) est dans \( I\), c'est à dire que soit \( a\) soit \( b\) est nul. Donc \( A\) est intègre.

    \item[Si \( A\) est intègre]

        Alors \( A\neq \{ 0 \}\) et l'idéal \( I=\{ 0 \}\) est strictement inclus à \( A\). Si \( ab\in I\), alors \( ab=0\) et comme \( A\) est intègre, soit \( a\) soit \( b\) est nul, c'est à dire appartient à \( I\).
    \end{subproof}
\end{proof}

\begin{proposition}[\cite{ooWEUDooQybvIx}]      \label{PROPooRUQKooIfbnQX}
    Soit un anneau commutatif\footnote{Tous les anneaux du Frido sont commutatifs} et un idéal \( I\) dans \( A\).
    \begin{enumerate}
        \item       \label{ITEMooUGBTooOGrnWl}
            \( I\) est un idéal premier si et seulement si \( A/I\) est un anneau intègre.
        \item   \label{ITEMooGLXSooUjINqR}
            \( I\) est un idéal maximal si et seulement si \( A/I\) est un corps.
        \item       \label{ITEMooTFFQooOUajFw}
            Tout idéal maximal propre est premier.
    \end{enumerate}
\end{proposition}


\begin{proof}
    En plein d'étapes.
    \begin{subproof}
        \item[\( I\) premier implique \( A/I\) intègre]
            Évacuons le cas trivial pour être sûr. Si \( I=\{ 0 \}\) alors \( A\) est intègre par le lemme \ref{LEMooYRPBooYxXXsi}. Donc \( A/I=A/\{ 0 \}=A\) est intègre également.

            Soient \( a,b\in A\) tels que \( [a][b]=[0]\). Donc \( [ab]=[0]\), c'est à dire \( ab\in I\). Vu que \( I\) est un idéal premier nous avons \( a\in I\) ou \( b\in I\), c'est à dire \( [a]=0\) ou \( [b]=0\); nous en déduisons que \( A/I\) est un anneau intègre.
        \item[\( A/I\) intègre implique \( I\) premier]
            Soit \( ab\in I\). Alors \( [ab]=0\), ce qui signifie que \( [a][b]=0\) donc que \( [a]=0\) ou que \( [b]=0\) parce que \( A/I\) est intègre. Mais la condition \( [a]=0\) signifie \( a\in I\), et \( [b]=0\) signifie \( b\in I\). Nous avons donc prouvé que soit \( a\) soit \( b\) est dans \( I\), c'est à dire que \( I\) est premier.
        \item[Si \( I\) est un idéal maximum]

            Nous devons montrer que tout élément non nul de \( A/I\) est inversible. Un élément non nul de \( A/I\) est \( [x]\) avec \( x\in A\setminus I\). 
            
            Nous considérons \( J=Ax+I\), qui est un idéal parce que pour tout \( a\in A\), \( aAx+aI\in Ax+I\). Mais comme \( I\) est maximal, \( J=I\) ou \( J=A\).

            Si \( J=I\), nous aurions que pour tout \( a\in A\) et pour tout \( i\in I\), \( ax+i\in I\). En particulier pour \( a=1\) et \( i=0\) nous aurions \( x\in I\), ce qui est contraire à l'hypothèse faite sur \( x\).

            Donc \( J=A\). En particulier, \( 1\in J\), c'est à dire qu'il existe \( a\in A\) et \( i\in I\) tels que \( ax+i=1\). En passant aux classes, \( [ax]=1\), c'est à dire \( [a][x]=1\) qui signifie que \( [a]\) est un inverse de \( [x]\) dans \( A/I\).

        \item[Si \( A/I\) est un corps]

            Si \( x\in A\setminus I\), il faut prouver que tout idéal contenant \( I\) et \( x\) est \( A\).

            Un idéal contenant \( I\) et \( x\) doit contenir l'idéal \( J=Ax+I\). Vu que \( x\notin I\), nous avons \( [x]\neq 0\) dans \( A/I\). Donc \( [x] \) est inversible et il existe \( a\in A\) tel que \( [ax]=[A]\). C'est à dire que $ax-1\in I$. Nous avons alors
            \begin{equation}
                1=ax+\underbrace{(1-ax)}_{\in I}.
            \end{equation}
            C'est à dire que \( 1\in Ax+I\) et donc \( Ax+I=A\).
    \end{subproof}
    Enfin nous prouvons que tout idéal maximal propre est premier. 

    Si \( I\) est maximal, \( A/I\) est un corps par le point \ref{ITEMooGLXSooUjINqR}, et vu que \( I\) est propre, le corps \( A/I\) n'est pas réduit à \( \{ 0 \}\). Donc le lemme \ref{LemAnnCorpsnonInterdivzer} dit que \( A/I\) est un anneau intègre. Le point \ref{ITEMooUGBTooOGrnWl} dit alors que \( I\) est un idéal premier.
\end{proof}

\begin{remark}
    Vu qu'un corps peut être réduit à \( \{0\}\), dans \ref{ITEMooGLXSooUjINqR}, l'idéal peut être \( A\). Mais pas dans \ref{ITEMooTFFQooOUajFw}, parce qu'un idéal premier est propre, ça fait partie de la définition \ref{DEFooAQSZooVhvQWv}.
\end{remark}

\begin{proposition}[\cite{ooOYKZooOJBDHS}]     \label{PROPooHABIooBZZQMj}
    Si \( A\) est un anneau commutatif intègre, alors un idéal \( I\) dans \( A\) est premier si et seulement si \( A/I\) est intègre.
\end{proposition}

\begin{proof}
    Supposons que \( I\) soit un idéal premier. Si \( \bar a,\bar b\in A/I\)  sont tels que \( \bar a\bar b=0\), alors \( \overline{ ab }=0\), ce qui signifie que \( ab\in I\). Mais alors, vu que \( I\) est premier, soit \( a\) soit \( b\) est dans \( I\). Cela signifie que soit \( \bar a\) soit \( \bar b\) est nul dans \( A/I\). Cela prouve que \( A/I\) est un anneau intègre.

    Dans l'autre sens, nous supposons que \( A/I\) est intègre. Cela implique immédiatement que \( I\neq A\) parce que \( A/A\) n'est pas un anneau intègre (tout le monde est évidemment diviseur de zéro).

    Soient donc \( a,b\in A\) tels que \( ab\in I\). Alors \( \bar a\bar b= \overline{ ab }=0\) dans \( A/I\), mais comme \( A/I\) est intègre, cela implique que soit \( \bar a\) soit \( \bar b\) est nul. Autrement dit, soit \( a\) soit \( b\) est dans \( I\).
\end{proof}

\begin{proposition}[Thème~\ref{THEMEooZYKFooQXhiPD}, \cite{MonCerveau}] \label{PropomqcGe}
    Soit \( A\) un anneau principal qui n'est pas un corps. Pour un idéal propre \( I\) de \( A\), les conditions suivantes sont équivalentes :
    \begin{enumerate}
        \item       \label{ITEMooNOVFooEHtcwE}
            \( I\) est un idéal maximal\footnote{Définition \ref{DEFIdealMax}.};
        \item       \label{ITEMooMQWVooNocVEU}
            \( I\) est un idéal premier non nul\footnote{Définition \ref{DEFooAQSZooVhvQWv}.};
        \item       \label{ITEMooJBXGooEISNuW}
            il existe \( p\) irréductible\footnote{Définition \ref{DeirredBDhQfA}.} dans \( A\) tel que \( I=(p)\).
    \end{enumerate}
\end{proposition}

\begin{proof}
    En plusieurs implications.
    \begin{subproof}
        \item[\ref{ITEMooNOVFooEHtcwE} implique~\ref{ITEMooMQWVooNocVEU}]

            Par hypothèse, \( I\) est un idéal propre, de plus il n'est pas égal à \( \{ 0 \}\), parce que lorsque \( A\) et \( \{ 0 \} \) sont les seuls idéaux, nous avons un corps (proposition~\ref{PROPooUOCVooZGAVVk}). Étant donné que \( I\) est un idéal maximal, le quotient \( A/I\) est un corps par la proposition~\ref{PROPooSHHWooCyZPPw}.

            Soient maintenant, pour entrer dans le vif du sujet, des éléments \( a,b\in A\) tels que \( ab\in I\). Dans le corps \( A/I\) nous avons \( \overline{ ab }=0\), et par définition du produit dans le quotient, \( \bar a\bar b=0\). Par intégrité de l'anneau \( A/I\) (un corps est un anneau intègre, exemple~\ref{EXooMXNTooZaRPPi}) nous avons soit \( \bar a=0\), soit \( \bar b=0\), soit les deux en même temps. Dans tous les cas, soit \( a\) soit \( b\) est dans \( I\).

        \item[\ref{ITEMooMQWVooNocVEU} implique~\ref{ITEMooJBXGooEISNuW}]

            Maintenant \( I\) est un idéal premier non réduit à \( \{ 0 \}\). Vu que \( A\) est un anneau principal, il existe \( x\in A\) tel que \( I=(x)\). Nous devons prouver que \( x\) peut être choisi irréductible; et nous allons faire plus : nous allons prouver que \( x\) ne peut être que irréductible\quext{ça me semble un peu trop facile. Lisez attentivement, et écrivez-moi pour dire si vous êtes d'accord ou pas.}.

            Supposons que \( x\) ne soit pas irréductible. Alors il existe \( a,b\in A\) non inversibles tels que \( x=ab\). Si \( a\in (x)\) alors il existe \( k\in A\) tel que \( a=xk\), et en particulier, \( a=abk\), c'est à dire \( 1=bk\) (parce que \( A\) est principal et donc intègre). Cela signifie que \( b\) est inversible alors que nous avions dit qu'il ne l'était pas. Nous en déduisons que \( a\) n'est pas dans \( (x)\). On montre de manière similaire que \( b\) n'est pas dans \( (x)\) non plus.

            Nous nous retrouvons donc avec \( a,b\in A\) tel que \( ab\in I\) sans que ni \( a\) ni \( b\) sont soient dans \( I\). Cela contredit le fait que \( I\) soit un idéal premier. En conclusion, \( x\) est irréductible.

        \item[\ref{ITEMooJBXGooEISNuW} implique~\ref{ITEMooNOVFooEHtcwE}]

            Nous avons \( I=(p)\) avec \( p\) irréductible dans \( A\). Supposons que \( J\) est un idéal différent de \( A\) contenant \( I\). Vu que \( A\) est principal, il existe \( y\in A\) tels que \( J=(y)\). En particulier \( p\in J\), donc \( p=ay\) pour un certain \( a\in A\). Mais \( p\) est irréductible, donc soit \( a\) est inversible, soit \( y\) est inversible. Si \( y\) est inversible, alors \( J=A\), ce qui est exclu. Si \( a\) est inversible, alors \( (y)=(p)\), et \( I=J\).
    \end{subproof}
\end{proof}

\begin{normaltext}
    Dans la proposition \ref{PropomqcGe}, l'hypothèse d'idéal propre est importante. En effet dans le cas \( I=A\), nous avons évidemment que \( I\) est un idéal maximum. Mais \( A\) n'est d'abord pas un idéal premier parce qu'un idéal premier doit être strictement inclus à l'anneau. Et ensuite, \( A\) est en général loin d'être garanti d'être égal à \( (p)\) pour un de ses éléments \( p\).
\end{normaltext}

\begin{proposition}     \label{PropoTMMXCx}
    Soit \( A \) un anneau principal, et soit \( p \in A \) un élément irréductible. Alors
    \begin{enumerate}
        \item
            \( (p)\) est un idéal maximum.
        \item       \label{ITEMooKPJQooWuPZXS}
            \( A/(p)\) est un corps.
    \end{enumerate}
\end{proposition}

\begin{proof}
    Nous notons \( I=(p)\). Soit un idéal \( J\) contenant \( I\). Vu que \( A\) est principal, \( J\) aussi est monogène : \( J=(q)\). Mais comme \( p\) est dans \( I\) qui est dans \( J\), il existe \( a\in A\) tel que \( p=qa\).

    Vu que \( p\) est irréductible, soit \( q\) soit \( a\) est inversible. Si \( q\) est inversible, alors \( J=A\). Si \( a\) est inversible, alors nous avons \( p=qa\), donc \( q=pa^{-1}\), ce qui signifie que \( q\in(p)\) et donc que \( J=I\).

    Cela prouve que \( (p)\) est un idéal maximum.

    Le fait que \( A/(p)\) soit un corps est maintenant la proposition~\ref{PROPooSHHWooCyZPPw}.
\end{proof}

\begin{example}
    L'anneau \( \eZ\) est principal parce qu'il est intègre et que ses seuls idéaux sont les \( n\eZ\) qui sont principaux : \( n\eZ\) est engendré par \( n\).
\end{example}

\begin{example}[Les idéaux de $\eZ/n\eZ$]       \label{EXooCJRPooYkWdyr}

    Les idéaux de \( \eZ/n\eZ\) sont principaux, mais l'anneau \( \eZ/n\eZ\) n'est pas principal lorsque \( n\) n'est pas premier. Nous allons voir ça.

    \begin{subproof}
        \item[Les idéaux de \( \eZ/n\eZ\) sont principaux]

            Soit un idéal \( S\) dans \( \eZ/n\eZ\). Nous considérons la projection canonique \( \phi\colon \eZ\to \eZ/n\eZ\). La proposition~\ref{PropIJJIdsousphi} dit que  \( S=\phi(J)\) où \( J\) est un idéal de \( \eZ\) contenant \( n\eZ\). Mais le corollaire~\ref{CORooLINXooBlUKPG} nous dit qu'alors \( J=m\eZ\) pour un certain \( m\). Pour que \( m\eZ\) contienne \( n\eZ\), il faut que \( m\) divise \( n\).

            Bref, \( S=\phi(m\eZ)\) avec \( m\divides n\). Nous montrons maintenant que \( S\) est engendré par \( [m]_n\). D'abord, l'élément \( [m]_n\) est bien dans \( \phi(m\eZ)\). Ensuite un élément de \( \phi(m\eZ)\) est de la forme
            \begin{equation}
                [km]_n=k[m]_n\in ([m]_n).
            \end{equation}
            Donc \( S\subset ([m]_n)\). Et l'inclusion dans l'autre sens est tout aussi immédiate : un élément de \( ([m]_n)\) est de la forme
            \begin{equation}
                k[m]_n=[km]_n=\phi(km)\in \phi(m\eZ).
            \end{equation}

        \item[Si \( n\) n'est pas premier, \( \eZ/n\eZ\) n'est pas principal]

            Le fait est que lorsque \( n\) n'est pas premier, \( \eZ/n\eZ\) n'est pas intègre (corollaire~\ref{CorZnInternprem}).

        \item[Moralité]

            Un anneau comme \( \eZ/6\eZ\) est un anneau dont tous les idéaux sont principaux, mais qui n'est pas principal.

    \end{subproof}
\end{example}

\begin{example}
    Nous verrons dans la proposition~\ref{PROPooVWRPooGQMenV} que l'anneau des fonctions holomorphes sur un compact de \( \eC\) est principal.
\end{example}

\begin{definition}      \label{DEFooXSPFooPumQSy}
Nous disons que deux éléments d'un anneau principal sont \defe{premiers entre eux}{premier!deux éléments d'un anneau principal} si leur PGCD est \( 1\).
\end{definition}

\begin{theorem}\index{théorème!chinois!anneau principal}        \label{ThofPXwiM}
    Si \( A\) est un anneau principal et si \( p\) et \( q\) sont premiers entre eux dans \( A\), alors on a l'isomorphisme d'anneaux
    \begin{equation}
        A/pqA\simeq A/pA\times A/qA.
    \end{equation}
\end{theorem}
% TODO : trouver une preuve. Je parie que recopier la même que celle dans Z fonctionne très bien.

%---------------------------------------------------------------------------------------------------------------------------
\subsection{Bézout}
%---------------------------------------------------------------------------------------------------------------------------

\begin{theorem}[\cite{XPXxPl}]
    Toute partie \( S\) d'un anneau principal admet un PGCD et un PPCM. De plus
    \begin{equation}
        \begin{aligned}[]
            \delta=\pgcd(S)\Leftrightarrow (\delta)=\sum_{s\in S}(s)
            \mu=\ppcm(S)\Leftrightarrow (\mu)=\bigcap_{s\in S}(s)
        \end{aligned}
    \end{equation}
\end{theorem}

\begin{proof}
    Vu que l'anneau \( A\) est principal, tous ses idéaux sont principaux et donc engendrés par un seul élément. En particulier il existe \( \delta,\mu\in A\) tels que
    \begin{subequations}
        \begin{align}
            (\delta)&=\sum_{s\in S}(s)\\
            (\mu)&=\bigcap_{s\in S}(s)
        \end{align}
    \end{subequations}
    \begin{subproof}
    \item[PGCD]
        Montrons ce que \( \delta\) est un PGCD de \( S\). Pour tout \( x\in S\), nous avons \( (x)\subset (\delta)\), et donc \( \delta\divides x\). Par ailleurs si \( d\divides x\) pour tout \( x\in S\), nous avons \( (x)\subset (d)\) et donc
        \begin{equation}
            \sum_{x\in S}(x)\subset (d),
        \end{equation}
        puis \( (\delta)\subset (d)\) et finalement \( d\divides \delta\).
        \item[PPCM]
            Si \( x\in S\) nous avons \( (\mu)\subset (x)\) et donc \( x\divides \mu\). D'autre part si \( x\divides m\) pour tout \( x\in S\), alors \( (m)\subset (x)\) et donc \( (m)\subset(\mu)\), finalement \( \mu\divides m\).
    \end{subproof}
\end{proof}

\begin{corollary}[Théorème de Bézout\cite{XPXxPl}]\index{Bézout!anneau principal}\label{CorimHyXy}
    Soit un anneau principal \( A\). Deux éléments \( a,b\in A\) sont premiers entre eux si et seulement s'il existe un couple \( (u, v)\in A^2 \) tel que
    \begin{equation}
        ua+vb=1.
    \end{equation}
    À la place de \( 1\) on aurait pu écrire n'importe quel inversible.
\end{corollary}
\index{anneau!principal}

\begin{proof}
    Pour cette preuve, nous allons écrire \( \pgcd(a,b)\) l'ensemble de PGCD de \( a\) et \( b\), c'est à dire la classe d'association d'un PGCD.

    Si \( a\) et \( b\) sont premiers entre eux, alors
    \begin{equation}
        1\in\pgcd(a,b)=\sum_{x=a,b}(x)=(a)+(b).
    \end{equation}

    À l'inverse, si nous avons \( ua+vb=1\), alors \( 1\in (a)+(b)\), mais vu que \( (a)+(b)\) est un idéal principal, \( (1)=(a)+(b)\) et donc \( 1\in \pgcd(a,b)\).
\end{proof}

Le lemme de Gauss est une application immédiate de Bézout. Il y aura aussi un lemme de Gauss à propos de polynômes (lemme~\ref{LemEfdkZw}), et une généralisation directe au théorème de Gauss, théorème~\ref{ThoLLgIsig}.
\begin{lemma}[\href{http://ljk.imag.fr/membres/Bernard.Ycart/mel/ar/node6.html}{lemme de Gauss}]    \label{LemSdnZNX}
    Soit \( A\) un anneau principal et \( a,b,c\in A\) tels que \( a\) divise \( bc\). Si \( a\) est premier avec \( c\), alors \( a\) divise \( b\).
\end{lemma}
\index{lemme!Gauss!dans un anneau principal}

\begin{proof}
    Vu que \( a\) est premier avec \( c\), nous avons \( \pgcd(a,c)=1\) et Bézout (\ref{ThoBuNjam}) nous donne donc \( s,t\in \eA\) tels que \( sa+tc=1\). En multipliant par \( b\),
    \begin{equation}
        sab+tbc=b.
    \end{equation}
    Mais les deux termes du membre de gauche sont multiples de \( a\) parce que \( a\) divise \( bc\). Par conséquent \( b\) est somme de deux multiples de \( a\) et donc est multiple de \( a\).
\end{proof}
Un cas usuel d'utilisation est le cas de \( A=\eN^*\).

%--------------------------------------------------------------------------------------------------------------------------- 
\subsection{Élément premier}
%---------------------------------------------------------------------------------------------------------------------------

\begin{definition}[\cite{ooWBLYooLYwALS}]       \label{DEFooZCRQooWXRalw}
    Soit un anneau commutatif \( A\). Un élément \( p\in A\) est \defe{premier}{élément premier} si il est
    \begin{enumerate}
        \item
            non nul,
        \item
            non inversible,
        \item       \label{ITEMooPMTTooCVHPIm}
            si \( p\) divise un produit \( ab\), alors il divise soit \( a\) soit \( b\) (ou le deux).
    \end{enumerate}
\end{definition}

\begin{proposition}[\cite{ooTGPAooQTbamu}]     \label{PROPooWMNPooZdvOBt}
    Dans un anneau intègre\footnote{Si pas intègre, voir l'exemple \ref{EXooEIUEooCZCPMC}.} tout élément premier est irréductible\footnote{Toutes les définitions dans le thème \ref{THEMEooVIQIooOcFAQS}.}.
\end{proposition}
    
\begin{proof}
    Soit \( p\), un élément premier dans un anneau intègre \( A\).
    \begin{subproof}
        \item[\( p\) n'est pas inversible]
            Cela fait partie de la définition d'un élément premier.
        \item[\( p\) n'est pas un produit d'inversibles]
            Soient \( a,b\in A\) tels que \( p=ab\). Par le point \ref{ITEMooPMTTooCVHPIm} de la définition \ref{DEFooZCRQooWXRalw}, \( p\) divise soit \( a\) soit \( b\). Supposons que \( p\) divise \( a\). Alors il existe \( x\in A\) tel que \( a=px\). En remettant dans \( p=ab\) nous avons :
            \begin{equation}        \label{EQooPYBGooLFHMJZ}
                p=pxb.
            \end{equation}
            Mais l'anneau est intègre et permet donc des simplifications par tout élément non nul. La relation \ref{EQooPYBGooLFHMJZ} donne donc 
            \begin{equation}
                1=xb,
            \end{equation}
            ce qui signifie que \( b\) est inversible.

            Un travail similaire montre que \( a\) est inversible si \( p\) divise \( b\).
    \end{subproof}
\end{proof}

\begin{example}
    Si nous avons l'égalité \( 7=ab\) dans \( \eZ\), alors soit \( a\) soit \( b\) vaut \( 1\) et est donc inversible.
\end{example}

Sur un anneau non intègre, la notion d'élément premier n'est pas aussi intéressante que sur un anneau intègre. Par exemple la proposition \ref{PROPooWMNPooZdvOBt} devient fausse.

\begin{example}     \label{EXooEIUEooCZCPMC}
    Soit l'anneau \( \eZ^2\). L'élément \( (1,0)\) est premier mais pas irréductible.
    \begin{subproof}
        \item[\( (1,0)\) est premier]
            L'élément \( (1,0)\) est non nul, ok. Pour qu'il soit inversible, il faudrait \( (1,0)(x,y)=(1,1)\). Entre autres, \( 0\times y=1\), ce qui est impossible. Donc il n'est pas inversible.

            Supposons que \( (1,0)\) divise le produit \( (a,b)(c,d)=(ac,b)\). Alors il existe \( (x,y)\) tel que \( (1,0)(x,y)=(ac,bd)\). Cela signifie que \( x=ac\) et \( 0\times y=bd\). En particulier, soit \( b=0\) soit \( d=0\). Si \( b=0\), nous avons \( (a,b)=(a,0)\) et effectivement, \( (1,0)\) le divise.
        \item[\( (1,0)\) n'est pas irréductible]
            Nous avons \( (1,0)=(1,0)(1,0)\). Donc l'élément \( (1,0)\) est le produit de deux éléments non inversibles.
    \end{subproof}
\end{example}

\begin{example}
    Si \( \eK\) est un corps, l'élément \( XY\) de \( \eK[X,Y]\) n'est pas premier parce que \( XY\divides X^2Y^2\) alors que \( XY\) ne divise ni \( X^2\) ni \( Y^2\).
\end{example}


\begin{proposition}[\cite{ooJHFCooSbHtEC,MonCerveau}, thème \ref{THEMEooVIQIooOcFAQS}]     \label{PROPooZICGooNmblhl}
    Soit un anneau principal \( A\) et un élément \( p\neq 0\) dans \( A\). Nous avons équivalence de :
    \begin{enumerate}
        \item   \label{ITEMooBTEAooWlFUTX}
            \( (p)\) est un idéal premier,
        \item   \label{ITEMooKQRMooBNPDMX}
            \( p\) est un élément premier,
        \item   \label{ITEMooZYYJooCWiBhL}
            \( p\) est un élément irréductible,
        \item   \label{ITEMooHPAIooYoQzqD}
            \( (p)\) est un idéal maximum propre\quext{Ce «propre» n'est pas dans l'énoncé sur Wikipédia. Je ne comprends pas pourquoi, et j'ai posé la question sur la page de discussion.\\\url{https://fr.wikipedia.org/wiki/Discussion:Idéal_premier}}.
    \end{enumerate}
\end{proposition}

\begin{proof}
    En plusieurs implications.
    \begin{subproof}
        \item[\ref{ITEMooBTEAooWlFUTX} implique \ref{ITEMooKQRMooBNPDMX}]
            En plusieurs points.
            \begin{itemize}
                \item La condition \( p\neq 0\) est dans les hypothèses de la proposition.
                \item Si \( p\) était inversible, nous aurions \( (p)=A\) et donc pas que \( (p)\) est un idéal premier.
                \item Soient \( a,b\in A\) tels que \( p\divides ab\). En particulier, \( (ab)\in (p)\). Mais comme \( (p)\) est un idéal premier, cela implique soit \( a\in (p)\) soit \( b\in (p)\). Donc soit \( p\) divise \( a\) soit \( p\) divise \( b\).
            \end{itemize}
        \item[\ref{ITEMooKQRMooBNPDMX} implique \ref{ITEMooZYYJooCWiBhL}]
            Un anneau principal est intègre; c'est dans la définition \ref{DEFooGWOZooXzUlhK}. Dans un anneau intègre, tout élément premier est irréductible, c'est la proposition \ref{PROPooWMNPooZdvOBt}.
        \item[\ref{ITEMooZYYJooCWiBhL} implique \ref{ITEMooHPAIooYoQzqD}]
            Soit un idéal \( I\) contenant \( (p)\). Vu que \( A\) est principal, \( I\) est engendré par un seul élément; soit \( I=(a)\). Vu que \( p\in I\), l'élément \( a\) divise \( p\). Mais comme \( p\) est un élément premier, \( a\divides p\) implique \( a=p\) ou \( a=1\). Dans le premier cas, \( I=(a)=(p)\), et dans le second cas, \( I=(a)=(1)=A\). Donc \( (p)\) est bien un idéal maximum.

            De plus l'idéal \( (p)\) est propre. En effet avoir \( (p)=A\) dirait en particulier que \( 1\in (p)\), c'est à dire qu'il existe \( x\in A\) tel que \( xp=1\). Or \( p\) étant irréductible, il est non inversible.
        \item[\ref{ITEMooHPAIooYoQzqD} implique \ref{ITEMooBTEAooWlFUTX}]
            C'est la proposition \ref{PROPooRUQKooIfbnQX}\ref{ITEMooTFFQooOUajFw}.
    \end{subproof}
\end{proof}

Un exemple d'élément premier non irréductible est \( [4]_6\) dans l'anneau non principal \( \eZ/6\eZ\). Voir \ref{NORMooAXOKooAQMXoB} et le lemme \ref{LEMooZSELooGOFEIz}.

%---------------------------------------------------------------------------------------------------------------------------
\subsection{Anneau noethérien}
%---------------------------------------------------------------------------------------------------------------------------

\begin{definition}      \label{DEFooPWMHooCnrQuJ}
    Un anneau est dit \defe{noethérien}{anneau!noethérien} si toute suite croissante d'idéaux est stationnaire (à partir d'un certain rang).
\end{definition}

Montrer que tout anneau principal est noethérien est le premier pas pour montrer que tout anneau principal est factoriel.

\begin{lemma}       \label{LEMooHQPVooTfkhRV}
    Tout anneau principal\footnote{Définition \ref{DEFooGWOZooXzUlhK}.} est noethérien.
\end{lemma}

\begin{proof}
    Soit \( (J_n)\) une suite croissante d'idéaux et \( J\) la réunion. L'ensemble \( J\) est encore un idéal parce que les \( J_i\) sont emboités. Étant donné que l'idéal est principal nous pouvons prendre \( a\in J\) tel que \( J=(a)\). Il existe \( N\) tel que \( a\in J_N\). Alors pour tout \( n\geq N\) nous avons
    \begin{equation}
        J\subset J_N\subset J_n\subset J.
    \end{equation}
    La première inclusion est le fait que \( J=(a)\) et \( a\in J_N\). La seconde est la croissance des idéaux et la troisième est le fait que \( J\) est une union. Par conséquent pour tout \( n\geq N\) nous avons \( J_N=J_n=J\). La suite est par conséquent stationnaire.
\end{proof}

\begin{example}
    Il y a moyen d'avoir un anneau noetherien non principal. C'est le cas de \( \eZ/6\eZ\) dont nous parlerons dans \ref{LEMooZSELooGOFEIz}.
\end{example}

\begin{theorem}[\cite{FSwlnf}]      \label{THOooANCAooBChmwp}
    Tout anneau principal est factoriel.
\end{theorem}

\begin{example}[\( \eZ\lbrack i\sqrt{ 5 }\rbrack\) n'est ni factoriel ni principal]     \label{EXooYCTDooGXAjGC}
    Vu que \( (i\sqrt{ 5 })^2=-5\), les éléments de \( \eZ[i\sqrt{ 5 }]\) sont les éléments de \( \eC\) de la forme \( a+bi\sqrt{ 5 }\) avec \( a,b\in \eZ\). Nous définissons la \defe{norme}{norme!sur \( \eZ[i\sqrt{ 5 }]\)} sur \( \eZ[i\sqrt{ 5 }]\) par\footnote{C'est le carré de la norme usuelle, mais c'est l'usage dans le milieu.}
    \begin{equation}
        \begin{aligned}
            N\colon \eZ[i\sqrt{ 5 }]&\to \eN \\
            z&\mapsto z\bar z.
        \end{aligned}
    \end{equation}
    Le fait que ce soit à valeurs dans \( \eN\) est un simple calcul :
    \begin{equation}
        N(x+iy\sqrt{ 5 })=(x+iy\sqrt{ 5 })(x-iy\sqrt{ 5 })=x^2+5y^2.
    \end{equation}
    De plus \( N\) est multiplicative : \( N(z_1z_2)=N(z_1)N(z_2)\).

    Nous pouvons maintenant déterminer les inversibles de \( \eZ[i\sqrt{ 5 }]\). Si \( \alpha\) est inversible, alors il existe \( \beta\) tel que \( \alpha\beta=1\). Au niveau de la norme,
    \begin{equation}
        N(\alpha)N(\beta)=1,
    \end{equation}
    ce qui implique que \( N(\alpha)=1\). Or l'équation \( x^2+5y^2=1\) dans \( \eN\) donne \( y=0\), \( x=\pm 1\).

    Au final, les inversibles de \( \eZ[i\sqrt{ 5 }]\) sont \( \pm 1\).

    L'anneau \( \eZ[i\sqrt{ 5 }]\) n'est alors pas factoriel (définition~\ref{DEFooVCATooPJGWKq}) parce que
    \begin{equation}
        2\times 3=(1+i\sqrt{ 5 })(1-i\sqrt{ 5 }).
    \end{equation}
    Cela donne deux décompositions du nombre \( 6\) en produit d'éléments non associés\footnote{Définition~\ref{DefrXUixs}.} (\( 2\) n'est associé qu'à \( 2\) et \( -2\)) parce que les inversibles sont \( 1\) et \( -1\).

    Le fait que \( \eZ[i\sqrt{ 5 }]\) ne soit pas factoriel implique qu'il ne soit pas principal, théorème~\ref{THOooANCAooBChmwp}.
\end{example}

%+++++++++++++++++++++++++++++++++++++++++++++++++++++++++++++++++++++++++++++++++++++++++++++++++++++++++++++++++++++++++++ 
\section{Anneau \( \eZ/6\eZ\)}
%+++++++++++++++++++++++++++++++++++++++++++++++++++++++++++++++++++++++++++++++++++++++++++++++++++++++++++++++++++++++++++
\label{SECooSWGKooEeOZTO}

Nous allons donner quelque propriétés de cet anneau, et en particulier voir que dans cet anneau non intègre, la notion d'élément irréductible n'est pas très intéressante.

Voici pour commencer un calcul la table de multiplication de \( A=\eZ/6\eZ\). Pour les multiples de (par exemple) \( [4]_6\) nous écrivons
\begin{equation}
    1\times [4]_6=[4_6]
\end{equation}
et ensuite
\begin{equation}
    2\times [4]_6=[8]_6=[2]_6,
\end{equation}
puis
\begin{equation}
    3\times [4]_6=[2+4]_6=[6]_6=[0]_6,
\end{equation}
et caetera. Le résultat est :
\begin{equation}
\begin{array}{c|c|c|c|c|c|c}
    \times & [0]_6 & [1]_6  & [2]_6  & [3]_6 & [4]_6 & [5]_6  \\
\hline\hline
[0]_6 & 0 & 0 & 0 & 0 & 0 & 0 \\ 
\hline
[1]_6  & 0 & 1 & 2 & 3 & 4 & 5 \\ 
\hline
[2]_6 & 0 & 2 & 4 & 0 & 2 & 4 \\ 
\hline
[3]_6 & 0 & 3 & 0 & 3 & 0 & 3 \\ 
\hline
[4]_6 & 0 & 4 & 2 & 0 & 4 & 2 \\ 
\hline
[5]_6 & 0 & 5 & 4 & 3 & 2 & 1 \\ 
\hline
\end{array}
\end{equation}
Pour ne pas alourdir, nous n'avons pas écrit \( [x]_6\) partout au lieu de \( x\).

\begin{normaltext}[Inversibles]
    Les éléments inversibles de \( \eZ/6\eZ\) sont ceux qui ont un \( [1]_6\) dans leur table de multiplication. Ce sont donc
    \begin{equation}
        U(\eZ/6\eZ)=\big\{ [1]_6,[5]_6 \big\}.
    \end{equation}
\end{normaltext}

\begin{normaltext}[Diviseurs de zéro]
    Les diviseurs de zéro sont ceux qui ont un \( [0]_6\) dans leur table de multiplication, c'est à dire
    \begin{equation}
        \big\{ [2]_6,[3]_6,[4]_6 \big\}.
    \end{equation}
\end{normaltext}

\begin{normaltext}[Irréductibles]
    Les irréductibles sont ceux qui ne sont ni inversibles ni produit de deux éléments non inversibles. Les non inversibles sont :
    \begin{equation}
        \big\{ [0]_6,[2]_6,[3]_6,[4]_6 \}.
    \end{equation}
    Ils sont candidats à être irréductibles. Les produits de ces éléments (on oublie les crochets) sont :
    \begin{subequations}
        \begin{align}
            2\times 2&=4\\
            2\times 3&=0\\
            2\times 4&=2\\
            3\times 3&=3\\
            3\times 4&=0\\
            4\times 4&=4.
        \end{align}
    \end{subequations}
    Donc \( [0]_6\), \( [2]_6\), \( [3]_6\) et \( [4]_6\) ne sont plus candidats à être irréductible. Bref, il ne reste aucun candidats.

    L'anneau \( \eZ/6\eZ\) n'a aucun élément irréductible.
\end{normaltext}

\begin{normaltext}[Éléments premiers]       \label{NORMooAXOKooAQMXoB}
    Les éléments non nuls et non inversibles sont \( 2\), \( 3\) et \( 4\).
    \begin{subproof}
    \item[Pour \( 2\)]
        L'élément \( [2]_6\) divise \( 2\), \( 4\) et \( 0\).
        \begin{itemize}
            \item Les \( (a,b)\) tels que \( ab=2\) sont : $(1,2)$, \( (2,4)\) et \( (5,4)\). L'élément \( 2\) divise donc toujours \( a\) ou \( b\).
            \item Les \( (a,b)\) tels que \( ab=4\) sont : $(1,4)$, \( (2,5)\) et \( (4,4)\). L'élément \( 2\) divise donc toujours \( a\) ou \( b\).
            \item Les \( (a,b)\) tels que \( ab=0\) sont : \( (0,x)\),  $(3,2)$ et \( (4,3)\). L'élément \( 2\) divise donc toujours \( a\) ou \( b\). En particulier, \( [2]_6\) divise \( [0]_6\); c'est important.
        \end{itemize}
        Donc \( [2]_6\) est un élément premier.
    \item[Pour \( 3\)]
        L'élément \( [3]_6\) divise \( 3\) et \( 0\).
        \begin{itemize}
            \item Les \( (a,b)\) tels que \( ab=3\) sont : $(1,3)$ et \( (3,5)\). L'élément \( 3\) divise donc toujours \( a\) ou \( b\).
            \item Les \( (a,b)\) tels que \( ab=0\) sont : \( (0,x)\),  $(3,2)$ et \( (4,3)\). L'élément \( 3\) divise donc toujours \( a\) ou \( b\).
        \end{itemize}
        Donc \( [3]_6\) est un élément premier.
        L'élément \( [4]_6\) divise \( 4\), \( 2\) et \( 0\).
        \begin{itemize}
            \item Les \( (a,b)\) tels que \( ab=4\) sont : $(1,4)$, \( (2,5)\) et \( (4,4)\). L'élément \( 4\) divise donc toujours \( a\) ou \( b\).
            \item Les \( (a,b)\) tels que \( ab=2\) sont : $(1,2)$, \( (2,4)\) et \( (5,4)\). L'élément \( 4\) divise donc toujours \( a\) ou \( b\).
            \item Les \( (a,b)\) tels que \( ab=0\) sont : \( (0,x)\),  $(3,2)$ et \( (4,3)\). L'élément \( 4\) divise donc toujours \( a\) ou \( b\).
        \end{itemize}
        Donc \( [4]_6\) est un élément premier.
    \end{subproof}
    Au final, les éléments premiers dans \( \eZ/6\eZ\) sont 
    \begin{equation}
        \big\{ [2]_6, [3]_6, [4]_6  \big\}.
    \end{equation}
\end{normaltext}

Vous noterez que \( \eZ/6\eZ\) a des éléments premiers non irréductibles. Cela est un contre-exemple à la proposition \ref{PROPooZICGooNmblhl} dans le cas d'un anneau non-intègre.


\begin{lemma}[\cite{MonCerveau}]    \label{LEMooZSELooGOFEIz}
    L'anneau \( \eZ/6\eZ\) est noetherien, mais ni intègre ni principal\footnote{Toutes les définitions dans le thème \ref{THEMEooVIQIooOcFAQS}.}.
\end{lemma}

\begin{proof}
    Vu que c'est un anneau fini, toute suite croissante de quoi que ce soit devient stationnaire; donc \( \eZ/6\eZ\) est noetherien.

    Vu que \( \eZ/6\eZ\) a des diviseurs de zéro, il n'est pas intègre. Et vu qu'il n'est pas intègre, il n'est pas factoriel non plus.
\end{proof}
