% This is part of Mes notes de mathématique
% Copyright (c) 2011-2017
%   Laurent Claessens
% See the file fdl-1.3.txt for copying conditions.

%+++++++++++++++++++++++++++++++++++++++++++++++++++++++++++++++++++++++++++++++++++++++++++++++++++++++++++++++++++++++++++
\section{Généralités}
%+++++++++++++++++++++++++++++++++++++++++++++++++++++++++++++++++++++++++++++++++++++++++++++++++++++++++++++++++++++++++++

On rappelle la définition \ref{DefHXJUooKoovob} d'un anneau ainsi que les notations $0$ pour le neutre additif et $1$ pour le neutre multiplicatif. Certains ouvrages ne demandent pas spécialement de neutre pour la multiplication, et ajoutent le terme unitaire pour spécifier qu'il y a un neutre.

\begin{lemma}       \label{LEMooVUSMooWisQpD}
    Pour tout élément \( a\) d'un anneau nous avons \( a\cdot 0=0\).
\end{lemma}

\begin{proof}
    L'élément \( 0\) est le neutre de l'addition. Il peut être écrit \( 1-1\), et en utilisant la distributivité,
    \begin{equation}
        a\cdot 0=a\cdot (1-1)=a-a=0.
    \end{equation}
    Notons que la dernière égalité s'écrit en détail \( a-a=a+(-a)\) qui donne le neutre de l'addition.
\end{proof}

\begin{proposition}     \label{PROPooNCCGooXjVyVt}
    Dans un anneau non nul, le neutre pour l'addition est distinct du neutre pour la multiplication.
\end{proposition}
\begin{proof}
    Supposons par contraposée que dans un anneau $A$, \( 1 = 0 \). Alors, pour tout \( a \in A \), on a \( a = 1a = 0a = (1 - 1)a = a - a=0 \), d'où l'on déduit \( -a = 0  \) et par suite, \( a = 0. \)
\end{proof}

Soit \( X\) un ensemble et un anneau $(A, +, \times)$. Nous considérons \( \Fun(X,A)\)\nomenclature[A]{\( \Fun(X,Y)\)}{les applications de \( X\) vers \( Y\)} l'ensemble des applications \( X\to A\). Cet ensemble devient un anneau avec les définitions
\begin{subequations}
    \begin{align}
        (f+g)(x)=f(x)+g(x)\\
        (fg)(x)=f(x)g(x).
    \end{align}
\end{subequations}
Cela est la \defe{structure canonique}{structure d'anneau canonique} d'anneau sur \( \Fun(X,A)\).

Le \defe{centralisateur}{centralisateur} de \( x\in A\) dans \( A\) est l'ensemble
\begin{equation}
    \{ y\in A\tq xy=yx \},
\end{equation}
le \defe{centre}{centre!d'un anneau} de \( A\) est
\begin{equation}
    \{ y\in A\tq xy=yx,\forall x\in A \}.
\end{equation}

\begin{definition}[Diviseurs dans un anneau]\label{DiviseursAnneau}
	Soient \( a, b \in A \). On dit que $a$ divise $b$, ou que $a$ est un \defe{diviseur (à gauche)}{diviseur!dans un anneau} de $b$ s'il existe \( c \in A \) tel que \( ac = b \). On dit que c'est un diviseur de $b$ à droite si \( ca = b \) pour un certain \( c in A \).
\end{definition}
Un cas particulier est le cas des diviseurs de zéro. L'absence de tels diviseurs dans un anneau est une propriété intéressante: on dit dans ce cas que l'anneau est intègre. Nous étudions ces anneaux plus en détail en section \ref{SECAnneauxIntegres}.

Un élément \( a\in A\) est \defe{régulier à droite}{régulier à droite} si \( ba=0\) implique \( b=0\). Il est régulier à gauche si \( ab=0\) implique \( b=0\).

\begin{definition}[Éléments nilpotents, unipotents et inversibles]
	On dit que \( a \in A \) est \defe{nilpotent}{nilpotent} s'il existe \( n \in \eN \) tel que \( a^n = 0 \). Il est dit \defe{unipotent}{unipotent} si \( a-1\) est nilpotent, c'est à dire si \( (a-1)^n =0\) pour un certain \( n \in \eN \).
	
	Un élément \( a \in A \) est dit \defe{inversible}{élément!inversible!dans un anneau} s'il existe \( b \in A \) tel que \( ab = 1 \).
\end{definition}

L'ensemble \( U(A)\)\nomenclature[A]{\( U(A)\)}{ensemble des inversibles} des éléments inversibles de \( A\) est un groupe pour la multiplication. Nous notons \( A^*=A\setminus\{ 0 \}\).

La proposition suivante donne une caractérisation d'un corps, en disant un tout petit peu plus que la définition \ref{DefTMNooKXHUd}.
\begin{proposition}
    L'anneau $A$ est un corps si et seulement si \( U(A) = A^* \).
\end{proposition}

\begin{proof}
    L'inclusion \( A^*\subset U(A)\) est la définition.
    
    Pour l'inclusion inverse, il faut montrer que les inversibles ne peuvent pas être zéro, c'est à dire que zéro n'est pas inversible. Cela n'est autre que le lemme \ref{LEMooVUSMooWisQpD} couplé à la proposition \ref{PROPooNCCGooXjVyVt} : \( a\cdot 0=0\neq 1\) pour tout \( a\).
\end{proof}

\begin{lemma}
    Si \( a\) et \( b\) commutent, nous avons, pour tout \( r \in \eN \) et \( r > 0\), la formule
    \begin{equation}        \label{Eqarpurmkbk}
        a^{r+1}-b^{r+1}=(a-b)\left(\sum_{k=0}^ra^{r-k}b^k\right).
    \end{equation}
\end{lemma}

\begin{proof}
  Démontrons cela par récurrence. Le cas \( r = 0 \) est évident. Pour
  un \( r \) donné, si \eqref{Eqarpurmkbk} est vraie, alors
  \begin{align*}
    a^{r+2}-b^{r+2}&= a^{r+1}a - a^{r+1}b +a^{r+1}b - b^{r+1}b\\
    &= a^{r+1}(a - b) + (a^{r+1} - b^{r+1})b\\
    &= a^{r+1}(a - b) + (a-b)\left(\sum_{k=0}^ra^{r-k}b^k\right)b\\
    &= (a - b) \left(a^{r+1} + \left(\sum_{k=0}^ra^{r-k}b^k\right)b\right)\\
    &= (a - b) \left(a^{r+1} + \sum_{k=0}^ra^{r-k}b^{k + 1}\right)\\
    &= (a - b) \left(a^{r+1} + \sum_{k'=1}^{r+1}a^{(r+1)-k'}b^{k'}\right)\\
    &= (a - b) \left(\sum_{k'=0}^{r+1}a^{(r+1)-k'}b^{k'}\right).
  \end{align*}
\end{proof}

\begin{proposition}
    Si \( a\) est un élément nilpotent de l'anneau \( A\), alors \( 1-a\) est inversible. Si \( a\) est nilpotent non nul, alors il est diviseur de zéro.
\end{proposition}

\begin{proof}
    Soit \( n\) le minimum tel que \( a^n=0\). En vertu de la formule \eqref{Eqarpurmkbk} nous avons
    \begin{equation}
        1=1-a^n=(1-a)(1+a+\cdots+a^{n-1})=(1+a+\cdots+a^{n-1})(1-a).
    \end{equation}
    La somme \( 1+a+\cdots+a^{n-1}\) est donc un inverse de \( (1-a)\).
\end{proof}

\begin{definition}      \label{DEFooTCUOooWHlbee}
    Soient \( A\) un anneau et \( a,b\in A\). Nous disons que \( d\) est un \( \pgcd\)\index{pgcd} de \( a\) et \( b\) si tout diviseur commun de \( a\) et \( b\) divise \( d\).
\end{definition}

Nous rappelons également la définition \ref{DEFooQBGJooKJqHXr} de morphisme d'anneaux. Remarquons que si \( f\) est un morphisme, nous avons \( f(0)=0\) et \( f(x)^{-1}=f(x^{-1})\).

%+++++++++++++++++++++++++++++++++++++++++++++++++++++++++++++++++++++++++++++++++++++++++++++++++++++++++++++++++++++++++++ 
\section{Binôme de Newton et morphisme de Frobenius}
%+++++++++++++++++++++++++++++++++++++++++++++++++++++++++++++++++++++++++++++++++++++++++++++++++++++++++++++++++++++++++++

\begin{proposition}     \label{PropBinomFExOiL}
Pour tout $x$, $y\in\eR$ et $n\in\eN$, nous avons
\begin{equation}        \label{EqNewtonB}
    (x+y)^n=\sum_{k=0}^n{n\choose k}x^{n-k}y^k
\end{equation}
où
\begin{equation}
    {n\choose k}=\frac{ n! }{ k!(n-k)! }
\end{equation}
sont les \defe{coefficients binomiaux}{coefficients binomiaux}.
\end{proposition}

La preuve qui suit provient de \href{http://fr.wikipedia.org/wiki/Formule_du_binôme_de_Newton}{wikipédia}.
\begin{proof}
    La preuve se fait par récurrence. La vérification pour $n=0$ et $n=1$ se fait aisément pour peu que l'on se rappelle que \( x^0=1\) et que \( 0!=1\), ce qui donne entre autres \( {0\choose 0}=1\).
    
    Supposons que la formule \eqref{EqNewtonB} soit vraie pour $n\geq1$, et prouvons la pour $n+1$. Nous avons
\begin{equation}        \label{EqBinTrav}
    \begin{aligned}[]
        (x+y)^{n+1} &=(x+y)\cdot  \sum_{k=0}^n{n\choose k}x^{n-k}y^k\\
                &= \sum_{k=0}^n{n\choose k}x^{n-k+1}y^k+\sum_{k=0}^n{n\choose k}x^{n-k}y^{k+1}\\
                &=x^{n+1}+ \sum_{k=1}^n{n\choose k}x^{n-k+1}y^k+\sum_{k=0}^{n-1}{n\choose k}x^{n-k}y^{k+1}+y^{n+1}.
    \end{aligned}
\end{equation}
La seconde grande somme peut être transformée en posant $i=k+1$ :
\begin{equation}
    \sum_{k=0}^{n-1}{n\choose k}x^{n-k}y^{k+1}  =\sum_{i=1}^n{n\choose i-1}x^{n-(i-1)}y^{i-1+1},
\end{equation}
dans lequel nous pouvons immédiatement renommer $i$ par $k$. En remplaçant dans la dernière expression de \eqref{EqBinTrav}, nous trouvons
\begin{equation}
    (x+y)^{n+1}=x^{n+1}+y^{n+1}+\sum_{k=1}^n\left[ {n\choose k}+{n\choose k-1} \right]x^{n-k+1}y^k.
\end{equation}
La thèse découle maintenant de la formule
\begin{equation}
    {n\choose k}+{n\choose k-1}={n+1\choose k}
\end{equation}
qui est vraie parce que
\begin{equation}
    \frac{ n! }{ k!(n-k)! }+\frac{ n! }{ (k-1)(n-k+1)! }=\frac{ n!(n-k+1)+n!k }{ k!(n-k+1)! }=\frac{ n!(n+1) }{  k!(n-k+1)!  },
\end{equation}
par simple mise au même dénominateur.
\end{proof}

%+++++++++++++++++++++++++++++++++++++++++++++++++++++++++++++++++++++++++++++++++++++++++++++++++++++++++++++++++++++++++++ 
\section{Idéal dans un anneau}
%+++++++++++++++++++++++++++++++++++++++++++++++++++++++++++++++++++++++++++++++++++++++++++++++++++++++++++++++++++++++++++

La définition d'un idéal dans un anneau est la définition \ref{DefooQULAooREUIU}.


\begin{definition}[Idéal engendré par un élément]  \label{DefSKTooOTauAR}
    Si \( p\) est un élément d'un anneau \( A\) alors nous notons \( (p)\)\nomenclature[A]{\( (p)\)}{idéal engendré par \( p\)}\index{engendré!idéal dans un anneau} l'idéal dans \( A\) \defe{engendré}{engendré} par \( p\), c'est à dire \( pA\).
\end{definition}

\begin{definition}  \label{DefAJVTPxb}
    Un sous-ensemble \( B\subset A\) d'un anneau est un \defe{sous anneau}{sous anneau} si
    \begin{enumerate}
        \item
            \( 1\in B\)
        \item
            \( B\) est un sous-groupe pour l'addition
        \item
            \( B\) est stable pour la multiplication.
    \end{enumerate}
\end{definition}

\begin{remark}
    Un idéal n'est pas toujours un anneau parce que l'identité pourrait manquer. Un idéal qui contient l'identité est l'anneau complet.
\end{remark}

\begin{example}
    L'ensemble \( 2\eZ\) est un idéal de \( \eZ\). On peut aussi le noter \( (2) \).
\end{example}

Soit \( A\), un anneau, \( I\) un idéal bilatère\footnote{Définition \ref{DefooQULAooREUIU}.} de \( A\). Nous considérons la relation d'équivalence \( x\sim y\) si et seulement si \( x-y\in I\). Dans ce cas, le quotient
\begin{equation}
    A/\sim=A/I
\end{equation}
est un anneau appelé \defe{anneau quotient}{anneau!quotient par un idéal}. La surjection \( A\to A/I\) est un morphisme.

\begin{proposition}[Premier théorème d'isomorphisme pour les anneaux]
    Soient \( A\) et \( B\) des anneaux et un homomorphisme \( f\colon A\to B\). Nous considérons l'injection canonique \( j\colon f(A)\to B\) et la surjection canonique \( \phi\colon A\to A/\ker f\). Alors il existe un unique isomorphisme
    \begin{equation}
        \tilde f \colon A/\ker f\to f(A)
    \end{equation}
    tel que \( f=j\circ\tilde f\circ\phi\).

    \begin{equation}
        \xymatrix{%
        A \ar[r]^{f}\ar[d]_{\phi}        &   B\ar[d]^{j}\\
           A/\ker f \ar[r]_{\tilde f}   &   f(A)\subset B
           }
    \end{equation}
\end{proposition}
\index{théorème!isomorphisme!premier!pour les anneaux}

\begin{proposition}     \label{PropIJJIdsousphi}
    Soient \( I\) un idéal de \( A\) et \( \phi\colon A\to A/I\) la surjection canonique. Les idéaux de \( A/I\) sont les \( \phi(J)\) où \( J\) est un idéal de \( A\) contenant \( I\). De plus cette relation est bijective :
    \begin{equation}        \label{EqKbrizu}
        \{ \text{idéaux de } A\text{ contenant } I\}\simeq\{ \text{idéaux de } A/I \}.
    \end{equation}
\end{proposition}

\begin{proof}
    Si \( I\subset J\) et si \( J \) est un idéal de \( A\), alors \( \phi(J)\) est un idéal dans \( A/I\). En effet un élément de \( \phi(J)\) est de la forme \( \phi(j)\) et un élément de \( A/I\) est de la forme \( \phi(i)\). Leur produit vaut
    \begin{equation}
        \phi(i)\phi(j)=\phi(ij)\in\phi(J).
    \end{equation}
    
    Soit maintenant \( K\) un idéal dans \( A/I\) et soit \( J=\phi^{-1}(K)\). Étant donné qu'un idéal doit contenir \( 0\) (parce qu'un idéal est un groupe pour l'addition), \( [0]\in K\) et par conséquent \( I\subset\phi^{-1}(K)\).
\end{proof}
% TODO : il faudrait dire à peu près ici qu'une des utilités de Z_2 est le groupe modulaire PSL(2,Z)=SL(2,Z)/Z_2


%--------------------------------------------------------------------------------------------------------------------------- 
\subsection{Résultats supplémentaires sur l'anneau des entiers}
%---------------------------------------------------------------------------------------------------------------------------

\begin{corollary}       \label{CORooLINXooBlUKPG}
    Les quotients de \( \eZ\) sont \( \eZ/n\eZ\).
\end{corollary}

\begin{proof}
    Tous les idéaux de \( \eZ\) sont de la forme \( n\eZ\). En effet en vertu de la proposition \ref{PropSsgpZestnZ}, les seuls sous-groupes de \( \eZ\) (en tant que groupe additif) sont les \( n\eZ\). Tous les idéaux sont donc de cette forme. De plus les \( n\eZ\) sont effectivement tous des idéaux : si \( a\in n\eZ\) et si \( k\in \eZ\) alors \( ak\in n\eZ\). Cela est donc un idéal.
\end{proof}

\begin{proposition}     \label{PropZpintssiprempUzn}
    Soient \( n\geq 2\) un entier et \( \phi\colon \eZ\to \eZ/n\eZ\) la surjection canonique. Nous noterons \( \tilde a=\phi(a)\). Alors l'ensemble des inversibles de \( \eZ/n\eZ\) est donné par
    \begin{equation}
        U(\eZ/n\eZ)=\phi(P_n)=\{ \tilde x\tq 0\leq x\leq n\tq\pgcd(x,n)=1 \}.
    \end{equation}
    où \( P_n\) est l'ensemble $P_n=\{ x\in\{ 0,\ldots,n-1 \}\tq\pgcd(x,n)=1 \}$. En particulier, \( \Card\big( U(\eZ/n\eZ) \big)=\varphi(n)\).
\end{proposition}

\begin{proof}
    Soit \( 0\leq x\leq n\) tel que \( \pgcd(x,n)=1\). Il existe donc\footnote{Théorème de Bézout \ref{ThoBuNjam}} \( u,v\in\eZ\) tels que \( ux+vn=1\). En passant aux classes,
    \begin{equation}
        \tilde u\tilde x=\tilde 1,
    \end{equation}
    donc \( \tilde u\) est l'inverse de \( \tilde x\). Cela prouve que \( \phi(P_n)\subset U(\eZ/n\eZ)\).

    Nous prouvons maintenant l'inclusion inverse. Soit \( \tilde x\) et \( \tilde y\) inverses l'un de l'autre : $\tilde x\tilde y=\tilde 1$. Il existe donc \( q\in\eZ\) tel que \( xy-qn=1\), ce qui prouve\footnote{À nouveau avec le Théorème de Bézout.} que \( \pgcd(x,n)=1\).
\end{proof}

%+++++++++++++++++++++++++++++++++++++++++++++++++++++++++++++++++++++++++++++++++++++++++++++++++++++++++++++++++++++++++++ 
\section{Caractéristique}
%+++++++++++++++++++++++++++++++++++++++++++++++++++++++++++++++++++++++++++++++++++++++++++++++++++++++++++++++++++++++++++

\begin{lemmaDef}        \label{LEMDEFooVEWZooUrPaDw}
    Soit l'application
    \begin{equation}
        \begin{aligned}
            \mu\colon \eZ&\to A \\
            n&\mapsto n\cdot 1_A .
        \end{aligned}
    \end{equation}
    \begin{enumerate}
        \item
            C'est un morphisme d'anneaux.
        \item
            Le noyau est un sous-groupe de \( \eZ\)
        \item
            Il existe un unique \( p\in \eZ\) tel que \( \ker(\mu)=p\eZ\).
    \end{enumerate}
    Ce \( p\) est la \defe{caractéristique}{caractéristique!d'un anneau} de \( A\).
\end{lemmaDef}

Par exemple la caractéristique que \( \eQ\) est zéro parce qu'aucun multiple de l'unité n'est nul.

À propos de diagonalisation en caractéristique \( 2\), voir l'exemple \ref{ExewINgYo}.

\begin{lemma}
    Si \( A\) est de caractéristique nulle, alors \( A\) est infini.
\end{lemma}

\begin{proof}
    En effet, \( \ker\mu=\{0\} \) implique que \( n1_A \neq  m1_A\) dès que \(n \neq m \) et par conséquent \( A\) contient \(\eZ 1_A \), et  est infini.
\end{proof}

\begin{lemma}       \label{LemHmDaYH}
    Si \( p\) est la caractéristique de l'anneau \( A\), alors nous avons l'isomorphisme d'anneaux
    \begin{equation}
         \eZ 1_A\simeq\eZ/p\eZ.
    \end{equation}
\end{lemma}

\begin{proof}
    L'isomorphisme est donné par l'application \( n1_A\mapsto \phi(n)\) si \( \phi\) est la projection canonique \( \eZ\to \eZ/p\eZ\).
\end{proof}

\begin{proposition}     \label{PropGExaUK}
    La caractéristique d'un anneau fini divise son cardinal.
\end{proposition}

\begin{proof}
    Si \( A\) est un anneau, le groupe \( \eZ\) agit sur \( A\) par
    \begin{equation}
        n\cdot a=a+n1_A.
    \end{equation}
    Chaque orbite de cette action est de la forme
    \begin{equation}
        \mO_a=\{ a+n1_A\tq n=0,\ldots, p-1 \}
    \end{equation}
    où \( p\) est la caractéristique de \( A\). Les orbites ont \( p\) éléments et forment une partition de \( A\), donc le cardinal de \( A\) est un multiple de \( p\).
\end{proof}

\begin{lemma}[\cite{ooIBWOooSjOvXd}]        \label{LEMooJQIKooQgukqn}
    Un anneau totalement ordonné est de caractéristique nulle.
\end{lemma}

\begin{proof}
    Le morphisme \( \mu\colon \eZ\to A\), \( n\mapsto n 1_A\) est strictement croissant, en particulier \( \mu(x)\neq \mu(y)\) dès que \( x\neq y\). Donc \( \ker(\mu)=\{ 0 \}\).
\end{proof}

L'ensemble typique de caractéristique \( p\) est \( \eF_p=\eZ/p\eZ\).



\begin{proposition}     \label{Propqrrdem}
    Soit \( A\) un anneau commutatif de caractéristique première \( p\). Alors \( \sigma(x)=x^p\) est un automorphisme de l'anneau \( A\). Nous avons la formule
    \begin{equation}
        (a+b)^p=a^p+b^p
    \end{equation}
    pour tout \( a,b\in A\).
\end{proposition}

\begin{proof}
    Nous utilisons la formule du binôme de la proposition \ref{PropBinomFExOiL} et le fait que les coefficients binomiaux non extrêmes sont divisibles par \( p\) et donc nuls.
\end{proof}

\begin{proposition} \label{PropFrobHAMkTY}
    Soit \( A\) un anneau commutatif unitaire de caractéristique \( p\). L'application
    \begin{equation}
        \begin{aligned}
            \Frob_A\colon A&\to A \\
            x&\mapsto x^p 
        \end{aligned}
    \end{equation}
    est un automorphisme d'anneau unitaire.
\end{proposition}
Nous le nommons le \defe{morphisme de Frobenius}{morphisme!Frobenius}\index{Frobenius!morphisme}. Nous utiliserons aussi les itérés du morphisme de Frobenius : \( \Frob^k\colon x\mapsto x^{p^k}\).

\begin{example}
    Soit à factoriser \( X^p-1\) dans \( \eF_p\). Grâce au morphisme de Frobenius, nous avons immédiatement
    \begin{equation}
        X^p-1=(X-1)^p.
    \end{equation}
\end{example}

%+++++++++++++++++++++++++++++++++++++++++++++++++++++++++++++++++++++++++++++++++++++++++++++++++++++++++++++++++++++++++++
\section{Modules}
%+++++++++++++++++++++++++++++++++++++++++++++++++++++++++++++++++++++++++++++++++++++++++++++++++++++++++++++++++++++++++++

Si \( A\) est un anneau et si \( (M,+)\) est un groupe commutatif, nous disons que \( M\) est un \defe{\wikipedia{fr}{Module_sur_un_anneau}{module}}{module!sur un anneau} à gauche sur \( A\) si nous avons une application \( A\times M\to M\) notée \( a\cdot x\) telle que
\begin{enumerate}
    \item
       $\alpha\cdot(x + y) = a\cdot x + a\cdot y$  (distributivité de $\cdot$ par rapport à l'addition dans $M$)
   \item $(a + b) \cdot x = a \cdot x + b \cdot x$ (distributivité de $\cdot$ par rapport à l'addition dans \( \eA\)). Remarque :  la loi $+$ du membre de gauche est celle de l'anneau $A$ et la loi $+$ du membre de droite est celle du groupe $M$
   \item $(ab) \cdot x = a \cdot (b \cdot x$)
   \item $1 \cdot x = x$ 
\end{enumerate}

\begin{remark}
    La notion de module généralise celle d'espace vectoriel que nous verrons plus loin.
\end{remark}

Soient \( M\) un \( A\)-module et \( x=(x_i)_{i\in I}\) une famille d'éléments de \( M\) paramétrée par l'ensemble \( I\). Nous considérons l'application
\begin{equation}
    \begin{aligned}
        \mu_x\colon A^{(I)}&\to M \\
        (a_i)_{i\in I}&\mapsto \sum_{i\in I}a_ix_i.
    \end{aligned}
\end{equation}
Ici \( A^{(I)}\) désigne l'ensemble de toutes les applications \( I\to A\) de support fini.  

\begin{definition}      \label{DefBasePouyKj}
    À l'instar des espaces vectoriels, les modules ont une notion de partie libre, génératrice et de bases :
    \begin{enumerate}
        \item
            Si \( \mu_x\) est surjective, nous disons que \( x\) est une partie \defe{génératrice}{génératrice!partie d'un module}.
        \item
            Si \( \mu_x\) est injective, nous disons que la partie \( x\) est \defe{libre}{libre!partie d'un module}.
        \item
            Si \( \mu_x\) est bijective, nous disons que la partie \( x\) est une \defe{base}{base!d'un module}.
    \end{enumerate}
\end{definition}

\begin{definition}
  Un sous-ensemble \( N\subset M\) est un \defe{sous-module}{sous-module} si \( (N,+)\) est un sous-groupe de \( (M,+)\) et si \( a\cdot x\in N\) pour tout \( x\in N\) et pour tout \( a\in A\).
\end{definition}

\begin{example}
    Un anneau \( A\) est lui-même un \( A\)-module et ses sous-modules sont les idéaux.
\end{example}

\begin{definition}
    Soit \( M\) un module sur un anneau commutatif \( A\). Un \defe{projecteur}{projecteur!dans un module} est une application linéaire \( p\colon M\to M\) telle que \( p^2=p\).

    Une famille \( (p_i)_{i\in I}\) sur \( M\) est \defe{orthogonale}{orthogonal!famille de projecteurs} si \( p_i\circ p_j=0\) pour tout \( i\neq j\). La famille est \defe{complète}{complète!famille de projecteurs} si \( \sum_{i\in I}p_i=\mtu\).
\end{definition}

\begin{theorem}     \label{ThoProjModpAlsUR}
    Soient des sous modules \( M_1,\ldots,M_n\) du module \( M \) tels que \( M=M_1\oplus\ldots\oplus M_n\). Les applications \( p_i\) définies par
    \begin{equation}
        p_i(x_1+\ldots+x_n)=x_i
    \end{equation}
    forment une famille orthogonale de projecteurs et \( p_1+\cdots +p_n=\id\).

    Inversement, si \( (p_1,\ldots, p_n)\) est une famille orthogonale de projecteurs dans un module \( \modE\) tel que \( \sum_{i=1}^np_i=\id\), alors
    \begin{equation}
        M=\bigoplus_{i=1}^np_i(M).
    \end{equation}
\end{theorem}

Un module est \defe{simple}{simple!module}\index{module!simple} ou \defe{irréductible}{irréductible!module}\index{module!irréductible} s'il n'a pas d'autre sous-modules que \( \{ 0 \}\) et lui-même. Un module est \defe{indécomposable}{indécomposable!module}\index{module!indécomposable} s'il ne peut pas être écrit comme somme directe de sous-modules.

Un module simple est a fortiori indécomposable. L'inverse n'est pas vrai comme le montre l'exemple suivant.

\begin{example}
    Soit \( \modE=\eC[X]/(X^2)\) vu comme \( \eC[X]\)-module. C'est le \( \eC[X]\)-module des polynômes de la forme \( aX+b\) avec \( a,b\in \eC\). L'ensemble des polynômes de la forme \( aX\) est un sous-module. Le module \( \modE\) n'est donc pas simple. Il est cependant indécomposable parce que \( \{ aX \}\) est le seul sous-module non trivial. En effet si \( \modF\) est un sous-module de \( \modE\) contenant \( aX+b\) avec \( b\neq 0\), alors \( \modF\) contient \( X(aX+b)=bX\) et donc contient tout \( \modE\).
\end{example}

\begin{definition}[Algèbre\cite{ZSyHmiy}]   \label{DefAEbnJqI}
    Si \( \eK\) est un corps commutatif\footnote{Définition \ref{DefTMNooKXHUd}}, une \( \eK\)-\defe{algèbre}{algèbre} \( A\) est un espace vectoriel muni d'une opération bilinéaire \( \times\colon A\times A\to A\), c'est à dire telle que pour tout \( x,y,z\in A\) et pour tout \( \alpha,\beta\in\eK\),
    \begin{enumerate}
        \item
            \( (x+y)\times z=x\times z+y\times z\)
        \item
            \( x\times (y+z)=x\times y+x\times z\)
        \item
            \( (\alpha x)\times (\beta y)=(\alpha\beta)(x\times y)\).
    \end{enumerate}
    Si \( A\) et \( B\) sont deux \( \eK\)-algèbres, une application \( f\colon A\to B\) est un \defe{morphisme d'algèbres}{morphisme!d'algèbres} entre \( A\) et \( B\) si pour tout \( x,y\in A\) et pour tout \( \alpha\in \eK\),
    \begin{enumerate}
        \item
            \( f(xy)=f(x)f(y)\)
        \item
            \( f(x+\alpha y)=f(x)+\alpha f(y)\)
    \end{enumerate}
    où nous avons noté \( xy\) pour \( x\times y\).
\end{definition}

\begin{lemma}[\cite{MonCerveau}]   \label{LEMooVKLKooSAHmpZ}
    Soient une algèbre \( A\) et une famille \( (X_i)_{i\in I}\) de sous-algèbres de \( A\) (ici \( I\) est un ensemble quelconque). Alors la partie \( X=\bigcap_{i\in I}X_i\) est une sous-algèbre de \( A\).
\end{lemma}

\begin{proof}
    Nous devons prouver que si \( x\) et \( y\) sont dans \( X\) et \( \lambda\in \eK\), alors \( xy\), \( x+y\) et \( \lambda x\) sont dans \( X\). Pour tout \( i\in I\) nous avons \( x,y\in X_i\) et donc \( xy\in X_i\), \( x+y\in X_i\) et \( \lambda x\in X_i\) (parce que \( X_i\) est une algèbre). Donc \( xy\),\( x+y\) et \( \lambda x\) sont dans \( X_i\) pour tout \( I\), et donc dans \( X\).
\end{proof}

\begin{definition}\label{DefkAXaWY}
    L'\defe{algèbre engendrée}{algèbre!engendrée} par \( X\) est l'intersection de toutes les sous-algèbres de \( A\) contenant \( X\) (qui est une algèbre par le lemme \ref{LEMooVKLKooSAHmpZ}).
\end{definition}

%+++++++++++++++++++++++++++++++++++++++++++++++++++++++++++++++++++++++++++++++++++++++++++++++++++++++++++++++++++++++++++
\section{Anneau intègre}
%+++++++++++++++++++++++++++++++++++++++++++++++++++++++++++++++++++++++++++++++++++++++++++++++++++++++++++++++++++++++++++
\label{SECAnneauxIntegres}

Rappelons la notion de diviseurs que nous avons fixée par la définition \ref{DiviseursAnneau}.
Un élément \( a\neq 0\) est un \defe{diviseur de zéro à gauche}{diviseur!de zéro} s'il existe \( x\neq 0\) tel que $xa=0$. L'élément \( a\) est un diviseur de zéro \defe{à droite}{diviseur!de zéro à droite} s'il existe \( b\) tel que \( ab=0\).

\begin{definition}      \label{DEFooTAOPooWDPYmd}
    Un anneau est \defe{intègre}{intègre!anneau}\index{anneau!intègre} s'il est non nul et ne possède pas de diviseurs de zéro.
\end{definition}

\begin{proposition}     \label{PROPooZWUEooUHTtUI}
    Soit $A$ un anneau. Les assertions suivantes sont équivalentes:
    \begin{enumerate}
        \item
            $A$ est un anneau intègre.
        \item
            La règle du produit nul s'applique dans $A$: pour tous \( a, b \in A \), si \( ab=0\), alors \( a = 0\) ou \( b = 0\).
            \index{règle!du produit nul}
        \item       \label{ITEMooQNTFooSRrVPK}
            On peut simplifier par un même élément non-nul, deux expressions produit dans $A$ qui sont égales: pour tous \( a, b, c \in A \) avec \( a \neq 0 \), si \( ab = ac \), alors \( b = c \).
    \end{enumerate}
\end{proposition}

Un corps dans lequel l'élément nul est inversible est appelé un mensonge, comme le prouve le lemme suivant.
\begin{lemma}       \label{LemAnnCorpsnonInterdivzer}
    En tant que anneau, un corps est un anneau intègre : il n'a pas de diviseurs de zéro. 
\end{lemma}

\begin{proof}
    En effet si \( a\) est un diviseur de zéro, alors \( ax=0\) pour un certain \( x\neq 0\). Si \( a\) était inversible, nous aurions \( x=a^{-1}ax=0\), ce qui est impossible.
\end{proof}
Conséquence : dans un corps nous avons toujours la règle du produit nul.

\begin{example}     \label{EXooLDXRooSxUAXs}
    L'ensemble \( \eZ\) avec les opérations usuelles est un anneau intègre.
\end{example}


\begin{lemma}
    Si l'anneau \( \eZ/n\eZ\) est intègre, alors \( n\) est premier.
\end{lemma}

\begin{proof}
    Si \( n\) n'est pas premier, alors \( n=pq\) avec \( 1<p\leq q<n\). Alors
    \begin{equation}
        [p]_n[q]_n=[pq]_n=[0]_n.
    \end{equation}
    Donc lorsque \( n\) n'est pas premier,  l'anneau \( \eZ/n\eZ\) possède des diviseurs de zéro et n'est alors pas intègre.
\end{proof}
<++>

\begin{example}
    L'anneau \( \eZ/6\eZ\) n'est pas intègre parce que \( 3\cdot 2=0\) alors que ni \( 3\) ni \( 2\) ne sont nuls.
\end{example}

Nous verrons au théorème \ref{ThoBUEDrJ} que l'anneau \( A\) est intègre si et seulement si \( A[X]\) est intègre.

\begin{corollary}   \label{CorZnInternprem}
    L'anneau \( \eZ/n\eZ\) est intègre si et seulement si \( n\) est premier.
\end{corollary}

\begin{proof}
    Si \( n\) est premier, tous les éléments de \( \eZ/n\eZ\) sont inversibles parce que tous les éléments rentrent dans \( \phi(P_n)\). Donc \( \eZ/n\eZ\) est intègre.

    Si \( n\) n'est pas premier, il existe \( p,q\in\eN^*\) tels que \( pq=n\). Dans ce cas au niveau des classes nous avons \( \tilde p\tilde q=0\) avec \( \tilde p\neq 0\neq\tilde q\), ce qui montre que \( \eZ/n\eZ\) a des diviseurs de zéro et n'est pas intègre.
\end{proof}

%---------------------------------------------------------------------------------------------------------------------------
\subsection{Caractéristique d'un anneau intègre}
%---------------------------------------------------------------------------------------------------------------------------

\begin{lemma}       \label{LemCaractIntergernbrcartpre}
    La caractéristique\footnote{Définition \ref{LEMDEFooVEWZooUrPaDw}.} d'un anneau intègre est zéro ou un nombre premier.
\end{lemma}

\begin{proof}
    Si \( A\) est intègre, alors \( \eZ 1_A\) est intègre (a fortiori), et soit $A$ est de caractéristique nulle (auquel cas c'est fini), soit $A$ est de caractéristique $p$, et \( \eZ/p\eZ\) est intègre parce qu'il est isomorphe à \( \eZ 1_A\). Mais nous savons que \( \eZ/p\eZ\) est intègre si et seulement si \( p\) est premier (proposition \ref{CorZnInternprem}).
\end{proof}

\begin{example}
    Il existe des corps dont la caractéristique n'est pas égale au cardinal (contrairement à ce que laisserait penser l'exemple des \( \eZ/p\eZ\)). En effet les matrices \( n\times n\) inversibles sur \( \eF_{3}\) forment un corps qui n'est pas de cardinal trois alors que la caractéristique est \( 3\) :
    \begin{equation}
        \begin{pmatrix}
            1    &       \\ 
                &   1    
            \end{pmatrix}+\begin{pmatrix}
                1    &       \\ 
                    &   1    
                \end{pmatrix}+\begin{pmatrix}
                    1    &       \\ 
                        &   1    
                \end{pmatrix}=0.
    \end{equation}
\end{example}

\begin{example}
    Si \( \eK\) est un corps de caractéristique \( 2\), alors l'égalité \( x=-x\) n'implique pas \( x=0\), vu que \( 2x=0\) est vérifiée pour tout \( x\). Cela se répercute sur un certain nombre de résultats. Par exemple, en caractéristique deux, une forme antisymétrique n'est pas toujours alternée: voir le lemme \ref{LemHiHNey}.
\end{example}

%---------------------------------------------------------------------------------------------------------------------------
\subsection{Divisibilité et classes d'association}  
%---------------------------------------------------------------------------------------------------------------------------
\label{DivisibiliteAnneauxIntegres}

\begin{lemma}\label{LemRmVTRq}
    Si \( A\) est un anneau intègre et si \( a,b\in A\) sont tels que \( a\divides b\) et \( b\divides a\), alors il existe un inversible \( u\in A\) tel que \( a=ub\).
\end{lemma}

\begin{proof}
    Les hypothèses à propos de la divisibilité nous indiquent que \( a=xb\) et \( b=ya\) pour certains \( x,y\in A\). Du coup,
    \begin{equation}
        b(1-yx)=0.
    \end{equation}
    Étant donné que \( \eA\) est intègre, cela montre que \( b=0\) ou \( 1-yx=0\). Si \( b=0\) nous avons immédiatement \( a=0\) et le lemme est prouvé. Si au contraire \( yx=1\), c'est que \( y\) et \( x\) sont inversibles et inverses l'un de l'autre.
\end{proof}

\begin{definition}\label{DefrXUixs}
    On dit de deux éléments \( a,b\in A\) qu'ils sont \defe{associés}{associés!éléments d'un anneau} si ils vérifient les hypothèses du lemme \ref{LemRmVTRq}; en d'autres termes, $a$ et $b$ sont associés s'il existe un inversible \( u\in A\) tel que \( a=ub\).

    La \defe{classe d'association}{classe d'association}\index{classe!d'association} d'un élément \( a \in A \) est l'ensemble des éléments qui lui sont associés; en d'autres termes, c'est \( a  U(A) \).
\end{definition}

\begin{example}
    Dans \( \eZ[i]\), les inversibles sont \( \pm 1\) et \( \pm i\). Donc les éléments associés à \( z\) sont \( z\), \( -z\), \( iz\) et \( -iz\).

    Notons au passage que la notion de divisibilité dans \( \eZ[i]\) n'est pas immédiatement intuitive. En effet bien que \( 7\) ne soit pas divisible par \( 2\) (ni dans \( \eZ\) ni dans \( \eZ[i]\)), le nombre \( 7+6i\) est divisible par \( 2+i\) dans \( \eZ[i]\). En effet :
    \begin{equation}
        (2+i)(4+i)=7+6i.
    \end{equation}
\end{example}

\begin{probleme}
    Est-ce que quelqu'un connais un anneau contenant \( \eZ\) dans lequel \( 7\) est divisible en \( 2\) ?

    Peut-être \( \eZ\) étendu par tous les \( 1/2^n\) ?
\end{probleme}

%---------------------------------------------------------------------------------------------------------------------------
\subsection{PGCD et PPCM}
%---------------------------------------------------------------------------------------------------------------------------

Pour le théorème de Bézout et autres opérations avec des molulo, voir le thème \ref{THEMEooNRZHooYuuHyt}.

Commençons par donner une autre vision de la divisibilité dans les anneaux intègres.
\begin{proposition}\label{PropDiviseurIdeaux}
    Dans un anneau intègre\footnote{Définition \ref{DEFooTAOPooWDPYmd} et proposition \ref{PROPooZWUEooUHTtUI}.} $A$, on a l'équivalence suivante concernant deux éléments \( a, b \in A \):
\begin{equation}
    a\divides b\Leftrightarrow (b)\subset (a).
\end{equation}
\end{proposition}

Donc la divisibilité devient en réalité une relation d'ordre dont nous pouvons chercher un maximum et un minimum. Si \( S\) est une partie de \( A\), nous notons \( a\divides S\) pour exprimer que \( a\divides x\) pour tout \( x\in S\); de la même façon, \( S\divides b\) signifie que \( x\divides b\) pour tout \( x\in S\)

\begin{definition}\label{DefrYwbct}
    Soient \( A\) un anneau intègre et \( S\subset A\). Nous disons que \( \delta\in A\) est un \defe{PGCD}{PGCD!dans un anneau intègre} de \( S\) si
    \begin{enumerate}
        \item
            \( \delta\divides S\)
        \item
            si \( d\divides S\) alors\footnote{Il me semble qu'à ce niveau il y a une faute de frappe dans \cite{XPXxPl}.} \( d\divides \delta\).
    \end{enumerate}
    Nous disons que \( \mu\in A\) est un \defe{PPCM}{PPCM!dans un anneau intègre} de \( S\) si
    \begin{enumerate}
        \item
            \( S\divides \mu\),
        \item
            si \( S\divides m\), alors \( \mu\divides m\).
    \end{enumerate}
\end{definition}
Notons qu'il n'y a en général pas unicité du PGCD ou du PPCM d'un ensemble.

\begin{lemma}
    Soent \( A\) un anneau intègre et \( S\subset A\). Si \( \delta\) est un PGCD de \( S\), alors l'ensemble des PGCD de \( S\) est la classe d'association de \( \delta\).

    De la même façon si \( \mu\) est un PPCM de \( S\), alors l'ensemble des PPCM de \( S\) est la classe d'association de \( \mu\).
\end{lemma}

\begin{proof}
    Soient \( \delta\) un PGCD de \( S\) et \( u\) un inversible dans \( A\). Si \( x\in S\) nous avons \( \delta\divides x\) et donc \( x=a\delta\). Par conséquent \( x=au^{-1}u\delta\) et donc \( u\delta\) divise \( x\). De la même manière, si \( d\) divise \( x\) pour tout \( x\in S\), alors \( d\) divise \( \delta\) et donc \( \delta=ad\) et \( u\delta=aud\), ce qui signifie que \( d\) divise \( u\delta\).

    Dans l'autre sens nous devons prouver que si \( \delta'\) est un autre PGCD de \( S\), alors il existe un inversible \( u\in \eA\) tel que \( \delta'=u\delta\). Vu que \( \delta'\) divise \( x\) pour tout \( x\in S\), nous avons \( \delta'\divides \delta\), et symétriquement nous trouvons \( \delta\divides\delta'\). Par conséquent (lemme \ref{LemRmVTRq}), il existe un inversible \( u\) tel que \( \delta=u\delta'\).

    Le même type de raisonnement tient pour le PPCM.
\end{proof}

Si \( \delta\) est un PGCD de \( S\), nous dirons \emph{par abus de langage} que \( \delta\) est \emph{le} PGCD de \( S\), gardant en tête qu'en réalité toute sa classe d'association est PGCD. Nous noterons aussi, toujours par abus que \( \delta=\pgcd(S)\).

\begin{remark}
    La classe d'association d'un élément n'est pas toujours très grande. Les inversibles dans \( \eZ\) étant seulement \( \pm 1\), nous pouvons obtenir l'unicité du PGCD et du PPCM en imposant qu'ils soient positifs.

    Pour les polynômes, nous obtenons l'unicité en demandant que le PGCD soit unitaire.

    Dans les cas pratiques, il y a donc en réalité peu d'ambiguïté à parler du PGCD ou du PPCM d'un ensemble.
\end{remark}


%---------------------------------------------------------------------------------------------------------------------------
\subsection{Anneaux intègres et corps}
%---------------------------------------------------------------------------------------------------------------------------


Le fait d'être intègre pour un anneau n'assure pas le fait d'être un corps. Nous avons cependant ce résultat pour les anneaux finis.

\begin{proposition}     \label{PropanfinintimpCorp}
    Un anneau fini intègre est un corps.
\end{proposition}

\begin{proof}
    Soit \( A\) un tel anneau. Soit \( a\neq 0\). Les applications 
    \begin{subequations}
        \begin{align}
            l_a\colon x\to ax\\
            r_a\colon x\to xa
        \end{align}
    \end{subequations}
    sont injectives. En tant que applications injectives entre ensembles finis, elles sont surjectives. Il existe donc \( b\) et \( c\) tels que \( 1=ba=ac\). Il se fait que \( b\) et \( c\) sont égaux parce que
    \begin{equation}
        b=b(ac)=(ba)c=c.
    \end{equation}
    Par conséquent \( b\) est un inverse de \( a\).
\end{proof}

Il faut être un peu souple avec les notations : ici \( b(ac)\) est le produit de \( b\) par l'élément \( bc\), et non quelque chose comme le produit de \( b\) avec l'idéal \( (bc)\).

\begin{proposition}     \label{PropzhFgNJ}
    Soit \( n\in\eN^*\). Les conditions suivantes sont équivalentes :
    \begin{enumerate}
        \item
            \( n\) est premier.
        \item
            \( \eZ/n\eZ\) est un anneau intègre.
        \item
            \( \eZ/n\eZ\) est un corps.
    \end{enumerate}
\end{proposition}

\begin{proof}
    L'équivalence entre les deux premiers points est le contenu du corollaire \ref{CorZnInternprem}. Le fait que \( \eZ/n\eZ\) soit un corps lorsque \( \eZ/n\eZ\) est intègre est la proposition \ref{PropanfinintimpCorp}. Le fait que \( \eZ/n\eZ\) soit intègre lorsque \( \eZ/n\eZ\) est un corps est une propriété générale des corps : ce sont en particulier des anneaux intègres (lemme \ref{LemAnnCorpsnonInterdivzer}).
\end{proof}

\begin{proposition}     \label{AnnCorpsIdeal}
    Si \( A\) est un anneau, nous avons les équivalences
    \begin{enumerate}
        \item
            \( A\) est un corps.
        \item
            \( A\) est non nul et ses seuls idéaux à gauche sont \( \{ 0 \}\) et \( A\).
        \item
            \( A\) est non nul et ses seuls idéaux à droite sont \( \{ 0 \}\) et \( A\).
    \end{enumerate}
\end{proposition}

\begin{proposition}
    Soient \( A\) un anneau commutatif non nul et \( I\) un idéal dans \( A\). L'ensemble \( I\) est un idéal maximal de \( A\) si et seulement si \( A/I\) est un corps.
\end{proposition}

\begin{proof}
    Par la proposition \ref{AnnCorpsIdeal}, le fait pour \( A/I\) d'être un corps signifie qu'il n'a pas d'idéaux non triviaux. Si \( \phi\) est la projection canonique \( A\mapsto A/I\), nous savons que les idéaux de \( A/I\) sont les \( \phi(J)\) où \( J\) est un idéal de \( A\) contenant \( I\) (proposition \ref{PropIJJIdsousphi}). Si \( I\) est un idéal maximal de \( A\), un tel \( J\) n'existe pas. Inversement si \( A/I\) n'a pas d'autres idéaux que \( A/I\) et \( \phi(0)\), c'est que \( I\) est un idéal maximal.
\end{proof}

%---------------------------------------------------------------------------------------------------------------------------
\subsection{Corps des fractions}
%---------------------------------------------------------------------------------------------------------------------------

\begin{theoremDef}     \label{ThogbhWgo}
    Soit \( \eA\) un anneau commutatif intègre. 
    
    \begin{enumerate}
        \item
            
    Il existe un corps commutatif \( \eK\) et un morphisme d'anneaux injectif \( \epsilon\colon \eA\to \eK\) tels que pour tout \( \lambda\in\eK\), il existe \( (a,b)\in \eA\times \eA^*\) tels que
    \begin{equation}
        \lambda=\epsilon(a)\big( \epsilon(b) \big)^{-1}
    \end{equation}

\item
    Si \( (\eK',\epsilon')\) est un autre couple qui vérifie la propriété, les corps \( \eK\) et \( \eK'\) sont isomorphes.

    \end{enumerate}

    Le corps \( \eK\) associé à l'anneau \( \eA\) est le \defe{corps des fractions}{corps!des fractions}\index{fractions (corps)} de \( \eA\), et sera noté \( \Frac(\eA)\).\nomenclature[A]{\( \Frac(\eA)\)}{Le corps des fractions de l'anneau \( \eA\)}
\end{theoremDef}

\begin{lemma}[Simplification de fraction]
    L'application \( \eA\times \eA^*\to \eK\) donnée par \( (a,b)\mapsto \epsilon(a)\big( \epsilon(b) \big)^{-1}\) envoie \( (xa,xb)\) sur le même que \( (a,b)\).
\end{lemma}

\begin{proposition} \label{PROPooXHMPooEavWLt}
    À propos des rationnels.
    \begin{enumerate}
        \item
            L'ensemble \( \eQ\)\nomenclature[A]{\( \eQ\)}{corps des fractions sur \( \eZ\)} est le corps des fractions de \( \eZ\). 
        \item
            Cet ensemble n'est pas l'ensemble de tous les réels.
    \end{enumerate}
\end{proposition}

%+++++++++++++++++++++++++++++++++++++++++++++++++++++++++++++++++++++++++++++++++++++++++++++++++++++++++++++++++++++++++++
\section{Anneau factoriel}
%+++++++++++++++++++++++++++++++++++++++++++++++++++++++++++++++++++++++++++++++++++++++++++++++++++++++++++++++++++++++++++

\begin{definition}[Élément irréductible\cite{ooWUNIooXKxRya}]  \label{DeirredBDhQfA}
    Un élément d'un anneau commutatif intègre est \defe{irréductible}{irréductible!dans un anneau} si il n'est ni inversible, ni le produit de deux éléments non inversibles.
\end{definition}

Pour rappel, nous notons \( U(A)\) l'ensemble des éléments inversibles de \( A\).

\begin{example}
    Un corps n'a pas d'éléments irréductibles parce qu'à part zéro tous les éléments sont inversibles. Mais \( 0\) n'est pas irréductible parce qu'il peut être écrit comme produit d'éléments non inversibles : \( 0=0\cdot 0\).
\end{example}

\begin{example}
    Les éléments irréductibles de l'anneau \( \eZ\) sont les nombres premiers. En effet les seuls inversibles de \( \eZ\) sont \( \pm 1\). Si \( p\) est premier et \( p=ab\) avec \( a,b\in \eZ\), alors nous avons soit \( a=\pm 1\) soit \( b=\pm 1\).
\end{example}

\begin{definition}[Anneau factoriel]        \label{DEFooVCATooPJGWKq}
    Un anneau commutatif \( A\) est \defe{factoriel}{factoriel!anneau}\index{anneau!factoriel} s'il vérifie les propriétés suivantes.
    \begin{enumerate}
        \item
            L'anneau \( A\) est intègre (pas de diviseurs de zéro).
        \item
            Si \( a\in A\) est non nul et non inversible, alors il admet une décomposition en facteurs irréductibles: \( a=p_1\ldots p_k\) où les \( p_i\) sont irréductibles.
        \item
            Si \( a=q_1\ldots q_m\) est une autre décomposition de \( a\) en irréductibles, alors \( m=k\) et il existe une permutation\footnote{Définition \ref{DEFooJNPIooMuzIXd}.} \( \sigma\in S_k\) telle que \( p_i\) et \( q_{\sigma(i)}\) soient associés\footnote{Définition \ref{DefrXUixs}.}.
    \end{enumerate}
\end{definition}

Un anneau factoriel permet de caractériser le \( \pgcd\) et le \( \ppcm\) de nombres.

\begin{proposition}
Soit une famille \( \{ a_n \}\) d'éléments de \( A\) qui se décomposent en irréductibles comme
\begin{equation}
    a_i=\prod_k p_k^{\alpha_{k,i}}.
\end{equation}
Alors
\begin{equation}
    \pgcd\{ a_n \}=\prod_k p_k^{\min_i\{ \alpha_{k,i} \}}.
\end{equation}

De plus le PGCD est :
\begin{enumerate}
    \item
        Un multiple de tous les diviseurs communs des \( a_i\).
    \item
        Unique pour cette propriété à multiple près par un inversible\quext{Soyez prudent avec cette affirmation : je n'en n'ai pas de démonstrations sous la main et ne suis pas certain que ce soit vrai.}.
\end{enumerate}

\end{proposition}

De la même manière,
\begin{equation}
    \ppcm\{ a_n \}=\prod_kp_k^{\max_i\{ \alpha_{k,i} \}}.
\end{equation}
Un anneau factoriel a une relation de préordre partiel\index{ordre!sur un anneau factoriel} donnée par \( a<b\) si \( a\) divise \( b\). En termes d'idéaux, cela donne l'ordre inverse de celui de l'inclusion\footnote{Voir proposition \ref{PropDiviseurIdeaux}.} : \( a<b\) si et seulement si \( (b)\subset (a)\).

\begin{example} \label{EXooCWJUooCDJqkr}
    L'anneau \( \eZ[i\sqrt{3}]\) n'est pas factoriel parce que
    \begin{equation}
        4=2\cdot 2=(1+i\sqrt{3})(1-i\sqrt{3})
    \end{equation}
    donnent deux décompositions distinctes de \( 4\) en irréductibles.
\end{example}

Nous allons voir dans l'exemple \ref{ExeDufyZI} que \( \eZ[i\sqrt{2}]\) est factoriel parce qu'il sera euclidien.

%+++++++++++++++++++++++++++++++++++++++++++++++++++++++++++++++++++++++++++++++++++++++++++++++++++++++++++++++++++++++++++
\section{Anneau principal}
%+++++++++++++++++++++++++++++++++++++++++++++++++++++++++++++++++++++++++++++++++++++++++++++++++++++++++++++++++++++++++++

\begin{definition} 
    Un idéal \( I\) dans \( A\) est \defe{principal à gauche}{idéal!principal!à gauche} s'il existe \( a\in I\) tel que \( I= A a\). Il est \defe{principal à droite}{idéal!principal!à droite} s'il existe \( a\in I\) tel que \( I=a A\). Nous disons qu'il est \defe{principal}{principal!idéal} s'il est principal à gauche et à droite.
    
    Un idéal \( I\) dans l'anneau \( A\) est \defe{maximal}{maximal!idéal}\index{idéal!maximal} si les seuls idéaux de \( A\) contenant \( I\) sont \( I\) et \( A\).
\end{definition}

\begin{definition}          \label{DEFooGWOZooXzUlhK}
    Un anneau est \defe{principal}{principal!anneau} si 
    \begin{enumerate}
        \item
            il est commutatif et intègre
        \item
            tous ses idéaux sont principaux.
    \end{enumerate}
\end{definition}

Souvent pour prouver qu'un anneau est principal, nous prouvons qu'il est euclidien (définition \ref{DefAXitWRL}) et nous utilisons la proposition \ref{Propkllxnv} qui dit qu'un anneau euclidien est principal.

Une manière de prouver qu'un anneau n'est pas principal est de prouver qu'il n'est pas factoriel, théorème \ref{THOooANCAooBChmwp}.

\begin{definition}
    Nous disons qu'un idéal \( I\) dans \( A\) est \defe{premier}{premier!idéal} si \( I\) est strictement inclus dans \( A\) et si pour tout \( a,b\in A\) tels que \( ab\in I\) nous avons \( a\in I\) ou \( b\in I\).
\end{definition}

\begin{proposition}
    Si \( A\) est un anneau commutatif intègre, alors un idéal \( I\) dans \( A\) est premier si et seulement si \( A/I\) est intègre. 
\end{proposition}

\begin{proposition} \label{PropomqcGe}
    Soit \( A\) un anneau principal qui n'est pas un corps. Pour un idéal \( I\subset A\), les conditions suivantes sont équivalentes :
    \begin{enumerate}
        \item
            \( I\) est un idéal maximal;
        \item
            \( I\) est un idéal premier non nul;
        \item
            il existe \( p\) irréductible dans \( A\) tel que \( I=(p)\).
    \end{enumerate}
\end{proposition}

\begin{proposition}     \label{PropoTMMXCx}
    Si \( A\) est un anneau principal et si \( p\) est irréductible, alors \(  A/ (p) \) est un corps.
\end{proposition}

\begin{example}
    L'anneau \( \eZ\) est principal parce que ses seuls idéaux sont les \( n\eZ\) qui sont principaux : \( n\eZ\) est engendré par \( n\).
\end{example}

\begin{example}[Les idéaux de $\eZ/n\eZ$]       \label{EXooCJRPooYkWdyr}

    Les idéaux de \( \eZ/n\eZ\) sont principaux, mais l'anneau \( \eZ/n\eZ\) n'est pas principal lorsque \( n\) n'est pas premier. Nous allons voir ça.

    \begin{subproof}
        \item[Les idéaux de \( \eZ/n\eZ\) sont principaux]

            Soit un idéal \( S\) dans \( \eZ/n\eZ\). Nous considérons la projection canonique \( \phi\colon \eZ\to \eZ/n\eZ\). La proposition \ref{PropIJJIdsousphi} dit que  \( S=\phi(J)\) où \( J\) est un idéal de \( \eZ\) contenant \( n\eZ\). Mais le corollaire \ref{CORooLINXooBlUKPG} nous dit qu'alors \( J=m\eZ\) pour un certain \( m\). Pour que \( m\eZ\) contienne \( n\eZ\), il faut que \( m\) divise \( n\).

            Bref, \( S=\phi(m\eZ)\) avec \( m\divides n\). Nous montrons maintenant que \( S\) est engendré par \( [m]_n\). D'abord, l'élément \( [m]_n\) est bien dans \( \phi(m\eZ)\). Ensuite un élément de \( \phi(m\eZ)\) est de la forme 
            \begin{equation}
                [km]_n=k[m]_n\in ([m]_n).
            \end{equation}
            Donc \( S\subset ([m]_n)\). Et l'inclusion dans l'autre sens est tout aussi immédiate : un élément de \( ([m]_n)\) est de la forme
            \begin{equation}
                k[m]_n=[km]_n=\phi(km)\in \phi(m\eZ).
            \end{equation}

        \item[Si \( n\) n'est pas premier, \( \eZ/n\eZ\) n'est pas principal]
            
            Le fait est que lorsque \( n\) n'est pas premier, \( \eZ/n\eZ\) n'est pas intègre (corollaire \ref{CorZnInternprem}).

        \item[Moralité]

            Un anneau comme \( \eZ/6\eZ\) est un anneau dont tous les idéaux sont principaux, mais qui n'est pas principal.
    
    \end{subproof}
\end{example}

\begin{example}
    Nous verrons dans la proposition \ref{PROPooVWRPooGQMenV} que l'anneau des fonctions holomorphes sur un compact de \( \eC\) est principal.
\end{example}

\begin{definition}      \label{DEFooXSPFooPumQSy}
Nous disons que deux éléments d'un anneau principal sont \defe{premiers entre eux}{premier!deux éléments d'un anneau principal} si leur PGCD est \( 1\).
\end{definition}

\begin{theorem}\index{théorème!chinois!anneau principal}        \label{ThofPXwiM}
    Si \( A\) est un anneau principal et si \( p\) et \( q\) sont premiers entre eux dans \( A\), alors on a l'isomorphisme d'anneaux
    \begin{equation}
        A/pqA\simeq A/pA\times A/qA.
    \end{equation}
\end{theorem}
% TODO : trouver une preuve. Je parie que recopier la même que celle dans Z fonctionne très bien.

%---------------------------------------------------------------------------------------------------------------------------
\subsection{Bézout}
%---------------------------------------------------------------------------------------------------------------------------

\begin{theorem}[\cite{XPXxPl}]
    Toute partie \( S\) d'un anneau principal admet un PGCD et un PPCM. De plus
    \begin{equation}
        \begin{aligned}[]
            \delta=\pgcd(S)\Leftrightarrow (\delta)=\sum_{s\in S}(s)
            \mu=\ppcm(S)\Leftrightarrow (\mu)=\bigcap_{s\in S}(s)
        \end{aligned}
    \end{equation}
\end{theorem}

\begin{proof}
    Vu que l'anneau \( A\) est principal, tous ses idéaux sont principaux et donc engendrés par un seul élément. En particulier il existe \( \delta,\mu\in A\) tels que
    \begin{subequations}
        \begin{align}
            (\delta)&=\sum_{s\in S}(s)\\
            (\mu)&=\bigcap_{s\in S}(s)
        \end{align}
    \end{subequations}
    \begin{subproof}
    \item[PGCD]
        Montrons ce que \( \delta\) est un PGCD de \( S\). Pour tout \( x\in S\), nous avons \( (x)\subset (\delta)\), et donc \( \delta\divides x\). Par ailleurs si \( d\divides x\) pour tout \( x\in S\), nous avons \( (x)\subset (d)\) et donc 
        \begin{equation}
            \sum_{x\in S}(x)\subset (d),
        \end{equation}
        puis \( (\delta)\subset (d)\) et finalement \( d\divides \delta\).
        \item[PPCM]
            Si \( x\in S\) nous avons \( (\mu)\subset (x)\) et donc \( x\divides \mu\). D'autre part si \( x\divides m\) pour tout \( x\in S\), alors \( (m)\subset (x)\) et donc \( (m)\subset(\mu)\), finalement \( \mu\divides m\).
    \end{subproof}
\end{proof}

\begin{corollary}[Théorème de Bézout\cite{XPXxPl}]\index{Bézout!anneau principal}\label{CorimHyXy}
    Soit un anneau principal \( A\). Deux éléments \( a,b\in A\) sont premiers entre eux si et seulement s'il existe un couple \( (u, v)\in A^2 \) tel que
    \begin{equation}
        ua+vb=1.
    \end{equation}
    À la place de \( 1\) on aurait pu écrire n'importe quel inversible.
\end{corollary}
\index{anneau!principal}

\begin{proof}
    Pour cette preuve, nous allons écrire \( \pgcd(a,b)\) l'ensemble de PGCD de \( a\) et \( b\), c'est à dire la classe d'association d'un PGCD.

    Si \( a\) et \( b\) sont premiers entre eux, alors
    \begin{equation}
        1\in\pgcd(a,b)=\sum_{x=a,b}(x)=(a)+(b).
    \end{equation}
    
    À l'inverse, si nous avons \( ua+vb=1\), alors \( 1\in (a)+(b)\), mais vu que \( (a)+(b)\) est un idéal principal, \( (1)=(a)+(b)\) et donc \( 1\in \pgcd(a,b)\).
\end{proof}

Le lemme de Gauss est une application immédiate de Bézout. Il y aura aussi un lemme de Gauss à propos de polynômes (lemme \ref{LemEfdkZw}), et une généralisation directe au théorème de Gauss, théorème \ref{ThoLLgIsig}.
\begin{lemma}[\href{http://ljk.imag.fr/membres/Bernard.Ycart/mel/ar/node6.html}{lemme de Gauss}]    \label{LemSdnZNX}
    Soit \( A\) un anneau principal et \( a,b,c\in A\) tels que \( a\) divise \( bc\). Si \( a\) est premier avec \( c\), alors \( a\) divise \( b\).
\end{lemma}
\index{lemme!Gauss!dans un anneau principal}

\begin{proof}
    Vu que \( a\) est premier avec \( c\), nous avons \( \pgcd(a,c)=1\) et Bézout (\ref{ThoBuNjam}) nous donne donc \( s,t\in \eA\) tels que \( sa+tc=1\). En multipliant par \( b\),
    \begin{equation}
        sab+tbc=b.
    \end{equation}
    Mais les deux termes du membre de gauche sont multiples de \( a\) parce que \( a\) divise \( bc\). Par conséquent \( b\) est somme de deux multiples de \( a\) et donc est multiple de \( a\).
\end{proof}
Un cas usuel d'utilisation est le cas de \( A=\eN^*\).

%---------------------------------------------------------------------------------------------------------------------------
\subsection{Anneau noethérien}
%---------------------------------------------------------------------------------------------------------------------------

\begin{definition}
    Un anneau est dit \defe{noethérien}{anneau!noethérien} si toute suite croissante d'idéaux est stationnaire (à partir d'un certain rang). 
\end{definition}

Montrer que tout anneau principal est noethérien est le premier pas pour montrer que tout anneau principal est factoriel.

\begin{lemma}
    Tout anneau principal est noethérien.
\end{lemma}

\begin{proof}
    Soit \( (J_n)\) une suite croissante d'idéaux et \( J\) la réunion. L'ensemble \( J\) est encore un idéal parce que les \( J_i\) sont emboités. Étant donné que l'idéal est principal nous pouvons prendre \( a\in J\) tel que \( J=(a)\). Il existe \( N\) tel que \( a\in J_N\). Alors pour tout \( n\geq N\) nous avons
    \begin{equation}
        J\subset J_N\subset J_n\subset J.
    \end{equation}
    La première inclusion est le fait que \( J=(a)\) et \( a\in J_N\). La seconde est la croissance des idéaux et la troisième est le fait que \( J\) est une union. Par conséquent pour tout \( n\geq N\) nous avons \( J_N=J_n=J\). La suite est par conséquent stationnaire.
\end{proof}

\begin{theorem}[\cite{FSwlnf}]      \label{THOooANCAooBChmwp}
    Tout anneau principal est factoriel.
\end{theorem}

\begin{example}[\( \eZ\lbrack i\sqrt{ 5 }\rbrack\) n'est ni factoriel ni principal]     \label{EXooYCTDooGXAjGC}
    Vu que \( (i\sqrt{ 5 })^2=-5\), les éléments de \( \eZ[i\sqrt{ 5 }]\) sont les éléments de \( \eC\) de la forme \( a+bi\sqrt{ 5 }\) avec \( a,b\in \eZ\). Nous définissons la \defe{norme}{norme!sur \( \eZ[i\sqrt{ 5 }]\)} sur \( \eZ[\i\sqrt{ 5 }]\) par\footnote{C'est le carré de la norme usuelle, mais c'est l'usage dans le milieu.}
    \begin{equation}
        \begin{aligned}
            N\colon \eZ[i\sqrt{ 5 }]&\to \eN \\
            z&\mapsto z\bar z. 
        \end{aligned}
    \end{equation}
    Le fait que ce soit à valeurs dans \( \eN\) est un simple calcul :
    \begin{equation}
        N(x+iy\sqrt{ 5 })=(x+iy\sqrt{ 5 })(x-iy\sqrt{ 5 })=x^2+5y^2.
    \end{equation}
    De plus \( N\) est multiplicative : \( N(z_1z_2)=N(z_1)N(z_2)\).

    Nous pouvons maintenant déterminer les inversibles de \( \eZ[i\sqrt{ 5 }]\). Si \( \alpha\) est inversible, alors il existe \( \beta\) tel que \( \alpha\beta=1\). Au niveau de la norme, 
    \begin{equation}
        N(\alpha)N(\beta)=1,
    \end{equation}
    ce qui implique que \( N(\alpha)=1\). Or l'équation \( x^2+5y^2=1\) dans \( \eN\) donne \( y=0\), \( x=\pm 1\).

    Au final, les inversibles de \( \eZ[i\sqrt{ 5 }]\) sont \( \pm 1\).

    L'anneau \( \eZ[i\sqrt{ 5 }]\) n'est alors pas factoriel (définition \ref{DEFooVCATooPJGWKq}) parce que 
    \begin{equation}
        2\times 3=(1+i\sqrt{ 5 })(1-i\sqrt{ 5 }).
    \end{equation}
    Cela donne deux décompositions du nombre \( 6\) en produit d'éléments non associés\footnote{Définition \ref{DefrXUixs}.} (\( 2\) n'est associé qu'à \( 2\) et \( -2\)) parce que les inversibles sont \( 1\) et \( -1\).

    Le fait que \( \eZ[i\sqrt{ 5 }]\) ne soit pas factoriel implique qu'il ne soit pas principal, théorème \ref{THOooANCAooBChmwp}.
\end{example}

%+++++++++++++++++++++++++++++++++++++++++++++++++++++++++++++++++++++++++++++++++++++++++++++++++++++++++++++++++++++++++++
\section{Anneau euclidien}
%+++++++++++++++++++++++++++++++++++++++++++++++++++++++++++++++++++++++++++++++++++++++++++++++++++++++++++++++++++++++++++

\begin{definition}[\wikipedia{fr}{Anneau_euclidien}{Wikipédia}] \label{DefAXitWRL}
    Soit \( A\) un anneau intègre. Un \defe{stathme euclidien}{stathme euclidien} sur \( A\) est une application \( \alpha\colon A\setminus\{ 0 \}\to \eN\) tel que
    \begin{enumerate}
        \item       \label{ITEMooLVJAooLpjgEz}
            \( \forall a,b\in A\setminus\{ 0 \}\), il existe \( q,r\in A\) tel que
            \begin{equation}
                a=qb+r
            \end{equation}
            et \( \alpha(r)<\alpha(b)\).
        \item
            Pour tout \( a,b\in A\setminus\{ 0 \}\), \( \alpha(b)\leq \alpha(ab)\).
    \end{enumerate}
    Un anneau est \defe{euclidien}{euclidien!anneau} s'il accepte un stathme euclidien.
\end{definition}
Le stathme est la fonction qui donne le «degré» à utiliser dans la division euclidienne. La contrainte est que le degré du reste soit plus petit que le degré du dividende.

\begin{example} \label{ExwqlCwvV}
    Le stathme de \( \eN\) pour la division euclidienne usuelle est \( \alpha(n)=n\). Si \( a,b\in \eN\) nous écrivons
    \begin{equation}
        a=qb+r
    \end{equation}
    où \( q\) est l'entier le plus proche \emph{inférieur} à \( a/b\) (on veut que le reste soit positif) et \( r=a-qb\). Nous avons donc
    \begin{equation}
        r-b=a-b(q+1)<a-b\frac{ a }{ b }=0,
    \end{equation}
    ce qui montre que \( r<b\).
\end{example}

Cet exemple ne fonctionne pas avec \( \eZ\) au lieu de \( \eN\) parce que le stathme doit avoir des valeurs dans \( \eN\). Cela ne veut cependant pas dire qu'il n'existe pas de stathme sur \( \eZ\); cela veut seulement dire que \( \alpha(x)=x\) ne fonctionne pas. 

\begin{proposition}[\cite{ooELVSooZIZCRn}]\label{Propkllxnv}
    Un anneau euclidien est principal.
\end{proposition}

\begin{proof}
    Soit \( A\) un anneau principal et \( \alpha\) un stathme sur \( A\). Nous considérons un idéal \( I\) non nul de \( A\). Nous devons montrer que \( I\) est généré par un élément. En l'occurrence nous allons montrer qu'un élément \( a\in I\setminus\{ 0 \}\) qui minimise \( \alpha(a)\) va générer\footnote{Un tel élément existe\dots}. Soit \( x\in I\). Par construction, il existe \( q,r\in A\) tels que \( x=aq+r\) avec \( r=0\) ou \( \alpha(r)<\alpha(a)\). Étant donné que \( x,a\in I\), \( r\in I\). Si \( r\neq 0\), alors \( r\) contredirait la minimalité de \( \alpha(a)\). Donc \( r=0\) et \( x=aq\), ce qui signifie que \( I\) est principal.
\end{proof}

\begin{proposition}     \label{PROPooPJGLooQSrJTU}
    L'anneau \( \eZ\) est principal et euclidien.
\end{proposition}

\begin{proof}
    Nous allons seulement montrer que \( \alpha(x)=| x |\) est un stathme euclidien. Ainsi \( \eZ\) sera euclidien et donc principal par la proposition \ref{Propkllxnv}.

    D'abord \( \eZ\) est intègre, c'est l'exemple \ref{EXooLDXRooSxUAXs}.

    La condition \( \alpha(b)\leq \alpha(ab)\) est immédiate : \( | a |\leq | ab |\) pour tout \( a,b\in \eZ\).

    Soient maintenant \( a,b\in \eZ\). Nous définissons \( q_0,r_0\in \eN\) tels que
    \begin{equation}
        | a |=q_0| b |+r_0
    \end{equation}
    avec \( r_0<| b |\). Cela existe parce que \( \alpha(x)=x\) est un stathme sur \( \eN\) par l'exemple \ref{ExwqlCwvV}.

    \begin{subproof}
        \item[Si \( a>0\) et \( b>0\)]

            Alors \( a=q_0b+r_0\) et le couple \( (q_0,r_0)\) vérifie les conditions de la définition \ref{DefAXitWRL}\ref{ITEMooLVJAooLpjgEz}.

        \item[Si \( a>0\) et \( b<0\)]

            Alors \( a=-q_0b+r_0\), et le couple \( (-q_0,r_0)\) vérifie les conditions de la définition \ref{DefAXitWRL}\ref{ITEMooLVJAooLpjgEz}.


        \item[Si \( a<0\) et \( b>0\)]
            Alors \( a=-q_0b-r_0\), et le couple \( (-q_0,-r_0)\) vérifie les conditions de la définition \ref{DefAXitWRL}\ref{ITEMooLVJAooLpjgEz} parce que
            \begin{equation}
                \alpha(-r_0)=r_0<| b |=\alpha(b).
            \end{equation}
        \item[Si \( a<0\) et \( b<0\)]
            Alors \( a=q_0b-r_0\), et le couple \( (q_0,-r_0)\) vérifie les conditions de la définition \ref{DefAXitWRL}\ref{ITEMooLVJAooLpjgEz}.

    \end{subproof}
\end{proof}

Nous venons de voir que \( \eZ\) est principal; le lemme suivant nous dit que $\eZ[X]$ n'est pas principal, lui.
\begin{lemma}[\cite{ooRQHSooEBZpKe}]        \label{LEMooDJSUooJWyxCL}
    Si $A$ est un anneau intègre qui n'est pas un corps, alors \( A[X]\) n'est pas principal.
\end{lemma}

\begin{proof}
    Soit un élément non nul \( a\in A\).
    \begin{subproof}
        \newcommand{\foo}{A[X]}
    \item[Un idéal principal contenant \( a\) et \( X\) est $\foo$]
            
            Soit \( (P)\) un idéal principal contenant \( a\) et \( X\). Vu que \( a\in(P)\), il existe \( Q\) tel que \( a=QP\). Donc \( P\) divise \( a\) dans \( \eZ[X]\). Les degrés font que \( P\) est un polynôme constant, c'est à dire en réalité un élément de \( A\). Soit \( P=k\in A\).

            Vu que \( P\) divise \( X\), nous avons aussi \( X=kQ\) pour un certain \( Q\in \eZ[X]\). Les degrés disent qu'il existe \( k'\in A\) tel que \( Q=k'X\) et donc \( Q=k'X=k'kQ\), ce qui implique que \( kk'=1\). L'idéal engendré par \( k\) contient donc en particulier \( kk'=1\) et donc contient \( A[X]\) en entier :
            \begin{equation}
                1=k'k\in k'(P)=(P).
            \end{equation}

        \item[Si \( (a,X)=\foo\) alors \( a\) est inversible]

            Si \( (a,X)=A[X]\), en particulier, \( 1\in (a,X)\), ce qui signifie qu'il existe des polynômes \( U,V\in A[X]\) tels que
            \begin{equation}
                1=UX+Va.
            \end{equation}
            Nous évaluons cette égalité en \( 0\) : vu que \( (UX)(0)=0\) nous avons \( 1=V(0)a\), ce qui signifie que \( V(0)\) est un inverse de \( A\). Donc \( a\) est inversible.


        \item[Si \( a\) n'est pas inversible alors \( (a,X)\) n'est pas principal]

            Si \( (a,X)\) était principal, alors nous aurions, par ce qui est dit plus haut, \( (a,X)=A[X]\). Mais cette dernière égalité impliquerait que \( a\) est inversible.
    \end{subproof}
    En conclusion, si \( A\) n'est pas un corps, il possède un élément ni nul ni inversible. Dans ce cas, l'idéal \( (a,X)\) n'est pas principal dans \( A[X]\) et nous en déduisons que \( A[X]\) n'est pas un anneau principal.
\end{proof}

Nous verrons dans le lemme \ref{LEMooIDSKooQfkeKp} que si $\eK$ est un corps, alors \( \eK[X]\) est principal.

\begin{example} \label{ExeDufyZI}
    Prouvons que \( \eZ[i\sqrt{2}]\) est une anneau euclidien. Pour cela nous démontrons que
    \begin{equation}    \label{EqOZUIooZGmHWl}
        \begin{aligned}
            N\colon \eZ[i\sqrt{2}]&\to \eN \\
            a+bi\sqrt{2}&\mapsto a^2+2b^2 
        \end{aligned}
    \end{equation}
    est un stathme euclidien.    

    Soient \( z=a+bi\sqrt{2}\), \( t=a'+b'i\sqrt{2}\). Nous cherchons \( q\) et \( r\) tels que la division euclidienne s'écrive \( z=qt+r\). Soient \( \alpha,\beta\in \eQ\) tels que 
    \begin{equation}
        \frac{ z }{ t }=\alpha+\beta i\sqrt{2}.
    \end{equation}
    Nous désignons par \( \alpha+\epsilon_1\) et \( \beta+\epsilon_2\) les entiers les plus proches de \( \alpha\) et \( b\). Nous avons \( | \epsilon_1 |,| \epsilon_2 |\leq \frac{ 1 }{2}\). Nous posons alors naturellement 
    \begin{equation}
        q=(\alpha+\epsilon_1)+(\beta+\epsilon_2)i\sqrt{2}
    \end{equation}
    et nous calculons \( r=z-qt\) :
    \begin{equation}
        2b'\epsilon_2-a'\epsilon_1+i\sqrt{2}\big( \epsilon_1b'-a'\epsilon_2 \big).
    \end{equation}
    Nous trouvons 
    \begin{equation}
        N(r)=a'^2\epsilon_1^2+4b'^2\epsilon_2^2+2a'^2\epsilon_1^2+2b'^2\epsilon_2^2\leq \frac{ 3 }{ 4 }a'^2+\frac{ 3 }{2}b'^2.
    \end{equation}
    D'autre part \( N(t)=a'^2+2b'^2\), et nous avons donc bien \( N(r)<N(t)\).

    En ce qui concerne la seconde propriété du stathme, un petit calcul montre que
    \begin{equation}
        N(zt)=(a^2+2b^2)(a'^2+2b'^2),
    \end{equation}
    et tant que \( t\neq 0\) nous avons bien \( N(zt)>N(z)\).
\end{example}

Notons en particulier que \( \eZ[i\sqrt{2}]\) est factoriel et principal.

\begin{example} \label{ExluqIkE}
    Décomposition en facteurs irréductibles dans \( \eZ[i\sqrt{2}]\). Les éléments inversibles de \( \eZ[i\sqrt{2}]\) sont \( \pm 1\), donc deux éléments \( a\) et \( b\) sont associés (définition \ref{DefrXUixs}) si et seulement si \( a=\pm b\).

    De plus si \( p\) est irréductible, alors \( -p\) est irréductible. Les éléments irréductibles de \( \eZ[i\sqrt{2}]\) arrivent donc par pairs d'éléments associés. Soit \( \{ p_i \}\) une sélection de un élément irréductible parmi chaque paire. Tout élément \( x\) de \( \eZ[i\sqrt{2}]\) peut alors être écrit \( x=\pm p_1^{\alpha_1}\ldots p_n^{\alpha_n}\). Ce fait va être pratique pour comparer des décomposition en facteurs irréductibles d'éléments.
\end{example}

Le lemme suivant fait en pratique partie de l'exemple \ref{ExmuQisZU}, mais nous l'isolons pour plus de clarté\footnote{Merci à \href{http://fr.wikipedia.org/wiki/Utilisateur:Marvoir}{Marvoir} pour m'avoir souligné le manque.}.
\begin{lemma}       \label{LemTScCIv}
    Si \( a\) et \( b\) sont deux éléments premiers entre eux de \( \eZ[i\sqrt{2}]\), et s'il existe \( y \in  \eZ[i\sqrt{2}\) tel que \( ab=y^3\), alors \( a\) et \( b\) sont des cubes (dans \( \eZ[i\sqrt{2}]\)).
\end{lemma}

\begin{proof}
    D'après l'exemple \ref{ExluqIkE} nous pouvons écrire
    \begin{subequations}
        \begin{align}
            y&=\pm p_1^{\sigma_1}\ldots p_n^{\sigma_n}\\
            a&=\pm p_1^{\alpha_1}\ldots p_n^{\alpha_n}\\
            b&=\pm p_1^{\beta_1}\ldots p_n^{\beta_n}
        \end{align}
    \end{subequations}
    où les \( p_i\) sont les irréductibles de \( \eZ[i\sqrt{2}]\) «modulo \( \pm 1\)» au sens où la liste des irréductibles est \( \{ p_i \}\cup\{ -p_i \}\) (union disjointe). Étant donné que \( a\) et \( b\) sont premiers entre eux, \( \alpha_i\) et \( \beta_i\) ne peuvent pas être non nuls en même temps alors que leur somme doit faire \( 3\sigma_i\). Nous avons donc pour chaque \( i\) soit \( \alpha_i=3\sigma_i\) soit \( \beta=3\sigma_i\) (et bien entendu si \( \sigma_i=0\) alors \( \alpha_i=\beta_i=0\)).

    Étant donné que \( \pm 1\) sont également deux cubes, \( a\) et \( b\) sont bien des cubes.

    Notons que nous avons utilisé de façon capitale le fait que \( \eZ[i\sqrt{2}]\) était factoriel.
\end{proof}

%---------------------------------------------------------------------------------------------------------------------------
\subsection{Équations diophantiennes}
%---------------------------------------------------------------------------------------------------------------------------
%TODO : il y a une équation diophantienne qui semple pas mal ici : http://fr.wikipedia.org/wiki/Entier_quadratique#x2_.2B_5.y2_.3D_p

\begin{example} \label{ExZPVFooPpdKJc}
    L'équation diophantienne
    \begin{equation}
        x^2=3y^2+8
    \end{equation}
    n'a pas de solutions. En effet si nous prenons l'équation modulo \( 3\) nous obtenons
    \begin{equation}
        [x^2]_3=[3y^2+8]_3=[8]_3=[2]_3.
    \end{equation}
    Or dans \( \eZ/3\eZ\), aucun carré n'est égal à deux : \( 0^2=0\neq 2\), \( 1^2=1\neq 2\) et \( 2^2=4=1\neq 2\).
\end{example}

\begin{example}     \label{ExmuQisZU}
    Résolvons l'équation diophantienne\index{équation!diophantienne} 
    \begin{equation}
        x^2+2=y^3.
    \end{equation}
    Une première remarque est que \( x\) doit être impair. En effet si \( x=2k\), nous devons avoir \( y^3\) pair. Mais si un cube pair est divisible par \( 8\), donc \( y^3=8l\). L'équation devient \( 4k^2+2=8l^3\), c'est à dire \( 2k^2+1=4l^3\). Le membre de gauche est impair tandis que celui de droite est pair. Impossible.

    Nous pouvons écrire l'équation sous la forme \( x^2+2=(x+i\sqrt{2})(x-i\sqrt{2})\). Et nous considérons \( \eZ[i\sqrt{2}]\) muni de son stathme \( N\) donné par \eqref{EqOZUIooZGmHWl}.

    L'élément \( i\sqrt{2}\) est irréductible parce que \( N(i\sqrt{2})=2\), et si nous avions \( i\sqrt{2}=pq\), alors nous aurions \( N(p)N(q)=2\), ce qui n'est possible que si \( N(p)\) ou \( N(q)\) vaut \( 1\).

    Nous prouvons maintenant que les éléments \( x+i\sqrt{2}\) et \( x-i\sqrt{2}\) sont premiers entre eux. Supposons que \( d\) soit un diviseur commun; alors il divise aussi la somme et la différence. Donc \( d\) divise à la fois \( 2x\) et \( 2i\sqrt{2}\).

    Étant donné que \( i\sqrt{2}\) est irréductible et que \( 2i\sqrt{2}=(-i\sqrt{2})^3\), les diviseurs de \( 2i\sqrt{2}\) sont les puissances de \( (-i\sqrt{2})\). Du coup nous devrions avoir \( d=(i\sqrt{2})^{\beta}\) et donc
    \begin{equation}
        x=(i\sqrt{2})^{\beta}q
    \end{equation}
    pour un certain \( q\in\eZ[i\sqrt{2}]\). Dans ce cas nous avons \( N(x)=2^{\beta}N(q)\), mais nous avons déjà précisé que \( x\) ne pouvait pas être pair, donc \( \beta=0\) et nous avons \( d=1\).

    Vu que les nombres \( x\pm i\sqrt{2}\) sont premiers entre eux et que leur produit doit être un cube, ils doivent être séparément des cubes (lemme \ref{LemTScCIv}). Nous devons donc résoudre séparément \( x\pm i\sqrt{2}=y^3\).

    Cherchons les \( x\) et \( y\) entiers tels que \( x+i\sqrt{2}=y^3\). Si nous posons \( z=a+bi\sqrt{2}\), il suffit de calculer \( z^3\) :
    \begin{verbatim}
----------------------------------------------------------------------
| Sage Version 4.8, Release Date: 2012-01-20                         |
| Type notebook() for the GUI, and license() for information.        |
----------------------------------------------------------------------
sage: var('a,b')
(a, b)
sage: z=a+I*sqrt(2)*b
sage: (z**3).expand()
3*I*sqrt(2)*a^2*b - 2*I*sqrt(2)*b^3 + a^3 - 6*a*b^2
    \end{verbatim}
    En identifiant cela à \( x+i\sqrt{2}\) nous trouvons le système
    \begin{subequations}
        \begin{numcases}{}
            x=a^3-6ab^2\\
            1=3a^2b-2b^3
        \end{numcases}
    \end{subequations}
    où, nous le rappelons, \( x\), \( a\) et \( b\) sont des entiers. La seconde équation montre que \( b\) doit être inversible : \( b(3a^2-2b^2)=1\). Il y a donc les possibilités \( b=\pm 1\). Pour \( b=1\) l'équation devient \( 3a^2-2=1\), c'est à dire \( a=\pm 1\). Pour \( b=-1\) l'équation devient \( 3a^2-2=-1\), impossible. En conclusion les possibilités sont
    \begin{subequations}
        \begin{align}
            (x,z)=(-5,1+i\sqrt{2})\\
            (x,z)=(5,-1+i\sqrt{2})\\
        \end{align}
    \end{subequations}
    Le travail avec \( x-i\sqrt{2}\) donne les mêmes résultats.

    Les deux solutions de l'équation \( x^2+2=y^3\) sont alors \( (5,3)\) et \( (-5,3)\).
\end{example}

%--------------------------------------------------------------------------------------------------------------------------- 
\subsection{Triplets pythagoriciens et équation de Fermat pour \texorpdfstring{$ n=4$}{n=4}}
%---------------------------------------------------------------------------------------------------------------------------

\begin{definition}
    Les solutions entières (positives) de l'équation \( x^2+y^2=z^2\) sont appelés \defe{triplets pythagoriciens}{triplet!pythagoricien}. 
\end{definition}

Ils donnent toutes les possibilités de triangles rectangles dont les côtés ont des longueurs entières.

\begin{definition}
  On dit qu'un triplet pythagoricien est \defe{primitif}{primitif!triplet pythagoricien} si les trois nombres sont premiers dans leur ensemble\footnote{Définition
    \ref{DefZHRXooNeWIcB}.}.
\end{definition}

Remarquons que cela est équivalent à montrer que les trois nombres sont premiers deux à deux: en effet, si deux parmi \( x\), \( y\) et \( z\) sont divisibles par un nombre, alors tous les trois sont divisibles par ce nombre\footnote{Parce que \( k\) et \( k^2\) ont les mêmes facteurs premiers.}, donc les nombres \( x\), \( y\) et \( z\) sont premiers deux à deux.

\begin{lemma}    \label{LemTripletsPythagoriciensPrimitifs}
  Dans un triplet pythagoricien primitif \( (x, y, z) \), on a toujours $z$ impair et:
  \begin{itemize}
  \item
    soit $x$ impair et $y$ pair;
  \item
    soit $x$ pair et $y$ impair.
  \end{itemize}
\end{lemma}

\begin{proof}
  Remarquons que le fait d'imposer que le triplet soit primitif, interdit aux nombres $x$ et $y$ d'être pairs en même temps. En effet, si c'était le cas, alors \( x^2 \) et \( y^2 \) seraient aussi pairs, donc leur somme \( z^2 \) aussi, d'ou $z$ serait pair et les trois nombres ne seraient pas premiers entre eux.

  Nous montrons à présent que les nombres \( x\) et \( y\) ne sont pas tous les deux impairs. Par l'absurde, si \( x=2a+1\), nous avons \( x^2=4a^2+4a+1\in [1]_4\); de la même manière,  \( y^2 \in [1]_4\). On en déduit alors que \( z^2=x^2+y^2\in [2]_4\). Le nombre \(  z^2\) est donc pair, donc \( z\) est pair : disons \( z=2c\). Alors, \( z^2=4c^2\in [0]_4\). Comme les classes modulo 4 sont disjointes, nous aboutissons à une contradiction.
\end{proof}

\begin{proposition}[Triplets pythagoriciens\cite{fJhCTE,HARRooBvzbXo}]  \label{PropXHMLooRnJKRi}
    Un triplet \( (x,y,z)\in(\eN^*)^3\) est solution de \( x^2+y^2=z^2\) si et seulement s'il existe \( d\in \eN\) et \( u,v\in \eN^*\) premiers entre eux tels que
    \begin{subequations}        \label{subeqLVHFooVgWsFx}
        \begin{numcases}{}
            x=d(u^2-v^2)\\
            y=2duv\\
            z=d(u^2+v^2)
        \end{numcases}
    \end{subequations}
    ou
    \begin{subequations}
        \begin{numcases}{}
            x=2duv\\
            y=d(u^2-v^2)\\
            z=d(u^2+v^2)
        \end{numcases}
    \end{subequations}
    La différence entre les deux est seulement d'inverser les rôles de \( x\) et \( y\).
\end{proposition}

\begin{proof}
    Montrons d'abord que les formules proposées sont bien des solutions; nous vérifions \eqref{subeqLVHFooVgWsFx} :
    \begin{equation}
        x^2+y^2=d^2(u^2-v^2)+4d^2u^2v^2=d^2(u^2+v^2)^2,
    \end{equation}
    qui correspond bien au \( z^2\) proposé.

    Nous allons maintenant prouver la réciproque : toute solution est d'une des deux formes proposées. Déterminer les triplets primitifs suffira parce que si \( (x,y,z)\) n'est pas une solution primitive, alors en posant \( k=\pgcd(x,y,z)\), le triplet \( \big( \frac{ x }{ k },\frac{ y }{ k },\frac{ z }{ k } \big)\) est primitif. Connaissant les triplets primitifs, nous obtenons tous les autres par simple multiplication.

    Soit donc \( (x,y,z)\) un triplet pythagoricien primitif. Sans
    perte de généralité\footnote{En échangeant les rôles de $x$ et $y$
      ici, nous obtenons à la fin la seconde forme des solutions.},
    grâce au lemme \ref{LemTripletsPythagoriciensPrimitifs}, on peut
    supposer \( x\) pair, \( y\) impair, \( z\) impair. Comme \(
    x^2=(z+y)(z-y)\), nous avons
    \begin{equation}
        \left( \frac{ x }{2} \right)^2=\left( \frac{ z+y }{2} \right)^2\left( \frac{ z-y }{ 2 } \right)^2.
    \end{equation}
    Vu que \( z\) et \( y\) sont premiers entre eux, les nombres \( z-y\) et \( z+y\) sont également premiers entre eux\footnote{Si \( z-y=kn\) et \( z+y=km\), faisant la somme et la différence on voit que \( y\) et \( z\) sont divisibles par \( k\).}. Donc les facteurs premiers (qui arrivent au moins au carré) de \( (x/2)^2\) sont chacun soit dans \( (z+y)/2\) soit dans \( (z-y)/2\). Nous en déduisons que ces derniers sont des carrés d'entiers. Nous posons
    \begin{equation}
        \begin{aligned}[]
            \frac{ z-y }{2}=u^2&&\frac{ z+y }{2}=v^2.
        \end{aligned}
    \end{equation}
    Bien entendu \( u\) et \( v\) sont non nuls parce que nous avons exclu la possibilité de triplets dont un élément serait nul. Avec tout cela nous avons \( (x/2)^2=u^2v^2\) et donc \( x=2uv\) puis par somme et différence :
    \begin{subequations}
        \begin{numcases}{}
            x=2uv\\
            y=v^2-u^2\\
            z=u^2+v^2,
        \end{numcases}
    \end{subequations}
    ce qu'il fallait.
\end{proof}

\begin{remark}
    Les solutions dans lesquelles \( x\), \( y\) ou \( z\) sont nuls sont faciles à classer. La solution \( (1,0,1)\) n'est pas dans les formes proposées. En effet elle ne peut pas être de la première forme : avoir \( y=0\) demanderait qu'un nombre parmi \( d\), \( u\) et \( v\) soit nul, ce qui est interdit. La solution \( (1,0,1) \) ne peut pas non plus être de la seconde forme parce que \( x\) y est pair.
\end{remark}

\begin{proposition}[\cite{fJhCTE}]      \label{propFKKKooFYQcxE}
    Les équations \( x^4+y^4=z^2\) et \( x^2+y^4=z^4\) n'ont pas de solutions dans \( (\eN^*)^3\).
\end{proposition}
\index{équation!diophantienne}

\begin{proof}
  Si la première équation n'a pas de solutions, alors la seconde n'en
  n'a pas non plus parce que \( z^4\) est un carré. Nous nous
  concentrons donc sur l'équation \( x^4+y^4=z^2\) et nous supposons
  qu'il existe au moins une solution dans \( (\eN^*)^3\). Nous en choisissons une \( (x,y,z)\) avec \( z\) minimum (les \( z\) dans différentes solutions étant dans \( \eN\), il en existe forcément un minimum\footnote{Voir quelque chose comme le lemme \ref{PropQEPoozLqOQ}.}). Du coup, les trois nombres \( x\), \( y\) et \( z\) sont premiers dans leur ensembles parce que une
  division par leur \( \pgcd\) donnerait une nouvelle solution qui
  contredirait la minimalité de \( z\).

    Nous posons \( x^4=\bar x^2\) et \( y^4=\bar y^2\). Ils vérifient
    l'équation \( \bar x^2+\bar y^2=z^2\) et par la proposition
    \ref{PropXHMLooRnJKRi}, il existe \( u,v\in \eN^*\) premiers entre
    eux tels que, sans perte de généralité\footnote{En inversant les
      rôles de $x$ et $y$ au besoin.}, on ait
    \begin{subequations}
        \begin{numcases}{}
            \bar x=2uv\\
            \bar y=u^2-v^2\label{eqnFKKKooFYQcxE1}\\
            z=u^2+v^2.\label{eqnFKKKooFYQcxE2}
        \end{numcases}
    \end{subequations}
    Si \( u\) est pair, alors \( v\) est impair (et inversement) parce
    que \( \pgcd(u,v)=1\) Si \( u\) est pair, alors \( u=2a\) et \(
    v=2b+1\), ce qui donne \( \bar y=4a^2-4b^2-4b-1\in[-1]_4\). Or
    nous avons déjà vu qu'un carré est dans \( [0]_4\) ou dans \(
    [1]_4\). Il faut donc que \( u\) soit impair. Le lemme
    \ref{LemTripletsPythagoriciensPrimitifs} implique alors que \( v\)
    soit pair.

    De l'équation \ref{eqnFKKKooFYQcxE1}, nous en déduisons que \(
    v^2+\bar y=u^2\); de plus \( u^2\), \( v^2\) et \( \bar y\) sont
    premiers dans leur ensemble: en effet, $u$ et $v$ sont premiers
    entre eux, et si l'un parmi \( u^2\) et \( v^2\) a un facteur
    commun avec \( \bar y\), alors l'autre doit l'avoir aussi (dans
    une égalité \( a+b=c\), si deux des nombres ont un diviseur
    commun, le troisième l'a aussi). Comme \( \bar y=y^2\), le triplet
    \( (v,y,u)\) est un triplet pythagoricien primitif. Nous
    réappliquons la proposition \ref{PropXHMLooRnJKRi}, en se
    souvenant que $v$ est pair: il existe donc deux nombres \( r\) et
    \( s\) premiers entre eux tels que
    \begin{subequations} \label{eqnFKKKooFYQcxE3}
        \begin{numcases}{}
            v=2rs\\
            y=r^2-s^2\\
            u=r^2+s^2.
        \end{numcases}
    \end{subequations}
    Avec cela, \( \bar x=2uv=4rs(r^2+s^2)\). Remarquons que les trois
    nombres \( r\), \( s\) et \( r^2+s^2\) sont premiers entre
    eux dans leur ensemble; or, comme \( \bar x\) est un
    carré ces nombres doivent séparément être des carrés :
    \begin{subequations}
        \begin{numcases}{}
            r=\alpha^2\\
            s=\beta^2\\
            r^2+s^2=\gamma^2.
        \end{numcases}
    \end{subequations}
    En mettant les deux premiers dans la troisième, nous avons \( \alpha^4+\beta^4=\gamma^2\). Donc \( (\alpha^2,\beta^2,\gamma)\) est une solution. Nous allons prouver que \( \gamma<z\), ce qui terminera la preuve, puisque $z$ était supposé minimal. Nous avons :
    \begin{align*}
        z&=u^2+v^2&&\text{par \ref{eqnFKKKooFYQcxE2}}\\
          &=r^2+s^2+4r^2s^2&&\text{par \ref{eqnFKKKooFYQcxE3}}\\
          &=\gamma^2+4r^2s^2\\
          &> \gamma^2,
    \end{align*}
    et a fortiori \( \gamma<z\).
\end{proof}

%--------------------------------------------------------------------------------------------------------------------------- 
\subsection{Lignes et colonnes de matrices}
%---------------------------------------------------------------------------------------------------------------------------

Nous nommons \( E_{ij}\) la matrice remplie de zéros sauf à la case \( ij\) qui vaut \( 1\). Autrement dit
\begin{equation}
    (E_{ij})_{kl}=\delta_{ik}\delta_{jl}.
\end{equation}
\begin{definition}
    Une \defe{matrice de transvection}{transvection (matrice)}\index{matrice!de transvection} est une matrice de la forme
    \begin{equation}
        T_{ij}(\lambda)=\id+\lambda E_{ij}
    \end{equation}
    avec \( i\neq j\).

    Une \defe{matrice de dilatation}{matrice!de dilatation}\index{dilatation (matrice)} est une matrice de la forme
    \begin{equation}
        D_i(\lambda)=\id+(\lambda-1)E_{ii}.
    \end{equation}
    Ici le \( (\lambda-1)\) sert à avoir \( \lambda\) et non \( 1+\lambda\). C'est donc une matrice qui dilate d'un facteur \( \lambda\) la direction \( i\) tout en laissant le reste inchangé.

    Si \( \sigma\) est une permutation (un élément du groupe symétrique \( S_n\)) alors la \defe{matrice de permutation}{matrice!de permutation}\index{permutation!matrice} associée est la matrice d'entrées
    \begin{equation}
        (P_{\sigma})_{ij}=\delta_{i\sigma(j)}.
    \end{equation}
\end{definition}

\begin{lemma}   \label{LemyrAXQs}
    La matrice \( T_{ij}(\lambda)A=(\mtu+\lambda E_{ij})A\) est la matrice \( A\) à qui on a effectué la substitution
    \begin{equation}
        L_i\to L_i+\lambda L_j.
    \end{equation}
    La matrice \( AT_{ij}(\lambda)\) est la substitution 
    \begin{equation}
        C_j\to C_j+\lambda C_i.
    \end{equation}

    La matrice \( AP_{\sigma}\) est la matrice \( A\) dans laquelle nous avons permuté les colonnes avec \( \sigma\).

    La matrice \( P_{\sigma}A\) est la matrice \( A\) dans laquelle nous avons permuté les lignes avec \( \sigma^{-1}\).
\end{lemma}

\begin{proof}
    Calculons la composante \( kl\) de la matrice \( E_{ij}A\) :
    \begin{subequations}
        \begin{align}
            (E_{ij}A)_{kl}&=\sum_m(E_{ij})_{km}A_{ml}\\
            &=\sum_m\delta_{ik}\delta_{jm}A_{ml}\\
            &=\delta_{ik}A_{jl}.
        \end{align}
    \end{subequations}
    C'est donc la matrice pleine de zéros, sauf la ligne \( i\) qui est donnée par la ligne \( j\) de \( A\). Donc effectivement la matrice
    \begin{equation}
        A+\lambda E_{ij}A
    \end{equation}
    est la matrice \( A\) à laquelle on a substitué la ligne \( i\) par la ligne \( i\) plus \( \lambda\) fois la ligne \( j\).

    En ce qui concerne l'autre assertion sur les transvections, le calcul est le même et nous obtenons
    \begin{equation}
        (AE_{ij})=A_{ki}\delta_{jl}.
    \end{equation}

    Pour les matrices de permutation, nous avons 
    \begin{equation}
        (AP_{\sigma})_{kl}=A_{k\sigma(l)}
    \end{equation}
    et
    \begin{equation}
        (P_{\sigma}A)_{kl}=\sum_m\delta_{k\sigma(m)}A_{ml}=\sum_m\delta_{\sigma^{-1}(k)m}A_{ml}=A_{\sigma^{-1}(k)l}.
    \end{equation}
\end{proof}


%--------------------------------------------------------------------------------------------------------------------------- 
\subsection{Algorithme des facteurs invariants}
%---------------------------------------------------------------------------------------------------------------------------

\begin{probleme}
Définir les anneaux de matrices, et en particulier \(GL_n(A) \).
\end{probleme}

\begin{proposition}[Algorithme des facteurs invariants\cite{KXjFWKA}]   \label{PropPDfCqee}
    Soit \( (A,\delta)\) un anneau euclidien muni de son stathme  et \( U\in \eM(n,m,A)\). Alors il existe \( d_1,\ldots, d_s\in A^*\) et des matrices \( P\in\GL(m,A)\), \( Q\in \GL(n,A)\) tels que nous ayons
    \begin{equation}
        U=P \begin{pmatrix}
            \begin{matrix}
                d_1    &       &       \\
                    &   \ddots    &       \\
                    &       &   d_s
            \end{matrix}&   0    \\ 
            0    &   0    
        \end{pmatrix}Q
    \end{equation}
    avec \( d_i\divides d_{i+1}\) pour tout \( i\).
\end{proposition}
\index{anneau!euclidien!facteurs invariants}
\index{algorithme!facteurs invariants}

\begin{proof}
    Nous allons donner la preuve plus ou moins sous forme d'algorithme.

    D'abord si \( U=0\) c'est bon, on a la réponse. Sinon, nous prenons l'élément \( (i_0,j_0)\) dont le stathme est le plus petit et nous l'amenons en \( (1,1)\) par les permutations
    \begin{equation}
        \begin{aligned}[]
            C_1&\leftrightarrow C_{j_0}\\
            L_1&\leftrightarrow L_{i_0}
        \end{aligned}
    \end{equation}
    Ensuite nous traitons la première colonne jusqu'à amener des zéros partout en dessous de \( u_{11}\) de la façon suivante : pour chaque ligne successivement nous calculons la division euclidienne
    \begin{equation}
        u_{i1}=qu_{11}+r_i,
    \end{equation}
    et nous faisons
    \begin{equation}
        L_i\to L_i-qL_1,
    \end{equation}
    c'est à dire que nous enlevons le maximum possible et il reste seulement \( r_i\) en \( u_{i1}\). Vu que le but est de ne laisser que des zéros dans la première colonne, si le reste n'est pas zéro nous ne sommes pas content\footnote{S'il est zéro, nous passons à la ligne suivante}. Dans ce cas nous permutons \( L_1\leftrightarrow L_i\), ce qui aura pour effet de strictement diminuer le stathme de \( u_{11}\) parce qu'on va mettre en \( u_{11}\) le nombre \( r_i\) dont le stathme est strictement plus petit que celui de \( u_{11}\).

    En faisant ce jeu de division euclidienne puis échange, on diminue toujours le stathme de \( u_{11}\), donc ça finit par s'arrêter, c'est à dire qu'à un certain moment la division euclidienne de \( u_{i1}\) par \( u_{11}\) va donner un reste zéro et nous serons content.

    Une fois la première colonne ramenée à la forme
    \begin{equation}
        C_1=\begin{pmatrix}
            u_{11}    \\ 
            0    \\ 
            \vdots    \\ 
            0    
        \end{pmatrix},
    \end{equation}
    nous faisons tout le même jeu avec la première ligne en faisant maintenant des sommes divisions et permutations de colonnes. Notons que ce faisant nous ne changeons plus la première colonne.

    En fin de compte nous trouvons une matrice\footnote{Nous nommons toujours par la même lettre \( U\) la matrice originale et la modifiée, comme il est d'usage en informatique.}
    \begin{equation}
        U=\begin{pmatrix}
            u_{11}   &   0    &   \ldots    &   0    \\
             0   &       &       &       \\
             \vdots   &       &   A    &       \\ 
             0   &       &       &        
         \end{pmatrix}
    \end{equation}
    Si l'élément \( u_{11}\) ne divise pas un des éléments de \( A\), disons \( a_{ij}\), alors nous faisons 
    \begin{equation}
        C_1\to C_1-C_j.
    \end{equation}
    Cela nous détruit un peu la première colonne, mais ne change pas \( u_{11}\). Nous avons maintnant
    \begin{equation}
        U=\begin{pmatrix}
            u_{11}   &   0    &   \ldots    &   0    \\
             0   &       &       &       \\
             *   &       &       &       \\ 
             u_{ij}   &       &   A    &       \\ 
             *   &       &       &       \\ 
             0   &       &       &        
         \end{pmatrix}
    \end{equation}
    Et nous refaisons tout le jeu depuis le début. Cependant lorsque nous allons nous attaquer à la ligne \( i\), \( u_{11}\) ne divisera pas \( u_{ij}\), ce qui donnera lieu à une division euclidienne et un échange \( L_1\leftrightarrow L_i\). L'échange consistant à mettre \( r_i\) à la place de \( u_{11}\) et inversement  diminuera encore strictement le stathme. Encore une fois nous allons travailler jusqu'à avoir la matrice sous la forme
    \begin{equation}    \label{EqADcNVgI}
        U=\begin{pmatrix}
            u_{11}   &   0    &   \ldots    &   0    \\
             0   &       &       &       \\
             \vdots   &       &   A    &       \\ 
             0   &       &       &        
         \end{pmatrix},
    \end{equation}
    sauf que cette fois le stathme de \( u_{11}\) est strictement plus petit que la fois précédente. Si \( u_{11}\) ne divise toujours pas tous les éléments de \( A\), nous recommençons encore et encore. En fin de compte nous finissons par avoir une matrice de la forme \eqref{EqADcNVgI} avec \( u_{11}\) qui divise tous les éléments de \( A\).

    Une fois que cela est fait, il faut continuer en recommençant tout sur la matrice \( A\). Nous avons maintenant
    \begin{equation}
        U=\begin{pmatrix}
            \begin{matrix}
                u_{11}  &       \\ 
                &   u_{22}    
            \end{matrix}&   0    \\ 
            0    &   B    
        \end{pmatrix}.
    \end{equation}
    Sous cette forme nous avons \( u_{11}\divides u_{22}\) et \( u_{11}\) divise tous les éléments de \( B\). En effet \( u_{11}\) divisant tous les éléments de \( A\), il divise toutes les combinaisons de ces éléments. Or tout l'algorithme ne consiste qu'à prendre des combinaisons d'éléments.

    Nous finissons donc bien sûr une matrice comme annoncée. De plus n'ayant effectué que des combinaisons de lignes, nous avons seulement multiplié par des matrices inversibles (lemme \ref{LemyrAXQs}).
\end{proof}

%+++++++++++++++++++++++++++++++++++++++++++++++++++++++++++++++++++++++++++++++++++++++++++++++++++++++++++++++++++++++++++
\section{Polynômes à coefficients dans un anneau commutatif}
%+++++++++++++++++++++++++++++++++++++++++++++++++++++++++++++++++++++++++++++++++++++++++++++++++++++++++++++++++++++++++++
\label{SECooVMABooVdhbPo}

Soit \( A\) un anneau commutatif. Nous considérons \( \polyP\) l'ensemble des suites presque nulles d'éléments de \( A\), ce sont les suites \( (a_n)_{n\in\eN}\) telles que il existe \( N\) tel que \( a_i=0\) pour tout \( i>N\).

Cela est un \( A\)-module libre de base\footnote{Définition \ref{DefBasePouyKj}.}
\begin{equation}
    (e_n)_k=\delta_{nk}.
\end{equation}
Si \( (a_n)_{n\in \eN}\) et \( (b_n)_{n\in\eN}\) sont des éléments de \( \polyP\), nous définissons le produit \( ab\) par
\begin{equation}
    (ab)_n=\sum_{p+q=n}a_pb_q.
\end{equation}
Cela est bien un élément de \( \polyP\) parce qu'il existe \( N\in\eN\) tel que \( a_n=b_n=0\) pour tout \( n\geq N\). Avec la somme et le produit par un scalaire, le module \( \polyP\) devient une \( A\)-algèbre commutative unitaire. L'unité est 
\begin{equation}
    e_0=(1,0,\ldots).
\end{equation}

\begin{definition}  \label{DefRGOooGIVzkx}
    En tant que \( A\)-algèbre, l'ensemble \( \polyP\) est l'\defe{algèbre des polynômes en une indéterminée}{algèbre!polynômes} à coefficients dans \( A\).
\end{definition}

\begin{definition}  \label{DefDegrePoly}
    Soit \( P \in \polyP\), \( P \neq 0 \). On appelle \defe{degré}{degré!d'un polynôme} de $P$ le plus grand nombre naturel $n$ pour lequel le coefficient correspondant est non-nul. Ce naturel est noté \( \deg(P) \).
\end{definition}

Si nous posons que \( X=e_1\), et que nous prenons la convention \( X^0=1\), alors nous avons \( e_k=X^k\) et nous notons \( A[X]\)\nomenclature[A]{\( A[X]\)}{tous les polynômes de degré fini à coefficients dans \( A\)} l'anneau \( \polyP\) exprimé avec \( X\). Les éléments de la forme \( \lambda X^k\) avec \( \lambda\in A\) et \( k\in\eN\) sont des \defe{monômes}{monôme}. Nous allons aussi considérer\nomenclature[A]{\( A_n[X]\)}{les polynômes à coefficients dans \( A\) et de degré inférieur à \( n\)}
\begin{equation}
    A_n[X]=\{ P\in A[X]\tq \deg(P)\leq n \}.
\end{equation}
Cela est un sous module libre.

\begin{remark}  \label{RemLIOooXHePSd}
    L'ensemble \( A[X]\) est une algèbre et donc un espace vectoriel. Il possède un unique élément nul qui est celui dont tous les coefficients sont nuls; cela est immédiat par la construction en tant que suites presque nulles.

    Il n'y a a priori pas équivalence entre le fait d'être un polynôme nul et le fait de s'évaluer à zéro sur tous les éléments de \( A\). Cela sera discuté dans le théorème \ref{ThoLXTooNaUAKR} et l'exemple \ref{exVQBooBMPLkD}.
\end{remark}

\begin{theorem}     \label{ThoBUEDrJ}
    L'anneau \( A\) est intègre si et seulement si \( A[X]\) est intègre.
\end{theorem}

\begin{proof}
    Soient \( P\) et \( Q\) des éléments non nuls de \( A[X]\). Vu que l'anneau \( A\) est intègre, nous avons
    \begin{equation}
        \deg(PQ)=\deg(P)+\deg(Q)
    \end{equation}
    et le produit ne peut pas être nul. L'anneau \( A[X]\) est donc intègre.

    Si \( A[X]\) est intègre, \( A\) est intègre parce qu'il peut être vu comme sous anneau.
\end{proof}

\begin{normaltext}
    Si \( A\) n'est pas intègre, soient \( \alpha,\beta\in A\) non nuls tels que \( \alpha\beta=0\). Le produit est polynômes \( X\mapsto \alpha X\) et \( X\mapsto \beta\) est \( (\alpha X)(\beta)=0\); le degré du produit n'est pas la somme des degrés.

    Les personnes qui ont tout compris jusqu'ici remarqueront que la notation «\( X\mapsto P(X)\)» n'est pas correcte parce que du point de vue que nous adoptons ici, un polynôme n'est pas une application.
\end{normaltext}

\begin{corollary}
    Si \( A\) est intègre, les inversibles de \( A[X]\) sont les éléments de \( U(A)\).
\end{corollary}

\begin{proof}
    Pour que \( Q\) soit inversible, il faut un \( P\) tel que \( PQ=1\). Mais l'anneau \( A\) étant intègre, les degrés s'additionnent. Par conséquent ils doivent être de degrés zéro et il faut que \( P,Q\in A\). Enfin pour qu'ils soient inversibles, ils doivent être dans \( U(A)\).
\end{proof}

La \defe{valuation}{valuation!d'un polynôme} du polynôme \( P=\sum_n a_nX^n\), notée \( \val(P)\), est 
\begin{equation}
    \val(P)=\min\{ n\tq a_n\neq 0 \}.
\end{equation}
Nous avons \( \val(P)\leq \deg(P)\) et \( \val(P)=\deg(P)\) si et seulement si \( P\) est un monôme. Si \( P=0\), nous convenons que \( \val(0)=\infty\) et \( \deg(0)=-\infty\).

%---------------------------------------------------------------------------------------------------------------------------
\subsection{Division euclidienne}
%---------------------------------------------------------------------------------------------------------------------------

Le théorème suivant établit la \defe{division euclidienne}{division!euclidienne} dans \( A[X]\) du polynôme \( P\) par un polynôme \( D\).
\begin{theorem}     \label{ThodivEuclPsFexf}
    Soit \( D\neq 0\) dans \( A[X]\) de coefficient dominant inversible dans \( A\). Pour tout \( P\in A[X]\), il existe \( Q,R\in A[X]\) tels que
    \begin{equation}
        P=QD+R
    \end{equation}
    avec \( \deg(R)<\deg(D)\).

    Les polynômes \( Q\) et \( R\) sont déterminés de façon univoque par cette condition. 
\end{theorem}

\begin{definition}\label{DefMPZooMmMymG}
    Le polynôme \( Q\) est le \defe{quotient}{quotient} et \( R\) est le \defe{reste}{reste} de la division euclidienne de \( P\) par \( D\). Si le reste de la division de \( P\) par $D$ est nul on dit que \( D\) \defe{divise}{diviseur!polynôme} \( P\) et on note \( D\divides P\)\nomenclature[A]{\( D\divides P\)}{\( D\) divise \( P\)}. Autrement dit \( D\) divise \( P\) s'il existe \( Q\) tel que \( P=QD\).\footnote{Ceci se rapproche tout naturellement des notions de divisibilité dans un anneau intègre général, vues en sous-section \ref{DivisibiliteAnneauxIntegres}.}
\end{definition}

\begin{normaltext}
    Le théorème \ref{ThodivEuclPsFexf} nous incite à utiliser le degré comme stathme euclidien sur \( A[X]\) dès que \( A\) est un anneau intègre. Or cela ne fonctionne en général pas parce que très peu de polynômes ont a priori un coefficient dominant inversible.
\end{normaltext}

\begin{lemma}       \label{LEMooIDSKooQfkeKp}
    Si \( \eK\) est un corps\footnote{Définition \ref{DefTMNooKXHUd}.}, alors l'anneau \( \eK[X]\) est euclidien et principal.
\end{lemma}

\begin{proof}
    Vu que \( \eK\) est un corps, tous les éléments sont inversibles et le degré donne un stathme par le théorème \ref{ThodivEuclPsFexf}. L'anneau \( \eK[X]\) est donc euclidien et par conséquent principal (proposition \ref{Propkllxnv}). 
\end{proof}

Dans le théorème \ref{ThoCCHkoU} nous donnerons une preuve directe du fait que \( \eK[X]\) est principal en montrant que tous ses idéaux sont principaux. Nous y démontrerons donc un peu moins pour un peu plus cher, mais avec le plaisir de ne pas devoir passer par un stathme.

\begin{definition}[\cite{ooSXFEooEehobn}]  \label{DefDSFooZVbNAX}
    Soit un anneau \( A\). Deux polynômes \( P\) et \( Q\) dans \( A[X]\) sont dits \defe{étrangers}{etranger@étrangers!polynômes} entre eux si \( 1\) est un pgcd\footnote{Définition \ref{DEFooTCUOooWHlbee}.} de \( P\) et \( Q\). Un ensemble de polynômes \( (P_i)_{i\in I}\) est étranger \defe{dans leur ensemble}{étranger!dans leur ensemble} si \( 1\) est un \( \pgcd\) des \( P_i\).
    
Les polynômes \( P\) et \( Q\) sont \defe{premiers entre eux}{premier!deux polynômes entre eux} si les seuls diviseurs communs de \( P\) et \( Q\) sont les inversibles.
\end{definition}

Les notions de polynômes étrangers entre eux ou de polynômes premiers entre eux ne sont pas identiques, comme le montre l'exemple suivant.

\begin{example}[\cite{MonCerveau}]
    Soient dans \( \eZ[X]\) les polynômes \( P(X)=2X+2\) et \( Q(X)=2X^2+2\). Le nombre \( 2\) est diviseur commun et n'est pas un diviseur de \( 1\). Donc \( 1\) n'est pas un pgcd de \( P\) et \( Q\). Ils ne sont pas étrangers.

    Mais ils sont premiers entre eux parce qu'ils n'ont pas d'autres diviseurs communs que les inversibles (\( 1\) et \( -1\)).
\end{example}

%--------------------------------------------------------------------------------------------------------------------------- 
\subsection{Polynôme primitif}
%---------------------------------------------------------------------------------------------------------------------------

\begin{definition}\label{DefContenuPolynome}
    Le \defe{contenu}{contenu}\index{polynôme!contenu} du polynôme \( P=\sum_ia_iX^i\in\eK[X]\) est le pgcd de ses coefficients : $c(P)=\pgcd(a_i)$.
\end{definition}

\begin{definition}[Ordre d'un polynôme]
    Soit \( P\) un polynôme irréductible de degré \( n\) sur \( \eF_p[X]\). L'\defe{ordre}{ordre!d'un polynôme} de \( P\) est
    \begin{equation}
        \min\{ k\tq P\divides X^k-1 \}.
    \end{equation}
\end{definition}

\begin{definition}[Polynôme primitif]           \label{DEFooDVOOooKaPZQC}
    Soit \( p\), un nombre premier et \( P\) un polynôme de degré $n$ dans \( \eF_p[X]\). Nous disons que \( P\) est \defe{primitif}{primitif!polynôme} si 
    \begin{enumerate}
        \item
            \( P\) est unitaire et irréductible,
        \item
            les racines de \( P\) sont d'ordre \( p^n-1\) dans \( \eF_p[X]/P\).
    \end{enumerate}
\end{definition}

\begin{definition}[Polynôme primitif au sens du pgcd]       \label{DEFooAIYGooRAEfHU}
    Soit un anneau \( A\). Un polynôme \( P\in A[X]\) est \defe{primitif au sens du pgcd}{primitif!polynôme!au sens du pgcd} si ses coefficients sont premiers entre eux.
\end{definition}

\begin{normaltext}
    Pour rappel, il y a plusieurs façons de périphraser le fait que les coefficients soient premiers entre eux. Nous pouvons dire \ldots
    \begin{enumerate}
        \item
            Le pgcd de ses coefficients est \( 1\) parce que c'est la définition \ref{DEFooXSPFooPumQSy} d'avoir des nombres premiers entre eux.
        \item
            Le contenu de ses coefficients est \( 1\). Parce que le contenu est précisément le pgcd, définition \ref{DefContenuPolynome}.
    \end{enumerate}
\end{normaltext}

La notion de polynôme primitif au sens du pgcd est particulière aux polynôme à coefficients dans un anneau comme le montre le lemme suivant.

\begin{lemma}
    Si \( \eK\) est un corps, tout polynôme unitaire dans \( \eK[X]\) non nul est primitif au sens du pgcd.
\end{lemma}

\begin{proof}
    Un polynôme unitaire a un \( 1\) parmi ses coefficients, donc le pgcd est forcément \( 1\). 
\end{proof}

Lorsque nous utiliserons la notion de polynôme primitif au sens du \( \pgcd\), nous le mentionnerons explicitement. C'est pas exemple le cas pour le corollaire \ref{CORooZCSOooHQVAOV}.

%--------------------------------------------------------------------------------------------------------------------------- 
\subsection{Racines des polynômes}
%---------------------------------------------------------------------------------------------------------------------------

\begin{definition}
  Soient \( A \) un anneau et \( P \in A[X] \). On appelle
  \defe{racine}{racine!d'un polynôme} un élément \( \alpha \in A \)
  tel que \( P(\alpha) = 0 \); c'est-à-dire que, en remplaçant toutes
  les occurrences de $X$ par $\alpha$ dans l'expression de $P$, on
  obtient $0$.
\end{definition}

\begin{proposition} \label{PropHSQooASRbeA}
    Soient \( A\) un anneau et \( P\) un polynôme non nul dans \( A[X]\). Si \( \alpha\in A\) est une racine de \( P\) alors \( X-\alpha\) divise \( P\), et réciproquement.
\end{proposition}

\begin{proof}
  Nous notons le polynôme \( \mu=X-\alpha\) par analogie avec le polynôme minimal dont il sera question dans la très semblable proposition \ref{PropXULooPCusvE}. Le sens réciproque est clair: si $\mu$ divise $P$, alors $\alpha$ est racine de $P$.

  Pour le sens direct, remarquons que si $\alpha$ est racine de $P$, alors $P$ est de degré au moins égal à \( 1\), et nous pouvons donc effectuer la division euclidienne\footnote{Théorème \ref{ThodivEuclPsFexf}.} de \( P\) par \( \mu\) : il existe des polynômes \( Q\) et \( R\) tels que
    \begin{equation} \label{PropHSQooASRbeA1}
        P=Q\mu+R
    \end{equation}
    avec \( \deg(R)<\deg(\mu)\). Donc \( R\) est une constante,
    élément de $A$: appelons-le $a$. En évaluant
    \eqref{PropHSQooASRbeA1} en \( \alpha\), il vient
    \begin{equation}
        0 = P(\alpha)=Q(\alpha)\mu(\alpha)+a,
    \end{equation}
    et nous en déduisons que \( a=0\), ce qui montre que \( P=Q\mu\) et que \( \mu\) divise \( P\).
\end{proof}

\begin{definition}[Racine simple et multiple d'un polynôme]
  Soit \( A\) un anneau ainsi qu'un polynôme \( P\in A[X]\) et \( \alpha\in A\) racine de $P$. La \defe{multiplicité}{multiplicité!racine d'un polynôme} de \( \alpha\) par rapport à \( P\) est l'entier \( h\) tel que \( P\) est divisible par \( (X-\alpha)^h\) mais pas divisible par \( (X-\alpha)^{h+1}\).  Nous noterons \( \theta_{\alpha}(P)\)\nomenclature[A]{\( \theta_{\alpha}(P)\)}{la multiplicité de \( \alpha\) par rapport à \( P\)} la multiplicité de \( \alpha\) par rapport à \( P\).
\end{definition}

Pour une définition générale d'une racine simple de l'équation \( f(x)=0\), voir la définition \ref{DEFooXSOQooAnWqKM}.

La proposition \ref{PropHSQooASRbeA} nous indique que toute racine est de multiplicité au moins \( 1\).

\begin{proposition} \label{PropahQQpA}
  L'élément \( \alpha\in A\) est une racine de multiplicité \( h\) du
  polynôme \( P\) si et seulement s'il existe \( Q\in A[X]\) tel que
  \( P=(X-\alpha)^hQ\) avec \( Q(\alpha)\neq 0\).
\end{proposition}

\begin{lemma}       \label{LemIeLhpc}
    Soient \( P\) et \( Q\) des polynômes non nuls de \( A[X]\) et \( \alpha\in A\). Alors
    \begin{enumerate}
        \item
            \( \theta_{\alpha}(P+Q)\leq\min\{
            \theta_{\alpha}(P),\theta_{\alpha}(Q) \}\), et l'égalité a
            lieu si \( \theta_{\alpha}(P)\neq \theta_{\alpha}(Q)\);
        \item     \label{ItemIeLhpciv}
            \( \theta_{\alpha}(PQ)\geq
            \theta_{\alpha(P)}+\theta_{\alpha}(Q)\), et l'égalité a
            lieu si \( A \) est intègre.
    \end{enumerate}
\end{lemma}

\begin{theorem} \label{ThoSVZooMpNANi} Soit \( A\) un anneau intègre
  et \( P\in A[X]\setminus\{ 0 \}\), un polynôme de degré \( n\). Si
  \( \alpha_1,\ldots, \alpha_p\in A\) sont des racines deux à deux
  distinctes de multiplicités \( k_1,\ldots, k_p\), alors il existe \(
  Q\in A[X]\), de degré \( n-p\), tel que \(
  P=Q\prod_{i=1}^p(X-\alpha_i)^{k_i}\) et \( Q(\alpha_i)\neq 0\) pour
  tout $i$.
    De plus la somme des multiplicités des racines de \( P\) est au plus \( \deg(P)\).
\end{theorem}
\index{factorisation!de polynôme}

\begin{proof}
    Si \( p=1\), soit \( \alpha\) une racine de multiplicité \( k\) de \( P\). La définition de la multiplicité d'une racine nous dit que \( P\) est divisible par \( (X-\alpha)^k\) mais pas par \( (X-\alpha)^{k+1}\). Donc il existe \( Q\in \eA[X]\) tel que \( P=Q(X-\alpha)^k\). Il reste à voir que \( Q(\alpha)\neq 0\). Cela est une conséquence de la proposition \ref{PropHSQooASRbeA} : si \( Q(\alpha)\) était nul, on pourrait lui factoriser \( (X-\alpha)\) et donc avoir \( (X-\alpha)^{k+1}\) qui se factorise dans \( P\), ce qui n'est pas possible.

    Nous supposons que \( p\geq 2\) et nous effectuons une récurrence sur \( p\). Nous considérons donc les \( p-1\) premières racines \( \alpha_1,\ldots, \alpha_{p-1}\) et un polynôme \( R\in\eA[X]\) tel que \( R(\alpha_i)\neq 0\) pour \( i=1,\ldots, p-1\) et
    \begin{equation}
        P=\underbrace{(X-\alpha_1)^{k_1}\ldots (X-\alpha_{p-1})^{k_{p-1}}}_SR.
    \end{equation}
    Par hypothèse \( P(\alpha_p)=S(\alpha_p)R(\alpha_p)=0\). L'anneau \( \eA\) étant intègre, \( S(\alpha_p)\neq 0\) parce que \( \alpha_i\neq \alpha_p\) pour \( i\neq p\). Par conséquent, \( R(\alpha_p)=0\).
    
    Nous devons encore vérifier que la multiplicité \( \alpha_p\) est \( k_p\) par rapport à \( R\). Pour cela nous utilisons le point \ref{ItemIeLhpciv} du lemme \ref{LemIeLhpc} afin de dire que le degré de \( \alpha_p\) pour \( P=SR\) est \( k_p\). Par conséquent
    \begin{equation}
        R=(X-\alpha_p)^{k_p}T
    \end{equation}
    avec \( T(\alpha_p)\neq 0\) et enfin
    \begin{equation}
        P=\prod_{i=1}^p(X-\alpha_i)T.
    \end{equation}
    De plus \( T(\alpha_i)\neq 0\), sinon \( R(\alpha_i)\) serait nul.
\end{proof}

\begin{corollary}[Conséquence du lemme de Gauss\cite{ooCDLEooEQGSvn}]       \label{CORooZCSOooHQVAOV}
    Soient \( A\) un anneau factoriel et \( \Frac(A)\) son corps des fractions. Un polynôme non constant \( P\in A[X]\) est irréductible (sur \( A\)) si et seulement s'il est irréductible et primitif au sens du pgcd\footnote{Définition \ref{DEFooAIYGooRAEfHU}.} sur \( \Frac(A)[X]\). 
\end{corollary}

%--------------------------------------------------------------------------------------------------------------------------- 
\subsection{Quelques identités}
%---------------------------------------------------------------------------------------------------------------------------

\begin{lemma}   \label{LemISPooHIKJBU}
    Quelques identités de polynômes.
    \begin{enumerate}
        \item   \label{ItemLTBooAcyMtN}
            Si \( n\) est impair, alors \( 1+X\) divise \( 1+X^n\).
        \item\label{ItemLTBooAcyMtNii}
            Pour tout \( n\) nous avons \( X^n-1=(X-1)(1+X+\cdots +X^{n-1})\).
        \item
            \( X^n-a^n=(X-a)\sum_{i=0}^{n-1}a^iX^{n-1-i}\).
    \end{enumerate}
\end{lemma}

\begin{proof}
  Nous démontrons uniquement le point \ref{ItemLTBooAcyMtNii}, puisque
  les autres ont été vus en début de chapitre\footnote{Voir l'égalité
    \eqref{Eqarpurmkbk}.}. Le cas \( n=1\) est évident. Procédons
  alors par récurrence en considérant un nombre entier impair \( n\) :
    \begin{subequations}
        \begin{align}
            1+X^{n+2}&=1+X^n+X^{n+2}-X^n\\
                    &=(1+X)P+X^n(X^2-1)\\
                    &=(1+X)P+X^n(X+1)(X-1)\\
                    &=(1+X)\big( P+X^n(X-1) \big).
        \end{align}
    \end{subequations}
\end{proof}
