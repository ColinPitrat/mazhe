% This is part of Le Frido
% Copyright (c) 2006-2018
%   Laurent Claessens, Carlotta Donadello
% See the file fdl-1.3.txt for copying conditions.


\begin{proposition}[Règles de calculs]  \label{PROPooBWZFooTxKavX}
    Soient $f$ et $g$ des fonctions
  différentiables en $g(a)$ et $a$ respectivement, alors la composée
  $f\circ g$ est différentiable en $a$ et
  \begin{equation*}
    d (f\circ g)_a = d f_{g(a)} \circ d g_a
  \end{equation*}
  et de plus les jacobiennes correspondantes vérifient
  \begin{equation*}
      J_{f\circ g}(a) = J_f\big( g(a) \big)J_g(a)
  \end{equation*}
  où le membre de droite est le produit (non-commutatif !) des deux matrices.
\end{proposition}

\begin{corollary}[Chain rule] Si $f : \eR^p \to \eR$ et $g : \eR \to
  \eR^p$, alors
  \begin{equation*}
    (f\circ g)^\prime(t) = \sum_{i=1}^p \pder f {x_i}(g(t)) g_i^\prime(t).
  \end{equation*}
\end{corollary}

\begin{remark}
    Quelque remarques à propos de la règle de dérivation en chaîne.
  \begin{enumerate}
  \item Si $p = 1$, on retrouve la règle usuelle de dérivation de
    fonctions composées.

  \item
      Si $g$ est à plusieurs variables, cette règle permet de déterminer les dérivées partielles de $f \circ g$, puisqu'une dérivée partielle peut être vue comme dérivée usuelle par rapport à une seule variable (voir \ref{deriveepartielles}).

  \item Si $f$ est à valeurs vectorielles, cette formule permet de
    retrouver la jacobienne de $f \circ g$ puisqu'il suffit de traiter
    chaque composante de $f$ séparément.
  \end{enumerate}
\end{remark}


%--------------------------------------------------------------------------------------------------------------------------- 
\subsection{Gradient}
%---------------------------------------------------------------------------------------------------------------------------

 \begin{definition}
	 Soit $f$ une fonction différentiable de $\eR^m$ dans $\eR$. On appelle \defe{gradient}{gradient} de $f$ la fonction $\nabla f : \eR^m\to \eR^m$\nomenclature{$\nabla f$}{gradient de la fonction $f$} de composantes
\[
\partial_{1}f,\ldots,\partial_{m}f.
\]
Soit $f$ une fonction de $\eR^m$ dans $\eR^n$, $f(a)=(f_1(a),\ldots,f_n(a))^T$. On appelle \defe{matrice jacobienne}{matrice!jacobienne} de $f$ la fonction $J(f) : \eR^m\to \eR^m\times\eR^n$ définie par
\begin{equation}
a\mapsto  \begin{pmatrix}
    \partial_{1}f_1(a) &\ldots&\partial_{m}f_1(a)\\
\vdots&\ddots&\vdots\\
\partial_{1}f_n (a)&\ldots&\partial_{m}f_n(a)\\
  \end{pmatrix}
\end{equation}
\end{definition}

%---------------------------------------------------------------------------------------------------------------------------
\subsection{Linéarité}
%---------------------------------------------------------------------------------------------------------------------------

La proposition suivante signifie que différentiation est une opération linéaire sur l'ensemble des fonctions différentiables.
\begin{proposition}		\label{PropDiffLineaire}
  Soient $f$ et $g$ deux fonctions de $U\subset\eR^m$ dans $\eR^n$ différentiables au point $a\in U$, et soit $\lambda$ dans $\eR$. Alors les fonctions $f+g$ et $\lambda f$ sont différentiables au point $a$ et on a
  \begin{equation}
    \begin{aligned}
 &     d(f+g)(a)=df(a)+dg(a), \\
& d(\lambda f)(a)=\lambda df(a),
    \end{aligned}
\end{equation}
\end{proposition}
\begin{proof}
  \begin{equation}
    \begin{aligned}
     & \lim_{h\to 0_m}\frac{\left\|\left(f(a+h)+g(a+h)\right)-\left(f(a)+g(a)\right)-df(a).h-dg(a).h\right\|_n}{\|h\|_m}\leq\\
&\lim_{h\to 0_m}\frac{\|f(a+h)-f(a)-df(a).h\|_n}{\|h\|_m}+\lim_{h\to 0_m}\frac{\|g(a+h)-g(a)-dg(a).h\|_n}{\|h\|_m}=0.
    \end{aligned}
  \end{equation}
  De même on démontre la  propriété $d(\lambda f)(a)=\lambda df(a)$.
\end{proof}

%---------------------------------------------------------------------------------------------------------------------------
\subsection{Produit}
%---------------------------------------------------------------------------------------------------------------------------

Soient $f$ et $g$ deux fonctions de $\eR^m$ dans $\eR^n$. Nous notons $f\cdot g$ la fonction de $\eR^n$ dans $\eR$ donnée par le produit scalaire point par point, c'est à dire
\begin{equation}
	(f\cdot g)(x)=f(x)\cdot g(x)
\end{equation}
pour tout $x\in\eR^m$. Le point dans le membre de droite est le produit scalaire dans $\eR^n$. Le cas particulier $n=1$ revient au produit usuel de fonctions :
\begin{equation}
	(fg)(x)=f(x)g(x).
\end{equation}

\begin{lemma}		\label{LemDiffProsuid}
	Si $f$ et $g$ sont des fonctions différentiables sur $\eR^m$ à valeurs dans $\eR$, alors la fonction produit $fg$ est également différentiable et
	\begin{equation}		\label{EqDifffgProd}
		d(fg)(a)=df(a)g(a)+f(a)dg(a)
	\end{equation}
	au sens où pour chaque $u$ dans $\eR^m$,
	\begin{equation}
		d(fg)(a).u=g(a)df(a).u+f(a)dg(a).u.
	\end{equation}
\end{lemma}
Remarquons qu'ici, $f(a)$ et $g(a)$ sont des réels, donc nous pouvons écrire $f(a)dg(a)$ aussi bien que $dg(a)f(a)$ sans ambigüités.

\begin{proof}
	Ce que nous devons faire pour vérifier la formule~\ref{EqDifffgProd}, c'est de vérifier le critère \eqref{EqCritereDefDiff} en remplaçant $f$ par $fg$ et $T(h)$ par $g(a)df(a).h+f(a)dg(a).h$.

	Ce que nous avons au numérateur est
	\begin{equation}
		\begin{aligned}[]
			\clubsuit&=(fg)(a+h)-(fg)(a)-g(a)df(a).h-f(a)dg(a).h\\
				&=f(a+h)g(a+h)-f(a)g(a)-g(a)df(a).h-f(a)dg(a).h.
		\end{aligned}
	\end{equation}
	Maintenant, nous allons faire apparaître $\big( f(a+h)-f(a)-df(a) \big)g(a+h)$ en ajoutant et soustrayant ce qu'il faut pour conserver $\clubsuit$ :
	\begin{equation}
		\begin{aligned}[]
			\clubsuit&=\big( f(a+h)-f(a)-df(a).h \big)g(a+h)\\
					&\quad +f(a)g(a+h)+g(a+h)df(a).h\\
					&\quad -f(a)g(a)-g(a)df(a).h-f(a)dg(a).h.
		\end{aligned}
	\end{equation}
	Nous mettons maintenant $f(a)$ et $fd(a).h$ en évidence là où c'est possible :
	\begin{equation}
		\begin{aligned}[]
			\clubsuit&=\big( f(a+h)-f(a)-df(a).h \big)g(a+h)\\
				&\quad+f(a)\big( g(a+h)-g(a)-dg(a).h \big)\\
				&\quad+\big( g(a+h)-g(a) \big)df(a).h.
		\end{aligned}
	\end{equation}
    Nous devons maintenant considérer la limite
	\begin{equation}
		\lim_{h\to 0}\frac{ \| \clubsuit \| }{ \| h \| }.
	\end{equation}
    Étant donné que $f$ et $g$ sont différentiables, les deux premiers termes sont nuls :
    \begin{equation}
        \begin{aligned}[]
            \lim_{h\to 0}\frac{ \big( f(a+h)-f(a)-df(a).h \big)}{\| h \|}g(a+h)=0\\
            \lim_{h\to 0} f(a)\frac{ \big( g(a+h)-g(a)-dg(a).h \big)}{\| h \|}=0.
        \end{aligned}
    \end{equation}
    En ce qui concerne le troisième terme, en utilisant la norme d'une application linéaire, nous avons
	\begin{equation}
		\lim_{h\to 0} \frac{ \| df(a).h \| }{ \| h \| }\leq\sup_{h\in\eR^m}\frac{ \| df(a).h \| }{ \| h \| }=\| df(a) \|,
	\end{equation}
    et par conséquent
    \begin{equation}
        \begin{aligned}[]
            0&\leq\lim_{h\to 0} \| g(a+h)-g(a) \|\frac{ \| df(a).h \|\| h \| }{ \| h \| }\\
            &\leq \lim_{h\to 0} \| g(a+h)-g(a) \|\| df(a) \|=0
        \end{aligned}
    \end{equation}
    parce que $g$ est continue (la limite du premier facteur est nulle tandis que la norme de $df(a)$ est un nombre constant). Nous avons donc bien prouvé que la formule \eqref{EqDifffgProd} est la différentielle de $fg$ au point $a$.
\end{proof}

Ce résultat se généralise pour des fonctions $f$ et $g$ de $\eR^m$ dans $\eR^n$ dans la proposition suivante qui généralise tout en même temps la proposition \ref{PROPooFKKHooQZGXhE}.
\begin{proposition}
	Soient $f$ et $g$ deux fonctions de $U\subset\eR^m$ dans $\eR^n$ différentiables au point $a\in U$. Alors la fonction $f\cdot g$ est différentiable  au point $a$ et on a
	\begin{equation}
        d(f\cdot g)(a)=g(a)\cdot df(a)+f(a)\cdot dg(a)
	\end{equation}
	au sens où
	\begin{equation}		\label{Eqdfcdotgexpl}
		d(f\cdot g)_a(u)=g(a)\cdot\big( df_a(u) \big)+f(a)\cdot\big( dg_a(u) \big)
	\end{equation}
	pour tout $u\in\eR^m$.
\end{proposition}

\begin{proof}
	La preuve du cas $n=1$ est déjà faite; c'est la formule \eqref{EqDifffgProd}. Pour le cas général $n\geq 2$, nous passons au composantes en nous rappelant que
	\begin{equation}
		(f\cdot g)(a)=\sum_{i=1}^nf_i(a)g_i(a)=\sum_{i=1}^n(f_ig_i)(a).
	\end{equation}
	En utilisant la linéarité de la différentiation, nous nous réduisons donc au cas des produits $f_ig_i$ qui sont des fonctions de $\eR^m$ dans $\eR$ :
	\begin{equation}
		\begin{aligned}[]
			d(f\cdot g)(a)&=d\left( \sum_{i=1}^n f_ig_i \right)(a)\\
			&=\sum_{i=1}^n\big( df_i(a)g_i(a)+f_i(a)dg_i(a) \big)\\
			&=g(a)\cdot df(a)+f(a)\cdot dg(a).
		\end{aligned}
	\end{equation}
	Ceci termine la preuve.
\end{proof}

Nous avons aussi une formule importante pour la différentielle des formes bilinéaires.
  \begin{lemma}\label{bilin_diff}
    Toute application bilinéaire
    \begin{equation}
	    \begin{aligned}
		    B\colon \eR^m\times\eR^n&\to \eR^p \\
		    B(a_1,a_2)&=a_1 \star a_2
	    \end{aligned}
    \end{equation}
    est différentiable en tout point $(a_1,a_2)$ de $\eR^m\times\eR^n$, et on a
\[
dB(a_1,a_2).(h_1,h_2)=h_1\star a_2 + a_1\star h_2.
\]
  \end{lemma}
  \begin{proof}
    \begin{equation}
      \begin{aligned}
  & \frac{\|B(a_1+h_1,a_2+h_2)-B(a_1,a_2)-(h_1\star a_2 + a_1\star h_2)\|_p}{\|(h_1,h_2)\|_{\eR^m\times\eR^n}} = \\
&= \frac{\|(a_1+h_1)\star(a_2+h_2)-a_1\star a_2-(h_1\star a_2 + a_1\star h_2)\|_p}{\|(h_1,h_2)\|_{\eR^m\times\eR^n}}=\spadesuit
 \end{aligned}
    \end{equation}
on rajoute et on enlève la quantité $(a_1+h_1)\star a_2$ dans le numérateur, et on obtient
   \begin{equation}
      \begin{aligned}
%&= \frac{\|(a_1+h_1)\star(a_2+h_2)-(a_1+h_1)\star a_2 +(a_1+h_1)\star a_2- a_1\star a_2-}{\|(h_1,h_2)\|_{\eR^m\times\eR^n}}\\
%&\hspace{7cm}\frac{-(h_1\star a_2 + a_1\star h_2)\|_p}{\quad}=\\
&\spadesuit= \frac{\|(a_1+h_1)\star h_2+h_1\star a_2-(h_1\star a_2 + a_1\star h_2)\|_p}{\|(h_1,h_2)\|_{\eR^m\times\eR^n}}=\\
&= \frac{\|h_1\star h_2\|_p}{\|(h_1,h_2)\|_{\eR^m\times\eR^n}}\leq C\frac{\|h_1\|_m\|h_2\|_n}{\|(h_1,h_2)\|_{\eR^m\times\eR^n}}\leq\\
&\leq C\frac{\|(h_1,h_2)\|^2_{\eR^m\times\eR^n}}{\|(h_1,h_2)\|_{\eR^m\times\eR^n}}= C\|(h_1,h_2)\|_{\eR^m\times\eR^n}.
      \end{aligned}
    \end{equation}
Si on prend la limite de cette expression pour $(h_1,h_2)\to (0_m,0_n)$ on obtient $0$, donc la preuve est complète. À noter, que dans l'avant-dernier passage on a utilisé la continuité des applications linéaires $\pr_m:\eR^m\times\eR^n\to \eR^m$ et $\pr_n: \eR^m\times\eR^n\to \eR^n$ qui à chaque point $(a_1,a_2)$ de $\eR^m\times\eR^n$ associent $a_1$ et $a_2$ respectivement.
\end{proof}

\begin{proposition}     \label{PropEKLTooSvZjdW}
    Soit \( V\) et \( W\) deux espaces vectoriels et \( \varphi\colon V\to W\) un isomorphisme. Soit \( f\colon \eC\to V\) une application telle que \(\varphi\circ f\colon \eC\to W\) soit différentiable.

    Alors \( f\) est différentiable et \( df=\varphi^{-1}\circ d(\varphi\circ f)\).
\end{proposition}

\begin{proof}
    Si \( T\) est la différentielle de \( \varphi\circ f\) au point \( z\) nous avons
    \begin{equation}
        \lim_{\substack{h\to 0\\h\in \eC}}\frac{ (\varphi\circ f)(z+h)-(\varphi\circ f)(z)+T(h) }{ h }=0.
    \end{equation}
    En appliquant \( \varphi\) aux deux membres, et en permutant avec la limite (parce que \( \varphi\) est continue),
    \begin{equation}
        \varphi\lim_{h\to 0} \frac{ f(z+h)-f(z)+\varphi^{-1} T(h) }{ h }=0,
    \end{equation}
    ce qui signifie que \( f\) est différentiable et que \( df=\varphi^{-1}\circ T=\varphi^{-1}\circ d(\varphi\circ f)\).
\end{proof}

%---------------------------------------------------------------------------------------------------------------------------
\subsection{Différentielle de fonction composée}
%---------------------------------------------------------------------------------------------------------------------------

La plus importante entre les règles de différentiation est la règle de différentiation d'une fonction composée (\emph{chain rule} dans les livres anglais et américains). Cette règle généralise la règle de dérivation pour fonctions de $\eR$ dans $\eR$. Il est utile d'introduire d'abord une formulation équivalente de la définition de différentielle
\begin{lemma}\label{Def_diff2}
  Soit $U$ un ouvert de $\eR^m$. La fonction $f: U\to\eR^n$ est différentiable au point $a$ dans $U$, si et seulement s'il existe une fonction $\sigma_f: U\times U\to \eR^n$ telle que
  \begin{subequations}		\label{SubEqsDiff2}
	  \begin{align}
  		\sigma_f(a,a)&=\lim_{x\to a} \sigma_f(a,x)=0\\
		 f(x)&=f(a)+T(x-a)+\sigma_f(a,x)\|x-a\|_m,   \label{def_diff2}
	  \end{align}
  \end{subequations}
pour une certaine application linéaire $T\in\mathcal{L}(\eR^m,\eR^n)$.
\end{lemma}
\begin{proof}
	Si les conditions \eqref{SubEqsDiff2} sont satisfaites alors $T$ est la différentielle de $f$ en $a$. En effet, dans ce cas nous avons
	\begin{equation}
		f(a+h)=f(a)+T(h)+\sigma_f(a,a+h)\| h \|,
	\end{equation}
	et la condition \eqref{EqCritereDefDiff} devient
	\begin{equation}
		\lim_{h\to 0} \frac{ \| \sigma_f(a,a+h) \|\| h \| }{ \| h \| }=\lim_{h\to 0} \| \sigma_f(a,a+h)\| =0
	\end{equation}


Si $f$ est différentiable au point $a$ il suffit de prendre $T=df(a)$ et
\[
\sigma_f(a,x)=\frac{f(x)-f(a)-df(a).(x-a)}{\|x-a\|_m}.
\]
\end{proof}

\begin{remark}
	La fonction $\sigma_f(a,x)\| x-a \|_m$ est ce qui avait été appelle $\epsilon(h)$ sur la figure~\ref{LabelFigDifferentielle}.
\end{remark}

\begin{proposition}		\label{PropDiffCompose}
Soient $U$ un ouvert de $\eR^m$ et $V$ un ouvert de $\eR^n$. Soient $f: U\to V$  et $g: V \to \eR^p$ deux fonctions différentiables respectivement au point $a$ dans $U$ et $b=f(a)$ dans $V$. Alors la fonction composée $g\circ f: U\to \eR^p $ est différentiable au point $a$ et
\begin{equation}	\label{EqDiffCompose}
    d(g\circ f)_a=dg_{f(a)}\circ df_a.
\end{equation}
\end{proposition}

\begin{proof}
 En tenant compte du lemme~\ref{Def_diff2} on peut écrire
 \begin{subequations}
	 \begin{align}
		f(a+h)-f(a)&=df_a(h)+\sigma_f(a,a+h)\|h\|_m,	&&\forall h\in U-a,\\
		g(b+k)-g(b)&=dg_b(k)+\sigma_g(b,b+k)\|k\|_n,	&&\forall k\in V-b.
	 \end{align}
 \end{subequations}
On sait que $f(a)=b$ et que $f(a+h)$ est  un élément de $V$ et $f(a+h)=f(a)+k$ pour $k=df(a).h+\sigma_f(a,a+h)\|h\|_m$.  Par substitution dans la deuxième équation on obtient
\begin{equation}
	\begin{aligned}
		g\big(f(a+h)\big)& - g\big(f(a)\big)\\
        &=dg_{f(a)}\Big(df_a(h)+\sigma_f(a,a+h)\|h\|_m\Big)\\
		&\quad+\sigma_g\left(f(a), f(a+h)\right)\left\| df_a(h)+\sigma_f(a,a+h)\|h\|_m\right \|_n\\
		&=g\circ f (a+h) - g\circ f (a)\\
        &= dg_{f(a)}\circ df_a(h) \\
        &\quad +\|h\|_m\Big[ dg_{f(a)}\sigma_f(a,a+h)\\
		&\quad+\sigma_g\left(f(a), f(a+h)\right)\big\| df_a\frac{h}{\|h\|_m}+\sigma_f(a,a+h)\big \|_n\Big],
	\end{aligned}
\end{equation}
donc
\begin{equation}
	(g\circ f) (a+h) - (g\circ f) (a) = dg_f(a)\circ df_a(h) + S(a,a+h) \|h\|_m
\end{equation}
où $S$ représente le contenu du dernier grand crochet. Il ne reste plus qu'à prouver que $S(a,a+h)$ est $o(\|h\|_m)$. En tenant compte du fait que $\sigma_f(a,a+h)$ et $\sigma_g\left(f(a), f(a+h)\right)$ sont $o (\|h\|_m)$,
\begin{equation}
  \begin{aligned}
      & \lim_{h\to 0_m} \frac{S(a,a+h)}{\|h\|_m}= \lim_{h\to 0_m}\frac{dg_{f(a)}\sigma_f(a,a+h)}{\|h\|_m}+ \\
& + \lim_{h\to 0_m}\frac{\sigma_g\left(f(a), f(a+h)\right)\left\| df_a\frac{h}{\|h\|_m}+\sigma_f(a,a+h)\right \|_n}{\|h\|_m} = 0.
  \end{aligned}
\end{equation}
\end{proof}

\begin{remark}
    Note : la formule \eqref{EqDiffCompose} est à comprendre de la façon suivante. Si $u\in\eR^m$, alors
    \begin{equation}
        d(g\circ f)_a(u)=\underbrace{dg_{f(a)}}_{\in\aL(\eR^n,\eR^p)}\Big( \underbrace{df_a(u)}_{\in\eR^n} \Big)\in\eR^p.
    \end{equation}
\end{remark}

Le lemme suivant sert à prouver les théorèmes \ref{ThoLDpRmXQ} et \ref{ThoUJMhFwU}. Il est fondamentalement la raison de la formule définissant l'intégrale d'une forme sur un chemin (définition \ref{DEFooRMHGooFtMEPB}).
\begin{lemma}[\cite{MonCerveau}]        \label{LEMooKNBVooQSowos}
    Soient un espace vectoriel normé \( F\) de dimension finie, ainsi que \( E\), un espace vectoriel normé. Nous considérons un chemin de classe \( C^1\)
    \begin{equation}
        \gamma\colon \mathopen[ a , b \mathclose]\to E
    \end{equation}
    et une application de classe \( C^1\)
    \begin{equation}
        f\colon E\to F.
    \end{equation}
    Si \( g=f\circ\gamma\), alors
    \begin{equation}
        g'(t)=(df)_{\gamma(t)}\big( \gamma'(t) \big)
    \end{equation}
    pour tout \( t\in \mathopen[ a , b \mathclose]\).
\end{lemma}
 
\begin{proof}
    Nous écrivons la dérivée de \( g\) de la façon suivante : 
    \begin{subequations}        \label{SUBEQSooWKRYooGyVgNl}
        \begin{align}
            g'(t)&=\Dsdd{ g(t+s) }{s}{0}\\
            &=\Dsdd{ (f\circ \gamma)(t+s) }{s}{0}\\
            &=d(f\circ \gamma)_t(1) \label{SUBEQooHUZVooWECnVZ}\\
            &=(df)_{\gamma(t)}\big( d\gamma_t(1) \big) \label{SUBEQooOAYLooIzltaY}\\
        \end{align}
    \end{subequations}
    Justifications :
    \begin{itemize}
        \item Pour \eqref{SUBEQooHUZVooWECnVZ} : les formules \eqref{LemdfaSurLesPartielles}. Notez que le \( 1\) à qui s'applique la différentielle de \( f\circ\gamma\) est le vecteur de \( \eR\) qui est multiplié par \( s\) dans l'expression \( (f\circ\gamma)(t+s)\).
        \item
            Pour \eqref{SUBEQooOAYLooIzltaY} : la différentiation de fonction composées de la proposition \ref{PropDiffCompose}.
    \end{itemize}
    Mais
    \begin{equation}
        d\gamma_t(1)=\Dsdd{ \gamma(t+s) }{s}{0}=\gamma'(t).
    \end{equation}
    En remettant au bout de \eqref{SUBEQSooWKRYooGyVgNl}, nous obtenons le résultat.
\end{proof}

%---------------------------------------------------------------------------------------------------------------------------
  \subsection{Différentielle et dérivées partielles}
%---------------------------------------------------------------------------------------------------------------------------

\begin{proposition}		\label{Diff_totale}
 Soit $U$ un ouvert dans $\eR^m$ et $a$ un point dans $U$. Soit $f$ une application de $U$ dans $\eR^n$. Si toutes les dérivées partielles de $f$ existent sur \( U\) et sont continues au point $a$ alors $f$ est différentiable au point $a$.
\end{proposition}
\begin{proof}
 On se limite au cas $m=2$.  Pour rendre les calculs plus simples on utilise ici la norme $\|\cdot\|_\infty$ dans l'espace $\eR^2$, mais comme on a vu plus en haut, cela ne peut pas avoir des conséquences sur la différentiabilité de $f$. Si la différentielle de $f$ au point $a$ existe alors elle est définie par la formule
\[
    df_a(v)=\frac{ \partial f }{ \partial x }(a)v_1+\frac{ \partial f }{ \partial y }(a)v_2
\]
pour tout $v$ dans $\eR^m$.

On commence par prouver le résultat en supposant que les dérivées partielles de $f$ au point $a$ sont nulles. La différentiabilité de $f$ signifie que pour toute constante  $\varepsilon> 0$ il y a une constante $\delta>0$ telle que si $\|v\|_\infty\leq \delta $ alors
\[
\frac{\|f(a_1+v_1, a_2+v_2)-f(a_1, a_2)\|_n}{\|v\|_\infty}\leq \varepsilon.
\]
On écrit alors
\begin{equation}
  \begin{aligned}
   & \|f(a_1+v_1, a_2+v_2)-f(a_1, a_2)\|_n=\\
&=\|f(a_1+v_1, a_2+v_2)-f(a_1+v_1, a_2)+f(a_1+v_1, a_2)-f(a_1, a_2)\|_n\leq\\
&\leq \|f(a_1+v_1, a_2+v_2)-f(a_1+v_1, a_2)\|_n+\|f(a_1+v_1, a_2)-f(a_1, a_2)\|_n.
  \end{aligned}
\end{equation}
Comme la dérivée partielle $\partial_x f$ est  nulle au point $a$  on sait que  pour toute constante  $\varepsilon> 0$ il y a une constante $\delta_1>0$ telle que si $|v_1|\leq \delta_1 $ alors
\[
\|f(a_1+v_1, a_2)-f(a_1, a_2)\|_n\leq \varepsilon |v_1|.
\]
Pour l'autre terme on a, par la proposition~\ref{val_medio_1},
\begin{equation}
  \begin{aligned}
   & \|f(a_1+v_1, a_2+v_2)-f(a_1+v_1, a_2)\|_n\leq \\
&\leq \sup\{\|\partial_yf(x)\|_n\,\vert\, x\in S\}|v_2|.
  \end{aligned}
\end{equation}
où $S$ est le segment d'extrémités  $(a_1+v_1, a_2)$ et $ (a_1+v_1, a_2+v_2)$. Comme la  dérivée partielle $\partial_y f$ est continue et nulle au point $a$ on sait que  pour toute constante  $\varepsilon> 0$ il existe une constante $\delta_2>0$ telle que si $\|(u_1,u_2)\|_\infty\leq \delta_2 $ alors $\|\partial_yf(a_1+u_1,a_2+u_2)\|_n\leq \varepsilon$. Si on choisit $\delta = \min\{\delta_1,\,\delta_2\}$ le segment $S$ est contenu dans la boule de rayon $\delta$ centrée au point $a$ et on obtient
\[
 \|f(a_1+v_1, a_2+v_2)-f(a_1, a_2)\|_n\leq \varepsilon |v_1|+\varepsilon |v_2|\leq 2\varepsilon \|v\|_\infty.
\]
Cela prouve que \( f\) est différentiable en \( (a_1,a_2)\) et que la différentielle est nulle :
\begin{equation}
    df_{(a_1,a_2)}=0.
\end{equation}

Dans le cas général, où les dérivées partielles de $f$ au point $a$ ne sont pas spécialement nulles, on peut considérer la fonction\footnote{Vous verrez dans la discussion à propos de la fonction \eqref{EqCJVooJOuXdN} pourquoi cette fonction ne fonctionne pas dans le cas de la dimension infinie.}
\begin{equation}    \label{EqXHVooJeQKrB}
    g(x,y)=f(x,y)-\partial_1 f(a)x-\partial_2 f(a)y,
\end{equation}
qui a dérivées partielles nulles au point $a$. La fonction $g$ est donc différentiables. La fonction $f$ est maintenant la somme de $g$ et de la fonction linéaire et continue $(x,y)\mapsto \partial_1 f(a)x-\partial_2 f(a)y$. On verra dans la prochaine section que la somme de deux fonctions différentiables est une fonction différentiable. Par conséquent, la fonction $f$ est différentiable.
\end{proof}

\begin{remark}
    En dimension infinie, il n'est pas vrai que l'existence et la continuité de toutes les dérivées partielles en un point implique la différentiabilité en ce point. Pour donner un exemple, nous allons continuer l'exemple~\ref{ExHKsIelG}
    avec la fonction~\ref{EqCJVooJOuXdN} sur un espace de Hilbert.

    En dimension infinie nous aurons le théorème~\ref{ThoOYwdeVt} qui donnera quelque chose de moins fort.
\end{remark}

Étant donné que pour tout vecteur $u$ dans $\eR^m$ on a $\partial_uf(a)=\nabla f(a)\cdot u$, le gradient de $f$ nous donne la direction dans laquelle la croissance de $f$ est maximale. Soit $C$ une colline et soit $f$ la fonction que a chaque point $(x,y)$ de la Terre associe son altitude. Si nous voulons monter la colline le plus vite possible nous n'avons qu'a suivre la direction $\nabla f$ à chaque point. Elle est la projection sur le plan $x$-$y$ de la direction de pente maximale. Au contraire, la direction $-\nabla f$ est la direction de croissance minimale.

La matrice jacobienne calculé au point $a$ est la matrice associée canoniquement à l'application linéaire $df_a:\eR^m\to\eR^n$.

%---------------------------------------------------------------------------------------------------------------------------
\subsection{Plan tangent}
%---------------------------------------------------------------------------------------------------------------------------

On a dit au début de cette section que si $f$ est une fonction de $\eR^2$ dans $\eR$ alors le graphe de $f$ est une surface à deux paramètres et que l'application affine tangente au graphe de $f$ au point $(a, f(a))$ est un plan. Maintenant on sait que ce plan est celui d'équation
\begin{equation}
	T_a(x,y)=f(a_1,a_2)+\frac{ \partial f }{ \partial x }(a_1,a_2)(x-a_1)+\frac{ \partial f }{ \partial y }(a_1,a_2)(y-a_2).
\end{equation}
Le plan tangent au graphe de $f$ au point $a$ est le graphe de cette fonction $T_a$.

\begin{remark}
	Il existe cependant des fonctions différentiables dont les dérivées partielles ne sont pas continues. La construction d'un tel exemple est cependant délicate, et nous le ferons pas ici. Retenez cependant que si dans un exercice vous obtenez que les dérivées partielles ne sont pas continues, vous ne pouvez pas immédiatement en conclure que la fonction ne sera pas différentiable.
\end{remark}

%---------------------------------------------------------------------------------------------------------------------------
\subsection{Calcul de différentielles}
%---------------------------------------------------------------------------------------------------------------------------

\begin{normaltext} \label{deriveepartielles}
En pratique, ayant une formule pour la fonction $f$, nous la dérivons par rapport à la variable $x_i$ en utilisant les règles usuelle de dérivation en considérant que les autres ($x_j$ avec $j \neq i$) sont des constantes.
\end{normaltext}

\begin{example}Pour $f(x,y) = xy + x^2$, les dérivées partielles
  s'écrivent
  \begin{equation*}
    \frac{\partial f}{\partial x} = y + 2x \quad\text{et}\quad \frac{\partial f}{\partial y} = x
  \end{equation*}
\end{example}


Des \emph{règles de calcul} sont d'application. En particulier, quand
ces opérations existent, les sommes, différences, produits, quotients
et compositions d'applications différentiables sont différentiables.

Toute application linéaire est différentiable, et sa différentielle en
tout point est égale à l'application elle-même\footnote{Lemme \ref{LEMooZSNMooCfjzOB}.}. En particulier, les
\Defn{projections canoniques}, c'est-à-dire les applications du type
$(x,y,z) \mapsto y$, sont linéaires donc différentiables.

\begin{example}
Les cas suivants sont faciles :
  \begin{enumerate}
  \item En combinant les projections canoniques avec les règles de
    calculs, on obtient que toute fonction polynomiale à $n$ variables
    est différentiable comme application de $\eR^n$ dans $\eR$.

  \item Toute fonction rationnelle, du type $f(x) \pardef
    \frac{P(x)}{Q(x)}$ où $P$ et $Q$ sont des polynômes, est
    différentiable en tout point $a$ tel que $Q(a) \neq 0$.

  \item Pour une fonction d'une variable $f : D \subset \eR \to
    \eR$, le caractère différentiable et le caractère dérivable
    coïncident. De plus, on a
    \begin{equation*}
      d f_a(u) = f'(a) u.
    \end{equation*}
  \end{enumerate}
\end{example}

%---------------------------------------------------------------------------------------------------------------------------
                    \subsection{Notes idéologiques quant au concept de plan tangent}
%---------------------------------------------------------------------------------------------------------------------------
\label{ssecConceptPlanTag}

Notons $G$, le graphe d'une fonction $f$, c'est à dire
\begin{equation}
    G=\{ (x,y,z)\in\eR^3\tq z=f(x,y) \}.
\end{equation}
Première affirmation : si $\gamma\colon \eR\to G$ est une courbe telle que $\gamma(0)=\big( a,f(a) \big)$, alors $\gamma'(0)\in\eR^n$ est dans le plan tangent à $G$ au point $\big( a,f(a) \big)$.

Plus fort : tous les éléments du plan tangent sont de cette forme.

Le plan tangent à $G$ en un point $x\in G$ est donc constitué des vecteurs vitesse de tous les chemins qui passent par $x$.

Prenons maintenant $S$, une courbe de niveau de $G$, c'est à dire
\begin{equation}
    S=\{ (x,y)\in\eR^2\tq f(x,y)=C \}.
\end{equation}
Si nous prenons un chemin dans $G$ qui est, de plus, contraint à $S$, c'est à dire tel que $\gamma(t)\in S$, alors $\gamma'(0)$ sera tangent à $G$ (ça, on le savait déjà), mais en plus, $\gamma'(0)$ sera tangent à $S$, ce qui est logique.

La morale est que si vous prenez un chemin qui se ballade dans n'importe quoi, alors la dérivée du chemin sera un vecteur tangent à ce n'importe quoi.

En outre, si $\gamma(t)\in S$ et $\gamma(0)=a$, alors
\begin{equation}
    \scal{\nabla f(a)}{\gamma'(0)}=0,
\end{equation}
c'est à dire que le vecteur tangent à la courbe de niveau est perpendiculaire au gradient. Cela est intuitivement logique parce que la tangente à la courbe de niveau correspond à la direction de \emph{moins} grande pente.

%---------------------------------------------------------------------------------------------------------------------------
                    \subsection{Gradient et recherche du plan tangent}
%---------------------------------------------------------------------------------------------------------------------------

Nous avons maintenant en main les concepts utiles pour trouver l'équation du plan tangent à une surface.

De la même manière que la tangente à une courbe était la droite de coefficient directeur donné par la dérivée, maintenant, le plan tangent à une surface est le plan dont les vecteurs directeurs sont les dérivées partielles :

La généralisation de l'équation \eqref{EqDiffRapTgDer} est
\begin{equation}        \label{EqDefPlanTag}
    T_a(x)=f(a)+\sum_i\frac{ \partial f }{ \partial x_i }(a)(x-a)^i
\end{equation}

Nous introduisons aussi souvent l'opérateur différentiel abstrait \defe{nabla}{nabla}, noté $\nabla$ et qui est donné par le vecteur
\begin{equation}
    \nabla=\left( \frac{ \partial  }{ \partial x_1 },\ldots,\frac{ \partial  }{ \partial x_n } \right).
\end{equation}
Les égalités suivantes sont juste des notations, sommes toutes logiques, liées à $\nabla$ :
\begin{equation}
    \nabla f=\left( \frac{ \partial f }{ \partial x_1 },\ldots,\frac{ \partial f }{ \partial x_n } \right),
\end{equation}
et
\begin{equation}        \label{EqDefGradient}
    \nabla f(a) = \left(\frac{\partial f}{\partial x_1}(a), \frac{\partial f}{\partial x_2}(a), \ldots, \frac{\partial f}{\partial x_n}(a)\right).
\end{equation}
Ce dernier est un élément de $\eR^n$ : chaque entrée est un nombre réel.

\begin{definition}
Le vecteur gradient de $f$ au point $a$ est le vecteur donné par la formule \eqref{EqDefGradient}.
\end{definition}
La notation $\nabla$ permet d'écrire la différentielle sous forme un peu plus compacte. En effet, la formule \eqref{EqDiffPartRap} peut être notée
\begin{equation}
    df_a(u)=\scal{\nabla f(a)}{u}.
\end{equation}

En utilisant ce produit scalaire, l'équation \eqref{EqDefPlanTag} peut se récrire
\begin{equation}
    T_a(x)=f(a)+\sum_i\frac{ \partial f }{ \partial x_i }(a)(x-a)^i=f(a)+\scal{\nabla f(a)}{x-a}.
\end{equation}

Afin d'éviter les confusions, il est parfois souhaitable de bien mettre les parenthèses et noter $(\nabla f)(a)$ au lieu de $\nabla f(a)$.

\begin{proposition}
$\nabla f(a)\,\bot \,S_a$
\end{proposition}


\begin{equation}        \label{EqPlanTgSansNabla}
    z=f(a)+\sum_i\frac{ \partial f }{ \partial f }(a)(x-a)^i.
\end{equation}

\subsubsection*{Cas particulier où $n=2$:}
Le plan $T_a$ avec $a=(a_1,a_2)$ a pour équation dans $\eR^3$:
\begin{equation}        \label{EqPlanTgEnDimDeux}
    z = f(a_1,a_2) + \frac{\partial f}{\partial x}(a_1,a_2)\,(x-a_1)+ \frac{\partial f}{\partial y}(a_1,a_2)\,(y-a_2).
\end{equation}

\begin{definition}
  Soit $f : \eR^n \to\eR$ une fonction différentiable en un point
  $a$. Le \emph{plan tangent} au graphe de $f$ en $(a,f(a))$ est
  l'ensemble des points
  \begin{equation*}
    \begin{split}
      T_af &= \{ (x,z) \in \eR^n \times \eR \tq z = f(a) + d f_a (x-a)\}\\
      &= \{ (x,z) \in \eR^n \times \eR \tq z = f(a) + \scalprod{\nabla f(a)}{x-a}\}
    \end{split}
  \end{equation*}
\end{definition}

Nous avons vu que, de la même façon qu'en deux dimensions nous avions l'approximation \eqref{Eqfxsimesfa} d'une fonction par sa tangente, en trois dimensions nous avons l'approximation suivante d'une fonction de deux variables :
\begin{equation}
    f(x,y)\simeq f(a,b)+\frac{ \partial f }{ \partial x }(a,b)(x-a)+\frac{ \partial f }{ \partial y }(a,b)(y-b)
\end{equation}
lorsque $(x,y)$ n'est pas trop loin de $(a,b)$. Cela signifie que le graphe de $f$ ressemble au graphe de la fonction $T_{(a,b)}$ donnée par
\begin{equation}
    T_{(a,b)}(x,y)=f(a,b)+\frac{ \partial f }{ \partial x }(a,b)(x-a)+\frac{ \partial f }{ \partial y }(a,b)(x-a).
\end{equation}
En notations compactes :
\begin{equation}
    T_p(x)=f(p)+\nabla f(p)\cdot (x-p).
\end{equation}
Le graphe de la fonction $T_p$ sera le \defe{plan tangent}{plan!tangent} au graphe de $f$ au point $p$. L'équation du plan tangent sera donc
\begin{equation}
    z-f(p)=\nabla f(p)\cdot (x-p).
\end{equation}

\begin{remark}
    Lorsque nous utilisons la notation vectorielle, la lettre «$x$» désigne le vecteur $(x,y)$. Il faut être attentif. Dans un cas $x$ est un vecteur dans l'autre c'est une composante d'un vecteur.
\end{remark}

%+++++++++++++++++++++++++++++++++++++++++++++++++++++++++++++++++++++++++++++++++++++++++++++++++++++++++++++++++++++++++++
\section{Calcul différentiel dans un espace vectoriel normé}
%+++++++++++++++++++++++++++++++++++++++++++++++++++++++++++++++++++++++++++++++++++++++++++++++++++++++++++++++++++++++++++

%---------------------------------------------------------------------------------------------------------------------------
\subsection{Formule des accroissements finis}
%---------------------------------------------------------------------------------------------------------------------------

\begin{proposition} \label{PropDQLhSoy}
    Soient \( a<b\) dans \( \eR\) et deux fonctions
    \begin{subequations}
        \begin{align}
            f\colon \mathopen[ a , b \mathclose]\to E\\
            g\colon \mathopen[ a , b \mathclose]\to \eR
        \end{align}
    \end{subequations}
    continues sur \( \mathopen[ a , b \mathclose]\) et dérivables sur \( \mathopen] a , b \mathclose[\). Si pour tout \( t\in\mathopen] a , b \mathclose[\) nous avons \( \| f'(t) \|\leq g'(t)\) alors
        \begin{equation}
            \| f(b)-f(a) \|\leq g(b)-g(a).
        \end{equation}
\end{proposition}

\begin{proof}
    Soit \( \epsilon>0\) et la fonction
    \begin{equation}
        \begin{aligned}
            \varphi_{\epsilon}\colon \mathopen[ a , b \mathclose]&\to \eR \\
            t&\mapsto \| f(t)-f(a) \|-g(t)-\epsilon t.
        \end{aligned}
    \end{equation}
    Cela est une fonction continue réelle à variable réelle. En particulier pour tout \( u\in\mathopen] a , b \mathclose[\) la fonction \( \varphi_{\epsilon}\) est continue sur le compact \( \mathopen[ u , b \mathclose]\) et donc y atteint son minimum en un certain point \( c\in\mathopen[ u , b \mathclose]\); c'est le bon vieux théorème de Weierstrass~\ref{ThoWeirstrassRn}. Nous commençons par montrer que pour tout \( u\), ledit minimum ne peut être que \( b\). Pour cela nous allons montrer que si \( t\in\mathopen[ u , b [\), alors \( \varphi_{\epsilon}(s)<\varphi_{\epsilon}(t)\) pour un certain \( s>t\). Par continuité si \( s\) est proche de \( t\) nous avons
        \begin{equation}
            \left\|  \frac{ f(s)-f(t) }{ s-t }  \right\|-\frac{ \epsilon }{2}<\| f'(t) \|<g'(t)+\frac{ \epsilon }{2}=\frac{ g(s)-g(t) }{ s-t }+\frac{ \epsilon }{2}.
        \end{equation}
        Ces inégalités proviennent de la limite
        \begin{equation}
            \lim_{s\to t} \frac{ f(s)-f(t) }{ s-t }=f'(t),
        \end{equation}
        donc si \( s\) et \( t\) sont proches,
        \begin{equation}
            \left\| \frac{ f(s)-f(t) }{ s-t }-f'(t) \right\|
        \end{equation}
        est petit. Si \( s>t\) nous pouvons oublier des valeurs absolues et transformer l'inégalité en
        \begin{equation}
            \| f(s)-f(t) \|<g(s)-g(t)+\epsilon(s-t).
        \end{equation}
        Utilisant cela et l'inégalité triangulaire,
        \begin{subequations}
            \begin{align}
                \varphi_{\epsilon}(s)&\leq\| f(s)-f(t) \|+\| f(t)-f(a) \|-g(s)-\epsilon s\\
                &\leq g(s)-g(t)+\epsilon s-\epsilon t+\| f(t)-f(a) \|-g(s)-\epsilon s\\
                &=\varphi_{\epsilon}(t).
            \end{align}
        \end{subequations}
        Donc nous avons bien \( \varphi_{\epsilon}(s)<\varphi_{\epsilon}(t)\) avec l'inégalité stricte. Par conséquent pour tout \( u\in\mathopen] a , b \mathclose[\) nous avons \( \varphi_{\epsilon}(b)<\varphi_{\epsilon}(u)\) et en prenant la limite \( u\to a\) nous avons
        \begin{equation}
            \varphi_{\epsilon}(b)\leq \varphi_{\epsilon}(a).
        \end{equation}
        Cette inégalité donne immédiatement
        \begin{equation}
            \| f(b)-f(a) \|\leq g(b)-g(a)+\epsilon(b-a)
        \end{equation}
         pour tout \( \epsilon>0\) et donc
         \begin{equation}
            \| f(b)-f(a) \|\leq g(b)-g(a).
         \end{equation}
\end{proof}

\begin{theorem}[Théorème des accroissements finis]\label{ThoNAKKght}
    Soient \( E\) et \( F\) des espaces vectoriels normés, \( U \) ouvert dans \( E\) et une application différentiable \( f\colon U\to F\). Pour tout segment \( \mathopen[ a , b \mathclose]\subset U\) nous avons
    \begin{equation}
        \| f(b)-f(a) \|\leq\left( \sup_{x\in\mathopen[ a , b \mathclose]}\| df_x \| \right)\| b-a \|.
    \end{equation}
\end{theorem}
\index{théorème!accroissements finis}


\begin{proof}
    Nous prenons les applications
    \begin{equation}
        \begin{aligned}
            k\colon \mathopen[ 0 , 1 \mathclose]&\to E \\
            t&\mapsto f\big( (1-t)a+tb \big)
        \end{aligned}
    \end{equation}
    et
    \begin{equation}
        \begin{aligned}
            g\colon \mathopen[ 0 , 1 \mathclose]&\to \eR \\
            t&\mapsto t\sup_{x\in\mathopen[ a , b \mathclose]}\| df_x \|\| b-a \|.
        \end{aligned}
    \end{equation}
    Pour tout \( t\) nous avons \( g'(t)=M\| b-a \|\) où il n'est besoin de dire ce qu'est \( M\). D'un autre côté nous avons aussi
    \begin{equation}
        \begin{aligned}[]
            k'(t)&=\lim_{\epsilon\to 0}\frac{ f\big( (1-t-\epsilon)a+(t+\epsilon)b \big)-f\big( (1-t)a+tb \big) }{ \epsilon }\\
            &=\Dsdd{ f\big( (1-t)a+tb+\epsilon(b-a) \big)  }{\epsilon}{0}\\
            &=df_{(1-t)a+tb}(b-a)
        \end{aligned}
    \end{equation}
    où nous avons utilisé l'hypothèse de différentiabilité de \( f\) sur \( \mathopen[ a , b \mathclose]\) et donc en \( (1-t)a+tb\). Nous avons donc
    \begin{equation}
        \| k'(t) \|\leq \| b-a \|\| df_{(1-t)a+tb} \|\leq M\| b-a \|=g'(t)
    \end{equation}
    La proposition~\ref{PropDQLhSoy} est donc utilisable et
    \begin{equation}
        \| k(1)-k(0) \|=g(1)-g(0),
    \end{equation}
    c'est à dire
    \begin{equation}
        \| f(b)-f(a) \|=M\| b-a \|
    \end{equation}
    comme il se doit.
\end{proof}

\begin{proposition} \label{ProFSjmBAt}
    Soient \( E\) et \( F\) des espaces vectoriels normés, \( U \) ouvert dans \( E\) et une application \( f\colon U\to F\). Soient \( a,b\in U\) tels que \( \mathopen[ a , b \mathclose]\subset U\). Nous posons \( u=(b-a)/\| b-a \|\) et nous supposons que pour tout \( x\in\mathopen[ a , b \mathclose]\), la dérivée directionnelle
    \begin{equation}
        \frac{ \partial f }{ \partial u }(x)=\Dsdd{ f(x+tu) }{t}{0}
    \end{equation}
    existe. Nous supposons de plus que \( \frac{ \partial f }{ \partial u }(x)\) est continue en \( x=a\). Alors
    \begin{equation}
        \| f(b)-f(a) \|\leq\left( \sup_{x\in\mathopen[ a , b \mathclose]}\| \frac{ \partial f }{ \partial u }(x) \| \right)\| b-a \|.
    \end{equation}
\end{proposition}

\begin{proof}
    Nous posons évidemment
    \begin{equation}
        M=\sup_{x\in\mathopen[ a , b \mathclose]}\| \frac{ \partial f }{ \partial u }(x) \|
    \end{equation}
    et nous considérons les fonctions
    \begin{equation}
        k(t)=f\big( (1-t)a+tb \big)
    \end{equation}
    et
    \begin{equation}
        g(t)=tM\| b-a \|.
    \end{equation}
    Pour alléger les notations nous posons \( x=(1-t)a+tb\) et nous calculons avec un petit changement de variables dans la limite :
    \begin{equation}
        k'(t)=\Dsdd{  f\big( x+\epsilon(b-a) \big)  }{\epsilon}{0}=\| b-a \|\Dsdd{ f\big( x+\frac{ \epsilon }{ \| b-a \| }(b-a) \big) }{\epsilon}{0}=\| b-a \|\frac{ \partial f }{ \partial u }(x),
    \end{equation}
    et donc encore une fois nous avons
    \begin{equation}
        \| k'(t) \|\leq g'(t),
    \end{equation}
    ce qui donne
    \begin{equation}
        \| k(1)-k(0) \|=g(1)-g(0),
    \end{equation}
    c'est à dire
    \begin{equation}
        \| f(b)-f(a) \|\leq \sup_{x\in\mathopen[ a , b \mathclose]}\| \frac{ \partial f }{ \partial u }(x) \|\| b-a \|.
    \end{equation}
\end{proof}

\begin{theorem} \label{ThoOYwdeVt}
    Soient \( E,V\) deux espaces vectoriels normés, une application \( f\colon E\to V\), un point \( a\in E\) tel que pour tout \( u\in E\), la dérivée
    \begin{equation}
        \Dsdd{ f(x+tu) }{t}{0}
    \end{equation}
    existe pour tout \( x\in B(a,r)\) et est continue (par rapport à \( x\)) en \( x=a\). Nous supposons de plus que
    \begin{equation}
        \frac{ \partial f }{ \partial u }(a)=0
    \end{equation}
    pour tout \( u\in E\). Alors \( f\) est différentiable en \( a\) et
    \begin{equation}
        df_a=0
    \end{equation}
\end{theorem}

\begin{proof}
    Soit \( \epsilon>0\). Pourvu que \( \| h \|\) soit assez petit pour que \( a+h\in B(a,r)\), la proposition~\ref{ProFSjmBAt} nous donne
    \begin{equation}
        \| f(a+h)-f(a) \|\leq \sup_{x\in\mathopen[ a , a+h \mathclose]}\| \frac{ \partial f }{ \partial u }(x) \|  |h |
    \end{equation}
    où \( u=h/\| h \|\). Par continuité de \( \partial_uf(x)\) en \( x=a\) et par le fait que cela vaut \( 0\) en \( x=a\), il existe un \( \delta>0\) tel que si \( \| h \|<\delta\) alors
    \begin{equation}
        \| \frac{ \partial f }{ \partial u }(a+h) \|\leq \epsilon.
    \end{equation}
    Pour de tels \( h\) nous avons
    \begin{equation}
        \| f(a+h)-f(a) \|\leq \epsilon\| h \|,
    \end{equation}
    ce qui prouve que l'application linéaire \( T(u)=0\) convient parfaitement pour faire fonctionner la définition \ref{DefDifferentiellePta}.
%
%    Nous ne supposons plus que les dérivées directionnelles de \( f\) sont nulles en \( x=a\). Alors nous posons, pour \( x\in U\),
%    \begin{equation}    \label{EqCUgHXHy}
%        g(x)=f(x)-\Dsdd{ f(a+s(x-a)) }{s}{0}.
%    \end{equation}
%    Le fait que cette fonction soit bien définie est encore un coup de hypothèses sur les dérivées directionnelles de \( f\) qui sont bien définies autour de \( a\). Cette nouvelle fonction \( g\) satisfait à \( \frac{ \partial g }{ \partial v }(a)=0\) pour tout \( v\in E\) parce que
%    \begin{subequations}
%        \begin{align}
%            \frac{ \partial g }{ \partial v }(a)&=\Dsdd{ g(a+tv) }{t}{0}\\
%            &=\Dsdd{ f(a+tv)-\Dsdd{ f\big( a+s(tv) \big) }{s}{0} }{t}{0}\\
%            &=\frac{ \partial f }{ \partial v }(a)-\Dsdd{ t\frac{ \partial f }{ \partial v }(a) }{t}{0}\\
%            &=0.
%        \end{align}
%    \end{subequations}
%    Pour la dérivée par rapport à \( s\) nous avons effectué le changement de variables \( s\to ts\), ce qui explique la présence d'un \( t\) en facteur. La fonction \( g\) est donc différentiable en \( a\).
%
%
% Position 229262367
    % Attention : ce qui suit est faux. Mais il y a peut-être moyen d'adapter.
%\item[Dérivées non nulles]
%
%    Nous allons montrer que la fonction
%    \begin{equation}
%        l(x)=\Dsdd{ f\big( a+s(x-a) \big) }{t}{0}
%    \end{equation}
%    est différentiable en \( x=a\), de différentielle \( T(u)=l(u+a)\). Cela fournira la différentiabilité de \( f\) parce que \eqref{EqCUgHXHy} donnerait alors \( f\) comme somme de deux fonctions différentiables.
%
%    En premier lieu nous devons montrer que \( T\) ainsi définie est linéaire.
%
%    Notre but est donc de prouver que
%    \begin{equation}
%        \lim_{h \to 0}\frac{ \| l(x+h)-l(x)-l(h) \| }{ \| h \| }=0.
%    \end{equation}
%    Un premier pas est de calculer
%    \begin{subequations}
%        \begin{align}
%            l(x+h)-l(x)-l(h)&=\lim_{s\to 0}\frac{ f\big( s(x+h) \big)-f(0)-f(sx)+f(0)-f(sh)+f(0) }{ s }\\
%            &=\lim_{s\to 0}\frac{ f\big( s(x+h) \big)-f(sx)-f(sh)+f(0) }{ s }.
%        \end{align}
%    \end{subequations}
%    Ensuite nous étudions le numérateur en utilisant la proposition~\ref{ProFSjmBAt}:
%    \begin{subequations}
%        \begin{align}
%            \| f\big( s(x+h) \big)-f(sx)-f(sh)+f(0) \|&\leq  \| f\big( s(x+h) \big)-f(sx)\| + \|f(sh)-f(0) \|  \\
%            &\leq \sup_{z\in\mathopen[ sx , sx+sh \mathclose]}\| \frac{ \partial f }{ \partial h }(z) \|\| sh \|\\
%            &\quad +\sup_{z\in\mathopen[ 0 , sh \mathclose]}\| \frac{ \partial f }{ \partial h }(z) \|\| sh \|.
%        \end{align}
%    \end{subequations}
%    La division par \( s\) se passe bien et nous avons
%    \begin{subequations}
%        \begin{align}
%            \| l(x+h)-l(x)-l(h) \|&\leq \lim_{s\to 0}  \sup_{z\in\mathopen[ sx , sx+sh \mathclose]}\| \frac{ \partial f }{ \partial h }(z) \|\| h \|+ \sup_{z\in\mathopen[ 0 , sh \mathclose]}\| \frac{ \partial f }{ \partial h }(z) \|\| h \|\\
%            &=2\| h \|\| \frac{ \partial f }{ \partial h }(0) \|        \label{SubeqVMMoSDH}\\
%            &=2\| h \|^2\| \frac{ \partial f }{ \partial u }(0) \|
%        \end{align}
%    \end{subequations}
%    où nous avons posé \( u=h/\| h \|\). Pour l'égalité \eqref{SubeqVMMoSDH} nous avons utilisé la continuité de \( \frac{ \partial f }{ \partial h }(z)\) en \( z=0\). Du coup
%    \begin{equation}
%        \lim_{y\to 0} \frac{ \| f(x+h)-f(x)-f(h) \| }{ \| h \| }=\lim_{h\to 0} 2\| h \|\| \frac{ \partial f }{ \partial u }(0) \|=0.
%    \end{equation}
%    Cela prouve que \( l\) est bien différentiable en \( x=0\).
%
%    \end{subproof}
%
\end{proof}

%---------------------------------------------------------------------------------------------------------------------------
\subsection{L'inverse, sa différentielle}
%---------------------------------------------------------------------------------------------------------------------------

Si \( E\) est un espace de Banach, nous sommes intéressés à l'espace \( \GL(E)\) des endomorphismes inversibles de \( E\) sur \( E\). Cet ensemble est métrique par la formule usuelle
\begin{equation}
    \| T \|=\sup_{\| x \|=1}\| T(x) \|_E.
\end{equation}

\begin{theorem}[Inverse dans \( \GL(E)\)\cite{laudenbach2000calcul,SNPdukn}]    \label{ThoCINVBTJ}
    Soient \( E\) et \( F\) des espaces vectoriels normés.
    \begin{enumerate}
        \item
        L'ensemble \( \GL(E)\) est ouvert dans \( \End(E)\).
    \item
        L'application inverse
    \begin{equation}
        \begin{aligned}
        i\colon \GL(E,F)&\to \GL(F,E) \\
        u&\mapsto u^{-1}
        \end{aligned}
    \end{equation}
    est de classe \( C^{\infty}\) et
    \begin{equation}
        di_{u_0}(h)=-u_0^{-1}\circ h\circ u_0^{-1}
    \end{equation}
    pour tout \( h\in\End(E)\)
    \end{enumerate}
\end{theorem}
\index{différentielle!de $u\mapsto u^{-1}$}

\begin{proof}
Nous supposons que \( \GL(E,F)\) n'est pas vide, sinon ce n'est pas du jeu.
        \begin{subproof}

        \item[Cas de dimension finie]

            Si la dimension de \( E\) et \( F\) est finie, elles doivent être égales, sinon il n'y a pas de fonctions inversibles \( E\to F\). L'ensemble \( \GL(E,F)\) est donc naturellement \( \GL(n,\eR)\). Un élément de \( \eM(n,\eR)\) est dans \( \GL(n,\eR)\) si et seulement si son déterminant est non nul. Le déterminant étant une fonction continue (polynomiale) en les entrées de la matrice, l'ensemble \( \GL(n,\eR)\) est ouvert dans \( \eM(n,\eR)\).

            Même idée pour la régularité de la fonction \( i\colon \GL(n,\eR)\to \GL(n,\eR)\), \( X\mapsto X^{-1}\). Les entrées de \( X^{-1}\) sont les cofacteurs de \( X\) divisé par \( \det(X)\), et donc des polynômes en les entrées de \( X\) divisés par un polynôme qui ne s'annule pas sur \( \GL(n,\eR)\), et donc sur un ouvert autour de \( X\) et de \( X^{-1}\). Bref, tout est \(  C^{\infty}\).

            Le reste de la preuve parle de la dimension infinie.

        \item[Ouvert autour de l'identité]

        Nous commençons par prouver que \( B(\mtu,1)\subset \GL(E)\). Pour cela il suffit de remarquer que si \( \| u \|<1\) alors le lemme~\ref{PropQAjqUNp} nous donne un inverse de \( (1+u)\) en la personne de \( \sum_{k=0}^{\infty}(-u)^k\).

    \item[Ouvert en général]

        Soit maintenant \( u_0\in\GL(E)\). Si \( \| u \|<\frac{1}{ \| u_0^{-1} \| }\) alors \( \| u_0^{-1}u \|<1\), ce qui signifie que
        \begin{equation}
            \mtu+u_0^{-1}u
        \end{equation}
    est inversible. Mais \( u_0+u=u_0(\mtu+u_0^{-1}u)\), donc \( u_0+u\in\GL(E)\) ce qui signifie que
    \begin{equation}
    B\left( u_0,\frac{1}{ \| u_0^{-1} \| } \right)\subset \GL(E).
    \end{equation}

    \item[Différentielle en l'identité]

    Nous commençons par prouver que \( di_{\mtu}(u)=-u\). Pour cela nous posons
    \begin{equation}
        \alpha(h)=\sum_{k=2}^{\infty}(-1)^kh^k
    \end{equation}
    et nous calculons
    \begin{equation}
    di_{\mtu}(u)=\Dsdd{ i(\mtu+tu) }{t}{0}=\Dsdd{ \mtu-tu+\alpha(tu) }{t}{0}.
    \end{equation}
    Il suffit de prouver que \( \Dsdd{ \alpha(tu) }{t}{0}=0\) pour conclure que \( di_{\mtu}(u)=-u\). Pour cela, nous remarquons que \( \alpha(0)=0\) et donc que
    \begin{subequations}
        \begin{align}
        \Dsdd{ \alpha(tu) }{t}{0}&=\lim_{t\to 0} \frac{ \alpha(tu)-\alpha(0) }{ t }\\
        &=\lim_{t\to 0} \sum_{k=2}^{\infty}(-1)^k\frac{ (tu)^k }{ t }\\
        &=-\lim_{t\to 0} u\sum_{k=1}^{\infty}(-1)^kt^ku^k.
        \end{align}
    \end{subequations}
    La norme de ce qui est dans la limite est majorée par
    \begin{equation}
    \| u \|\sum_{k=1}^{\infty}\| tu \|^k=\| u \|\left( \frac{1}{ 1-\| tu \| }-1 \right),
    \end{equation}
    et cela tend vers zéro lorsque \( t\to\infty\). Nous avons utilisé la somme~\ref{EqRGkBhrX} de la série géométrique. Nous avons bien prouvé que \( di_{\mtu}(u)=-u\).

    \item[Différentielle en général]
    Soit maintenant \( u_0\in\GL(E)\) et \( h\in\End(E)\) tel que \( u_0+h\in \GL(E)\); par le premier point, il suffit de prendre \( \| h \|\) suffisamment petit. Vu que \( u_0+h=u_0(\mtu+u_0^{-1}h)\) nous avons
    \begin{equation}
        (u_0+h)^{-1}=(\mtu+u_0^{-1}h)^{-1}u_0^{-1}.
    \end{equation}
    Nous pouvons donc calculer
    \begin{equation}
        (u_0+h)^{-1}=\big( \mtu-u_0^{-1}h+\alpha(u_0^{-1}h) \big)u_0^{-1}=u_0^{-1}-u_0^{-1}hu_0^{-1}+\alpha(u_0^{-1}h)u_0^{-1},
    \end{equation}
    et ensuite
    \begin{equation}
        di_{u_0}(h)=\Dsdd{ i(u_0+th) }{t}{0}=\Dsdd{ u_0^{-1}-tu_0^{-1}hu_0^{-1}+\alpha(tu_0^{-1}h)u_0^{-1} }{t}{0},
    \end{equation}
    mais nous avons déjà vu que
    \begin{equation}
        \Dsdd{ \alpha(th) }{t}{0}=0,
    \end{equation}
    donc
    \begin{equation}
        di_{u_0}(h)=-u_0^{-1}hu_0^{-1}
    \end{equation}
    Cela donne la différentielle de l'application inverse.

    \item[Continuité de l'inverse]

        L'application \( i\) est continue parce que différentiable.
    \item[L'inverse est \(  C^{\infty}\)]

        Nous allons écrire la fonction inverse comme une composée. Soient les applications
        \begin{equation}
            \begin{aligned}
                B\colon \cL(F,E)\times \cL(F,E)&\to \cL\big( \cL(E,F),\cL(F,E) \big) \\
                B(\psi_1,\psi_2)(A)&= -\psi_1\circ A\circ\psi_2
            \end{aligned}
        \end{equation}
        et
        \begin{equation}
            \begin{aligned}
                \Delta\colon \cL(F,E)&\to \cL(F,E)\times \cL(F,E) \\
                \varphi&\mapsto (\varphi,\varphi)
            \end{aligned}
        \end{equation}
        Nous avons alors
        \begin{equation}
            di=B\circ\Delta\circ i.
        \end{equation}
        L'application \( \Delta\) est de classe \(  C^{\infty}\). Nous devons voir que \( B\) l'est aussi. Pour le voir nous commençons par prouver qu'elle est bornée :
        \begin{equation}
            \begin{aligned}[]
                \| B \|&=\sup_{\| \psi_1 \|,\| \psi_2 \|=1}\| B(\psi_1,\psi_2) \|_{\aL\big( L(E,F),L(F,E) \big)}\\
                &=\sup_{  \| \psi_1 \|,\| \psi_2 \|=1 }\sup_{\| A \|=1}\| \psi_1\circ A\circ\psi_2 \|_{L(F,E)}\\
                &\leq \sup_{\| \psi_1 \|,\| \psi_2 \|=1}\sup_{\| A \|=1}\| \psi_1 \|\| A \|\| \psi_2 \|\\
                &\leq 1.
            \end{aligned}
        \end{equation}
        Donc \( B\) est bien bornée et par conséquent continue. Une application bilinéaire continue est \(  C^{\infty}\) par le lemme~\ref{LemFRdNDCd}. La décomposition \( di=B\circ \Delta\circ i\) nous donne donc que \( i\in C^{\infty}\) dès que \( i\) est continue, ce que nous avions déjà montré.
        \end{subproof}
\end{proof}

%---------------------------------------------------------------------------------------------------------------------------
\subsection{Projection orthogonale}
%---------------------------------------------------------------------------------------------------------------------------

Le théorème suivant n'est pas indispensablissime parce qu'il est le même que le théorème de la projection sur les espaces de Hilbert\footnote{Théorème~\ref{ThoProjOrthuzcYkz}}. Cependant la partie existence est plus simple en se limitant au cas de dimension finie.
\begin{theorem}[Théorème de la projection]  \label{ThoWKwosrH}
    Soit \( E\) un espace vectoriel réel ou complexe de dimension finie, \( x\in E\), et \( C\) un sous-ensemble fermé convexe de \(E\).
    \begin{enumerate}
        \item
            Les deux conditions suivantes sur \( y\in E\) sont équivalentes:
    \begin{enumerate}
        \item   \label{zzETsfYCSItemi}
            \( \| x-y \|=\inf\{ \| x-z \|\tq z\in C \}\),
        \item\label{zzETsfYCSItemii}
            pour tout \( z\in C\), \( \real\langle x-y, z-y\rangle \leq 0\).
    \end{enumerate}
\item
    Il existe un unique \( y\in E\), noté \( y=\pr_C(x)\) vérifiant ces conditions.
    \end{enumerate}
\end{theorem}
%TODO : il y a sûrement un endroit plus adapté pour mettre ce théorème.

\begin{proof}
    Nous commençons par prouver l'existence et l'unicité d'un élément dans \( C\) vérifiant la première condition. Ensuite nous verrons l'équivalence.

    \begin{subproof}
        \item[Existence]

            Soit \( z_0\in C\) et \( r=\| x-z_0 \|\). La boule fermée \( \overline{ B(x,r) }\) est compacte\footnote{C'est ceci qui ne marche plus en dimension infinie.} et intersecte \( C\). Vu que \( C\) est fermé, l'ensemble \( C'=C\cap\overline{ B(x,r) }\) est compacte. Tous les points qui minimisent la distance entre \( x\) et \( C\) sont dans \( C'\); la fonction
            \begin{equation}
                \begin{aligned}
                     C'&\to \eR \\
                    z&\mapsto d(x,z)
                \end{aligned}
            \end{equation}
            est continue sur un compact et donc a un minimum qu'elle atteint\footnote{Théorème~\ref{ThoMKKooAbHaro}.}. Un point \( P\) réalisant ce minimum prouve l'existence d'un point vérifiant la première condition.

        \item[Unicité]
            Soient \( y_1\) et \( y_2\), deux éléments de \( C\) minimisant la distance avec \( x\), et soit \( d\) ce minimum. Nous avons par l'identité du parallélogramme \eqref{EqYCLtWfJ} que
            \begin{equation}
                \| y_1-y_2 \|^2=-4\left\| \frac{ y_1+y_2-x }{2} \right\|^2+2\| y_1-x \|^2+2\| y_2-x \|^2\leq -4d+2d+2d=0.
            \end{equation}
            Par conséquent \( y_1=y_2\).

        \item[\ref{zzETsfYCSItemi}\( \Rightarrow\)~\ref{zzETsfYCSItemii}]

            Soit \( z\in C\) et \( t\in \mathopen] 0 , 1 \mathclose[\); nous notons \( P=\pr_Cx\). Par convexité le point \( z=ty+(1-t)P\) est dans \( C\), et par conséquent,
                \begin{equation}
                    \| x-P \|^2\leq\| x-tz-(1-t)P \|^2=\| (x-P)-t(z-P) \|^2.
                \end{equation}
                Nous sommes dans un cas \( \| a \|^2\leq | a-b |^2\), qui implique \( 2\real\langle a, b\rangle \leq \| b \|^2\). Dans notre cas,
                \begin{equation}
                    2\real\langle x-P , t(z-P)\rangle \leq t^2\| z-P \|^2.
                \end{equation}
                En divisant par \( t\) et en faisant \( t\to 0\) nous trouvons l'inégalité demandée :
                \begin{equation}
                    2\real\langle x-P, z-P\rangle \leq 0.
                \end{equation}

        \item[\ref{zzETsfYCSItemii}\( \Rightarrow\)~\ref{zzETsfYCSItemi}]

            Soit un point \( P\in C\) vérifiant
            \begin{equation}
                \real\langle x-P, z-P\rangle \leq 0
            \end{equation}
            pour tout \( z\in C\). Alors en notant \( a=x-P\) et \( b=P-z\),
            \begin{equation}
                \begin{aligned}[]
                \| x-z \|^2=\| x-P+P-z \|^2&=\| a+b \|^2\\
                &=\| a \|^2+\| b \|^2+2\real\langle a, b\rangle \\
                &=\| a \|^2+\| b \|^2-2\real\langle x-P, z-P\rangle \\
                &\geq \| b \|^2,
                \end{aligned}
            \end{equation}
            ce qu'il fallait.
    \end{subproof}
\end{proof}

%+++++++++++++++++++++++++++++++++++++++++++++++++++++++++++++++++++++++++++++++++++++++++++++++++++++++++++++++++++++++++++
\section{Jacobienne}
%+++++++++++++++++++++++++++++++++++++++++++++++++++++++++++++++++++++++++++++++++++++++++++++++++++++++++++++++++++++++++++

\subsection{Rappels et définitions}

Dans cette section nous considérons des fonctions $f : D \to \eR^m$
où $D \subset \eR^n$, et un point $a \in \Int D$ où $f$ est
différentiable.
\begin{remark}
  La définition de continuité (resp. différentiabilité) pour une
  fonction à valeurs vectorielles est celle introduite précédemment,
  et on remarque que pour avoir la continuité
  (resp. différentiabilité) de $f$ en un point, il faut et il suffit
  de chacune des composantes de $f = (f_1,\ldots, f_m)$, vues
  séparément comme fonctions à $n$ variables et à valeurs réelles,
  soit continue (resp. différentiable) en ce point.
\end{remark}

\begin{definition}
    La \defe{jacobienne}{matrice!jacobienne} de $f$ en $a$ est la matrice de l'application linéaire donnée par la différentielle. Elle a de nombreuses notations
  \begin{equation}
      J_f(a) = \frac{ \partial (f_1,\ldots, f_m) }{ \partial x_1,\ldots, x_m }=
    \begin{pmatrix}
      \pder {f_1} {x_1}(a) & \ldots &\pder {f_1} {x_n}(a)\\
      \vdots& & \vdots\\
      \pder {f_m} {x_1}(a) & \ldots &\pder {f_m} {x_n}(a)
    \end{pmatrix}
  \end{equation}
  Autrement dit, c'est la matrice composée de l'ensemble des dérivées partielles de $f$. Le \defe{jacobien}{jacobien} de \( f\) au point \( a\) est le déterminant de cette matrice.

  Si $m = 1$, cette matrice ne contient qu'une ligne ; c'est donc un vecteur appelé le \defe{gradient}{gradient} de $f$ au point $a$ et noté $\nabla f(a)$.
\end{definition}

\begin{remark}
  \begin{enumerate}
  \item Si la fonction est supposée différentiable, calculer la
    jacobienne revient à connaître la différentielle. En effet, par
    linéarité de la différentielle et par définition des dérivées
    partielles, nous avons
    \begin{equation*}
      d f_a (u) =%
      \begin{pmatrix}
        \pder {f_1} {x_1}(a) & \ldots &\pder {f_1} {x_n}(a)\\
        \vdots& & \vdots\\
        \pder {f_m} {x_1}(a) & \ldots &\pder {f_m} {x_n}(a)
      \end{pmatrix}
      \begin{pmatrix}u_1\\\vdots\\u_n\end{pmatrix}
    \end{equation*}
    où $u = (u_1, \ldots, u_n)$ et où le membre de droite est un
    produit matriciel

  \item Remarquons que la jacobienne peut exister en un point donné
    sans que la fonction soit différentiable en ce point !
  \end{enumerate}
\end{remark}


\begin{normaltext}
    Le théorème de différentiation de fonctions composées \ref{PropDiffCompose} peut également se lire au niveau des matrices jacobiennes. La matrice jacobienne de $g\circ f$ au point $a$ est le produit matriciel des matrices jacobiennes de $f$ et de $f$. Plus précisément, nous avons
    \begin{equation}
        J_{g\circ f}(a)=J_g\big( f(a) \big)J_f(a).
    \end{equation}
    Remarquez que nous considérons la matrice jacobienne de $g$ au point $f(a)$.

    Dans la cas particulier où $m=1$ et $f$ est une fonction d'un intervalle $I$ dans $\eR^n$, dérivable au point $a$, on a que la fonction composée $g\circ f$ est dérivable au point $a$ si $g$ est différentiable et alors
    \[
    (g\circ f)'(a)= dg\left(f(a)\right).f'(a).
    \]
    En fait, pour les fonctions d'une seule variable la dérivabilité coïncide avec la différentiabilité.
\end{normaltext}<++>
