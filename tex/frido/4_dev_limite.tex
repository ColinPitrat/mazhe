% This is part of Analyse Starter CTU
% Copyright (c) 2014,2017
%   Laurent Claessens,Carlotta Donadello
% See the file fdl-1.3.txt for copying conditions.

%+++++++++++++++++++++++++++++++++++++++++++++++++++++++++++++++++++++++++++++++++++++++++++++++++++++++++++++++++++++++++++ 
\section{Nombres de Bell}
%+++++++++++++++++++++++++++++++++++++++++++++++++++++++++++++++++++++++++++++++++++++++++++++++++++++++++++++++++++++++++++

\begin{theorem}[Nombres de Bell\cite{KXjFWKA}]  \label{ThoYFAzwSg}
    Soit \( n\geq 1\) et \( B_n\) le nombre de partitions distinctes de l'ensemble \( \{ 1,\ldots, n \}\) avec la convention que \( B_0=0\). Alors
    \begin{enumerate}
        \item
            La série entière
            \begin{equation}    \label{EqYCMGBmP}
                \sum_{n=0}^{\infty}\frac{ B_n }{ n! }x^n
            \end{equation}
            a un rayon de convergence \( R>0\) et sa somme est donnée par
            \begin{equation}
                f(x)= e^{ e^{x}-1}
            \end{equation}
            pour tout \( x\in\mathopen] -R , R \mathclose[\).
        \item
            Pour tout \( k\in \eN\),
            \begin{equation}
                B_n=\frac{1}{ e }\sum_{k=0}^{\infty}\frac{ k^n }{ k! }.
            \end{equation}
            \item
                Le rayon de convergence de la série \eqref{EqYCMGBmP} est en réalité infini : \( R=\infty\).
    \end{enumerate}
\end{theorem}
\index{anneau!de séries formelles}
\index{dénombrement!partitions de \( \{ 1,\ldots, n \} \)}
\index{série!numérique}
\index{série!entière}
\index{limite!inversion}

\begin{proof}
    \begin{enumerate}
        \item
            Soit \( n\geq 1\) et \( 0\leq k\leq n\). Nous notons \( E_k\) l'ensemble des partitions de \( \{ 1,\ldots, n+1 \}\) pour lesquelles le «paquet» contenant \( n+1\) soit de cardinal \( k+1\). Calculons le cardinal de \( E_k\).

            Pour construire un élément de \( E_k\), il faut d'abord prendre le nombre \( n+1\) et lui adjoindre \( k\) éléments choisis dans \( \{ 1,\ldots, n \}\), ce qui donne \( n\choose k\) possibilités. Ensuite il faut trouver une partition des \( (n+1)-(k+1)=n-k\) éléments restants, ce qui fait \( B_{n-k}\) possibilités. Donc
            \begin{equation}
                \Card(E_k)={n\choose k}B_{n-k}.
            \end{equation}
            L'intérêt des ensembles \( E_k\) est que \( \{ E_0,\ldots, E_n \}\) est une partition de l'ensemble des partitions de \( \{ 1,\ldots, n+1 \}\), c'est à dire que \( B_{n+1}=\sum_{k=0}^n\Card(E_k)\), ce qui va nous donner une relation de récurrence pour les \( B_n\) :
\begin{equation}
                    B_{n+1}=\sum_{k=0}^n\Card(E_k)
                   =\sum_{k=0}^n{n\choose k}B_{n-k}
                    =\sum_{l=0}^n{n\choose n-l}B_l 
                    =\sum_{l=0}^n{n\choose l}B_l.
\end{equation}
où nous avons utilisé un petit changement de variables \( l=n-k\). Afin d'étudier la convergence de la série \eqref{EqYCMGBmP}, nous allons montrer par récurrence que pour tout \( n\), \( B_n<n!\). D'abord pour \( n=0\) c'est bon : \( B_1=1\) parce que la seule partition de \( \{ 1 \}\) est \( \{ 1 \}\). Supposons que l'inégalité soit vraie pour une certaine valeur \( k\), et montrons qu'elle est vraie pour la valeur \( k+1\) :
\begin{equation}
                    B_{k+1}=\sum_{l=0}^n{n\choose k}B_k
                   \leq \sum_{l=0}^n{n\choose k}k!
                    =k!\sum_{l=0}^k\underbrace{\frac{1}{ (n-k)! }}_{\leq 1}
                    \leq n!(n+1)
                    =(n+1)!
\end{equation}
            où nous avons utilisé la formule \( {n\choose k}=\frac{ n! }{ k!(n-k)! }\).

            Donc pour tout \( x\in \eR\) nous avons
            \begin{equation}
                0\leq \frac{ B_n }{ n! }| x^n |\leq | x |^n,
            \end{equation}
            et donc la série a un rayon de convergence au moins aussi grand que celui de la série géométrique, c'est à dire que \( 1\). Donc \( R\geq 1\). Nous nommons \( R\) ce rayon de convergence.

        \item

            Soit \( x\in\mathopen] -R , R \mathclose[\). Pour une telle valeur de \( x\) à l'intérieur du disque de convergence, la proposition \ref{ProptzOIuG} nous permet de dériver terme à terme la série\footnote{C'est ici qu'on utilise la convention \( B_0=0\) et ça aura une influence sur le choix de la constante \( K\) plus bas.}
                \begin{equation}
                    f(x)=\sum_{k=0}^{\infty}\frac{ B_k }{ k! }x^k=1+\sum_{k=0}^{\infty}\frac{ B_{k+1} }{ (k+1)! }x^{k+1},
                \end{equation}
                pour obtenir
                \begin{equation}
                    f'(x)=\sum_{k=0}^{\infty}\frac{ B_{k+1} }{ (k+1) }(k+1)x^k=\sum_{k=0}^{\infty}\frac{ B_{k+1} }{ k! }x^k=\sum_{k=0}^{\infty}\left( \sum_{l=0}^k{k\choose l}B_l \right)\frac{ x^k }{ k! }=\sum_{k=0}^{\infty}\left( \sum_{l=0}^k\frac{ B_l }{ l!(l-k)! } \right)x^k.
                \end{equation}
                En cette expression, nous reconnaissons un produit de Cauchy (proposition \ref{ThokPTXYC}) avec \( a_l=\frac{ B_l }{ l! }\) et \( b_n=\frac{ 1 }{ n! }\). Vu que ce sont deux séries ayant un rayon de convergence plus grand que zéro, le produit a encore un rayon de convergence plus grand que zéro et nous pouvons prendre le produit des séries :
                \begin{equation}
                    f'(x)=\left( \sum_{l=0}^{\infty}\frac{ B_l }{ l! }x^l \right)\left( \sum_{k=0}^{\infty}\frac{1}{ k! }x^k \right)=f(x) e^{x}.
                \end{equation}
            Étudions l'équation différentielle \( y'=ye^x\). D'abord par un argument en lacet de chaussure\footnote{Genre ce qui est fait pour prouver \ref{ThoRWOZooYJOGgR}\ref{ItemYTLTooSnfhOu}.}, une solution est de classe \(  C^{\infty}\). Ensuite si une solution est non nulle, elle est de signe constant. En effet si \( y(x_0)<0\) et \( y(x_1)=0\) (on choisit \( x_1\) minimum pour cette propriété parmi les nombres plus grands que \( x_0\)) alors il existe\footnote{Théorème de Rolle \ref{ThoRolle}.} un \( t\in\mathopen] x_0 , x_1 \mathclose[\) tel que \( y'(t)>0\), ce qui donnerait \( y(t)>0\), ce qui contredirait la minimalité de \( x_1\).

                Nous prétendons\footnote{Ou alors on utilise le théorème \ref{ThoNYEXqxO} avec \( M(x)=e^x\) dans les cas \( n=1\) et \( I=\mathopen] -R , R \mathclose[\).} que cette équation différentielle a un espace de solutions de dimension \( 1\). En effet, si \( y'=ye^x\) et \( g'=ge^x\) alors en posant \( \varphi=y/g\) nous obtenons tout de suite \( \varphi'=0\), ce qui signifie que \( \varphi\) est constante, ou encore que \( y\) et \( g\) sont multiples l'un de l'autre.

                 Si nous en trouvons une non nulle par n'importe quel moyen, c'est bon. Une solution étant dérivable est continue, donc l'équation \( f'=f e^{x}\) nous indique que \( f'\) est continue. Une solution non nulle va automatiquement accepter un petit voisinage sur lequel la manipulation suivante a un sens :
                    \begin{equation}
                        \frac{ f'(x) }{ f(x) }= e^{x},
                    \end{equation}
                    donc \( \ln\big( | f(x) | \big)= e^{x}+C\) et \( f(x)=K e^{ e^{x}}\) pour une certaine constante. Il est vite vérifié que cette fonction est une solution de l'équation différentielle \( y'(x)=y(x) e^{x}\) et par unicité, toutes les solutions sont de cette forme. Autrement dit, l'espace des solution est l'espace vectoriel \( \Span\{ x\mapsto e^{e^x} \}\). Étant donné que \( f(0)=0\), nous devons choisir \( K=\frac{1}{ e }\) et donc 
                    \begin{equation}
                        f(x)=\frac{1}{ e } e^{e^x}= e^{e^x-1}.
                    \end{equation}

                \item

                    Nous commençons par écrire la fonction \( f\) comme une série de puissance. La partie simple du calcul : pour \( x\in \mathopen] -R , R \mathclose[\), nous avons
                        \begin{equation}    \label{EqODjgjDN}
                        e^{e^x}=\sum_{k=0}^{\infty}\frac{ (e^x)^k }{ k! }=\sum_{k=0}^{\infty}\frac{1}{ k! }\sum_{l=0}^{\infty}\frac{ (kx)^l }{ l! }=\sum_{k=0}^{\infty}\sum_{l=0}^{\infty}\frac{k^l}{k! }\frac{ x^l }{ l! }.
                    \end{equation}
                    Notons que cela n'est pas une série de puissance en \( x\) parce qu'il y a la double somme. Nous allons inverser les sommes au moyen du théorème de Fubini sous la forme du corollaire \ref{CorTKZKwP}. Pour cela nous considérons la fonction
                    \begin{equation}
                        \begin{aligned}
                            a\colon \eN\times \eN&\to \eR \\
                            (k,l)&\mapsto \frac{ (kx)^l }{ k!l! } 
                        \end{aligned}
                    \end{equation}
                    et nous mettons la mesure de comptage\footnote{Nous passons outre les avertissements et menaces de Arnaud Girand.} sur \( \eN\) et \( \eN^2\). Nous commençons donc à vérifier l'intégrabilité variable par variable de \( | a |\) :
                    \begin{subequations}    \label{SubEqsFHsBfhk}
                        \begin{align}
                            \int_{\eN}\left( \int_{\eN}| a(k,l) |dm(l) \right)dm(k)&=\sum_{k=0}^{\infty}\frac{1}{ k! }\frac{ (k| x |)^l }{ l! }\\
                            &=\sum_{k=0}^{\infty}\frac{1}{ k! } e^{k| x |}.
                        \end{align}
                    \end{subequations}
                    Nous devons montrer que cette dernière somme va bien. Pour cela nous posons \( u_k=\frac{ e^{k| x |} }{ k! }\) et nous remarquons que \( \frac{ u_{k+1} }{ u_k }\to 0\). Donc la double intégrale \eqref{SubEqsFHsBfhk} converge, ergo \( a\in L^1(\eN\times \eN)\), ce qui nous permet d'utiliser le théorème de Fubini \ref{ThoFubinioYLtPI} pour inverser les \sout{sommes} \sout{intégrales} sommes dans l'équation \eqref{EqODjgjDN} :
                    \begin{equation}
                        \frac{1}{ e }e^{e^x}=\frac{1}{ e }\sum_{k=0}^{\infty}\sum_{l=0}^{\infty}\frac{1}{ k! }\frac{1}{ l! }(kx)^l=\sum_{l=0}\frac{1}{ e }\frac{1}{ l! }\left( \sum_{k=0}^{\infty}\frac{ k^l }{ k! } \right)x^l.
                    \end{equation}
                    Cela est un développement en série entière pour la fonction \( \frac{1}{ e } e^{e^x}\), dont nous savions déjà le développement \eqref{EqYCMGBmP}; par unicité du développement nous pouvons identifier les coefficients :
                    \begin{equation}
                        B_l=\frac{1}{ e }\sum_{k=0}^{\infty}\frac{ k^l }{ k! }.
                    \end{equation}
                    
                \item

                    Le développement \eqref{EqODjgjDN} étant en réalité valable pour tout \( x\) et tous les calculs subséquents l'étant aussi, le développement
                    \begin{equation}
                        e^{e^x-1}=\sum_{n=0}^{\infty}\frac{ B_n }{ n! }x^n
                    \end{equation}
                    est en fait valable pour tout \( x\), ce qui donne à la série entière un rayon de convergence infini.
    \end{enumerate}
\end{proof}



%+++++++++++++++++++++++++++++++++++++++++++++++++++++++++++++++++++++++++++++++++++++++++++++++++++++++++++++++++++++++++++ 
\section{Étude d'asymptote}
%+++++++++++++++++++++++++++++++++++++++++++++++++++++++++++++++++++++++++++++++++++++++++++++++++++++++++++++++++++++++++++

Lorsqu'une fonction tend vers l'infini pour \( x\to \infty\), une question qui peut venir est : à quelle vitesse tend-t-elle vers l'infini ?

Il est «visible» que la fonction logarithme ne tend pas très vite vers l'infini : certes
\begin{equation}
    \lim_{x\to \infty} \ln(x)=+\infty,
\end{equation}
mais par exemple \( \ln(100000)\simeq 11.5\) tandis que \(  e^{100000}\simeq 10^{43429}\). Sans contestations possibles, l'exponentielle croit plus vite que le logarithme.

Soient \( f\) et \( g\) deux fonctions dont la limite \( x\to \infty\) est \( \infty\). Si
\begin{equation}
    \lim_{x\to \infty} \frac{ f(x) }{ g(x) }=0
\end{equation}
nous disons que \( g\) tend vers \( \infty\) plus vite que \( f\); si
\begin{equation}
    \lim_{x\to \infty} \frac{ f(x) }{ g(x) }=\infty
\end{equation}
nous disons que \( f\) tend vers \( \infty\) plus vite que \( g\), et si
\begin{equation}
    \lim_{x\to \infty} \frac{ f(x) }{ g(x) }=a\in \eR
\end{equation}
avec \( a\neq 0\) alors nous disons que \( f\) tend vers l'infini à la même vitesse que \( ag(x)\).

\begin{example}
    La fonction \( x\mapsto x^2\) tend vers l'infini plus vite que la fonction \( x\mapsto \sqrt{x}\).
\end{example}

Dans cette section nous allons nous contenter de déterminer les fonctions qui tendent vers l'infini aussi vite qu'une droite oblique, que nous appellons asymptote et que nous voulons déterminer.


\begin{example}
    Déterminer les asymptotes obliques (s'ils existent) de la fonction
    \begin{equation}
        f(x)= e^{1/x}\sqrt{1+4x^2}.
    \end{equation}
    Tout d'abord nous remarquons que \( \lim_{x\to \infty} f(x)=\infty\). Nous sommes donc en présence d'une branche du graphe qui tend vers l'infini. Ensuite,
    \begin{equation}
        \lim_{x\to \infty} \frac{ f(x) }{ x }=\lim_{x\to \infty}  e^{1/x}\sqrt{\frac{1}{ x^2 }+4}=2.
    \end{equation}
    Donc le graphe de \( f\) tend vers l'infini à la même vitesse que le graphe de la fonction \( y=2x\). Nous aurons donc une asymptote oblique de coefficient directeur \( 2\). De façon imagée, nous pouvons penser que le graphe de \( f\) et celui de \( y=2x\) sont presque parallèles si \( x\) est assez grand. Afin de déterminer l'ordonnée à l'origine de l'asymptote, il nous reste à voir quelle est la «distance» entre le graphe de \( f\) et celui de \( y=2x\) :
    \begin{equation}
        \lim_{x\to \infty} f(x)-2x=\lim_{x\to \infty}  e^{1/x}\sqrt{1+4x^2}-2x.
    \end{equation}
    Cette limite a été calculée dans l'exemple \ref{ExBCDookjljhjk} et vaut $2$.

	Nous concluons que le graphe de la fonction $f$ admet l'asymptote
    \begin{equation}
	y=2x+2.
    \end{equation}
\end{example}
