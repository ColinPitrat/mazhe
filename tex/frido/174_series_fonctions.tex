% This is part of Mes notes de mathématique
% Copyright (c) 2011-2017
%   Laurent Claessens
% See the file fdl-1.3.txt for copying conditions.

%+++++++++++++++++++++++++++++++++++++++++++++++++++++++++++++++++++++++++++++++++++++++++++++++++++++++++++++++++++++++++++ 
\section{Exponentielle et logarithme}
%+++++++++++++++++++++++++++++++++++++++++++++++++++++++++++++++++++++++++++++++++++++++++++++++++++++++++++++++++++++++++++

La méthode adoptée ici est la suivante :
\begin{itemize}
    \item L'exponentielle est définie par la série.
    \item Nous démontrons qu'elle vérifie l'équation différentielle \( y'=y\), \( y(0)=1\).
    \item Nous démontrons l'unicité de la solution à cette équation différentielle.
    \item Nous démontrons qu'elle est égale à \( x\mapsto y(1)^x\). Cela donne la définition du nombre \( e\) comme valant \( y(1)\).
    \item Nous définissons le logarithme comme l'application réciproque de l'exponentielle (définition \ref{DEFooELGOooGiZQjt}).
    \item Les fonctions trigonométriques (sinus et cosinus) sont définies par leurs séries. Il est alors montré que \(  e^{ix}=\cos(x)+i\sin(x)\).
\end{itemize}

\begin{theorem}[Existence de l'exponentielle] \label{ThoKRYAooAcnTut}
    La série entière
    \begin{equation}    \label{EqEIGZooKWSvPS}
        y(x)=\sum_{k=0}^{\infty}\frac{ x^k }{ k! }
    \end{equation}
    définit une fonction dérivable solution de
    \begin{subequations}
        \begin{numcases}{}
            y'=y\\
            y(0)=1.
        \end{numcases}
    \end{subequations}
\end{theorem}
\index{exponentielle!existence}

\begin{proof}
    La formule de Hadamard (théorème \ref{ThoSerPuissRap}) donne le rayon de convergence de la série \eqref{EqEIGZooKWSvPS} par
    \begin{equation}
        \frac{1}{ R }=\lim_{k\to \infty} \frac{ \frac{1}{ (k+1)! } }{ \frac{1}{ k! } }=\lim_{k\to \infty} \frac{1}{ k+1 }=0.
    \end{equation}
    Donc nous avons un rayon de convergence infini. La fonction \( y\) est définie sur \( \eR\) et la proposition \ref{ProptzOIuG} nous dit que \( y\) est dérivable. Nous pouvons aussi dériver terme à terme :
    \begin{equation}
            y'(x)=\sum_{k=0}^{\infty}\frac{ kx^{k-1} }{ k! }=\sum_{k=1}^{\infty}\frac{ kx^{k-1} }{ k! }=\sum_{k=1}^{\infty}\frac{ x^{k-1} }{ (k-1)! }=\sum_{k=0}^{\infty}\frac{ x^k }{ k! }=y(x).
    \end{equation}
    Notez le petit jeu d'indice de départ de \( k\). Dans un premier temps, nous remarquons que \( k=0\) donne un terme nul et nous le supprimons, et dans un second temps nous effectuons la simplification des factorielles (qui ne fonctionne pas avec \( k=0\)).
\end{proof}

Pour la suite nous notons \( y\) une solution de l'équation \( y'=y\), \( y(0)=1\), et nous allons en donner des propriétés indépendamment de l'existence, donnée par le théorème \ref{ThoKRYAooAcnTut}.

\begin{proposition} \label{PropTLECooEiLbPP}
    Quelques propriétés de \( y\) (si elle existe) :
    \begin{enumerate}
        \item
            Pour tout \( x\in \eR\) nous avons \( y(x)y(-x)=1\).
        \item
            \( y(x)>0\) pour tout \( x\).
        \item
            \( y\) est strictement croissante.
    \end{enumerate}
\end{proposition}

\begin{proof}
    Nous posons \( \varphi(x)=y(x)y(-x)\) et nous dérivons :
    \begin{equation}
        \varphi'(x)=y'(x)y(-x)-y(x)y'(-x)=0.
    \end{equation}
    Donc \( \varphi\) est constante\footnote{Proposition \ref{PropGFkZMwD}.}. Vu que \( \varphi(0)=1\) nous avons automatiquement \( y(x)y(-x)=1\) pour tout \( x\).

Les deux autres allégations sont simples : si \( y(x_0)<0\) alors il existe \( t\in\mathopen] x_0 , 1 \mathclose[\) tel que \( y(t)=0\), ce qui est impossible parce que \( y(t)y(-t)=1\). La stricte croissance de \( y\) s'ensuit.
\end{proof}

\begin{proposition}[Unicité de l'exponentielle] \label{PropDJQSooYIwwhy}
    Si elle existe, la solution au problème 
    \begin{subequations}
        \begin{numcases}{}
            y'=y\\
            y(0)=1
        \end{numcases}
    \end{subequations}
    est unique.
\end{proposition}
\index{exponentielle!unicité}

\begin{proof}
    Soient \( y\) et \( g\) deux solutions et considérions la fonction \( h(x)=g(x)y(-x)\). Un calcul immédiat donne
    \begin{equation}
        h'(x)=0
    \end{equation}
    et donc \( h\) est constante. Vu que \( h(0)=1\) nous avons \( g(x)y(-x)=1\) pour tout \( x\), c'est à dire
    \begin{equation}
        g(x)=\frac{1}{ y(-x) }=y(x).
    \end{equation}
\end{proof}

\begin{proposition}     \label{PROPooGGUIooExVHPM}
    Quelques formules pour tout \( a,b\in \eR\) et \( n\in \eZ\) :
    \begin{enumerate}
        \item       \label{ITEMooMPSUooWQpVQJ}
            \( y(a+b)=y(a)y(b)\)
        \item
            \( y(na)=y(a)^n\)
        \item
            \( y\left( \frac{ a }{ n } \right)=\sqrt[n]{y(a)}\).
    \end{enumerate}
\end{proposition}

\begin{proof}
    Nous posons \( h(x)=y(a+b-x)y(x)\) et nous avons encore \( h'(x)=0\) dont nous déduisons que $h$ est constante. De plus
    \begin{equation}
        h(0)=y(a+b)y(0)=y(a+b)
    \end{equation}
    et
    \begin{equation}
        h(b)=y(a)y(b).
    \end{equation}
    Vu que \( h\) est constante, ces deux expressions sont égales : \( y(a+b)=y(a)y(b)\).

    Forts de cette relation, une récurrence donne \( y(na)=y(a)^n\) pour tout \( n\in \eN\). De plus
    \begin{equation}
        y(a)=y\left( \frac{ a }{ n }\times n \right)=y\left( \frac{ a }{ n } \right)^n,
    \end{equation}
    ce qui donne \( y(a)=y(a/n)^n\) ou encore \( y(a/n)=\sqrt[n]{y(a)}\).

    Enfin pour les négatifs, si \( n\in \eN\),
    \begin{equation}
        y(-na)=\frac{1}{ y(na) }=\frac{1}{ y(a)^n }=y(a)^{-n}.
    \end{equation}
    Et de la même façon,
    \begin{equation}
        y\left( -\frac{ a }{ n } \right)=\frac{1}{ y\left( \frac{ a }{ n } \right) }=\sqrt[n]{\frac{1}{ y(a) }}=\sqrt[-n]{y(a)}.
    \end{equation}
\end{proof}

\begin{proposition} \label{PropCELWooLBSYmS}
    Pour tout \( x\in \eR\), nous avons
    \begin{equation}
        y(x)=y(1)^x.
    \end{equation}
\end{proposition}

\begin{proof}
    Si \( q\in \eQ\) alors \( q=a/b\) et
    \begin{equation}
        y(q)=y\left( \frac{ a }{ b } \right)=y\left( a\times \frac{1}{ b } \right)=y\left( \frac{1}{ b } \right)^a=\big( \sqrt[b]{y(1)} \big)^a=y(1)^{a/b}=y(1)^{q}.
    \end{equation}

    Par ailleurs si \( a\in \eR\), alors la fonction \( x\mapsto a^x\) est continue. Les fonctions \( y\) et \( x\mapsto y(1)^x\) sont deux fonctions continues égales sur \( \eQ\). Elles sont donc égales par la proposition \ref{PropCJGIooZNpnGF}.
\end{proof}
Nous notons \( y(1)=e\), le \defe{nombre de Néper}{nombre!de Néper}, de telle sorte que
\begin{equation}
    y(x)=e^x.
\end{equation}

Une conséquence est que 
\begin{subequations}    \label{EqLOIUooHxnEDn}
    \begin{align}
        \lim_{x\to -\infty}  e^{x}=0\\
        \lim_{x\to +\infty}  e^{x}=+\infty,
    \end{align}
\end{subequations}
et en particulier, 
\begin{equation}
    \begin{aligned}
    \exp\colon \eR&\to \mathopen] 0 , \infty \mathclose[ \\
        x&\mapsto  e^{x} 
    \end{aligned}
\end{equation}
est une bijection.

Nous donnons maintenant quelque approximations numériques de \( e\), particulièrement inefficaces.

\begin{lemma}
    Nous avons
    \begin{equation}
        2<e<3.
    \end{equation}
\end{lemma}

\begin{proof}
    Nous savons que \( y(0)=1\) et \( y'(0)=1\). La fonction \( y\) est strictement croissante (et donc sa dérivée aussi). Nous avons donc \( y'(x)>1\) pour tout \( x\in\mathopen] 0 , 1 \mathclose]\), et donc
    \begin{equation}
        y(1)>1+1\times 1=2.
    \end{equation}
    Sachant que \( 2>y'(x)\) pour tout \( x\in \mathopen] 0 , 1 \mathclose[\) nous pouvons refaire le coup de l'approximation affine, cette fois en majorant :
        \begin{equation}
            y(1)<1+2\times 1=3.
        \end{equation}
\end{proof}

De la même façon nous savons que
\begin{equation}
    y(\frac{1}{ n })>1+\frac{1}{ n }
\end{equation}
parce que \( y'\) est minoré par \( 1\) sur \( \mathopen] 0 , \frac{1}{ n } \mathclose[\). Avec cela nous avons aussi la majoration
\begin{equation}
    y(\frac{1}{ n })<1+\frac{1}{ n }\times \left( 1+\frac{1}{ n } \right)=1+\frac{1}{ n }+\frac{1}{ n^2 }.
\end{equation}
Et enfin nous pouvons donner l'encadrement, valable pour tout \( n\) :
\begin{equation}
    \left( 1+\frac{1}{ n } \right)^n<y(1)<\left( 1+\frac{1}{ n }+\frac{1}{ n^2 } \right)^n.
\end{equation}
Pour \( n=10\) nous trouvons
\begin{equation}
    2.50<e<2.83.
\end{equation}

Bien que ce soit à mon avis humainement pas possible à faire à la main nous avons, pour \( n=100\) :
\begin{equation}
    2.70<e<2.7317
\end{equation}
Cela reste un encadrement très modeste.

Une méthode plus efficace consiste à calculer directement le développement de définition
\begin{equation}
    e=\exp(1)=\sum_{k=0}^{\infty}\frac{1}{ n! }.
\end{equation}
\lstinputlisting{tex/sage/sageSnip013.sage}

\begin{probleme}
    Comment trouver, avec cette méthode, un \emph{encadrement pour \( e\) ?}
\end{probleme}
    
Ce petit programme, avec \( 5\) termes donne \( e\simeq 65/24\simeq 2.708\). Avouez que c'est déjà bien mieux.

\begin{theorem}[Définition de l'exponentielle]  \label{ThoRWOZooYJOGgR}
    Les choses que nous savons sur l'exponentielle :
    \begin{enumerate}
        \item
            Il y a unicité de la solution à l'équation différentielle
            \begin{subequations}    \label{subeqBKJNooJQtbBD}
        \begin{numcases}{}
            y'=y\\
            y(0)=1.
        \end{numcases}
    \end{subequations}
    \item
        L'équation différentielle \eqref{subeqBKJNooJQtbBD} possède une solution donnée par la série entière\nomenclature[Y]{\( \exp\)}{exponentielle}
        \begin{equation}    \label{EqUARSooKXnQxu}
        \exp(x)=\sum_{k=0}^{\infty}\frac{ x^k }{ k! }
    \end{equation}
\item
    Cette solution est une bijection \( y\colon \eR\to \mathopen] 0 , \infty \mathclose[\).
    \item   \label{ItemYTLTooSnfhOu}
        La fonction \( y\) ainsi définie est de classe \(  C^{\infty}\).
\item
    Elle est également donnée par la formule
    \begin{equation}
        \exp(x)=e^x
    \end{equation}
    où \( e\) est définit par \( e=\exp(1)\).
\item
    Elle vérifie
    \begin{equation}        \label{EQooVFXUooBfwjJY}
        e^{a+b}= e^{a} e^{b}
    \end{equation}
    \end{enumerate}
\end{theorem}
Nous nommons \defe{exponentielle}{exponentielle} cette fonction.

\begin{proof}
    \begin{enumerate}
        \item
            C'est la proposition \ref{PropDJQSooYIwwhy}.
        \item 
            C'est le théorème \ref{ThoKRYAooAcnTut}.
        \item
            Le rayon de convergence de la série \eqref{EqUARSooKXnQxu} est infini (théorème \ref{ThoKRYAooAcnTut}); elle est donc définie sur \( \eR\). Le fait que ce soit une bijection est dû au fait qu'elle est strictement croissante (proposition \ref{PropTLECooEiLbPP}) ainsi qu'aux limite \eqref{EqLOIUooHxnEDn}.
        \item
            Vu que \( y=y'\), \( y\) est dérivable. Mais comme \( y'\) est alors égale à une fonction dérivable, \( y'\) est dérivable. En dérivant l'égalité \( y'=y\) nous obtenons \( y''=y'\) et le jeu continue.
        \item
            C'est la proposition \ref{PropCELWooLBSYmS}.
        \item
            C'est la proposition \ref{PROPooGGUIooExVHPM}\ref{ITEMooMPSUooWQpVQJ}.
    \end{enumerate}
\end{proof}

\begin{example}[Un endomorphisme sans polynôme annulateur\cite{RombaldiO}]     \label{ExooLRHCooMYLQTU}
    l'exponentielle permet de donner un exemple d'un endomorphisme n'ayant pas de polynôme annulateur\footnote{Voir la définition \ref{DefooOHUXooNkPWaB} et ce qui suit.} : l'endomorphisme de dérivation
    \begin{equation}
        \begin{aligned}
            D\colon C^{\infty}(\eR,\eR)&\to  C^{\infty}(\eR,\eR) \\
            f&\mapsto f' 
        \end{aligned}
    \end{equation}
    n'a pas de polynôme annulateur. En effet supposons que \( P=\sum_{k=0}^{p}a_kX^k\) en soit un, et considérons les fonction \( f_{\lambda}\colon t\mapsto  e^{\lambda t}\). Nous avons
    \begin{equation}
            0=P(D)f_{\lambda}
            =\sum_ka_kD^k(f_{\lambda})
            =\sum_ka_k\lambda^kf_{\lambda}
            =P(\lambda)f_{\lambda}.
    \end{equation}
    Par conséquent \( \lambda\) est une racine de \( P\) pour tout \( \lambda\in \eR\). Cela implique que \( P=0\).
    
    D'ailleurs si on y pense bien, cet exemple n'est qu'un habillage de l'exemple \ref{ExooDTUJooIMqSKn}.
\end{example}


\begin{propositionDef}    \label{DEFooELGOooGiZQjt}
            L'application \(\exp\colon \eR\to \mathopen] 0 , \infty \mathclose[\) est une bijection.  L'application réciproque
            \begin{equation}
                \ln\colon \mathopen] 0 , \infty \mathclose[\to \eR
            \end{equation}
            est le \defe{logarithme}{logarithme!sur les réels positifs}.
\end{propositionDef}

\begin{proof}
Le fonction exponentielle est dérivable, toujours strictement positive, donc strictement croissante. Les limites en \( \pm \infty\) sont \( 0\) et \( +\infty\). Le théorème des valeurs intermédiaires \ref{ThoValInter} nous dit que c'est une bijection. En effet, l'injectivité est la stricte croissance. En ce qui concerne la surjection, soit \( y\in \mathopen] 0 , \infty \mathclose[\). Vu que la limite en \( -\infty\) est zéro, il existe \( A\in \eR\) tel que \( \exp(x)<y\) pour tout \( x<A\), et de la même façon, il existe \( B\in \eR\) tel que \( \exp(x)>y\) pour tout \( x>B\). Si \( a<A\) et \( b>B\) alors \( \exp(a)<y\) et \( \exp(b)>y\), donc \( y\) est dans l'image de \( \mathopen[ a , b \mathclose]\) par l'exponentielle.
\end{proof}

\begin{proposition}\label{ExZLMooMzYqfK}
    Quelque propriétés du logarithme.
    \begin{enumerate}
        \item
            Le logarithme est une application dérivable et strictement croissante.
        \item
            Le logarithme est la primitive de \( x\mapsto\frac{1}{ x }\) qui s'annule en \( x=1\).
    \end{enumerate}
\end{proposition}

\begin{proof}
    Elle est donc bijective, d'inverse continue et dérivable par le théorème \ref{ThoKBRooQKXThd} et la proposition \ref{PropMRBooXnnDLq}. 

    La dérivée de la fonction logarithme peut être calculée en utilisant la formule \eqref{EqWWAooBRFNsv}, mais aussi de façon plus piettone en écrivant l'expression suivante, valable pour tout \( x\in \eR\) :
    \begin{equation}
        \ln\big( \exp(x) \big)=x,
    \end{equation}
    que nous pouvons dériver en utilisant le théorème de dérivation des fonctions composées :
    \begin{equation}
        \ln'\big( \exp(x) \big)\exp'(x)=1.
    \end{equation}
    Mais \( \exp'(x)=x\), donc
    \begin{equation}
        \ln'(y)=\frac{1}{ y }
    \end{equation}
    pour tout \( y\) dans l'image de \( \exp\), c'est à dire pour tout \( y\) dans l'ensemble de définition de \( \ln\).

    Par ailleurs, \( \exp(0)=1\) donc
    \begin{equation}
        \ln(1)=\ln\big( \exp(0) \big)=0.
    \end{equation}

    En ce qui concerne l'unicité d'une primitive s'annulant en \( x=1\), c'est le corollaire \ref{CorZeroCst}.
\end{proof}

\begin{lemma}
Si \( u\colon \eR\to \mathopen] 0 , \infty \mathclose[\) est dérivable alors \( \ln(u)'=\dfrac{ u' }{ u }\).
\end{lemma}

\begin{proof}
    Cela est une conséquence du théorème de dérivation des fonctions composées : si \( g(x)=\ln(u(x))\) alors
    \begin{equation}
        g'(x)=\ln'\big( u(x) \big)u'(x)=\frac{1}{ u(x) }u'(x).
    \end{equation}
\end{proof}

\begin{example}[Primitive du logarithme]\label{primln}
    La primitive de la fonction logarithme définie en \ref{DEFooELGOooGiZQjt} nous offre un bon moment d'intégration par partie.

    Trouver la primitive de la fonction \( x\mapsto \ln(x)\). Pour calculer
    \begin{equation}
        \int\ln(x)dx
    \end{equation}
    nous écrivons \( \ln(x)=1\times \ln(x)\) et nous posons \( u'=1\) et \( v=\ln(x)\), c'est à dire
    \begin{equation}
        \begin{aligned}[]
            u'&=1&v=\ln(x)\\
            u&=x&v'=\frac{1}{ x }.
        \end{aligned}
    \end{equation}
    La formule d'intégration par parties \eqref{EQooKISBooQvGMQT} donne donc 
    \begin{equation}
        \int \ln(x)=x\ln(x)-\int x\times \frac{1}{ x }=x\ln(x)-\int 1=x\ln(x)-x+C, \qquad C\in\eR.
    \end{equation}
    Il est facile de vérifier par un petit calcul que
    \begin{equation}
        \big( x\ln(x)-x \big)'=\ln(x).
    \end{equation}
\end{example}

\begin{lemma}   \label{LemPEYJooEZlueU}
Si \( a,b\in\mathopen] 0 , \infty \mathclose[\) alors
    \begin{equation}
        \ln(ab)=\ln(a)+\ln(b)
    \end{equation}
    et
    \begin{equation}    \label{EqOOZGooOWkGlA}
        \ln\left( \frac{1}{ b } \right)=-\ln(b).
    \end{equation}
\end{lemma}

\begin{proof}
    Nous posons \( f(x)=\ln(ax)\) qui est une fonction dérivable. Alors \( f'(x)=\frac{ a }{ ax }=\frac{1}{ x }\). Cette fonction \( f\) est donc une primitive de \( \frac{1}{ x }\) et il existe une constante \( K\) telle que
    \begin{equation}
        f(x)=\ln(x)+K.
    \end{equation}
    Vu que \( \ln(1)=0\) nous avons \( K=f(1)= \ln(a)\). Donc
    \begin{equation}
        \ln(ax)=\ln(x)+\ln(a).
    \end{equation}

    En ce qui concerne la seconde formule à démontrer, nous avons
    \begin{equation}
        \ln(1)=\ln\left( \frac{1}{ b }b \right)=\ln\left( \frac{1}{ b } \right)+\ln(b).
    \end{equation}
    Étant donné que $\ln(1)=0$ nous en déduisons la formule \eqref{EqOOZGooOWkGlA}.
\end{proof}

\begin{example}
    Montrons que la fonction\footnote{Pour la définition du logarithme, c'est la définition \ref{DEFooELGOooGiZQjt}.}
    \begin{equation}
        \begin{aligned}
            f\colon \eR_+\setminus\{ 0,1 \}&\to \eR \\
            x&\mapsto \frac{ \ln(x) }{ x-1 } 
        \end{aligned}
    \end{equation}
    admet un prolongement \( C^{\infty}\) sur \( \eR_+\setminus\{ 0 \}\).

    Nous allons étudier la fonction
    \begin{equation}
        f(x)=\frac{ \ln(1+x) }{ x }
    \end{equation}
    autour de \( x=0\). Le logarithme ne pose pas de problèmes à développer dans un voisinage :
    \begin{subequations}
        \begin{align}
            f(x)&=\frac{1}{ x }\sum_{n=1}^{\infty}\frac{ (-1)^{n+1} }{ n }x^n\\
            &=\sum_{n=1}^{\infty}\frac{ (-1)^{n+1} }{ n }x^{n-1}\\
            &=\sum_{n=0}^{\infty}\frac{ (-1)^k }{ k+1 }x^k.
        \end{align}
    \end{subequations}
    Cette série a un rayon de convergence égal à \( 1\), et donc définit sans problèmes une fonction \( C^{\infty}\) dans un voisinage de \( x=0\). Notons que par convention \( x^0=1\) même si \( x=0\).
\end{example}

\begin{example}     \label{EXooKNTPooKiRExX}
    Montrons que pour tout \( x\in\mathopen] -1 , 1 \mathclose[\) nous avons
    \begin{equation}        \label{EqweEZnV}
        -\ln(1-x)=\sum_{n=1}^{\infty}\frac{ x^n }{ n }.
    \end{equation}
    Nous calculerons ensuite la valeur de la série
    \begin{equation}    \label{EqKUQmOZ}
        \sum_{n=1}^{\infty}\frac{ (-1)^n }{ n }.
    \end{equation}

    La série \eqref{EqKUQmOZ} serait \( f(-1)=-\ln(2)\) où \( f\) est la série de fonctions \eqref{EqweEZnV}. Nous utilisons le théorème de convergence radiale d'Abel (théorème \ref{ThoLUXVjs}) pour justifier cette réponse :
    \begin{equation}
        \sum_n\frac{ (-1)^n }{ n }
    \end{equation}
    converge.
\end{example}

%+++++++++++++++++++++++++++++++++++++++++++++++++++++++++++++++++++++++++++++++++++++++++++++++++++++++++++++++++++++++++++
\section{Vitesses de $x^{\alpha}$, de l'exponentielle et du logarithme}
%+++++++++++++++++++++++++++++++++++++++++++++++++++++++++++++++++++++++++++++++++++++++++++++++++++++++++++++++++++++++++++

\begin{lemma}   \label{LemSYHKooUiSMFJ}
    Pour tout \( \alpha>0\), il existe \( N\) tel que \( \ln(n)\leq n^{\alpha}\) pour tout \( n\geq N\).
\end{lemma}

\begin{proof}
En effet, nous avons
\begin{equation}
    \lim_{x\to\infty} \frac{ x^{\alpha} }{ \ln(x) }=\lim_{x\to\infty} \frac{ \alpha x^{\alpha-1} }{ 1/x }=\lim_{x\to\infty} \alpha x^{\alpha}=\infty
\end{equation}
quand $\alpha>0$. 
\end{proof}
Cela tient également lorsque nous considérons $\ln(x)^p$ au lieu de $\ln(x)$. De cela, nous disons que le logarithme croit moins vite que n'importe quel polynôme. 

\begin{lemma}
    L'exponentielle croit plus vite que tout polynôme, et plus vite que que logarithme :
    \begin{equation}        \label{EqExpDecrtPlusVite}
        \lim_{t\to\infty} e^{-t}(\ln t)^{n}t^{\alpha}=0
    \end{equation}
    pour tout $n$ et pour tout $\alpha$.
\end{lemma}

\begin{lemma}       \label{LemVKDKooEftNzG}
    Nous avons aussi la limite utile suivante 
    \begin{equation}
        \lim_{n\to \infty} n^{\alpha}a^n
    \end{equation}
    pour tout \( \alpha>0\) et \( a<1\).
\end{lemma}

\begin{proof}
    En passant à l'exponentielle, pour chaque \( n\) nous avons
    \begin{equation}        \label{EqLKLQooLIlWgm}
        n^{\alpha}a^n= e^{\alpha\ln(n)+n\ln(a)}.
    \end{equation}
    Ce qui est dans l'exponentielle est
    \begin{equation}
        \alpha\ln(n)+n\ln(a)=n\big(\alpha \frac{ \ln(n) }{ n }+\ln(a) \big).
    \end{equation}
    Dans la parenthèse, \( \ln(a)<0\) et \( \frac{ \ln(n) }{ n }\to 0\). Donc ce qui est dans l'exponentielle \eqref{EqLKLQooLIlWgm} tend vers \( -\infty\) et au final l'expression demandée tend vers zéro.
\end{proof}

\begin{proposition} \label{PropBQGBooHxNrrf}
    Pour tout polynôme \( P\) et pour tout \( a>0\) la fonction \( f(x)=P(x) e^{-ax}\) est intégrable\footnote{Définition \ref{DefTCXooAstMYl}.} sur \( \mathopen[ 0 , \infty [\).
\end{proposition}

\begin{proof}
    Nous avons \( f(x)=P(x) e^{-ax/2} e^{-ax/2}\), et par la vitesse comparée des exponentielles et polynômes, pour un certain \( M>0\) nous pouvons affirmer que \( P(x) e^{-ax/2}<1\) sur \( \mathopen[ M , 0 [\). Dès lors
        \begin{equation}
            | f(x) |< e^{-ax/2},
        \end{equation}
        qui est intégrable.
\end{proof}

\begin{theorem} \label{ThonfVruT}
    Soit \( P\), l'ensemble des nombres premiers. Alors la somme \( \sum_{p\in P}\frac{1}{ p }\) diverge et plus précisément,
    \begin{equation}
        \sum_{\substack{p\leq x\\p\in P}}\frac{1}{ p }\geq \ln(\ln(x))-\ln(2).
    \end{equation}
\end{theorem}
\index{nombre!premier}
\index{convergence!rapidité}
\index{série!numérique}

\begin{proof}
    Nous posons
    \begin{equation}
        S_x=\{  q\leq x\text{ avec } q\text{ sans facteurs carrés} \}
    \end{equation}
    et
    \begin{equation}
        P_x=\{ p\in P\tq p\leq x \}.
    \end{equation}
    Si
    \begin{equation}
        K_x=\{  (q,m)\text{ tels que } q\text{ n'a pas de facteurs carrés et } qm^2\leq x \},
    \end{equation}
    alors nous avons
    \begin{equation}
        K_x=\bigcup_{q\in S_x}\bigcup_{m\leq \sqrt{x/q}}(q,m).
    \end{equation}
    Par définition et par le lemme \ref{LemheKdsa} nous avons aussi
    \begin{equation}
        \{ n\leq x \}=\{ qm^2\tq (q,m)\in K_x \}.
    \end{equation}
    Tout cela pour décomposer la somme
    \begin{equation}        \label{EqpoJpuC}
        \sum_{n\leq x}\frac{1}{ n }=\sum_{q\in S_x}\sum_{m\leq\sqrt{x/q}}\frac{1}{ m^2 }\leq \sum_{q\in S_x}\frac{1}{ q }\underbrace{\sum_{m\geq 1}\frac{1}{ m^2 }}_{=C}.
    \end{equation}
    Nous avons aussi
    \begin{subequations}
        \begin{align}
            \prod_{p\in P_x}\left( 1+\frac{1}{ p } \right)&=1+\sum_{p\in P_x}\frac{1}{ p }+\sum_{\substack{p,q\in P_x\\p<q}}\frac{1}{ pq }+\sum_{\substack{p,q,r\in P_x\\p<q<r}}\frac{1}{ pqr }+\ldots\\
            &\geq 1+\sum_{p\in P_x}\frac{1}{ p }+\sum_{\substack{p,q\in P_x\\pq\leq x}}\frac{1}{ pq }+\sum_{\substack{p,q,r\in P_x\\pqr\leq x}}\frac{1}{ pqr }+\ldots
        \end{align}
    \end{subequations}
    Les sommes sont finies. Les sommes s'étendent sur toutes les façons de prendre des produits de nombres premiers distincts de telle sorte de conserver un produit plus petit que \( x\); c'est à dire que les sommes se résument en une somme sur les éléments de \( S_x\) :
    \begin{equation}        \label{EqooilOz}
        \exp\left( \sum_{p\in P_x}\frac{1}{ p } \right)\geq\prod_{p\in P_x}\left( 1+\frac{1}{ p } \right)\geq \sum_{q\in S_x}\frac{1}{ q }.
    \end{equation}
    La première inégalité est simplement le fait que \( 1+u\leq e^u\) si \( u\geq 0\) (directe de la définition \ref{ThoRWOZooYJOGgR}). Les inégalités suivantes proviennent du fait que le logarithme est une primitive de la fonction inverse (proposition \ref{ExZLMooMzYqfK}) :
    \begin{equation}
        \ln(x)\leq \sum_{n\geq x}\int_{n}^{n+1}\frac{dt}{ t }\leq \sum_{n\geq x}\frac{1}{ n }.
    \end{equation}
    Nous prolongeons ces inégalités avec les inégalités \eqref{EqpoJpuC} et \eqref{EqooilOz} :
    \begin{equation}
        \ln(x)\leq \sum_{n\geq x}\frac{1}{ n }\leq C\sum_{q\in S_x}\frac{1}{ q }\leq C\leq \exp\left( \sum_{p\in P_x}\frac{1}{ p } \right).
    \end{equation}
    En passant au logarithme,
    \begin{equation}
        \ln\big( \ln(x) \big)\leq\ln(C)+\sum_{p\in P_x}\frac{1}{ p }.
    \end{equation}
    Ceci montre la divergence de la série de droite. Nous cherchons maintenant une borne pour \( C\). Pour cela nous écrivons
    \begin{subequations}
        \begin{align}
            \sum_{n=1}^N\frac{1}{ n^2 }&\leq 1+\sum_{n=2}\frac{1}{ n(n-1) }\\
            &=1+\sum_{n=2}^N\left( \frac{1}{ n-1 }-\frac{1}{ n } \right)\\
            &=1+1-\frac{1}{ N }\\
            &\leq 2.
        \end{align}
    \end{subequations}
    Donc \( C\leq 2\).
\end{proof}
Ce théorème prend une nouvelle force en considérant le théorème de Müntz \ref{ThoAEYDdHp} qui dit qu'alors l'ensemble \( \Span\{ x^p\tq  p\text{ est premier} \}\) est dense dans les fonctions continues sur \( \mathopen[ 0 , 1 \mathclose]\) muni de la norme uniforme ou \( \| . \|_2\).

%+++++++++++++++++++++++++++++++++++++++++++++++++++++++++++++++++++++++++++++++++++++++++++++++++++++++++++++++++++++++++++ 
\section{Trigonométrie hyperbolique}
%+++++++++++++++++++++++++++++++++++++++++++++++++++++++++++++++++++++++++++++++++++++++++++++++++++++++++++++++++++++++++++

\begin{definition}
    Les fonction \defe{sinus hyperbolique}{sinus!hyperbolique} et \defe{cosinus hyperbolique}{cosinus!hyperbolique} sont les fonctions définies sur $\eR$ par les formules suivantes :
    \begin{subequations}
        \begin{align}
            \cosh(x)&=\frac{  e^{x}+ e^{-x} }{2}\\
            \sinh(x)&=\frac{  e^{x}- e^{-x} }{2}
        \end{align}
    \end{subequations}
\end{definition}

Leurs principales propriétés sont :
\begin{enumerate}
    \item
        \( \cosh^2(x)-\sinh^2(x)=1\)
    \item
        \( \cosh'(x)=\sinh(x)\) 
    \item
        \( \sinh'(x)=\cosh\).
\end{enumerate}

Les représentations graphiques sont ceci :
\begin{center}
   \input{auto/pictures_tex/Fig_UNVooMsXxHa.pstricks}
\end{center}

La \defe{tangente hyperbolique}{tangente hyperbolique} est donnée par le quotient
\begin{equation}
    \tanh(x)=\frac{ \sinh(x) }{ \cosh(x) }.
\end{equation}

%Les fonction réciproques de $\sinh$, $\cosh$ et $\tanh$ sont traitées dans les exercices.



%+++++++++++++++++++++++++++++++++++++++++++++++++++++++++++++++++++++++++++++++++++++++++++++++++++++++++++++++++++++++++++ 
\section{Dénombrement des solutions d'une équation diophantienne}
%+++++++++++++++++++++++++++++++++++++++++++++++++++++++++++++++++++++++++++++++++++++++++++++++++++++++++++++++++++++++++++

\begin{theorem}[\cite{fJhCTE,NHXUsTa}] \label{ThoDIDNooUrFFei}
    Soient des entiers naturels premiers dans leur ensemble\footnote{Définition \ref{DefZHRXooNeWIcB}.} \( \alpha_1,\ldots, \alpha_p\) et l'équation
    \begin{equation}
        \alpha_1n_1+\cdots +\alpha_pn_p=n
    \end{equation}
    pour les naturels \( n_i\) où \( n\) est un naturel donné. Nous notons \( S_n\) le nombre de solutions de cette équation. Alors :
    \begin{enumerate}
        \item
            Il existe un algorithme (en temps fini) pour calculer \( S_n\) en fonction des \( \alpha_i\) et de \( n\).
        \item
            Nous avons le comportement asymptotique
            \begin{equation}
                S_n\sim\frac{1}{ \alpha_1\ldots\alpha_p }\frac{ n^{p-1} }{ (p-1)! }.
            \end{equation}
    \end{enumerate}
\end{theorem}

\begin{proof}
    Pour \( | z |<1\) dans \( \eC\), utilisant le lemme \ref{LemPQFDooGUPBvF}, nous écrivons le développement
    \begin{equation}
        F(z)=\prod_{i=1}^p\frac{1}{ 1-z^{\alpha_i} }=\prod_{i=1}^p\sum_{n\geq 0}z^{n\alpha_i}.
    \end{equation}
    Nous allons maintenant à la pêche au terme de degré \( k\) dans ce produit de sommes en utilisant \( p\) fois le produit de Cauchy de la formule \eqref{EqFPGGooDQlXGe}. Nous avons
    \begin{equation}
        F(z)=\sum_{k\geq 0}\left( \sum_{n_1\alpha_1+\cdots +n_p\alpha_p=n}1 \right)z^k=\sum_{k\geq 0}S_kz^k.
    \end{equation}
    
    La technique pour déterminer la valeur de \( S_n\) est alors de développer \( F(z)\) en série de façon un peu explicite et d'identifier le coefficient de \( z^n\) parce que nous venons de voir que ce coefficient est \( S_n\). Nous commençons par une décomposition en éléments simples, expliquée autour de l'équation \eqref{EqDWYBooJIMBAt} :
    \begin{equation}
        \frac{1}{ 1-z^{\alpha_i} }=\sum_{\alpha\in U_{\alpha_i}}\frac{ A_{\omega,i} }{ \omega-z }.
    \end{equation}
    où \( U_{\alpha_i}\) est le groupe des racines \( \alpha_i\)\ieme de l'unité décrit en \ref{SecGJOLooWdMYVl}. La raison de ce développement est que, comme mentionné dans le lemme \ref{LemKYGBooAwpOHD}, \( \prod_{\omega\in\gU_{\alpha_i}}(z-\omega)=z^{\alpha_1}-1\). Lorsque nous effectuons la somme, le dénominateur commun est donc bien\footnote{Pour le signe, c'est ajustable avec le signe de \( A_{\omega,i}\).} \( 1-z^{\alpha_i}\).
    En récrivant le produit :
    \begin{equation}
        F(z)=\prod_{i=1}^{p}\frac{1}{ 1-z^{\alpha_i} }=\prod_{i=1}^p\sum_{\omega\in U_{\alpha_i}}\frac{ A_{\omega,i} }{ \omega-z }
    \end{equation}
    Les coefficients \( A_{\omega,i}\) sont calculables explicitement, en temps fini.

    Vu que \( 1\) est dans tous les \( \gU_{\alpha_i}\), le produit fait intervenir au dénominateur des puissances de \( (1-z)\) jusqu'à la puissance \( p\). Les autres racines de l'unité appartiennent au maximum à \( p-1\) des groupes \( \gU_{\alpha_i}\) parce que les nombres \( \alpha_i\) sont premiers dans leur ensemble, voir la proposition \ref{PropFDDHooEyYxBC}.

    La fonction \( F\) peut alors s'écrire sous la forme
    \begin{equation}    \label{EqLISXooSlwIWD}
        F(z)=\frac{ A }{ (1-z)^p }+G(z)
    \end{equation}
    où \( G(z)\) est une somme de termes de la forme
    \begin{equation}
        \frac{ a_{i,1} }{ 1-\omega_i }+\cdots +\frac{ a_{i,p} }{ (1-\omega_i)^{p-1} }
    \end{equation}
    où les \( \omega_i\) sont les racines \( \alpha_i\)\ieme de l'unité et \( a_{k,r}\) sont des nombres complexes. Trouvons \( A\). D'abord grâce au lemme \ref{LemISPooHIKJBU}\ref{ItemLTBooAcyMtNii} nous avons
    \begin{equation}
        F(z)(1-z)^p=\prod_{l=1}^p\frac{ 1-z }{ 1-z^{\alpha_i} }=\prod_{i=1}^p\frac{ 1 }{ 1+z+\cdots +z^{\alpha_i-1} },
    \end{equation}
    et donc 
    \begin{equation}
        \lim_{z\to 1}F(z)(1-z)^p=\prod_{i=1}^p\frac{1}{ \alpha_i }.
    \end{equation}
    Mais vu ce que contient \( G(z)\), nous avons aussi \( \lim_{z\to 1}F(z)=A\). Nous avons donc déjà déterminé \( A=\frac{1}{  \alpha_1\ldots\alpha_p }\).

    Pour la suite nous avons besoin des développements du lemme \ref{LemPQFDooGUPBvF}. Nous utiliserons en particulier celle-ci :
    \begin{equation}
        \frac{1}{ (\omega-z)^k }=\frac{1}{ (k-1)! }\sum_{s=0}^{\infty}\omega^{-s-1-k}\frac{ (s+k-1)! }{ s! }z^s.
    \end{equation}
    En particulier le module du coefficient de \( z^n\) là dedans est : \(  \frac{(n+k-1)! }{ n!(k-1)! } \). Dans la partie \( G\) de la décomposition \eqref{EqLISXooSlwIWD}, \( k\) est majoré par \( p-2\) et la dépendance en \( n\) est donc au maximum du type
    \begin{equation}
        \frac{ (n+p-2)! }{ n!(p-2)! }\sim  \frac{ n^{n+p-2} }{ n^n(p-2)! }=\frac{ n^{p-2} }{ (p-2)! }.
    \end{equation}
    Dans le premier terme par contre, il y a des termes jusqu'à \( k=p\). Le terme dominant est alors en \( \frac{ n^{p-1} }{ (p-1)! }\) et son coefficient est \( A\) qui est déjà calculé. Au final le terme dominant du coefficient de \( z^n\) dans \( F(z)\) est
    \begin{equation}
        S_n\sim \frac{ A }{ (p-1)! }n^{p-1}=\frac{1}{ \alpha_1\ldots \alpha_p }\frac{ n^{p-1} }{ (p-1)! }.
    \end{equation}
\end{proof}

\begin{example}
    Pour \( p=1\), l'équation est \( \alpha x=n\), qui possède au maximum une solution, quel que soit \( n\). Et de plus pour avoir une solution il faut et suffit que \( \alpha\) divise \( n\), c'est à dire que \( n\) soit un multiple de \( \alpha\). Il n'y a que un nombre sur \( \alpha\) à être multiple de \( \alpha\). D'où le comportement en \( \frac{1}{ \alpha }\).

    Pour \( p=2\), c'est l'équation \eqref{EqTOVSooJbxlIq} déjà étudiée. Il y a une famille à un paramètre de solutions dont seulement un certain nombre sont positives. A priori, le nombre de solutions positives croît linéairement en \( n\).
\end{example}
