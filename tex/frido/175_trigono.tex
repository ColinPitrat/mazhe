% This is part of (everything) I know in mathematics
% Copyright (c) 2011-2013,2016-2017
%   Laurent Claessens
% See the file fdl-1.3.txt for copying conditions.

%+++++++++++++++++++++++++++++++++++++++++++++++++++++++++++++++++++++++++++++++++++++++++++++++++++++++++++++++++++++++++++
\section{Isométries de l'espace euclidien}
%+++++++++++++++++++++++++++++++++++++++++++++++++++++++++++++++++++++++++++++++++++++++++++++++++++++++++++++++++++++++++++

Nous considérons l'espace affine euclidien \( A=\affE_n(\eR)\) modelé sur \( \eR^n\) avec sa métrique usuelle. Un premier grand résultat sera le théorème \ref{ThoDsFErq} qui dira que les isométries de cet espace sont des applications linéaires.

%--------------------------------------------------------------------------------------------------------------------------- 
\subsection{Structure du groupe  \texorpdfstring{\( \Isom(\eR^n)\)}{Isom(Rn)} }
%---------------------------------------------------------------------------------------------------------------------------

\begin{example}
    La forme quadratique \( q(x)=x_1^2+x_2^2\) donne la norme euclidienne. La forme bilinéaire associée est \( b(x,y)=x_1y_1+x_2y_2\), qui est le produit scalaire usuel.
\end{example}

Il ne faudrait pas déduire trop vite que la formule \( \| x \|^2=q(x)\) donne une norme dès que \( q\) est non dégénérée. En effet \( q\) peut ne pas être définie positive. La forme \( q(x)=x_1^2-x_2^2\) prend des valeurs positives et négatives. A fortiori \( d(x,y)=q(x-y)\) ne donne pas toujours une distance.

\begin{definition}      \label{DEFooECTUooRxBhHf}
    Une \defe{isométrie}{isométrie!de forme quadratique} pour la forme quadratique \( q\) est une application bijective \( f\colon V\to V\) telle que \( q(x-y)=q\big( f(x)-f(y) \big)\). Dans les cas où \( q\) donne une distance, alors c'est une isométrie au sens usuel.
\end{definition}

\begin{lemma}   \label{LemewGJmM}
    Soit \( q\) une forme quadratique et \( b\) la forme bilinéaire associée par le lemme \ref{LEMooLKNTooSfLSHt}.  Pour une application bijective \( f\colon E\to E\) telle que \( f(0)=0\), les conditions suivantes sont équivalentes: 
    \begin{enumerate}
        \item
            \( b\big( f(x),f(y) \big)=b(x,y)\) pour tout \( x,y\in E\);
        \item
            \( q\big( f(x)-f(y) \big)=q(x-y)\) pour tout \( x,y\in E\).
    \end{enumerate}
\end{lemma}

\begin{proof}
    Dans le sens direct, en posant \( x=y\) nous trouvons tout de suite \( q(f(x))=q(f)\); ensuite en utilisant la distributivité de \( b\),
    \begin{subequations}
        \begin{align}
            q\big( f(x)-f(y) \big)&=b\big( f(x)-f(y),f(x)-f(y) \big)\\
            &=q\big( f(x) \big)-2b\big( f(x),f(y) \big)+q\big( f(y) \big)\\
            &=q(x)+q(y)-2b(x,y)\\
            &=q(x-y).
        \end{align}
    \end{subequations}
    
    Dans l'autre sens, nous commençons par remarquer que l'hypothèse \( f(0)=0\) donne \( q(x)=q\big( f(x) \big)\). Ensuite nous utilisons l'identité de polarisation \eqref{EqMrbsop} :
    \begin{subequations}
        \begin{align}
            b\big( f(x),f(y) \big)&=\frac{ 1 }{2}\big[ q\big( f(x) \big)+q\big( f(y) \big)-q\big( f(x-y) \big) \big]\\
            &=\frac{ 1 }{2}\big[ q(x)+q(y)-q(x-y) \big]\\
            &=b(x,y).
        \end{align}
    \end{subequations}
\end{proof}

\begin{theorem}[\cite{ooQFKAooFnllQU}]     \label{ThoDsFErq}
    Soit \( f\colon E\to E\) une bijection telle que
    \begin{equation}
        q(x-y)=q\big( f(x)-f(y) \big)
    \end{equation}
    pour tout \( x,y\in E\). Alors
    \begin{enumerate}
        \item
            si \( f(0)=0\), alors \( f\) est linéaire;
        \item
            si \( f(0)\neq 0\) alors \( f\) est affine\footnote{Lemme \ref{LEMooZZAIooOMiayy}.}
    \end{enumerate}
\end{theorem}

\begin{proof}
    Si \( f(0)=0\), nous savons par le lemme \ref{LemewGJmM} que \( b\big( f(x),f(y) \big)=b(x,y)\). Soit \( z\in E\); étant donné que \( f\) est bijective nous pouvons considérer l'élément \( f^{-1}(z)\in E\) et calculer
    \begin{subequations}
        \begin{align}
            b\big( f(x+y),z \big)&=b\big( f(x+y),f(f^{-1}(z)) \big)\\
            &=b(x+y,f^{-1}(z))\\
            &=b(x,f^{-1}(z))+b(y,f^{-1}(z))\\
            &=b(f(x),z)+b(f(y),z)\\
            &=b\big( f(x)+f(y),z \big),
        \end{align}
    \end{subequations}
    donc \( f(x+y)=f(x)+f(y)\) par le lemme \ref{LemyKJpVP}. 

    De la même façon on trouve \( b\big( f(\lambda x),z \big)=b\big( \lambda f(x),z \big)\) qui prouve que \( f(\lambda x)=\lambda f(x)\) et donc que \( f\) est linéaire.

    Si \( f(0)\neq 0\), alors nous posons \( g(x)=f(x)-f(0)\) qui vérifie \( g(0)=0\) et
    \begin{equation}
        q\big( g(x)-g(y) \big)=q\big( f(x)-f(0)-f(y)+f(0) \big)=q(x-y).
    \end{equation}
    Nous pouvons donc appliquer le premier point à \( g\), déduire que \( g\) est linéaire et donc que \( f\) est affine.
\end{proof}

\ifbool{isEverything}{
\begin{remark}
    Des preuves alternatives.
    \begin{enumerate}
        \item
            En utilisant un peut plus d'indices et un peu plus de mots comme «tenseurs», peut être trouvée  \href{http://physics.stackexchange.com/questions/12664/proving-that-interval-preserving-transformations-are-linear}{ici}. Le fait que la preuve donnée soit tensorielle me fait penser que le résultat peut encore être généralisé.
        \item
            Et encore une autre preuve, utilisant des techniques de groupes de Lie sera la proposition \ref{PROPooDVIWooAFDNPy}.
    \end{enumerate}
\end{remark}
}
{}

Nous pouvons maintenant particulariser tout cela au cas de \( \eR^n\) muni du produit scalaire usuel et de la norme associée pour voir quel résultat nous avons à peine prouvé.

\begin{lemma}[\cite{ooYPVPooYGSlNU}]        \label{LEMooJPYZooHETCqt}
    Une isométrie est bijective (nous sommes en dimension finie).
\end{lemma}

\begin{proof}
    Si \( f\colon E\to E\) est une isométrie, elle est linéaire par le théorème \ref{ThoDsFErq}. Elle vérifie également \( \| f(x) \|=\| x \|\), et donc \( f(x)=0\) si et seulement si \( x=0\), c'est à dire que \( f\) est injective. Elle est alors bijective par le corollaire \ref{CORooCCXHooALmxKk} du théorème du rang.
\end{proof}

Nous notons ici \( T(n)\) le groupe des translations sur \( \eR^n\). Un élément de \( T(n)\) est une translation \( \tau_v\) donnée par un vecteur \( v\) et agissant sur \( \eR^n\) par
\begin{equation}
    \begin{aligned}
        \tau_v\colon \eR^n&\to \eR^{n} \\
        x&\mapsto x+v. 
    \end{aligned}
\end{equation}
Ce groupe est isomorphe au groupe abélien \( (\eR^n,+)\), et nous allons souvent identifier \( \tau_v\) à \( v\).

Si vous ne voulez pas savoir ce qu'est un produit semi-direct de groupes, vous pouvez lire seulement le point \ref{ITEMooLLUIooIGsknv} du théorème suivant, et passer directement à la remarque \ref{REMooLUEZooIwvTqu}.
\begin{theorem}     \label{THOooQJSRooMrqQct}
    Un peu de structure sur \( \Isom(\eR^n)\).
    \begin{enumerate}
        \item       \label{ITEMooLLUIooIGsknv}
            L'application
            \begin{equation}
                \begin{aligned}
                    \psi\colon T(n)\times \gO(n)&\to \Isom(\eR^n) \\
                    (v,\Lambda)&\mapsto \tau_v\circ\Lambda 
                \end{aligned}
            \end{equation}
            est une bijection. Ici,  \( T(n)\) est le groupe des translations de \( \eR^n\).
        \item
            Un couple \( (v,\Lambda)\in T(n)\times\SO(n)\) agit sur \( x\in \eR^n\) par
            \begin{equation}
                (v,\Lambda)x=\Lambda x+v
            \end{equation}
            au sens où \( \psi(v,\Lambda)x=\Lambda x+v\).
        \item       \label{ITEMooEWSIooNKzRxB}
            En tant que groupes,
            \begin{equation}
                \Isom(\eR^n)\simeq T(n)\times_{\rho}\gO(n)
            \end{equation}
            où \( \rho\) représente l'action adjointe de \( \gO(n)\) sur \( T(n)\) et \( \times_{\rho}\) dénotes le produit semi-direct de la définition \ref{DEFooKWEHooISNQzi}.
    \end{enumerate}
\end{theorem}

\begin{proof}
    Point par point.
    \begin{enumerate}
        \item
            Prouvons que l'application proposée est injective et surjective. Notons aussi que ce point ne parle pas de structure de groupe, mais seulement d'une bijection en tant qu'ensembles.    
            \begin{subproof}
                \item[Injection]
                    Si \( \psi(v,\Lambda)=\psi(w,\Lambda')\) alors en appliquant sur \( x=0\) nous avons tout de suite \( v=w\). Et ensuite \( \Lambda=\Lambda'\) est immédiat.
                \item[Surjection]
                    Une isométrie \( g\in\Isom(\eR^n)\) est une application \( g\colon \eR^n\to \eR^n\) telle que \( d(x,y)=d\big( g(x),g(y) \big)\). Dans le cas de \( \eR^n\) cela se traduit par
                    \begin{equation}
                        \| x-y \|=\big\| g(x)-g(y) \big\|,
                    \end{equation}
                    Vu que \( x\mapsto\| x \|\) est une forme quadratique, elle tombe sous le coup du théorème  \ref{ThoDsFErq}, ce qui nous permet de dire que \( g\) est affine. Or par définition une application est affine lorsqu'elle est la composée d'une translation et d'une application linéaire.
            \end{subproof}
        \item
            C'est seulement le fait que \( (\tau_v\circ\Lambda)x=\tau_v\big( \Lambda x \big)=\Lambda(x)+v\).
        \item
            Nous allons étudier l'application
            \begin{equation}
                \psi\colon T(n)\times_{\rho}O(n)\to \Isom(\eR^n).
            \end{equation}
            \begin{subproof}
            \item[Le produit semi-direct est bien définit]
                Il faut montrer que
                \begin{equation}
                    \begin{aligned}
                        \rho\colon O(n)&\to \Aut\big( T(n) \big) \\
                        \Lambda&\mapsto \AD(\Lambda) 
                    \end{aligned}
                \end{equation}
                est correcte.

                D'abord pour \( \Lambda\in O(n)\), nous avons bien \( \rho_{\Lambda}(\tau_v)\in T(n)\) parce qu'en appliquant à \( x\in \eR^n\),
                    \begin{equation}
                        (\Lambda\tau_v\Lambda^{-1})(x)=\Lambda\big( \tau_v(\Lambda^{-1} x) \big)=\Lambda\big( \Lambda^{-1}x+v \big)=x+\Lambda(v)=\tau_{\Lambda(v)}(x).
                    \end{equation}
                    Donc \( \rho_{\Lambda}(\tau_v)=\tau_{\Lambda(v)}\).

                    De plus, \( \rho_{\Lambda}\in\Aut\big( T(n) \big)\) parce que 
                    \begin{equation}
                        \rho_{\Lambda}\big( \tau_v\circ \tau_w \big)=\rho_{\Lambda}(\tau_v)\circ\rho_{\Lambda}(\tau_v),
                    \end{equation}
                    comme on peut aisément vérifier que les deux membres sont égaux à \( \tau_{\Lambda(v+w)}\).
                \item[\( \psi\) est une bijection]
                    Cela est déjà vérifié.
                \item[\( \psi\) est un homomorphisme]
                    Nous avons d'une part
                    \begin{equation}
                        \psi\big( (v,g)(w,h) \big)=\psi\big( v\rho_g(w),gh \big)=\tau_v\circ g\circ\tau_w\circ g^{-1}\circ g\circ h=\tau_v\circ g\circ\tau_w\circ h.
                    \end{equation}
                    Et d'autre part,
                    \begin{equation}
                        \psi(v,g)\circ\psi(w,h)=\tau_v\circ g\circ \tau_w\circ h,
                    \end{equation}
                    ce qui est la même chose.
            \end{subproof}
    \end{enumerate}
\end{proof}

\begin{remark}      \label{REMooLUEZooIwvTqu}
    Notons au passage la loi de groupe sur les couples qui est donnée, pour tout \( v,v'\in \eR^n\), \( \Lambda,\Lambda'\in\SO(n)\), par
    \begin{equation}    \label{EqDiHcut}
            (v,\Lambda)\cdot(v',\Lambda')=(\Lambda v'+v,\Lambda\Lambda')
    \end{equation}
    comme le montre le calcul suivant :
    \begin{subequations}
        \begin{align}
            (v,\Lambda)\cdot(v',\Lambda')x&=(v,\Lambda)(\Lambda'x+v')\\
            &=\Lambda\Lambda'x+\Lambda v'+v\\
            &=(\Lambda v'+v,\Lambda\Lambda')x.
        \end{align}
    \end{subequations}
\end{remark}

\begin{proposition}[\cite{ooZYLAooXwWjLa}]      \label{PROPooDHYWooXxEXvl}
    Soit \( n\geq 1\) et un élément \( R\) de \( \gO(n)\) de déterminant \( -1\) tel que \( R^2=\id\). En posant \( C_2=\{ \id,R \}\) nous avons
    \begin{equation}
        \gO(n)=\SO(n)\times_{\rho} C_2
    \end{equation}
\end{proposition}

\begin{proof}
    Notons que pour \( R\) nous pouvons prendre par exemple \( (x_1,\ldots, x_n)\mapsto (-x_1,x_2,\ldots, x_n)\). Ce que nous allons montrer être un isomorphisme est :
    \begin{equation}
        \begin{aligned}
            \psi\colon \SO(n)\times C_2&\to \gO(n) \\
            (A,h)&\mapsto Ah. 
        \end{aligned}
    \end{equation}
    \begin{subproof}
        \item[Injectif]
            Soient \( A,B\in \SO(n)\) et \( h,k\in C_2\) tels que \( \psi(A,h)=\psi(B,k)\), c'est à dire tels que \( Ah=Bk\). Vu que \( \det(A)=\det(B)=1\) nous avons \( \det(h)=\det(k)\). Mais comme \( C_2\) contient un élément de déterminant \( 1\) et un élément de déterminant \( -1\), nous avons \( h=k\). De là \( A=B\).
        \item[Surjectif]
            Soit \( X\in\gO(n)\). Si \( \det(X)=1\) alors \( X\in \SO(n)\) et \( X=\psi(X,\mtu)\). Si par contre \( \det(X)=-1\) alors \( XR\in\SO(n)\) parce que \( \det(XR)=1\) et nous avons
            \begin{equation}
                \psi(XR,R)=XR^2=X.
            \end{equation}
        \item[Homomorphisme]
            Nous avons
            \begin{equation}
                \psi\Big( (A,h)(B,k) \Big)=\psi\big( A\rho_h(B),hk \big)=A(hBh^{-1})hk=AhBk,
            \end{equation}
            tandis que
            \begin{equation}
                \psi(A,h)\psi(B,k)=AhBk,
            \end{equation}
            qui est la même chose.
    \end{subproof}
\end{proof}

\begin{lemma}[\cite{JGAdTA}]
    Si \( n\geq 3\), alors toute droite est intersection de deux plans non isotropes.
\end{lemma}

\begin{proposition}[\cite{ooZYLAooXwWjLa}]      \label{PROPooVEEUooJQmmkN}
    Si une isométrie de \( \eR^n\) fixe un ensemble \( F\) de points, alors elle fixe l'espace affine engendrée par \( F\).
\end{proposition}

\begin{proof}
    Soit \( f\in \Isom(\eR^n)\) fixant \( F\). Par le théorème \ref{ThoDsFErq}, c'est une application affine et l'ensemble \( \Fix(f)\) des points fixés par \( f\) est un sous-espace affine de \( \eR^n\), grâce à la proposition \ref{PROPooYRCJooIcmUVI}.

    Donc \( \Fix(f)\) est un espace affine contenant \( F\). Vu que l'espace affine engendré par \( F\) est l'intersection de tous les espace affines contenant \( F\), il est en particulier contenu dans \( \Fix(f)\).
\end{proof}

\begin{corollary}       \label{CORooZHZZooDgTzsW}
    Si \( f\) et \( g\) sont des isométries de \( \eR^n\) qui coïncident sur \( F\), alors elles coïncident sur l'espace affine engendré par \( F\).
\end{corollary}

\begin{proof}
    Nous considérons \( h=g^{-1}\circ f\) qui est une isométrie de \( \eR^n\) fixant \( F\). Elle fixe donc, par la proposition \ref{PROPooVEEUooJQmmkN}, l'espace affine engendré par $F$. Or tout point fixé par \( h\) est un point sur lequel \( g\) et \( f\) coïncident.
\end{proof}

%--------------------------------------------------------------------------------------------------------------------------- 
\subsection{Classification des isométries de \( \eR\)}
%---------------------------------------------------------------------------------------------------------------------------

\begin{definition}
    Soit \( x\in \eR\); nous notons \( \sigma_x\) la \defe{réflexion}{réflexion}\nomenclature[R]{\( \sigma_x\)}{réflexion par rapport à \( x\)} par rapport à \( x\), c'est à dire
    \begin{equation}
        \sigma_x(y)=2x-y.
    \end{equation}
\end{definition}

\begin{theorem}[\cite{ooZYLAooXwWjLa}]
    Toute isométrie de \( \eR\) est composée d'au plus \( 2\) réflexions. Plus précisément toute isométrie de \( \eR\) est dans une des trois catégories suivantes :
    \begin{itemize}
        \item l'identité (\( 0\) réflexions),
        \item les réflexions,
        \item les translations (\( 2\) réflexions)
    \end{itemize}
\end{theorem}

\begin{proof}
    Nous divisions la preuve en fonction du nombre de points fixés par l'isométrie \( f\in\Isom(\eR)\).
    \begin{subproof}
        \item[\( f\) fixe deux points distincts]
            Alors elle fixe l'espace affine engendrée par ces deux points par la proposition \ref{PROPooVEEUooJQmmkN}. Donc \( f\) fixe tout \( \eR\) et est l'identité.
        \item[\( f\) fixe un unique point]
            Soit \( x\) l'unique point fixé par \( f\) et considérons \( y\neq x\). Vu que \( x=f(x)\) et que \( f\) est une isométrie,
            \begin{equation}
                d\big( x,f(y) \big)=d\big( f(x),f(y) \big)=d(x,y).
            \end{equation}
            Donc \( f(y)\) est à égale distance de \( x\) que \( y\). Autrement dit, \( f(y)\) est soit \( y\) soit \( \sigma_x(y)\). Mais comme \( x\) est unique point fixe, \( f(y)=\sigma_x(y)\). Ce raisonnement étant valable pour tout \( y\neq x  \) nous avons \( f=\sigma_x\).
        \item[\( f\) n'a pas de points fixes]
            Soit \( x\in \eR\) et \( y=\frac{ x+f(x) }{ 2 }\). Nous posons \( g=\sigma_y\circ f\). Alors \( x\) est un point fixe de \( g\) parce que
            \begin{equation}
                g(x)=\sigma_y\big( f(x) \big)=2y-f(x)=x.
            \end{equation}
            Donc soit \( g\) est l'identité soit \( g\) est une réflexion (par les points précédents). La possibilité \( g=\id\) est exclue parce que cela ferait \( f=\sigma_y\) alors que \( f\) n'a pas de points fixes. Donc \( g\) est une réflexion; et comme \( x\) est un point fixe de \( g\) nous avons \( g=\sigma_x\). Au final
            \begin{equation}
                f=\sigma_y\circ\sigma_x.
            \end{equation}
            Montrons que cela implique que \( f\) est une translation :
            \begin{equation}
                \sigma_y\sigma_x(z)=\sigma_y(2x-z)=2y-2x+z=z+2(y-x).
            \end{equation}
            Donc \( \sigma_y\circ\sigma_x\) est la translation de vecteur \( 2(y-x)\).
    \end{subproof}
\end{proof}

%+++++++++++++++++++++++++++++++++++++++++++++++++++++++++++++++++++++++++++++++++++++++++++++++++++++++++++++++++++++++++++ 
\section{Classification des isométries dans \( \eR^2\)}
%+++++++++++++++++++++++++++++++++++++++++++++++++++++++++++++++++++++++++++++++++++++++++++++++++++++++++++++++++++++++++++

%--------------------------------------------------------------------------------------------------------------------------- 
\subsection{Réflexions}
%---------------------------------------------------------------------------------------------------------------------------

\begin{lemma}   \label{LEMooSZZWooPDHnGl}
    Un point \( M\) est sur la médiatrice du segment \( [A,B]\) si et seulement si \( \| M-A \|=\| M-B \|\).
\end{lemma}

Soit un espace vectoriel \( E\) de dimension \( 2\) muni d'un produit scalaire\footnote{Définition \ref{DefVJIeTFj}.}. Cela pourrait très bien être \( \eR^2\), mais nous allons nous efforcer de l'appeler \( E\) pour reste un peu général.

\begin{lemmaDef}[Caractérisation des réflexions]        \label{DEFooLJKDooUaamen}
    Soit une droite \( \ell\) de \( \eR^2\). Il existe une unique application \( f\colon \eR^2\to \eR^2\) telle que
    \begin{enumerate}
        \item
            \( f(x)=x\) pour tout \( x\in \ell\).
        \item
            \( f\) échange les côtés de \( \ell\).
        \item
            \( f\) laisse invariants les droites perpendiculaires à \( \ell\) et les cercles dont le centre est sur \( \ell\).
    \end{enumerate}
    Cette application est la \defe{réflexion}{réflexion!dans \( \eR^2\)} d'axe \( \ell\).
\end{lemmaDef}

\begin{proof}
    Soit \( x\) hors de \( \ell\) et \( p\) la droite perpendiculaire à \( \ell\) et passant par \( x\). Nous avons \( f(x)\in p\). En nommant \( P\) l'intersection entre \( \ell\) et \( p\), nous considérons le cercle \( S(P,\| Px \|)\) qui est un cercle dont le centre est sur \( \ell\). Il contient \( x\) et donc \( f(x)\in S(P,\| Px \|)\).

    Donc \( f(x)\in p\cap S(P,\| Px \|)\). L'intersection entre un cercle et une droite contient de façon générique deux point. L'un est \( x\), mais \( f(x)=x\) n'est pas possible parce que \( x\) est hors de \( \ell\) et \( f\) doit inverser les côtés de \( \ell\). Donc \( f(x)\) est l'autre.

    Cela prouve l'unicité. En ce qui concerne l'existence, il suffit de noter que la réflexion \( \sigma_{\ell}\) satisfait les contraintes.
\end{proof}

\begin{lemma}       \label{LEMooZSDRooUkNYer}
    Soit une droite \( \ell\) est \( A\in E\). Alors
    \begin{equation}        \label{EQooVUQDooKuwszl}
        \sigma_{\ell}(A)=2\pr_{\ell}(A)-A
    \end{equation}
    où \( \pr_{\ell}\) est l'opération de projection orthogonale sur la droite \( \ell\).
\end{lemma}

\begin{proof}
    Nous posons \( f(X)=2\pr_{\ell}(X)-X\) et nous allons montrer que \( f=\sigma_{\ell}\) en vérifiant les conditions de la définition \ref{DEFooLJKDooUaamen}. Nous nous gardons bien de faire un raisonnement du type «nous allons montrer que \( f\) et \( \sigma_{\ell}\) coïncident sur deux points, et sont donc égales par le corollaire \ref{CORooZHZZooDgTzsW}» parce que nous ne savons pas encore que \( \sigma_{\ell}\) est une application affine, ni même que c'est une isométrie.

    Si \( X\in\ell\) alors \( \pr_{\ell}(X)=X\) et nous avons \( f(X)=2X-X=X\). Donc \( \ell\) est conservée.

    En ce qui concerne les deux côtés de \( \ell\), il existe une application linéaire \( s\colon E\to \eR\) et une constante \( c\in \eR\) telles qu'en posant \( \ell(X)=s(X)+c\), la droite \( \ell\) soit le lieux des points \( X\) tels que \( \ell(X)=0\). Un côté de la droite est \( \ell<0\) et l'autre côté est \( \ell>0\). Nous avons :
    \begin{subequations}
        \begin{align}
            \ell\big( f(A) \big)&=\ell\big( 2\pr_{\ell}(A)-A \big)\\
            &=s(2\pr_{\ell}(A)-A)+c\\
            &=2s\big( \pr_{\ell}(A) \big)-s(A)+c\\
            &=s\big( \pr_{\ell}(A) \big)-s(A)\\
            &=-c-s(A)\\
            &=-\ell(A)
        \end{align}
    \end{subequations}
    où nous avons utilisé le fait que, \( \pr_{\ell}(A)\) étant sur \( \ell\), \( s\big( \pr_{\ell}(A) \big)+c=0\). Nous avons donc \( \ell\big( f(A) \big)=-\ell(A)\), ce qui indique que \( A\) et \( f(A)\) sont de part et d'autre de \( \ell\).

    Si \( d\) est une droite perpendiculaire à \( \ell\) et si \( A\in d\) alors \( f(A)=2\pr_{\ell}(A)-A=  \big( \pr_{\ell}(A)-A \big)+A\in d  \) parce que \( \pr_{\ell}(A)\in d\) du fait que \( d\) soit précisément perpendiculaire à \( \ell\). Nous avons aussi utilisé le fait que si \( A,B,C\in d\) alors \( (B-A)+C\in d\); pensez que \( B-A\) est un vecteur directeur et que \( C\) est un point de \( d\).

    Enfin soit \( K\in\ell\) et un cercle \( S(K,r)\) centré en \( K\). Soit \( A\in S(K,r)\); nous devons vérifier que \( f(A)=S(K,r)\). Le segment \( [A,f(A)]\) est par définition perpendiculaire à \( \ell\). Soit \( M\), le milieu, qui est sur la droite \( \ell\). Les triangles \( AMK\) et \( f(A)MK\) sont rectangles en \( M\), et \( \| AM \|=\| Mf(A) \|\). Le théorème de Pythagore donne \( \| AK \|=\| f(A)K \|\). Donc le cercle centré en \( K\) est donc préservé par \( f\).

    Nous en déduisons que \( f=\sigma_{\ell}\).
\end{proof}

\begin{proposition}[\cite{ooIIMKooJpdFyk}]      \label{PROPooFSVEooWmJsnv}
    Une réflexion est une isométrie de \( (E,d)\) où \( d(A,B)=\| A-B \|\).
\end{proposition}

\begin{proof}
    Soient \( A,B\in E\); il faut vérifier que \( \| A-B \|=\| \sigma_{\ell}(A)-\sigma_{\ell}(B) \|\). Pour cela nous écrivons
    \begin{equation}
        B-A=\underbrace{B-\pr_{\ell}(B)}_{=a}+\underbrace{\pr_{\ell}(B)-\pr_{\ell}(A)}_{=b}+\underbrace{\pr_{\ell}(A)-A}_{=c}.
    \end{equation}
    Vu que \( b\perp a\) et \( b\perp c\) nous avons
    \begin{equation}
        \| B-A \|=\langle B-A, B-A\rangle =\| a \|^2+2\langle a, c\rangle +\| b \|^2+\| c \|^2.
    \end{equation}
    Nous pouvons faire le même jeu avec \( \sigma_{\ell}(B)-\sigma_{\ell}(A)\) en tenant compte du fait que \( \pr_{\ell}\big( \sigma_{\ell}(X) \big)=\pr_{\ell}(X)\) et que
    \begin{equation}
        \sigma_{\ell}(A)-\pr_{\ell}(A)=2\pr_{\ell}(A)-A-\pr_{\ell}(A)=-\big( A-\pr_{\ell}(A) \big).
    \end{equation}
    Là nous avons utilisé le lemme \ref{LEMooZSDRooUkNYer}. Ce que nous trouvons est que
    \begin{equation}
        \sigma_{\ell}(B)-\sigma_{\ell}(A)=-a+b-c,
    \end{equation}
    et donc encore une fois 
    \begin{equation}
        \| \sigma_{\ell}(B)-\sigma_{\ell}(A) \|=\| a \|^2-2\langle a, c\rangle +\| b \|^2+\| c \|^2.
    \end{equation}
\end{proof}

\begin{remark}
    Il faut bien comprendre que si l'axe de la réflexion ne passe par par \( 0\) (le zéro de l'espace vectoriel normé \( (E,\| . \|)\)), la réflexion n'est pas une isométrie de \( (E,\| . \|)\) au sens où nous n'avons pas \( \| \sigma_{\ell}(x) \|=\| x \|\).
\end{remark}

\begin{lemma}       \label{LEMooTCIEooXdyuHu}
    Si \( A'\) est l'image de \( A\) par \( \sigma_{\ell}\) alors \( \ell\) est la médiatrice du segment \( [A,A']\).
\end{lemma}

\begin{proof}
    Soit \( M\in\ell\). Nous avons
    \begin{equation}
        \| A-M \|^2=\| \pr_{\ell}(A)-A \|^2+\| \pr_{\ell}(A)-M \|^2
    \end{equation}
    parce que \( A-\pr_{ell}(A)\perp M-\pr_{\ell}(A)\). Par ailleurs, vu que \( \sigma_{\ell}(A)=2\pr_{\ell}(A)-A\) et que \( \pr_{\ell}(A)=\pr_{\ell}(A')\),
    \begin{equation}
        \| \pr_{\ell}(A)-A \|=\| \pr_{\ell}(A')-A' \|.
    \end{equation}
    Nous avons donc
    \begin{equation}
        \| \sigma_{\ell}(A)-M \|^2=\| A-M \|^2,
    \end{equation}
    ce qui prouve que \( M\) est sur la médiatrice de \( [A',A]\) par le lemme \ref{LEMooSZZWooPDHnGl}.
\end{proof}

\begin{normaltext}
    Si \( l\) est une droite dans \( \eR^2\), nous avons la réflexion \( \sigma_l\in\Isom(\eR^2)\) d'axe \( l\). Cela est une isométrie et donc une application affine par le théorème \ref{ThoDsFErq}. Le lemme suivant détermine comment la réflexion \( \sigma_{\ell}\) se décompose en une translation et une application linéaire.
\end{normaltext}

\begin{lemma}   \label{LEMooVOJLooCFgdNG}
    Soit une droite \( \ell\). Alors
    \begin{equation}
        \sigma_{\ell}=\tau_{2w}\circ\sigma_{\ell_0}
    \end{equation}
    où \( \ell_0\) est la droite parallèle à \( \ell\) passant par l'origine, et \( w\) est le vecteur perpendiculaire à \( \ell\) tel que \( \ell_0=\ell+v\).
\end{lemma}

\begin{proof}
    Il faut trouver trois points non alignés sur lesquels les deux applications coïncident; cela suffira par le corollaire \ref{CORooZHZZooDgTzsW}. 

    Pour tous les points de \( \ell_0\), l'égalité fonctionne parce que si \( x\in\ell_0\),
    \begin{equation}
        \sigma_{\ell}(x)=x+2w,
    \end{equation}
    tandis que
    \begin{equation}
        \sigma_{\ell_0}(x)+2w=x+2w
    \end{equation}
    du fait que \( \sigma_{\ell_0}(x)=x\).

    Si \( x\in\ell\), alors
    \begin{equation}
        \sigma_{\ell}(x)=x
    \end{equation}
    tandis que
    \begin{equation}
        \sigma_{\ell_0}(x)+2w=x-2w+2w=x.
    \end{equation}
    Donc les applications affines \( \sigma_{\ell}\) et \( x\mapsto \sigma_{\ell_0}(x)+2w\) coïncident sur \( \ell\) et \( \ell_0\). Elles coïncident donc partout.
\end{proof}

%--------------------------------------------------------------------------------------------------------------------------- 
\subsection{Rotations}
%---------------------------------------------------------------------------------------------------------------------------

\begin{definition}[Rotation en dimension \( 2\)]        \label{DEFooFUBYooHGXphm}
    Une \defe{rotation}{rotation!en dimension \( 2\)} d'un espace euclidien de dimension \( 2\) est une composée de deux réflexions d'axes non parallèles. L'identité est une rotation.
\end{definition}

\begin{normaltext}
    Quelque remarques à propos de cette définition.
    \begin{enumerate}
        \item
            Attention : nous ne parlons pas encore de rotations «vectorielles» : ici le centre de la rotation (que nous n'avons pas encore défini) peut ne pas être \( 0\).
        \item
            Dans la même veine : plus tard, lorsque nous saurons que les rotations sont des isométries de \( (E,d)\) où \( d(X,Y)=\| X-Y \|\), nous allons en réalité beaucoup plus souvent parler de rotations centrées en l'origine qu'en un point quelconque. C'est pourquoi à partir de \ref{NORMooOUDJooRfbDEX} nous dirons le plus souvent «rotation»  pour «rotation centrée en \( 0\)». D'où les énoncés comme «les rotations sont les matrice orthogonales» (corollaire \ref{CORooVYUJooDbkIFY}) , qui \emph{stricto senus} de la définition \ref{DEFooFUBYooHGXphm} sont faux.
        \item
            Une rotation est composée de deux réflexions d'axes non parallèles. Il est cependant trop tôt pour décréter que l'intersection de ces axes est le centre de la rotation. Rien ne dit en effet pour l'instant que deux décompositions différentes de la même rotation, avec des axes différents donnent le même point d'intersection.
        \item
            Pourquoi ajouter l'identité  ? Pour avoir un groupe. Dans le cas vectoriel, il est suffisant de demander d'être une composée de deux réflexions, parce que toutes les réflexions vectorielles ont des axes qui s'intersectent en \( 0\). Le cas des axes parallèles est seulement le cas des axes confondus et revient à l'identité.

            Si nous voulons avoir un groupe même pour les rotations centrées ailleurs qu'en zéro, nous devons ajouter «à la main» l'identité.
    \end{enumerate}

    Toutes ces remarques se résument par : «tout devient compliqué du fait que nous voulons considérer également les rotations centrées ailleurs qu'en zéro». En se contentent du cas vectoriel, de nombreuses choses sont plus simples.
\end{normaltext}

\begin{corollary}       \label{CORooNKKIooPGOUJl}
    Si \( A\neq B\) dans \( E\) alors il existe une unique réflexion envoyant \( A\) sur \( B\).
\end{corollary}

\begin{proof}
    En ce qui concerne l'existence, la réflexion dont l'axe est la médiatrice de \( [A,B]\) fait l'affaire. En ce qui concerne l'unicité, le lemme \ref{LEMooTCIEooXdyuHu} nous dit que si \( A\) est envoyé sur \( B\), l'axe est forcément la médiatrice de \( [A,B]\).
\end{proof}

\begin{lemma}[\cite{ooYPVPooYGSlNU}]        \label{LEMooIJELooLWqBfE}
    La rotation \( r=\sigma_1\circ\sigma_2\), si elle est différente de l'identité, ne fixe que le point d'intersection \( \ell_1\cap\ell_2\).
\end{lemma}

\begin{proof}
    Nous nommons \( O=\ell_1\cap\ell_2\). Soit \( A\in E\), et supposons que \( r(A)=A\). Nous avons \( \sigma_1\circ r=s_2\) et donc
    \begin{equation}
        \sigma_1(A)=(\sigma_1\circ r)(A)=s_2(A).
    \end{equation}
    On pose \( B=\sigma_1(A)\). Alors \( \sigma_1\) et \( \sigma_2\) envoient tout deux \( A\) sur \( B\). 
    
    Si \( A=B\) alors \( A\) est fixé par \( \sigma_1\) et donc appartient à \( \ell_1\). Même chose pour \( A\) est fixé par \( \sigma_2\) et donc \( A\in\ell_2\). Cela donne \( A=B=O\), et donc le point fixé par \( r\) est \( O\).

    Si \( A\neq B\) alors il existe une unique réflexion envoyant \( A\) sur \( B\) (corollaire \ref{CORooNKKIooPGOUJl}). L'unicité signifie que \( \sigma_1=\sigma_2\). Dans ce cas, \( r=\sigma_1\circ\sigma_2=\id\).
\end{proof}

\begin{normaltext}      \label{NORMooDPBOooKkRuTn}
    La rotation \( \sigma_1\circ\sigma_2\) laisse évidemment fixé le point \( \ell_1\cap \ell_2\). Si \( \sigma_1\circ\sigma_2=\sigma_a\circ\sigma_b\) alors rien n'oblige les axes de \( \sigma_1\) et \( \sigma_2\) d'être identiques à ceux que \( \sigma_a\) et \( \sigma_b\). Mais l'intersection \( \ell_1\cap\ell_2\) doit être la même que l'intersection \( \ell_a\cap \ell_b\) parce que c'est l'unique point fixé par la composée. Cela nous permet de poser les définition suivante.
\end{normaltext}

\begin{definition}
    Le \defe{centre}{centre!d'une rotation} d'une rotation (autre que l'identité) est l'unique point fixé par la rotation.
\end{definition}
Comme expliqué dans \ref{NORMooDPBOooKkRuTn}, le centre de la rotation s'obtient en l'écrivant comme composée de deux réflexions et en considérant le point d'intersection des axes.

\begin{lemma}       \label{LEMooTZNWooTVOklu}
    Les rotation sont des isométries pour la distance : \( \| X-Y \|=\| r(X)-r(Y) \|\).
\end{lemma}

\begin{proof}
    Si \( r=\sigma_1\circ\sigma_2\), en utilisant le fait que \( \sigma_1\) et \( \sigma_2\) sont des isométries de \( (E,d)\) (\ref{PROPooFSVEooWmJsnv}) nous avons :
    \begin{equation}
        d(X,Y)=d\big( \sigma_2(X),\sigma_2(Y) \big)=d\big( \sigma_1\sigma_2(X),\sigma_1\sigma_2(Y) \big)=d\big( r(X),r(Y) \big).
    \end{equation}
\end{proof}

Ce lemme nous dit qu'une rotation de centre \( O\) vérifie \( \| OX \|=\| Or(X) \|\) pour tout \( X\).

\begin{proposition}[\cite{ooYPVPooYGSlNU}]      \label{PROPooNXJKooEDOczh}
    Soient \( A,B,O\in E\) tels que \( \| AO \|=\| BO \|\neq 0\). Alors il existe une unique rotation \( r\) centrée en \( O\) telle que \( r(A)=B\).
\end{proposition}

\begin{probleme}
    Attention : la preuve qui suit contient de nombreuses galipettes et improvisations personnelles. Relisez-la attentivement avant de la prendre pour argent comptant.

    La difficulté tient essentiellement à ce que cette preuve traite de façon vectorielle (tous les points sont des éléments de \( E\)) un énoncé qui est essentiellement affine : tous les points doivent être vus comme vecteurs partant de \( O\).

    Si vous comparez la preuve donnée ici avec celle de \cite{ooYPVPooYGSlNU}, vous remarquerez que dans ce dernier, seule la partie «\( A\) et \( O\) son alignés» est présente. C'est parce que lui, il se met directement dans le cas vectoriel et \( O=0\). Il a donc une preuve un tout petit peu moins générale, mais au moins ses isométries sont linéaires et non affines.
\end{probleme}

\begin{proof}
    Existence et unicité séparément.
    \begin{subproof}
        \item[Existence]
            Si \( A=B\), l'identité fait l'affaire. Sinon, \( \| A-O \|=\| B-O \|\) implique que la médiatrice de \( [A,B]\) contient \( O\). Soit \( \sigma_m\) la réflexion selon cette médiatrice. La rotation \( \sigma_m\circ\sigma_{(AO)}\) convient.

        \item[Unicité]

            Soit \( r\) une rotation de centre \( O\) et telle que \( r(A)=B\). Si \( A=B\) alors \( r=\id\) parce qu'une rotation autre que l'identité ne fixe que son centre par le lemme \ref{LEMooIJELooLWqBfE}. Nous supposons que \( A\neq B\).

            Nous posons \( g=\sigma_m\circ r\). Alors \( g(A)=\sigma_m(B)=A\) parce que \( \sigma_m(B)=A\) et \( r(A)=B\). Cela signifie que \( g\) est une isométrie qui fixe \( A\). 
            
            \begin{subproof}
                \item[Si \( A\) et \( O\) ne sont pas alignés]
                
                    Attention : ici \( O\) est un point de \( E\), pas le zéro de l'espace vectoriel \( E\). Lorsqu'on dit que \( A\) et \( O\) ne sont pas alignés, nous parlons bien d'alignement avec le zéro de \( E\).

                    Nous avons \( g(A)=A\) et \( g(O)=O\). Donc \( g\) coïncide avec \( \sigma_{(AO)}\) en deux points non alignés, c'est à dire en deux points pour lesquels l'espace engendré est tout \( E\). Nous en déduisons que \( g=\sigma_{(AO)}\).

                \item[Si \( A\) et \( O\) sont alignés]
            
            
            Soit maintenant un point \( C\) tel que \( A-O\perp C-O\) et 
            \begin{equation}
                \| OC \|=\| OA \|=\| OB \|.
            \end{equation}
            Vu que \( g\) est une isométrie pour la distance sur \( E\), pas pour la norme, nous ne pouvons pas écrire \( g(C-O)\perp g(A-O)\) à partir de \( C-O\perp A-O\). Nous décomposons \( g(X)=s(X)+G\) où \( s\) est linéaire sur \( E\). Il est vite vu que \( s\) est une isométrie de \( (E,\| . \|)\) :
            \begin{equation}
                \| X-Y \|=\| g(X)-g(Y) \|=\| s(X)+G-s(y)-G \|=\| s(X)-s(Y) \|=\| s(X-Y) \|
            \end{equation}
            pour tout \( X,Y\in E\). Nous avons de plus \( g(A)=A\) et \( g(O)=O\), ce qui donne \( O=s(O)+G\) et \( A=s(A)+G\). En égalisant les valeurs de \( G\) nous avons
            \begin{equation}        \label{EQooPEWGooABHUvu}
                O-s(O)=A-s(A).
            \end{equation}
            Vu que \( s\) est une isométrie (une vraie) nous avons
            \begin{equation}
                s(A-O)\perp s(C-O),
            \end{equation}
            mais \( s(A-O)=s(A)-s(O)=A-O\) par \eqref{EQooPEWGooABHUvu}. Donc
            \begin{equation}
                A-O\perp s(C-O).
            \end{equation}
            Nous en concluons que \( s(C-O)=\pm (C-O)\). Parce que les vecteurs \( \pm(C-O)\) sont les deux seuls de norme \( \| AO \| =\| CO \|\) à être perpendiculaire à \( A-O\). Rappel : la définition de \( C\) et le fait que nous soyons en dimension \( 2\).o
            
            Est-il possible d'avoir \( s(C-O)=C-O\) ? Cela donnerait
            \begin{subequations}
                \begin{align}
                    g(A-O)&=s(A)-s(O)+G=s(A)-O+O-s(O)+G=A-O+G\\
                    g(C-O)&=C-O+G,
                \end{align}
            \end{subequations}
            ce qui signifierait que \( g\) et \( \tau_G\) coïncideraient sur les points \( A-O\) et \( C-O\), et donc seraient égaux par le corollaire \ref{CORooZHZZooDgTzsW}. Cela est cependant impossible parce que \( g\) fixe au moins les points \( A\) et \( O\) alors que la translation ne fixe aucun point. Nous en déduisons \( s(C-O)=-(C-O)\).

            Nous avons aussi, parce que \( (AO)\) est une droite passant par l'origine que
            \begin{equation}
                \sigma_{(AO)}(A-O)=A-O
            \end{equation}
            et parce que \( C-O\) est perpendiculaire à cette droite :
            \begin{equation}
                \sigma_{(AO)}(C-O)=-(C-O).
            \end{equation}
            Nous avons donc quand même que \( g\) et \( \sigma_{(AO)}\) coïncident sur deux points non alignés : \( A-O\) et \( C-O\).

            \end{subproof}

            Dans tous le cas, \( g=\sigma_{(AO)}\). Nous avons donc
            \begin{equation}
                \sigma_{(OA)}=\sigma_m\circ r,
            \end{equation}
            et donc \( r\) est fixé à
            \begin{equation}
                r=\sigma_m\circ\sigma_{(OA)}.
            \end{equation}
    \end{subproof}
\end{proof}

\begin{normaltext}
    Anticipons un peu et faisons semblant de déjà connaitre les matrices et les fonctions trigonométriques. La proposition \ref{PROPooNXJKooEDOczh} nous dit qu'il existe une seule rotation amenant \( A\) sur \( B\). Vous pourriez objecter que le point \( (1,0)\) peut être amené sur \( (0,-1)\) soit par la rotation d'angle \( 3\pi/2\), soit par celle d'angle \( -\pi/2\). Il n'en est rien parce que ces deux rotations sont les mêmes ! Pensez-y. En tant qu'application \( \eR^2\to \eR^2\), la rotation \( R_{3\pi/2}\) est égale à \( R_{-\pi/2}\).
\end{normaltext}

Une rotation donnée peut être écrite de beaucoup de façons comme composée de deux réflexions. En fait d'autant de façons qu'il y a de réflexions.
\begin{proposition}[\cite{ooYPVPooYGSlNU}]      \label{PROPooKAZEooLTHWKe}
    Soit une rotation \( r\) de \( E\) centrée en \( O\). Pour toute réflexion \( \sigma_{\ell}\) telle que le centre de \( r\) soit sur \( \ell\), il existe une réflexion \( \sigma_1\) tells que \( r=\sigma_1\circ\sigma_{\ell}\). Il existe aussi une réflexion \( \sigma_2\) telle que \( r=\sigma_{\ell}\circ s_2\).
\end{proposition}

\begin{proof}
    Si \( r=\id\) c'est bon avec \( s_1=s_2=\sigma_{\ell}\). Sinon nous considérons \( A\neq O\) sur \( \ell\), et \( B=r(A)\). Nous savons que \( B\neq A\) parce que \( O\) est le seul point de \( E\) fixé par \( r\) (proposition \ref{LEMooIJELooLWqBfE}). Il existe une réflexion (unique) \( \sigma_1\) faisant \( \sigma_1(A)=B\), et c'est le réflexion dont l'axe est la médiatrice de \( [A,B]\). Le point \( O\) est sur cette médiatrice parce que les rotations sont des isométries de \( (E,d)\) (lemme \ref{LEMooTZNWooTVOklu}).

    La rotation \( \sigma_1\circ \sigma_{\ell}\) vérifie
    \begin{equation}
        (\sigma_1\circ\sigma_{\ell})(A)=\sigma_1(A)=B.
    \end{equation}
    Or \( \| OA \|=\| OB \|\), donc il y a unicité de la rotation centrée en \( O\) portant \( A\) sur \( B\) (proposition \ref{PROPooNXJKooEDOczh}); nous avons donc \( r=\sigma_1\circ\sigma_{\ell}\).

    En ce qui concerne \( r=\sigma_{\ell}\circ\sigma_2\), il faut appliquer ce que nous venons de faire à la rotation \( r^{-1}\): il existe \( \sigma_2\) tel que \( r^{-1}=\sigma_2\circ\sigma_{\ell}\), ce qui donne
    \begin{equation}
        r=\sigma_{\ell}\circ\sigma_2.
    \end{equation}
\end{proof}

\begin{proposition}[\cite{ooYPVPooYGSlNU}]      \label{PROPooWMESooNJMdxf}
    Les rotations basées en \( O\) forment un groupe abélien.
\end{proposition}

\begin{proof}
    L'identité est une rotation par définition. En ce qui concerne l'inverse, si \( r=\sigma_1\sigma_2\) alors \( r^{-1}=\sigma_2\sigma_1\). Nous commençons maintenant les choses pas tout à fait évidentes.
    \begin{subproof}
        \item[Composition]
            Soient des rotations \( r,r'\) centrées en \( O\). Soit également une réflexion \( \sigma\) dont l'axe contient \( O\). Alors la proposition \ref{PROPooKAZEooLTHWKe} nous donne l'existence de \( \sigma_1\) et \( \sigma_2\) tels que \( r=\sigma_1\sigma\) et \( r'=\sigma\sigma_2\). Avec ça, la composition donne
            \begin{equation}
                rr'=\sigma_1\sigma\sigma\sigma_2=\sigma_1\sigma_2,
            \end{equation}
            qui est encore une rotation.
        \item[Commutativité]
            Soient deux rotations \( r\) et \( r'\) ainsi que des décompositions \( r=\sigma_1\sigma\), \( r'=\sigma\sigma_2\). Nous avons
            \begin{subequations}
                \begin{align}
                    rr'&=\sigma_1\sigma_2\\
                    r'r&=\sigma\sigma_2\sigma_2\sigma.
                \end{align}
            \end{subequations}
            Vu que \( t=\sigma_2\sigma_1\) est une rotation nous pouvons encore appliquer la proposition \ref{PROPooKAZEooLTHWKe} pour avoir \( t=\sigma_2\sigma_1=\sigma\sigma_3\). Avec ça,
            \begin{equation}
                r'r=\sigma\sigma\sigma_3\sigma=\sigma_3\sigma.
            \end{equation}
            Mais aussi \( rr'=\sigma_1\sigma_2=t^{-1}=\sigma_3\sigma\). Nous avons donc bien \( rr'=r'r\), et le groupe est commutatif.
    \end{subproof}
\end{proof}

%--------------------------------------------------------------------------------------------------------------------------- 
\subsection{Rotations vectorielles}
%---------------------------------------------------------------------------------------------------------------------------

\begin{normaltext}      \label{NORMooOUDJooRfbDEX}
    Jusqu'à présent nous avons parlé de rotations «affines». Parmi elles, Les rotations centrées en \( 0\) (zéro, l'origine de \( E\) comme espace vectoriel) sont de particulière importance. Ce sont des applications linéaires, et même des isométries. Dans la suite, nous allons souvent dire simplement «rotation» pour dire «rotation centrée en \( 0\)».

    Vu que nous allons maintenant prendre un point de vue plus vectoriel, nous allons noter les points de \( E\) avec des lettres comme \( x\), \( y\), \( u\), \( v\) et plus avec des majuscules, comme quand on avait un point de vue affin. En même temps, nous allons noter les applications \( E\to E \) par des lettres comme \( A\) et ne plus écrire les parenthèses. Bref, nous écrivons \( Au\) au lieu de \( r(A)\).
\end{normaltext}

\begin{lemma}       \label{LEMooSYZYooWDFScw}
    En dimension \( 2\), les réflexions (vectorielles, c'est à dire dont l'axe passe par \( 0\)) ont un déterminant \( -1\).
\end{lemma}

\begin{proof}
    Soit une réflexion d'axe \( \ell\). Prenons une base orthonormale de \( E\) constituée de \( e_1\) sur \( \ell\) et de \( e_2\perp \ell\). Alors \( \sigma_{\ell}(e_1)=e_1\) et \( \sigma_{\ell}(e_2)=-e_2\). la formule du déterminant donne
    \begin{equation}
        \det(\sigma_{\ell})=e_1^*\big( \sigma_{\ell}(e_1) \big)e_1^*\big( \sigma_{\ell}(e_2) \big)-e_2^*\big( \sigma_{\ell}(e_1) \big)e_1^*\big( \sigma_{\ell}(e_2) \big)=1\times (-1)-0\times 0=-1.
    \end{equation}
    Nous utilisons de façon cruciale le fait que le calcul du déterminant ne dépende pas de la base choisie, lemme \ref{LEMooQTRVooAKzucd}.
\end{proof}

\begin{proposition}     \label{PROPooTUJWooAjtEnQ}
    Les rotations\footnote{Centrées en \( 0\), nous ne le répéterons pas !} sont 
    \begin{enumerate}
        \item
            des applications orthogonales au sens de la définition \ref{DEFooYKCSooURQDoS},
        \item
             des applications de déterminant \( 1\),
    \end{enumerate}
\end{proposition}

\begin{proof}
    Nous avons, pour tout \( u\in E\) l'égalité de la norme \( \| u \|\) et \( \| Au \|\) par le lemme \ref{LEMooTZNWooTVOklu} appliqué à \( Y=0\). En terme de produits scalaires nous avons alors \( \langle Au, Au\rangle =\langle u, u\rangle \), et donc
    \begin{equation}
        \langle A^*Au, u\rangle =\| u \|^2.
    \end{equation}
    En particulier si \( \{ e_i \}_{i=1,\ldots, n}\) est une base orthonormée de \( E\) nous avons
    \begin{equation}
        (A^*Ae_i)_i=\| e_i \|^2=1,
    \end{equation}
    ce qui donne \( \| A^*Ae_i \|\geq 1\), avec égalité si et seulement si \( A^*Ae_i=e_i\). Ici nous avons utilisé le fait que \( \langle x, e_i\rangle =x_i\), et le fait que pour tout \( i\) nous ayons \( \| x \|\geq | x_i |\), avec égalité seulement si \( x\) est un multiple de \( e_i\).

    Par ailleurs l'inégalité de Cauchy-Schwarz \ref{ThoAYfEHG} nous donne
    \begin{equation}
        \| u \|^2=| \langle A^*Au, u\rangle  | \leq \| A^*Au \|\| u \|
    \end{equation}
    et donc
    \begin{equation}
        \| u \|\leq \| A^*Au \|.
    \end{equation}
    Encore une fois, en appliquant cela à \( u=e_i\) nous trouvons \( 1\leq \| A^*Ae_i \|\). Vu que nous avions déjà l'inégalité dans l'autre sens, \( \| A^*Ae_i \|=1\). Et le cas d'égalité est uniquement possible avec \( A^*Ae_i=e_i\).

    Donc pour tout \( i\) de la base nous avons \( A^*Ae_i=e_i\). Nous avons donc \( A^*A=\id\) et l'application \( A\) est orthogonale. 

    En ce qui concerne le déterminant, les réflexions sont de déterminant \( -1\) par le lemme \ref{LEMooSYZYooWDFScw}, donc \( A=\sigma_1\circ\sigma_2\) est de déterminant \( 1\). Nous avons utilisé le fait que le déterminant était un morphisme : proposition \ref{PropYQNMooZjlYlA}\ref{ItemUPLNooYZMRJy}.
\end{proof}

\begin{remark}
    Nous ne savons pas encore que les rotations forment tout le groupe \( \SO(2)\) des endomorphismes orthogonaux de déterminant \( 1\). Il faudra attendre le corollaire \ref{CORooVYUJooDbkIFY} pour le savoir.
\end{remark}

\begin{lemma}       \label{LEMooMIJXooCjiQqP}
    L'application \( -\id\) est une rotation.
\end{lemma}

\begin{proof}
    Soit une base orthonormée \( \{ e_1,e_2 \}\) de \( E\) et la rotation \( r=\sigma_1\sigma_2\) où \( \sigma_i\) est la réflexion le long de l'axe \( \ell_i=\{ te_i \}_{t\in \eR}\). Faut-il vous prouver que \( r=-\id\) ? La réflexion \( \sigma_2\) retourne la composante \( y\) d'un vecteur écrit dans la base \( \{ e_1,e_2 \}\) sans toucher à la composante \( x\). La réflexion \( \sigma_1\) fait le contraire.
\end{proof}

%--------------------------------------------------------------------------------------------------------------------------- 
\subsection{Angle}
%---------------------------------------------------------------------------------------------------------------------------

Avant d'aborder la classification des isométries, nous devons parler de l'angle entre deux droites. Si \( \ell_1\) et \( \ell_2\) sont deux droites, alors il est bien clair deux angles peuvent prétendre être «l'angle entre \( \ell_1\) et \( \ell_2\)». De plus chacun de ces deux angles sont doubles parce que si \( \alpha\) peut prétendre être l'angle entre \( \ell_1\) et \( \ell_2\), alors \( -\alpha\) peut également prétendre.

\begin{lemmaDef}        \label{DEFooEGKOooRPGOAs}
    Si \( \ell_1\) et \( \ell_2\) sont deux droites sécantes au point \( O\) et si \( x\in\ell_1\) n'est pas \( O\), alors il existe un unique \( \alpha\in \mathopen[ 0 , \pi \mathclose[\) tel que \( R_O(\alpha)x\in \ell_2\). La valeur de \( \alpha\) ne dépend pas du choix du point \( x\in \ell_1\).

        Cet angle \( \alpha\) est l'\defe{angle}{angle!entre deux droites} de \( \ell_1\) à \( \ell_2\).
\end{lemmaDef}

\begin{remark}
    Nous ne parlons pas de l'angle entre \( \ell_1\) et \( \ell_2\) mais bien de l'angle \emph{de} \( \ell_1\) \emph{à} \( \ell_2\). L'ordre des droites est important.
\end{remark}

\begin{normaltext}
    Pour la suite, \( R_O(\alpha)\) est la rotation d'angle \( \alpha\) autour du point \( O\) tandis que \( R(\alpha)\) est la rotation d'angle \( \alpha\) autour de l'origine. En termes matriciels, la rotation d'angle \( \alpha\) est donnée par
    \begin{equation}        \label{EQooQBEJooAHaBbJ}
        R(\alpha)=\begin{pmatrix}
            \cos(\alpha)    &   -\sin(\alpha)    \\ 
            \sin(\alpha)    &   \cos(\alpha)    
        \end{pmatrix},
    \end{equation}
\end{normaltext}

\begin{lemma}[\cite{MonCerveau}]        \label{LEMooJLHGooQIpKIE}
    Soit \( A\in \eR^2\) et une droite \( \ell_1\). Soit \( \ell_2\) une droite passant par \( A\) et intersectant \( \ell_1\) en \( O\). Alors
    \begin{equation}
        \sigma_{\ell_1}(A)=R_O(-2\alpha)A
    \end{equation}
    où \( \alpha\) est l'angle de \( \ell_1\) à \( \ell_2\).
\end{lemma}

\begin{proof}
    Nous allons utiliser des coordonnées autour de \( O\). Il existe un vecteur \( v\) tel que
    \begin{equation}
        A=O+v
    \end{equation}
    Par définition de l'angle \( \alpha\), la droite \( \ell_2\) s'obtient par rotation d'angle \( \alpha\) depuis la droite \( \ell_1\). Donc le point
    \begin{equation}
        B=R_O(-\alpha)A
    \end{equation}
    est sur \( \ell_1\).

    Nous allons prouver que le point
    \begin{equation}
        D=R_O(-2\alpha)A
    \end{equation}
    est \( D=\sigma_{\ell_1}A\).

    Nous commençons par montrer que la droite \( (DA)\) est perpendiculaire à \( \ell_1\), c'est à dire que
    \begin{equation}
        (D-A)\cdot (B-O)=0.
    \end{equation}
    En utilisant le fait que 
    \begin{equation}
        R_O(\alpha)(O+X)=O+R(\alpha)X,
    \end{equation}
    nous avons
    \begin{equation}
        D-A=R_O(-2\alpha)(O+v)-(O+v)=O+R(-2\alpha)v-O-v=R(-2\alpha)v-v
    \end{equation}
    et de la même façon,
    \begin{equation}
        B-O=R(-\alpha)v.
    \end{equation}
    Notons que tous les \( O\) se sont simplifiés et qu'il ne reste que des rotation usuelles. En utilisant le fait que \( R(\alpha)\) est une isométrie, nous pouvons alors calculer
    \begin{subequations}
        \begin{align}
            (D-A)\cdot (B-O)&=\langle R(-2\alpha)v-v, R(-\alpha)v\rangle \\
            &=\langle R(-\alpha)v-R(\alpha)v, v\rangle.
        \end{align}
    \end{subequations}
    En utilisant la matrice de rotation \eqref{EQooQBEJooAHaBbJ} nous trouvons
    \begin{equation}
        \big( R(-\alpha)-R(\alpha) \big)v=\begin{pmatrix}
            2\sin(\alpha)v_2    \\ 
            -2\sin(\alpha)v_1    
        \end{pmatrix}
    \end{equation}
    et donc
    \begin{equation}
        \langle  \big( R(-\alpha)-R(\alpha) \big)v  , v\rangle =0.
    \end{equation}

    Le point \( D\) est bien sûr la droite perpendiculaire à \( \ell_1\) et passant par \( A\). Mais vu que \( D\) est obtenu à partir de \( A\) par une rotation, le point \( D\) est également sur le cercle de rayon \( \| OA \|\) et centré en \( O\). Ce cercle possède exactement deux intersections avec cette droite. Le premier est \( A\) et le second est \( \sigma_{\ell_1}(A)\). Vu que \( D\) n'est pas \( A\), nous avons \( D=\sigma_{\ell}(A)\).
\end{proof}

%--------------------------------------------------------------------------------------------------------------------------- 
\subsection{Classification}
%---------------------------------------------------------------------------------------------------------------------------

\begin{theorem}[\cite{ooZYLAooXwWjLa}]      \label{THOooRORQooTDWFdv}
    Toute isométrie du plan est une composition d'au plus \( 3\) réflexions. 
\end{theorem}

\begin{proof}
    Encore une fois nous décomposons la preuve en fonction du nombre de points fixes.
    \begin{subproof}
        \item[Si \( f\) n'a pas de points fixes]
            Soit \( x\in \eR^2\) et \( l\), la médiatrice du segment \( [x,f(x)]\). Par construction, \( f(x)=\sigma_l(x)\). Nous posons \( g=\sigma_l\circ f\), et nous avons
            \begin{equation}
                g(x)=x.
            \end{equation}
            Donc nous avons \( f=\sigma_l\circ g\) avec \( x\in\Fix(g)\).
        \item[Si \( f\) a un unique point fixe]
            Soit \( x\) cet unique point fixe. Soit \( y\neq x\) et \( l\) la médiatrice de \( [y,f(y)]\). En posant \( g=\sigma_l\circ f\) nous avons 
            \begin{equation}
                g(y)=y
            \end{equation}
            et \( g(x)=x\) parce que
            \begin{equation}
                d\big( x,f(y) \big)=d\big( f(x),f(y) \big)=d(x,y),
            \end{equation}
            ce qui donne que \( x\) est à égale distance de \( y\) et de \( f(y)\), c'est à dire que \( x\in l\) et par conséquent \( g(x)=(\sigma_l\circ f)(x)=\sigma_l(x)=x\). 

            Donc \( g\) fixe \( x\) et \( y\) et donc toute la droite \( (xy)\).
        \item[Si \( f\) fixe une droite]
            Soit \( l\) une droite fixée par \( f\), et soient \( x,y\in l\) et \( z\notin l\) (avec \( x\neq y\)). Le fait que \( x\) et \( y\) soient des points fixes de \( f\) implique
            \begin{subequations}
                \begin{numcases}{}
                    d\big( x,f(z) \big)=d(x,z)\\
                    d\big( y,f(z) \big)=d(y,z)
                \end{numcases}
            \end{subequations}
            ce qui signifie que \( f(z)\) est sur l'intersection des deux cercles\footnote{L'intersection existe pare que \( d(x,z)+d(y,z)>d(x,y)\).} \( S\big( x,d(x,z) \big)\) et \( S\big( y, d(y,z) \big)\), et comme ce sont deux cercles centrés sur la droite \( l\), les intersections sont liées par \( \sigma_l\). Autrement dit, les intersections sont \( z\) et \( \sigma_l(z)\).

            Si \( f(z)=z\) alors \( f\) fixe trois points non alignés et fixe dont \( \eR^2\), c'est à dire \( f=\id\).

            Si par contre \( f(z)=\sigma_l(z)\) alors les isométries \( f\) et \( \sigma_l\) coïncident sur trois points et coïncident donc partout par le corollaire \ref{CORooZHZZooDgTzsW} : \( f=\sigma_l\).
        \item[Conclusion]

            Nous avons montré que si \( \Fix(f)\) a dimension \( m\), alors il existe une droite pour laquelle \( f=\sigma_l\circ g\) avec \( \dim\big( \Fix(g) \big)>m\). Donc il faux au maximum trois pas pour avoir \( \dim\big( \Fix(g) \big)=2\) c'est à dire pour avoir \( g=\id\).
    \end{subproof}
\end{proof}

\begin{definition}
    Une \defe{réflexion glissée}{réflexion!glissée} est une transformation du plan de la forme \( \tau_v\circ\sigma_{\ell}\) où le vecteur \( v\) est parallèle à la droite \( \ell\).
\end{definition}

\begin{theorem}[\cite{ooZYLAooXwWjLa}]      \label{THOooVRNOooAgaVRN}
    Les isométries du plan sont exactement
    \begin{enumerate}
        \item
            l'identité (composée de \( 0\) réflexions),
        \item
            les réflexions,
        \item
            les translations (composées de \( 2\) translations d'axes parallèles),
        \item
            les rotations (composées de \( 2\) réflexions d'axes non parallèles),
        \item
            les réflexions glissées (composées de \( 3\) réflexions)
    \end{enumerate}
\end{theorem}

\begin{proof}
    Nous savons déjà que \( f\in \Isom(\eR^2)\) est une composée de \( 0\), \( 1\), \( 2\) ou \( 3\) réflexions. 
    \begin{subproof}
        \item[Zéro réflexions]
            Alors c'est l'identité. Ce n'est pas très profond.
        \item[Une réflexion]
            Alors \( f\) est une réflexion. Toujours pas très profond.
        \item[Deux réflexions]
            Soit \( f=\sigma_{\ell_1}\circ\sigma_{\ell_2}\). Maintenant ça s'approfondit un bon coup.
            
            Nous supposons d'abord que \( \ell_1\parallel\ell_2\). Dans ce cas nous allons prouver que \( f=\tau_{2v}\) où \( v\) est le vecteur perpendiculaire à \(  \ell_1 \) tel que \( \ell_1+v=\ell_2\). Nous allons utiliser le lemme \ref{LEMooVOJLooCFgdNG} pour montrer que \( \sigma_{\ell_1}\circ\sigma_{\ell_2}=\tau_{2v}\). Nous avons
            \begin{subequations}
                \begin{align}
                    \ell_1=\ell_0+w\\
                    \ell_2=\ell_0+w+v
                \end{align}
            \end{subequations}
            où \( w\) est un vecteur perpendiculaire à \( \ell_1\) et \( \ell_0\) est la droite passant par l'origine et parallèle à \( \ell_1\) et \( \ell_2\). Avec cela,
            \begin{subequations}
                \begin{align}
                    (\sigma_{\ell_1}\circ\sigma_{\ell_2})(x)&=\sigma_{\ell_1}\big( \sigma_{\ell_0}(x)+2w \big)\\
                    &=\sigma_{\ell_0}\big( \sigma_{\ell_0}(x)+2w \big)+2(v+w)\\
                    &=x+\underbrace{\sigma_{\ell_0}(2w)}_{-2w}+2v+2w\\
                    &=x+2v.
                \end{align}
            \end{subequations}
            Donc si \( f\) est composée de deux réflexions d'axes parallèles, alors \( f\) est une translation.

            Toujours dans le cas où \( f\) est composée de deux réflexions, nous supposons que \( f=\sigma_{\ell_2}\circ\sigma_{\ell_1}\) avec \( \ell_1\) et \( \ell_2\) non parallèles. Nous notons \( O\) le point d'intersection, et nous allons voir que \( f=R_O(2\alpha)\) où \( \alpha\) est l'angle de \( \ell_1\) à \( \ell_2\) donné par le lemme \ref{DEFooEGKOooRPGOAs}.

            Soit \( x\in \ell_1\). Alors 
            \begin{equation}
                f(x)=\sigma_{\ell_2}(x),
            \end{equation}
            et le lemme \ref{LEMooJLHGooQIpKIE} nous donne un moyen de calculer \( \sigma_{\ell_2}(x)\) parce que \( \ell_1\) est une droite passant par \( x\) et coupant \( \ell_1\) au point \( O\). Le lemme dit que \( \sigma_{\ell_2}(x)=R_O(2\alpha)\). Remarque : c'est bien \( 2\alpha\) et non \( -2\alpha\) parce qu'il s'agit de l'angle de \( \ell_2\) à \( \ell_2\); il y a inversion des numéros entre ici et l'énoncé du lemme.

            Nous avons donc bien \( f(x)=R_O(2\alpha)x\) pour \( x\in \ell_1\).

            Si \( y\in\ell_2\) alors
            \begin{equation}
                f(y)=\sigma_{\ell_2}\big( R_O(-2\alpha)y \big)
            \end{equation}
            Nous posons \( z=\sigma_{\ell_1}(y)=R_O(-2\alpha)y\). Soit la droite \( \ell_3\) passant par \( O\) et \( z\). Vu que \( R_O(2\alpha)z=y\in \ell_2\), l'angle de \( \ell_3\) à \( \ell_2\) est \( 2\alpha\). Par conséquent 
            \begin{equation}
                \sigma_{\ell_2}(z)=R_O\big( -2\times (-2\alpha) \big)z=R_O(4\alpha)z=R_O(4\alpha)R_O(-2\alpha)y=R_O(2\alpha)y.
            \end{equation}
            
            Donc les transformations \( f\) et \( R_O(2\alpha)\) coïncident pour tous les points des droites \( \ell_1\) et \( \ell_2\), qui ne sont pas parallèles. Cela prouve que \( f=R_{O}(2\alpha)\).

        \item[Trois réflexions]
            Nous écrivons \( f=\sigma_{\ell_3}\circ\sigma_{\ell_2}\circ\sigma_{\ell_1}\). Nous allons transformer cela progressivement en une symétrie glissée en passant par plusieurs étapes :
            \begin{enumerate}
                \item       \label{ITEMooHVYCooPhFMiv}
                    \( f=\sigma_{\ell}\circ\tau_v\),
                \item       \label{ITEMooUKGLooFlCcjt}
                    \( f=\tau_v\circ\sigma_{\ell}\),
                \item       \label{ITEMooWUCWooZSjofe}
                    \( f=\tau_v\circ\sigma_{\ell} \) avec \( v\parallel\ell\).
            \end{enumerate}
            À chacune de ces étapes, \( v\) et \( \ell\) vont changer. La dernière est une réflexion glissée.

            Nous commençons par supposer \( \ell_2\parallel\ell_3\). Dans ce cas, \( \sigma_{\ell_3}\circ\sigma_{\ell_2}\) est une translation, comme nous l'avons déjà vu. Alors \( f= \tau_v\circ\sigma_{\ell_1}\) et nous sommes déjà dans le cas \ref{ITEMooUKGLooFlCcjt}.

            Nous supposons que \( \ell_2\) n'est pas parallèle à \( \ell_3\). Dans ce cas, si \( O=\ell_2\cap\ell_3\) nous avons 
            \begin{equation}
                \sigma_{\ell_3}\circ\sigma_{\ell_2}=R_O(2\alpha)
            \end{equation}
            où \( \alpha\) est l'angle de \( \ell_2\) à \( \ell_3\). En réalité tant que l'angle de \( \ell'_3\) à \( \ell'_2\) est \( \alpha\) nous avons
            \begin{equation}
                \sigma_{\ell'_3}\circ\sigma_{\ell'_2}= \sigma_{\ell_3}\circ\sigma_{\ell_2}=R_O(2\alpha).
            \end{equation}
            Nous choisissons \( \ell'_2\) parallèle à \( \ell_1\), de telle sorte à ce que \( \sigma_{\ell'_2}\circ\sigma_{\ell_1}\) soit une translation. Alors nous avons
            \begin{equation}
                f=\sigma_{\ell_3}\circ\sigma_{\ell_2}\circ\sigma_{\ell_1}=\sigma_{\ell_3}\circ\sigma_{\ell'_2}\circ\sigma_{\ell'_1}=\sigma_{\ell_3}\circ\tau_v.
            \end{equation}
            où \( v\) est le vecteur de la translation en question.

            Nous avons donc prouvé que toute composition de trois réflexions peut être écrite soit sous la forme \ref{ITEMooHVYCooPhFMiv} soit sous la forme \ref{ITEMooUKGLooFlCcjt}.

            Nous prouvons à présent que toute transformation de la forme \ref{ITEMooHVYCooPhFMiv} peut être écrite sous la forme \ref{ITEMooUKGLooFlCcjt}. Plus précisément nous allons prouver que si \( \ell\) est une droite, \( v\) un vecteur et \( \ell_0\) la droite parallèle à \( \ell\) passant par l'origine, alors
            \begin{equation}
                \sigma_{\ell}\circ\tau_v=\tau_{\sigma_{\ell_0}(v)}\circ\sigma_l
            \end{equation}
            D'abord nous savons que \( \sigma_{\ell}(x)=\sigma_{\ell_0}(x)+2w\) où \( w\) est le vecteur tel que \( \ell=\ell_0+w\). Ensuite c'est un simple calcul utilisant le fait que \( \sigma_{\ell_0}\) est linéaire :
            \begin{equation}
                (\sigma_{\ell}\circ\tau_v)(x)=\sigma_l(x+v)=\sigma_{\ell_0}(x)+\sigma_{\ell_0}(v)+2w,
            \end{equation}
            et
            \begin{equation}
                (\tau_{\sigma_{\ell_0}(v)}\circ\sigma_{\ell})(x)=\sigma_{\ell_0}(v)+\sigma_{\ell}(x)=\sigma_{\ell_0}(v)+\sigma_{\ell_0}(x)+2w.
            \end{equation}
            L'égalité est faite.

            Nous montrons maintenant que toute transformation de la forme \ref{ITEMooUKGLooFlCcjt} peut être mise sous la forme \ref{ITEMooWUCWooZSjofe}. Soit donc \( f=\tau_v\circ\sigma_{\ell}\) où \( v\) et \( \ell\) ne sont pas spécialement parallèles.

            Pour cela nous décomposons \( v=v_1+v_2\) avec \( v_1\perp \ell\) et \( v_2\parallel\ell\) et nous posons \( \ell'=\ell+\frac{ 1 }{2}v_1\). Nous montrons que
            \begin{itemize}
                \item \( \tau_v\circ\sigma_{\ell}=\tau_{v_2}\circ\sigma_{\ell'}\)
                \item \( v_2\parallel \ell'\).
            \end{itemize}
            Pour le deuxième point, \( v_2\parallel\ell\) et bien entendu \( \ell'\parallel\ell\). Donc \( v_2\parallel\ell'\).

            Soit \( \ell_0\) la droite parallèle à \(  \ell\) et \( \ell'\) et passant par l'origine. Soit aussi le vecteur \( w\) tel que \( \ell=\ell_0+w\). Alors nous avons
            \begin{subequations}
                \begin{numcases}{}
                    \sigma_{\ell}=\sigma_{\ell_0}+2w\\
                    \sigma_{\ell'}=\sigma_{\ell_0}+2w+v_1
                \end{numcases}
            \end{subequations}
            Nous avons
            \begin{equation}
                (\tau_v\circ\sigma_{\ell})(x)=v+\sigma_{\ell_0}(x)+2w
            \end{equation}
            et
            \begin{subequations}
                \begin{align}
                    (\tau_{v_2}\circ\sigma_{\ell'})(x)&=v_2+\sigma_{\ell_0}(x)+2w+v_1\\
                    &=\sigma_{\ell_0}(x)+v+2w
                \end{align}
            \end{subequations}
            où dans la dernière ligne, nous avons regroupé \( v_1+v_2=v\). Et voila.
    \end{subproof}
\end{proof}

\begin{corollary}[\cite{MonCerveau}] \label{CORooVYUJooDbkIFY}
    Les rotations vectorielles forment exactement le groupe \( \SO(2)\).
\end{corollary}

\begin{proof}
    Nous savons déjà de la proposition \ref{PROPooTUJWooAjtEnQ} que les rotations (composées de \( 2\) réflexions, par la définition \ref{DEFooFUBYooHGXphm}) sont toutes des applications de \( \SO(2)\).

    Pour l'autre sens, une application de \( \SO(2)\) est une isométrie et est donc, par le théorème \ref{THOooVRNOooAgaVRN} une des choses suivantes : 
    \begin{itemize}
        \item identité
        \item réflexion
        \item translation
        \item réflexion glissée
        \item rotation (composée de réflexions d'axes non parallèles).
    \end{itemize}
    L'identité est possible et c'est une rotation. La réflexion est impossible parce qu'une réflexion linéaire (non affine) est de déterminant \( -1\) (lemme \ref{LEMooSYZYooWDFScw}). Une translation c'est impossible parce que les éléments de \( \SO(2)\) laissent invariant un point alors que les translations ne laissent aucun point invariant.

    Quid des réflexions glissées ? Vue la forme \eqref{EQooVUQDooKuwszl} de la réflexion, la réflexion glissée \( \tau_v\circ \sigma_{\ell}\) a pour formule
    \begin{equation}
        \alpha(x)=(\tau_v\circ\sigma_{\ell})(x)=v+2\pr_{\ell}(x)-x.    
    \end{equation}
    Nous allons étudier les conditions nécessaires pour que cela décrive un élément de \( \SO(2)\), et nous allons voir que c'est impossible.

    D'abord si \( \ell\) passe par l'origine, alors \( \alpha(0)=v\) et nous devons avoir \( v=0\). Dans ce cas \( \alpha=\sigma_{\ell}\), ce qui n'est pas possible parce que \( \alpha\) doit avoir un déterminant \( 1\) alors que la réflexion a un déterminant \( -1\).

    Donc \( \ell\) ne passe pas par l'origine. La condition \( \alpha(0)=0\) impose \( v=-2\pr_{\ell}(0)\). Alors
    \begin{equation}
        \alpha(x)=-2\pr_{\ell}(0)+2\pr_{\ell}(x)-x.
    \end{equation}
    Le calcul de \( \alpha\big( \pr_{\ell}(0) \big)\) donne :
    \begin{equation}
        \alpha\big( \pr_{\ell}(0) \big)=-2\pr_{\ell}(0)+2\pr_{\ell}\big( \pr_{\ell}(0) \big)-\pr_{\ell}(0).
    \end{equation}
    Vu que la projection de la projection est la projection, cela se réduit à
    \begin{equation}
        \alpha\big( \pr_{\ell}(0) \big)=-\pr_{\ell}(0).
    \end{equation}
    Le vecteur \( \pr_{\ell}(0)\) est donc renversé par l'action de \( \alpha\).

    Considérons une base orthonormale \( \{ e_1,e_2 \}\) de \( E\) où \(e_1 \) est un multiple de norme \( 1\) de \( \pr_{\ell}(0)\). Le déterminant de \( \alpha\) pour cette base est\footnote{La définition \ref{EQooOJEXooXUpwfZ} avec la formule \ref{EQooOJEXooXUpwfZ}.} :
    \begin{equation}
        \det(\alpha)=\langle e_1, \alpha(e_1)\rangle \langle e_2, \alpha(e_2)\rangle -\langle e_1, \alpha(e_2)\rangle \langle e_2, \alpha(e_1)\rangle.
    \end{equation}
    Là dedans nous avons \( \langle e_1, \alpha(e_1)\rangle =-1\) et \( \langle e_2, \alpha(e_1)\rangle =0\), et le tout doit faire \( 1\). Donc
    \begin{equation}
        \langle e_2, \alpha(e_2)\rangle =-1,
    \end{equation}
    ce qui implique que \( \alpha(e_2)=-e_2\). En particulier \( \alpha=-\id\). Nous avons prouvé jusqu'ici que si \( \alpha\) est une réflexion glissée dans \( \SO(2)\) alors \( \alpha=-\id\). 

    Même pas besoin de se poser de grandes questions sur savoir si il est possible de construire une symétrie glissée égale à \( -\id\) (c'est pas possible), il suffit de dire que de toutes façons, \( -\id\) est tout de même une rotation (lemme \ref{LEMooMIJXooCjiQqP}).
    
    Il reste les rotations. Tout élément de \( \SO(2)\) est une rotation.
\end{proof}

%+++++++++++++++++++++++++++++++++++++++++++++++++++++++++++++++++++++++++++++++++++++++++++++++++++++++++++++++++++++++++++ 
\section{Isométries dans \( \eR^n\)}
%+++++++++++++++++++++++++++++++++++++++++++++++++++++++++++++++++++++++++++++++++++++++++++++++++++++++++++++++++++++++++++

\begin{definition}
    Un \defe{hyperplan}{hyperplan} de \( \eR^n\) est un sous-espace affine de dimension \( n-1\). 
\end{definition}
    
\begin{lemmaDef}
    Si un hyperplan \( H\) de \( \eR^n\) est donné, et si \( x\in \eR^n\), il existe un unique point \( y\in \eR^n\) tel que
    \begin{enumerate}
        \item
            \( x-y\perp H\),
        \item
            Le segment \( [x,y]\) coupe \( H\) en son milieu.
    \end{enumerate}
    La \defe{réflexion}{réflexion!par rapport à un hyperplan} \( \sigma_H\) est l'application $\sigma_H\colon \eR^n\to \eR^n $ qui à \( x\) fait correspondre ce \( y\).
\end{lemmaDef}

\begin{proof}
    Il faut vérifier que les conditions données définissent effectivement un unique point de \( \eR^n\). Soit \( H_0\) le sous-espace vectoriel parallèle à \( H\) et une base orthonormée \( \{ e_1,\ldots, e_{n-1} \}\) de \( H_0\). Nous complétons cela en une base orthonormée de \( \eR^n\) avec un vecteur \( e_n\). Si \( H=H_0+v\), quitte à décomposer \( v\) en une partie parallèle et une partie perpendiculaire à \( H\), nous avons
    \begin{equation}
        H=H_0+\lambda e_n
    \end{equation}
    pour un certain \( \lambda\).

    Une droite passant par \( x\) et perpendiculaire à \( H\) est de la forme \( t\mapsto x+te_n\). Si \( x=\sum_{i=1}^{n}x_ie_i\) alors l'unique point de cette droit à être dans \( H\) est le point tel que \(   x_ne_n+te_n=\lambda e_n   \), c'est à dire \( t=-x_n\). L'unique point \( y\) sur cette droite à être tel que \( [x,y ]\) coupe \( H\) en son milieu est celui qui correspond à \( t=-2x_n\). 
\end{proof}

Notons au passage que cette preuve donne une formule pour \( \sigma_H\) :
\begin{equation}        \label{EQooRTWLooLPsUpY}
    \sigma_H(x)=\sum_{i=1}^{n-1}x_ie_i-x_ne_n.
\end{equation}
Il s'agit donc de changer le signe de la composante perpendiculaire à \( H\).

\begin{lemma}       \label{LEMooWYVRooQmWqvM}
    Dans cette même base si \( H_0\) est l'hyperplan parallèle à \( H\) et passant par l'origine, nous écrivons \( H=H_0+\lambda e_n\) pour un certain \( \lambda\). Alors
    \begin{equation}
        \sigma_H=\sigma_{H_0}+2\lambda e_n.
    \end{equation}
\end{lemma}

\begin{proof}
    Un élément \( x\in \eR^n\) peut être décomposé dans la base adéquate en \( x=x_H+x_ne_n\). Nous savons de la formule \eqref{EQooRTWLooLPsUpY} que 
    \begin{equation}
        \sigma_H(x)=x_H-x_ne_n.
    \end{equation}
    Mais vu que \( \sigma_{H_0}(x_H)=x_H-2\lambda e_n\) nous avons
    \begin{equation}
            \sigma_{H_0}(x)+2\lambda e_n=\sigma_{H_0}(x_H+x_ne_n)+2\lambda e_N=x_H-2\lambda e_n-x_ne_n+2\lambda e_n=x_H-x_ne_n.
    \end{equation}
\end{proof}

Le lemme suivant est une généralisation du fait que tous les points de la médiatrice d'un segment sont à égale distance des deux extrémités du segment (très utile lorsqu'on étudie les triangles isocèles).
\begin{lemma}[\cite{ooZYLAooXwWjLa}]        \label{LEMooDPLYooJKZxiM}
    Soient deux points distincts \( x_0,y_0\in \eR^n\) l'ensemble \( H\subset \eR^n\) donné par
    \begin{equation}
        H=\{ x\in \eR^n\tq d(x,x_0)=d(x,y_0) \}.
    \end{equation}
    Alors \( H\) est l'hyperplan orthogonal au vecteur \( v=y_0-x_0\) et \( H\) passe par le milieu du segment \( [x_0,y_0] \).
\end{lemma}

\begin{proof}
    Nous savons que
    \begin{equation}
        d(x,x_0)^2=\langle x-x_0, x-x_0\rangle =\| x \|^2+\| x_0 \|^2-2\langle x, x_0\rangle,
    \end{equation}
    ou encore
    \begin{equation}
        \| x_0 \|^2-\| y_0 \|^2=2\langle x, x_0-y_0\rangle .
    \end{equation}
    En posant \( v=y_0-x_0\) et en considérant la forme linéaire
    \begin{equation}
        \begin{aligned}
            \beta\colon \eR^n&\to \eR \\
            x&\mapsto \langle x, v\rangle , 
        \end{aligned}
    \end{equation}
    Nous avons \( x\in H\) si et seulement si \( \beta(x)=\frac{ 1 }{2}\big( \| y_0 \|^2-\| x_0 \|^2 \big)=\lambda\). En d'autres termes, \( H=\beta^{-1}(\lambda)\). Par la proposition \ref{PROPooAKJBooMkmsiV} la partie \( H\) est un sous-espace affine. C'est même un translaté de \( \ker(\beta)\), et comme \( \ker(\beta)\) est l'espace vectoriel des vecteurs perpendiculaires à \( v\), nous avons \( \dim(H)=\dim\big( \ker(\beta) \big)=n-1\).

    Le fait que \( H\) contienne le milieu du segment \( [x_0,y_0]\) est par définition.
\end{proof}

Pour le lemme suivant, et pour que la récurrence se passe bien nous disons que l'ensemble vide est un espace vectoriel de dimension \( -1\).
\begin{lemma}       \label{LEMooJCDRooGAmlwp}
    Si \( f\in\Isom(\eR^n)\) satisfait 
    \begin{equation}
        \dim\big( \Fix(f) \big)=n-k
    \end{equation}
    alors \( f\) peut être écrit comme composition d'au plus \( k\) réflexions hyperplanes.
\end{lemma}

\begin{proof}
    Nous faisons une récurrence sur \( k\geq 0\). 
    
    Pour l'initialisation, si \( k=0\) alors \( \dim\big( \Fix(f) \big)=n\), c'est à dire que \( f\) fixe tout \( \eR^n\), autant dire que \( f\) est l'identité, une composition de zéro réflexions.

    Pour la récurrence, nous supposons que le lemme est démontré jusqu'à \( k\geq 0\). Soit donc \( f\in\Isom(\eR^n)\) tel que 
    \begin{equation}
        \dim\big( \Fix(f) \big)=n-(k+1).
    \end{equation}
    Vu que \( k\geq 0\), la dimension de \( \Fix(f)\) est strictement plus petite que \( n\), donc il existe un \( x_0\in \eR^n\) tel que \( f(x_0)\neq x_0\). Nous posons
    \begin{equation}
        H=\{ x\in \eR^n\tq d(x,x_0)=d\big( x,f(x_0) \big)  \}.
    \end{equation}
    Par le lemme \ref{LEMooDPLYooJKZxiM}, ce \( H\) est l'hyperplan orthogonal à \( v=f(x_0)-x_0\) et passant par le milieu du segment \( [x_0,f(x_0)]\).

    Nous posons \( g=\sigma_H\circ f\). Vu que \( g(x_0)=\sigma_H(f(x_0))=x_0\), ce \( x_0\) est un point fixe de \( g\). Le fait que \( \sigma_H\big( f(x_0) \big)=x_0\) est vraiment la définition de l'hyperplan \( H\).

    Nous avons donc
    \begin{equation}
        x_0\in\Fix(g)\setminus\Fix(f).
    \end{equation}
    Mais nous prouvons de plus que \( \Fix(f)\subset\Fix(g)\). En effet si \( y\in Fix(f)\) alors \( y\in H\) parce que 
    \begin{equation}
        d(y,x_0)=d\big( f(y),f(x_0) \big)=d\big( y, f(x_0) \big).
    \end{equation}
    Vu que \( y\in H\) nous avons \( y\in \Fix(g)\) parce que 
    \begin{equation}
        g(y)=\sigma_H\big( f(y) \big)=\sigma_H(y)=y.
    \end{equation}
    Tout cela pour dire que l'ensemble \( \Fix(g)\) est \emph{strictement} plus grand que \( \Fix(f)\). Et comme ce sont des espaces affines nous pouvons parler de dimension :
    \begin{equation}
        \dim\big( \Fix(g) \big)>\dim\big( \Fix(f) \big).
    \end{equation}
    Par hypothèse de récurrence, l'application \(  g\) peut être écrite comme composition de \( k\) réflexions. Donc l'application
    \begin{equation}
        f=\sigma_H\circ g
    \end{equation}
    est une composition de \( k+1\) réflexions.
\end{proof}

\begin{lemma}       \label{LEMooMCVKooKzmlAg}
    Soit un hyperplan \( H\) et un vecteur \( v\) de \( \eR^n\). Nous avons
    \begin{equation}
        \tau_v\circ \sigma_H\circ\tau_v^{-1}=\sigma_{\tau_v(H)}.
    \end{equation}
\end{lemma}

\begin{proof}
    Pour ce faire nous considérons une base adaptée. Les vecteurs \( \{ e_1,\ldots, e_{n-1} \}\) forment une base orthonormée de \( H_0\) et \( e_n\) complète en une base orthonormée de \( \eR^n\). Soit \( H_0\) l'hyperplan parallèle à \( H\) et passant par l'origine; nous avons, pour un certain \( \lambda\in \eR\),
    \begin{equation}
        H=H_0+\lambda e_n
    \end{equation}
    D'un autre côté, le vecteur \( v\) peut être décomposé en \( v=v_1+v_2\) où \( v_1\perp H\) et \( v_2\parallel H\). Alors 
    \begin{equation}
        \tau_v(H)=H+v=H+v_2=H_0+\lambda e_n+v_2.
    \end{equation}
    Nous pouvons maintenant utiliser le lemme \ref{LEMooWYVRooQmWqvM} pour exprimer la transformation \( \sigma_{\tau_v(H)}\) :
    \begin{equation}        \label{EQooNYKFooXprXav}
        \sigma_{\tau_v(H)}(x)=\sigma_{H_0}(x)+ 2\lambda e_n+2v_2
    \end{equation}
    
    Mais d'autre part, 
    \begin{equation}
        (\tau_v\circ \sigma_H\circ\tau_{v}^{-1})(x)=v+\sigma_H(x-v)=v+\sigma_{H_0}(x-v)+2\lambda e_n.
    \end{equation}
    Vue la décomposition de \( v=v_1+v_2\) nous avons \( \sigma_{H_0}(v)=-v_1+v_2\) et donc
    \begin{equation}        \label{EQooGOHEooALPRFB}
        (\tau_v\circ \sigma_H\circ\tau_{v}^{-1})(x)= v+  \sigma_{H_0}(x)+v_1-v_2+2\lambda e_n=\sigma_{H_0}+2v_1+2\lambda e_n.
    \end{equation}
    Les expressions \eqref{EQooNYKFooXprXav} et \eqref{EQooGOHEooALPRFB} coïncident, d'où l'égalité recherchée.
\end{proof}

\begin{theorem}[\cite{ooZYLAooXwWjLa}]
    Toute isométrie de \( \eR^n\) peut être écrite comme composition d'au plus \( n+1\) réflexions.
    
    Une isométrie de \( \eR^n\) préserve l'orientation si et seulement si est elle composition d'un nombre pair de réflexions.
\end{theorem}

\begin{proof}
    La première partie de ce théorème n'est rien d'autre que le lemme \ref{LEMooJCDRooGAmlwp} parce que le pire cas est celui où le fixateur de \( f\) est réduit à l'ensemble vide, et dans ce cas l'application \( f\) est une composition de \( n+1\) réflexions.

    Pour la seconde partie nous définissons
    \begin{equation}
        \begin{aligned}
            \epsilon\colon \Isom(\eR^n)&\to \{ \pm 1 \} \\
            \tau_v\circ \alpha&\mapsto \det(\alpha)
        \end{aligned}
    \end{equation}
    où nous nous référons à la décomposition unique d'un élément de \( \Isom(\eR^n)\) sous la forme \( \tau_v\circ \alpha\) avec \( \alpha\in O(n)\) donnée par le théorème \ref{THOooQJSRooMrqQct}\ref{ITEMooEWSIooNKzRxB}.

    Le noyau de \( \epsilon\) est alors la partie 
    \begin{equation}
        \ker(\epsilon)=\eR^n\times_{\AD}\SO(n).
    \end{equation}
    Une isométrie \( f\) préserve l'orientation si et seulement si \( \epsilon(f)=1\). Vu que toutes les isométries sont des composition de réflexions (première partie), il nous suffit de montrer que \( \epsilon(\epsilon_H)=-1\) pour qu'une isométrie préserve l'orientation si et seulement si elle est composition d'un nombre pair de réflexions.

    Nous commençons par prouver que pour tout vecteur \( v\), \( \epsilon\big( \sigma_H \big)=\epsilon\big( \sigma_{\tau_v(H)} \big)\). Pour cela nous utilisons le lemme \ref{LEMooMCVKooKzmlAg} et le fait que \( \epsilon\) est un homomorphisme :
    \begin{equation}
        \epsilon(\sigma_{\tau_v(H)})=\epsilon(\tau_v)\epsilon(\sigma_H)\epsilon(\tau_v^{-1})=\epsilon(\sigma_H)
    \end{equation}
    parce que la partie linéaire d'une translation est l'identité (et donc \( \epsilon(\tau_v)=1\) pour tout \( v\)).

    Nous avons donc \( \epsilon(\sigma_H)=\epsilon(\sigma_{H_0})\). En ce qui concerne \( \sigma_{H_0}\), dans la base adaptée la matrice est
    \begin{equation}
        \sigma_{H_0}=\begin{pmatrix}
             1   &       &       &       \\
                &   \ddots    &       &       \\
                &       &   1    &       \\ 
                &       &       &   -1     
         \end{pmatrix},
    \end{equation}
    dont le déterminant est \( -1\).
\end{proof}

Pour en savoir plus sur le groupe des isométries, il faut lire le théorème de Cartan-Dieudonné dans \cite{JGAdTA}.

%--------------------------------------------------------------------------------------------------------------------------- 
\subsection{Groupes finis d'isomorphismes}
%---------------------------------------------------------------------------------------------------------------------------

\begin{definition}
    Si \( X\) est une partie finie de \( \eR^n\), le \defe{barycentre}{barycentre!cas vectoriel} de \( X\) est le point
    \begin{equation}
        B_X=\frac{1}{ | X | }\sum_{x\in X}x
    \end{equation}
    où \( | X |\) est le cardinal de \( X\).
\end{definition}
Cela est à mettre en relation avec la définition dans le cadre affine \ref{LemtEwnSH}.

\begin{lemma}[\cite{ooZYLAooXwWjLa}]
    Soit une partie finie \( X\) de \( \eR^n\) et une application affine \( f\in\Aff(\eR^n)\). Alors
    \begin{equation}
        f(B_X)=B_{f(X)}.
    \end{equation}
\end{lemma}

\begin{proof}
    Nous savons que toute application affine est une composée de translation et d'une application linéaire : \( f=\tau_v\circ g\) avec \( v\in \eR^n\) et \( g\in \GL(n,\eR)\). Nous vérifions le résultat séparément pour \( \tau_v\) et pour \( g\).

    D'une part,
    \begin{equation}
        B_{\tau_v(X)}=\frac{1}{ | \tau_v(X) | }\sum_{y\in \tau_v(X)}y=\frac{1}{ | X | }\sum_{x\in X}(x+v)=B_x+\frac{1}{ | X | }\sum_{x\in X}v=B_x+v=\tau_v(B_X).
    \end{equation}
    Nous avons utilisé le fait que \( X\) et \( \tau_v(X)\) possèdent le même nombre d'éléments, ainsi que le fait d'avoir une somme de \( | X |\) termes tous égaux à \( v\).

    D'autre part,
    \begin{equation}
        B_{g(X)}=\frac{1}{ | X | }\sum_{x\in X}g(x)=g\big( \frac{1}{ |X | }\sum_{x\in X}x \big)=g(B_X)
    \end{equation}
    où nous avons utilisé la linéarité de \( g\) dans tous ses retranchements.
\end{proof}

\begin{proposition}     \label{PROPooLAEBooWdcBoe}
    Points fixes d'un sous-groupe.
    \begin{enumerate}
        \item
            Soit \( H\) un sous-groupe finie de \( \Isom(\eR^n)\). Alors il existe \( v\in \eR^n\) tel que \( f(v)=v\) pour tout \( f\in H\).
        \item
            Si \( H\) est un sous-groupe de \( \Isom(\eR^n)\) n'acceptant pas de points fixes, alors il est infini.
    \end{enumerate}
\end{proposition}

\begin{proof}
    Le groupe \( H\) agit sur \( \eR^n\), et si \( x\in \eR^n\) nous pouvons considérer son orbite \( Hx\), qui est une partie finie de \( \eR^n\). Considérons son barycentre
    \begin{equation}
        v=B_{Hx}
    \end{equation}
    Soit \( f\in H\). Alors \( f(v)=f(B_{Hx})=B_{f(Hx)}=B_{Hx}=v\), donc \( v\) est fixé par \( H\).

    La seconde affirmation n'est rien d'autre que la contraposée de la première.
\end{proof}

\begin{proposition}     \label{PROPooEUFIooDUIYzi}
    À propos de groupes finis d'isométries.
    \begin{enumerate}
        \item
            Tout sous groupe finie de \( \Isom(\eR^n)\) est isomorphe à un sous-groupe fini de \( \gO(n)\).
        \item
            Tout sous-groupe fini de \( \Isom^+(\eR^n)\) est isomorphe à un sous-groupe fini de \( \SO(n)\).
    \end{enumerate}
\end{proposition}

\begin{proof}
    Soit \( H\) un sous-groupe fini de \( \Isom(\eR^n)\) et \( v\in \eR^n\) un élément fixé par \( H\) (comme garantit par la proposition \ref{PROPooLAEBooWdcBoe}). Nous posons
    \begin{equation}
        \begin{aligned}
            \phi\colon H&\to \Isom(\eR^n) \\
            f&\mapsto \tau_v^{-1}\circ f\circ \tau_v. 
        \end{aligned}
    \end{equation}

    \begin{subproof}
        \item[\( \phi\) est un homomorphisme]
            Les opération du type \( \phi=\AD(\tau_v)\) sont toujours des homomorphismes.
        \item[\( \phi\) consiste à extraire la partie linéaire]
            Si \( f=\tau_w\circ g\) alors
            \begin{subequations}
                \begin{align}
                    \phi(f)(x)&=(\tau_{-v}\circ\tau_w\circ g\circ\tau_v)(x)\\
                    &=\tau_{w-v}(   g(x)+g(v)  )\\
                    &=g(x)+g(v)-v+w
                \end{align}
            \end{subequations}
            Mais \( g(v)+w=f(v)\) et nous savons que \( f(v)=v\). Donc il ne reste que \( \phi(f)(x)=g(x)\).
        \item[\( \phi\) est injective]
            Si \( f=\tau_w\circ g\) vérifie \( \phi(f)=\id\), il faut en particulier que \( g=\id\). Mais \( H\) est fini et ne peut donc pas contenir de translations non triviales. Donc \( w=0\) et \( f=\id\).
    \end{subproof}
    Donc \( \phi\) est une injection à valeur dans les transformation linéaires de \( \Isom(\eR^n)\). Autrement dit, \( \phi\) est un isomorphisme entre \( H\) et son image, laquelle image est dans \( \gO(n)\).

    En ce qui concerne la seconde partie, si \( f\in\Isom^+(\eR^n)\), alors \( \phi(f)\) y est aussi, tout en étant linéaire. Donc \( \phi(f)\in \SO(n)\).
\end{proof}

L'extraction de la partie linéaire est injective ? Certe c'est prouvé, mais on peut se demander ce qu'il se passe si \( H\) contient deux éléments qui ont la même partie linéaire. Cela n'est pas possible parce si \( f_1=\tau_{w_1}\circ g\) et \( f_2=\tau_{w_2}\circ g\) sont dans \( H\) alors \( f_1f_2^{-1}=\tau_{w_1+w_2}\) est également dans \( H\), ce qui n'est pas possible si \( H\) est fini.

\begin{definition}[Groupe de symétrie d'une partie de \( \eR^n\)\cite{ooZYLAooXwWjLa}]
    Si \( Y\) est une partie de \( \eR^n\), nous définissons le \defe{groupe des symétries}{groupe!des symétries} de \( Y\) par
    \begin{equation}
        \Sym(Y)=\{ f\in\Isom(\eR^n)\tq f(Y)=Y \}.
    \end{equation}
    Nous définissons aussi le groupe des symétries propres de \( Y\) par
    \begin{equation}
        \Sym^+(Y)=\{ f\in\Isom^+(\eR^n)\tq f(Y)=Y \}.
    \end{equation}
\end{definition}

\begin{theorem}[\cite{ooZYLAooXwWjLa}]
    Soit \( Y\subset \eR^2\) tel que le groupe \( \Sym^+(Y)\) soit fini d'ordre \( n\). Alors c'est un groupe cyclique d'ordre \( n\).

    Si \( \Sym^+(Y)\) est fini, alors \( \Sym(Y)\) est soit cyclique d'ordre \( n\) soit isomorphe au groupe diédral d'ordre \( 2n\).
\end{theorem}

\begin{proof}
    Nous savons déjà par la proposition \ref{PROPooEUFIooDUIYzi} que \( \Sym^+(Y)\) est isomorphe à un sous-groupe \( H^+\) d'ordre \( n\) de \( \SO(2)\). Vérifions que ce groupe est cyclique. Si \( n=1\), c'est évident. Si \( n\geq 2\) alors nous savons que \( H^+\) est constitué de rotations d'angles dans \( \mathopen[ 0 , 2\pi \mathclose[\) et vu que c'est un ensemble fini, il possède une rotation d'angle minimal (à part zéro). Notons \( \alpha_0\) cet angle.

        Nous montrons que \( H^+\) est engendré par la rotation d'angle \( \alpha_0\). Soit une rotation d'angle \( \alpha\). Étant donné que \( \alpha_0<\alpha\) nous pouvons effectuer la division euclidienne\footnote{Théorème \ref{ThoDivisEuclide}.} de \( \alpha\) par \( \alpha_0\) et obtenir
        \begin{equation}
            \alpha=k\alpha_0+\beta
        \end{equation}
        avec \( \beta<\alpha_0\). Mézalors \( R(\beta)=R(\alpha)R(\alpha_0)^{-k}\) est également un élément du groupe. Cela contredit la minimalité dès que \( \beta\neq 0\). Avoir \( \beta=0\) revient à dire que \( \alpha\) est un multiple de \( \alpha_0\), ce qui signifie que le groupe \( H^+\) est cyclique engendré par \( \alpha_0\). 

        Notons au passage que nous avons automatiquement \( \alpha_0=\frac{ 2\pi }{ n }\) parce qu'il faut \( R(\alpha_0)^n=\id\). Nous avons prouvé que \( \Sym^+(Y) \) est cyclique d'ordre \( n\).

        Nous étudions maintenant le groupe \( \Sym(Y)\). Par la proposition \ref{PROPooEUFIooDUIYzi} nous avons un homomorphisme injectif
        \begin{equation}
            \phi\colon \Sym(Y)\to \gO(2),
        \end{equation}
        et en posant \( H=\phi\big( \Sym(Y) \big)\) nous avons un isomorphisme de groupes \( \phi\colon \Sym(Y)\to H\). Nous savons aussi que ce \( \phi\) se restreint en
        \begin{equation}
            \phi\colon   \Sym^+(Y) \to H^+\subset\SO(2)
        \end{equation}
        où \( H^+=\phi\big( \Sym^+(Y) \big)=H\cap\SO(2)\). Le groupe \( H^+\) est cyclique et est engendré par la rotation \( R(2\pi/n)\).

        Supposons un instant que \( H\subset \SO(2)\). Alors nous avons \( H=H^+\) et \( \phi\) est un isomorphisme entre \( \Sym(Y)\) et le groupe cyclique engendré par \( R(2\pi/n)\).

        Nous supposons à présente que \( H\) n'est pas un sous-ensemble de \( \SO(2)\). Quelles sont les isométries de \( \eR^2\) qui ne sont pas de déterminant \( 1\) ? Il faut regarder dans le théorème \ref{THOooVRNOooAgaVRN} quelles sont les isométries contenant un nombre impair de réflexions. Ce sont les réflexions et les réflexions glissées. Or il ne peut pas y avoir de réflexions glissées dans un groupe fini parce que si \( f\) est une réflexion glissée, tous les \( f^k\) sont différents.

        Nous en déduisons que si \( H\) n'est pas inclus à \( \SO(2)\), il contient une réflexion que nous nommons \( \sigma\). Nous allons en déduire que \( H\simeq H^+\times_{\AD}C_2\) où \( C_2=\{ \id,\sigma \}\). Si \( h\in H\) nous pouvons écrire 
        \begin{equation}
            h=(h\sigma^{\epsilon})\sigma^{\epsilon}
        \end{equation}
        pour n'importe quelle valeur de \( \epsilon\), et en particulier pour \( \epsilon=\pm 1\). 

        Si \( h\in \SO(2)\) alors nous écrivons \( h=h\epsilon^{0}\) et si \( h\notin\SO(2)\) nous écrivons \( h=(h\sigma)\sigma\). Vu que \( h\sigma\in\SO(2)\), cette dernière écriture est encore de la forme \( \SO(2)\times C_2\). Quoi qu'il en soit tout élément de \( H\) s'écrit comme un produit 
        \begin{equation}
            H=H^+C_2.
        \end{equation}
        Cette décomposition est unique parce que si \( h_1c_1=h_2c_2\) alors \( h_2^{-1}h_1=c_2c_1^{-1}\), et comme \( h_2^{-1}h_1\in H^+\) nous avons \( c_2c_1^{-1}\in H^+\) et donc \( c_1=c_2\). Partant nous avons aussi \( h_1=h_2\). Pour avoir le produit semi-direct il faut encore montrer que \( \AD(C_2)H^+\subset H^+\). Le seul cas à vérifier est \( \AD(\sigma)H^+\subset H^+\). Vu que les éléments de \( H^+\) sont caractérisés par le fait d'avoir un déterminant positif, nous avons 
        \begin{equation}
            \AD(\sigma)R(\alpha)=\sigma R(\alpha)\sigma^{-1}\in H^+.
        \end{equation}
\end{proof}

\begin{remark}
    Tout ceci est cohérent avec le théorème de Burnside \ref{ThooJLTit} parce que le sous-groupe fini de \( \SO(n)\) engendré par la rotation \( R(2\pi/n)\) est un groupe d'exposant fini, à savoir que si \( h\) est dans ce groupe, \( h^n=\id\).
\end{remark}

%--------------------------------------------------------------------------------------------------------------------------- 
\subsection{Théorème de Sylvester}
%---------------------------------------------------------------------------------------------------------------------------

% TODO : Il y a une démonstration sur wikipédia, à voir.

\begin{theorem}[de Sylvester]   \label{ThoQFVsBCk}
    Soit $Q$ une forme quadratique réelle de signature \( (p,q)\). Alors pour toute base orthonormée on a
    \begin{subequations}
        \begin{align}
            p&=\Card\{ i\tq Q(e_i)>0 \}\\
            q&=\Card\{ i\tq Q(e_i)<0 \}.
        \end{align}
    \end{subequations}
    Le rang de \( Q\) est \( p+q\).

    Si \( A\) est la matrice de \( Q\) dans une base, alors il existe une matrice inversible \( P\) telle que
    \begin{equation}
        P^tAP=\begin{pmatrix}
            -\mtu_q    &       &       \\
                &   \mtu_p    &       \\
                &       &   0
        \end{pmatrix}.
    \end{equation}
\end{theorem}
\index{théorème!Sylvester}
\index{rang}
\index{matrice!semblables}
\index{forme!quadratique}

%---------------------------------------------------------------------------------------------------------------------------
\subsection{Groupe diédral}
%---------------------------------------------------------------------------------------------------------------------------
\label{subsecHibJId}

%///////////////////////////////////////////////////////////////////////////////////////////////////////////////////////////
\subsubsection{Définition et générateurs : vue géométrique}
%///////////////////////////////////////////////////////////////////////////////////////////////////////////////////////////

\begin{definition}  \label{DEFooIWZGooAinSOh}
    Le \defe{groupe diédral}{groupe!diédral} \( D_n\)\nomenclature[R]{\( D_n\)}{groupe diédral} est le groupe des isométries de \( \eR^2\) laissant invariant un polygone régulier à \( n\) côtés. 
\end{definition}
Le groupe diédral peut être vu comme le stabilisateur de l'ensemble
\begin{equation}
    \{  e^{2ik\pi/n},k=0,\ldots, n-1 \}
\end{equation}
dans le groupe des isométries affines de \( \eC^*\).
\index{groupe!agissant sur un ensemble!diédral}
\index{groupe!en géométrie}
\index{groupe!fini!diédral}
\index{groupe!permutation!diédral}
% TODO : prouver que les racines de l'unité forment un polygone régulier.

Si \( f\in D_n\), alors \( f( e^{2ik\pi/n}) \) doit être l'un des \(  e^{2ik'\pi/n}\), et vu que \( f\) conserve les longueurs dans \( \eC\), nous devons avoir
\begin{equation}
    1=d(0, e^{2ik\pi/n})=d\big( f(0), e^{2ik'\pi/n} \big).
\end{equation}
Donc \( f(0)\) est à l'intersection de tous les cercles de rayon \( 1\) centrés en les \(  e^{2ik\pi/n}\), ce qui montre que \( f(0))0\) (dès que \( n\geq 3\)). Par conséquent notre étude du groupe diédral ne doit prendre en compte que les isométries vectorielles de \( \eR^2\). En d'autres termes
\begin{equation}
    D_n\subset O(2,\eR).
\end{equation}

\begin{proposition}[\cite{tzHydF}]
    Le groupe \( D_n\) contient un sous groupe cyclique d'ordre \( 2\) et un sous groupe cyclique d'ordre \( n\).
\end{proposition}

\begin{proof}
    Si \( s\) est la réflexion d'axe \( \eR\), alors \( s\) est d'ordre \( 2\). De plus \( s\) est bien dans \( D_n\) parce que
    \begin{equation}    \label{EqSUshknP}
        s\big(  e^{2ki\pi/n} \big)= e^{2(n-k)i\pi/n}.
    \end{equation}

    De la même façon, la rotations d'angle \(2\pi/n\), que l'on note \( r\), agit sur les racines de l'unité et engendre un le groupe d'ordre \( n\) des rotations d'angle \(2 k\pi/n\).
\end{proof}

Notons que la conjugaison complexe ne fait pas spécialement partie du groupe \( D_n\). En effet pour \( n=3\) par exemple les points fixes sont \( A_1=(1,0)\), \( A_2=(-\frac{ 1 }{2},\frac{ \sqrt{3} }{2})\) et \( A_3=(\frac{ 1 }{2},-\frac{ \sqrt{3} }{2})\). La conjugaison complexe envoie évidemment \( A_1\) sur \( A_1\), mais pas du tout \( A_2\) sur \( A_3\).
%TODO : un dessin du triangle équilatéral serait pas mal ici.

\begin{proposition}[\cite{tzHydF}]
    Nous avons \( (sr)^2=\id\).
\end{proposition}

\begin{proof}
    Si \( z^n=1\), alors
    \begin{equation}
        (srsr)z=srs e^{2 i\pi/n}z=sr\big( e^{-2\pi i/n\bar z}\big)=s\bar z=z.
    \end{equation}
\end{proof}

\begin{proposition}[\cite{tzHydF}] \label{PropLDIPoZ}
    Le groupe diédral \( D_n\) est engendré par \( s\) et \( r\). De plus tous les éléments de \( D_n\) s'écrivent sous la forme \( s\circ r^m\).
\end{proposition}
\index{groupe!diédral!générateurs (preuve)}
\index{racine!de l'unité}
\index{géométrie!avec nombres complexes}
\index{géométrie!avec des groupes}
\index{isométrie!de l'espace euclidien \( \eR^2\)}

\begin{proof}
    Nous considérons les points \( A_0=1\) et \( A_k= e^{2ki\pi/n}\) avec \( k\in\{ 1,\ldots, n-1 \}\). Par convention, \( A_n=A_0\). L'action des éléments \( s\) et \( r\) sur ces points est
    \begin{subequations}
        \begin{align}
            r(A_k)&=A_{k+1}\\
            s(A_k)&=A_{n-k}.
        \end{align}
    \end{subequations}
    Cette dernière est l'équation \eqref{EqSUshknP}.
    
    Soit \( f\in D_n\). Étant donné que c'est une isométrie de \( \eR^2\) avec un point fixe (le point \( 0\)), \( f\) est soit une rotation soit une réflexion.
    %TODO : il faut démontrer ce point et mettre un lien vers ici.

    Supposons pour commencer que un des \( A_k\) est fixé par \( f\). Dans ce cas \( f\) a deux points fixes : \( O\) et \( A_k\) et est donc la réflexion d'axe \( (OA_k)\). Dans ce cas, nous avons \( f=s\circ r^{n-2k}\). En effet
    \begin{equation}
        s\circ r^{n-2k}(A_k)=s(A_{k+n-2k})=s(A_{n-k})=A_k.
    \end{equation}
    Donc \( O\) et \( A_k\) sont deux points fixes de l'isométrie \( f\); donc \( f\) est bien la réflexion sur le bon axe.

    Nous passons à présent au cas où \( f\) ne fixe aucun des \( A_k\). 
    \begin{enumerate}
        \item
            Supposons que \( f\) soit une rotation. Si \( f(A_k)=A_m\), alors l'angle de la rotation est 
            \begin{equation}
                \frac{ 2(m-k)\pi }{ n },
            \end{equation}
            et donc \( f=r^{m-k}\), qui est de la forme demandée.
        \item
            Supposons à présent que \( f\) soit une réflexion d'axe \( \Delta\). Cette fois, \( \Delta\) ne passe par aucun des points \( A_k\), par contre \( \Delta\) passe par \( 0\). Nous commençons par montrer que \( \Delta\) doit être la médiatrice d'un des côtés \( [A_p,A_{p+1}]\) du polygone. Vu que \( \Delta\) passe par \( O\) et n'est aucune des droites \( (OA_k)\), cette droite passe par l'intérieur d'un des triangles \( OA_pA_{p+1}\) et intersecte donc le côté correspondant.

            Notre tâche est de montrer que \( \Delta\) coupe \( [A_p,A_{p+1}]\) en son milieu. Dans ce cas, \( \Delta\) sera automatiquement perpendiculaire parce que le triangle \( OA_pA_{p+1}\) est isocèle en \( O\). Nommons \( l\) la longueur des côtés du polygone, \( P=\Delta\cap[A_p,A_{p+1}]\), \( x=d(A_p,P)\) et \( \delta=d(A_p,\Delta)\). Vu que \( f\) est la symétrie d'axe \( \Delta\), nous avons aussi \( d\big( f(A_p),\Delta \big)=\delta\) et \( d\big( A_p,f(A_p) \big)=2\delta\). D'autre part, par la définition de la distance, \( \delta<x\). Si \( x<\frac{ l }{2}\), alors \( \delta<\frac{ \delta }{2}\) et donc \( d\big( A_p,f(A_p) \big)<l\). Or cela est impossible parce que le polygone ne possède aucun sommet à distance plus courte que \( l\) de \( A_p\).

            De la même manière si \( x>\frac{ l }{2}\), nous raisonnons avec \( A_{p+1}\) pour obtenir une contradiction. Nous en concluons que la seule possibilité est \( x=\frac{ l }{2}\), et donc \( f(A_p)=A_{p+1}\). Montrons alors que \( f=s\circ r^{n-2p-1}\). Il faut montrer que c'est une réflexion qui envoie \( A_p\) sur \( A_{p+1}\). D'abord c'est une réflexion parce que
            \begin{equation}
                \det(sr^{n-2p-1})=\det(s)\det(r^{n-2p-1})=-1
            \end{equation}
            parce que \( \det(s)=-1\) alors que \( \det(r^k)=1\) parce que \( r\) est une rotation dans \( \SO(2)\). Ensuite nous avons
            \begin{equation}
                s\circ r^{n-2p-1}(A_p)=s(A_{p+n-2p-1})=s(A_{n-p-1})=A_{n-(n-p-1)}=A_{p+1}.
            \end{equation}

            Donc \( s\circ r^{n-2p-1}\) est bien une réflexion qui envoie \( A_p\) sur \( A_{p+1}\).

    \end{enumerate}
\end{proof}

\begin{corollary}   \label{CorWYITsWW}
La liste des éléments de \( D_n\) est 
\begin{equation}
    D_n=\{ 1,r,\ldots, r^{n-1},s,sr,\ldots, sr^{n-1} \}
\end{equation}
et \( | D_n |=2n\).
\end{corollary}

\begin{proof}
    Nous savons par la proposition \ref{PropLDIPoZ} que tous les élément de \( D_n\) s'écrivent sous la forme \( r^k\) ou \( sr^k\). Vu que \( r\) est d'ordre \( n\), il ne faut considérer que \( k\in\{ 1,\ldots, n-1 \}\). Les éléments \( 1\), \( r\),\ldots, \( r^{n-1}\) sont tous différents, et sont (pour des raisons de déterminant) tous différents des \( sr^k\). Les isométries \( sr^k\) sont toutes différentes entre elles pour essentiellement la même raison :
    \begin{equation}
        sr^k(A_p)=s(A_{p+k})=A_{n-p+k}
    \end{equation}
    donc si \( k\neq k'\), \( sr^k(A_p)\neq sr^{k'}(A_p)\). La liste des éléments de \( D_n\) est donc
    \begin{equation}
        D_n=\{ 1,r,\ldots, r^{n-1},s,sr,\ldots, sr^{n-1} \}
    \end{equation}
    et donc \( | D_n |=2n\).
\end{proof}

\begin{example}     \label{EXooHNYYooUDsKnm}
    Nous considérons le carré \( ABCD\) dans \( \eR^2\) et nous cherchons les isométries de \( \eR^2\) qui laissent le carré invariant. Nous nommons les points comme sur la figure \ref{LabelFigIsomCarre}. La symétrie d'axe vertical est nommée \( s\) et la rotation de \( 90\) degrés est notée \( r\).
    \newcommand{\CaptionFigIsomCarre}{Le carré dont nous étudions le groupe diédral.}
    \input{auto/pictures_tex/Fig_IsomCarre.pstricks}

    Il est facile de vérifier que toutes les symétries axiales peuvent être écrites sous la forme \( r^is\). De plus le groupe engendré par \( s\) agit sur le groupe engendré par \( r\) parce que
    \begin{equation}
        (srs^{-1})(A,B,C,D)=sr(B,A,D,C)=s(A,D,C,B)=(B,C,D,A),
    \end{equation}
    c'est à dire \( srs^{-1}=r^{-1}\). Nous sommes alors dans le cadre du corollaire \ref{CoroGohOZ} et nous pouvons écrire que
    \begin{equation}
        D_4=\gr(r)\times_{\sigma}\gr(s).
    \end{equation}
\end{example}

%///////////////////////////////////////////////////////////////////////////////////////////////////////////////////////////
\subsubsection{Générateurs : vue abstraite}
%///////////////////////////////////////////////////////////////////////////////////////////////////////////////////////////

Nous allons montrer que \( D_n\) peut être décrit de façon abstraite en ne parlant que de ses générateurs. Nous considérons un groupe \( G\) engendré par des éléments \( a\) et \( b\) tels que
\begin{enumerate}
    \item
        \( a\) est d'ordre \( 2\),
    \item
        \( b\) est d'ordre \( n\) avec \( n\geq 3\),
    \item
        \( abab=e\).
\end{enumerate}
Nous allons prouver que ce groupe doit avoir la même liste d'éléments que celle du corollaire \ref{CorWYITsWW}.

\begin{proposition}[\cite{tzHydF}]
    Le groupe \( G\) n'est pas abélien.
\end{proposition}

\begin{proof}
    Nous savons que \( abab=e\), donc \( abab^{-1}=b^{-2}\), mais \( b^{-2}\neq e\) parce que \( b\) est d'ordre \( n>2\). Donc \( abab^{-1}\neq e\). En manipulant un peu :
    \begin{equation}
        e\neq abab^{-1}=(ab)(ba^{-1})^{-1}=(ab)(ba)^{-1}
    \end{equation}
    parce que \( a^{-1}=a\). Donc \( ab\neq ba\).
\end{proof}

\begin{lemma}[\cite{tzHydF}]        \label{LemKKXdqdL}
    Pour tout \( k\) entre \( 1\) et \( n-1\) nous avons
    \begin{equation}
        \AD(a)b^k=ab^ka^{-1}=ab^ka=b^{-k}.
    \end{equation}
\end{lemma}

\begin{proof}
    Nous faisons la démonstration par récurrence. D'abord pour \( k=1\), nous devons avoir \( aba=b^{-1}\), ce qui est correct parce que par construction de \( G\) nous avons \( abab=e\). Ensuite nous supposons que le lemme tient pour \( k\) et nous regardons ce qu'il se passe avec \( k+1\) :
    \begin{equation}
            ab^{k+1}ba=ab^kba=\underbrace{ab^ka}_{b^{-k}}\underbrace{aba}_{b^{-1}}=b^{-k}b^{-1}=b^{-(k+1)}.
    \end{equation}
\end{proof}

\begin{proposition}     \label{PROPooVQARooWuKHMZ}
    L'élément \( a\) n'est pas une puissance de \( b\).
\end{proposition}

\begin{proof}
    Supposons le contraire : \( a=b^k\). Dans ce cas nous aurions
    \begin{equation}
        e=(ab)(ab)=b^{k+1}b^{k+1}=b^{2k+2}=b^{2k}b^2=a^2b^2=b^2,
    \end{equation}
    ce qui signifierait que \( b\) est d'ordre \( 2\), ce qui est exclu par construction.
\end{proof}

\begin{proposition}[\cite{tzHydF}]      \label{PROPooEPVGooQjHRJp}
    La liste des éléments de \( G\) est donnée par
    \begin{equation}
        G=\{ 1,b,\cdots,b^{n-1},a,ab,\ldots, ab^{n-1} \}=\{ a^{\epsilon}b^k\}_{\substack{\epsilon=0,1\\k=0,\ldots, n-1}}
    \end{equation}
    Les éléments de ces listes sont distincts.
\end{proposition}

\begin{proof}
    Étant donné que \( a\) n'est pas une puissance de \( b\), les éléments \( 1\), \( a\), \( b\),\ldots, \( b^{n-1}\) sont distincts. De plus si \( k\) et \( m=k+p\) sont deux éléments distincts de \( \{ 1,\ldots, n-1 \}\), nous avons \( ab^k\neq ab^m\) parce que si \( ab^k=ab^{k+p}\), alors \( a=ab^p\) avec \( p<n\), ce qui est impossible. Pour la même raison, \( ab^k\neq e\), et \( ab^k\neq b^m\).

    Au final les éléments \( 1,a,b,\ldots, b^{n-1},ab,\ldots, ab^{n-1}\) sont tous différents. Nous devons encore voir qu'il n'y en a pas d'autres.

    Par définition le groupe \( G\) est engendré par \( a\) et \( b\), donc tout élément \( x\in G\) s'écrit $x=a^{m_1}b^{k_1}\ldots a^{m_r}b^{k_r}$ pour un certain \( r\) et avec pour tout \( i\), \( k_i\in\{ 1,\ldots, n-1 \}\) (sauf \( k_r\) qui peut être égal à zéro) et \( m_i=1\), sauf \( m_1\) qui peut être égal à zéro. Donc
    \begin{equation}
        x=a^mb^{k_1}ab^{k_2}a\ldots b^{k_{r-1}}ab^{k_r}
    \end{equation}
    où \( m\) et \( k_r\) peuvent éventuellement être zéro. En utilisant le lemme \ref{LemKKXdqdL} sous la forme \( b^{k_i}a=ab^{-k_i}\), quitte à changer les valeurs des exposants, nous pouvons passer tous les \( a \) à gauche et tous les \( b\) à droite pour finir sous la forme \( x=a^kb^m\). 

    Donc non, il n'existe pas d'autres éléments dans \( G\) que ceux déjà listés.
\end{proof}

\begin{lemma}[\cite{MonCerveau}]        \label{LemooNFRIooPWuikH}
    Tout élément de \( G\) s'écrit de façon unique sous la forme \( a^{\epsilon}b^k\) ou \( b^ka^{\epsilon}\) avec \( \epsilon=0,1\) et \( k=0,\ldots, n-1\).
\end{lemma}

\begin{proof}
    Nous commençons par la forme \( a^{\epsilon}b^k\). L'existence est la proposition \ref{PROPooEPVGooQjHRJp}. Pour l'unicité nous supposons \( a^{\epsilon}b^k=a^{\sigma}b^l\) et nous décomposons en \( 4\).
    \begin{subproof}
        \item[\( \epsilon=0\), \( \sigma=0\)]
            Alors \( b^k=b^l\). Mais \( b\) étant d'ordre \( n\) et \( k,l\) étant égaux au maximum à \( n-1\), cette égalité implique \( k=l\).
        \item[\( \epsilon=0\), \( \sigma=1\)]
            Alors \( b^k=ab^l\), ce qui donne \( a=b^{k-l}\), ce qui est interdit par la proposition \ref{PROPooVQARooWuKHMZ}.
        \item[\( \epsilon=1\), \( \sigma=0\)]
            Même problème.
        \item[\( \epsilon=1\), \( \sigma=1\)]
            Encore une fois \( b^k=b^l\) implique \( k=l\).
    \end{subproof}
    En ce qui concerne la forme \( b^ka^{\epsilon}\), l'existence est à montrer. Soit l'élément \( g=a^{\epsilon}b^k\) et cherchons à le mettre sous la forme \( b^la^{\sigma}\). Si \( \epsilon=0\) c'est évident. Sinon \( \epsilon=1\) et nous avons par le lemme \ref{LemKKXdqdL}
    \begin{equation}
        ab^k=b^{-k}a^{-1}=b^{-k}b^na=b^{-k}a.
    \end{equation}
    En ce qui concerne l'unicité, nous refaisons \( 4\) cas pour \( b^ka^{\epsilon}=b^la^{\sigma}\) comme précédemment et ils se traitement exactement comme précédemment.
\end{proof}

\begin{theorem}
    Les groupes \( G\) et \( D_n\) sont isomorphes.
\end{theorem}

\begin{proof}
        Nous utilisons l'application
    \begin{equation}
        \begin{aligned}
            \psi\colon G&\to D_n \\
            a^kb^m&\mapsto s^kr^m. 
        \end{aligned}
    \end{equation}
    C'est évidemment bien défini et bijectif, mais c'est également un homomorphisme parce que si nous calculons \( \psi\) sur un produit, nous devons comparer
    \begin{equation}        \label{EqBULPilp}
        \psi\big( a^{k_1}b^{m_1}a^{k_2}b^{m_2} \big)
    \end{equation}
    avec
    \begin{equation}        \label{EqIVEIphI}
        \psi\big( a^{k_1}b^{m_1}\big)\psi\big(a^{k_2}b^{m_2} \big)= s^{k_1}r^{m_1}s^{k_2}r^{m_2}.
    \end{equation}
    Vu que \( D_n\) et \( G\) ont les mêmes propriétés qui permettent de permuter \( a\) et \( b\) ou \( s\) et \( r\), l'expression à l'intérieur du \( \psi\) dans \eqref{EqBULPilp} se simplifie en \( a^kb^m\) avec les même \( k\) et \( n\) que l'expression à droite dans \eqref{EqIVEIphI} ne se simplifie en \( s^kr^m\).
\end{proof}

\begin{corollary}
    Toutes les propriétés démontrées pour \( G\) sont vraies pour \( D_n\). En particulier, avec quelques redites :
    \begin{enumerate}
        \item
            Le groupe \( D_n\) peut être défini comme étant le groupe engendré par un élément \( s\) d'ordre \( 2\) et un élément \( r\) d'ordre \( n-1\) assujettis à la relation \( srsr=e\).
        \item
            Le groupe \( D_n\) n'est pas abélien.
        \item
            Pour tout \( k\in\{ 1,\ldots, n-1 \}\) nous avons \( sr^ks=r^{-k}\).
        \item
            L'élément \( s\) ne peut pas être obtenu comme une puissance de \( r\).
        \item
            La liste des éléments de \( D_n\) est
            \begin{equation}
                D_n=\{ 1,r,\ldots, r^{n-1},s,sr,\ldots, sr^{n-1} \}
            \end{equation}
        \item
            Le groupe diédral \( D_n\) est d'ordre \( 2n\).
    \end{enumerate}
\end{corollary}

\begin{proposition}
    En posant \( C_n=\{ r^k \}_{k=0,\ldots, n-1}\) et \( C_2=\{ a^{\epsilon} \}_{\epsilon=0,1}\), nous pouvons exprimer \( D_n\) comme le produit semi-direct
    \begin{equation}
        D_n=C_n\times_{\rho}C_2
    \end{equation}
    où \( \rho\) désigne l'action adjointe.
\end{proposition}

\begin{proof}
    L'isomorphisme est :
    \begin{equation}
        \begin{aligned}
            \psi\colon C_n\times_{\rho}C_2&\to D_n \\
            (b^k,a^{\epsilon})&\mapsto b^ka^{\epsilon}.
        \end{aligned}
    \end{equation}
    \begin{subproof}
        \item[Action adjointe]
            L'application \( \rho_{a^{\epsilon}}=\AD(a^{\epsilon})\) est toujours un homomorphisme. Vu que \( a^{\epsilon}\) est soit \( e\) soit \( a\), nous allons nous restreindre à \( a\) et oublier l'exposant \( \epsilon\). Il faut montrer que\( \AD(a)\in\Aut(C_n)\). En utilisant le lemme \ref{LemKKXdqdL},
            \begin{equation}
                \AD(a)b^k=ab^ka^{-1}=b^{-k}=b^{n-k}.
            \end{equation}
            L'application \( \AD(a)\colon C_n\to C_n\) est donc bijective et homomorphique. Ergo isomorphisme.
        \item[Injectif]
            Si \( \psi(b^k,a^{\epsilon})=\psi(b^l,a^{\sigma})\), alors par unicité du lemme \ref{LemooNFRIooPWuikH} nous avons \( k=l\) et \( \epsilon=\sigma\).
        \item[Surjectif]
            Par la partie «existence»  du lemme \ref{LemooNFRIooPWuikH}.
        \item[Homomorphisme]
            L'homomorphisme est toujours de mise lorsque l'on prend deux sous-groupes d'un même groupe (ici le groupe des isométries de \( \eR^2\)) et que l'on tente de faire un produit semi-direct en utilisant l'action adjointe. Dans notre cas, le calcul est : 
            \begin{equation}
                \psi\big( (b^k,a^{\epsilon})(b^l,a^{\sigma}) \big)=b^k\rho_{a^{\epsilon}}(b^l)a^{\epsilon+\sigma}=b^ka^{\epsilon}b^la^{-\epsilon}a^{\epsilon+\sigma}=b^ka^{\epsilon}b^la^{\sigma}=\psi(b^k,a^{\epsilon})\psi(b^l,a^{\sigma}).
            \end{equation}
    \end{subproof}
\end{proof}

%///////////////////////////////////////////////////////////////////////////////////////////////////////////////////////////
\subsubsection{Classes de conjugaison}
%///////////////////////////////////////////////////////////////////////////////////////////////////////////////////////////
\label{subsubsecZQnBcgo}

Pour les classes de conjugaison du groupe diédral nous suivons \cite{HRIMAJJ}.

D'abord pour des raisons de déterminants\footnote{Vous notez qu'ici nous utilisons un argument qui utilise la définition de \( D_n\) comme isométries de \( \eR^2\). Si nous avions voulu à tout prix nous limiter à la définition «abstraite» en termes de générateurs, il aurait fallu trouver autre chose.}, les classes des éléments de la forme \( r^k\) et de la forme \( sr^k\) ne se mélangent pas. Nous notons \( C(x)\) la classe de conjugaison de \( x\), et \( y\cdot x=yxy^{-1}\).

Les relations que nous allons utiliser sont 
\begin{subequations}
    \begin{align}
        sr^ks=r^{-k}\\
        rs=sr^{-1}=sr^{n-1}.
    \end{align}
\end{subequations}

La classe de conjugaison qui ne rate jamais est bien entendu \( C(1)={1}\). Nous commençons les vraies festivités \( C(r^{m})\). D'abord \( r^k\cdot r^m=r^m\), ensuite
\begin{equation}
    (sr^k)\cdot r^m=sr^kr^mr^{-k}s^{-1}=sr^ms^{-1}=r^{-m}.
\end{equation}
Donc
\begin{equation}    \label{EqVFfFxgi}
    C(r^m)=\{ r^m,r^{-m} \}.
\end{equation}
À ce niveau il faut faire deux remarques. D'abord si \( m>\frac{ n }{2}\), alors \( C(r^m)\) est la classe de \( C^{n-m}\) avec \( n-m<\frac{ n }{2}\). Donc les classes que nous avons trouvées sont uniquement à lister avec \( m<\frac{ n }{2}\). Ensuite si \( m=\frac{ n }{2}\) alors \( r^m=r^{-m}\) et la classe est un singleton. Cela n'arrive que si \( n\) est pair.

Nous passons ensuite à \( C(s)\). Nous avons
\begin{equation}
    r^k\cdot s=r^ksr^{-k}=ssr^ksr^{-k}=sr^{-k}r^{-k}=sr^{n-2k},
\end{equation}
et
\begin{equation}
    (sr^k)\cdot s=\underbrace{sr^ks}_{r^{-k}}r^{-k}s^{-1}=r^{-2k}s=r^{n-2k}s=sr^{(n-1)(n-2k)}=sr^{n^2-2kn-n+2k}=sr^{2k}.
\end{equation}
donc
\begin{equation}
    C(s)=\{ sr^{n-2k},sr^{2k} \}_{k=0,\ldots, n-1}.
\end{equation}
Ici aussi l'écriture n'est pas optimale : peut-être que pour certains \( k\) il y a des doublons. Nous reportons l'écriture exacte à la discussion plus bas qui distinguera \( n\) pair de \( n\) impair. Notons juste que si \( n\) est pair, l'élément \( sr\) n'est pas dans la classe \( C(s)\).

Nous en faisons donc à présent le calcul en gardant en tête le fait qu'il n'a de sens que si \( n\) est pair. D'abord
\begin{equation}
    s\cdot (sr)=ssrs=rs=sr^{n-1}.
\end{equation}
Ensuite
\begin{equation}
    (sr^k)\cdot (sr)=sr^ksrr^{-k}s=r^{-2k+1}s=sr^{2k-1}.
\end{equation}
Avec \( k=\frac{ n }{2}\), cela rend \( s\cdot (sr)\), donc pas besoin de le recopier. Nous avons
\begin{equation}
    C(sr)=\{ sr^{2k-1} \}_{k=1,\ldots, n-1}.
\end{equation}

%///////////////////////////////////////////////////////////////////////////////////////////////////////////////////////////
\subsubsection{Le compte pour $ n$ pair}
%///////////////////////////////////////////////////////////////////////////////////////////////////////////////////////////
\label{SubsubsecROVmHuM}

Si \( n\) est pair, nous avons les classes
\begin{subequations}
    \begin{align}
        C(1)&=\{ 1 \}       &&& 1\text{ élément}\\
        C(r^m)&=\{ r^m,r^{m-1} \}&\text{ pour }&0<m<\frac{ n }{2}   & \frac{ n }{2}-1\text{ fois } 2\text{ éléments}\\
        C(r^{n/2})&=\{ r^{n/2} \}   &&&  1\text{ élément}\\ 
        C(s)&=\{ sr^{2k} \}_{k=0,\ldots, \frac{ n }{2}-1} &&&  \frac{ n }{2}\text{ éléments}\\
        C(sr)&=\{ sr^{2k+1} \}_{k=0,\ldots, \frac{ n }{2}-1} &&&  \frac{ n }{2}\text{ éléments}.
    \end{align}
\end{subequations}
Au total nous avons bien listé \( 2n\) éléments comme il se doit, dans \(  \frac{ n }{2}+3\) classes différentes.

%///////////////////////////////////////////////////////////////////////////////////////////////////////////////////////////
\subsubsection{Le compte pour $ n$ impair}
%///////////////////////////////////////////////////////////////////////////////////////////////////////////////////////////
\label{Subsubsec*GJIzDEP}

Si \( n\) est impair, nous avons les classes
\begin{subequations}
    \begin{align}
        C(1)&=\{ 1 \}       &&& 1\text{ élément}\\
        C(r^m)&=\{ r^m,r^{m-1} \}&\text{ pour }&0<m<\frac{ n-1 }{2}   & \frac{ n-1 }{2}\text{ fois } 2\text{ éléments}\\
        C(s)&=\{ sr^k \}_{k=0,\ldots, n-1} &&&  n\text{ éléments}
    \end{align}
\end{subequations}
Au total nous avons bien listé \( 2n\) éléments comme il se doit, dans \(  \frac{ n+3 }{2}\) classes différentes.

%--------------------------------------------------------------------------------------------------------------------------- 
\subsection{Applications : du dénombrement}
%---------------------------------------------------------------------------------------------------------------------------

%///////////////////////////////////////////////////////////////////////////////////////////////////////////////////////////
\subsubsection{Le jeu de la roulette}
%///////////////////////////////////////////////////////////////////////////////////////////////////////////////////////////
\label{pTqJLY}
\index{groupe!fini}
\index{groupe!de permutations}
\index{groupe!et géométrie}
\index{combinatoire}
\index{dénombrement}

Soit une roulette à \( n\) secteurs que nous voulons colorier en \( q\) couleurs\cite{HEBOFl}. Nous voulons savoir le nombre de possibilités à rotations près. Soit d'abord \( E\) l'ensemble des coloriages possibles sans contraintes; il y a naturellement \( q^n\) possibilités. Sur l'ensemble \( E\), le groupe cyclique \( G\) des rotations d'angle \( 2\pi/n\) agit. Deux coloriages étant identiques si ils sont reliés par une rotation, la réponse à notre problème est donné par le nombre d'orbites de l'action de \( G\) sur \( E\) qui sera donnée par la formule du théorème de Burnside \ref{THOooEFDMooDfosOw}. 

Nous devons calculer \( \Card\big( \Fix(g) \big)\) pour tout \( g\in G\). Soit \( g\), un élément d'ordre \( d\) dans \( G\). Si \( g\) agit sur la roulette, chaque secteur a une orbite contenant \( d\) éléments. Autrement dit, \( g\) divise la roulette en \( n/d\) secteurs. Un élément de \( E\) appartenant à \( \Fix(g)\) doit colorier ces \( n/d\) secteurs de façon uniforme; il y a \( q^{n/d}\) possibilités.

Il reste à déterminer le nombre d'éléments d'ordre \( d\) dans \( G\). Un élément de \( G\) est donné par un nombre complexe de la forme \(  e^{2ik\pi/n}\). Les éléments d'ordre \( d\) sont les racines primitives\footnote{Une racine non primitive \( 8\)ième de l'unité est par exemple \( i\). Certes \( i^8=1\), mais \( i^4=1\) aussi. Le nombre \( i\) est d'ordre \( 4\).} \( d\)ièmes de l'unité. Nous savons que --par définition-- il y a \( \varphi(d)\) telles racines primitives de l'unité. Bref il y a \( \varphi(d)\) éléments d'ordre \( d\) dans \( G\). 

La formule de Burnside nous donne maintenant le nombre d'orbites :
\begin{equation}
    \frac{1}{ n }\sum_{d|n}\varphi(d)q^{n/d}.
\end{equation}
Cela est le nombre de coloriage possibles de la roulette à \( n\) secteurs avec \( q\) couleurs.

%///////////////////////////////////////////////////////////////////////////////////////////////////////////////////////////
\subsubsection{L'affaire du collier}
%///////////////////////////////////////////////////////////////////////////////////////////////////////////////////////////
\label{siOQlG}

Nous avons maintenant des perles de \( q\) couleurs différentes et nous voulons en faire un collier à \( n\) perles. Cette fois non seulement les rotations donnent des colliers équivalents, mais en outre les symétries axiales (il est possible de retourner un collier, mais pas une roulette). Le groupe agissant sur \( E\) est maintenant le groupe diédral\footnote{Définition \ref{DEFooIWZGooAinSOh}.}\index{diédral}\index{groupe!diédral} \( D_n\) conservant un polygone a \( n\) sommets.

Nous devons séparer le cas \( n\) impair du cas \( n\) pair.

Si \( n\) est impair, alors les axes de symétries passent par un sommet par le milieu du côté opposé. Le groupe \( D_n\) contient \( n\) symétries axiales. Nous avons donc maintenant
\begin{equation}
    | G |=2n.
\end{equation}
Nous écrivons la formule de Burnside
\begin{equation}
    \Card(\Omega)=\frac{1}{ 2n }\sum_{g\in G}\Card\big( \Fix(g) \big).
\end{equation}
Si \( g\) est une rotation, le travail est déjà fait. Si \( g\) est une symétrie, nous avons le choix de la couleur du sommet par lequel passe l'axe et le choix de la couleur des \( (n-1)/2\) paires de sommets. Cela fait
\begin{equation}
    qq^{(n-1)/2}=q^{\frac{ n+1 }{2}}
\end{equation}
possibilités. Nous avons donc
\begin{equation}
    \Card(\Omega)=\frac{1}{ 2n }\left( \sum_{d|n}q^{n/d}\varphi(d)+nq^{\frac{ n+1 }{2}} \right).
\end{equation}

Si \( n\) est pair, le choses se compliquent un tout petit peu. En plus de symétries axiales passant par un sommet et le milieu du côté opposé, il y a les axes passant par deux sommets opposés. Pour colorier un collier en tenant compte d'une telle symétrie, nous pouvons choisir la couleur des deux perles par lesquelles passe l'axe ainsi que la couleur des \( (n-2)/2\) paires de perles. Cela fait en tout
\begin{equation}
    q^2q^{\frac{ n-2 }{2}}=q^{\frac{ n+2 }{2}}.
\end{equation}
Le groupe \( G\) contient \( n/2\) tels axes.

Notons que cette fois \( G\) ne contient plus que \( n/2\) symétries passant par un sommet et un côté. L'ordre de $G$ est donc encore \( 2n\). La formule de Burnside donne
\begin{equation}
    \Card(\Omega)=\frac{1}{ 2n }\left( \sum_{d\divides n}\varphi(d)q^{n/d}+\frac{ n }{2}q^{(n+2)/2}+\frac{ n }{2}q^{n/2} \right).
\end{equation}
