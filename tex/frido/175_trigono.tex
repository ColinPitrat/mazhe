% This is part of (everything) I know in mathematics
% Copyright (c) 2011-2013,2016-2017
%   Laurent Claessens
% See the file fdl-1.3.txt for copying conditions.

%+++++++++++++++++++++++++++++++++++++++++++++++++++++++++++++++++++++++++++++++++++++++++++++++++++++++++++++++++++++++++++
\section{Isométries de l'espace euclidien}
%+++++++++++++++++++++++++++++++++++++++++++++++++++++++++++++++++++++++++++++++++++++++++++++++++++++++++++++++++++++++++++

Nous considérons l'espace affine euclidien \( A=\affE_n(\eR)\) modelé sur \( \eR^n\) avec sa métrique usuelle. Un premier grand résultat sera le théorème \ref{ThoDsFErq} qui dira que les isométries de cet espace sont des applications linéaires.

%--------------------------------------------------------------------------------------------------------------------------- 
\subsection{Structure du groupe  \texorpdfstring{\( \Isom(\eR^n)\)}{Isom(Rn)} }
%---------------------------------------------------------------------------------------------------------------------------

\begin{example}
    La forme quadratique \( q(x)=x_1^2+x_2^2\) donne la norme euclidienne. La forme bilinéaire associée est \( b(x,y)=x_1y_1+x_2y_2\), qui est le produit scalaire usuel.
\end{example}

Il ne faudrait pas déduire trop vite que la formule \( \| x \|^2=q(x)\) donne une norme dès que \( q\) est non dégénérée. En effet \( q\) peut ne pas être définie positive. La forme \( q(x)=x_1^2-x_2^2\) prend des valeurs positives et négatives. A fortiori \( d(x,y)=q(x-y)\) ne donne pas toujours une distance.

\begin{definition}      \label{DEFooECTUooRxBhHf}
    Une \defe{isométrie}{isométrie!de forme quadratique} pour la forme quadratique \( q\) est une application bijective \( f\colon V\to V\) telle que \( q(x-y)=q\big( f(x)-f(y) \big)\). Dans les cas où \( q\) donne une distance, alors c'est une isométrie au sens usuel.
\end{definition}

\begin{lemma}   \label{LemewGJmM}
    Soit \( q\) une forme quadratique et \( b\) la forme bilinéaire associée par le lemme \ref{LEMooLKNTooSfLSHt}.  Pour une application bijective \( f\colon E\to E\) telle que \( f(0)=0\), les conditions suivantes sont équivalentes: 
    \begin{enumerate}
        \item
            \( b\big( f(x),f(y) \big)=b(x,y)\) pour tout \( x,y\in E\);
        \item
            \( q\big( f(x)-f(y) \big)=q(x-y)\) pour tout \( x,y\in E\).
    \end{enumerate}
\end{lemma}

\begin{proof}
    Dans le sens direct, en posant \( x=y\) nous trouvons tout de suite \( q(f(x))=q(f)\); ensuite en utilisant la distributivité de \( b\),
    \begin{subequations}
        \begin{align}
            q\big( f(x)-f(y) \big)&=b\big( f(x)-f(y),f(x)-f(y) \big)\\
            &=q\big( f(x) \big)-2b\big( f(x),f(y) \big)+q\big( f(y) \big)\\
            &=q(x)+q(y)-2b(x,y)\\
            &=q(x-y).
        \end{align}
    \end{subequations}
    
    Dans l'autre sens, nous commençons par remarquer que l'hypothèse \( f(0)=0\) donne \( q(x)=q\big( f(x) \big)\). Ensuite nous utilisons l'identité de polarisation \eqref{EqMrbsop} :
    \begin{subequations}
        \begin{align}
            b\big( f(x),f(y) \big)&=\frac{ 1 }{2}\big[ q\big( f(x) \big)+q\big( f(y) \big)-q\big( f(x-y) \big) \big]\\
            &=\frac{ 1 }{2}\big[ q(x)+q(y)-q(x-y) \big]\\
            &=b(x,y).
        \end{align}
    \end{subequations}
\end{proof}

\begin{theorem}[\cite{ooQFKAooFnllQU}]     \label{ThoDsFErq}
    Soit \( f\colon E\to E\) une bijection telle que
    \begin{equation}
        q(x-y)=q\big( f(x)-f(y) \big)
    \end{equation}
    pour tout \( x,y\in E\). Alors
    \begin{enumerate}
        \item
            si \( f(0)=0\), alors \( f\) est linéaire;
        \item
            si \( f(0)\neq 0\) alors \( f\) est affine\footnote{Lemme \ref{LEMooZZAIooOMiayy}.}
    \end{enumerate}
\end{theorem}

\begin{proof}
    Si \( f(0)=0\), nous savons par le lemme \ref{LemewGJmM} que \( b\big( f(x),f(y) \big)=b(x,y)\). Soit \( z\in E\); étant donné que \( f\) est bijective nous pouvons considérer l'élément \( f^{-1}(z)\in E\) et calculer
    \begin{subequations}
        \begin{align}
            b\big( f(x+y),z \big)&=b\big( f(x+y),f(f^{-1}(z)) \big)\\
            &=b(x+y,f^{-1}(z))\\
            &=b(x,f^{-1}(z))+b(y,f^{-1}(z))\\
            &=b(f(x),z)+b(f(y),z)\\
            &=b\big( f(x)+f(y),z \big),
        \end{align}
    \end{subequations}
    donc \( f(x+y)=f(x)+f(y)\) par le lemme \ref{LemyKJpVP}. 

    De la même façon on trouve \( b\big( f(\lambda x),z \big)=b\big( \lambda f(x),z \big)\) qui prouve que \( f(\lambda x)=\lambda f(x)\) et donc que \( f\) est linéaire.

    Si \( f(0)\neq 0\), alors nous posons \( g(x)=f(x)-f(0)\) qui vérifie \( g(0)=0\) et
    \begin{equation}
        q\big( g(x)-g(y) \big)=q\big( f(x)-f(0)-f(y)+f(0) \big)=q(x-y).
    \end{equation}
    Nous pouvons donc appliquer le premier point à \( g\), déduire que \( g\) est linéaire et donc que \( f\) est affine.
\end{proof}

\ifbool{isEverything}{
\begin{remark}
    Des preuves alternatives.
    \begin{enumerate}
        \item
            En utilisant un peut plus d'indices et un peu plus de mots comme «tenseurs», peut être trouvée  \href{http://physics.stackexchange.com/questions/12664/proving-that-interval-preserving-transformations-are-linear}{ici}. Le fait que la preuve donnée soit tensorielle me fait penser que le résultat peut encore être généralisé.
        \item
            Et encore une autre preuve, utilisant des techniques de groupes de Lie sera la proposition \ref{PROPooDVIWooAFDNPy}.
    \end{enumerate}
\end{remark}
}
{}

Nous pouvons maintenant particulariser tout cela au cas de \( \eR^n\) muni du produit scalaire usuel et de la norme associée pour voir quel résultat nous avons à peine prouvé.

\begin{lemma}[\cite{ooYPVPooYGSlNU}]        \label{LEMooJPYZooHETCqt}
    Une isométrie est bijective (nous sommes en dimension finie).
\end{lemma}

\begin{proof}
    Si \( f\colon E\to E\) est une isométrie, elle est linéaire par le théorème \ref{ThoDsFErq}. Elle vérifie également \( \| f(x) \|=\| x \|\), et donc \( f(x)=0\) si et seulement si \( x=0\), c'est à dire que \( f\) est injective. Elle est alors bijective par le corollaire \ref{CORooCCXHooALmxKk} du théorème du rang.
\end{proof}

Nous notons ici \( T(n)\) le groupe des translations sur \( \eR^n\). Un élément de \( T(n)\) est une translation \( \tau_v\) donnée par un vecteur \( v\) et agissant sur \( \eR^n\) par
\begin{equation}
    \begin{aligned}
        \tau_v\colon \eR^n&\to \eR^{n} \\
        x&\mapsto x+v. 
    \end{aligned}
\end{equation}
Ce groupe est isomorphe au groupe abélien \( (\eR^n,+)\), et nous allons souvent identifier \( \tau_v\) à \( v\).

Si vous ne voulez pas savoir ce qu'est un produit semi-direct de groupes, vous pouvez lire seulement le point \ref{ITEMooLLUIooIGsknv} du théorème suivant, et passer directement à la remarque \ref{REMooLUEZooIwvTqu}.
\begin{theorem}     \label{THOooQJSRooMrqQct}
    Un peu de structure sur \( \Isom(\eR^n)\).
    \begin{enumerate}
        \item       \label{ITEMooLLUIooIGsknv}
            L'application
            \begin{equation}
                \begin{aligned}
                    \psi\colon T(n)\times \gO(n)&\to \Isom(\eR^n) \\
                    (v,\Lambda)&\mapsto \tau_v\circ\Lambda 
                \end{aligned}
            \end{equation}
            est une bijection. Ici,  \( T(n)\) est le groupe des translations de \( \eR^n\).
        \item
            Un couple \( (v,\Lambda)\in T(n)\times\SO(n)\) agit sur \( x\in \eR^n\) par
            \begin{equation}
                (v,\Lambda)x=\Lambda x+v
            \end{equation}
            au sens où \( \psi(v,\Lambda)x=\Lambda x+v\).
        \item       \label{ITEMooEWSIooNKzRxB}
            En tant que groupes,
            \begin{equation}
                \Isom(\eR^n)\simeq T(n)\times_{\rho}\gO(n)
            \end{equation}
            où \( \rho\) représente l'action adjointe de \( \gO(n)\) sur \( T(n)\) et \( \times_{\rho}\) dénotes le produit semi-direct de la définition \ref{DEFooKWEHooISNQzi}.
    \end{enumerate}
\end{theorem}

\begin{proof}
    Point par point.
    \begin{enumerate}
        \item
            Prouvons que l'application proposée est injective et surjective. Notons aussi que ce point ne parle pas de structure de groupe, mais seulement d'une bijection en tant qu'ensembles.    
            \begin{subproof}
                \item[Injection]
                    Si \( \psi(v,\Lambda)=\psi(w,\Lambda')\) alors en appliquant sur \( x=0\) nous avons tout de suite \( v=w\). Et ensuite \( \Lambda=\Lambda'\) est immédiat.
                \item[Surjection]
                    Une isométrie \( g\in\Isom(\eR^n)\) est une application \( g\colon \eR^n\to \eR^n\) telle que \( d(x,y)=d\big( g(x),g(y) \big)\). Dans le cas de \( \eR^n\) cela se traduit par
                    \begin{equation}
                        \| x-y \|=\big\| g(x)-g(y) \big\|,
                    \end{equation}
                    Vu que \( x\mapsto\| x \|\) est une forme quadratique, elle tombe sous le coup du théorème  \ref{ThoDsFErq}, ce qui nous permet de dire que \( g\) est affine. Or par définition une application est affine lorsqu'elle est la composée d'une translation et d'une application linéaire.
            \end{subproof}
        \item
            C'est seulement le fait que \( (\tau_v\circ\Lambda)x=\tau_v\big( \Lambda x \big)=\Lambda(x)+v\).
        \item
            Nous allons étudier l'application
            \begin{equation}
                \psi\colon T(n)\times_{\rho}O(n)\to \Isom(\eR^n).
            \end{equation}
            \begin{subproof}
            \item[Le produit semi-direct est bien définit]
                Il faut montrer que
                \begin{equation}
                    \begin{aligned}
                        \rho\colon O(n)&\to \Aut\big( T(n) \big) \\
                        \Lambda&\mapsto \AD(\Lambda) 
                    \end{aligned}
                \end{equation}
                est correcte.

                D'abord pour \( \Lambda\in O(n)\), nous avons bien \( \rho_{\Lambda}(\tau_v)\in T(n)\) parce qu'en appliquant à \( x\in \eR^n\),
                    \begin{equation}
                        (\Lambda\tau_v\Lambda^{-1})(x)=\Lambda\big( \tau_v(\Lambda^{-1} x) \big)=\Lambda\big( \Lambda^{-1}x+v \big)=x+\Lambda(v)=\tau_{\Lambda(v)}(x).
                    \end{equation}
                    Donc \( \rho_{\Lambda}(\tau_v)=\tau_{\Lambda(v)}\).

                    De plus, \( \rho_{\Lambda}\in\Aut\big( T(n) \big)\) parce que 
                    \begin{equation}
                        \rho_{\Lambda}\big( \tau_v\circ \tau_w \big)=\rho_{\Lambda}(\tau_v)\circ\rho_{\Lambda}(\tau_v),
                    \end{equation}
                    comme on peut aisément vérifier que les deux membres sont égaux à \( \tau_{\Lambda(v+w)}\).
                \item[\( \psi\) est une bijection]
                    Cela est déjà vérifié.
                \item[\( \psi\) est un homomorphisme]
                    Nous avons d'une part
                    \begin{equation}
                        \psi\big( (v,g)(w,h) \big)=\psi\big( v\rho_g(w),gh \big)=\tau_v\circ g\circ\tau_w\circ g^{-1}\circ g\circ h=\tau_v\circ g\circ\tau_w\circ h.
                    \end{equation}
                    Et d'autre part,
                    \begin{equation}
                        \psi(v,g)\circ\psi(w,h)=\tau_v\circ g\circ \tau_w\circ h,
                    \end{equation}
                    ce qui est la même chose.
            \end{subproof}
    \end{enumerate}
\end{proof}

\begin{remark}      \label{REMooLUEZooIwvTqu}
    Notons au passage la loi de groupe sur les couples qui est donnée, pour tout \( v,v'\in \eR^n\), \( \Lambda,\Lambda'\in\SO(n)\), par
    \begin{equation}    \label{EqDiHcut}
            (v,\Lambda)\cdot(v',\Lambda')=(\Lambda v'+v,\Lambda\Lambda')
    \end{equation}
    comme le montre le calcul suivant :
    \begin{subequations}
        \begin{align}
            (v,\Lambda)\cdot(v',\Lambda')x&=(v,\Lambda)(\Lambda'x+v')\\
            &=\Lambda\Lambda'x+\Lambda v'+v\\
            &=(\Lambda v'+v,\Lambda\Lambda')x.
        \end{align}
    \end{subequations}
\end{remark}

\begin{proposition}[\cite{ooZYLAooXwWjLa}]      \label{PROPooDHYWooXxEXvl}
    Soit \( n\geq 1\) et un élément \( R\) de \( \gO(n)\) de déterminant \( -1\) tel que \( R^2=\id\). En posant \( C_2=\{ \id,R \}\) nous avons
    \begin{equation}
        \gO(n)=\SO(n)\times_{\rho} C_2
    \end{equation}
\end{proposition}

\begin{proof}
    Notons que pour \( R\) nous pouvons prendre par exemple \( (x_1,\ldots, x_n)\mapsto (-x_1,x_2,\ldots, x_n)\). Ce que nous allons montrer être un isomorphisme est :
    \begin{equation}
        \begin{aligned}
            \psi\colon \SO(n)\times C_2&\to \gO(n) \\
            (A,h)&\mapsto Ah. 
        \end{aligned}
    \end{equation}
    \begin{subproof}
        \item[Injectif]
            Soient \( A,B\in \SO(n)\) et \( h,k\in C_2\) tels que \( \psi(A,h)=\psi(B,k)\), c'est à dire tels que \( Ah=Bk\). Vu que \( \det(A)=\det(B)=1\) nous avons \( \det(h)=\det(k)\). Mais comme \( C_2\) contient un élément de déterminant \( 1\) et un élément de déterminant \( -1\), nous avons \( h=k\). De là \( A=B\).
        \item[Surjectif]
            Soit \( X\in\gO(n)\). Si \( \det(X)=1\) alors \( X\in \SO(n)\) et \( X=\psi(X,\mtu)\). Si par contre \( \det(X)=-1\) alors \( XR\in\SO(n)\) parce que \( \det(XR)=1\) et nous avons
            \begin{equation}
                \psi(XR,R)=XR^2=X.
            \end{equation}
        \item[Homomorphisme]
            Nous avons
            \begin{equation}
                \psi\Big( (A,h)(B,k) \Big)=\psi\big( A\rho_h(B),hk \big)=A(hBh^{-1})hk=AhBk,
            \end{equation}
            tandis que
            \begin{equation}
                \psi(A,h)\psi(B,k)=AhBk,
            \end{equation}
            qui est la même chose.
    \end{subproof}
\end{proof}

\begin{lemma}[\cite{JGAdTA}]
    Si \( n\geq 3\), alors toute droite est intersection de deux plans non isotropes.
\end{lemma}

\begin{proposition}[\cite{ooZYLAooXwWjLa}]      \label{PROPooVEEUooJQmmkN}
    Si une isométrie de \( \eR^n\) fixe un ensemble \( F\) de points, alors elle fixe l'espace affine engendrée par \( F\).
\end{proposition}

\begin{proof}
    Soit \( f\in \Isom(\eR^n)\) fixant \( F\). Par le théorème \ref{ThoDsFErq}, c'est une application affine et l'ensemble \( \Fix(f)\) des points fixés par \( f\) est un sous-espace affine de \( \eR^n\), grâce à la proposition \ref{PROPooYRCJooIcmUVI}.

    Donc \( \Fix(f)\) est un espace affine contenant \( F\). Vu que l'espace affine engendré par \( F\) est l'intersection de tous les espace affines contenant \( F\), il est en particulier contenu dans \( \Fix(f)\).
\end{proof}

\begin{corollary}       \label{CORooZHZZooDgTzsW}
    Si \( f\) et \( g\) sont des isométries de \( \eR^n\) qui coïncident sur \( F\), alors elles coïncident sur l'espace affine engendré par \( F\).
\end{corollary}

\begin{proof}
    Nous considérons \( h=g^{-1}\circ f\) qui est une isométrie de \( \eR^n\) fixant \( F\). Elle fixe donc, par la proposition \ref{PROPooVEEUooJQmmkN}, l'espace affine engendré par $F$. Or tout point fixé par \( h\) est un point sur lequel \( g\) et \( f\) coïncident.
\end{proof}

%--------------------------------------------------------------------------------------------------------------------------- 
\subsection{Classification des isométries de \( \eR\)}
%---------------------------------------------------------------------------------------------------------------------------

\begin{definition}
    Soit \( x\in \eR\); nous notons \( \sigma_x\) la \defe{réflexion}{réflexion}\nomenclature[R]{\( \sigma_x\)}{réflexion par rapport à \( x\)} par rapport à \( x\), c'est à dire
    \begin{equation}
        \sigma_x(y)=2x-y.
    \end{equation}
\end{definition}

\begin{theorem}[\cite{ooZYLAooXwWjLa}]
    Toute isométrie de \( \eR\) est composée d'au plus \( 2\) réflexions. Plus précisément toute isométrie de \( \eR\) est dans une des trois catégories suivantes :
    \begin{itemize}
        \item l'identité (\( 0\) réflexions),
        \item les réflexions,
        \item les translations (\( 2\) réflexions)
    \end{itemize}
\end{theorem}

\begin{proof}
    Nous divisions la preuve en fonction du nombre de points fixés par l'isométrie \( f\in\Isom(\eR)\).
    \begin{subproof}
        \item[\( f\) fixe deux points distincts]
            Alors elle fixe l'espace affine engendrée par ces deux points par la proposition \ref{PROPooVEEUooJQmmkN}. Donc \( f\) fixe tout \( \eR\) et est l'identité.
        \item[\( f\) fixe un unique point]
            Soit \( x\) l'unique point fixé par \( f\) et considérons \( y\neq x\). Vu que \( x=f(x)\) et que \( f\) est une isométrie,
            \begin{equation}
                d\big( x,f(y) \big)=d\big( f(x),f(y) \big)=d(x,y).
            \end{equation}
            Donc \( f(y)\) est à égale distance de \( x\) que \( y\). Autrement dit, \( f(y)\) est soit \( y\) soit \( \sigma_x(y)\). Mais comme \( x\) est unique point fixe, \( f(y)=\sigma_x(y)\). Ce raisonnement étant valable pour tout \( y\neq x  \) nous avons \( f=\sigma_x\).
        \item[\( f\) n'a pas de points fixes]
            Soit \( x\in \eR\) et \( y=\frac{ x+f(x) }{ 2 }\). Nous posons \( g=\sigma_y\circ f\). Alors \( x\) est un point fixe de \( g\) parce que
            \begin{equation}
                g(x)=\sigma_y\big( f(x) \big)=2y-f(x)=x.
            \end{equation}
            Donc soit \( g\) est l'identité soit \( g\) est une réflexion (par les points précédents). La possibilité \( g=\id\) est exclue parce que cela ferait \( f=\sigma_y\) alors que \( f\) n'a pas de points fixes. Donc \( g\) est une réflexion; et comme \( x\) est un point fixe de \( g\) nous avons \( g=\sigma_x\). Au final
            \begin{equation}
                f=\sigma_y\circ\sigma_x.
            \end{equation}
            Montrons que cela implique que \( f\) est une translation :
            \begin{equation}
                \sigma_y\sigma_x(z)=\sigma_y(2x-z)=2y-2x+z=z+2(y-x).
            \end{equation}
            Donc \( \sigma_y\circ\sigma_x\) est la translation de vecteur \( 2(y-x)\).
    \end{subproof}
\end{proof}


%--------------------------------------------------------------------------------------------------------------------------- 
\subsection{Segment, plan médiateur et équidistance}
%---------------------------------------------------------------------------------------------------------------------------

\begin{lemma}   \label{LEMooSZZWooPDHnGl}
    Un point \( M\) est sur la médiatrice du segment \( [A,B]\) si et seulement si \( \| M-A \|=\| M-B \|\).
\end{lemma}

\begin{lemma}       \label{LEMooVBVUooOTFFXT}
    Soient \( A\) et \( B\) de points de \( \eR^3\). Alors le plan médiateur du segment \( [A,B]\) est le lieu des points de \( \eR^3\) à être équidistants de \( A\) et \( B\).
\end{lemma}

\begin{proof}
    Nous nommons \( \sigma\) ce plan.

    Soit \( X\) un point équidistant de \( A\) et \( B\). Alors dans le plan \( (A,B,X)\), le triangle \( ABX\) est isocèle en \( X\), et la hauteur issue de \( X\) coupe perpendiculairement \( [A,B]\) en son milieu. Cela prouve que \( X\) est dans le plan médiateur du segment \( [A,B]\) (lemme \ref{LEMooSZZWooPDHnGl}).

    Mettons au contraire que \( X\) est dans le plan médiateur de \( [A,B]\). Nous avons \( (X,M)\perp (A,B)\). Donc le triangle \( A,B,X\) est isocèle en \( X\) et donc \( X\) est équidistant de \( A\) et \( B\).
\end{proof}

%--------------------------------------------------------------------------------------------------------------------------- 
\subsection{Isométries du tétraèdre régulier}
%---------------------------------------------------------------------------------------------------------------------------

\begin{definition}
    Un polyèdre \defe{régulier}{régulier!polyèdre} est un polyèdre dont les faces sont des polygones réguliers identiques dont tous les sommets joignent le même nombre d'arrêtes.
\end{definition}

Le \defe{tétraèdre}{tétraèdre} est une pyramide à base triangulaire dont toutes les faces sont des triangles équilatéraux.

\begin{proposition}[Isométries affines du tétraèdre régulier]       \label{PROPooVNLKooOjQzCj}
    Soit un tétraèdre régulier \( T\) et son groupe d'isométries affines \( \Iso(T)\) (définition \ref{DEFooZGKBooGgjkgs}). Alors
    \begin{equation}
        \Iso(T)\simeq S_4
    \end{equation}
    où \( S_4\) est le groupe des permutations de quatre objets.
\end{proposition}

\begin{proof}
    Commençons par prouver qu'une isométrie préserve les sommets : l'image d'un sommet est un sommet. Pour cela nous considérons \( g\in \Iso(T)\) et nous supposons que l'image d'un sommet \( x\) soit à l'intérieur d'une arrête. Soient \( g(y)\) et \( g(z)\) deux points distincts de cette arrête situés à égale distance de \( g(x)\). Cela est possible parce que \( g\) est une bijection de \( \eR^3\). Aussi : \( y\neq z\). Mais une application affine préserve l'alignement (vous ne le croyez pas  ? regardez la forme donnée par le lemme \eqref{LEMooZZAIooOMiayy}), donc \( x\), \( y\) et \( z\) foment un triangle isocèle en \( x\) de points alignés et appartenant à \( T\). Cela est impossible si \( x\) est un sommet.

    Donc l'image d'un sommet est un sommet. Si nous numérotons les sommets \( x_1\),\ldots, \( x_4\), nous obtenons un morphisme de groupe \( \varphi\colon \Isom(T) \to S_4\) qui envoie \( g\) sur la permutation qui envoie \( 1\) sur le numéro du sommet \( g(x_1)\), \( 2\) sur le numéro du sommet \( g(x_2)\), etc.

    \begin{subproof}
    \item[Le morphisme \( \varphi\) est injectif]
        Supposons \( \varphi(g_1)=\varphi(g_2)\). Alors \( g_1^{-1}\circ g_2\) est une isométrie de \( (\eR^3,d)\) qui fixe les quatre sommets. Une application affine \( \eR^4\to\eR^3\) fixant \( 4\) point est l'identité par le lemme \ref{LEMooDUMVooFtfFOe}. Donc \( g_1^{-1}g_2=\id\), ce qui prouve que \( g_1=g_2\). Vous noterez que nous utilisons l'unicité de l'inverse dans un groupe.

    \item[\( \varphi\) est surjectf]

        Nous savons que \( S_4\) est engendré par les transpositions (proposition \ref{PropPWIJbu}). Or les transpositions sont dans l'image de \( \varphi\). En effet, notons les sommets de notre tétraèdre par \( A\), \( B\), \( D\) et \( D\) et considérons la transposition \( A\leftrightarrow B\). Elle est l'image par \( \varphi\) de la réflexion selon le plan \( \sigma\), médiateur du segment \( [A,B]\). Pour nous assurer de cela, nous devons nous assurer que \( C\) et \( D\) appartiennent à \( \sigma\). Cela est le contenu du lemme \ref{LEMooVBVUooOTFFXT}.

    \item[Conclusion]

        L'application \( \varphi\) est un morphisme bijectif, c'est à dire un isomorphisme.

    \end{subproof}
\end{proof}

\begin{normaltext}
    Lorsque le tétraèdre a son barycentre en l'origine de \( \eR^3\), l'isomorphisme \( \varphi\colon \Iso(T)\to S_4\) donne une représentation de dimension \( 3\) de \( S_4\). Nous allons calculer les caractères de \( S_4\) en la section \ref{SecUMIgTmO} sans avoir besoin de savoir que l'une des représentations de dimension \( 3\) est cella que nous venons de trouver via le groupe des isométries du tétraèdre. Nous allons cependant également y calculer les caractères de la représentations \( \varphi\), pour le sport.

    Plus de détails en \ref{SUBSECooLEUAooGGjGIZ}.
\end{normaltext}


