% This is part of (everything) I know in mathematics
% Copyright (c) 2011-2013,2016-2019
%   Laurent Claessens
% See the file fdl-1.3.txt for copying conditions.

%+++++++++++++++++++++++++++++++++++++++++++++++++++++++++++++++++++++++++++++++++++++++++++++++++++++++++++++++++++++++++++
\section{Isométries de l'espace euclidien}
%+++++++++++++++++++++++++++++++++++++++++++++++++++++++++++++++++++++++++++++++++++++++++++++++++++++++++++++++++++++++++++

Nous considérons l'espace affine euclidien \( A=\affE_n(\eR)\) modelé sur \( \eR^n\) avec sa métrique usuelle. Un premier grand résultat sera le théorème~\ref{ThoDsFErq} qui dira que les isométries de cet espace sont des applications linéaires.

%---------------------------------------------------------------------------------------------------------------------------
\subsection{Structure du groupe  \texorpdfstring{\( \Isom(\eR^n)\)}{Isom(Rn)} }
%---------------------------------------------------------------------------------------------------------------------------

Si vous ne voulez pas savoir ce qu'est un produit semi-direct de groupes, vous pouvez lire seulement le point~\ref{ITEMooLLUIooIGsknv} du théorème suivant, et passer directement à la remarque~\ref{REMooLUEZooIwvTqu}.
\begin{theorem}     \label{THOooQJSRooMrqQct}
    Un peu de structure sur \( \Isom(\eR^n)\).
    \begin{enumerate}
        \item       \label{ITEMooLLUIooIGsknv}
            L'application
            \begin{equation}
                \begin{aligned}
                    \psi\colon T(n)\times \gO(n)&\to \Isom(\eR^n) \\
                    (v,\Lambda)&\mapsto \tau_v\circ\Lambda
                \end{aligned}
            \end{equation}
            est une bijection. Ici,  \( T(n)\) est le groupe des translations de \( \eR^n\).
        \item
            Un couple \( (v,\Lambda)\in T(n)\times\SO(n)\) agit sur \( x\in \eR^n\) par
            \begin{equation}
                (v,\Lambda)x=\Lambda x+v
            \end{equation}
            au sens où \( \psi(v,\Lambda)x=\Lambda x+v\).
        \item       \label{ITEMooEWSIooNKzRxB}
            En tant que groupes,
            \begin{equation}
                \Isom(\eR^n)\simeq T(n)\times_{\rho}\gO(n)
            \end{equation}
            où \( \rho\) représente l'action adjointe de \( \gO(n)\) sur \( T(n)\) et \( \times_{\rho}\) dénotes le produit semi-direct de la définition~\ref{DEFooKWEHooISNQzi}.
    \end{enumerate}
\end{theorem}

\begin{proof}
    Point par point.
    \begin{enumerate}
        \item
            Prouvons que l'application proposée est injective et surjective. Notons aussi que ce point ne parle pas de structure de groupe, mais seulement d'une bijection en tant qu'ensembles.
            \begin{subproof}
                \item[Injection]
                    Si \( \psi(v,\Lambda)=\psi(w,\Lambda')\) alors en appliquant sur \( x=0\) nous avons tout de suite \( v=w\). Et ensuite \( \Lambda=\Lambda'\) est immédiat.
                \item[Surjection]
                    Une isométrie \( g\in\Isom(\eR^n)\) est une application \( g\colon \eR^n\to \eR^n\) telle que \( d(x,y)=d\big( g(x),g(y) \big)\). Dans le cas de \( \eR^n\) cela se traduit par
                    \begin{equation}
                        \| x-y \|=\big\| g(x)-g(y) \big\|,
                    \end{equation}
                    Vu que \( x\mapsto\| x \|\) est une forme quadratique, elle tombe sous le coup du théorème~\ref{ThoDsFErq}, ce qui nous permet de dire que \( g\) est affine. Or par définition une application est affine lorsqu'elle est la composée d'une translation et d'une application linéaire.
            \end{subproof}
        \item
            C'est seulement le fait que \( (\tau_v\circ\Lambda)x=\tau_v\big( \Lambda x \big)=\Lambda(x)+v\).
        \item
            Nous allons étudier l'application
            \begin{equation}
                \psi\colon T(n)\times_{\rho}O(n)\to \Isom(\eR^n).
            \end{equation}
            \begin{subproof}
            \item[Le produit semi-direct est bien définit]
                Il faut montrer que
                \begin{equation}
                    \begin{aligned}
                        \rho\colon O(n)&\to \Aut\big( T(n) \big) \\
                        \Lambda&\mapsto \AD(\Lambda)
                    \end{aligned}
                \end{equation}
                est correcte.

                D'abord pour \( \Lambda\in O(n)\), nous avons bien \( \rho_{\Lambda}(\tau_v)\in T(n)\) parce qu'en appliquant à \( x\in \eR^n\),
                    \begin{equation}
                        (\Lambda\tau_v\Lambda^{-1})(x)=\Lambda\big( \tau_v(\Lambda^{-1} x) \big)=\Lambda\big( \Lambda^{-1}x+v \big)=x+\Lambda(v)=\tau_{\Lambda(v)}(x).
                    \end{equation}
                    Donc \( \rho_{\Lambda}(\tau_v)=\tau_{\Lambda(v)}\).

                    De plus, \( \rho_{\Lambda}\in\Aut\big( T(n) \big)\) parce que
                    \begin{equation}
                        \rho_{\Lambda}\big( \tau_v\circ \tau_w \big)=\rho_{\Lambda}(\tau_v)\circ\rho_{\Lambda}(\tau_v),
                    \end{equation}
                    comme on peut aisément vérifier que les deux membres sont égaux à \( \tau_{\Lambda(v+w)}\).
                \item[\( \psi\) est une bijection]
                    Cela est déjà vérifié.
                \item[\( \psi\) est un homomorphisme]
                    Nous avons d'une part
                    \begin{equation}
                        \psi\big( (v,g)(w,h) \big)=\psi\big( v\rho_g(w),gh \big)=\tau_v\circ g\circ\tau_w\circ g^{-1}\circ g\circ h=\tau_v\circ g\circ\tau_w\circ h.
                    \end{equation}
                    Et d'autre part,
                    \begin{equation}
                        \psi(v,g)\circ\psi(w,h)=\tau_v\circ g\circ \tau_w\circ h,
                    \end{equation}
                    ce qui est la même chose.
            \end{subproof}
    \end{enumerate}
\end{proof}

\begin{remark}      \label{REMooLUEZooIwvTqu}
    Notons au passage la loi de groupe sur les couples qui est donnée, pour tout \( v,v'\in \eR^n\), \( \Lambda,\Lambda'\in\SO(n)\), par
    \begin{equation}    \label{EqDiHcut}
            (v,\Lambda)\cdot(v',\Lambda')=(\Lambda v'+v,\Lambda\Lambda')
    \end{equation}
    comme le montre le calcul suivant :
    \begin{subequations}
        \begin{align}
            (v,\Lambda)\cdot(v',\Lambda')x&=(v,\Lambda)(\Lambda'x+v')\\
            &=\Lambda\Lambda'x+\Lambda v'+v\\
            &=(\Lambda v'+v,\Lambda\Lambda')x.
        \end{align}
    \end{subequations}
\end{remark}

\begin{proposition}[\cite{ooZYLAooXwWjLa}]      \label{PROPooDHYWooXxEXvl}
    Soient \( n\geq 1\) et \( R\) un élément de \( \gO(n)\) de déterminant \( -1\) tels que \( R^2=\id\). En posant \( C_2=\{ \id,R \}\) nous avons
    \begin{equation}
        \gO(n)=\SO(n)\times_{\rho} C_2
    \end{equation}
\end{proposition}

\begin{proof}
    Notons qu'un élément \( R\) comme décrit dans l'énoncé existe. Par exemple il y a l'application  \( (x_1,\ldots, x_n)\mapsto (-x_1,x_2,\ldots, x_n)\). 

    Cela étant dit, nous allons montrer que
    \begin{equation}
        \begin{aligned}
            \psi\colon \SO(n)\times C_2&\to \gO(n) \\
            (A,h)&\mapsto Ah.
        \end{aligned}
    \end{equation}
    est un isomorphisme.
    \begin{subproof}
        \item[Injectif]
            Soient \( A,B\in \SO(n)\) et \( h,k\in C_2\) tels que \( \psi(A,h)=\psi(B,k)\), c'est-à-dire tels que \( Ah=Bk\). Vu que \( \det(A)=\det(B)=1\) nous avons \( \det(h)=\det(k)\). Mais comme \( C_2\) contient un élément de déterminant \( 1\) et un élément de déterminant \( -1\), nous avons \( h=k\). De là \( A=B\).
        \item[Surjectif]
            Soit \( X\in\gO(n)\). Si \( \det(X)=1\) alors \( X\in \SO(n)\) et \( X=\psi(X,\mtu)\). Si par contre \( \det(X)=-1\) alors \( XR\in\SO(n)\) parce que \( \det(XR)=1\) et nous avons
            \begin{equation}
                \psi(XR,R)=XR^2=X.
            \end{equation}
        \item[Homomorphisme]
            Nous avons
            \begin{equation}
                \psi\Big( (A,h)(B,k) \Big)=\psi\big( A\rho_h(B),hk \big)=A(hBh^{-1})hk=AhBk,
            \end{equation}
            tandis que
            \begin{equation}
                \psi(A,h)\psi(B,k)=AhBk,
            \end{equation}
            qui est la même chose.
    \end{subproof}
\end{proof}

\begin{lemma}[\cite{JGAdTA}]
    Si \( n\geq 3\), alors toute droite est intersection de deux plans non isotropes.
\end{lemma}

\begin{proposition}[\cite{ooZYLAooXwWjLa}]      \label{PROPooVEEUooJQmmkN}
    Si une isométrie de \( \eR^n\) fixe un ensemble \( F\) de points, alors elle fixe l'espace affine engendrée par \( F\).
\end{proposition}

\begin{proof}
    Soit \( f\in \Isom(\eR^n)\) fixant \( F\). Par le théorème~\ref{ThoDsFErq}, c'est une application affine et l'ensemble \( \Fix(f)\) des points fixés par \( f\) est un sous-espace affine de \( \eR^n\), grâce à la proposition~\ref{PROPooYRCJooIcmUVI}.

    Donc \( \Fix(f)\) est un espace affine contenant \( F\). Vu que l'espace affine engendré par \( F\) est l'intersection de tous les espaces affines contenant \( F\), il est en particulier contenu dans \( \Fix(f)\).
\end{proof}

\begin{corollary}       \label{CORooZHZZooDgTzsW}
    Si \( f\) et \( g\) sont des isométries de \( \eR^n\) qui coïncident sur \( F\), alors elles coïncident sur l'espace affine engendré par \( F\).
\end{corollary}

\begin{proof}
    Nous considérons \( h=g^{-1}\circ f\) qui est une isométrie de \( \eR^n\) fixant \( F\). Elle fixe donc, par la proposition~\ref{PROPooVEEUooJQmmkN}, l'espace affine engendré par $F$. Or tout point fixé par \( h\) est un point sur lequel \( g\) et \( f\) coïncident.
\end{proof}

%---------------------------------------------------------------------------------------------------------------------------
\subsection{Classification des isométries de \( \eR\)}
%---------------------------------------------------------------------------------------------------------------------------

\begin{definition}
    Soit \( x\in \eR\); nous notons \( \sigma_x\) la \defe{réflexion}{réflexion}\nomenclature[R]{\( \sigma_x\)}{réflexion par rapport à \( x\)} par rapport à \( x\), c'est-à-dire
    \begin{equation}
        \sigma_x(y)=2x-y.
    \end{equation}
\end{definition}

\begin{theorem}[\cite{ooZYLAooXwWjLa}]
    Toute isométrie de \( \eR\) est composée d'au plus \( 2\) réflexions. Plus précisément toute isométrie de \( \eR\) est dans une des trois catégories suivantes :
    \begin{itemize}
        \item l'identité (\( 0\) réflexions),
        \item les réflexions,
        \item les translations (\( 2\) réflexions)
    \end{itemize}
\end{theorem}

\begin{proof}
    Nous divisions la preuve en fonction du nombre de points fixés par l'isométrie \( f\in\Isom(\eR)\).
    \begin{subproof}
        \item[\( f\) fixe deux points distincts]
            Alors elle fixe l'espace affine engendrée par ces deux points par la proposition~\ref{PROPooVEEUooJQmmkN}. Donc \( f\) fixe tout \( \eR\) et est l'identité.
        \item[\( f\) fixe un unique point]
            Soit \( x\) l'unique point fixé par \( f\) et considérons \( y\neq x\). Vu que \( x=f(x)\) et que \( f\) est une isométrie,
            \begin{equation}
                d\big( x,f(y) \big)=d\big( f(x),f(y) \big)=d(x,y).
            \end{equation}
            Donc \( f(y)\) est à égale distance de \( x\) que \( y\). Autrement dit, \( f(y)\) est soit \( y\) soit \( \sigma_x(y)\). Mais comme \( x\) est unique point fixe, \( f(y)=\sigma_x(y)\). Ce raisonnement étant valable pour tout \( y\neq x  \) nous avons \( f=\sigma_x\).
        \item[\( f\) n'a pas de points fixes]
            Soient \( x\in \eR\) et \( y=\frac{ x+f(x) }{ 2 }\). Nous posons \( g=\sigma_y\circ f\). Alors \( x\) est un point fixe de \( g\) parce que
            \begin{equation}
                g(x)=\sigma_y\big( f(x) \big)=2y-f(x)=x.
            \end{equation}
            Donc soit \( g\) est l'identité soit \( g\) est une réflexion (par les points précédents). La possibilité \( g=\id\) est exclue parce que cela ferait \( f=\sigma_y\) alors que \( f\) n'a pas de points fixes. Donc \( g\) est une réflexion; et comme \( x\) est un point fixe de \( g\) nous avons \( g=\sigma_x\). Au final
            \begin{equation}
                f=\sigma_y\circ\sigma_x.
            \end{equation}
            Montrons que cela implique que \( f\) est une translation :
            \begin{equation}
                \sigma_y\sigma_x(z)=\sigma_y(2x-z)=2y-2x+z=z+2(y-x).
            \end{equation}
            Donc \( \sigma_y\circ\sigma_x\) est la translation de vecteur \( 2(y-x)\).
    \end{subproof}
\end{proof}


%---------------------------------------------------------------------------------------------------------------------------
\subsection{Segment, plan médiateur et équidistance}
%---------------------------------------------------------------------------------------------------------------------------

\begin{lemma}   \label{LEMooSZZWooPDHnGl}
    Un point \( M\) est sur la médiatrice du segment \( [A,B]\) si et seulement si \( \| M-A \|=\| M-B \|\).
\end{lemma}

\begin{lemma}       \label{LEMooVBVUooOTFFXT}
    Soient \( A\) et \( B\) de points de \( \eR^3\). Alors le plan médiateur du segment \( [A,B]\) est le lieu des points de \( \eR^3\) à être équidistants de \( A\) et \( B\).
\end{lemma}

\begin{proof}
    Nous nommons \( \sigma\) ce plan.

    Soit \( X\) un point équidistant de \( A\) et \( B\). Alors dans le plan \( (A,B,X)\), le triangle \( ABX\) est isocèle en \( X\), et la hauteur issue de \( X\) coupe perpendiculairement \( [A,B]\) en son milieu. Cela prouve que \( X\) est dans le plan médiateur du segment \( [A,B]\) (lemme~\ref{LEMooSZZWooPDHnGl}).

    Mettons au contraire que \( X\) est dans le plan médiateur de \( [A,B]\). Nous avons \( (X,M)\perp (A,B)\). Donc le triangle \( A,B,X\) est isocèle en \( X\) et donc \( X\) est équidistant de \( A\) et \( B\).
\end{proof}

%---------------------------------------------------------------------------------------------------------------------------
\subsection{Isométries du tétraèdre régulier}
%---------------------------------------------------------------------------------------------------------------------------

\begin{definition}
    Un polyèdre \defe{régulier}{régulier!polyèdre} est un polyèdre dont les faces sont des polygones réguliers identiques dont tous les sommets joignent le même nombre d'arrêtes.
\end{definition}

Le \defe{tétraèdre}{tétraèdre} est une pyramide à base triangulaire dont toutes les faces sont des triangles équilatéraux.

\begin{proposition}[Isométries affines du tétraèdre régulier]       \label{PROPooVNLKooOjQzCj}
    Soient \( T\) un tétraèdre régulier et \( \Iso(T)\) son groupe d'isométries affines (définition~\ref{DEFooZGKBooGgjkgs}). Alors
    \begin{equation}
        \Iso(T)\simeq S_4
    \end{equation}
    où \( S_4\) est le groupe des permutations de quatre objets.
\end{proposition}

\begin{proof}
    Commençons par prouver qu'une isométrie préserve les sommets : l'image d'un sommet est un sommet. Pour cela nous considérons \( g\in \Iso(T)\) et nous supposons que l'image d'un sommet \( x\) soit à l'intérieur d'une arrête. Soient \( g(y)\) et \( g(z)\) deux points distincts de cette arrête situés à égale distance de \( g(x)\). Cela est possible parce que \( g\) est une bijection de \( \eR^3\). Aussi : \( y\neq z\). Mais une application affine préserve l'alignement (vous ne le croyez pas  ? regardez la forme donnée par le lemme \eqref{LEMooZZAIooOMiayy}), donc \( x\), \( y\) et \( z\) foment un triangle isocèle en \( x\) de points alignés et appartenant à \( T\). Cela est impossible si \( x\) est un sommet.

    Donc l'image d'un sommet est un sommet. Si nous numérotons les sommets \( x_1\),\ldots, \( x_4\), nous obtenons un morphisme de groupe \( \varphi\colon \Isom(T) \to S_4\) qui envoie \( g\) sur la permutation qui envoie \( 1\) sur le numéro du sommet \( g(x_1)\), \( 2\) sur le numéro du sommet \( g(x_2)\), etc.

    \begin{subproof}
    \item[Le morphisme \( \varphi\) est injectif]
        Supposons \( \varphi(g_1)=\varphi(g_2)\). Alors \( g_1^{-1}\circ g_2\) est une isométrie de \( (\eR^3,d)\) qui fixe les quatre sommets. Une application affine \( \eR^4\to\eR^3\) fixant \( 4\) point est l'identité par le lemme~\ref{LEMooDUMVooFtfFOe}. Donc \( g_1^{-1}g_2=\id\), ce qui prouve que \( g_1=g_2\). Vous noterez que nous utilisons l'unicité de l'inverse dans un groupe.

    \item[\( \varphi\) est surjectif]

        Nous savons que \( S_4\) est engendré par les transpositions (proposition~\ref{PropPWIJbu}). Or les transpositions sont dans l'image de \( \varphi\). En effet, notons les sommets de notre tétraèdre par \( A\), \( B\), \( D\) et \( D\) et considérons la transposition \( A\leftrightarrow B\). Elle est l'image par \( \varphi\) de la réflexion selon le plan \( \sigma\), médiateur du segment \( [A,B]\). Pour nous assurer de cela, nous devons nous assurer que \( C\) et \( D\) appartiennent à \( \sigma\). Cela est le contenu du lemme~\ref{LEMooVBVUooOTFFXT}.

    \item[Conclusion]

        L'application \( \varphi\) est un morphisme bijectif, c'est-à-dire un isomorphisme.

    \end{subproof}
\end{proof}

%---------------------------------------------------------------------------------------------------------------------------
\subsection{Représentation de \( S_4\) via les isométries du tétraèdre}
%---------------------------------------------------------------------------------------------------------------------------
\label{SUBSECooVEASooDUbsBh}


\begin{normaltext}
    Lorsque le tétraèdre a son barycentre en l'origine de \( \eR^3\), l'isomorphisme \( \varphi\colon \Iso(T)\to S_4\) donne une représentation de dimension \( 3\) de \( S_4\). Nous avons calculé les caractères de \( S_4\) en la section \ref{SecUMIgTmO} sans avoir besoin de savoir que l'une des représentations de dimension \( 3\) est cella que nous venons de trouver via le groupe des isométries du tétraèdre. Nous allons cependant également y calculer les caractères de la représentation \( \varphi\), pour le sport.
\end{normaltext}

Une des représentations trouvées (la représentation \( \rho_s\)) peut être vue comme le groupe \( \Iso(T)\) des isométries affine du tétraèdre grâce à la proposition \ref{PROPooVNLKooOjQzCj} qui donne un isomorphisme de groupe \( S_4\simeq \Iso(T)\) lorsque \( T\) est un tétraèdre régulier de \( \eR^3\).

Si le barycentre de \( T\) est situé à l'origine de \( \eR^3\), alors les éléments de \( \Iso(T)\) sont des applications linéaires parce que
\begin{itemize}
    \item les affinités laissent invariantes les barycentres (proposition~\ref{PROPooGSPZooRnVgiU}),
    \item les affinités qui laissent l'origine invariante sont linéaires (corolaire~\ref{CORooATCNooUwEPNI}).
\end{itemize}
Nous allons à présent calculer la trace de cette représentation, en utilisant le fait que nous la connaissions explicitement. Nous savons que les caractères sont constants sur les classes de conjugaison; nous allons donc écrire une matrice par classe de conjugaison (qui sont données dans l'exemple~\ref{EXooQAXRooBsPURs}).

Pour tout cela nous allons considérer un tétraèdre dont le centre (isobary) est en \( (0,0,0)\) et une base de \( \eR^3\) formée de trois sommets \( e_1\), \( e_2\) et \( e_3\). Vu que l'isobarycentre des quatre sommes est en \( (0,0,0)\), le quatrième somme est forcément le point de coordonnées \( e_4(-1,-1,-1)\), de telle sorte que \( e_1+e_2+e_3+e_4=0\).

\begin{description}
    \item[Les transpositions]

        Quelle isométrie de $\eR^3$ permute deux sommets du tétraèdre sans bouger les autres ? Pour permuter les sommets \( e_1\) et \( e_2\) en laissant \( e_3\) et \( e_4\), c'est le symétrie par rapport au plan médiateur de \( [e_1,e_2]\). Ce plan passe par les sommets \( e_3\) et \( e_4\), parce que le tétraèdre étant régulier, les points \( e_3\) \( e_4\) sont équidistants de \( e_1\) et \( e_2\). Le lemme~\ref{LEMooVBVUooOTFFXT} dit qu'alors ces points dont partie du plan médiateur.

        Dans notre base, la matrice de la transposition précédemment nommée \( (12)\) est
        \begin{equation}
            \begin{pmatrix}
                0    &   1    &   0    \\
                1    &   0    &   0    \\
                0    &   0    &   1
            \end{pmatrix},
        \end{equation}
        dont la trace est \( 1\). Donc \( \chi_s(12)=1\).

    \item[Les bitranspositions]

        La bitransposition \( (12)(34)\) est le produit des transpositions selon les plans médiateur de \( [e_1,e_2]\) et \( [e_3,e_4]\). Ces deux plans sont perpendiculaires, et l'intersection est la droite qui passe par les milieux. Cette droite est perpendiculaire aux deux segments en même temps. La matrice est :
        \begin{equation}
            \begin{pmatrix}
                0    &    1   &   -1    \\
                1    &   0    &   -1    \\
                0    &   0    &   -1
            \end{pmatrix}
        \end{equation}
        parce que \( e_1\mapsto e_2\), \( e_2\mapsto e_1\) et \( e_3\mapsto e_4\). Pour rappel, la matrice est formée des images des vecteurs de base. Cela donne
        \begin{equation}
            \chi_s\big( (12)(34) \big)=-1.
        \end{equation}

    \item[Les \( 3\)-cycles]

        La symétrie qui permute cycliquement les points \( e_1\), \( e_2\) et \( e_3\) est la rotation d'angle \( 2\pi/3\) dans le plan formé par les extrémités de ces trois vecteurs. Heureusement, la trace est invariante par changement de base; donc nous pouvons calculer la trace d'une rotation d'angle \( 2\pi/3\) dans n'importe quelle base. Par exemple :
        \begin{equation}
            \chi_s\big( (12)(34) \big)=\tr\begin{pmatrix}
                1    &   0    &   0    \\
                0    &   \cos(2\pi/3)    &   \sin(2\pi/3)    \\
                0    &   -\sin(2\pi/3)    &   \cos(2\pi/3)
            \end{pmatrix}=1+2\cos(2\pi/3)=0.
        \end{equation}

        Notons que, sans cette interprétation géométrique, nous y arrivons aussi facilement : dans notre base le \( 3\)-cycle est \( e_1\mapsto e_2\mapsto e_3\mapsto e_1\), donc la matrice est :
        \begin{equation}
            \begin{pmatrix}
                0    &   0    &   1    \\
                1    &   0    &   0    \\
                0    &   1    &   0
            \end{pmatrix},
        \end{equation}
        dont la trace est manifestement nulle : \( \chi_s\big( (123) \big)=0\).

    \item[Le \( 4\)-cycle]

        Il fait \( e_1\mapsto e_2\mapsto e_3\mapsto e_4\mapsto e_1\), dont la matrice est
        \begin{equation}        \label{EQooONDUooYlduup}
            \begin{pmatrix}
                0    &   0    &   -1    \\
                1    &   0    &   -1    \\
                0    &   1    &   -1
            \end{pmatrix},
        \end{equation}
        et la trace est \( \chi_s\big( (1,2,3,4) \big)=-1\).
\end{description}
Nous avons retrouvé les caractères de la représentation \( \rho_s\), et nous pouvons vérifier qu'elle est irréductible.

%+++++++++++++++++++++++++++++++++++++++++++++++++++++++++++++++++++++++++++++++++++++++++++++++++++++++++++++++++++++++++++ 
\section{Transformations de Lorentz}
%+++++++++++++++++++++++++++++++++++++++++++++++++++++++++++++++++++++++++++++++++++++++++++++++++++++++++++++++++++++++++++

Nous considérons dans cette section un nombre réel \( c>0\) ainsi l'espace \( \eR^2\) muni du produit pseudo-scalaire\footnote{Définition \ref{DEFooLPBGooXLxubc}.} donné par la matrice
\begin{equation}
    \eta=\begin{pmatrix}
        c^2    &   0    \\ 
        0    &   -1    
    \end{pmatrix}.
\end{equation}
Et pour faire plus vrai, nous notons \( (x_0,x_1)\) les coordonnées sur \( \eR^2\). Ainsi
\begin{equation}
    x\cdot y=c^2x_0y_0-x_1y_1.
\end{equation}
Nous insistons sur le fait que cela n'est pas un produit scalaire.

\begin{lemma}[\cite{MonCerveau}]        \label{LEMooPZPZooVAdPVj}
    Soit \( c>0\). L'application
    \begin{equation}
        \begin{aligned}
        \varphi\colon \mathopen] -c , c \mathclose[&\to \eR \\
            v&\mapsto \frac{ -v/c }{ \sqrt{ 1-\frac{ v^2 }{ c^2 } } } 
        \end{aligned}
    \end{equation}
    est une bijection.
\end{lemma}

\begin{proof}
    Nous commençons par mentionner le fait que \( \varphi\) est continue du fait que le dénominateur ne s'annule pas. Une petite étude fonction montre que
    \begin{equation}
        \lim_{v\to -c} \varphi(v)=\infty,
    \end{equation}
    et
    \begin{equation}
        \lim_{v\to c} \varphi(v)=-\infty,
    \end{equation}
    et
    \begin{equation}
        \varphi'(v)=-\frac{1}{ c\sqrt{ 1-\frac{ v^2 }{ c^2 } } }-\frac{ v^2/c^3 }{ \left( 1-\frac{ v^2 }{ c^2 } \right)^{3/2} }<0.
    \end{equation}
    Tout cela fait que \( \varphi\) est bijective (entre autres par le théorème des valeurs intermédiaires \ref{ThoValInter} et la théorème dérivée et croissance \ref{PropGFkZMwD}).
\end{proof}

\begin{theorem}     \label{THOooYHDWooWxVovH}
    Soit une bijection\quext{À mon avis, il y a moyen d'affaiblir cette hypothèse. Écrivez-moi si vous avez une idée.} \( f\colon \eR^2\to \eR^2\) telle que 
    \begin{equation}
        f(x)\cdot f(y)=x\cdot y
    \end{equation}
    pour tout \( x,y\in \eR^2\). Alors :
    \begin{enumerate}
        \item
            \( f\) est linéaire.
        \item
            Il existe un unique choix de \( (x,\sigma_1,\sigma_2)\in \eR\times \{ \pm1 \}\times \{ \pm1 \}\) tel que la matrice de \( f\) ait la forme
            \begin{equation}
                f=\begin{pmatrix}
                    \sigma_1\cosh(\xi)    &   \frac{ \sigma_1\sigma_2 }{ c }\sinh(\xi)    \\ 
                    c\sinh(\xi)    &   \sigma_2\cosh(\xi)    
                \end{pmatrix}.
            \end{equation}
        \item
        Il existe un unique \( v\in\mathopen] -c , c \mathclose[\) tel que la matrice de \( f\) ait la forme
            \begin{equation}
                f=\begin{pmatrix}
                    \frac{ \sigma_1 }{ \sqrt{ 1-\frac{ v^2 }{ c^2 } } }    &   -\frac{ \sigma_1\sigma_2 }{ c^2 }\frac{ v }{ \sqrt{ 1-\frac{ v^2 }{ c^2 } } }    \\ 
                    \frac{ -v }{ \sqrt{ 1-\frac{ v^2 }{ c^2 } } }    &   \frac{ \sigma_2 }{ \sqrt{ 1-\frac{ v^2 }{ c^2 } } }    
                \end{pmatrix}.
            \end{equation}
    \end{enumerate}
\end{theorem}

\begin{proof}
    Le fait que \( f\) doive être linéaire est la proposition \ref{ThoDsFErq}. Nous posons 
    \begin{equation}
         A=\begin{pmatrix}
            \alpha    &   \beta    \\ 
            \gamma    &   \delta    
        \end{pmatrix}
    \end{equation}
    et, conformément à la proposition \ref{PROPooSYQMooEnZFdp} nous imposons \( A^t\eta A=\eta\). Après un petit produit matriciel nous obtenons :
    \begin{equation}
        \begin{pmatrix}
            c^2\alpha^2-\gamma^2    &   c^2\alpha\beta-\gamma\delta    \\ 
            c^2\alpha\beta-\gamma\delta    &   c^2\beta^2-\delta^2    
        \end{pmatrix}=\begin{pmatrix}
            c^2    &   0    \\ 
            0    &   -1    
        \end{pmatrix}.
    \end{equation}
    Voila quatre équations à résoudre pour les quatre inconnues \( \alpha, \beta,\gamma, \delta\). Déjà les équations des termes anti-diagonaux sont les mêmes. Nous recopions le reste :
    \begin{subequations}
        \begin{numcases}{}
            c^2\alpha^2-\gamma^2=c^2            \label{SUBEQooXZUGooITKZnH}\\
            c^2\alpha\beta-\gamma\delta=0       \label{SUBEQooDWQRooBeDaPw}\\
            c^2\beta^2-\delta^2=1.              \label{SUBEQooJAFLooGxmbaO}
        \end{numcases}
    \end{subequations}
    C'est le moment d'utiliser la proposition \ref{PROPooWEHGooOBqSHY}. La relation \eqref{SUBEQooXZUGooITKZnH} donne
    \begin{equation}
        \alpha^2-\left( \frac{ \gamma }{ c } \right)^2=1,
    \end{equation}
    ce qui implique l'existence (unique) de \( \xi_1\in \eR\) et \( \sigma_1\in \{ \pm 1 \}\) tels que
    \begin{subequations}        \label{SUBEQSooQUSIooRZRYSW}
        \begin{align}
            \gamma&=c\sinh(\xi_1)\\
            \alpha&=\sigma_1\cosh(\xi_1).
        \end{align}
    \end{subequations}
    La relation \eqref{SUBEQooJAFLooGxmbaO} implique quant à elle l'existence de \( \xi_2\in \eR\) et \( \sigma_2\in\{ \pm 1 \}\) tels que
    \begin{subequations}        \label{SUBEQSooLFHCooXVetmK}
        \begin{align}
            \delta&=\sigma_2\cosh(\xi_2)\\
            \beta&=\frac{1}{ c }\sinh(\xi_2).
        \end{align}
    \end{subequations}
    
    Nous substituons maintenant toutes les valeurs \eqref{SUBEQSooQUSIooRZRYSW} et \eqref{SUBEQSooLFHCooXVetmK} dans \eqref{SUBEQooDWQRooBeDaPw}. Cela donne
    \begin{equation}        \label{EQooHTMSooVYzJUS}
        \sigma_1\cosh(\xi_1)\sinh(\xi_2)=\sinh(\xi_1)\cosh(\xi_2).
    \end{equation}
    Nous mettons cette relation au carré et nous substituons \( \cosh(\xi_1)^2=1+\sinh^2(\xi_1)\). Ce que nous trouvons est
    \begin{equation}
        \sinh(\xi_1)^2=\sinh(\xi_2)^2,
    \end{equation}
    qui implique que \( \xi_1=\pm\xi_2\). Nous posons donc \( \xi_2=\sigma_3\xi_1\) pour un certain \( \sigma_3\in \{ \pm 1 \}\). Cela nous permet d'alléger la notation et d'écrire \( \xi\) au lieu de \( \xi_1\).
    
    Nous remettons la valeur \( \xi=\xi_1=\sigma_3\xi_2\) dans l'équation \eqref{EQooHTMSooVYzJUS} en tenant compte du fait que \( \sinh\) est impaire et \( \cosh\) est paire :
    \begin{equation}
        \sigma_1\sigma_3\cosh(\xi)\sinh(\xi)=\sigma_2\sinh(\xi)\cosh(\xi).
    \end{equation}
    Et cela nous enseigne que \( \sigma_3=\sigma_1\sigma_2\).

    Jusqu'à présent nous avons prouvé qu'il existe un unique \( \xi\in \eR\) et \( \sigma_1,\sigma_2\in \{ \pm 1 \}\) tels que
    \begin{equation}        \label{EQooYZIVooCTdmSh}
        A=\begin{pmatrix}
            \sigma_1\cosh(\xi)    &   \frac{ \sigma_1\sigma_2 }{ c }\sinh(\xi)    \\ 
            c\sinh(\xi)    &   \sigma_2\cosh(\xi)    
        \end{pmatrix}.
    \end{equation}
    
Nous utilisons à présent la bijection du lemme \ref{LEMooPZPZooVAdPVj}. Il existe un unique \( v\in \mathopen] -v , v \mathclose[\) tel que \( \sinh(\xi)=\varphi(v)\). En utilisant \( \cosh(\xi)^2=1+\varphi(v)^2\), nous trouvons
    \begin{equation}
        \cosh(\xi)^2=\frac{1}{ 1-\frac{ v^2 }{ c^2 } }.
    \end{equation}
    Mais comme le cosinus hyperbolique est toujours strictement positif, nous pouvons prendre la racine carrée des deux côtés :
    \begin{equation}
        \cosh(\xi)=\frac{1}{ \sqrt{ 1-\frac{ v^2 }{ c^2 } } }.
    \end{equation}
    En substituant dans \eqref{EQooYZIVooCTdmSh}, nous trouvons le résultat annoncé.
\end{proof}

%+++++++++++++++++++++++++++++++++++++++++++++++++++++++++++++++++++++++++++++++++++++++++++++++++++++++++++++++++++++++++++
\section{Trigonométrie}
%+++++++++++++++++++++++++++++++++++++++++++++++++++++++++++++++++++++++++++++++++++++++++++++++++++++++++++++++++++++++++++

%---------------------------------------------------------------------------------------------------------------------------
\subsection{Définitions, périodicité et quelques valeurs remarquables}
%---------------------------------------------------------------------------------------------------------------------------

\begin{propositionDef}[Défintion du cosinus et du sinus]        \label{PROPooZXPVooBjONka}
    La série
    \begin{equation}
        \cos(x)=\sum_{n=0}^{\infty}\frac{ (-1)^n }{ (2n)! }x^{2n}
    \end{equation}
    définit une fonction \( \cos\colon \eR\to \eR\) de classe \(  C^{\infty}\). Nous l'appelons \defe{cosinus}{cosinus}.

    La série
    \begin{equation}        \label{EQooCMRFooCTtpge}
        \sin(x)=\sum_{n=0}^{\infty}\frac{ (-1)^n }{ (2n+1)! }x^{2n+1}
    \end{equation}
    définit une fonction \( \sin\colon \eR\to \eR\) de classe \(  C^{\infty}\). Nous l'appelons \defe{sinus}{sinus}.
\end{propositionDef}

\begin{proof}
    La série entière définissant \( \cos(x)\) a pour coefficients
    \begin{equation}
        a_n=\begin{cases}
            0    &   \text{si } n\text{ est impair}\\
            \frac{ (-1)^{n/2} }{ n! }    &   \text{si } n\text{ est pair}.
        \end{cases}
    \end{equation}
    Nous pouvons la majorer par la série entière donnée par les coefficients
    \begin{equation}
        b_n=\begin{cases}
            1/n!    &   \text{si } n\text{ est impair}\\
            \frac{ (-1)^{n/2} }{ n! }    &   \text{si } n\text{ est pair}.
        \end{cases}
    \end{equation}
    Quelle que soit la parité de \( k\) nous avons toujours
    \begin{equation}
        | \frac{ b_{k+1} }{ b_k } |=\frac{1}{ k+1 },
    \end{equation}
    de telle sorte que la formule d'Hadamard \eqref{EqAlphaSerPuissAtern} nous donne \( R=\infty\) pour la série \( \sum_{k=0}^{\infty}b_kx^k\). A fortiori\footnote{Remarque~\ref{REMooYOTEooKvxHSf}.} le rayon de convergence pour la série du cosinus est infini.

    L'assertion concernant le sinus se démontre de même.

    En ce qui concerne le fait que les fonctions \( \sin\) et \( \cos\) sont de classe \(  C^{\infty}\) sur \( \eR\), il faut invoquer le corolaire~\ref{CorCBYHooQhgara}.
\end{proof}

Par substitution directe dans les séries, nous avons immédiatement
\begin{subequations}        \label{SUBEQooTTNNooXzApSM}
    \begin{align}
        \cos(0)&=1\\
        \sin(0)&=1.
    \end{align}
\end{subequations}

\begin{lemma}       \label{LEMooBBCAooHLWmno}
    En ce qui concerne la dérivation, nous avons
    \begin{subequations}
        \begin{align}
            \sin'&=\cos\\
            \cos'&=-\sin.
        \end{align}
    \end{subequations}
\end{lemma}

\begin{proof}
    Il s'agit de se permettre de dériver terme à terme (proposition~\ref{ProptzOIuG}) les séries qui définissent le sinus et le cosinus.
\end{proof}

\begin{lemma}       \label{LEMooAEFPooGSgOkF}
    Les fonctions sinus et cosinus vérifient
    \begin{equation}        \label{EQooNYCZooApyyRd}
        \cos^2(x)+\sin^2(x)=1
    \end{equation}
    pour tout \( x\in \eR\).
\end{lemma}

\begin{proof}
    Posons \( f(x)=\sin^2(x)+\cos^2(x)\) et dérivons :
    \begin{equation}
        f'(x)=2\sin(x)\cos(x)+2\cos(x)(-)\sin(x)=0.
    \end{equation}
    La fonction \( f\) est donc constante par le corolaire~\ref{CORooEOERooYprteX}. Nous avons donc pour tout \( x\) :
    \begin{equation}
        f(x)=f(0)=\sin^2(0)+\cos^2(0)=1.
    \end{equation}
    Le dernier calcul s'obtient en substituant directement \( x\) par zéro dans les séries : \( \sin(0)=0\) et \( \cos(0)=1\).
\end{proof}

%--------------------------------------------------------------------------------------------------------------------------- 
\subsection{Fonction puissance (pour les complexes)}
%---------------------------------------------------------------------------------------------------------------------------

La fonction puissance a déjà fait l'objet de nombreuses définitions et extensions. Voir le thème \ref{THEMEooBSBLooWcaQnR}. Nous allons maintenant définir \( a^z\) pour \( a>0\) et \( z\in \eC\). 

\begin{definition}      \label{DEFooRBTDooNLcWGj}
    Pour le nombre \( e\in \eR\) et le nombre imaginaire pur \( iy\) (\( y\in \eR\)), nous définissons
    \begin{equation}
        e^{iy}=\exp(iy)
    \end{equation}
    où \( \exp\) est la série usuelle de la définition \ref{DEFooSFDUooMNsgZY}. Pour un nombre complexe général \( x+yi\) nous définissons
    \begin{equation}
        e^{x+iy}= e^{x} e^{iy}.
    \end{equation}
    Et enfin, si \( a>0\) et si \( z\in \eC\) nous définissons
    \begin{equation}
        a^z= e^{z\ln(a)},
    \end{equation}
    la fonction logarithme ici étant celle \( \ln\colon \mathopen] 0 , \infty \mathclose[\to \eR\) définie par la proposition \ref{DEFooELGOooGiZQjt}.
\end{definition}

Si \( z\in \eC\) et si \( n\in \eZ\), la définition de \( z^n\) ne pose pas de problèmes, c'est la définition \ref{DEFooGVSFooFVLtNo}.

\begin{normaltext}  \label{DefJilXoM}
    Soit \( z=x+iy\in \eC\). L'exponentielle \( \exp(x+yi)\) est déjà définie en \ref{DEFooSFDUooMNsgZY}; elle est la fonction donnée par
    \begin{equation}
        \begin{aligned}
            \exp\colon \eC&\to \eC \\
            z&\mapsto \sum_{n=0}^{\infty}\frac{ z^n }{ n! }.
        \end{aligned}
    \end{equation}
\end{normaltext}

\begin{proposition}     \label{PROPooXEYFooIEaPvU}
Le rayon de convergence\footnote{Définition \ref{DefZWKOZOl}.} de la série exponentielle est infini.
\end{proposition}

\begin{proof}
    La formule de Hadamard de la proposition \ref{PROPooMXCDooBffXbl} est à utiliser avec \( a_k=j!\). Nous avons
    \begin{equation}
        \frac{1}{ R }=\lim_{k\to \infty} \left| \frac{ (n+1)! }{ n! } \right| =\lim_{k\to \infty} (n+1)=\infty.
    \end{equation}
    Donc \( R=\infty\).
\end{proof}

\begin{proposition}
    Pour tout \( z\in \eC\) nous avons
    \begin{equation}
        \exp(z)= e^{z}.
    \end{equation}
\end{proposition}

\begin{proposition}[\cite{RomainBoilEnt}]     \label{PropdDjisy}
    Quelques propriétés de l'exponentielle.
    \begin{enumerate}
        \item
            Le fonction \( \exp\) est continue.
        \item       \label{ITEMooRLHCooJTuYKV}
            Nous avons la formule \(  e^{z+w}= e^{z}e^w\) pour tout \( z,w\in \eC\).
        \item
            \( (e^z)^{-1}= e^{-z}\)
        \item
            \( (\exp(z))^n=\exp(nz)\).
    \end{enumerate}
\end{proposition}

\begin{proof}
    La proposition \ref{PROPooXEYFooIEaPvU} nous enseigne que le rayon de convergence est infini. La fonction ainsi définie est alors continue par la proposition \ref{PropUEMoNF}.

    Les séries \( \exp(z)\) et \( \exp(w)\) ayant un rayon de convergence infini nous pouvons utiliser le produit de Cauchy (théorème~\ref{ThokPTXYC}) :
    \begin{subequations}
        \begin{align}
            e^{z} e^{w}&=\sum_{n=0}^{\infty}\left( \sum_{i+j=n}\frac{ z^iw^j }{ i!j! } \right)\\
            &=\sum_{n=0}^{\infty}\left( \sum_{i=0}^n\frac{ z^iw^{n-i} }{ i!(n-i)! } \right)\\
            &=\sum_{n=0}^{\infty}\frac{1}{ n! }\sum_{i=0}^{n}{n\choose i}z^iw^{n-i}\\
            &=\sum_{n=0}^{\infty}\frac{1}{ n! }(z+w)^{n}\\
            &=\exp(z+w).
        \end{align}
    \end{subequations}
    Nous avons utilisé la formule du binôme (proposition~\ref{PropBinomFExOiL}).

    Les autres propriétés énoncées sont des corolaires :
    \begin{equation}
        e^{z} e^{-z}= e^{0}=1.
    \end{equation}
\end{proof}

D'autres propriétés de l'exponentielle sur \( \eC\), entre autres l'holomorphie, sont données dans le théorème \ref{THOooNGOIooEECfAv}.


\begin{lemma}[\cite{MonCerveau}]        \label{LEMooTDGKooWdpUTD}
    Soient \( a>0\), \( z\in \eC\) et \( n\in \eZ\). Alors
    \begin{equation}
        (a^z)^n=a^{nz}.
    \end{equation}
\end{lemma}

%--------------------------------------------------------------------------------------------------------------------------- 
\subsection{Formules de trigonométrie}
%---------------------------------------------------------------------------------------------------------------------------

Le lemme suivant est un premier pas pour le paramétrage du cercle dont nous parlerons dans la proposition \ref{PROPooZEFEooEKMOPT}.
\begin{lemma}       \label{LEMooHOYZooKQTsXW}
    Nous avons la formule
    \begin{equation}        \label{EQooRVPJooTMwNTU}
        e^{ix}=\cos(x)+i\sin(x)
    \end{equation}
    pour tout \( x\in \eR\).

    En particulier pour tout \( x\), nous avons \( |  e^{ix} |=1\).
\end{lemma}

\begin{proof}
    La définition de l'exponentielle sur \( \eC\) est la définition~\ref{DEFooSFDUooMNsgZY}. Cette définition fonctionne parce que \( \eC\) est une algèbre normée, et que \( \eC\) est un \( \eC\)-module vérifiant l'inégalité \(  | zz' |\leq | z | |z' | \) (en l'occurrence, une égalité).

    Nous remarquons que que \( i^k\) vaut \( 1\), \( i\), \( -1\), \( -i\). Donc un terme sur deux est imaginaire pur et parmi ceux-là, un sur deux est positif. À bien y regarder, les termes imaginaires purs forment la série du sinus et ceux réels la série du cosinus.

    Si vous aimez les formules,
    \begin{equation}
            e^{iy}=\sum_{n=0}^{\infty}\frac{ (iy)^n }{ n! }
            =\sum_{n=0}^{\infty}(-1)^n\frac{ y^{2n} }{ (2n)! }+i\sum_{n=0}^{\infty}(-1)^n\frac{ y^{2n+1} }{ (2n+1)! }
            =\cos(y)+i\sin(y).
    \end{equation}
    Nous avons utilisé le fait que \( i^{2n}=(-1)^n\) et \( i^{2n+1}=i(-1)^n\).
\end{proof}

\begin{lemma}       \label{LEMooJAWBooJGfZIL}
    Nous avons les formules d'addition d'angles\footnote{Rien ne nous empêche de donner ce nom à ces formules, mais seriez-vous capable de définir précisément le mot «angle» ?}
    \begin{subequations}        \label{SUBEQSooFSSMooHcYwRc}
        \begin{align}
            \cos(a+b)=\cos(a)\cos(b)-\sin(a)\sin(b) \label{EQooJYEMooQaOMib}\\
            \sin(a+b)=\cos(a)\sin(b)+\sin(a)\cos(b) \label{EQooECAUooQzckDv}\\
            \cos(a-b)=\cos(a)\cos(b)+\sin(a)\sin(b) \label{EQooCVZAooQfocya}
        \end{align}
    \end{subequations}
    pour tout \( a\), \( b\) réels.
\end{lemma}

\begin{proof}
    Nous utilisons la formule d'addition dans l'exponentielle, proposition \eqref{EQooVFXUooBfwjJY} et la formule \eqref{EQooRVPJooTMwNTU} avant de séparer les parties réelles et imaginaires :
    \begin{equation}
        e^{i(a+b)}= e^{ia} e^{ib}=\cos(a)\cos(b)-\sin(a)\sin(b)+i\big( \cos(a)\sin(b)+\sin(a)\cos(b) \big).
    \end{equation}
    Cela est également égal à
    \begin{equation}
        \cos(a+b)+i\sin(a+b).
    \end{equation}
    En identifiant les parties réelle et imaginaires, nous obtenons les formules \eqref{EQooJYEMooQaOMib} et \eqref{EQooCVZAooQfocya} annoncées.

    Pour la formule \eqref{EQooCVZAooQfocya}, il suffit de se souvenir que \( \sin(-b)=-\sin(b)\) et \( \cos(-b)=\cos(b)\) (ces deux égalités sont immédiatement visibles sur les développements en série : l'un a uniquement des puissances paires et l'autre impaires) et d'écrire \eqref{EQooJYEMooQaOMib} avec \( -b\) au lieu de \( b\).
\end{proof}

\begin{corollary}       \label{CORooQZDQooWjMXTF}
    Les formules suivantes pour les duplications d'angles s'ensuivent :
    \begin{subequations}
        \begin{align}
            \cos(2a)&=\cos^2(a)-\sin^2(a)\\
            \sin(2a)&=2\cos(a)\sin(a).      \label{SUBEQooLRJDooQuFvux}
        \end{align}
    \end{subequations}
\end{corollary}

\begin{proof}
    Poser \( b=a\) dans les relations du lemme~\ref{LEMooJAWBooJGfZIL}.
\end{proof}

\begin{lemma}       \label{LEMooPQWWooMdPWUT}
    Un sous-groupe de \( (\eR,+)\) est soit dense dans \( \eR\) soit de la forme \( p\eZ\) pour un certain réel \( p\neq 0\).
\end{lemma}

\begin{proof}
    Soit \( A\), un sous-groupe de \( (\eR,+)\) qui ne soit pas dense. Soit un intervalle \( \mathopen] a , b \mathclose[\) qui n'intersecte pas \( A\) (si vous voulez frimer, vous noterez ici que nous utilisons le fait que les intervalles ouverts forment une base de la topologie de \( \eR\)). Si \( d=| b-a |\), l'ensemble \( A\) ne contient pas deux éléments séparés par strictement moins de \( d\). Soit \( p\), le plus petit élément strictement positif de \( A\); nous avons \( p\geq d\) (parce que \( 0\in A\) de toutes façons).

        Vu que \( A\) est un groupe nous avons \( p\eZ\subset A\).

        Pour l'inclusion inverse, si \( x\in A\) est hors de \( p\eZ\), il existe un \( y\in p\eZ\) avec \( | x-y |<p\). Et donc le nombre \( | x-y |\) est dans \( A\) tout en étant plus petit que \( p\). Contradiction.
\end{proof}

\begin{propositionDef}[Périodicité, le nombre \( \pi\)\cite{ooUMDHooHrJpfV}]      \label{PROPooFRVCooKSgYUM}
    Plusieurs choses à propos de la périodicité de la fonction \( \cos\).
    \begin{enumerate}
        \item
            La fonction \( \cos\) est périodique.
        \item
            Un nombre \( T>0\) est une période si et seulement si \( \cos(T)=1\) et \( \sin(T)=0\).
    \end{enumerate}
    
    Nous définissons le nombre \( \pi>0\) comme étant la moitié de la période de la fonction \( \cos\) :
    \begin{equation}
        2\pi=\min\{ T>0\tq \cos(x+T)=\cos(x)\,\forall x \}.
    \end{equation}

\end{propositionDef}

\begin{proof}
    Plusieurs étapes.
    \begin{subproof}
        \item[La fonction cosinus n'est pas toujours positive]
    Supposons d'abord que \( \cos(x)>0\) pour tout \( x\in \eR\). Dans ce cas, la fonction \( \sin\) est strictement croissante. Mais les deux fonctions sont bornées par \( 1\) du fait de la formule \( \cos^2(x)+\sin^2(x)=1\). La fonction \( \sin\) étant croissante et bornée, elle est convergente vers un réel par la proposition~\ref{PropMTmBYeU} :
    \begin{equation}
        \lim_{x\to \infty} \sin(x)=\ell
    \end{equation}
    pour un certain \( \ell>0\). Avec ça nous avons aussi (pour cause de dérivée) \( \lim_{x\to \infty} \sin'(x)=0\), c'est-à-dire \( \lim_{x\to \infty} \cos(x)=0\). Mais vu que \( \cos^2(x)+\sin^2(x)=1\) nous en déduisons que \( \lim_{x\to \infty} \sin(x)=1\). Mézalor \( \lim_{x\to \infty} \cos'(x)=-1\), ce qui donne que la fonction \( \cos\) n'est pas bornée. Cela est impossible. Nous en déduisons que \( \cos(x)\) n'est pas toujours positive.

\item[Il existe \( T>0\) tel que \( \cos(T)=1\) et \( \sin(T)=0\)]

    Par ce que nous venons de faire, il existe \( r>0\) tel que \( \cos(r)=0\). Pour cette valeur, nous avons aussi obligatoirement \( \sin(r)=\pm 1\). Nous avons aussi, en utilisant les formules \eqref{SUBEQSooFSSMooHcYwRc},
    \begin{subequations}
        \begin{align}
            \cos(2r)=\cos^2(r)-\sin^2(r)=-1\\
            \sin(2r)=2\cos(r)\sin(r)=0.
        \end{align}
    \end{subequations}
    et par conséquent
    \begin{subequations}
        \begin{align}
            \cos(4r)=\cos^2(2r)-\sin^2(2r)=1\\
            \sin(4r)=2\cos(2r)\sin(2r)=0.
        \end{align}
    \end{subequations}
    Donc \( T=4r\) fonctionne.

\item[Si \( T\) est une période]
    Nous entrons dans le vif de la preuve. Soit un \( T>0\) tel que \( \cos(x+T)=\cos(x)\) pour tout \( x\in \eR\). Avec la formule d'addition d'angle dans le cosinus nous cherchons un \( T\) tel que
    \begin{equation}
        \cos(x+T)=\cos(x)\cos(T)-\sin(x)\sin(T)=\cos(x)
    \end{equation}
    et donc tel que
    \begin{equation}        \label{EQooELSAooLNtBnm}
        \cos(x)\big( \cos(T)-1 \big)=\sin(x)\sin(T).
    \end{equation}
    Nous dérivons cette équation :
    \begin{equation}        \label{EQooCECFooLpxXaw}
        -\sin(x)\big( \cos(T)-1 \big)=\cos(x)\sin(T).
    \end{equation}
    Nous multiplions chacune des deux équations \eqref{EQooELSAooLNtBnm} et \eqref{EQooCECFooLpxXaw} par \( \sin(x)\) et \( \cos(x)\) pour obtenir les quatre relations suivantes :
    \begin{subequations}
        \begin{align}
            \cos^2(x)\big( \cos(T)-1 \big)-\sin(x)\cos(x)\sin(T)=0   \label{SUBEQooLGQXooIrLMLW}\\
            -\sin(x)\cos(x)\big( \cos(T)-1 \big)-\cos^2(x)\sin(T)=0     \label{SUBEQooCHTDooKwvyZF}\\
            \sin(x)\cos(x)\big( \cos(T)-1 \big)-\sin^2(x)\sin(T)=0 \label{SUBEQooEWPTooTLCUMf}\\
            \sin^2(x)\big( \cos(T)-1 \big)-\sin(x)\cos(x)\sin(T)=0  \label{SUBEQooGBXTooCFekGJ}
        \end{align}
    \end{subequations}
    En faisant \eqref{SUBEQooLGQXooIrLMLW} moins \eqref{SUBEQooGBXTooCFekGJ} nous trouvons \( \cos(T)=1\). Et en sommant \eqref{SUBEQooCHTDooKwvyZF} avec \eqref{SUBEQooEWPTooTLCUMf} nous avons \( -\sin(T)=0\).

\item[Si \( T>0\) est tel que \( \sin(T)=0\) et \( \cos(T)=1\)]

    Alors les formules d'addition d'angle du lemme \ref{LEMooJAWBooJGfZIL} donnent tout de suite
    \begin{equation}
        \cos(x+T)=\cos(x).
    \end{equation}

    \end{subproof}

    À de niveau nous croyons avoir prouvé que \( \cos\) était périodique et que la période est donnée par
    \begin{equation}
        \min\{ T>0\tq \sin(T)=0,\cos(T)=1 \}.
    \end{equation}
    Or rien n'est moins sûr parce qu'il pourrait arriver que ce minimum n'existe pas, c'est-à-dire que l'infimum soit zéro. Autrement dit, il peut arriver que l'ensemble des périodes soit dense. Plus précisément, soit \( P\subset \eR\) l'ensemble des périodes de \( \cos\). C'est un sous-groupe de \( (\eR,+)\) et le lemme~\ref{LEMooPQWWooMdPWUT} nous dit que \( P\) est soit dense dans \( \eR\) soit de la forme \( p\eZ\) pour un \( p>0\).

    Si \( P\) est dense, soit \( t\in \eR\) et une suite \( (t_n)\) dans \( P\) telle que \( t_n\to t\). Pour tout \( x\) et tout \( n\) nous avons
    \begin{equation}
        \cos(x+t_n)=\cos(x),
    \end{equation}
    Vu que la fonction cosinus est continue, nous pouvons passer à la limite et écrire \( \cos(x+t)=\cos(x)\). Cela étant valable pour tout \( x\) et pour tout \( t\), la fonction cosinus est constante. Or nous savons que ce n'est pas le cas, donc \( P\) n'est pas dense. Donc cosinus est périodique.
\end{proof}

\begin{proposition}     \label{PROPooKNLAooLwQHea}
    La fonction \( \sin\) est périodique de période \( \pi\) et
    \begin{equation}
        2\pi=\min\{ T>0\tq \sin(T)=0,\cos(T)=1 \}.
    \end{equation}
\end{proposition}

\begin{normaltext}
Notons que tout ceci ne nous donne pas la plus petite indication d'ordre de grandeur de la valeur de \( \pi\). Cela peut encore être \( 0.1\) autant que \( 500\).
\end{normaltext}

\begin{proposition}[\cite{ooUMDHooHrJpfV}]      \label{PROPooMWMDooJYIlis}
    Des propriétés à la chaine à propos des sinus, cosinus et de leurs périodes.
    \begin{enumerate}
        \item       \label{ITEMooRJZHooCXcKmM}
            Le nombre \( 2\pi\) est le plus petit nombre strictement positif tel que
            \begin{subequations}
                \begin{numcases}{}
                    \cos(2\pi)=1\\
                    \sin(2\pi)=0.
                \end{numcases}
            \end{subequations}
        \item
            Les fonctions \( \sin\) et \( \cos\) sont périodiques de période \( 2\pi\).
        \item
            Nous avons \( \cos(\pi)=- 1\) et \( \sin(\pi)=0\).
        \item
            Pour tout \( a\in \eR\) nous avons
            \begin{subequations}
                \begin{align}
                    \cos(a+\pi)&=-\cos(\pi)\\
                    \sin(a+\pi)&=-\sin(\pi).
                \end{align}
            \end{subequations}
        \item       \label{ITEMooHDQNooYHVCkg}
            Le nombre \( \pi\) est le plus petit \( T>0\) tel que \( \cos(T)=-1\) et \( \sin(T)=0\).
        \item       \label{ITEMooWFNUooYAybDB}
            Nous avons
            \begin{subequations}        \label{SUBEQSooBTNPooSvCAHO}
                \begin{numcases}{}
                    \cos(\pi/2)=0\\
                    \sin(\pi/2)=1.
                \end{numcases}
            \end{subequations}
        \item
            Nous avons les formules
            \begin{subequations}        \label{EQSooRJZGooCFVqbZ}
                \begin{numcases}{}
                    \cos(x+\pi/2)=-\sin(x)\\
                    \sin(x+\pi/2)=\cos(x)
                \end{numcases}
            \end{subequations}
            pour tout \( x\in \eR\).
        \item
            Le nombre \( \pi/2\) est le plus petit \( T\) vérifiant \( \sin(T)=1\), \( \cos(T)=0\).
        \item
            Nous avons les valeurs
            \begin{subequations}
                \begin{numcases}{}
                    \cos(\frac{ 3\pi }{ 2 })=0\\
                    \sin(\frac{ 3\pi }{ 2 })=-1.
                \end{numcases}
            \end{subequations}
        \item       \label{ITEMooQKPKooEPeHER}
            Le nombre \( \pi/2\) est le plus petit \( T>0\) tel que \( \cos(T)=0\).
    \end{enumerate}
\end{proposition}

\begin{proof}
    C'est parti.
    \begin{enumerate}
        \item
            Le fond de la proposition~\ref{PROPooFRVCooKSgYUM} est que toutes les périodes \( T>0\) vérifient \( \cos(T)=1\) et \( \sin(T)=0\). La définition de \( \pi\) est que c'est la plus petite période.
        \item
            En utilisant le fait que l'une est la dérivée de l'autre, si \( T\) est une période de \( \cos\) nous avons
            \begin{subequations}
                \begin{align}
                    \sin(x+T)&=-\cos'(x+T)\\
                    &=-\lim_{\epsilon\to 0}\frac{ \cos(x+T+\epsilon)-\cos(x+T) }{\epsilon  }\\
                    &=-\lim_{\epsilon\to 0}\frac{ \cos(x+\epsilon)-\cos(x) }{ \epsilon }\\
                    &=-\cos'(x)\\
                    &=\sin(x).
                \end{align}
            \end{subequations}
            Nous déduisons que toute période de \( \cos\) est une période de \( \sin\). De la même façon, nous pouvons prouver le contraire : toute période de \( \sin\) est une période de \( \cos\).
        \item
            D'un côté nous avons
            \begin{equation}
                \cos(2\pi)=\cos^2(\pi)-\sin^2(\pi)=1
            \end{equation}
            parce que \( \cos(2\pi)=\cos(0)=1\). Vu que \( \cos(\pi)\) et \( \sin(\pi)\) sont bornés par \( -1\) et \( 1\), nous devons avoir \( \sin(\pi)=0\) et \( \cos(\pi)=\pm 1\).

            Mais d'un autre côté, le nombre \( 2\pi\) est le plus petit \( T\) vérifiant \( \cos(T)=1\), \( \sin(T)=0\). Donc avoir \( \cos(\pi)=1\) n'est pas possible. Nous concluons
            \begin{subequations}
                \begin{numcases}{}
                    \cos(\pi)=-1\\
                    \sin(\pi)=0.
                \end{numcases}
            \end{subequations}
        \item
            Il s'agit d'utiliser les formules d'addition d'angles du lemme~\ref{LEMooJAWBooJGfZIL} pour calculer \( \cos(a+\pi)\) et \( \sin(a+\pi)\) en tenant compte du fait que \( \cos(\pi)=-1\) et \( \sin(\pi)=0\).
        \item
        Soit \( a\in\mathopen] 0 , \pi \mathclose[\) tel que \( \cos(a)=-1\) et \( \sin(a)=0\). Alors nous avons
            \begin{subequations}
                \begin{align}
                    \cos(a+\pi)=-\cos(\pi)=1\\
                    \sin(a+\pi)=-\sin(\pi)=0,
                \end{align}
            \end{subequations}
        ce qui donnerait \( a+\pi\in\mathopen] \pi , 2\pi \mathclose[\) dont le cosinus est \( 1\) et le sinus est zéro. Mais nous savons déjà que \( 2\pi\) est le minimum pour cette propriété.
        \item
            Nous avons
            \begin{equation}
                -1=\cos(\pi)=\cos^2(\pi/2)-\sin^2(\pi/2),
            \end{equation}
            donc \( \cos(\pi/2)=0\) et \( \sin^2(\pi/2)=1\), ce qui donne \( \sin(\pi/2)=\pm 1\).

        Nous devons départager le \( \pm\). Pour cela nous savons que \( \sin'(0)=\cos(0)=1\), donc il existe \( \epsilon>\epsilon\) tel que pour tout \( x\in\mathopen] 0 , \epsilon \mathclose[\) nous avons \( 0<\cos(x)<1\) et \( 0\sin(x)<1\). Nous choisissons \( \epsilon\) plus petit que \( \pi/2\) .

        Supposons que \( \sin(\pi/2)=-1\). Le théorème des valeurs intermédiaires~\ref{ThoValInter} dit qu'il existe \( x_0\in\mathopen] \epsilon , \pi/2 \mathclose[\) tel que \( \sin(x_0)=0\). Pour cette valeur de \( x_0\) nous devons aussi avoir \( \cos(x_0)=\pm 1\). Mais vu que \( 2\pi\) est minium pour avoir \( \cos=1\) et \( \sin=0\) nous devons avoir \( \cos(x_0)=-1\). Alors nous avons aussi
            \begin{subequations}
                \begin{align}
                    \cos(x_0+\pi)=\cos(x_0)\cos(\pi)-\sin(x_0)\sin(\pi)=-\cos(x_0)=1\\
                    \sin(x_0+\pi)=\cos(x_0)\sin(\pi)+\sin(x_0)\cos(\pi)=\sin(x_0)=0.
                \end{align}
            \end{subequations}
            Encore une fois par minimalité de \( 2\pi\), cela ne va pas. Conclusion : \( \sin(\pi/2)=1\).
        \item
            Il s'agit encore d'utiliser les formules d'addition d'angle en tenant compte du fait que \( \cos(\pi/2)=0\) et \( \sin(\pi/2)=1\).
        \item
        Supposons \( x_0\in\mathopen] 0 , \pi/2 \mathclose[\) tel que \( \sin(x_0)=1\) et \( \cos(x_0)=0\). En utilisant les formules \eqref{EQSooRJZGooCFVqbZ} nous avons
            \begin{subequations}
                \begin{align}
                    \cos(x_0+\pi/2)=-1\\
                    \sin(x_0+\pi/2)=0,
                \end{align}
            \end{subequations}
            avec \( x_0+\pi/2<\pi\). Cela contredirait la minimalité de \( \pi\).
        \item
            Il s'agit d'utiliser les formules \eqref{EQSooRJZGooCFVqbZ} :
            \begin{subequations}
                \begin{align}
                    \cos(\frac{ 3\pi }{ 2 })=\cos(\pi+\pi/2)=-\sin(\pi)=0\\
                    \sin(\frac{ 3\pi }{ 2 })=\sin(\pi+\pi/2)=\cos(\pi)=-1.
                \end{align}
            \end{subequations}
        \item
            Si \( \cos(x_0)=0\) alors \( \sin(x_0)=-1\) (parce que \( \sin(x_0)=1\) est déjà exclu). Alors \( \cos(x_0+\pi/2)=1\) et \( \sin(x_0+\pi/2)=0\), ce qui est également impossible.
    \end{enumerate}
\end{proof}

\begin{corollary}   \label{CORooTFMAooHDRrqi}
    Des nombres \( x,y\in \eR\) vérifient \(  e^{ix}= e^{iy}\) si et seulement si il existe \( k\in\eZ\) tel que \( y=x+2k\pi\).
\end{corollary}

Tout cela nous permet de calculer quelques valeurs remarquables de cosinus et sinus ainsi que d'écrire le tableau de variations de sinus et cosinus.

\begin{lemma}       \label{LEMooIGNPooPEctJy}
    Nous avons les valeurs remarquables
    \begin{equation}
        \sin(\frac{ \pi }{ 4 })=\cos(\frac{ \pi }{ 4 })=\frac{ \sqrt{ 2 } }{2}.
    \end{equation}
\end{lemma}

\begin{proof}
    La relation \eqref{SUBEQooLRJDooQuFvux} donne
    \begin{equation}
        0=\cos(\pi/2)=\cos^2(\pi/4)-\sin^2(\pi/4).
    \end{equation}
    Donc \( \cos^2(\pi/4)=\sin^2(\pi/4)\). Mais vu que \( \sin(\pi/4)\) et \( \cos(\pi/4)\) sont positifs, ils sont égaux.

    Mais \( \sin^2(\pi/4)+\cos^2(\pi/4)=1\). Donc le nombre \( x=\cos(\pi/4)=\sin(\pi/4)\) vérifie l'équation \( 2x^2=1\), donc l'unique solution positive est \( x=\frac{1}{ \sqrt{ 2 } }=\frac{ \sqrt{ 2 } }{2}\).
\end{proof}

\begin{lemma}       \label{LEMooRMHAooDEAPMw}
    Nous avons la valeur remarquable
    \begin{equation}
        \cos(\frac{ \pi }{ 3 })=\frac{ 1 }{2}.
    \end{equation}
\end{lemma}

\begin{proof}
    Il faut utiliser la formule \eqref{EQooJYEMooQaOMib} avec \( \cos(\pi)=\cos(2\pi/3+\pi/3)\) en sachant que \( \cos(\pi)=-1\). Ensuite \( \cos(2\pi/3)=\cos(\pi/3+\pi/3)\). En décomposant ainsi, nous exprimons \( -1=\cos(\pi)\) en termes de \( \cos(\pi/3)\) et de \( \sin(\pi/3)\). En substituant \( \sin^2(\pi/3)=1-\cos^2(\pi/3)\) nous trouvons que le nombre \( \cos(\pi/3)\) vérifie l'équation
    \begin{equation}
        4x^3-3x+1=0.
    \end{equation}
    Croyez-le ou non, les solutions de cette équation sont \( x=-1\) et \( x=1/2\). Allez. Faisons comme si nous le savions pas. En tout cas, ces deux nombres sont des solutions, et nous avons la factorisation
    \begin{equation}
        4x^3-3x+1=(2x-1)^2(x+1).
    \end{equation}
    Donc \( 1/2\) est de multiplicité \( 2\) et \( -1\) de multiplicité \( 1\). Le théorème~\ref{ThoSVZooMpNANi} nous dit qu'il n'y a alors pas d'autres racines que ces deux-là\footnote{Nous attirons votre attention sur le fait que cela n'est en aucun cas une trivialité.}.

    Nous en déduisons que la valeur de \( \cos(\pi/3)\) est soit \( 1/2\) soit \( -1\). La proposition~\ref{PROPooMWMDooJYIlis}\ref{ITEMooHDQNooYHVCkg} nous dit qu'il est impossible que \( \cos(\pi/3)\) soit égal à \( -1\) parce que \( \pi/3<\pi\). Donc \( \cos(\pi/3)=1/2\) comme annoncé.
\end{proof}

\begin{remark}
    Vous avez déjà sans doute vu la démonstration de \( \cos(\unit{30}{\degree})=1/2\) à partir de la figure~\ref{LabelFigGVDJooYzMxLW}. Il n'est pas possible de l'utiliser parce que cela n'est en réalité pas loin d'être la définition de l'angle entre deux droites.

    Si vous voulez savoir la définition de l'angle entre deux droites, il faut passer par la définition~\ref{DEFooFLGNooCZUkHY}, laquelle se base sur le lemme~\ref{LEMooHRESooQTrpMz} qui, elle-même, se base sur la proposition~\ref{PROPooKSGXooOqGyZj}.

    Bref, à notre niveau, nous sommes encore loin de pouvoir faire des raisonnements trigonométriques sur base de géométrie dans les triangles.
\end{remark}

\begin{example}\label{developcosenpisur3}
    Développer la fonction \( \cos\) autour de \( x=\frac{ \pi }{ 3 }\). Utiliser la valeur remarquable du lemme \ref{LEMooRMHAooDEAPMw}. Nous développons autour de \( h=0\) la fonction \( \cos(\frac{ \pi }{ 3 }+h)\) :
    \begin{equation}
        \cos\big( \frac{ \pi }{ 3 }+h \big)\sim \cos\big( \frac{ \pi }{ 3 } \big)+h\cos'(\frac{ \pi }{ 3 })+\frac{ h^2 }{2}\cos''\big( \frac{ \pi }{ 3 } \big)=\frac{ 1 }{2}-\frac{ \sqrt{3} }{2}h-\frac{1}{ 4 }h^2.
    \end{equation}
    Il est aussi possible d'écrire cela en notant \( x=x_0+h\), c'est-à-dire en remplaçant \( h\) par \( x-\frac{ \pi }{ 3 }\) :
    \begin{equation}
        \cos(x)\sim\frac{ 1 }{2}-\frac{ \sqrt{3} }{ 2 }(x-\frac{ \pi }{ 3 })-\frac{1}{ 4 }(x-\frac{ \pi }{ 3 })^2.
    \end{equation}
\end{example}

Pour donner une idée nous avons dessiné sur le graphe suivant la fonction sinus et ses développements d'ordre \( 4\) autour de zéro et autour de \( 3\pi/4\).
\begin{center}
   \input{auto/pictures_tex/Fig_WJBooMTAhtl.pstricks}
\end{center}

Voici un tableau qui rappelle les valeurs à retenir pour les fonctions sinus, cosinus et tangente.
\begin{equation}\label{PGooIMQFooTnBdIl}
    \begin{array}[]{|c|c|c|c|}
      \hline
      x&\sin(x)&\cos(x)&\tan(x)\\
      \hline
      0&0&1&0\\
      \hline
      \pi/6&1/2&\sqrt{3}/2&\sqrt{3}/3\\
      \hline
      \pi/6&1/2&\sqrt{3}/2&\sqrt{3}/3\\
      \hline
      \pi/4&\sqrt{2}/2&\sqrt{2}/2&1\\
      \hline
      \pi/3&\sqrt{3}/2&1/2&\sqrt{3}\\
      \hline
      \pi/2&1&0&\text{N.D.}\\
      \hline
    \end{array}
\end{equation}
où «N.D.»  signifie «non défini».

Rappelons le graphe de la fonction sinus :
\begin{center}
   \input{auto/pictures_tex/Fig_TWHooJjXEtS.pstricks}
\end{center}
celui de la fonction cosinus :
\begin{center}
   \input{auto/pictures_tex/Fig_JJAooWpimYW.pstricks}
\end{center}


\begin{lemma}
  Pour toute valeur de $x\in \eR$ on a $|\sin(x)|\leq |x|$.
\end{lemma}

\begin{proof}
        Nous séparons des cas en fonction des valeurs.
    \begin{itemize}
    \item Si $0\leq x\leq \pi/2$ alors le sinus de $x$ est la longueur du cathète verticale du triangle rectangle de sommets $O = (0,0)$, $A = (\cos(x), \sin(x))$ et $B = (\cos(x), 0)$. Le triangle de sommets $A$, $B$ et $C = (1, 0)$ est aussi rectangle et nous savons que chacun des cathètes ne peut pas être plus long que l'hypoténuse. Donc $\sin(x)$ est inférieur à la longueur du segment $AC$. À son tour le segment $AC$ ne peut pas être plus long que l'arc de cercle $\wideparen{A0C}$, car le chemin le plus court entre deux points du plan est toujours donné par un morceau de droite. La longueur de l'arc de cercle $\frown{AC}$ est \emph{par définition} la mesure en radiants de l'angle $\widehat{AOC}$, qui est $x$ et on a l'inégalité $\sin(x)\leq x$.
    \item Si $-\pi/2\leq x\leq 0$ le m\^eme raisonnement que au point précédent permet de conclure que $\sin(x)\leq |x|$.
    \item Nous savons par ailleurs que la fonction sinus prend ses valeurs dans l'intervalle $[-1,1]$ et donc pour tout $x$ tel que $|x|\geq \pi/2 \equiv 1,57\ldots$ on a forcement $|\sin(x)|\leq |x|$.
    \end{itemize}
\end{proof}

\begin{subequations}
    \begin{numcases}{}
        x=r\cos\theta\\
        y=r\sin\theta
    \end{numcases}
\end{subequations}
avec \( r\in\mathopen] 0 , \infty \mathclose[\) et \( \theta\in\mathopen[ 0 , 2\pi [\). Le jacobien vaut \( r\).
