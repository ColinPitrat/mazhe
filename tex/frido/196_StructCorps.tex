% This is part of Mes notes de mathématique
% Copyright (c) 2011-2017
%   Laurent Claessens
% See the file fdl-1.3.txt for copying conditions.

%+++++++++++++++++++++++++++++++++++++++++++++++++++++++++++++++++++++++++++++++++++++++++++++++++++++++++++++++++++++++++++
\section{Extension de corps}
%+++++++++++++++++++++++++++++++++++++++++++++++++++++++++++++++++++++++++++++++++++++++++++++++++++++++++++++++++++++++++++
\label{SECooLQVJooTGeqiR}

\begin{lemma}       \label{LemobATFP}
    Soit \( \eL\) un corps fini et \( \eK\) un sous corps de \( \eK\). Alors il existe \( s\in \eN\) tel que
    \begin{equation}        \label{EqUgqlJQ}
        \Card(\eK)=\Card(\eL)^s.
    \end{equation}
\end{lemma}

\begin{proof}
    Le corps \( \eL\) est un \( \eK\)-espace vectoriel de dimension finie. Si \( s\) est la dimension alors nous avons la formule \eqref{EqUgqlJQ} parce que chaque élément de \( \eL\) est un \( s\)-uple d'éléments de \( \eK\).
\end{proof}

\begin{definition}[\cite{ooOXISooAFtXsZ}]     \label{DEFooFLJJooGJYDOe}
    Soit \( \eK\) un corps commutatif. Une \defe{extension}{extension!de corps} de \( \eK\) est un couple \( (\eL,j)\) où \( \eL\) est un corps et \( j\colon \eK\to \eL\) est un morphisme de corps.
\end{definition}

    Nous identifions le plus souvent \( \eK\) avec \( i(\eK)\subset \eL\), mais il faut savoir que le corps \( \eL\) étendant \( \eK\) n'est pas toujours un sur-corps de \( \eK\).
    
\begin{definition}      \label{DEFooREUHooLVwRuw}
    Une extension est \defe{algébrique}{extension!de corps!algébrique}\index{algébrique!extension} de \( \eK\) est une extension dont tous les éléments sont racines de polynômes dans \( \eK[X]\).
\end{definition}

\begin{example}
    Le corps \( \eR\) n'est pas une extension algébrique de \( \eQ\). En effet il existe seulement une infinité \emph{dénombrable} de polynômes dans \( \eQ[X]\) et donc une infinité dénombrable de racines de tels polynômes. Toute extension algébrique de \( \eQ\) est donc dénombrable.
\end{example}

\begin{proposition}[\cite{ooLIOMooBuCPUS}]      \label{PROPooALFJooDjmIcb} 
    Soit une extension algébrique \( \eL\) du corps $\eK$.
    \begin{enumerate}
        \item
            Pour tout \( a\in \eL\), il existe un polynôme \( P\in \eK[X]\) tel que \( P(a)=0\).
        \item       \label{ITEMooEFNFooKYqXDk}
            Le polynôme minimal de \( a\) dans \( \eK[X]\) est l'unique polynôme unitaire irréductible annulant \( a\).
    \end{enumerate}
\end{proposition}
\index{polynôme!minimal}

\begin{proof}
    Le premier point est seulement la définition \ref{DEFooREUHooLVwRuw} d'une extension algébrique.

    L'idéal annulateur \( I_a=\{ P\in \eK[X]\tq P(a)=0 \}\) n'est pas réduit à \( \{ 0 \}\) parce que \( \eL\) est une extension algébrique. L'existence du polynôme minimal est le lemme \ref{DefCVMooFGSAgL} et le fait qu'il soit irréductible est la proposition \ref{PropRARooKavaIT}\ref{ItemDOQooYpLvXri}.

    Ce qui nous intéresse ici est l'unicité. Soit \( \mu_1\in \eK[X]\), un polynôme annulateur de \( a\) irréductible et unitaire. Vu que \( \mu_1\in I_a\) et que par définition, \( I_a=(\mu)\), il existe \( P\in \eK[X]\) tel que \( \mu_1=P\mu\). Vu que \( \mu\) n'est pas inversible et que \( \mu_1\) est irréductible, \( P\) doit être inversible : \( \mu_1=k\mu\) pour un certain \( k\in \eK\).

    Vu que \( \mu\) et \( \mu_1\) sont unitaires, \( k=1\). Donc \( \mu_1=\mu\).
\end{proof}

%--------------------------------------------------------------------------------------------------------------------------- 
\subsection{Extensions et polynômes}
%---------------------------------------------------------------------------------------------------------------------------

Nous savons déjà depuis la définition \ref{DefRGOooGIVzkx} ce qu'est \( A[X]\) pour tout anneau \( A\) et donc a fortiori pour un corps.

\begin{definition}  \label{DEFooHUWWooHiuRBr}
    Le corps \( \eK(X)\) est le corps des fractions\footnote{Définition \ref{DEFooGJYXooOiJQvP}.} de \( \eK[X]\).
\end{definition}


\begin{lemmaDef}        \label{DEFooZHBZooKlNfGZ}
    Si \( R\in \eK(X)\), avec \( R=P/Q\) et si \( \eL\) est une extension de \( \eK\) contenant l'élément \( \alpha\), alors nous définissons  
    \begin{equation}
        R(\alpha)=P(\alpha)Q(\alpha)^{-1}.
    \end{equation}
    Cela est une bonne définition au sens où elle ne dépend pas du choix du représentant \( (P,Q)\) pris dans la classe \( P/Q\).
\end{lemmaDef}

\begin{proof}
    Supposons \( R=P_1/Q_1=P_2/Q_2\). Par définition des classes (définition \ref{DEFooGJYXooOiJQvP}) nous avons
    \begin{equation}        \label{EQooKHVNooABuHaO}
        P_1Q_2=Q_1P_2.
    \end{equation}
    Vu que l'évaluation est un morphisme \( \eK[X]\to\eK\) \footnote{Lemme \ref{LEMooSFGGooGeVerf}. Certes ce lemme ne parle que d'anneaux, mais à y bien penser, dans le passage de \eqref{EQooKHVNooABuHaO} à \eqref{EQooKHVNooABuHaO}, nous ne considérons que les structures d'anneaux sur \( \eK[X]\) et \( \eK\).} nous pouvons évaluer l'équation \eqref{EQooKHVNooABuHaO} en \( \alpha\) :
    \begin{equation}        \label{EQooJAIGooRADgiD}
        P_1(\alpha)Q_2(\alpha)=Q_1(\alpha)P_2(\alpha).
    \end{equation}
    Cette dernière est une égalité dans le corps \( \eK\). Nous pouvons donc la multiplier par \( Q_2(\alpha)^{-1}P_2(\alpha)^{-1}\) (et utiliser toutes les hypothèse de commutativité des anneaux et corps) pour obtenir
    \begin{equation}
        P_1(\alpha)Q_1(\alpha)^{-1}=P_2(\alpha)Q_2(\alpha)^{-1},
    \end{equation}
    c'est à dire
    \begin{equation}
        (P_1/Q_1)(\alpha)=(P_2/Q_2)(\alpha).
    \end{equation}
\end{proof}

\begin{definition}  \label{DEFooVSKGooMyeGel}
    Soient un corps \( \eK\), une extension \( \eL\) de \( \eK\) et un élément \( \alpha\in\eL\). Nous définissons \( \eK(\alpha)_{\eL} \) comme étant l'intersection de tous les sous-corps de \( \eL\) contenant \( \eK\) et \( \alpha\).
\end{definition}

Notons que la définition de \( \eK(\alpha)_{\eL}\) donne bien un corps. Si \( a,b\in \eK(\alpha)_{\eL}\) alors il suffit de calculer \( ab\), \( a+b\) et \( a^{-1}\) dans n'importe quel sous-corps de \( \eL\) contenant \( \eK\) et \( \alpha\); nous avons une garantie que \( a\), \( b\), \( ab  \), \( a+b\) et \( a^{-1}\) sont dans tous les tels sous-corps.

\begin{example}      \label{EXooJRSUooYhAZkR}
    Est-ce que \( \eK(\alpha)_{\eL}\) dépend réellement de \( \eL\) ? Si \( \eL_2\) est une extension de \( \eL\) alors nous avons évidemment\footnote{Vérifiez-le tout de même.} \( \eK(\alpha)_{\eL}=\eK(\alpha)_{\eL_1}\).

    Mais il y a moyen de construire des exemples d'un peu de mauvaise foi. Nous avons 
    \begin{equation}
        \eQ(\sqrt{ 2 })_{\eR}=\{ a+b\sqrt{ 2 } \}_{a,b\in \eR}.
    \end{equation}
    À droite nous avons les lois de composition dans \( \eR\), avec en particulier la multiplication définie de telle sorte à avoir \( \sqrt{ 2 }\cdot\sqrt{ 2 }=2 \).

    Considérons le corps \( \eK\) donné en tant qu'ensemble par
    \begin{equation}
        \eK=\{ a+b\sqrt{ 2 } \}_{a,b\in \eR},
    \end{equation}
    et sur lequel nous mettons la multiplication usuelle sauf que \( \sqrt{ 2 }\cdot \sqrt{ 2 }=3\). Le fait est que si on y pense, l'objet \( \sqrt{ 2 }\) n'a aucun rapport avec \( \eQ\). En effet les objets de \( \eQ\) sont des classes d'équivalence de couples d'éléments de \( \eZ\), alors que l'élément \( \sqrt{ 2 }\) est une classe d'équivalence de suites de Cauchy dans \( \eQ\).

    Lorsque nous écrivons \( \eQ(\sqrt{ 2 })\), nous associons des objets de nature complètement différentes, et il n'y a aucune raison a priori de définir la multiplication entre eux d'une façon plutôt qu'une autre.

    Nous avons alors \( \eQ(\sqrt{ 2 })_{\eR}\neq \eQ(\sqrt{ 2 })_{\eK}\).
\end{example}

Ceci pour dire que oui, \( \eK(\alpha)_{\eL}\) dépend de \( \eL\). La proposition suivante semble indiquer que \( \eK(\alpha)\) est donné en termes de \( \eK(X)\), lequel est défini de façon très intrinsèque sans faire appel à un corps ambiant de \( \eK\).

Dans l'énoncé suivant, la notation \( R(\alpha)_{\eL}\) signifie que l'évaluation de \( R\) sur \( \alpha\) se fait en calculant dans le sur-corps \( \eL\) de \( \eK\).

\begin{proposition}[\cite{MonCerveau}]     \label{PROPooYSFNooFGbbCi}
    Soit une extension \( \eL\) du corps \( \eK\) et \( \alpha\in \eL\). Alors nous avons les isomorphismes de corps suivants :
    \begin{enumerate}
        \item
            \( \eK(\alpha)_{\eL}=\Frac\big( \eK[\alpha]_{\eL} \big)\),
        \item       \label{ITEMooATPTooVXKdlK}
            \( \eK(\alpha)_{\eL}=\{ R(\alpha)_{\eL}\tq R\in \eK(X) \}\).
    \end{enumerate}
\end{proposition}

\begin{proof}
    Le corps \( \eK(\alpha)\) est un sous-corps de \( \eL\) contenant \( \eK[\alpha]\) comme sous-anneau. La proposition \ref{PROPooGSHDooJOnDsp} nous dit alors que l'application suivante est un morphisme injectif de corps :
    \begin{equation}
        \begin{aligned}
            \epsilon\colon \Frac\big( \eK[\alpha] \big)&\to \eK(\alpha) \\
            P/Q&\mapsto PQ^{-1}. 
        \end{aligned}
    \end{equation}
    Pour rappel, la notation \( P/Q\) est bien une notation pour la classe d'équivalence du couple \( (P,Q)\) pour la relation définie en \ref{DEFooGJYXooOiJQvP}.

    Par ailleurs, la partie $\epsilon\Big( \Frac\big( \eK[\alpha] \big) \Big)$ est inclue à \( \eL\) et est un corps contenant \( \eK\) et \( \alpha\). Donc l fait partie des corps sur lesquels on prend l'intersection pour définir \( \eK(\alpha)\). Cela prouve que
    \begin{equation}
        \eK(\alpha)\subset  \epsilon\Big( \Frac\big( \eK[\alpha] \big) \Big).
    \end{equation}
    L'application \( \epsilon\) est donc surjective sur \( \eK(\alpha)\). Vu qu'elle était déjà injective, elle est bijective.

    Pour la seconde partie, veuillez lire la définition \ref{DEFooLBIWooCPCaSY} de l'évaluation d'une fraction rationnelle sur un élément de l'anneau. Si \( R=P/Q\in \eK(X)\) et si \( \alpha\in \eL\), nous avons
    \begin{equation}
        R(\alpha)=P(\alpha)Q(\alpha)^{-1}.
    \end{equation}
    Tout sous-corps de \( \eL\) contenant \( \eK\) et \( \alpha\) doit contenir en particulier \( \{ P(\alpha)\tq P\in \eK[X] \} \), les inverses \( \{ P(\alpha)^{-1}\tq P\in \eK[X],\,P(\alpha)\neq 0 \}\) et les produits d'iceux. Donc tout sous-corps de \( \eL\) contenant \( \eK\) et \( \alpha\) contient \( \{ R(\alpha)\tq R\in \eK(X) \}\).

    Nous avons donc
    \begin{equation}
        \{ R(\alpha)\tq R\in \eK(X) \}\subset \eK(\alpha).
    \end{equation}
    Mais vu que \( \eK(\alpha)\) est lui-même un sous-corps de \( \eL\) contenant \( \eK\) et \( \alpha\), il est contenu dans \( \{ R(\alpha)\tq R\in \eK(X) \}\). D'où l'égalité.
\end{proof}

Pourquoi cela ne contredit pas l'exemple \ref{EXooJRSUooYhAZkR} ? Lorsque nous écrivons
\begin{equation}
    \eK(\alpha)=\{ R(\alpha)\tq R\in \eK(X) \},
\end{equation}
certes \( \eK(X)\) est défini sans faire appel à un corps contenant \( \eK\). Mais l'évaluation \( R(\alpha)\), oui. Pour calculer \( R(\alpha)\), il faut écrire \( R=P/Q\) et calculer \( P(\alpha)Q(\alpha)^{-1}\). Tous les calculs de cette dernière expression doivent se faire dans un sur-corps de \( \eK\). Il suffit que le sur-corps en question soit un monceau de mauvaise foi comme celui de l'exemple \ref{EXooJRSUooYhAZkR}, et en réalité \( \eK(\alpha)\) peut ne pas être ce que l'on croit.

Le corollaire suivant montre que les chose s'arrangent. 

\begin{corollary}
    Soient un corps \( \eK\), une extension \( \eL_1\) de \( \eK\), un élément \( \alpha\in \eL_1\) et une extension \( \eL_2\) de \( \eL_1\). Alors
    \begin{equation}
        \eK(\alpha)_{\eL_1}=\eK(\alpha)_{\eL_2}.
    \end{equation}
\end{corollary}

\begin{proof}
    La proposition \ref{PROPooYSFNooFGbbCi} nous dit que
    \begin{subequations}
        \begin{align}
            \eK(\alpha)_{\eL_1}=\{ R(\alpha)_{\eL_1}\tq R\in\eK(X) \}\\
            \eK(\alpha)_{\eL_2}=\{ R(\alpha)_{\eL_2}\tq R\in\eK(X) \}.
        \end{align}
    \end{subequations}
    Mais lorsque \( R\in \eK(X)\), le calcul de \( R(\alpha)\) est exactement le même dans \( \eL_1\) et dans \( \eL_2\) parce que \( \eL_2\) est un sur-corps de \( \eL_1\) et que les calculs effectifs de \( R(\alpha)=P(\alpha)Q(\alpha)^{-1}\) ne font intervenir que des quantités de \( \eK\) et des puissances de \( \alpha\).
\end{proof}

Ce que ce corollaire nous dit est que si le contexte fixe une extension de \( \eK\), nous pouvons faire tous les calculs dans cette extension, même si il y a des piles d'extensions à côté.

Typiquement, à chaque fois que nous considérons des sous-corps de \( \eC\), les extensions se feront dans \( \eC\) : pour tout \( \alpha\in \eC\), les corps \( \eQ(\alpha)\), \( \eR(\alpha)\) se calculent dans \( \eC\).

%--------------------------------------------------------------------------------------------------------------------------- 
\subsection{Éléments algébriques et transcendants}
%---------------------------------------------------------------------------------------------------------------------------

%///////////////////////////////////////////////////////////////////////////////////////////////////////////////////////////
\subsubsection{Élements algébriques et transcendants}
%///////////////////////////////////////////////////////////////////////////////////////////////////////////////////////////

\begin{lemmaDef}[\cite{ooTGTKooFenWAc}] \label{LEMooLVPLooEkWYDN}
    Soit une extension \( \eL\) de \( \eK\) et \( \alpha\in \eL\). Nous considérons l'application
    \begin{equation}
        \begin{aligned}
            \varphi\colon \eK[X]&\to \eL \\
            P&\mapsto P(\alpha). 
        \end{aligned}
    \end{equation}
    Alors
    \begin{enumerate}
        \item
            L'application \( \varphi\) est un morphisme d'anneaux.
        \item
            L'application \( \varphi\) est un morphisme de \( \eK\)-espace vectoriel.
    \end{enumerate}
    Si \( \varphi\) est injective, nous disons que \( \alpha\) est \defe{transcendant}{transcendant}. Sinon, nous disons qu'il est algébrique.
\end{lemmaDef}

\begin{example}
    L'injectivité de \( \varphi\) n'est pas automatique. Prenons par exemple \( \eL=\eQ[\sqrt{ 2 }]\) dans \( \eR\). Les polynôme dans \( \eQ[X]\) ont des degrés arbitrairement élevés en \( X\), tandis que les éléments de \( \eL\) n'ont pas de degré très élevés en \( \sqrt{ 2 }\) parce que \( \sqrt{ 2 }\sqrt{ 2 }=2\). L'ensemble \( \eQ[\sqrt{ 2 }]\) ne contient donc que des éléments de la forme \( a+b\sqrt{ 2 }\) avec \( a,b\in \eQ\).

    Si par contre \( x_0\in \eR\) n'est racine d'aucun polynôme (cela existe parce que \( \eR\) n'est pas dénombrable), alors \( \eQ[x_0]\) contient tous les \( \sum_{k=0}^Na_kx_0^k\) avec \( N\) arbitrairement grand. Et tous ces nombres sont différents.
\end{example}

\begin{definition}
    Soit \( \eK\), un corps et \( \eL\), une extension de \( \eK\). Un élément \( a\in \eL\) est \defe{algébrique}{algébrique!nombre} sur \( \eK\) s'il existe un polynôme \( P\in \eK[X]\) tel que \( P(a)=0\).
\end{definition}

\begin{definition}
    Une \defe{clôture algébrique}{clôture algébrique} du corps \( \eK\) est une extension algébriquement close de \( \eK\) dont tous les éléments sont algébriques sur \( \eK\).
\end{definition}

\begin{remark}
    L'ensemble \( \eC\) n'est pas une clôture algébrique de \( \eQ\) parce qu'il existe des éléments de \( \eC\) qui ne sont pas des racines de polynômes à coefficients rationnels.
\end{remark}
L'existence d'une clôture algébrique pour tout corps est le théorème de Steinitz.
%TODO : à faire, le théorème de Steinitz.
% Lorsque ce sera faire, le référentier à la position EYRooJkxiFf

C'est cela que la proposition suivante tente de systématiser.
\begin{proposition}     \label{PROPooSYQWooFbfQtm}
    Soit un corps \( \eK\), une extension \( \eL\) et un élément \( \alpha\in \eL\). Nous considérons l'application
    \begin{equation}
        \begin{aligned}
            \varphi\colon \eK[X]&\to \eL \\
            P&\mapsto P(\alpha). 
        \end{aligned}
    \end{equation}
    \begin{enumerate}
        \item       \label{ITEMooUZDQooOasiRQ}
            Si \( \alpha\) est transcendant, alors \( \eK[\alpha]=\eK[X]\) (isomorphisme d'anneaux).
        \item
            Si \( \alpha\) est transcendant, alors \( \eK(\alpha)_{\eL}=\eK(X)\) (isomorphisme de corps),
        \item
            Si \( \alpha\) est algébrique, alors \( \ker(\varphi)\) est un idéal possédant un unique générateur unitaire, lequel est le polynôme minimal\footnote{Définition \ref{DefCVMooFGSAgL}.} de \( \alpha\) sur \( \eK\).
    \end{enumerate}
\end{proposition}

\begin{proof}
    Point par point.
    \begin{enumerate}
        \item
            Nous savons que \( \eK[\alpha]=\{ Q(\alpha)\tq Q\in \eK[X] \}\) (c'est la proposition \ref{PROPooPMNSooOkHOxJ}). Donc \( \varphi\) est surjective sur \( \eK[\alpha]\), et est donc bijective. Elle est un isomorphisme\footnote{Les amateurs d'écriture inclusive ne seront, je l'espère, pas choqué par «\emph{elle} est \emph{un} isomorphisme»; c'est une tournure que je propose ici sur le modèle de l'immonde «\emph{elle} est \emph{un} ministre» ou, à peine moins grave, «\emph{il} est \emph{une} sommité».} parce que le lemme \ref{LEMooLVPLooEkWYDN} dit déjà que c'est un morphisme.
        \item
            Nous supposons encore que \( \alpha\) est transcendant et nous considérons l'application
            \begin{equation}
                \begin{aligned}
                    \psi\colon \eK(X)&\to \eK(\alpha) \\
                    P&\mapsto R(\alpha). 
                \end{aligned}
            \end{equation}
            Note : cette application n'est pas \( \varphi\). En effet \( \varphi\) n'est définie que sur \( \eK[X]\); le corps des fractions \( \eK(X)\) est nettement plus grand (classes d'équivalence de couples).

            Le fait que cette application soit surjective est la proposition \ref{PROPooYSFNooFGbbCi}\ref{ITEMooATPTooVXKdlK}. Pour l'injectivité nous supposons que \( \psi(R)=0\), c'est à dire que \( R(\alpha)=0\). Nous considérons un représentant \( (P,Q)\) de \( R\); c'est à dire \( R=P/Q\). L'égalité \( R(\alpha)=0\) signifie \( P(\alpha)Q(\alpha)^{-1}=0\) (égalité dans \( \eL\)). Vu que \( \eL\) est un corps, c'est un anneau intègre et nous avons la règle du produit nul; soit \( P(\alpha)=0\), soit \( Q(\alpha)^{-1}=0\). La seconde possibilité est impossible parce que zéro n'est pas inversible. Donc \( P(\alpha)=0\). Donc \( \varphi(P)=0\) et \( \varphi\) étant injective, \( P=0\).

            Lorsque \( P=0\), la classe \( P/Q\) est nulle dans \( \eK(X)= \Frac\big(\eK[X]\big)\).

        \item

            C'est le lemme-définition \ref{DefCVMooFGSAgL}.
    \end{enumerate}
\end{proof}

\begin{proposition}\label{PropXULooPCusvE}
    Soit un corps \( \eK\) et une extension \( \eL\). Soit \( P\in \eK[X]\) et  \( a\in \eL\), une racine de \( P\). Alors le polynôme minimal d'une racine divise\footnote{Définition \ref{DefMPZooMmMymG}.} tout polynôme annulateur.

    Autrement dit, l'idéal engendré par le polynôme minimal est l'idéal des polynômes annulateurs.
\end{proposition}

\begin{proof}
    Nous considérons l'idéal
    \begin{equation}
        I=\{ Q\in \eK[X]\tq Q(a)=0 \}.
    \end{equation}
    Le fait que cela soit un idéal est simplement dû à la définition du produit : \( (PQ)(a)=P(a)Q(a)\). Par le théorème \ref{ThoCCHkoU}, le polynôme minimal \( \mu_a\) de \( a\) est dans \( I\) et qui plus est le génère : \( I=(\mu_a)\). Par conséquent tout polynôme annulateur de \( a\) est divisé par \( \mu_a\).
\end{proof}

%///////////////////////////////////////////////////////////////////////////////////////////////////////////////////////////
\subsubsection{Extension algébrique, degré}
%///////////////////////////////////////////////////////////////////////////////////////////////////////////////////////////

\begin{lemma}       \label{LemooOLIIooXzdppM}
    Si \( (\eL,i)\) est une extension de \( \eK\), alors \( \eL\) est un espace vectoriel sur \( \eK\).
\end{lemma}

\begin{proof}
    Il faut définir le produit d'un élément de \( \eL\) par un élément de \( \eK\); si \( \lambda\in \eK\) et \( x\in \eL\) nous la définissons par
    \begin{equation}
        \lambda\cdot x=i(\lambda)x
    \end{equation}
    où la multiplication du membre de droite est celle du corps \( \eL\). 
\end{proof}

\begin{definition}      \label{DefUYiyieu}
    Le \defe{degré}{degré!extension de corps} de \( \eL\) est la dimension de cet espace vectoriel. Il est noté \( [\eL:\eK]\)\nomenclature[A]{$ [\eL:\eK]$}{degré d'une extension de corps}; notons qu'il peut être infini.
\end{definition}

\begin{example}
    L'ensemble \( \eC\) est une extension de \( \eR\) et son degré est \( [\eC:\eR]=2\).
\end{example}

\begin{proposition}[Composition des degrés\cite{ooGIIFooVMVloY}]        \label{PROPooEGSJooBSocTf}
    Si \( \eL_2\) est une extension de \( \eL_1\) qui est elle-même une extension de \( \eK\), alors \( \eL_2\) est une extension de \( \eK\) et on a :
    \begin{equation}        \label{EQooOLLQooFdYtnh}
        [\eL_2:\eK]=[\eL_2:\eL_1][\eL_1:\eK].
    \end{equation}
    Dans ce cas, si \( \{ v_i \}_{i\in I}\) est une \( \eK\)-base de \( \eL_1\) et si \( \{ w_{\alpha} \}_{\alpha\in A}\) est une \( \eL_1\)-base de \( \eL_2\) alors \( \{ v_iw_{\alpha} \}_{\substack{i\in I\\\alpha\in A}}\) est une \( \eK\)-base de \( \eL_2\).
\end{proposition}

Notons que la formule \eqref{EQooOLLQooFdYtnh} n'est pas très instructive dans le cas des extensions non finies. La seconde partie, sur les bases, est en réalité nettement plus intéressante.

\begin{proof}
    Soit \( a\in \eL_2\). Vu que les \( w_{\alpha}\) forment une \( \eL_2\)-base nous avons une décomposition
    \begin{equation}
        a=\sum_{\alpha}a_{\alpha}w_{\alpha}
    \end{equation}
    pour des éléments \( a_{\alpha}\in \eL_1\). Mais les \( v_i\) forment une \( \eK\)-base de \( \eL_1\), donc chacun des \( a_{\alpha}\) peut être décomposé comme \( a_{\alpha}=\sum_ia_{\alpha i}v_iw_{\alpha}\). Donc :
    \begin{equation}
        a=\sum_{\alpha i}a_{\alpha i}v_iw_{\alpha},
    \end{equation}
    qui donne une décomposition de \( a\) en éléments de \( \{ v_iw_{\alpha} \}\) à coefficients dans \( \eK\). La partie proposée est donc génératrice.

    Pour prouver qu'elle est également libre, nous supposons avoir des éléments \( a_{\alpha i}\in \eK\) tels que
    \begin{equation}
        \sum_{\alpha i}a_{\alpha i}v_iw_{\alpha}=0.
    \end{equation}
    En récrivant sous la forme
    \begin{equation}
        \sum_{\alpha}\Big( \sum_ia_{\alpha i}v_i \Big)w_{\alpha}=0,
    \end{equation}
    nous reconnaissons une combinaison linéaire nulle des \( w_{\alpha}\) à coefficients dans \( \eL_1\). Les coefficients sont donc nuls : \( \sum_i a_{\alpha i}v_i=0\). Cela est une combinaison linéaire nulle des \( v_i\) à coefficients dans \( \eK\). Vu que les \( v_i\) forment une base, les coefficients sont nuls : \( a_{\alpha i}=0\).
\end{proof}

\begin{proposition}
    Toute extension finie est algébrique.
\end{proposition}

\begin{proof}
    Soient un corps \( \eK\), une extension \( \eL\) de degré \( n\) de \( \eK\) et \( a\in \eL\). Nous devons montrer qu'il existe un polynôme annulateur de \( a\) à coefficients dans \( \eK\).

    Soit la partie \( S=\{1,a,a^2,\ldots, a^n\}\) de \( \eL\). Si cette partie contient des éléments non distincts, alors c'est plié. En effet, si \( a^k=a^l\), alors le polynôme \( X^{k-l}\) est un polynôme annulateur de \( a\).

    Nous supposons donc que \( S\) contienne exactement \( n+1\) éléments distincts. Le lemme \ref{LemytHnlD} nous assure que \( S\) est une partie liée : il existe des éléments \( k_i\in \eK\) tels que \( \sum_{i=0}^nk_ia^i=0\).

    Donc le polynôme \( \sum_ia_iX^i\) est un polynôme annulateur de \( a\).
\end{proof}

\begin{proposition}[Propriétés d'extensions algébriques\cite{MonCerveau}]   \label{PropURZooVtwNXE}
    Soit \( \eK\) un corps commutatif\footnote{Juste en passant nous rappelons que tous les corps considérés ici sont commutatifs} et \( a\) un élément algébrique sur \( \eK\), de polynôme minimal \( \mu_a\) de degré \( n\). Alors
    \begin{enumerate}
        \item\label{ItemJCMooDgEHajmi}
            En considérant l'application d'évaluation
            \begin{equation}
                \begin{aligned}
                    \varphi_a\colon \eK[X]&\to \eL \\
                    Q&\mapsto Q(a), 
                \end{aligned}
            \end{equation}
            nous avons \( \eK[a]=\Image(\varphi_a)\).
        \item\label{ItemJCMooDgEHajiv}
            Une base de \( \eK[a]\) comme espace vectoriel sur \( \eK\) est donnée par \( \{ 1,a,a^2,\ldots, a^{n-1} \}\).
        \item\label{ItemJCMooDgEHajiii}
            Le degré de l'extension \( \eK[a]\) est égal au degré du polynôme minimal :
            \begin{equation}
                \big[ \eK[a]:\eK \big]=n.
            \end{equation}
         \item
            L'anneau \( \eK[a]\) est l'ensemble des polynômes en \( a\) de degré \( n-1\) à coefficient dans \( \eK\).
        \item\label{ItemJCMooDgEHaji}
            \( \eK(a)=\eK[a]\).
        \item   \label{ItemJCMooDgEHajii}
            \( \eK[a]\simeq\eK[X]/(\mu_a)\) (isomorphisme d'anneau).
    \end{enumerate}
\end{proposition}
\index{extension!de corps!algébrique}
L'intérêt de \ref{ItemJCMooDgEHajii} est qu'il permet de caractériser \( \eK[a]\) sans avoir recours à un sur-corps de \( \eK\). Le point \ref{ItemJCMooDgEHajiii} indique que le degré d'une extension algébrique est égal au degré du polynôme minimal.

\begin{proof}
    \begin{enumerate}
        \item
            Nous avons \( \eK[a]\subset \Image(\varphi_a)\) parce que \( \Image(\varphi_a)\) est lui-même un un sous-anneau de \( \eL\) contenant \( \eK\) et \( a\). Pour rappel, \( \eK[a]\) est l'intersection de tous les tels sous-anneaux.
            
            L'inclusion inverse est le fait que si \( Q\in \eK[X]\) alors \( Q(a)\in \eK[a]\) parce que \( \eK[a]\) est un anneau et contient donc tous les \( a^n\).
        \item
            La partie \( \{ 1,a,a^2,\ldots, a^{n-1} \}\) est libre parce qu'une combinaison linéaire de ces éléments est un polynôme de degré \( n-1\) en \( a\). Un tel polynôme ne peut pas être nul parce que nous avons mis comme hypothèse que le polynôme minimal de \( a\) est \( n\).

            Rappelons qu'en vertu de la définition \ref{DefCVMooFGSAgL}, le polynôme minimal \( \mu_a\) est unitaire; donc le polynôme \( \mu_a(X)-X^n\) est un polynôme de degré \( n-1\). Par conséquent en posant \( S(X)=X^n-\mu_a(X)\), le polynôme \( S\) est de degré \( n-1\) et vérifie \( a^n=S(a)\).  

            En vertu du point \ref{ItemJCMooDgEHajmi}, un élément de \( \eK[a]\) s'écrit \( Q(a)\) pour un certain \( Q\in\eK[X]\). Supposons que \( Q\) soit de degré \( p>n-1\); alors nous le décomposons en une partie contenant les termes de degré jusqu'à \( n-1\) et une partie contenant les autres :
            \begin{equation}
                Q(X)=Q_1(X)+X^nQ_2(X)
            \end{equation}
            où \( Q_1\) est de degré \( n-1\) et \( Q_2\) de degré \( p-n\). Nous évaluons cette égalité en \( a\) :
            \begin{equation}
                Q(a)=Q_1(a)+S(a)Q_2(a).
            \end{equation}
            Donc \( Q(a)\) est l'image de \( a\) par le polynôme \( Q_1+SQ_2\) qui est de degré \( p-1\). Par récurrence, \( Q(a)\) est l'image de \( a\) par un polynôme de degré \( n-1\).

            Notons que l'idée est très simple : il s'agit de remplacer récursivement tous les \( a^n\) par \( S(a)\).
    \item
        Conséquence immédiate de \ref{ItemJCMooDgEHajiv}.
    \item
        Conséquence immédiate de \ref{ItemJCMooDgEHajiv}.
    \item
        Un élément général non nul de \( \eK[a]\) est de la forme \( Q(a)\) avec \( Q\in\eK[X]\); il s'agit de lui trouver un inverse. Pour cela nous remarquons que les polynômes \( \mu_a(X)\) et \( Q(x)\) sont premiers entre eux, sinon \( \mu_a\) ne serait pas un polynôme minimal (voir la proposition \ref{PropRARooKavaIT}). Donc le théorème de Bézout \ref{ThoBezoutOuGmLB} affirme l'existence d'éléments \( U,V\in \eK[X]\) tels que
        \begin{equation}
            U\mu_a+VQ=1
        \end{equation}
        dans \( \eK[X]\). Nous évaluons cette égalité en \( a\) en tenant compte de \( \mu_a(a)=0\) dans \( \eK[a]\) :
        \begin{equation}
            U(a)\mu_a(a)+V(a)Q(a)=1
        \end{equation}
        dans \( \eK[a]\). Par conséquent \( V(a)Q(a)=1\), ce qui signifie que \( V(a)\) est l'inverse de \( Q(a)\).
        \item
            Nous considérons l'application
            \begin{equation}
                \begin{aligned}
                    \psi\colon \eK[X]/(\mu_a)&\to \eK[a] \\
                    \bar R&\mapsto R(a) 
                \end{aligned}
            \end{equation}
            et nous montrons qu'elle convient. Pour cela, nous nous souvenons que la proposition \ref{PropXULooPCusvE} nous enseigne que \( (\mu_a)\), l'idéal engendré par \( \mu_a\), est égal à l'idéal des polynômes annulateurs de \( a\) dans \( \eK[X]\). Le polynôme \( \mu_a\) divise tous les éléments de cet idéal; voir aussi la définition \ref{DefSKTooOTauAR} de l'idéal \( (\mu_a)\). Cela étant mis au point, nous passons à la preuve.
            \begin{subproof}
            \item[\( \psi\) est bien définie]
                
                Si \( \bar R=\bar S\) alors \( R=S+Q\) avec \( Q\in(\mu_a)\), et par conséquent \( R(a)=S(a)+Q(a)\) avec \( Q(a)=0\).

            \item[Surjective]

                Nous savons que \( \eK[a]=\Image(\varphi_a)\). Si \( x\in \eK[a]\) alors il existe \( Q\in \eK[X]\) tel que \( x=Q(a)\). Dans ce cas nous avons aussi \( x=\psi(\bar Q)\).

            \item[Injective]

                Si \( \psi(\bar R)=0\) alors \( R(a)=0\), mais comme mentionné plus haut, \( \mu_a\) engendre l'idéal est polynômes annulateurs de \( a\). Donc \( R\in (\mu_a)\) et nous avons \( \bar R=0\) dans \( \eK[X]/(\mu_a)\).

            \end{subproof}
                
    \end{enumerate}
\end{proof}

\begin{example}
    Un fait connu est que \( \frac{1}{ \sqrt{2} }=\frac{ \sqrt{2} }{ 2 }\). Donc l'inverse de \( \sqrt{2}\) s'exprime bien comme un polynôme en \( \sqrt{2}\) à coefficients dans \( \eQ\), ce qui confirme le point \ref{ItemJCMooDgEHaji} de la proposition \ref{PropURZooVtwNXE}. Du point de vue de Bézout, \( \mu_{\sqrt{2}}(X)=X^2-2\), et nous cherchons des polynômes \( U\) et \( V\) tels que
    \begin{equation}
        U(X^2-2)+VX=1.
    \end{equation}
    cette égalité est réalisée par \( U=-\frac{ 1 }{2}\) et \( V=\frac{ 1 }{2}X\). Et effectivement \( V(\sqrt{2})\) est bien l'inverse de \( \sqrt{2}\) :
    \begin{equation}
        V(\sqrt{(2)})=\frac{ 1 }{2}\sqrt{2}.
    \end{equation}
\end{example}

\begin{lemma}
    Un nombre complexe algébrique dont tous les conjugués sont de module \( 1\) est une racine de l'unité.
\end{lemma}

\begin{proposition}[\cite{ooTGTKooFenWAc}]
    Soient un corps \( \eK\), un extension \( \eL\) de \( \eK\) et un élément \( \alpha\) de \( \eL\). Il y a équivalence entre les trois points suivants :
    \begin{enumerate}
        \item   \label{ITEMooYTEBooUuEfBz}
            \( \alpha\) est algébrique sur \( \eK\),
        \item   \label{ITEMooWMQTooLnepQl}
            \( \eK[\alpha]=\eK(\alpha)\),
        \item   \label{ITEMooAQIUooMVZojp}
            \( \eK[\alpha]\) est un \( \eK\)-espace vectoriel de dimension finie.
    \end{enumerate}
    Si ces affirmation sont vraies, alors \( [\eK(\alpha):\eK]\) est le degré du polynôme minimal de \( \alpha\) sur \( \eK\).
\end{proposition}

\begin{proof}
    Démonstration décomposée en plusieurs implications.
    \begin{subproof}
        \item[\ref{ITEMooYTEBooUuEfBz} implique \ref{ITEMooWMQTooLnepQl}]
        
            Soit \( \alpha\) algébrique sur \( \eK\). Nous considérons le polynôme minimal de \( \alpha\) sur \( \eK\) (définition \ref{DefCVMooFGSAgL}). Nous savons par le lemme \ref{LEMooHKTMooKEoOuK} (qui fonctionne parce que \( \alpha\) est algébrique) que \( \eK[\alpha]=\eK[X]/(\mu)\) en tant qu'anneaux.

            Mais \( \eK[X]\) est un anneau principal et \( \mu\) en est un élément irréductible. Donc la proposition \ref{PropomqcGe} dit que \( (\mu)\) est un idéal maximum; la proposition \ref{PropoTMMXCx} avance encore un peu en disant que \( \eK[X]/(\mu)\) est un corps.

            Donc \( \eK[X]/(\mu)\) est un corps isomorphe à \( \eK[\alpha]\) en tant qu'anneaux. En conséquence de quoi \( \eK[\alpha]\) est un corps.

            Le corps \( \eK[\alpha]\) est un sous-corps de \( \eL\) contenant \( \eK\) et \( \alpha\); par définition nous avons donc \( \eK(\alpha)\subset \eK[\alpha]\). Mais d'autre part, \( \eK[\alpha]\) est contenu dans tout sous-corps de \( \eL\) contenant \( \eK\) et \( \alpha\), donc il est inclus à l'intersection de tout ces corps, donc \( \eK[\alpha]\subset \eK(\alpha)\).

            Les deux inclusions sont prouvées.

        \item[\ref{ITEMooWMQTooLnepQl} implique \ref{ITEMooYTEBooUuEfBz}]

            Nous montrons que non-\ref{ITEMooYTEBooUuEfBz} implique non-\ref{ITEMooWMQTooLnepQl}. Nous disons donc que \( \alpha\) est transcendant sur \( \eK\); cela implique par la proposition \ref{PROPooSYQWooFbfQtm}\ref{ITEMooUZDQooOasiRQ} que \( \eK[\alpha]=\eK[X]\) en tant qu'anneaux. Donc \( \eK[\alpha]\) n'est pas un corps parce que \( \eK[X]\) ne l'est pas.

            N'étant pas un corps, \( \eK[\alpha]\) ne peut pas être égal à \( \eK(\alpha)\) qui, lui, est un corps.

        \item[\ref{ITEMooYTEBooUuEfBz} implique \ref{ITEMooAQIUooMVZojp}]

            L'élément \( \alpha\) est maintenant algébrique et nous considérons son polynôme minimal \( \mu\). Nous savons par le lemme \ref{LEMooHKTMooKEoOuK} que \( \eK[\alpha]=\eK[X]/(\mu)\) en tant qu'espaces vectoriels. Or \( \eK[X]/(\mu)\) est de dimension finie \( \deg(\mu)\). Donc \( \eK[\alpha]\) est également de dimension finie.

        \item[\ref{ITEMooAQIUooMVZojp} implique \ref{ITEMooYTEBooUuEfBz}]

            Nous démontrons la contraposée. En supposant que \( \alpha\) est transcendant nous avons \( \eK[\alpha]=\eK[X]\) par la proposition \ref{PROPooSYQWooFbfQtm}. Or \( \eK[X]\) n'est pas de dimension finie sur \( \eK\), donc \( \eK[\alpha]\) non plus.

    \end{subproof}
\end{proof}

\begin{proposition}[\cite{ROZaSWZ}]     \label{PropGWazMpY}
    Si \( \eL\) est une extension du corps \( \eK\) et si \( \eM\) est une extension de \( \eL\), alors les degrés se multiplient :
    \begin{equation}
        [\eM:\eL][\eL:\eK]=[\eM;\eK].
    \end{equation}
\end{proposition}

\begin{proof}
    Soit \( \{ l_i \}\) une base de \( \eL\) sur \( \eK\) et \( \{ m_j \}\) une base de \( \eM\) sur \( \eL\). Un élément de \( \eM\) se note
    \begin{equation}
        \sum_{j}b_jm_j
    \end{equation}
    pour des \( b_j\in \eL\). Chacun de ces \( b_j\) s'écrit comme une combinaison des \( l_i\) :
    \begin{equation}
        \sum_{j}b_jm_j=\sum_{ij}(a_{ji}l_i)m_j,
    \end{equation}
    donc nous voyons que les éléments \( l_im_j\) de \( \eM\) sont une base de \( \eM\) sur \( \eK\).
\end{proof}

\begin{lemma}
    Soit \( \eL\) un corps commutatif et \( (\eK_i)_{i\in I}\) une famille de sous-corps de \( \eL\). Alors \( \bigcup_{i\in I}\eK_i\) est un sous-corps de \( \eL\).
\end{lemma}

\begin{definition}  \label{DefZCYIbve}
    Soit \( \eL\) une extension de \( \eK\) et \( A\subset \eL\). 
    \begin{enumerate}
        \item
            
    Nous notons \( \eK(A)\)\nomenclature[A]{$\eK(A)$}{corps contenant $\eK$ et $A$} le plus petit sous corps de \( \eL\) contenant \( \eK\) et \( A\). C'est l'intersection de tous les sous-corps de \( \eL\) contenant \( A\).
\item
    Nous notons \( \eK[A]\)\nomenclature[A]{$\eK[A]$}{anneau contenant $ \eK$ et $ A$} le plus petit sous anneau de \( \eL\) contenant \( \eK\) et \( A\). C'est l'intersection de tous les sous-anneaux de \( \eL\) contenant \( A\).
    \end{enumerate}
    Nous disons que l'extension \( \eL\) de \( \eK\) est \defe{monogène}{monogène!extension de corps} ou \defe{\wikipedia{fr}{Extension_simple}{simple}}{extension!de corps!simple}\index{simple!extension de corps} s'il existe \( \theta\in\eL\) tel que \( \eL=\eK(\theta)\). Un tel élément \( \theta\) est dit \defe{élément primitif}{primitif!élément d'une extension de corps} de \( \eL\). Il n'est pas nécessairement unique.
\end{definition}

\begin{remark}
    Les ensembles \( \eK(A)\) et \( \eK[A]\) sont aussi appelés corps et anneaux \defe{engendré}{engendré!corps et anneau} par \( A\). Cependant il faut bien remarquer que ce sont les parties de \( \eL\) engendrées par \( A\). Il n'est pas question a priori de parler de corps engendré par \( A\) sans dire dans quel corps plus grand nous nous plaçons.
\end{remark}

\begin{example}
    Nous savons que \( \eR\) est une extension de \( \eQ\). Si \( a\in \eR\) alors \( \eQ(a)\) est le plus petit corps contenant \( \eQ\) et \( a\).
\end{example}

\begin{example}
    Nous avons déjà vu à l'occasion de la définition \ref{DefRGOooGIVzkx} que \( A[X]\) est l'anneau de tous les polynômes de degré fini en \( X\). Cela rentre dans le cadre de la définition \ref{DefZCYIbve} parce un anneau contenant \( X\) doit contenir tous les \( X^n\).

    Notons que même si \( \eK\) est un corps, \( \eK[X]\) reste un anneau parce que \( A/X\) n'est pas dedans. Par contre, \( \eK(X)\) est un corps parce qu'il contient également les fractions rationnelles.
\end{example}

\begin{example} \label{ExLQhLhJ}
    Si nous prenons \( \eF_5\) et que nous l'étendons par \( i\), nous obtenons le corps \( \eK=\eF_5(i)\). Nous savons que tous les éléments \( a\in \eF_5\) sont racines de \( X^5-X\). Mais étant donné que \( i^5=i\), nous avons aussi \( x^5=x\) pour tout \( x\in \eF_5(i)\). Pour le prouver, utiliser le morphisme de Frobenius. Le polynôme \( X^5-X\) est donc le polynôme nul dans \( \eK\).

    Ceci est un cas très particulier parce que nous avons étendus \( \eF_p\) par un élément \( \alpha\) tel que \( \alpha^p=\alpha\). En général sur \( \eF_p(\alpha)\), le polynôme \( X^p-X\) n'est pas identiquement nul, et possède donc au maximum \( p\) racines. Pour \( x\in \eF_p(\alpha)\), nous avons \( x^p=x\) si et seulement si \( x\in \eF_p\).
\end{example}

\begin{lemma}
    Soit \( P\in\eK[X]\) un polynôme unitaire irréductible de degré \( n\). Il existe une extension \( \eL\) de \( \eK\) et \( a\in \eL\) telle que \( \eL=\eK(a)\) et \( P\) est le polynôme minimal de \( a\) dans \( \eL\).
\end{lemma}

\begin{proof}
    Nous prenons \( \eL=\eK[X]/(P)\) où \( (P)\) est l'idéal dans \( \eK[X]\) généré par \( P\). Cela est un corps par le corollaire \ref{CorsLGiEN}. Nous identifions \( \eK\) avec \( \phi(\eK)\) où
    \begin{equation}
        \phi\colon \eK[X]\to \eL 
    \end{equation}
    est la projection canonique. Nous considérons également \( a=\phi(X)\).

    Nous avons alors \( P(a)=0\) dans \( \eL\). En effet \( P(a)=P\big( \phi(X) \big)\) est à voir comme l'application du polynôme \( P\) au polynôme \( X\), le résultat étant encore un élément de \( \eL\). En l'occurrence le résultat est \( P\) qui vaut \( 0\) dans \( \eL\).

    Le polynôme \( P\) étant unitaire et irréductible, il est minimum dans \( \eL\).

    Nous devons encore montrer que \( \eL=\eK(a)\). Le fait que \( \eK(a)\subset \eL\) est une tautologie parce qu'on calcule \( \eK(a)\) dans \( \eL\). Pour l'inclusion inverse soit \( Q(X)=\sum_iQ_iX^i\) dans \( \eK[X]\). Dans \( \eL\) nous avons évidemment \( Q=\sum_iQ_ia^i\).
\end{proof}

\begin{proposition}[\cite{ooLIOMooBuCPUS}] \label{PropyMTEbH}
    Soit \( \eK\), un corps et \( P\in \eK[X]\) un polynôme. Soient \( a\) et \( b\), deux racines de \( P\) dans (éventuellement) une extension \( \eL\) de \( \eK\). Si \( \mu_a\) et \( \mu_b\) sont les polynômes minimaux de \( a\) et \( b\) (dans \( \eK[X]\)) et si \( \mu_a\neq \mu_b\), alors \( \mu_a\mu_b\) divise \( P\) dans \( \eK[X]\).
\end{proposition}

\begin{proof}
    Nous considérons les idéaux
    \begin{subequations}
        \begin{align}
            I_a=\{ Q\in \eK[X]\tq Q(a)=0 \};\\
            I_b=\{ Q\in \eK[X]\tq Q(b)=0 \}.
        \end{align}
    \end{subequations}
    Même si \( Q(a)\) est calculé dans \( \eL\), ce sont des idéaux de \( \eK[X]\). Le polynôme \( \mu_a\) est par définition le générateur unitaire de \( I_a\), et vu que \( a\) est une racine de \( P\), nous avons \( P\in I_a\) et il existe un polynôme \( Q\in \eK[X]\) tel que 
    \begin{equation}    \label{EqvTPoSq}
        P=\mu_aQ.
    \end{equation}

    Montrons que \( \mu_a(b)\neq 0\). Pour cela, nous  supposons que \( \mu_a(b)=0\), c'est à dire que \( \mu_a\in I_b\). Il existe alors \( R\in \eK[X]\) tel que \( \mu_a=\mu_bR\). Mais par la proposition \ref{PropRARooKavaIT}, le polynôme \( \mu_a\) est irréductible, donc soit \( \mu_b\) soit \( R\) est inversible. Vu que les inversibles sont les éléments de \( \eK\) (polynômes de degré zéro), \( \mu_b\) n'est pas inversible (sinon il serait constant et ne pourait pas être annulateur de \( b\)). Donc \( R\) est inversible. Disons \( R=k\).

    Donc \( \mu_a=k\mu_b\). Mais vu que \( \mu_a\) et \( \mu_b\) sont unitaires, nous avons obligatoirement \( k=1\). Cela donnerait \( \mu_a=\mu_b\), ce qui est contraire aux hypothèses. Nous en déduisons que \( \mu_a(b)\neq 0\).

    Étant donné que \( \mu_a(b)\neq 0\), l'évaluation de \eqref{EqvTPoSq} en \( b\) montre que \( Q(b)=0\), de telle sorte que \( Q\in I_b\) et il existe une polynôme \( S\) tel que \( Q=\mu_bS\), c'est à dire tel que \( P=\mu_a\mu_bS\), ce qui signifie que \( \mu_a\mu_b\) divise \( P\).
\end{proof}

\begin{example}
    Soit \( P=(X^2+1)(X^2+2)\) dans \( \eR[X]\). Dans \( \eC\) nous avons les racines \( a=i\) et \( b=\sqrt{2}i\) dont les polynômes minimaux sont \( \mu_a=X^2+1\) et \( \mu_2=X^2+2\). Nous avons effectivement \( \mu_a\mu_b\) divise \( P\) dans \( \eR[X]\).

    Si par contre nous considérions les racines \( a=i\) et \( b=-i\), nous aurions \( \mu_a=\mu_n=X^2+1\), et les polynôme \( \mu_a^2\) ne divise pas \( P\).
\end{example}

%--------------------------------------------------------------------------------------------------------------------------- 
\subsection{Polynômes à plusieurs variables}
%---------------------------------------------------------------------------------------------------------------------------

Nous avons déjà vu \( A[X,Y]\) lorsque \( A\) est un anneau en la définition \ref{DEFooZNHOooCruuwI}.

\begin{definition}      \label{DEFooRHRKooPqLNOp}
    Soit un corps \( \eK\). Le corps \( \eK(X_1,\ldots, X_n)\) est le corps des fractions de l'anneau \( A[X_1,\ldots, X_n]\).
\end{definition}

\begin{definition}  \label{DEFooOCPHooXneutp}
    Soient un corps \( \eK\) et une extension \( \eL\) de \( \eK\) contenant les éléments \( \alpha_1\),\ldots, \( \alpha_n\). Nous définissons \( \eK(\alpha_1,\ldots, \alpha_n)\) comme étant l'intersection de tous les sous-corps de \( \eL\) contenant \( \eK\) et les \( \alpha_i\).
\end{definition}

La proposition suivante est analogue à \ref{PROPooYSFNooFGbbCi}\ref{ITEMooATPTooVXKdlK}.

\begin{lemma}[\cite{MonCerveau}]        \label{LEMooQEJHooAmSNxU}
    Soient un corps \( \eK\), une extension \( \eL\) et des éléments \( \alpha_1,\ldots, \alpha_n\) dans \( \eL\). Alors
    \begin{equation}
        \eK(\alpha_1,\ldots, \alpha_n)=\{ r(\alpha_1,\ldots, \alpha_n)\tq r\in \eK(X_1,\ldots, X_n) \}.
    \end{equation}
\end{lemma}

\begin{proof}
    Ce que nous avons à droite est un corps : par exemple pour l'inverse, si \( r=P/Q\) alors \( r(\alpha_1,\ldots,\alpha_n)=P(\alpha_1,\ldots, \alpha_n)Q(\alpha_1,\ldots, \alpha_n)^{-1}\). Cet élément a un inverse en la personne de \( (Q/P)(\alpha_1,\ldots, \alpha_n)\).

    Donc à droite nous avons un sous-corps de \( \eL\) contenant \( \eK\) ainsi que les \( \alpha_i\). Donc
    \begin{equation}
        \eK(\alpha_1,\ldots, \alpha_n)\subset \big\{ r(\alpha_1,\ldots, \alpha_n)\tq r\in \eK(X_1,\ldots, X_n) \big\}.
    \end{equation}
    
    D'autre part, tout corps contenant \( \eK\) et les \( \alpha_i\) doit contenir tous les \( P(\alpha_1,\ldots, \alpha_n)\) (\( P\in \eK[X_1,\ldots, X_n]\)) et leurs inverses et leurs produits, bref tous les \( r(\alpha_1,\ldots, \alpha_n)\) avec \( r\in\eK[X_1,\ldots, X_n]\).
\end{proof}

%---------------------------------------------------------------------------------------------------------------------------
\subsection{Racines de polynômes}
%---------------------------------------------------------------------------------------------------------------------------

\begin{corollary}[Factorisation d'une racine]   \label{CorDIYooEtmztc}
    Soit \( P\in \eK[X]\), un polynôme de degré \( n\) et \( \alpha\in \eK\) tel que \( P(\alpha)=0\). Alors il existe un polynôme \( Q\) de degré \( n-1\) tel que \( P(x)=(X-\alpha)Q\).
\end{corollary}
\index{factorisation!de polynôme}

\begin{proof}
    Il s'agit d'un cas particulier de la proposition \ref{PropXULooPCusvE} : si \( \alpha\in \eK\) alors son polynôme minimal dans \( \eK\) est \( X-\alpha\); donc \( X-\alpha\) divise \( P\). Il existe un polynôme \( Q\) tel que \( P=(X-\alpha)Q\). Le degré est alors immédiat.
\end{proof}

\begin{theorem}[Polynôme qui a tellement de racines qu'il s'annule]\label{ThoLXTooNaUAKR}
    Soit \( \eK\) un corps et \( P\in \eK[X]\) un polynôme de degré \( n\) possédant \( n+1\) racines distinctes \( \alpha_1\),\ldots, \( \alpha_{n+1}\), alors \( P=0\).
\end{theorem}
\index{racine!de polynôme}
Voir \ref{NORMooQFTJooLBcPxl} pour savoir ce qu'est un polynôme nul.

\begin{proof}
    Si \( P\) est de degré \( 1\), il s'écrit \( P=aX+b\); s'il a comme racines \( \alpha\) et \( \beta\), nous avons le système
    \begin{subequations}
        \begin{numcases}{}
            a\alpha+b=0\\
            a\beta+b=0.
        \end{numcases}
    \end{subequations}
    La différence entre les deux donne \( a(\alpha-\beta)=0\). Vu que \( \alpha\neq \beta\), la règle du produit nul (lemme \ref{LemAnnCorpsnonInterdivzer}) nous donne \( a=0\). Maintenant que \( a=0\), l'annulation de \( b\) est alors immédiate.

    Nous faisons maintenant la récurrence en supposant le théorème vrai pour le degré \( n\) et en considérant un polynôme \( P\) de degré \( n+1\) possédant \( n+2\) racines distinctes. Vu que \( P(\alpha_1)=0\), le corollaire \ref{CorDIYooEtmztc} nous donne un polynôme \( Q\) de degré \( n\) tel que 
    \begin{equation}    \label{EqQGSooNdTWfz}
        P=(X-\alpha_1)Q.
    \end{equation}
    Étant donne que pour tout \( i\neq 1\) nous avons \( \alpha_i\neq \alpha_1\), 
    \begin{equation}
        0=P(\alpha_i)=\underbrace{(\alpha_i-\alpha_1)}_{\neq 0}Q(\alpha_i),
    \end{equation}
    et la règle du produit nul donne \( Q(\alpha_i)=0\). Par conséquent le polynôme \( Q\) est de degré \( n\) et possède \( n+1\) racines distinctes; tous ses coefficients sont alors nuls par hypothèse de récurrence. Tous les coefficients du produit \eqref{EqQGSooNdTWfz} sont alors également nuls.
\end{proof}

\begin{example}\label{ExGRHooBNpjSP}
    Un polynôme à plusieurs variables peut s'annuler en une infinité de points sans être nul. Par exemple le polynôme \( X^2+Y^2-1\in\eR[X,Y]\) s'annule sur tout un cercle de \( \eR^2\) mais n'est pas nul, loin s'en faut.

    Nous verrons dans la proposition \ref{PropTETooGuBYQf} une condition pour qu'un polynôme à plusieurs variables s'annule du fait qu'il ait «trop» de racines.
\end{example}

\begin{remark}
    L'intérêt du théorème \ref{ThoLXTooNaUAKR} est que si l'on prouve qu'un polynôme s'annule sur un corps infini, alors il s'annulera sur n'importe quel autre corps. Nous aurons un exemple d'utilisation de cela dans le théorème de Cayley-Hamilton \ref{ThoHZTooWDjTYI}.
\end{remark}

%---------------------------------------------------------------------------------------------------------------------------
\subsection{Corps de rupture}
%---------------------------------------------------------------------------------------------------------------------------

\begin{definition}      \label{DEFooVALTooDJJmJv}
    Soit \( P\in\eK[X]\) un polynôme irréductible. Une extension \( \eL\) de \( \eK\) est un \defe{corps de rupture}{corps!de rupture}\index{rupture!corps} pour \( P\) s'il existe \( a\in \eL\) tel que \( P(a)=0\) et \( \eL=\eK(a)\).
\end{definition}

\begin{example}     \label{ExemGVxJUC}
    Soit \( \eK=\eQ\) et \( P=X^2-2\). On pose \( a=\sqrt{2}\) et \( \eL=\eQ(\sqrt{2})\subset\eR\). De cette façon \( P\) est scindé :
    \begin{equation}
        P=(X-\sqrt{2})(X+\sqrt{2}).
    \end{equation}
    Le corps \( \eQ(\sqrt{2})\) est donc un corps de rupture pour \( P\).
\end{example}

\begin{example}
    Dans l'exemple \ref{ExemGVxJUC}, nous avions un corps de rupture dans lequel le polynôme \( P\) était scindé. Il n'en est pas toujours ainsi. Prenons 
    \begin{equation}
        P=X^3-2
    \end{equation}
    et \( a=\sqrt[3]{2}\). Nous avons, certes, \( P(a)=0\) dans \( \eQ(\sqrt[3]{2})\), mais \( P\) n'est pas scindé parce qu'il y a deux racines complexes.
\end{example}

\begin{example}
    Nous considérons le corps \( \eZ/p\eZ\) où \( p\) est un nombre premier. Si \( s\in \eZ/p\eZ\) n'est pas un carré, alors le polynôme \(P= X^2+s\) est irréductible et un corps de rupture de \( P\) sur \( \eZ/p\eZ\) est donné par \( (\eZ/p\eZ)[X]/(X^2+s)\), c'est à dire l'ensemble des polynômes de degrés \( 1\) en \( \sqrt{s}\). Le cardinal en est \( p^2\).
\end{example}


\begin{proposition}[Existence d'un corps de rupture]        \label{PROPooUBIIooGZQyeE}
    Soit un corps \( \eK\) et un polynôme non constant \( P\). Alors
    \begin{enumerate}
        \item
            Le corps \( \eL=\eK[X]/(P)\) est un corps de rupture pour \( P\).
        \item
            L'élément \( \bar X\) de \( \eL\) est une racine de \( P\).
        \item
            \( \eL=\eK(\bar X)_{\eL}\)
    \end{enumerate}
\end{proposition}

\begin{proof}
    Commençons par nous convaincre que \( \eK[X]/(P)\) est une extension de \( \eK\) (définition \ref{DEFooFLJJooGJYDOe}). Le morphisme \( j\colon \eK\to \eK[X]/(P)\) est simplement \( k\mapsto \bar k\) où à droite, \( \bar k\) voit \( k\) dans \( \eK[X]\) comme étant le polynôme constant. Notez qu'il est automatiquement injectif (lemme \ref{LEMooWBOPooZnsZgQ}).

    Il faut maintenant voir que \( \eK[X]/(P)=\eK(\alpha)\) pour un certain \( \alpha\in \eK[X]/(P)\). Grâce à notre compréhension des notations acquise dans \ref{SUBSUBSECooPNBYooWXEHrg}, nous savons que \( X\in\eK[X]\) et qu'il est donc parfaitement légitime de poser \( \alpha=\bar X\) dans \( \eK[X]/(P)\). Il s'agit simplement de l'ensemble \( \bar X=\{ X+QP\tq Q\in \eK[X] \}\) où \( X\) est une notation pour la suite \( (0,1,0,0,\ldots)\). 

    Bref, nous notons \( \alpha=\bar X\) et nous démontrons que \( P(\alpha)=0\) et que \( \eK[X]/(P)=\eK(\alpha)\) (isomorphisme de corps).
    \begin{subproof}
        \item[\( P(\bar X)=0\)]
        
            C'est le moment de nous souvenir comment la notation des \( X\) fonctionne, et en particulier la pirouette autour de \eqref{EQooABULooFCEasf}. D'abord la définition du produit sur \( \eK[X]/(P)\) est \( \bar P\bar Q=\overline{ PQ }\); en particulier si \( P=\sum_ka_kX^k\), alors \( P(\bar X)=\sum_ka_k\bar X^k=\sum_ka_k\overline{ X^k }\), et
            \begin{equation}
                P(\bar X)=\overline{ P(X) }=\bar P=0.
            \end{equation}
        \item[L'égalité]

            Nous montrons à présent que \( \eK(\bar X)_{\eL}=\eL\). C'est à dire que \( \eL\) est bien engendrée par \( \eK\) et un seul élément. D'abord, \( \eL=\eK[X]/(P)\) contient bien évidemment \( \eK\) et \( \bar X\). Ensuite nous devons prouver que tout sous-corps de \( \eL\) contenant \( \eK\) et \( \bar X\) est en réalité \( \eL\) entier.

            Soit \( Q\in \eK[X]\), et montrons que \( \bar Q\) est dans tout sous-corps de \( \eL\) contenant \( \eK\) et \( \bar X\). 
            
            Par le lemme \ref{LEMooXFMAooMBgIrN} nous avons \( \bar Q=Q(\bar X)\). Et si un corps contient \( \eK\) et \( \bar X\), il doit contenir tous les polynômes en \( \bar X\) à coefficients dans \( \eK\). Donc un tel corps doit contenir \( Q(\bar X)\) et donc \( \bar Q\).

    \end{subproof}
\end{proof}

\begin{example}
    Soit le polynôme \( P=X^2+1\in \eZ[X]\). Dans le quotient \( \eZ[X]/(P)\) nous avons \( \bar X^2+1=1\) et donc \( \bar X^2=-1\). C'est à dire que \( \eZ[X]/(P)\) contient un élément dont le carré est \( -1\). Avouez que c'est bien ce à quoi nous nous attendions.

    Notons que \( -\bar X\) est également une racine de \( P\) dans \( \eZ[X]/(P)\).

    En calculant dans les polynôme à coefficients dans \( \eZ(\bar X)\) nous avons :
    \begin{equation}
        (X+\bar X)(X-\bar X)=X^2-\bar X^2=X^2+1,
    \end{equation}
    c'est à dire que \( P\) est bien factorisé, et que nous avons retrouvé la multiplication \( x^2+1=(x+i)(x-i)\).
\end{example}

\begin{normaltext}
    Il n'y a évidemment pas unicité d'un corps de rupture pour un polynôme donné. Une raison est qu'un polynôme peut accepter plusieurs racines complètement indépendantes. Le corps étendu par l'une ou l'autre racine donne deux corps de rupture différents. Par exemple dans \( \eQ[X]\), le polynôme
    \begin{equation}
        P=X^4-X^2-2
    \end{equation}
    a pour racines (dans \( \eC\)) les nombres \( \sqrt{ 2 }\) et \( i\). Donc on a deux corps de ruptures complètement différents : \( \eQ(\sqrt{ 2 })\) et \( \eQ(i)\).
\end{normaltext}

\begin{normaltext}
    La proposition suivante donne une unicité du corps de rupture dans le cas d'un polynôme irréductible. Et nous comprenons pourquoi : un polynôme irréductible n'a fondamentalement qu'une seule racine «indépendante». Par exemple \( X^2-2\) a pour racines \( \pm\sqrt{ 2 }\). Autre exemple, le polynôme \( X^2+6X+13\) a pour racines, dans \( \eC\) les nombres complexes conjugués \( z=-3+2i\) et \( \bar z=-3-2i\).
\end{normaltext}

\begin{proposition}[\cite{ooUHHUooONXDDl}]          \label{PROPooVJACooNDmlfb}
    Soient un corps \( \eK\) et un polynôme irréductible \( P\in \eK[X]\). Alors toute extension \( \eL\) contenant une racine \( \alpha\) de \( P\) admet un unique morphisme de corps
            \begin{equation}
                \psi\colon \eK[X]/(P)\to \eL
            \end{equation}
            tel que \( \psi(\bar X)=\alpha\).

    Dans un tel cas,
    \begin{enumerate}
        \item
            l'image de \( \psi\) est $\eK(\alpha)_{\eL}$ ,
        \item       \label{ITEMooHRFHooWLIdWU}
            si \( \eL=\eK(\alpha)_{\eL}\) alors \( \psi\) est un isomorphisme.
    \end{enumerate}

\end{proposition}

\begin{proof}
    L'idéal annulateur de \( \alpha\) parmi les polynôme de \( \eK[X]\) n'est pas réduit à \( \{ 0 \}\) parce qu'il contient \( P\). Le lemme \ref{DefCVMooFGSAgL} s'applique donc et nous avons le polynôme minimal \( \mu\) de \( \alpha\) dans \( \eK[X]\). Il divise \( P\) qui est irréductible, donc
    \begin{equation}
        P=\lambda \mu
    \end{equation}
    pour un certain \( \lambda\in \eK\).

    Nous posons
    \begin{equation}
        \begin{aligned}
            \psi\colon \eK[X]/(P)&\to \eL \\
            \bar Q&\mapsto Q(\alpha). 
        \end{aligned}
    \end{equation}
    \begin{subproof}
        \item[Bien définie]
            Si \( \bar Q_1=\bar Q_2\) alors il existe un \( R\in \eK[X]\) tel que \( Q_1=Q_2+RP\). Mais alors \( \psi(\bar Q_1)=Q_1(\alpha)=Q_2(\alpha)+R(\alpha)P(\alpha)=Q_2(\alpha)\).
        \item[Injective]

            Si \( \psi(\bar Q_1)=\psi(\bar Q_2)\) alors \( Q_1-Q_2=R\) pour un certain \( R\in \eK[X]\) vérifiant \( R(\alpha)=0\). Nous avons alors un polynôme \( S\) tel que \( R=S\mu=\lambda^{-1}SP\). Donc \( \bar R=0\) et donc \( \bar Q_1=\bar Q_2\).

        \item[Morphisme]

            Laissé comme exercice; la paresse de l'auteur de ces lignes attend vos contributions.

        \item[La condition]

            Le morphisme \( \psi\) respecte de plus la condition
            \begin{equation}
                \psi(\bar X)=X(\alpha)=\alpha.
            \end{equation}

    \end{subproof}

    En ce qui concerne l'unicité, fixer \( \psi(\bar X)\) est suffisant pour fixer un morphisme. En effet si \( \psi(\bar X)=\alpha\), alors
    \begin{equation}
        \psi(\bar Q)=\psi\Big( \sum_ka_k\bar X^k \Big)=\sum_ka_k\psi(\bar X)^k=\sum_ka_k\alpha^k.
    \end{equation}
    
    Pour le second point de l'énoncé, il faut remarquer que \( \alpha\) est algébrique et non transcendant. Donc en utilisant les propositions \ref{PROPooPMNSooOkHOxJ} et \ref{PropURZooVtwNXE}\ref{ItemJCMooDgEHaji} nous trouvons
    \begin{equation}
        \Image(\psi)=\{ Q(\alpha)\tq Q\in \eK[X] \}=\eK[\alpha]=\eK(\alpha).
    \end{equation}

    Et finalement pour le dernier point, un morphisme de corps est toujours injectif. Si il est également surjectif, il sera bijectif.
\end{proof}

%--------------------------------------------------------------------------------------------------------------------------- 
\subsection{Pile d'extensions}
%---------------------------------------------------------------------------------------------------------------------------

\begin{lemma}[\cite{MonCerveau}]        \label{LEMooTURZooXnjmjT}
    Soient un corps \( \eK\), des extensions \( \eL_1\),\ldots, \( \eL_n\) et des éléments \( \alpha_i\in \eL_i\) tels que
    \begin{equation}
        \eL_k=\eL_{k-1}(\alpha_k)_{\eL_k}.
    \end{equation}
    Alors
    \begin{equation}
        \eL_n=\eK(\alpha_1,\ldots, \alpha_n)_{\eL_n}.
    \end{equation}
\end{lemma}

\begin{proof}
    Nous démontrons par récurrence sur \( n\), en supposant que c'est fait pour \( n=1\). Supposons donc que le lemme soit correct pour \( n\), et étudions le cas \( n+1\). Nous avons, par définition et par hypothèse de récurrence :
    \begin{equation}
        \eL_{n+1}=\eL_n(\alpha_{n+1})_{\eL_{n+1}}=\Big( \eK(\alpha_1,\ldots, \alpha_n) \Big)(\alpha_{n+1})_{\eL_{n+1}}.
    \end{equation}
    Notre tâche sera donc de montrer que
    \begin{equation}
        \Big( \eK(\alpha_1,\ldots, \alpha_n) \Big)(\alpha_{n+1})_{\eL_{n+1}}=\eK(\alpha_1,\ldots,\alpha_n)
    \end{equation}
    où nous n'écrivons plus les indices \( \eL_{n+1}\) partout.

    À gauche nous avons un sous-corps de \( \eL_{n+1}\) contenant tout ce qu'il faut, donc
    \begin{equation}
        \eK(\alpha_1,\ldots,\alpha_n)\subset \big( \eK(\alpha_1,\ldots, \alpha_n) \big)(\alpha_{n+1})_{\eL_{n+1}}.
    \end{equation}
    
    \begin{subproof}
        \item[Où chercher ?]
            Pour prouver l'inclusion inverse nous devons montrer que tout élément du corps \( \big( \eK(\alpha_1,\ldots, \alpha_n) \big)(\alpha_{n+1})\) est forcément dans tout sous-corps de \( \eL_{n+1}\) contenant \( \eK\) et les \( \alpha_i\).

            Un tel élément est, par la proposition \ref{PROPooYSFNooFGbbCi}\ref{ITEMooATPTooVXKdlK}, de la forme \( r(\alpha_{n+1})\) avec \( r\in \eK(\alpha_1,\ldots, \alpha_{n})(X)\), c'est à dire
            \begin{equation}
                P(\alpha_{n+1})Q(\alpha_{n+1})^{-1}
            \end{equation}
            avec \( P,Q\in \eK(\alpha_1,\ldots, \alpha_n)[X]\).

        \item[Ok pour un polynôme]

            Soit \( P\in \eK(\alpha_1,\ldots, \alpha_n)[X]\); nous allons montrer que \( P(\alpha_{n+1})\) est dans tout sous-corps de \( \eL_{n+1}\) contenant \( \eK\) et les \( \alpha_i\). Nous avons \( P=\sum_ia_iX^i\) pour \( a_i\in \eK(\alpha_1,\ldots, \alpha_n)\), et donc
            \begin{equation}
                P(\alpha_{n+1})=\sum_ia_i\alpha_{n+1}^i.
            \end{equation}
            Tout corps contenant \( \eK\) et les \( \alpha_1\),\ldots, \( \alpha_n\) contient les \( a_i\). Donc tout corps contenant \( \eK\), \( \alpha_1\),\ldots,  \( \alpha_{n+1}\) contient successivement \( a_i\), \( a_i\alpha_{n+1}^i\), \( P(\alpha_i)\).

        \item[Ok pour les fractions]

            Si \( r=P/Q\) alors \( r(\alpha_{n+1})=P(\alpha_{n+1})Q(\alpha_{n+1})^{-1}\). Nous avons déjà prouvé que si un corps contient \( \eK\) et les \( \alpha_i\) (\( i=1,\ldots, n+1\)) alors il contient \( P(\alpha_{n+1})\) et \( Q(\alpha_{n+1})\). Vu que c'est un corps, il contient l'inverse et le produit et donc l'élément
            \begin{equation}
                r(\alpha_{n+1})=P(\alpha_{n+1})Q(\alpha_{n+1})^{-1}.
            \end{equation}
    \end{subproof}
\end{proof}

