% This is part of Mes notes de mathématique
% Copyright (c) 2008-2018
%   Laurent Claessens
% See the file fdl-1.3.txt for copying conditions.

%+++++++++++++++++++++++++++++++++++++++++++++++++++++++++++++++++++++++++++++++++++++++++++++++++++++++++++++++++++++++++++
\section{Intervalles}
%+++++++++++++++++++++++++++++++++++++++++++++++++++++++++++++++++++++++++++++++++++++++++++++++++++++++++++++++++++++++++++

\begin{definition}[Intervalle]
    Une partie \( I\) de \( \eR\) est un \defe{intervalle}{intervalle} si pour tout \( a,b\in I\) nous avons \( t\in I\) dès que \( a\leq t\leq b\).

    Un intervalle est \defe{ouvert}{intervalle!ouvert} s'il est de la forme \( \mathopen] a , b \mathclose[\) avec éventuellement \( a=-\infty\) ou \( b=+\infty\). Un intervalle est \defe{fermé}{intervalle!fermé} s'il est de la forme \( \mathopen[ a , b \mathclose]\) ou \( \mathopen] -\infty , b \mathclose]\) ou \( \mathopen[ a , +\infty [\) avec \( a,b\in \eR\).
\end{definition}

\begin{remark}
  L'ensemble $\eR$ ne contient pas $-\infty$ et $-\infty$. L'intervalle $[-\infty, 5]$ par exemple, n'est pas une partie de $\eR$.
\end{remark}

\begin{example}
    \begin{enumerate}
        \item
        Les ensembles \( \mathopen] 3 , 7 \mathclose[\) et \( \mathopen] -\infty , \pi \mathclose[\) sont des intervalles ouverts.
        \item
            Les ensembles \( \mathopen[ 10 , 15 \mathclose]\) et \( \mathopen[ -1 , +\infty [\) sont des intervalles fermés.
        \item
        L'ensemble \( \mathopen] -4 , -2 \mathclose[\cup\mathopen] 2 , 9 \mathclose[\) n'est pas un intervalle (il y a un «trou» entre \(- 2\) et \( 2\)).
        \item
            L'ensemble \( \eR\) lui-même est un intervalle; par convention, il est à la fois ouvert et fermé.
    \end{enumerate}
Un intervalle peut n'être ni ouvert ni fermé; par exemple \( \mathopen] 4 , 8 \mathclose]\). Cet intervalle est «ouvert en \( 4\) et fermé en \( 8\)» .
\end{example}

\begin{definition}[Fonction, domaine, image, graphe]
  Soient $X$ et $Y$ deux ensembles. Une \defe{fonction}{fonction} $f$ définie sur $X$ et à valeurs dans $Y$ est une correspondence qui associe à chaque élément $x$ dans $X$ {\bf au plus} un élément $y$ dans $Y$. On écrit $y= f(x)$.
  \begin{itemize}
  \item La partie de $X$ qui contient tous les $x$ sur lesquels $f$ peut opérer est dite \defe{domaine}{domaine} de $f$. Le domaine de $f$ est indiqué par $\Dom f$.
  \item L'élément de $y\in Y$ associé par $f$ à un élément $x\in \Dom f$ (c'est à dire $f(x) = y$)  est appellé \defe{image}{image} de $x$ par $f$. L'\defe{image}{fonction!image} de la fonction $f$ est la partie de $Y$ qui contient les images de tous les éléments de $\Dom f$. L'image de $f$ est indiquée par $\Im f$.
  \item Le \defe{graphe}{graphe} de $f$ est l'ensemble de toutes les couples $(x, f(x))$ pour $x\in \Dom f$. Le graphe de $f$ est une partie de l'ensemble noté $X\times Y$ et il est indiqué par $\Graph f$. Dans ce cours $X = \eR$ et $Y = \eR$, donc le graphe de $f$ est contenu dans le plan cartésien.
  \end{itemize}
\end{definition}

\begin{definition}[Fonction croissante, décroissante et monotone]
    Soit \( f\colon \eR\to \eR\) une fonction définie sur un intervalle \( I\subset \eR\).
    \begin{enumerate}
        \item
            Le fonction \( f\) est \defe{croissante}{fonction!croissante} sur \( I\) si pour tout \( x<y\) dans \( I\) nous avons \( f(x)\leq f(y)\). Elle est \emph{strictement} croissante si \( f(x)<f(y)\) dès que \( x<y\).
        \item
            Le fonction \( f\) est \defe{décroissante}{fonction!décroissante} sur \( I\) si pour tout \( x<y\) dans \( I\) nous avons \( f(x)\geq f(y)\). Elle est \emph{strictement} décroissante si \( f(x)>f(y)\) dès que \( x<y\).
        \item
            La fonction \( f\) est dite \defe{monotone}{fonction!monotone} sur \( I\) si elle est soit croissante soit décroissante sur \( I\).
    \end{enumerate}
\end{definition}

\begin{example}
    La fonction \( x\mapsto x^2\) est décroissante sur l'intervalle \( \mathopen] -\infty , 0 \mathclose]\) et croissante sur l'intervalle \( \mathopen[ 0 , \infty \mathclose[\). Elle n'est par contre ni croissante ni décroissante sur l'intervalle \( \mathopen[ -4 , 3 \mathclose]\).
\end{example}

%+++++++++++++++++++++++++++++++++++++++++++++++++++++++++++++++++++++++++++++++++++++++++++++++++++++++++++++++++++++++++++
\section{Application réciproque}
%+++++++++++++++++++++++++++++++++++++++++++++++++++++++++++++++++++++++++++++++++++++++++++++++++++++++++++++++++++++++++++

%---------------------------------------------------------------------------------------------------------------------------
\subsection{Définitions}
%---------------------------------------------------------------------------------------------------------------------------

Les définitions d'injection, surjection, bijection et d'application réciproque sont les définitions~\ref{DEFooBFCQooPyKvRK} et~\ref{DEFooTRGYooRxORpY}.

\begin{example}     \label{EXooCWYHooLEciVj}
    \begin{enumerate}
        \item
            La fonction \( x\mapsto x^2\) n'est pas une bijection de \( \eR\) vers \( \eR\) parce qu'il n'existe aucun \( x\) tel que \( x^2=-1\).
        \item
            La fonction
            \begin{equation}
                \begin{aligned}
                    f\colon \mathopen[ 0 , +\infty [&\to \mathopen[ 0 , +\infty [ \\
                    x&\mapsto x^2
                \end{aligned}
            \end{equation}
            est une bijection. Notez que c'est la même fonction que celle de l'exemple précédent. Seul l'intervalle sur laquelle nous nous plaçons a changé.
        \item
            La fonction
            \begin{equation}
                \begin{aligned}
                    f\colon \eR&\to \mathopen[ 0 , \infty \mathclose[ \\
                    x&\mapsto x^2
                \end{aligned}
            \end{equation}
            n'est pas une bijections parce qu'il existe plusieurs \( x\) pour lesquels \( f(x)=4\).
        \item
            Nous verrons un peu plus tard (\ref{PROPooXQYFooPxoEHE}) que l'application
            \begin{equation}
                \begin{aligned}
                    f\colon \mathopen[ 0 , \infty \mathclose[\to \mathopen[ 0 , \infty \mathclose[\\
                    x&\mapsto x^2
                \end{aligned}
            \end{equation}
            est une bijection.
    \end{enumerate}
    En conclusion : il est très important de préciser les domaines des fonctions considérées.
\end{example}

\begin{remark}
    Dire que la fonction \( f\colon I\to J\) est bijective, c'est dire que l'équation \( f(x)=y\) d'inconnue \( x\) peut être résolue de façon univoque pour tout \( y\in J\).
\end{remark}

\begin{remark}
  Toute fonction strictement monotone sur un intervalle $I$ est injective.
\end{remark}

\begin{example}
    Trouvons la fonction réciproque de la fonction affine \( f\colon \eR\to \eR\), \( x\mapsto 3x-2\). Si \( y\in \eR\) le nombre \( f^{-1}(y)\) est la valeur de \( x\) pour laquelle \( f(x)=y\). Il s'agit donc de résoudre
    \begin{equation}
        3x-2=y
    \end{equation}
    par rapport à \( x\). La solution est \( x=\frac{ y+2 }{ 3 }\) et donc nous écrivons
    \begin{equation}
        f^{-1}(y)=\frac{ y+2 }{ 3 }.
    \end{equation}
    Notons que dans les calculs, il est plus simple d'écrire «\( y\)» que «\( x\)» la variable de la fonction réciproque. Il est néanmoins (très) recommandé de nommer «\( x\)» la variable dans la réponse finale. Dans notre cas nous concluons donc
    \begin{equation}
        f^{-1}(x)=\frac{ x+2 }{ 3 }.
    \end{equation}
\end{example}

%---------------------------------------------------------------------------------------------------------------------------
\subsection{Graphe de la fonction réciproque}
%---------------------------------------------------------------------------------------------------------------------------

Par définition le graphe de la fonction \( f\) est l'ensemble des points de la forme \( (x,y)\) vérifiant \( y=f(x)\). Afin de déterminer le graphe de la bijection réciproque nous pouvons faire le raisonnement suivant.

        Le point \( (x_0,y_0)\) est sur le graphe de \( f\)

\noindent\( \Leftrightarrow\)

        La relation \( f(x_0)=y_0\) est vérifiée

\noindent\( \Leftrightarrow\)

        La relation \( x_0=f^{-1}(y_0)\) est vérifiée

\noindent\( \Leftrightarrow\)

        Le point \( (y_0,x_0)\) est sur le graphe de \( f^{-1}\).

\begin{Aretenir}
    Dans un repère orthonormal, le graphe de la bijection réciproque est obtenu à partir du graphe de \( f\) en effectuant une symétrie par rapport à la droite d'équation \( y=x\).
\end{Aretenir}

Le dessin suivant montre le cas de la courbe de la fonction carré comparé à celle de la racine carrée.
\begin{center}
   \input{auto/pictures_tex/Fig_CELooGVvzMc.pstricks}
\end{center}

%+++++++++++++++++++++++++++++++++++++++++++++++++++++++++++++++++++++++++++++++++++++++++++++++++++++++++++++++++++++++++++
\section{Limite de fonctions}
%+++++++++++++++++++++++++++++++++++++++++++++++++++++++++++++++++++++++++++++++++++++++++++++++++++++++++++++++++++++++++++

%---------------------------------------------------------------------------------------------------------------------------
\subsection{Définition}
%---------------------------------------------------------------------------------------------------------------------------

La définition générale de la limite est~\ref{DefYNVoWBx}. Dans le cas de fonctions \( \eR\to \eR\), elle peut s'écrire de façon plus efficace. La proposition suivante montre comment fonctionne la limite pour une fonction définie sur tout \( \eR\).

\begin{proposition}[Caractérisation de la limite]       \label{PropAJQQooQQClfp}
	Soit une fonction $f\colon \eR\to \eR$ définie sur \( \eR\) et $a\in \eR$. La fonction \( f\) admet la limite \( \ell\) pour \( x\to a\) si et seulement si il existe un réel $\ell$ tel que pour tout \( \epsilon>0\), il existe un \( \delta>0\) tel que
	\begin{equation}\label{EqDefLimiteFonction}
		0<| x-a |<\delta\Rightarrow| f(x)-\ell |<\varepsilon.
	\end{equation}
\end{proposition}

\begin{proof}
    Il s'agit de montrer l'équivalence avec la définition~\ref{DefYNVoWBx}. Nous allons faire un usage intensif de la remarque~\ref{RemQDRooKnwKk}\ref{ITEMooUIHJooXAFaJa}.
    \begin{subproof}
    \item[Sens direct]
        Soient \( \epsilon>0\) et \( V=B(\ell,\epsilon)\). Alors il existe un voisinage \( W\) de \( a\) dans \( \eR\) tel que
        \begin{equation}
            f\big( W\setminus\{ a \} \big)\subset V.
        \end{equation}
        Soit \( \delta\) tel que \( B(a,\delta)\subset W\). Nous avons encore
        \begin{equation}
            f\big( B(a,\delta)\setminus\{ a \} \big)\subset V.
        \end{equation}
        Soit maintenant \( x\in \eR\) tel que $0<| x-a |<\delta$. Cela signifie \( x\in B(a,\delta)\setminus\{ a \}\). Pour un tel \( x\) nous avons donc \( f(x)\in B(\ell,\epsilon)\), c'est à dire \( | f(x)-\ell |<\epsilon\).
    \item[Dans l'autre sens]
        Soient un voisinage \( V\) de \( \ell\) et \( \epsilon>0\) tel que \( B(\ell,\epsilon)\subset V\). Nous considérons \( \delta\) tel que \( 0<| x-a |<\delta\) implique \( | f(x)-\ell |<\epsilon\).

        Avec tout cela nous posons \( W=B(x,\delta)\), et nous avons
        \begin{equation}
            f\big( W\setminus\{ a \} \big)\subset B(\ell,\alpha)\subset V.
        \end{equation}
    \end{subproof}
\end{proof}

Si aucun nombre $\ell$ ne vérifie la condition de la définition, alors on dit que la fonction n'admet pas de limite en $a$. Lorsque $f$ possède la limite $\ell$ en $a$, nous notons
\begin{equation}
	\lim_{x\to a} f(x)=\ell.
\end{equation}

La proposition suivante a déjà été démontrée dans la proposition~\ref{PropFObayrf}. Nous en donnons ici une démonstration adaptée au cas \( \eR\to \eR\).

\begin{proposition}
	Soit une fonction $f\colon D\to \eR$. Si $a$ est un point d'accumulation de $D$ et s'il existe une limite de $f$ en $a$, alors il en existe une seule.
\end{proposition}


\begin{proof}
    Nous prouvons qu'il ne peut pas exister deux nombres $\ell\neq\ell'$ vérifiant tout les deux la condition \eqref{EqDefLimiteFonction}.

	Soient $\ell$ et $\ell'$ deux limites de $f$ au point $a$. Par définition, pour tout $\varepsilon$ nous avons des nombres $\delta$ et $\delta'$ tels que
	\begin{equation}	\label{EqsContf2307Right}
		\begin{aligned}[]
			| x-a |<\delta&\Rightarrow \big| f(x)-\ell \big|<\varepsilon\\
			| x-a |<\delta'&\Rightarrow \big| f(x)-\ell' \big|<\varepsilon
		\end{aligned}
	\end{equation}
	Pour fixer les idées, supposons que $\delta<\delta'$ (le cas $\delta\geq\delta'$ se traite de la même manière).

	Étant donné que $a$ est un point d'accumulation du domaine $D$ de $f$, il existe un $x\in D$ tel que $| x-a |<\delta$. Évidemment, nous avons aussi $| x-a |<\delta'$. Les conditions \eqref{EqsContf2307Right} signifient alors que ce $x$ vérifie en même temps
	\begin{equation}
		| f(x)-\ell |<\varepsilon,
	\end{equation}
	et
	\begin{equation}
		| f(x)-\ell' |<\varepsilon.
	\end{equation}
	Afin de prouver que $\ell=\ell'$, nous allons maintenant calculer $| \ell-\ell' |$ et montrer que cette distance est plus petite que tout nombre. Nous avons (voir remarque~\ref{RemTechniqueIneqs})
	\begin{equation}	\label{EqInesq2307ellellepr}
		| \ell-\ell' |=| \ell-f(x)+f(x)-\ell' |\leq | \ell-f(x) |+| f(x)-\ell' |<\varepsilon+\varepsilon.
	\end{equation}
	En résumé, pour tout $\varepsilon>0$ nous avons
	\begin{equation}
		| \ell-\ell' |<2\varepsilon,
	\end{equation}
	et donc $| \ell-\ell' |=0$, ce qui signifie que $\ell=\ell'$.
\end{proof}

\begin{remark}		\label{RemTechniqueIneqs}
	Les inégalités \eqref{EqInesq2307ellellepr} utilisent deux techniques très classiques en analyse qu'il convient d'avoir bien compris. La première est de faire
	\begin{equation}
		| A-B |=| A-C+C-B |.
	\end{equation}
	Il s'agit d'ajouter $-C+C$ dans la norme. Évidemment, cela ne change rien.

	La seconde technique est l'inégalité
	\begin{equation}
		| A+B |\leq| A |+| B |.
	\end{equation}
\end{remark}

\begin{example}
	Considérons la fonction $f(x)=2x$, et calculons la limite $\lim_{x\to 3} f(x)$. Vu que $f(3)=6$, nous nous attendons à avoir $\ell=6$. C'est ce que nous allons prouver maintenant. Pour chaque $\varepsilon>0$ nous devons trouver un $\delta>0$ tel que $| x-3 |<\delta$ implique $| f(x)-6 |<\varepsilon$. En remplaçant $f(x)$ par sa valeur en fonction de $x$ et avec quelques manipulations nous trouvons :
	\begin{equation}
		\begin{aligned}[]
			| f(x)-6 |&<\varepsilon\\
			| 2x-6 |&<\varepsilon\\
			2| x-3 |&<\varepsilon\\
			| x-3 |&<\frac{ \varepsilon }{2}
		\end{aligned}
	\end{equation}
	Donc dès que $| x-3 |<\frac{ \varepsilon }{2}$, nous avons $| f(x)-6 |<\varepsilon$. Nous posons donc $\delta=\frac{ \varepsilon }{2}$.

	Plus généralement, nous avons $\lim_{x\to a} f(x)=2a$, et cela se prouve en étudiant $| f(x)-2a |$ exactement de la même manière.
\end{example}

%---------------------------------------------------------------------------------------------------------------------------
\subsection{Quelque règles de calcul}
%---------------------------------------------------------------------------------------------------------------------------

\begin{proposition}	\label{PropLimEstLineraure}
	La limite est une opération linéaire, c'est à dire que si $f$ et $g$ sont des fonctions qui admettent des limites en $a$ et si $\lambda$ est un nombre réel,
	\begin{enumerate}

		\item
			$\lim_{x\to a} (\lambda f)(x)=\lambda\lim_{x\to a} f(x)$,
		\item
			$\lim_{x\to a} (f+g)(x)=\lim_{x\to a} f(x)+\lim_{x\to a} g(x)$.
	\end{enumerate}
\end{proposition}
En combinant les deux propriétés de la proposition~\ref{PropLimEstLineraure}, nous pouvons écrire
\begin{equation}
	\lim_{x\to a} (\lambda f+\mu g)(x)=\lambda\lim_{x\to a} f(x)+\mu\lim_{x\to a} g(x).
\end{equation}
pour toutes fonctions $f$ et $g$ admettant une limite en $a$ et pour tout réels $\lambda$ et $\mu$.

En plus d'être linéaire, la limite possède les deux propriétés suivantes.
\begin{proposition}     \label{PROPooDQFIooMMwxxJ}
	Si $f$ et $g$ sont deux fonctions qui admettent une limite en $a$, alors
	\begin{equation}
		\lim_{x\to a} (fg)(x)=\lim_{x\to a} f(x)\cdot\lim_{x\to a} g(x).
	\end{equation}
	Si de plus $\lim_{x\to a} g(x)\neq 0$, alors
	\begin{equation}
		\lim_{x\to a} \frac{ f(x) }{ g(x) }=\frac{ \lim_{x\to a} f(x) }{ \lim_{x\to a} g(x) }.
	\end{equation}
\end{proposition}

\begin{theorem}     \label{ThoLimLinMul}
    Si
    \begin{equation} \label{Eqhypmullimlin}
      \lim_{x\to a}f(x)=b,
    \end{equation}
    alors
    \begin{equation} \label{Eqbutmultlim}
      \lim_{x\to a}(\lambda f)(x)=\lambda b
    \end{equation}
    pour n'importe quel $\lambda\in\eR$.
\end{theorem}

\begin{proof}
Soit $\epsilon>0$. Afin de prouver la propriété \eqref{Eqbutmultlim}, il faut trouver un $\delta$ tel que pour tout $x$ dans $[a-\delta,a+\delta]$, on ait $| (\lambda f)(x)- \lambda b |\leq\epsilon$. Cette dernière inégalité est équivalente à $|\lambda|| f(x)-b |\leq\epsilon$. Nous devons donc trouver un $\delta$ tel que
\begin{equation}
| f(x)-b |\leq\frac{ \epsilon }{ | \lambda | }.
\end{equation}
soit vraie pour tout $x$ dans $[a-\delta,a+\delta]$. Mais l'hypothèse \eqref{Eqhypmullimlin} dit précisément qu'il existe un $\delta$ tel que pour tout $x$ dans $[a-\delta,a+\delta]$ on ait cette inégalité.
\end{proof}

\begin{theorem}     \label{ThoLimLin}
    Si
    \begin{subequations}
    \begin{align}
        \lim_{x\to a}f(x)&=b_1\\
        \lim_{x\to a}g(x)&=b_2,
    \end{align}
    \end{subequations}
    alors
    \begin{equation}
        \lim_{x\to a}(f+g)(x)=b_1+b_2.
    \end{equation}
\end{theorem}

\begin{proof}
    Soit $\epsilon>0$. Par hypothèse, il existe $\delta_1$ tel que
    \begin{equation}    \label{Eqfbunepsdeux}
      | f(x)-b_1 |\leq \frac{ \epsilon }{ 2 }
    \end{equation}
    dès que $| x-a |\leq\delta_1$. Il existe aussi $\delta_2$ tel que
    \begin{equation}    \label{Eqgbdeuxepsdeux}
      | g(x)-b_2 |\leq \frac{ \epsilon }{ 2 }.
    \end{equation}
    dès que $| x-a |\leq \delta_2$. Tu notes l'astuce de prendre $\epsilon/2$ dans la définition de limite pour $f$ et $g$. Maintenant, ce qu'on voudrait c'est un $\delta$ tel que l'on ait $| (f+g)(x)-(b_1+b_2) |\leq \epsilon$ dès que $| x-a |\leq \delta$. Moi je dit que $\delta=\min\{ \delta_1,\delta_2 \}$ fonctionne. En effet, en utilisant l'inégalité $| a+b |\leq | a |+| b |$, nous trouvons :
    \begin{align}
    | (f+g)(x)-(b_1+b_2) |=| (f(x)-b_1)+(g(x)-b_2) |
            \leq | f(x)-b_1 |+| g(x)-b_2 |.     \label{Eqfplusgfbun}
    \end{align}
    Comme on suppose que $| x-a |\leq\delta$, on a évidemment $| x-a |\leq\delta_1$, et donc l'équation \eqref{Eqfbunepsdeux} tient. Mais si $| x-a |\leq\delta$, on a aussi $| x-a |\leq\delta_2$, et donc l'équation  \eqref{Eqfbunepsdeux} tient également. Chacun des deux termes de \eqref{Eqfplusgfbun} est donc plus petits que $\epsilon/2$, et donc le tout est plus petit que $\epsilon$, ce qu'il fallait montrer.

\end{proof}

Voici une formule qui résume la linéarité de la limite :
\begin{equation}    \label{EqLimLinRes}
    \lim_{x\to a}[\alpha f(x)+\beta g(x)]=\alpha\lim_{x\to a}f(x)+\beta\lim_{x\to a}g(x).
\end{equation}

\begin{proposition}[\cite{TrenchRealAnalisys}]      \label{PROPooOUPNooTrClHw}
    Soient des fonctions \( f,g\colon \eR\to \eR\) telles que \( \lim_{x\to a} f(x)=\ell\) et \( \lim_{x\to a} g(x)=\ell'\neq 0\). Alors
    \begin{equation}
        \lim_{x\to a} \frac{ f(x) }{ g(x) }=\frac{ \ell }{ \ell' }.
    \end{equation}
\end{proposition}

\begin{proof}
    Nous avons :
    \begin{equation}
        \left| \frac{ f(x) }{ g(x) }-\frac{ \ell }{ \ell' } \right| =\frac{ | \ell'f(x)-g(x)\ell | }{ |g(x)\ell| }.
    \end{equation}
    Soit \( s\), un minorant de \( | g(x) |\) sur un voisinage de \( a\); vu que la limite en \( a\) est \( \ell'\neq 0\), nous pouvons prendre par exemple \( s=\ell'/2\) : \( | g(x) |>\ell'/2\) sur \( B(a,\delta)\) dès que \( \delta\) est assez petit. Nous considérons \( x\in B(a,\delta)\). Avec cela nous avons :
    \begin{subequations}
        \begin{align}
            \left| \frac{ f(x) }{ g(x) }-\frac{ \ell }{ \ell' } \right| &=\frac{ | \ell'f(x)-g(x)\ell | }{ |g(x)\ell| }\\
            &\leq \frac{ 2 }{ | \ell'} |\left( \frac{ | \ell'f(x)-g(x)\ell | }{ | \ell' | } \right) \\
            &\leq \frac{ 2 }{ | \ell' |^2 }\big( | \ell'f(x)-\ell\ell' |+| \ell\ell'-g(x)\ell | \big)\\
            &=\frac{ 2 }{ | \ell' |^2 }\big( | \ell' | |f(x)-\ell |+| \ell | |\ell'-g(x) | \big).
        \end{align}
    \end{subequations}
    Soient \( \epsilon>0\) et \( \delta\) tel que \( | f(x)-\ell |<\epsilon\) et \( | g(x)-\ell' |<\epsilon\) pour tout \( x\in B(a,\delta)\). Avec cela nous avons
    \begin{equation}
        \left| \frac{ f(x) }{ g(x) }-\frac{ \ell }{ \ell' } \right| \leq\frac{ 2 }{ | \ell' |^2\big( | \ell' |+| \ell | \big) }\epsilon.
    \end{equation}
    D'où la limite attendue.
\end{proof}

\begin{theorem}     \label{Tholimfgabab}
    Si
    \begin{align}
        \lim_{x\to a}f(x)&=b_1&\text{et}&&\lim_{x\to a}g(x)=b_2,
    \end{align}
    alors
    \begin{equation}
        \lim_{x\to a}(fg)(x)=b_1b_2.
    \end{equation}
\end{theorem}

\begin{proof}
    Soit $\epsilon>0$, et tentons de trouver un $\delta$ tel que $| f(x)g(x)-b_1b_2 |\leq \epsilon$ dès que $| x-a |\leq \delta$. Nous avons
    \begin{equation}    \label{EqfgbunbdeuxMin}
    \begin{split}
    | f(x)g(x)-b_1b_2 |&=|  f(x)g(x)-b_1b_2 +f(x)b_2-f(x)b_2 |\\
            &=\left|   f(x)\big( g(x)-b_2 \big)+b_2\big( f(x)-b_1 \big)    \right|\\
            &\leq \left|  f(x)\big( g(x)-b_2 \big)  \right|+\left|  b_2\big( f(x)-b_1 \big)    \right|\\
            &= | f(x) | | g(x)-b_2  |+| b_2 | |f(x)-b_1 |.
    \end{split}
    \end{equation}
    À la première ligne se trouve la subtilité de la démonstration : on ajoute et on enlève\footnote{Comme exercice, tu peux essayer de refaire la démonstration en ajoutant et enlevant $g(x)b_1$ à la place.} $f(x)b_2$. Maintenant nous savons par le lemme~\ref{LemLimMajorableVois} que pour un certain $\delta_1$, la quantité $| f(x) |$ peut être majoré par un certain $M$ dès que $| x-a |\leq \delta_1$. Prenons donc un tel $\delta_1$ et supposons que $| x-a |\leq \delta_1$. Nous savons aussi que pour n'importe quel choix de $\epsilon_2$ et $\epsilon_3$, il existe des nombres $\delta_2$ et $\delta_3$ tels que $| f(x)-b_1 |\leq \epsilon_2$ et $| g(x)-b_1 |\leq \epsilon_3$ dès que $| x-a |\leq\delta_2$ et $| x-a |\leq\delta_3$. Dans ces conditions, la dernière expression \eqref{EqfgbunbdeuxMin} se réduit à
    \begin{equation}
    | f(x)g(x)-b_1b_2 |\leq M\epsilon_2+| b_2 |\epsilon_3.
    \end{equation}
    Pour terminer la preuve, il suffit de choisir $\epsilon_2$ et $\epsilon_3$ tels que $M\epsilon_2+| b_2 |\epsilon_3\leq\epsilon$, et puis prendre $\delta=\min\{ \delta_1,\delta_2,\delta_3 \}$.

    Remettons les choses dans l'ordre. L'on se donne $\epsilon$ au départ. La première chose est de trouver un $\delta_1$ qui permet de majorer $|f(x)|$ par $M$ selon le lemme~\ref{LemLimMajorableVois}, et puis choisissons $\epsilon_2$ et $\epsilon_3$ tels que $M\epsilon_2+| b_2 |\epsilon_3\leq\epsilon$. Ensuite nous prenons, en vertu des hypothèses de limites pour $f$ et $g$, les nombres $\delta_2$ et $\delta_3$ tels que $| f(x)-b_1 |\leq \epsilon_2$ et $| g(x)-b_2 |\leq \epsilon_3$ dès que $| x-a |\leq \delta_2$ et $| x-a |\leq \delta_3$.

    Si avec tout ça on prend $\delta=\min\{ \delta_1,\delta_2,\delta_3 \}$, alors la majoration et les deux inégalités sont valables en même temps et au final
    \[
      | f(x)g(x)-b_1b_2 |\leq M\epsilon_2+b_2\epsilon_3\leq \epsilon,
    \]
    ce qu'il fallait prouver.

\end{proof}

À l'aide de ces petits résultats, nous pouvons déjà calculer pas mal de limites. Nous pouvons déjà par exemple calculer les limites de tous les polynômes en tous les nombres réels. En effet, nous savons la limite de la fonction $f(x)=x$. La fonction $x\mapsto x^2$ n'est rien d'autre que le produit de $f$ par elle-même. Donc
\[
  \lim_{x\to a}x^2=\big( \lim_{x\to a}x\big)\cdot\big( \lim_{x\to a}x \big)=a^2.
\]
De la même façon, nous trouvons facilement que
\begin{equation}
 \lim_{x\to a}x^n=a^n.
\end{equation}

%---------------------------------------------------------------------------------------------------------------------------
\subsection{Limite en des nombres}
%---------------------------------------------------------------------------------------------------------------------------

Nous posons la définition suivante.
\begin{definition}      \label{DefInfNombre}
Lorsque $a\in\eR$, on dit que la fonction $f$ \defe{tend vers l'infini quand $x$ tend vers $a$}{} si
\[
  \forall M\in\eR,\exists \delta\tq (| x-a |\leq \delta )\Rightarrow f(x)\geq M\text{ quand }x\in\dom f.
\]
\end{definition}
Cela signifie que l'on demande que dès que $x$ est assez proche de $a$ (c'est à dire dès que $| x-a |\leq\delta$), alors $f(x)$ est plus grand que $M$, et que l'on peut trouver un $\delta$ qui fait ça pour n'importe quel $M$. Une autre façon de le dire est que pour toute hauteur $M$, on peut trouver un intervalle de largeur $\delta$ autour de $a$\footnote{C'est à dire un intervalle de la forme $[a-\delta,a+\delta]$.} tel que sur cet intervalle, la fonction $f$ est toujours plus grande que $M$.

Montrons sur un dessin pourquoi je disais que la fonction $x\to 1/x$ n'est pas de ce type.


Le problème est qu'il n'existe par exemple aucun intervalle autour de $0$ sur lequel $f$ serait toujours plus grande que $10$. En effet n'importe quel intervalle autour de $0$ contient au moins un nombre négatif. Or quand $x$ est négatif, $f$ n'est certainement pas plus grande que $10$. Nous y reviendrons.

Pour l'instant, montrons que la fonction $f(x)=1/x^2$ est une fonction qui vérifie la définition~\ref{DefInfNombre}.  Avant de prendre n'importe quel $M$, prenons par exemple $100$. Nous avons besoin d'un intervalle autour de zéro sur lequel $f$ est toujours plus grande que $100$. C'est vite vu que $f(0.1)=f(-0.1)=100$, donc l'intervalle $[-\frac{ 1 }{ 10 },\frac{1}{ 10 }]$ est le bon. Partout dans cet intervalle, $f$ est plus grande que $100$. Partout ? Ben non : en $x=0$, la fonction n'est même pas définie, donc c'est un peu dur de dire qu'elle est plus grande que $100$. C'est pour cela que nous avons ajouté la condition « quand $x\in\dom f$ » dans la définition de la limite.

Prenons maintenant un $M\in\eR$ arbitraire, et trouvons un intervalle autour de $0$ sur lequel $f$ est toujours plus grande que $M$. La réponse est évidemment l'intervalle de largeur $1/\sqrt{M}$, c'est à dire
\[
  \left[ -\frac{ 1 }{ \sqrt{M} },\frac{ 1 }{ \sqrt{M} } \right].
\]

\subsection{Limites quand tout va bien}
%--------------------------------------

D'abord définissons ce qu'on entend par la limite d'une fonction en un point quand il n'y a aucun infini en jeu.
\begin{definition}      \label{DefLimPointSansInfini}
 On dit que la fonction $f$ \defe{tend vers $b$ quand $x$ tend vers $a$}{} si
\[
  \forall \epsilon>0,\exists\delta\tq (| x-a |\leq\delta)\Rightarrow | f(x)-b |\leq \epsilon\text{ quand }x\in\dom f.
\]
Dans ce cas, nous notons
\begin{equation}
\lim_{x\to a}f(x)=b.
\end{equation}
\end{definition}

Commençons par un exemple très simple : prouvons que $\lim_{x\to 0}x=0$. C'est donc $a=b=0$ dans la définition. Prenons $\epsilon>0$, et trouvons un intervalle autour de zéro tel que partout dans l'intervalle, $x\leq \epsilon$. Bon ben c'est clair que $\delta=\epsilon$ fonctionne.

Plus compliqué maintenant, mais toujours sans surprises.

\begin{proposition}
\[
  \lim_{x\to 0}x^2=0.
\]

\end{proposition}

\begin{proof}
Soit $\epsilon>0$. On veut un intervalle de largeur $\delta$ autour de zéro tel que $x^2$ soit plus petit que $\epsilon$ sur cet intervalle. Cette fois-ci, le $\delta$ qui fonctionne est $\delta=\sqrt{\epsilon}$. En effet un élément de l'intervalle $[-\delta,\delta]$ est un $r$ de valeur absolue plus petite ou égale à $\delta$ :
\[
| r |\leq\delta=\sqrt{\epsilon}.
\]
En prenant le carré de cette inégalité on a :
\[
  r^2\leq\epsilon,
\]
ce qu'il fallait prouver.
\end{proof}

Calculer et prouver des valeurs de limites, mêmes très simples, devient vite de l'arrachage de cheveux à essayer de trouver le bon $\delta$ en fonction de $\epsilon$ si on n'a pas quelques théorèmes généraux. Heureusement nous en avons déjà quelque uns : \ref{PropLimEstLineraure}, \ref{PROPooDQFIooMMwxxJ}, \ref{ThoLimLinMul}, \ref{ThoLimLin}, \ref{PROPooOUPNooTrClHw}.

\begin{lemma}       \label{LemLimMajorableVois}
    Si $\lim_{x\to a}f(x)=b$ avec $a$, $b\in\eR$, alors il existe un $\delta>0$ et un $M>0$ tels que
    \[
        (| x-a |\leq\delta)\Rightarrow | f(x) |\leq M.
    \]
\end{lemma}

Ce que signifie ce lemme, c'est que quand la fonction $f$ admet une limite finie en un point, alors il est possible de majorer la fonction sur un intervalle autour du point.

\begin{proof}
    Cela va être démontré par l'absurde. Supposons qu'il n'existe pas de $\delta$ ni de $M$ qui vérifient la condition. Dans ce cas, pour tout $\delta$ et pour tout $M$, il existe un $x$ tel que $| x-a |\leq\delta$ et $| f(x) |> M$. Cela est valable pour tout $M$, donc prenons par exemple $b+1000$. Donc
    \begin{equation}
    \forall\delta>0,\exists x\text{ tel que } | x-a |\leq\delta\text{ et }| f(x) |>b+1000.
    \end{equation}
    Cela signifie qu'aucun $\delta$ ne peut convenir dans la définition de $\lim_{x\to a}f(x)=b$, ce qui contredit les hypothèses.
\end{proof}

Dans le même ordre d'idée, on peut prouver que si la limite de la fonction en un point est positive, alors elle est positive autour ce ce point. Plus précisément, nous avons la
\begin{proposition} \label{PropoLimPosFPos}
    Si $f$ est une fonction telle que $\lim_{x\to a}f(x)>0$, alors il existe un voisinage de $a$ sur lequel $f$ est positive.
\end{proposition}

\begin{proof}
    Supposons que $\lim_{x\to a}f(x)=y_0$. Par la définition de la limite fait que si pour tout $x$ dans un voisinage autour de $a$, on ait $| f(x)-a |<\epsilon$. Cela est valable pour tout $\epsilon$, pourvu que le voisinage soit assez petit. Si je choisis un voisinage pour lequel $| f(x)-a |<\frac{ y_0 }{ 2 }$, alors sur ce voisinage, $f$ est positive.
\end{proof}

\begin{proposition}[\cite{MonCerveau}]      \label{PROPooWXBAooAEweSF}
    Soit \( f\colon \eR^2\to \eR\) une application continue dont la variable \( y\) varie dans un compact \( I\) de \( \eR\). Alors la fonction
    \begin{equation}
        \begin{aligned}
            d\colon \eR&\to \eR \\
            x&\mapsto \sup_{y\in I} f(x,y)
        \end{aligned}
    \end{equation}
    est continue.
\end{proposition}

\begin{proof}
    Soit \( x_0\) fixé. Prouvons que \( d\) est continue en \( x_0\). Nous notons \( y_0\) la valeur de \( y\) qui réalise le maximum (par le théorème~\ref{ThoMKKooAbHaro} et le fait que les fonctions projection soient continues, lemme~\ref{LEMooHAODooYSPmvH}). Soit aussi \( \epsilon>0\) tellement fixé que même avec un tourne vis hydraulique, il ne bougerait pas. Nous considérons \( \delta\) tel que si \( \| (x,y)-(x_0,y_0) \|\leq \delta\) alors \( \| f(x,y)-f(x_0,y_0) \|<\epsilon\).

    Si \( | x-x_0 |<\delta\) alors pour \( y\) assez proche de \( y_0\) nous avons \( \| (x,y)-(x_0,y_0) \|\leq \delta\), et donc \( \| f(x,y)-f(x_0,y_0) \|\leq \epsilon \). Cela montre qu'il existe \( \delta\) tel que \( | x-x_0 |\leq \delta\) implique \( d(x)\geq d(x_0)-\epsilon\).

    Nous devons encore trouver un \( \delta\) tel que si \( | x-x_0 |\leq \delta\) alors \( d(x)\leq d(x_0)+\epsilon\). Supposons que non. Alors pour tout \( \delta\) il existe un \( x\) tel que \( | x-x_0 |\leq \delta\) et \( d(x)> d(x_0)+\epsilon\). Cela nous donne une suite \( x_i\to x_0\).

    Pour chaque \( x_i\) nous notons \( y_i\) la valeur de \( y\) qui réalise le supremum correspondant. La suite \( (y_i)\) étant contenue dans un compact nous supposons prendre une sous-suite de \( (x_i)\) telle que la suite \( (y_i)\) converge. Nous nommons \( a\) la limite (et non \( y_0\) parce que nous ne savons pas si \( y_i\to y_0\)). Pour chaque \( i\) nous avons
    \begin{equation}
        f(x_i,y_i)>\sup_{y\in I}f(x_0,y)+\epsilon.
    \end{equation}
    En prenant la limite et en utilisant la continuité de \( f\),
    \begin{equation}
        f(x_0,a)>\sup_{y\in I} f(x_0,y)+\epsilon,
    \end{equation}
    ce qui est impossible.
\end{proof}

%---------------------------------------------------------------------------------------------------------------------------
\subsection{Limites de fonctions}
%---------------------------------------------------------------------------------------------------------------------------

\begin{definition}\label{def_limite}
	Soit $f\colon D\subset\eR^m\to \eR$ une fonction et $a$ un point d'accumulation de $D$.  On dit que $f$ possède une \defe{limite}{limite!fonction de plusieurs variables} s'il existe un élément $\ell\in\eR$ tel que
	\begin{equation}		\label{Eq2807CondiionLimifnm}
		\forall\varepsilon>0,\,\exists\delta>0\tq 0<\| x-a \|<\delta\Rightarrow | f(x)-\ell |<\varepsilon.
	\end{equation}

	Pour une fonction $f\colon D\subset\eR^m\to \eR^n$, la définition est la même, sauf que nous remplaçons la valeur absolue par la norme dans $\eR^n$. Nous disons donc que $\ell$ est la limite de $f$ lorsque $x$ tend vers $a$, et nous notons $\lim_{x\to a} f(x)=\ell$ lorsque pour tout $\varepsilon>0$, il existe un $\delta>0$ tel que
	\begin{equation}		\label{EqDefLimRpRn}
		0<\| x-a \|_{\eR^m}<\delta\Rightarrow\,\| f(x)-\ell \|_{\eR^n}<\varepsilon.
	\end{equation}
\end{definition}

\begin{remark}
	Dans l'équation \eqref{EqDefLimRpRn}, nous avons explicitement écrit les normes $\| . \|_{\eR^m}$ et $\| . \|_{\eR^n}$. Dans la suite nous allons le plus souvent noter $\| . \|$ sans plus de précision. Il est important de faire l'exercice de bien comprendre à chaque fois de quelle norme nous parlons.
\end{remark}

\begin{remark}
	Il est important de remarquer à quel point les définitions~\ref{def_limite}, et les caractérisations~\ref{PropHOCWooSzrMjl},~\ref{PropAJQQooQQClfp} sont analogues. En réalité, la définition fondamentale est la définition de la limite dans les espaces vectoriels normés; les deux autres sont des cas particuliers, adaptés à $\eR$ et $\eR^m$. Il en sera de même pour les définitions de fonctions continues : il y aura une définition pour la continuité de fonctions entre espaces vectoriels normés, et ensuite une définition pour les fonctions de $\eR^m$ dans $\eR^n$ qui en sera un cas particulier.
\end{remark}

Tentons de comprendre ce que signifie qu'un nombre $\ell$ \emph{ne soit pas} la limite de $f$ lorsque $x\to a$. Il s'agit d'inverser la condition \eqref{Eq2807CondiionLimifnm}. Le nombre $\ell$ n'est pas une limite de $f$ pour $x\to a$ lorsque
\begin{equation}		\label{EqCaractNonLim}
	\exists\varepsilon>0\tq\,\forall\delta>0,\,\exists x\tq 0<\| x-a \|<\delta\text{ et }\| f(x)-\ell \|>\varepsilon,
\end{equation}
c'est à dire qu'il existe un certain seuil $\varepsilon$ tel qu'on a beau s'approcher aussi proche qu'on veut de $a$ (distance $\delta$), on trouvera toujours un $x$ tel que $f(x)$ n'est pas $\varepsilon$-proche de $\ell$.

\begin{lemma}[Unicité de la limite]
	Si $\ell$ et $\ell'$ sont deux limites de $f(x)$ lorsque $x$ tend vers $a$, alors $\ell=\ell'$.
\end{lemma}

\begin{proof}
	Soit $\varepsilon>0$. Nous considérons $\delta$ tel que $\| f(x)-\ell \|<\varepsilon$ pour tout $x$ tel que $\| x-a \|<\delta$. De la même manière, nous prenons $\delta'$ tel que $\| x-a \|<\delta'$ implique $\| f(x)-\ell' \|<\varepsilon$. Pour les $x$ tels que $\| x-a \|$ est plus petit que $\delta$ et $\delta'$ en même temps, nous avons
	\begin{equation}
		\| \ell-\ell' \|=\| \ell-f(x)+f(x)-\ell' \|\leq\| \ell-f(x) \|+\| f(x)-\ell' \|<2\varepsilon,
	\end{equation}
	et donc $\| \ell-\ell' \|=0$ parce que c'est plus petit que $2\varepsilon$ pour tout $\varepsilon$.
\end{proof}

%--------------------------------------------------------------------------------------------------------------------------- 
\subsection{Limite à gauche et à droite}
%---------------------------------------------------------------------------------------------------------------------------

Si \( a\) est à l'intérieur du domaine de \( f\), nous savons ce que signifie \( \lim_{x\to a} f(x)\). Nous donnons également une définition des limites à gauche et à droite.

\begin{definition}
    Soient \( D\subset \eR\) et une fonction \( f\colon D\to \eR\). Si \( a\in \Adh(D)\) nous définissons la \defe{limite à droite}{limite à droite} de \( f\) en \( a\) par
    \begin{equation}        \label{EQooQKHLooMoSXVe}
        \lim_{x\to a^+} f(x)=\lim_{x\to a} \tilde f(x)
    \end{equation}
    où \( \tilde f\) est la fonction \( f\) restreinte à \( D\cap\{ x\tq x>a \}\). La limite \eqref{EQooQKHLooMoSXVe} est souvent écrite sous la forme condensée
    \begin{equation}
        \lim_{\substack{x\to a\\x>a}}f(x).
    \end{equation}

    Pour la limite à gauche c'est un peu la même chose :
    \begin{equation}
        \lim_{x\to a^-} f(x)=\lim_{\substack{x\to a\\x<a}}f(x).
    \end{equation}
\end{definition}

\begin{proposition}[\cite{ooOMWZooZvUFiG}]      \label{PROPooGDDJooDCmydE}
    Soit une fonction \( f\colon D\to \eR\) où \( D\) est une partie de \( \eR\). Si \( a\in \Adh(D)\) alors la limite \( \lim_{x\to a} f(x)\) existe si et seulement si les limites à gauche et à droite existent et sont égales. Dans ce cas nous avons égalité :
    \begin{equation}
        \lim_{x\to a} f(x)=\lim_{x\to a^+} f(x)=\lim_{x\to a^-} f(x).
    \end{equation}
\end{proposition}

\begin{normaltext}
    Quelque remarques à propos de la proposition \ref{PROPooGDDJooDCmydE}.
    \begin{enumerate}
        \item
            
    Cette proposition ne se généralise pas du tout aux dimensions supérieures. Dans \( \eR^2\) par exemple, il ne faudrait pas croire que si les limites suivant toutes les directions existent alors la limite existe.
\item
    Cette proposition est souvent utilisée pour calculer des limites dans lesquelles arrivent des valeurs absolues. Par exemple durant la démonstration de la proposition \ref{PROPooCNDHooKRwils}.
    \end{enumerate}
\end{normaltext}

%+++++++++++++++++++++++++++++++++++++++++++++++++++++++++++++++++++++++++++++++++++++++++++++++++++++++++++++++++++++++++++ 
\section{Limite en compactifié d'Alexandrov}
%+++++++++++++++++++++++++++++++++++++++++++++++++++++++++++++++++++++++++++++++++++++++++++++++++++++++++++++++++++++++++++

Nous considérons l'espace topologique localement compact \( \eR\), et son compactifié d'Alexandrov défini en \ref{PROPooHNOZooPSzKIN}. Nous avons donc un point supplémentaire noté \( \infty\). Ce point n'est ni du côté des grands nombres positifs, ni du côté des grands nombres négatifs. Il n'est ni \( +\infty\) ni \( -\infty\).

\begin{proposition}
    Dans cet espace topologique \( \tilde \eR=\eR\cup\{ \infty \}\),
    \begin{equation}
        \lim_{x\to 0} \frac{1}{ x }=\infty.
    \end{equation}
\end{proposition}

\begin{proof}
    Soit un voisinage \( V\) de \( \infty\) dans \( \tilde \eR\). Il s'écrit \( V=K^c\cup\{ \infty \}\) pour un certain compact de \( \eR\). Le théorème \ref{ThoXTEooxFmdI} nous assure que \( K\) est borné. Donc il existe \( R>0\) tel que \( K\subset B(0,R)\). Pour \( x\in B(0,1/R)\) nous avons
    \begin{equation}
        | \frac{1}{ x } |>R,
    \end{equation}
    et donc \( 1/x\in K^c\). Donc aussi \( \frac{1}{ x }\in V\).
\end{proof}

De la même façon, dans \( \eC\cup\{ \infty \}\) nous avons
\begin{equation}
    \lim_{z\to 0} \frac{1}{ z }=\infty.
\end{equation}

\begin{normaltext}
    Je vous laisse deviner la topologie à considérer sur \( \bar \eR=\eR\cup\{ +\infty,-\infty \}\). Dans cet espace topologique la limite \( \lim_{x\to 0} \frac{1}{ x }\) n'existe pas.
\end{normaltext}

%+++++++++++++++++++++++++++++++++++++++++++++++++++++++++++++++++++++++++++++++++++++++++++++++++++++++++++++++++++++++++++
\section{Continuité}
%+++++++++++++++++++++++++++++++++++++++++++++++++++++++++++++++++++++++++++++++++++++++++++++++++++++++++++++++++++++++++++

\begin{normaltext}
    Nous allons considérer trois approches différentes de la continuité. La première sera de définir la continuité de fonctions de $\eR$ vers $\eR$ au moyen du critère usuel. Ensuite, nous définiront la continuité des applications entre n'importe quels espaces métriques, et nous montrerons que les deux définitions sont équivalentes dans le cas des fonctions sur $\eR$ à valeurs réelles.

    Enfin, un peu plus tard nous verrons que la continuité peut également être vue en termes de limites. Encore une fois nous verrons que dans le cas de fonctions de $\eR$ vers $\eR$ cette troisième approche est équivalentes aux deux premières.
\end{normaltext}

La définition de fonction continue est la définition~\ref{DefOLNtrxB}. Dans le cas d'une fonction \( f\colon \eR\to \eR\), elle devient ceci.
\begin{proposition}      \label{PROPooVNGEooPwbxXP}
    La fonction \( f\colon \eR\to \eR\) est \defe{continue en $a$}{continue sur \( \eR\)} si et seulement si
    \begin{equation}
        \forall \epsilon>0,\exists \delta\text{ tel que } \big(| x-a |\leq\delta\big)\Rightarrow | f(x)-f(a) |\leq \epsilon.
    \end{equation}
\end{proposition}

Nous allons maintenant étudier quelques conséquences de la continuité sur \( \eR\).

\begin{enumerate}
\item D'abord on voit que la continuité n'a été définie qu'en un point. On peut dire que la fonction $f$ est continue \emph{en tel point donné}, mais nous n'avons pas dit ce qu'est une fonction continue \emph{dans son ensemble}.

\item
    Le théorème \ref{ThoESCaraB} nous précise que si $I$ est un intervalle de $\eR$, la fonction $f$ est continue sur $I$ si et seulement si elle est continue en chaque point de $I$.

\item Comme la définition de $f$ continue en $a$ fait intervenir $f(x)$ pour tous les $x$ pas trop loin de $a$, il faut au moins déjà que $f$ soit définie sur ces $x$. En d'autres termes, dire que $f$ est continue en $a$ demande que $f$ existe sur un intervalle autour de $a$.

Ceci couplé à la définition précédente laisse penser qu'il est surtout intéressant d'étudier les fonctions qui sont continues sur un intervalle.

\item L'intuition comme quoi une fonction continue doit pouvoir être tracée sans lever la main correspond aux fonctions continues sur des intervalles. Au moins sur l'intervalle où elle est continue, elle est traçable en un morceau.
\end{enumerate}

\begin{example}
    Il est très possible d'être continue en un seul point. Par exemple la fonction
    \begin{equation}
        f(x)=x(1-\mtu_{\eQ}(x))
    \end{equation}
    où \( \mtu_{\eQ}\) est la fonction indicatrice de \( \eQ\) dans \( \eR\).
\end{example}

\begin{proposition}     \label{PROPooUBUAooNIxjfg}
    Si \( f\colon \eR\to \eR\) est continue au point \( a\in \eR\) et si \( f(a)\neq 0\), alors il existe un voisinage de \( a\) sur lequel \( f\) ne s'annule pas.
\end{proposition}

\begin{proof}
    Si \( f \) s'annulait sur tout voisinage de \( a\) (mais pas ne \( a\) lui-même), nous aurions, pour tout \( n\) un réel
    \begin{equation}
        x_n\in B\big( a,\frac{1}{ n } \big)\setminus\{ a \}
    \end{equation}
    tel que \( f(x_n)=0\). Cela donnerait une suite \( x_n\to a\) avec \( f(x_n)\to 0\), ce qui contredit la continuité de \( f\) en \( a\) en vertu de la proposition \ref{PropFnContParSuite}\ref{ItemWJHIooMdugfu} sur la continuité séquentielle en un point.
\end{proof}

Notons que ce résultat se généralise beaucoup : si \( f\) est continue et pas égale à \( r\) en \( a\), alors elle continue à n'être pas égale à \( r\) dans un voisinage de \( a\).

%--------------------------------------------------------------------------------------------------------------------------- 
\subsection{Opération sur la continuité}
%---------------------------------------------------------------------------------------------------------------------------

Nous allons démontrer maintenant une série de petits résultats qui permettent de simplifier la démonstration de la continuité de fonctions.
\begin{theorem}
Si la fonction $f$ est continue au point $a$, alors la fonction $\lambda f$ est également continue en $a$.
\end{theorem}

\begin{proof}
Soit $\epsilon>0$. Nous avons besoin d'un $\delta>0$ tel que pour chaque $x$ à moins de $\delta$ de $a$, la fonction $\lambda f$ soit à moins de $\epsilon$ de $(\lambda f)(a)=\lambda f(a)$. Étant donné que la fonction $f$ est continue en $a$, on sait déjà qu'il existe un $\delta_1$ (nous notons $\delta_1$ afin de ne pas confondre ce nombre dont on est sûr de l'existence avec le $\delta$ que nous sommes en train de chercher) tel que
\[
  (| x-a |\leq \delta_1)\Rightarrow | f(x)-f(a) |\leq \epsilon_1.
\]
Hélas, ce $\delta_1$ n'est pas celui qu'il faut faut parce que nous travaillons avec $\lambda f$ au lieu de $f$, ce qui fait qu'au lieu d'avoir $| f(x)-f(a) |$, nous avons $| \lambda f(x)-\lambda f(a) |=| \lambda |\cdot | f(x)-f(a) |$.  Ce que $\delta_1$ fait avec $(\lambda f)$, c'est
\[
  (| x-a |\leq\delta_1)\Rightarrow  | (\lambda f)(x)- (\lambda f)(a)|\leq | \lambda |\epsilon_1.
\]
Ce que nous apprend la continuité de $f$, c'est que pour chaque choix de $\epsilon_1$, on a un $\delta_1$ qui fait cette implication. Comme cela est vrai pour chaque choix de $\epsilon_1$, essayons avec $\epsilon_1=\epsilon/| \lambda |$ pour voir ce que ça donne. Nous avons donc un $\delta_1$ qui fait
\[
  (| x-a |\leq\delta_1)\Rightarrow  | (\lambda f)(x)- (\lambda f)(a)|\leq | \lambda |\epsilon_1=\epsilon.
\]
Ce $\delta_1$ est celui qu'on cherchait.
\end{proof}

\begin{theorem}
Si $f$ et $g$ sont deux fonctions continues en $a$, alors la fonction $f+g$ est également continue en $a$.
\end{theorem}

\begin{proof}
La continuité des fonctions $f$ et $g$ au point $a$ fait en sorte que pour tout choix de $\epsilon_1$ et $\epsilon_2$, il existe $\delta_1$ et $\delta_2$ tels que
\[
  (| x-a |\leq \delta_1)\Rightarrow | f(x)-f(a) |\leq \epsilon_1.
\]
et
\[
  (| x-a |\leq \delta_2)\Rightarrow | g(x)-g(a) |\leq \epsilon_2.
\]
La quantité que nous souhaitons analyser est $| f(x)+g(x)-f(a)-g(a) |$. Tout le jeu de la démonstration de la continuité est de triturer cette expression pour en tirer quelque chose en termes de $\epsilon_1$ et $\epsilon_2$. Si nous supposons avoir pris $| x-a |$ plus petit en même temps que $\delta_1$ et que $\delta_2$, nous avons
\[
| f(x)+g(x)-f(a)-g(a) |\leq| f(x)-g(x) |+| g(x)-g(a) |\leq\epsilon_1+\epsilon_2
\]
en utilisant la formule générale $| a+b |\leq | a |+| b |$. Maintenant si on choisit $\epsilon_1$ et $\epsilon_2$ tels que $\epsilon_1+\epsilon_2<\epsilon$, et les $\delta_1$, $\delta_2$ correspondants, on a
\[
| f(x)+g(x)-f(a)-g(a) |\leq\epsilon,
\]
pourvu que $| x-a |$ soit plus petit que $\delta_1$ et $\delta_2$. Le bon $\delta$ à prendre est donc le minimum de $\delta_1$ et $\delta_2$ qui eux-mêmes sont donnés par un choix de $\epsilon_1$ et $\epsilon_2$ tels que $\epsilon_1+\epsilon_2\leq\epsilon$.
\end{proof}

Pour résumer ces deux théorèmes, on dit que si $f$ et $g$ sont continues en $a$, alors la fonction $\alpha f+\beta g$ est également continue en $a$ pour tout $\alpha$, $\beta\in\eR$.

Parmi les propriétés immédiates de la continuité d'une fonction, nous avons ceci qui est souvent bien utile.

\begin{corollary}   \label{CorNNPYooMbaYZg}
Si la fonction $f$ est continue en $a$ et si $f(a)>0$, alors $f$ est positive sur un intervalle autour de $a$.
\end{corollary}

\begin{proof}
Prenons $\epsilon<f(a)$ et voyons\footnote{ici, nous insistons sur le fait que nous prenons $\epsilon$ \emph{strictement} plus petit que $f(a)$.} ce que la continuité de $f$ en $a$ nous offre : il existe un $\delta$ tel que
\[
  (| x-a |\leq \delta)\Rightarrow | f(x)-f(a) |\leq\epsilon < f(a).
\]
Nous en retenons que sur un intervalle (de largeur $\delta$), nous avons $| f(x)-f(a) |\leq f(a)$. Par hypothèse, $f(a)>0$, donc si $f(x)<0$, alors la différence $f(x)-f(a)$ donne un nombre encore plus négatif que $-f(a)$, c'est à dire que $| f(x)-f(a) |>f(a)$, ce qui est contraire à ce que nous venons de démontrer. D'où la conclusion que $f(x)>0$.
\end{proof}

\subsection{La fonction la moins continue du monde}
%--------------------------------------------------

Parmi les exemples un peu sales de fonctions non continues, il y a celle-ci :
\[
  \chi_{\eQ}(x)=
\begin{cases}
    1 \text{ si }x\in\eQ\\
    0 \text{ sinon.}
\end{cases}
\]
Par exemple, $\chi_{\eQ}(0)=1$, et\footnote{Pour prouver que $\sqrt{2}$ n'est pas rationnel, c'est pas trop compliqué, mais pour prouver que $\pi$ ne l'est pas non plus, il faudra encore manger de la soupe.} $\chi_{\eQ}(\pi)=\chi_{\eQ}(\sqrt{2})=0$. Malgré que $\chi_{\eQ}(0)=1$, il n'existe \emph{aucun} voisinage de $1$ sur lequel la fonction reste proche de $1$, parce que tout voisinage va contenir au moins un irrationnel. À chaque millimètre, cette fonction fait une infinité de bonds !

Cette fonction n'est donc continue nulle part.

À partir de là, nous pouvons construire la fonction suivante qui n'est continue qu'en un point :
\[
  f(x)=x\chi_{\eQ}(x)=
\begin{cases}
x\text{ si }x\in\eQ\\
0\text{ sinon.}
\end{cases}
\]
Cette fonction est continue en zéro. En effet, prenons $\delta>0$; il nous faut un $\epsilon$ tel que $| x |\leq\epsilon$ implique $f(x)\leq \delta$ parce que $f(0)=0$. Bon ben prendre simplement $\epsilon=\delta$ nous contente. Cette fonction est donc très facilement continue en zéro.

Et pourtant, dès que l'on s'écarte un tant soit peu de zéro, elle fait des bons une infinité de fois par millionième de millimètre ! Cette fonction est donc la plus discontinue du monde en tous les points saut un (zéro) où elle est une fonction continue !

\subsection{Approche topologique}
%--------------------------------

Nous avons vu que sur tout ensemble métrique, nous pouvons définir ce qu'est un ouvert : c'est un ensemble qui contient une boule ouverte autour de chacun de ses points. Quand on est dans un ensemble ouvert, on peut toujours un peu se déplacer sans sortir de l'ensemble.

Le théorème suivant est une très importante caractérisation des fonctions continues (de $\eR$ dans $\eR$) en termes de topologie, c'est à dire en termes d'ouverts.

\begin{theorem}     \label{ThoContInvOuvert}
Si $I$ est un intervalle ouvert contenu dans $\dom f$, alors $f$ est continue sur $I$ si et seulement si pour tout ouvert $\mO$ dans $\eR$, l'image inverse $f|_I^{^{-1}}(\mO)$ est ouvert.
\end{theorem}

Par abus de langage, nous exprimons souvent cette condition par « une fonction est continue si et seulement si l'image inverse de tout ouvert est un ouvert ».

\begin{proof}

Dans un premier temps, nous allons transformer le critère de continuité en termes de boules ouvertes, et ensuite, nous passerons à la démonstration proprement dite. Le critère de continuité de $f$ au point $x$ dit que
\begin{equation}        \label{EqDEfCOntAn}
  \forall \delta>0,\exists\,\epsilon>0\text{ tel que }\big( | x-a |< \epsilon \big)\Rightarrow| f(x)-f(a) |<\delta.
\end{equation}
Cette condition peut être exprimée sous la forme suivante :
\[
  \forall \delta>0,\exists\epsilon\text{ tel que } a\in B(x,\epsilon)\Rightarrow f(a)\in B\big( f(x),\delta \big),
\]
ou encore
\begin{equation}        \label{EqRedefContBoules}
  \forall \delta>0,\exists\epsilon\text{ tel que } f\big( B(x,\epsilon) \big)\subset B\big( f(x),\delta \big).
\end{equation}
Jusque ici, nous n'avons fait que du jeu de notations. Nous avons exprimé en termes de topologie des inégalités analytiques. La condition \eqref{EqRedefContBoules} est le plus souvent utilisée comme définition de la continuité d'une fonction en \( x\), lorsque le contexte ne demande pas de définitions plus générales. Si tel est le choix, il faut pouvoir retrouver \eqref{EqDEfCOntAn} à partir de \eqref{EqRedefContBoules}.

Passons maintenant à la démonstration proprement dite du théorème.

D'abord, supposons que $f$ est continue sur $I$, et prenons $\mO$, un ouvert quelconque. Le but est de prouver que $f|_I^{-1}(\mO)$ est ouvert. Pour cela, nous prenons un point $x_0\in f|_I^{-1}(\mO)$ et nous allons trouver un ouvert autour ce point contenu dans $f|_I^{-1}(\mO)$. Nous écrivons $y_0=f(x_0)$. évidemment, $y_0\in\mO$, donc on a une boule autour de $y_0$ qui est contenue dans $\mO$, soit donc $\delta>0$ tel que
\[
  B(y_0,\delta)\subset\mO.
\]
Par hypothèse, $f$ est continue en $x_0$, et nous pouvons donc y appliquer le critère \eqref{EqRedefContBoules}. Il existe donc $\epsilon>0$ tel que
\[
  f\big( B(x_0,\epsilon) \big)\subset B\big( f(x_0),\delta \big)\subset\mO.
\]
Cela prouve que $B(x_0,\epsilon)\subset f|_I^{-1}(\mO)$.

Dans l'autre sens, maintenant. Nous prenons $x_0\in I$ et nous voulons prouver que $f$ est continue en $x_0$, c'est à dire que pour tout $\delta$ nous cherchons un $\epsilon$ tel que $f\big( B(x_0,\epsilon) \big)\subset B\big( f(x_0),\delta \big)$. Oui, mais $B\big( f(x_0),\delta \big)$ est ouverte, donc par hypothèse, $f|_I^{-1}\Big( B\big( f(x_0),\delta \big) \Big)$ est ouvert, inclus dans $I$ et contient $x_0$. Donc il existe un $\epsilon$ tel que
\[
  B(x_0,\epsilon)\subset f|_I^{-1}\Big( B\big( f(x_0),\delta \big) \Big),
\]
et donc tel que
\[
  f\big( B(x_0,\epsilon) \big)\subset B\big( f(x_0),\delta \big),
\]
ce qu'il fallait prouver.
\end{proof}

\begin{lemma}   \label{LemConncontconn}
L'image d'un ensemble connexe par une fonction continue est connexe.
\end{lemma}

\begin{proof}
Nous allons encore faire la contraposée. Soit $A$ une partie de $\eR$ telle que $f(A)$ ne soit pas connexe. Nous allons prouver que $A$ elle-même n'est pas connexe. Dire que $f(A)$ n'est pas connexe, c'est dire qu'il existe $\mO_1$ et $\mO_2$, deux ouverts disjoints qui recouvrent $f(A)$. Je prétends que $f^{-1}(\mO_1)$ et $f^{-1}(\mO_2)$ sont ouverts, disjoints et qu'ils recouvrent $A$.
\begin{itemize}
\item Ces deux ensembles sont ouverts parce qu'ils sont images inverses d'ouverts par une fonction continue (théorème~\ref{ThoContInvOuvert}).
\item Si $x\in f^{-1}(\mO_1)\cap f^{-1}(\mO_2)$, alors $f(x)\in \mO_1\cap\mO_2$, ce qui contredirait le fait que $\mO_1$ et $\mO_2$ sont disjoints. Il n'y a donc pas d'éléments dans l'intersection de $f^{-1}(\mO_1)$ et de $f^{-1}(\mO_2)$.
\item Si $f^{-1}(\mO_1)$ et $f^{-1}(\mO_2)$ ne recouvrent pas $A$, il existe un $x$ dans $A$ qui n'est dans aucun des deux. Dans ce cas, $f(x)$ est dans $f(A)$, mais n'est ni dans $\mO_1$, ni dans $\mO_2$, ce qui contredirait le fait que ces deux derniers recouvrent $f(A)$.
\end{itemize}
Nous déduisons que $A$ n'est pas connexe. Et donc le lemme.
\end{proof}

\begin{theorem}[Théorème des valeurs intermédiaires]        \label{ThoValInter}
Soit $f$, une fonction continue sur $[a,b]$, et supposons que $f(a)<f(b)$. Alors pour tout $y$ tel que $f(a)\leq y\leq f(b)$, il existe un $x$ entre $a$ et $b$ tel que $f(x)=y$.
\end{theorem}
\index{connexité!théorème des valeurs intermédiaires}
\index{théorème!valeurs intermédiaires}

\begin{proof}
Nous savons que $[a,b]$ est connexe parce que c'est un intervalle (proposition~\ref{PropInterssiConn}). Donc $f\big( [a,b] \big)$ est connexe (lemme~\ref{LemConncontconn}) et donc est un intervalle (à nouveau la proposition~\ref{PropInterssiConn}). Étant donné que $f\big( [a,b] \big)$ est un intervalle, il contient toutes les valeurs intermédiaires entre n'importe quels deux de ses éléments. En particulier toutes les valeurs intermédiaires entre $f(a)$ et $f(b)$.
\end{proof}

\begin{corollary}       \label{CorImInterInter}
L'image d'un intervalle par une fonction continue est un intervalle.
\end{corollary}

\begin{proof}
Soient \( I\) un intervalle, \( \alpha<\beta\in f(I)\) et \( \gamma\in\mathopen] \alpha , \beta \mathclose[\). Nous considérons \(a,b\in I\) tels que \( \alpha=f(a)\) et \( \beta=f(b)\). Par le théorème des valeurs intermédiaires\ref{ThoValInter}, il existe \( t\in\mathopen] a , b \mathclose[\) tel que \( f(t)=\gamma\). Par conséquent \( \gamma\in f(I)\).
\end{proof}

\begin{corollaryDef}[Existence de la racine carrée]
    Si \( x\geq 0\) alors il existe un unique \( y\geq 0\) tel que \( y^2=x\). Ce nombre est noté \( \sqrt{x}\) et est nommé \defe{racine carrée}{racine carrée} de \( x\).
\end{corollaryDef}

\begin{proof}
    La fonction \( f\colon t\mapsto t^2\) est continue et strictement croissante. Nous avons \( f(0)=0\) et\footnote{Faites deux cas suivant \( x\geq 1\) ou non si vous le voulez, moi je prends \( x+1\).} \( f(x+1)>x\). Donc le théorème des valeurs intermédiaires~\ref{ThoValInter} nous assure qu'il existe un unique \( y\in\mathopen[ 0 , x+1 \mathclose]\) tel que \( f(y)=x\).
\end{proof}

Nous avons déjà vu dans la proposition~\ref{PropooRJMSooPrdeJb} que \( \sqrt{2}\) était irrationnel. En fait le théorème suivant va nous montrer que le nombre \( \sqrt{ n }\) est soit entier, soit irrationnel.
\begin{theorem}     \label{THOooYXJIooWcbnbm}
    Soit \( n\in \eN\). Le nombre \( \sqrt{n}\) est rationnel si et seulement si \( n\) est un carré parfait.
\end{theorem}

\begin{proof}
    Supposons que \( \sqrt{n}\) soit rationnel. Le théorème~\ref{THOooWYQVooRBaAAM} nous donne \( p,q\in \eN\) premiers entre eux tels que \( \sqrt{n}=p/q\). La proposition~\ref{PROPooRZDDooLJabov} nous enseigne de plus que \( p^2\) et \( q^2\) sont premiers entre eux. Nous avons
    \begin{equation}
        p^2=nq^2.
    \end{equation}
    Le nombre $q$ est alors un diviseur commun de \( q^2\) et de \( p\). Donc \( q=1\) et \( n=p^2\) est un carré parfait.
\end{proof}

\subsection{Continuité de la racine carrée, invitation à la topologie induite}
%-----------------------------------------

Pourquoi nous intéresser particulièrement à cette fonction ? Parce qu'elle a une sale condition d'existence : son domaine de définition n'est pas ouvert. Or dans tous les théorèmes de continuité d'approche topologique que nous avons vus, nous avons donné des conditions \emph{pour tout ouvert}. Nous nous attendons donc a avoir des difficultés avec la continuité de $\sqrt{x}$ en zéro.

Prenons $I$, n'importe quel intervalle ouvert dans $\eR^+$, et voyons que la fonction
\begin{equation}
\begin{aligned}
 f\colon \eR^+&\to \eR^+ \\
   x&\mapsto \sqrt{x}
\end{aligned}
\end{equation}
est continue sur $I$. Remarque déjà que si $I$ est un ouvert dans $\eR^+$, il ne peut pas contenir zéro. Avant de nous lancer dans notre propos, nous prouvons un lemme qui fera tout le travail\footnote{C'est toujours ingrat d'être un lemme : on fait tout le travail et c'est toujours le théorème qui est nommé.}.

\begin{lemma}
Soit $\mO$, un ouvert dans $\eR^+$. Alors $\mO^2=\{ x^2\tq x\in\mO \}$ est également ouvert .
\end{lemma}

\begin{proof}
Un élément de $\mO^2$ s'écrit sous la forme $x^2$ pour un certain $x\in\mO$. Le but est de trouver un ouvert autour de $x^2$ qui soit contenu dans $\mO^2$. Étant donné que $\mO$ est ouvert, on a une boule centrée en $x$ contenue dans $\mO$. Nous appelons $\delta$ le rayon de cette boule :
\[
  B(x,\delta)\subset\mO.
\]
Étant donné que cet ensemble est connexe, nous savons par le lemme~\ref{LemConncontconn} que $B(x,\delta)^2$ est également connexe (parce que la fonction $x\mapsto x^2$ est continue). Son plus grand élément est $(x+\delta)^2=x^2+\delta^2+2x\delta>x^2+\delta^2$, et son plus petit élément est $(x-\delta)^2=x^2+\delta^2-2x\delta$.

Ce qui serait pas mal, c'est que ces deux bornes entourent $x^2$; de cette façon elles définiraient un ouvert autour de $x^2$ qui soit dans $\mO^2$. Hélas, c'est pas gagné que $x^2+\delta^2-2x\delta$ soit plus petit que $x^2$.

Heureusement, en fait c'est vrai parce que d'une part, du fait que $\mO\subset\eR^+$, on a $x>0$, et d'autre part, pour que $\mO$ soit positif, il faut que $\delta<x$. Donc on a évidemment que $\delta<2x$, et donc que
\[
  x^2+\delta^2-2x\delta=x^2+\delta\underbrace{(\delta-2x)}_{<0}<x^2.
\]
Donc nous avons fini : l'ensemble
\[
  B(x,\delta)^2=]x^2+\delta^2-2x\delta,x^2+\delta^2+2x\delta[\subset\mO^2
\]
est un intervalle qui contient $x^2$, et donc qui contient une boule ouverte centrée en~$x^2$.

\end{proof}

Maintenant nous pouvons nous attaquer à la continuité de la racine carrée sur tout ouvert positif en utilisant le théorème~\ref{ThoContInvOuvert}. Soit $\mO$ n'importe quel ouvert de $\eR$, et prouvons que $f|_I^{-1}(\mO)$ est ouvert. Par définition,
\begin{equation}
  f|_I^{-1}(\mO)=\{ x\in I\tq \sqrt{x}\in\mO \}.
\end{equation}
Maintenant c'est un tout petit effort que de remarquer que $f|_I^{-1}(\mO)=\mO^2\cap I$. De là, on a gagné parce que $\mO^2$ et $I$ sont des ouverts. Or l'intersection de deux ouverts est ouvert.

Nous n'en avons pas fini avec la fonction $\sqrt{x}$. Nous avons la continuité de la racine carrée pour tous les réels strictement positifs. Il reste à pouvoir dire que la fonction est continue en zéro malgré qu'elle ne soit pas définie sur un ouvert autour de zéro.

Il est possible de dire que la racine carrée est continue en $0$, malgré qu'elle ne soit pas définie sur un ouvert autour de $0$\ldots en tout cas pas un ouvert au sens que tu as en tête. Nous allons rentabiliser un bon coup notre travail sur les espaces métriques.

Nous pouvons définir la notion de boule ouverte sur n'importe quel espace métrique $A$ en disant que
\[
  B(x,r)=\{ y\in A\tq d(x,y)<r \}.
\]
\begin{definition}      \label{DefContMetrique}
Soit $f\colon A\to B$, une application entre deux espaces métriques. Nous disons que $f$ est \defe{continue}{continue!sur espace métrique} au point $a\in A$ si $\forall \delta>0$, $\exists\epsilon>0$ tel que
\begin{equation}
  f\big( B(a,\epsilon) \big)\subset B\big( f(a),\delta \big).
\end{equation}
\end{definition}
Tu reconnais évidemment la condition \eqref{EqRedefContBoules}. Nous l'avons juste recopiée. Tu remarqueras cependant que cette définition généralise immensément la continuité que l'on avait travaillé à propos des fonctions de $\eR$ vers $\eR$. Maintenant tu peux prendre n'importe quel espace métrique et c'est bon.

Nous n'allons pas faire un tour complet des conséquences et exemples de cette définition. Au lieu de cela, nous allons juste montrer en quoi cette définition règle le problème de la continuité de la racine carrée en zéro.

La fonction que nous regardons est
\begin{equation}
\begin{aligned}
f \colon \eR^+&\to \eR^+ \\
   x&\mapsto \sqrt{x}.
\end{aligned}
\end{equation}
Mais cette fois, nous ne la voyons pas comme étant une fonction dont le domaine est une partie de $\eR$, mais comme fonction dont le domaine est $\eR^+$ vu comme un espace métrique en soi. Quelles sont les boules ouvertes dans $\eR^+$ autour de zéro ? Réponse : la boule ouverte de rayon $r$ autour de zéro dans $\eR^+$ est :
\[
  B(0,r)_{\eR^+}=\{ x\in\eR^+\tq d(x,0)<r \}=[0,r[.
\]
Cet intervalle est un ouvert. Aussi incroyable que cela puisse paraître !

Testons la continuité de la racine carrée en zéro dans ce contexte. Il s'agit de prendre $A=\eR^+$, $B=\eR^+$ et $a=0$ dans la définition~\ref{DefContMetrique}. Nous avons que $B(\sqrt{0},\delta)=B(0,\delta)=[0,\delta[$ pour la topologie de $\eR^+$.

Il s'agit maintenant de trouver un $\epsilon$ tel que $f\big( B(0,\epsilon) \big)\subset [0,\delta[$. Par définition, nous avons que
\[
  f\big( B(0,\epsilon) \big)=[0,\sqrt{\epsilon}[,
\]
le problème revient dont à trouver $\epsilon$ tel que $\sqrt{\epsilon}\leq\delta$. Prendre $\epsilon<\delta^2$ fait l'affaire.


Donc voilà. Au sens de la \href{http://fr.wikipedia.org/wiki/Topologie_induite}{topologie propre} à $\eR^+$, nous pouvons dire que la fonction racine carrée est partout continue.

%---------------------------------------------------------------------------------------------------------------------------
\subsection{Prolongement par continuité}
%---------------------------------------------------------------------------------------------------------------------------

%///////////////////////////////////////////////////////////////////////////////////////////////////////////////////////////
\subsubsection{Discussion avec mon ordinateur}
%///////////////////////////////////////////////////////////////////////////////////////////////////////////////////////////

Voici un extrait de ce peut donner Sage. Nous lui donnons la fonction
\begin{equation}    \label{EqyEHTBZ}
    f(x)=\frac{ x+4 }{ 3x^2+10x-8 }.
\end{equation}
Cette fonction est faite exprès pour que le dénominateur s'annule en \( -4\). En fait \( 3x^2+10x-8=(x+4)(3x-2)\), et la fraction peut se simplifier en
\begin{equation}
    f(x)=\frac{1}{ 3x-2 }.
\end{equation}
Et avec cela nous écririons \( f(-4)=-\frac{1}{ 14 }\). Voyons comment cela passe dans Sage.

\begin{verbatim}
----------------------------------------------------------------------
| Sage Version 5.2, Release Date: 2012-07-25                         |
| Type "notebook()" for the browser-based notebook interface.        |
| Type "help()" for help.                                            |
----------------------------------------------------------------------
sage: f(x)=(x+4)/(3*x**2+10*x-8)
sage: f(-4)
---------------------------------------------------------------------------
ValueError                                Traceback (most recent call last)
ValueError: power::eval(): division by zero
\end{verbatim}
Il produit donc une erreur de division par zéro. Cela n'est pas étonnant. Pourtant si on lui demande, il est capable de simplifier. En effet :
\begin{verbatim}
sage: f.simplify_full()
x |--> 1/(3*x - 2)
sage: f.simplify_full()(-4)
-1/14
\end{verbatim}

Nous considérons la question suivante : étant donné une fonction \( f\) définie sur \( I\setminus\{ x_0 \}\), est-il possible de définir \( f\) en \( x_0\) de telles façon à ce qu'elle soit continue ?

\begin{example}
    La fonction
    \begin{equation}
        \begin{aligned}
            f\colon \eR\setminus\{ 0 \}&\to \eR \\
            x&\mapsto \frac{1}{ x }
        \end{aligned}
    \end{equation}
    n'est pas définie pour \( x=0\) et il n'y a pas moyen de définir \( f(0)\) de telle sorte que \( f\) soit continue parce que \( \lim_{x\to 0} \frac{1}{ x }\) n'existe pas.
\end{example}

%///////////////////////////////////////////////////////////////////////////////////////////////////////////////////////////
\subsubsection{Limite et prolongement}
%///////////////////////////////////////////////////////////////////////////////////////////////////////////////////////////

Reprenons l'exemple de la fonction \eqref{EqyEHTBZ} que mon ordinateur refusait de calculer en zéro :
\begin{equation}
f(x)=\frac{ x+4 }{ 3x^2+10x-8 }=\frac{ x+4 }{ (x+4)\left( x-\frac{ 2 }{ 3 } \right) }.
\end{equation}
Cette fonction a une condition d'existence en $x=-4$. Et pourtant, tant que $x\neq 4$, cela a un sens de simplifier les $(x+4)$ et d'écrire
\[
  f(x)=\frac{ 1 }{ x-\frac{ 2 }{ 3 } }=\frac{ 3 }{ 3x-2 }.
\]
Étant donné que pour toute valeur de $x$ différente de $-4$, la fonction $f$ s'exprime de cette façon, nous avons que
\[
  \lim_{x\to -4}f(x)=\lim_{x\to -4}\left(\frac{ 3 }{ 3x-2 }\right).
\]
Oui, mais la fonction\footnote{Cette fonction $g$ n'est pas $f$ parce que $g$ a en plus l'avantage d'être définie en $-4$.} $g(x)=3/(3x-2)$ est continue en $-4$ et donc sa limite vaut sa valeur. Nous en déduisons que
\[
  \lim_{x\to -4}f(x)=-\frac{ 3 }{ 14 }.
\]
Que dire maintenant de la fonction ainsi définie ?
\begin{equation}
\tilde f(x)=
\begin{cases}
f(x)&\text{si }x\neq -4\\
-3/14&\text{si }x=-4.
\end{cases}
\end{equation}
Cette fonction est continue en $-4$ parce qu'elle y est égale à sa limite. Les étapes suivies pour obtenir ce résultat sont :
\begin{itemize}
\item Repérer un point où la fonction n'existe pas,
\item calculer la limite de la fonction en ce point, et en particulier vérifier que cette limite existe, ce qui n'est pas toujours le cas,
\item définir une nouvelle fonction qui vaut partout la même chose que la fonction originale, sauf au point considéré où l'on met la valeur de la limite.
\end{itemize}
C'est ce qu'on appelle \defe{prolonger la fonction par continuité}{prolongement!par continuité} parce que la fonction résultante est continue. La prolongation de $f$ par continuité est donc en général définie par
\begin{equation}
\tilde f(x)=
\begin{cases}
f(x)            &\text{si }f(x)\\
\lim_{y\to x}f(y)   &\text{si }f(x)
\end{cases}
\end{equation}
Dans le cas que nous regardions,
\[
    f(x)=\frac{ x+4 }{ 3x^2+10x-8 },
\]
le prolongement par continuité est donné par
\begin{equation}
\tilde f =\frac{ 3 }{ 3x-2 }.
\end{equation}
Remarquons que cette fonction n'est toujours pas définie en $x=2/3$.

%---------------------------------------------------------------------------------------------------------------------------
\subsection{Prolongement par continuité}
%---------------------------------------------------------------------------------------------------------------------------

\begin{propositionDef}[Prolongement par continuité]
    Soit \( f\colon I\setminus\{ x_0 \}\to \eR\) telle que \( \lim_{x\to x_{0}} f(x)=\ell\in \eR\). La fonction
    \begin{equation}
        \begin{aligned}
            \tilde f\colon I&\to \eR \\
            \tilde f(x)&=\begin{cases}
                f(x)    &   \text{si } x\neq x_0\\
                \ell    &    \text{si } x=x_0
            \end{cases}
        \end{aligned}
    \end{equation}
    est une fonction continue sur \( I\) et est appelée le \defe{prolongement par continuité}{prolongement!par continuité} de \( f\) en \( x_0\).
\end{propositionDef}
Vous noterez que dans cet énoncé nous demandons \( \ell\in \eR\). Les cas \( \ell=\pm\infty\) sont donc exclus.

\begin{normaltext}
    Le lemme~\ref{LEMooUAFBooAwiXxj} donnera un autre gros morceau de prolongement par continuité. Là, ce ne sera pas juste une valeur qui manquera, mais carrément la majorité des valeurs.
\end{normaltext}

\begin{example}
    La fonction
    \begin{equation}
        \begin{aligned}
            f\colon \eR\setminus\{ -3,2 \}&\to \eR \\
            x&\mapsto  \frac{ x^2+2x-3 }{ (x+3)(x-2) }
        \end{aligned}
    \end{equation}
    admet pour limite \( \lim_{x\to -3} f(x)=\frac{ 4 }{ 5 }\). Son prolongement par continuité en \( x=-3\) est donné par
    \begin{equation}
        \tilde f(x)=\frac{ x-1 }{ x-2 }.
    \end{equation}
    Notons que les fonctions \( f\) et \( \tilde f\) ne sont pas identiques : l'une est définie pour \( x=-3\) et l'autre pas. Lorsqu'on fait le calcul
    \begin{equation}
        \frac{ x^2+2x-3 }{ (x+3)(x-2) }=\frac{ (x-1)(x+3) }{ (x+3)(x-2) }=\frac{ x-1 }{ x-2 },
    \end{equation}
    la simplification n'est pas du tout un acte anodin. Le dernier signe «\( =\)» est discutable parce que les deux dernières expressions ne sont pas égales pour tout \( x\); elles ne sont égales «que» pour les \( x\) pour lesquels les deux expressions existent.
\end{example}

Les fonctions trigonométriques donneront quelques exemples intéressants de prolongements par continuité. Voir l'exemple~\ref{ExQWHooGddTLE}. Et une avec la fonction logarithme dans l'exemple~\ref{EXooAGEOooQdQkrS}.

%---------------------------------------------------------------------------------------------------------------------------
\subsection{Théorème de la bijection}
%---------------------------------------------------------------------------------------------------------------------------

\begin{proposition} \label{PropOARooUuCaYT}
    Une fonction monotone et surjective d'un intervalle $I$ sur un autre intervalle $J$ est continue sur $I$.
\end{proposition}

\begin{proposition}
    Soient \( f\colon I\to J\) une bijection et \( f^{-1}\colon J\to I\) sa réciproque. Alors pour tout \( x_0\in I\) nous avons
    \begin{equation}    \label{EqHQRooNmLYbF}
        f^{-1}\big( f(x_0) \big)=x_0
    \end{equation}
    et pour tout \( y_0\in J\) nous avons
    \begin{equation}    \label{EqIYTooQPvZDr}
        f\big( f^{-1}(y_0) \big)=y_0.
    \end{equation}
\end{proposition}

\begin{proof}
    Nous prouvons la relation \eqref{EqHQRooNmLYbF} et nous laissons \eqref{EqIYTooQPvZDr} comme exercice au lecteur.

    Soit \( x_0\in I\). Posons \( y_0=f(x_0)\). La définition de l'application réciproque est que pour \( y\in J\), \( f^{-1}(y)\) est l'unique élément \( x\) de \( I\) tel que \( f(x)=y\). Donc \( f^{-1}(y_0)\) est l'unique élément de \( I\) dont l'image est \( y_0\). C'est donc \( x_0\) et nous avons \( f^{-1}(y_0)=x_0\), c'est à dire
    \begin{equation}
        f^{-1}\big( f(x_0) \big)=x_0.
    \end{equation}
\end{proof}

\begin{theorem}[Théorème de la bijection] \label{ThoKBRooQKXThd}
    Soit $I$ un intervalle et $f$ une fonction continue et strictement monotone de $I$ dans \( \eR\). Nous avons alors :
    \begin{enumerate}
        \item
            $f(I)$ est un intervalle de \( \eR\) ;
        \item       \label{ITEMooMAWXooZXmVwA}
            La fonction \( f\colon I\to f(I)\) est bijective
        \item
            La fonction \( f^{-1}\colon f(I)\to I\) est strictement monotone de même sens que $f$ ;
        \item \label{ItemEJZooKuFoeFiv}
            La fonction \( f\colon I\to f(I)\) est un homéomorphisme, c'est-à-dire que \( f^{-1}\colon f(I)\to I\) est continue.
    \end{enumerate}
\end{theorem}

\begin{proof}

    Prouvons les choses point par point.

    \begin{enumerate}
    \item

        Supposons pour fixer les idées que \( f\) est monotone croissante\footnote{Traitez en tant qu'exercice le cas où $ f$ est décroissante.}.

        Soient \( a< b\) dans \( f(I)\). Par définition il existe \( x_1,x_2\in I\) tels que \( a=f(x_1)\) et \( b=f(x_2)\). La fonction \( f\) est continue sur l'intervalle \( \mathopen[ x_1 , x_2 \mathclose]\) et vérifie \( f(x_1)<f(x_2)\). Donc le théorème des valeurs intermédiaires~\ref{ThoValInter} nous dit que pour tout \( t\) dans \( \mathopen[ f(x_2) , f(x_2) \mathclose]\), il existe un \( x_0\in\mathopen[ x_1 , x_2 \mathclose]\) tel que \( f(x_0)=t\). Cela montre que toutes les valeurs intermédiaires entre \( a\) et \( b\) sont atteintes par \( f\) et donc que \( f(I)\) est un intervalle.

    \item

    Nous prouvons maintenant que \( f\) est bijective en prouvant séparément qu'elle est surjective et injective.

    \begin{subproof}

        \item[\( f\) est surjective]

            Une fonction est toujours surjective depuis un intervalle \( I\) vers l'ensemble \(\Im f \).

        \item[\( f\) est injective]

            Soit \( x\neq y\) dans \( I\); pour fixer les idées nous supposons que \( x<y\). La stricte monotonie de \( f\) implique que \( f(x)<f(y)\) ou que \( f(x)>f(y)\). Dans tous les cas \( f(x)\neq f(y)\).

    \end{subproof}

    La fonction \( f\) est donc bijective.

\item

    Comme d'accoutumée nous supposons que \( f\) est croissante. Soient \( y_1<y_2\) dans \( f(I)\); nous devons prouver que \( f^{-1}(y_1)\leq f^{-1}(y_2)\). Pour cela nous considérons les nombres \( x_1,x_2\in I\) tels que \( f(x_1)=y_1\) et \( f(x_2)=y_2\). Nous allons en prouver la contraposée en supposant que \( f^{-1}(y_1)>f^{-1}(y_2)\). En appliquant \( f\) (qui est croissante) à cette dernière inégalité il vient
    \begin{equation}
        f\big( f^{-1}(y_1) \big)\geq f\big( f^{-1}(y_2) \big),
    \end{equation}
    ce qui signifie
    \begin{equation}
        y_1\geq y_2
    \end{equation}
    par l'équation \eqref{EqIYTooQPvZDr}.

\item

    La fonction \( f^{-1}\colon f(I)\to I\) est une fonction monotone et surjective, donc continue par la proposition~\ref{PropOARooUuCaYT}.

    \end{enumerate}
\end{proof}

\begin{example}
    La fonction
    \begin{equation}
        \begin{aligned}
            f\colon \mathopen[ 2 , 3 \mathclose]&\to \mathopen[ 4 , 9 \mathclose] \\
            x&\mapsto x^2
        \end{aligned}
    \end{equation}
    est une bijection. Sa réciproque est la fonction
    \begin{equation}
        \begin{aligned}
            f^{-1}\colon \mathopen[ 4 , 9 \mathclose]&\to \mathopen[ 2 , 3 \mathclose] \\
            x&\mapsto \sqrt{x}.
        \end{aligned}
    \end{equation}
\end{example}

%+++++++++++++++++++++++++++++++++++++++++++++++++++++++++++++++++++++++++++++++++++++++++++++++++++++++++++++++++++++++++++
\section{Limite et continuité}
%+++++++++++++++++++++++++++++++++++++++++++++++++++++++++++++++++++++++++++++++++++++++++++++++++++++++++++++++++++++++++++
\label{SecLimiteFontion}

Voir les remarques dans l'index thématique~\ref{THEMEooGVCCooHBrNNd} pour comprendre la place et la portée de ce qui va venir à propos de limite et de continuité.

\begin{theorem}[Limite et continuité]           \label{ThoLimCont}
La fonction $f$ est continue au point $a$ si et seulement si $\lim_{x\to a}f(x)=f(a)$.
\end{theorem}

\begin{proof}
    Nous commençons par supposer que $f$ est continue en $a$, et nous prouvons que $\lim_{x\to a}f(x)=a$. Soit $\epsilon>0$; ce qu'il nous faut c'est un $\delta$ tel que $| x-a |\leq\delta$ implique $| f(x)-f(a) |\leq\epsilon$. La caractérisation \ref{PROPooVNGEooPwbxXP} de la continuité donne l'existence d'un $\delta$ comme il nous faut.

    Dans l'autre sens, c'est à dire prouver que $f$ est continue au point $a$ sous l'hypothèse que $\lim_{x\to a}f(x)=f(a)$, la preuve se fait de la même façon.
\end{proof}

Nous en déduisons que si nous voulons gagner quelque chose à parler de limites, il faut prendre des fonctions non continues. En effet, si une fonction est continue en un point, la limite ne donne aucune nouvelle information que la valeur de la fonction elle-même en ce point.

Prenons une fonction qui fait un saut. Pour se fixer les idées, prenons celle-ci :
\begin{equation}    \label{EqnCtOEL}
f(x)=
\begin{cases}
2x&\text{si }x\in]\infty,2[\\
x/2&\text{si }x\in[2,\infty[
\end{cases}
\end{equation}
Essayons de trouver la limite de cette fonction lorsque $x$ tend vers $2$. Étant donné que $f$ n'est pas continue en $2$, nous savons déjà que $\lim_{x\to 2}f(x)\neq f(2)$. Donc ce n'est pas $1$. Cette limite ne peut pas valoir $4$ non plus parce que si je prends n'importe quel $\epsilon$, la valeur de $f(2+\epsilon)$ est très proche de $2$, et donc ne peut pas s'approcher de $4$. En fait, tu peux facilement vérifier que \emph{aucun nombre ne vérifie la condition de limite pour $f$ en $2$}. Nous disons que la limite n'existe pas.

Il ne faudrait pas en déduire trop vite que si une fonction n'est pas continue en \( a\), alors la limite \( x\to a\) n'existe pas. Ce que dit le théorème~\ref{ThoLimCont} est que si une fonction n'est pas continue en \( a\), alors sa limite (si elle existe) ne vaut pas \( f(a)\).

\begin{example}[Un exemple de continuité Thème~\ref{THEMEooGVCCooHBrNNd}]     \label{EXooKREUooLeuIlv}
    Soit la fonction
    \begin{equation}        \label{EQooSYSWooSGsUfR}
        f(x)=\begin{cases}
            x    &   \text{si } x\neq 0\\
            4    &    \text{si } x=0.
        \end{cases}
    \end{equation}
    Cette fonction n'est pas continue en \( x=0\), et pourtant la limite existe : \( \lim_{x\to 0} f(x)=0\). Faisons cela en détail pour nous assurer de ce qu'il se passe.

    Considérons l'ouvert \( \mathopen] 3 , 5 \mathclose[\). L'image réciproque de cet ouvert par \( f\) est la partie \( \mathopen] 3 , 5 \mathclose[\cup\{ 0 \}\) qui n'est pas ouvert. Donc la fonction \( f\) n'est pas continue comme fonction \( \eR\to \eR\).

    Considérons pour comprendre la restriction \( f\colon \mathopen[ -1 , 1 \mathclose]\to \eR\). L'image inverse de \( \mathopen] 3 , 5 \mathclose[\) par cette fonction est \( \{ 0 \}\) qui n'est pas un ouvert.

    Plus généralement tant qu'on considère des restrictions de \( f\) sur des parties contenant un voisinage de \( 0\), la fonction ne peut pas être continue\footnote{Les plus acharnés se demanderont ce qu'il se passe pour la restriction de \( f\) à la partie \( \{ 0 \}\) munie de la topologie induite de $\eR$.}.

    Voyons ce qui en est de la continuité ponctuelle de \( f\) en \( x=0\). La définition~\ref{DefOLNtrxB} est celle de la continuité en un point; elle dit que \( f\) sera continue en \( 0\) si \( f(0)=4\) est une limite de \( f\). Nous voila parti vers la définition~\ref{DefYNVoWBx}.

Soit le voisinage \( V=\mathopen] 3 , 5 \mathclose[\) de \( f(0)\). Quel que soit le voisinage \( W\) de \( 0\) dans \( \eR\), il existe un \( \epsilon>0\) tel que \( W\subset B(0,\epsilon)\). Nous avons alors
    \begin{equation}
        f\big( W\setminus \{ a \} \big)\subset f\big( B(0,\epsilon)\setminus\{ 0 \} \big).
    \end{equation}
    Mais le nombre \( \epsilon/2\) fait partie de \( f\big( B(0,\epsilon)\setminus\{ 0 \} \big)\) et n'est pas dans \( V\). Donc \( f(0)\) n'est pas une limite de \( f\) en zéro. Cette fonction n'est donc pas continue en zéro.
\end{example}

\begin{example}[Même exemple, limite]
    Nous avons vu que, pour la fonction \eqref{EQooSYSWooSGsUfR}, le nombre \( 4\) n'est pas une limite de \( f\) en zéro. Nous montrons à présent que \( 0\) est une limite (et même la seule par la proposition~\ref{PropFObayrf} que nous ne rappelleront plus à chaque fois) de \( f\).

    Montrons que \( 0\) est une limite de \( f\) en zéro, c'est à dire que \( \lim_{x\to 0} f(x)=0\).

    Nous suivant la définition~\ref{DefYNVoWBx}. Soit un voisinage \( V\) de \( 0\) dans \( \eR\). Il existe \( \delta\) tel que \( B(0,\delta)\subset V\). En posant \( \epsilon=\delta\) et en définissant \( W=B(0,\epsilon)\) nous avons
    \begin{equation}
        f\big( B(0,\epsilon)\setminus\{ 0 \} \big)=B(0,\epsilon)\setminus\{ 0 \}\subset  B(0,\delta)\subset V.
    \end{equation}
    Donc \( 0\) est une limite de \( f\) en zéro.
\end{example}

Nous avons déjà vu par le corollaire~\ref{CorFHbMqGGyi} qu'une suite croissante et bornée était convergente. Il en va de même pour les fonctions.
\begin{proposition}[\cite{MonCerveau}] \label{PropMTmBYeU}
    Si la fonction réelle \( f\colon I=\mathopen[ a , b [\to \eR\) est croissante et bornée, alors la limite
    \begin{equation}
        \lim_{x\to b} f(x)
    \end{equation}
    existe et est finie.
\end{proposition}

\begin{proof}
    Commençons par prouver que si \( (x_n)\) est une suite dans \( I\) convergent vers \( b\), alors \( f(x_n)\) est une suite convergente. Dans un second temps nous allons prouver que si \( (x_n)\) et \( (x'_n)\) sont deux suites qui convergent vers \( b\), alors les suites convergentes \( f(x_n)\) et \( f(x'_n)\) convergent vers la même limite. Alors le critère séquentiel de la limite d'une fonction conclura (proposition~\ref{PROPooJYOOooZWocoq}).

    Nous pouvons extraire de \( x_n\) une sous-suite croissante \( (x_{\alpha(n)})\). Alors la suite \( f\big( x_{\alpha(n)} \big)\) est une suite croissante et majorée, donc convergente par le corollaire~\ref{CorFHbMqGGyi}\footnote{En gros nous sommes en train de dire que toute la théorie des fonctions convexes est un vulgaire corollaire de Bolzano-Weierstrass.}. Nommons \( \ell\) la limite et montrons qu'elle est aussi limite de \( f\) sur la suite originale.

    Pour tout \( \epsilon>0\), il existe \( K\) tel que si \( n>K\) alors \( \big| f\big( x_{\alpha(n)} \big)-\ell \big|<\epsilon\). Soit \( K'\) tel que pour tout \( n>K'\) nous ayons \( x_n>x_{\alpha(K')}\). Cela est possible parce que la suite est bornée par \( b\) et converge vers \( b\) : il suffit de prendre \( K'\) de telle sorte que \( | x_n-b |\leq | x_{\alpha(n)}-b |\). Si \( n>K'\) alors \( x_n>x_{\alpha(K)}\) et
    \begin{equation}
        f(x_n)\geq f(x_{\alpha(n)})\geq \ell-\epsilon;
    \end{equation}
    en résumé si \( n>K\) alors \( | f(x_n)-\ell |<\epsilon\). Cela prouve que \( f(x_n)\to\ell\).

    Soit maintenant une autre suite \( (x'_n)\) qui converge également vers \( b\). Comme nous venons de le voir la suite \( f(x'_n)\) est convergente et nous nommons \( \ell'\) la limite. Si nous considérons \( (x''_n)\) la suite «alternée» (\( x_1,x'_1,x_2,x'_2,\cdots\)) alors nous avons encore une suite qui converge vers \( b\) et donc \( f(x''_n)\to \ell'\).

    Mais étant donné que \( f(x_n)\) et \( f(x'_n)\) sont des sous-suites, elles doivent converger vers la même valeur. Donc \( \ell=\ell'=\ell''\).
\end{proof}

%TODO : écrire un truc sur la limite à gauche et la limite pour la topologie induite.

%---------------------------------------------------------------------------------------------------------------------------
\subsection{Règles simples de calcul}
%---------------------------------------------------------------------------------------------------------------------------

Les opérations simples passent à la limite, sauf la division pour laquelle il faut faire attention au dénominateur.
\begin{proposition}     \label{PropOpsSimplesLimites}
    Soient \( f\) et \( g\) deux fonctions telles que \( \lim_{x\to a} f(x)=\alpha\) et \( \lim_{x\to a} g(x)=\beta\). Alors
    \begin{enumerate}
        \item
            \( \lim_{x\to a} f(x)+g(x)=\alpha+\beta\),
        \item
            \( \lim_{x\to a} f(x)g(x)=\alpha\beta\),
        \item
            s'il existe un voisinage de \( a\) sur lequel \( g\) ne s'annule pas, alors \( \lim_{x\to a} \frac{ f(x) }{ g(x) }=\frac{ \alpha }{ \beta }\).
    \end{enumerate}
\end{proposition}

Le résultat suivant est pratique pour le calcul des limites.
\begin{proposition}     \label{PropChmVarLim}
Quand la limite existe, nous avons
\[
  \lim_{x\to a}f(x)=\lim_{\epsilon\to 0}f(a+\epsilon),
\]
ce qui correspond à un «changement de variables» dans la limite.
\end{proposition}

\begin{proof}
Si $A=\lim_{x\to a}f(x)$, par définition,
\begin{equation}        \label{EqCondFaplusespLim}
\forall\epsilon'>0,\,\exists\delta\text{ tel que }| x-a |\leq\delta\Rightarrow| f(x)-A |\leq\epsilon'.
\end{equation}
La seule subtilité de la démonstration est de remarquer que si $| x-a |\leq\delta$, alors $x$ peut être écrit sous la forme $x=a+\epsilon$ pour un certain $| \epsilon |\leq\delta$. En remplaçant $x$ par $a+\epsilon$ dans la condition~\ref{EqCondFaplusespLim}, nous trouvons
\begin{equation}
\forall\epsilon'>0,\,\exists\delta\text{ tel que }| \epsilon |\leq\delta\Rightarrow| f(x+\epsilon)-A |\leq\epsilon',
\end{equation}
ce qui signifie exactement que $\lim_{\epsilon\to 0}f(x+\epsilon)=A$.
\end{proof}

Il y a une petite différence de point de vue entre $\lim_{x\to a}f(x)$ et $\lim_{\epsilon\to 0}f(a+\epsilon)$. Dans le premier cas, on considère $f(x)$, et on regarde ce qu'il se passe quand $x$ se rapproche de $a$, tandis que dans le second, on considère $f(a)$, et on regarde ce qu'il se passe quand on s'éloigne un tout petit peu de $a$. Dans un cas, on s'approche très près de $a$, et dans l'autre on s'en éloigne un tout petit peu. Le contenu de la proposition~\ref{PropChmVarLim} est de dire que ces deux points de vue sont équivalents.

% Il y a des techniques de calcul de limites décrites sur le site
% http://bernard.gault.free.fr/terminale/limites/limite.html

%---------------------------------------------------------------------------------------------------------------------------
\subsection{Prolongement des rationnels vers les réels}
%---------------------------------------------------------------------------------------------------------------------------

\begin{lemma}[\cite{MonCerveau}]        \label{LEMooUAFBooAwiXxj}
    Soit une fonction continue\footnote{Au sens de la topologie induite; nous en avons discuté dans l'exemple~\ref{EXooHWIIooYYbfGE}.} \( f\colon \eQ\to \eR\).
    \begin{enumerate}
        \item
            La limite \( \lim_{q\to x} f(q)\) existe pour tout \( x\in \eR\).
        \item
            Il existe un unique prolongement continu \( \tilde f\colon \eR\to \eR\).
        \item
            Ce prolongement est donné par
            \begin{equation}
            \tilde f(x)=\begin{cases}
                f(x)    &   \text{si } x\in \eQ\\
                \lim_{q\to x} f(q)    &    \text{sinon }
            \end{cases}
        \end{equation}
    \end{enumerate}
\end{lemma}

\begin{proof}
    Imprégniez vous bien de la la définition~\ref{DefYNVoWBx} de la limite avant de commencer.

    \begin{subproof}

    \item[Unicité]

        Prouvons rapidement l'unicité avant l'existence parce que c'est facile.

        L'unicité du prolongement est la proposition~\ref{PropCJGIooZNpnGF} à propos de fonctions continues égales sur une partie denses. La densité de \( \eQ\) dans \( \eR\), si vous la cherchez est la proposition~\ref{PropooUHNZooOUYIkn}.

        \item[Candidat limite]

        Soit \( x\in \eR\). Vu que \( x\in \bar \eQ\), nous pouvons chercher à savoir si \( \lim_{q\to x} f(q) \) existe. Si elle existe, elle sera unique.

        Soit une suite \( (q_i)\) d'éléments de \( \eQ\) qui converge vers \( x\) dans \( \eR\) (i.e. pour la topologie de \( \eR\)). Les nombres réels \( f(q_i)\) forment une suite dans \( \eR\). Nous allons montrer que cette suite converge vers un réel qui vérifie la définition de \( \lim_{q\to x}f(q)\).

        Soit \( \epsilon>0\). Par la continuité de \( f\), il existe \( \delta\) tel que si \( | p-q |<\delta\) (pour \( p,q\in \eQ\)), alors \( | f(p)-f(q) |<\epsilon\). Par ailleurs, \( (q_i)\) est de Cauchy dans \( \eR\) (théorème~\ref{THOooNULFooYUqQYo} qu'il est toujours bon de citer de temps en temps). Il existe donc \( N\) tel que \( i,j>N\) implique \( | q_i-q_j |<\delta\).

        Avec de tels \( i\) et \( j\), nous avons
        \begin{equation}
            | f(q_i)-f(q_j) |<\epsilon,
        \end{equation}
        ce qui prouve que \( i\mapsto f(q_i)\) est une suite de Cauchy dans \( \eR\) et donc qu'elle converge.

    \item[C'est bien la limite]

        Nous prouvons à présent que le nombre réel \( \lim_{i\to \infty} f(q_i)\) vérifie bien la définition de la limite \( \lim_{q\to x}f(q)\).

        Soit un voisinage \( V\) de \( \lim f(q_i)\) dans \( \eR\). Nous devons trouver un voisinage \( W\) de \( x\) dans \( \eR\) tel que
        \begin{equation}
            f\big( W\cap\eQ\setminus\{ x \} \big)\subset V.
        \end{equation}
        Pour cela nous considérons \( \epsilon>0\) tel que \( B\big( \lim f(q_i),\epsilon \big)\subset V\). Vu que \( f\) est continue sur \( \eQ\), il existe \( \delta\) tel que
        \begin{equation}
            | p-q |<2\delta\Rightarrow\,| f(p)-f(q) |<\epsilon.
        \end{equation}
        Nous posons \( W=B(x,\delta)\).

        Soit \( q\in W\cap\eQ\setminus\{ x \}\). Nous nous proposons de majorer la quantité $| f(q)-\lim f(q_i) |$ par un multiple de \( \epsilon\).

        Pour cela nous considérons \( k\) suffisamment grand pour que \( | f(q_k)-\lim f(q_i)  |<\epsilon\). Et de plus, vu que \( q_i\to x\) nous considérons \( k\) suffisamment grand pour que \( | q_k-x |<\delta\). L'indice \( k\) est choisi pour vérifier les deux conditions en même temps.

        Nous écrivons alors la majoration suivante :
        \begin{equation}
                | f(q)-\lim f(q_i) |\leq | f(q)-f(q_k) |+| f(q_k)+\lim f(q_i) |.
        \end{equation}
        Le second terme est majoré par \( \epsilon\). Pour le premier terme, \( q\in B(x,\delta)\) et \( q_k\in B(x,\delta)\), donc \( | q-q_k |\leq 2\delta\), ce qui implique \( | f(q)-f(q_k) |<\epsilon\).

        Au final, \( | f(q)-\lim f(q_i) |\leq 2\epsilon\). En reprenant tout le travail avec \( \epsilon/2\) au lieu de \( \epsilon\) nous trouvons \( f(q)\in B\big( \lim f(q_i),\epsilon \big)\subset V\).

    \item[Intermède]

        Jusqu'à présent, nous avons prouvé que
        \begin{equation}        \label{EQooUJJKooTYRNDo}
            \lim_{q\to x} f(q)
        \end{equation}
        existe et vaut
        \begin{equation}        \label{EQooNSYCooTmECjs}
            \lim f(q_i)
        \end{equation}
        lorsque \( (q_{i})\) est une suite quelconque de rationnels qui converge vers \( x\). Nous l'écrivons pour la référentier plus tard :
        \begin{equation}        \label{EQooSGCMooKtpVMy}
            \lim_{q\to x} f(q)=\lim f(q_i).
        \end{equation}
        La limite \eqref{EQooUJJKooTYRNDo} est une limit de fonction définie sur \( \eQ\subset \eR\) en un pour adhérent à l'ensemble de définition de \( f\). La limite \eqref{EQooNSYCooTmECjs} est une limite usuelle d'une suite dans \( \eR\).

    \item[Le prolongement]

        Nous posons
        \begin{equation}
            \tilde f(x)=\begin{cases}
                f(x)    &   \text{si } x\in \eQ\\
                \lim_{q\to x} f(q)    &    \text{sinon }
            \end{cases}
        \end{equation}
        et nous allons prouver que \( \tilde f\) est une fonction continue sur \( \eR\).

    \item[Continuité]

        Soit \( a\in \eR\); nous allons montrer la continuité de \( \tilde f\) en \( a\). Nous fixonx bien entendu \( \epsilon>0\), et nous nous acharnons à majorer la quantité \( | \tilde f(x)-\tilde f(a) |\).

        Vu que \( f\) est continue sur \( \eQ\) nous considérons \( \delta'\) tel que (dans \( \eQ\)) \( 0<| q-q' |<\delta'\) implique \( | f(q)-f(q') |<\epsilon\).

        \begin{subproof}
            \item[\( a\in \eQ\), \( x\in \eQ\)]
                Alors \( \tilde f(x)=f(x)\) et \( \tilde f(a)=f(a)\). Par la continuité de \( f\) sur \( \eQ\), il existe un \( \delta\) tel que \( 0<| x-a |<\delta\) implique \( | f(x)-f(a) |<\epsilon\).

            \item[\( a\in \eQ\), \( x\) irrationnel]

                Nous considérons une suite de rationnels \( q_k\to x\) (vous penserez à l'utilisation du lemme \ref{LemooRTGFooYVstwS}). Nous avons la majoration
                \begin{equation}        \label{EQooPDEMooHlwTcm}
                    | \tilde f(x)-\tilde f(a) |=| \lim_{q\to x} f(q)-f(a) |\leq | \lim_{q\to x} f(q)-f(q_k) |+| f(q_k)-f(a) |.
                \end{equation}
                Nous considérons \( \delta<\delta'\) et \( k\) suffisament grand pour que \( | q_k-x |<\delta'-\delta\). Avec ces choix,
                \begin{equation}
                    | q_k-a |\leq | q_k-x |+| x-a |\leq \delta'.
                \end{equation}
                Enfin nous prenons également \( k\) suffisament grand pour avoir \( | \lim_{q\to x} f(q)-f(q_k) |\leq \epsilon\).

                Les inégalités \eqref{EQooPDEMooHlwTcm} peuvent alors être prolongées pour avoir
                \begin{equation}
                    | \tilde f(x)-\tilde f(a) |\leq 2\epsilon.
                \end{equation}
                
            \item[\( a\) irrationnel, \( x\in \eQ\)]

                Nous faisons encore la majoration
                \begin{equation}
                    | \tilde f(x)-\tilde f(a) |=| f(x)-\lim_{q\to a} f(q) |\leq | f(x)-f(q_k) |+| f(q_k)-\lim_{q\to a} f(a) |.
                \end{equation}
                Nous prenons \( \delta<\delta'/2\) et nous choisissons \( k\) assez grand pour que \( | q_k-a |<\delta'/2\). De ces choix il ressort que
                \begin{equation}
                    | q_k-x |\leq | q_k-a |+| a-x |\leq \frac{ \delta' }{2}+\frac{ \delta' }{2}\leq \delta'.
                \end{equation}
                Donc \( | f(x)-f(q_k) |<\epsilon\). De plus, pour \( k\) assez grand, \( | f(q_k)-\lim_{q\to a} f(q) |\leq \epsilon\).

            \item[\( a\) et \( x\) irrationnels]

                Nous avons
                \begin{equation}
                    | \tilde f(x)-\tilde f(a) |=| \lim_{q\to x} f(q)-\lim_{r\to a} f(r) |,
                \end{equation}
                et nous considérons des suites de rationnels \( q_k\to x\) et \( r_i\to a\). De plus nous considérons \( \delta<\delta'/4\), et \( k,i\) suffisament grands pour avoir \( | q_k-x |\leq \delta'/4\) et \( | r_i-a |<\delta'/4\). Avec tout cela nous avons
                \begin{equation}
                    | q_k-r_i |\leq | q_k-x |+| x-a |+| a-r_i |\leq 3\delta'/4<\delta'.
                \end{equation}
                Enfin, en choisissant \( i\) et \( k\) de telle sorte à avoir \( | \lim_{q\to x} f(q)-f(q_k) |\leq \epsilon\) et \( | f(r_i)-\lim_{r\to a} f(r) |<\epsilon\) nous avons les majorations
                \begin{subequations}
                    \begin{align}
                        | \tilde f(x)-\tilde f(a) |&=| \lim_{q\to x} f(q)-\lim_{r\to a} f(r) |\\
                        &\leq | \lim_{q\to x} f(x)-f(q_k) |+| f(q_k)-f(r_i) |+| f(r_i)-\lim_{r\to q} f(r) |\\
                        &\leq 3\epsilon.
                    \end{align}
                \end{subequations}
        \end{subproof}

    \end{subproof}
\end{proof}

\begin{proposition}[\cite{MonCerveau}]      \label{PROPooTNIAooNAJDzL}
    Soit une fonction strictement croissante \( f\colon \eQ\to \eR\). Alors la prolongation continue \( \tilde f\colon \eR\to \eR\) est également strictement croissante.
\end{proposition}

\begin{proof}
    Soient \( x,y\in \eR\) avec \( x<y\). Notons \( d=y-x\). Nous considérons des suites de rationnels \( x_k\to x\) et \( y_l\to y\) telles que pour tout \( k\), \( x_k\in B(x,d/3)\) et \( y_k\in B(y,d/3)\). En particulier, \( x_k<y_l\) pour tout \( k\) et \( l\).

    Soient des rationnels \( q\) et \( q'\) tels que pour tout \( k\),
    \begin{equation}
        x_k<q<q'<y_k.
    \end{equation}
    Pour trouver de tels rationnels, il suffit de les chercher dans \( \mathopen] x+\frac{ d }{ 3 } , y-\frac{ d }{ 3 } \mathclose[\). Cet intervalle étant de longueur \( d/3\), il contient des rationnels.

    Vue la croissance de \( f\) sur \( \eQ\), nous avons, pour tout \( k\) :
    \begin{equation}
        f(x_k)<f(q)<f(q')<f(y_k),
    \end{equation}
    et à la limite :
    \begin{equation}
        \tilde f(x)\leq f(q)<f(q')\leq \tilde f(y).
    \end{equation}
    Notez que les inégalités strictes se changent en inégalités larges au passage à la limite. D'où l'utilisé de prendre \emph{deux} rationnels entre \( x_k\) et \( y_k\) pour maintenir une inégalité stricte entre \(\tilde f(x)\) et \( \tilde f(y)\).

\end{proof}

%+++++++++++++++++++++++++++++++++++++++++++++++++++++++++++++++++++++++++++++++++++++++++++++++++++++++++++++++++++++++++++ 
\section{La fonction puissance}
%+++++++++++++++++++++++++++++++++++++++++++++++++++++++++++++++++++++++++++++++++++++++++++++++++++++++++++++++++++++++++++

Si \( x\) et \( y\) sont des réels, définir \( x^y\) n'est pas une mince affaire. Pour l'instant nous savons déjà définir \( x^n\) lorsque \( x\in \eR\) et \( n\in \eN\).

Pour la suite nous notons
\begin{subequations}
    \begin{numcases}{}
        f_{\alpha}(x)=x^{\alpha}\\
        g_{a}(x)=a^x
    \end{numcases}
\end{subequations}
pour autant que ces fonctions sont définies\footnote{L'objet des pages suivantes est de déterminer pour quelles valeurs de $a$, $\alpha$ et $ x$ nous pouvons trouver des définitions raisonnables pour ces fonctions.}.

%--------------------------------------------------------------------------------------------------------------------------- 
\subsection{Sur les naturels}
%---------------------------------------------------------------------------------------------------------------------------

\begin{definition}
    La fonction puissance définie sur \( \eN\) s'étend à \( \eZ\) de la façon suivante :
    \begin{equation}
        x^{-n}=\frac{1}{ x^n }
    \end{equation}
    pour \( n\geq 0\). Cela donne donne donc \( x^n\) pour \( x\in \eR\) et \( n\in \eZ\) a l'exception de \( x=0\) lorsque \( n<0\).
\end{definition}

Nous étudions quelques propriétés de cette fonction pour \( n>0\) fixé.

\begin{proposition}     \label{PROPooXQYFooPxoEHE}
    Soit \( n\in \eN\setminus\{ 0 \}\); nous posons \( f_n(x)=x^n\).

    Si \( n\) est pair,
    \begin{equation}
        f_n\colon \mathopen[ 0 , \infty \mathclose[\to \mathopen[ 0 , \infty \mathclose[
    \end{equation}
    est bijective.

    Si \( n\) est impair,
    \begin{equation}
        f_n\colon \eR \to \eR
    \end{equation}
    est bijective.

    Toutes les fonctions \( f_n\) sont continues sur \( \eR\).
\end{proposition}

\begin{proof}
    En plusieurs morceaux, pas spécialement dans l'ordre auquel on s'attend.
    \begin{subproof}
        \item[Continuité]

            Soit \( x\in \eR\). En vertu de~\ref{ThoLimCont} nous allons prouver que \( \lim_{\epsilon\to 0}f_n(x+\epsilon)=f_n(x)\). Pour cela nous utilisons la formule du binôme~\ref{PropBinomFExOiL} avec \( x,h>0\) :
            \begin{equation}
                f_n(x+h)=(x+h)^n=\sum_{k=0}^n{n\choose k}x^{n-k}h^k.
            \end{equation}
            Nous fixons \( x_0\in \eR\). Calcul :
            \begin{subequations}
                \begin{align}
                    | f_n(x_0+h)-f_n(x) |&=| \sum_{k=1}^n{n\choose k}x_0^{n-k}h^k |\\
                    &\leq \sum_{k=1}^n{n\choose k}| x_0 |^{n-k} |h|^k\\
                    &=h\sum_{k=1}^n{n\choose k}| x_0 |^{n-k}| h |^{k-1}\\
                    &\leq h\sum_{k=1}^n{n\choose k}| x_0 |^{n-k}.
                \end{align}
            \end{subequations}
            Justifications :
            \begin{itemize}
                \item
                   Le terme \( k=0\) est égal à \( x^n=f_n(x)\) parce que \( {n\choose 0}=1\).
               \item
                   Dans la somme nous avons majoré \( | h |\) par \( 1\), opération justifiée par le fait que nous ayons dans l'idée de faire \( h\to 0\).
            \end{itemize}
            Nous avons donc
            \begin{equation}
                \lim_{h\to 0} | f_n(x_0+h)-f_n(x) | \leq\lim_{h\to 0}  h\sum_{k=1}^n{n\choose k}| x_0 |^{n-k}=0.
            \end{equation}
            D'où la continuité de \( f_n\) en tout point \( x_0\in \eR\).

        \item[Pour \( n\) pair ou impair, bijection sur les positifs]
            Ceci sera déjà le résultat complet pour les \( n\) pairs, et a moitié du résultat pour les \( n\) impairs.
            \begin{subproof}
                \item[Stricte croissance]
                    Soit \( n\neq 0\) dans \( \eN\). Nous commençons par prouver que \( f_n\) est strictement croissante sur \( \mathopen[ 0 , \infty \mathclose[\). Nous repartons de la formule du binôme, mais cette fois, nous séparons les termes \( k=0\) et \( k=n\) des autres (si \( n=1\), il y a un peu de réécriture) en tenant compte de \( {n\choose 0}={n\choose n}=1\) :
                        \begin{equation}
                            f_n(x+h)=x^n+h^n+\sum_{k=1}^{n-1}{n\choose k}x^{n-k}h^k>x^n=f_n(x).
                        \end{equation}
                        Vous noterez que l'inégalité est stricte même si \( n=1\).

                        Vu que nous avons stricte monotonie, le théorème~\ref{ThoKBRooQKXThd}\ref{ITEMooMAWXooZXmVwA} nous dit que
                        \begin{equation}
                            f_n\colon \mathopen[ 0 , \infty \mathclose[\to f_n\big( \mathopen[ 0 , \infty \mathclose[ \big)
                        \end{equation}
                        est une bijection.
                    \item[Bijection]

                        Nous prouvons que \( f_n\big( \mathopen[ 0 , \infty \mathclose[ \big)=\mathopen[ 0 , \infty \mathclose[\). Si \( x>0\) alors \( f_n(x)>0\), cela prouve une inclusion.

                            Pour l'autre inclusion nous savons que \( f_n(x)>x\) dès que \( x>1\). Donc \( \lim_{x\to \infty} f_n(x)=\infty\). Si \( y\in \mathopen[ 0 , \infty \mathclose[\), alors il existe \( x_0\) tel que \( f_n(x_0)>y\). Étant donné que \( f_n(0)=0\) et que nous avons déjà prouvé que \( f_n\) était continue (proposition~\ref{PROPooXQYFooPxoEHE}), le théorème des valeurs intermédiaires~\ref{ThoValInter} nous indique l'existence de \( x_1\in \mathopen[ 0 , x_0 \mathclose[\) tel que \( f_n(x_1)=y\).

            \end{subproof}

            Nous avons prouvé que pour tout \( n\), la fonction
            \begin{equation}        \label{EQooYWHGooJWMTUI}
                f_n\colon \mathopen[ 0 , \infty \mathclose[\to \mathopen[ 0 , \infty \mathclose[
            \end{equation}
            est une bijection.

        \item[Pour \( n\) impair]

            Nous montrons à présent que si \( n\) est impair, alors
            \begin{equation}        \label{EQooTSLJooMAAUXH}
                f_n\colon \mathopen] -\infty , 0 \mathclose]\to \mathopen] -\infty , 0 \mathclose]
            \end{equation}
            est une bijection.

            Tout se base sur le fait que si \( x>0\) alors \( f_n(-x)=-f_n(x)\). Le fait que \eqref{EQooYWHGooJWMTUI} soit injective et surjective montre alors tout de suite le fait que \eqref{EQooTSLJooMAAUXH} soit également injective et surjective.
    \end{subproof}
\end{proof}

Vous noterez que la continuité de \( f_n\) démontrée dans la proposition \ref{PROPooXQYFooPxoEHE} est indépendant de la proposition \ref{LEMooUAFBooAwiXxj} qui sera invoquée plus tard pour définir \( a^x\) lorsque \( a>0\) dans \( \eR\).

%--------------------------------------------------------------------------------------------------------------------------- 
\subsection{Sur les rationnels, racines}
%---------------------------------------------------------------------------------------------------------------------------

\begin{definition}[Exposant rationnels]        \label{DEFooJWQLooWkOBxQ}
    La proposition \ref{PROPooXQYFooPxoEHE} nous dit entre autres que pour tout \( n\in \eN\), la fonction
    \begin{equation}
        \begin{aligned}
            f_n\colon \mathopen[ 0 , \infty \mathclose[&\to \mathopen[ 0 , \infty \mathclose[ \\
            x&\mapsto x^n 
        \end{aligned}
    \end{equation}
    est bijective. Nous définissions alors, pour \( a\in \mathopen[ 0 , \infty \mathclose[\),
    \begin{equation}
        a^{1/n}=f_n^{-1}(a).
    \end{equation}
    Autrement dit, le nombre \( a^{1/n}\) est l'unique solution positive de
    \begin{equation}
        x^n=a.
    \end{equation}
\end{definition}

\begin{normaltext}      \label{NORMooDUNZooUNdUKg}
    Nous ne définissons pas \( a^{1/n}\) pour \( a<0\), du moins pas encore. Vu que \( f_3\) est bijective sur \( \eR\), il serait tentant de définir \( (-1)^{1/3}=f_3^{-1}(-1)=-1\).

    Cela causera un certain nombre de problèmes plus tard vu que nous aurons envie de deux choses en même temps :
    \begin{itemize}
        \item d'une part \( \ln(-1)=i\pi\),
        \item d'autre part, \( a^x= e^{x\ln(a)}\).
    \end{itemize}
    De cette façon, nous devrions avoir
    \begin{equation}
        (-1)^{1/3}= e^{i\pi /3},
    \end{equation}
    qui est un nombre complexe non réel. Voici un exemple de ce que ça donne avec Sage :
    \lstinputlisting{tex/sage/sageSnip019.sage}
\end{normaltext}

\begin{definition}[Racince]
    Pour \( n\in \eN\) nous définissons \( \sqrt[n]{ x }=f_n^{-1}(x)\). Lorsque \( n\) est pair, la fonction \( x\mapsto\sqrt[n]{ x }\) n'est définie que sur \( \eR^+\), et lorsque \( n\) est impair, elle est définie sur tout \( \eR\).
\end{definition}

\begin{normaltext}      \label{NORMooYPRNooWCjEgR}
    Notons que les fonctions \( x\mapsto \sqrt[3]{ x }\) et \( x\mapsto x^{1/3}\) ne sont pas les mêmes : la première est définie sur tout \( \eR\) et donne des valeurs réelles tandis que la seconde n'est (pour l'instant) définie que sur les positifs, et donnera (quand on l'aura définie par l'exponentielle) des nombres complexes sur les négatifs.

    En suivant cette convention, c'est à dire en réservant la notation \( \sqrt{  }\) pour l'inverse de \( f_2\), nous ne devrions pas écrire des choses comme «\( \sqrt{ -1 }=i\)», mais plutôt «\( (-1)^{1/2}=i \)». En effet, \( \sqrt{ -1 }\) n'est pas défini et ne sera jamais défini alors que \( (-1)^{1/2}\) n'est pas encore défini, mais sera défini par 
    \begin{equation}
        (-1)^{1/2}= e^{\frac{ 1 }{2}\ln(-1)}= e^{i\pi/2}=i.
    \end{equation}
\end{normaltext}

En résumé, nous avons les fonctions suivantes :
\begin{enumerate}
    \item
        \( \sqrt[n]{  }\colon \eR\to \eR\) si \( n\) est impair,
    \item
        \( \sqrt[n]{  }\colon \mathopen[ 0 , \infty \mathclose[\to \mathopen[ 0 , \infty \mathclose[ \) si \( n\) est pair,
    \item
        \( x^{1/n}\colon \mathopen[ 0 , \infty \mathclose[\to \mathopen[ 0 , \infty \mathclose[\) pour tout \( n\in \eN\).
\end{enumerate}
Cependant nous n'hésiterons pas à utiliser la notation \( \sqrt{ x }\) pour \( x^{1/2}\) même lorsque \( x\) est négatif, parce c'est une notation très pratique. Il faut garder en tête que cette façon de faire est incohérente parce qu'elle inciterait à penser que \( \sqrt[3]{-1  }= e^{i\pi/3}\) au lieu de \( \sqrt[3]{-1  }=-1\).

Pour toute la suite de cette section, nous allons considérer \( a^x\) uniquement pour \( a>0\).

\begin{definition}
    Pour \( m,n\in \eN\) nous définissons 
    \begin{equation}        \label{EQooZFOAooTsMbub}
        a^{m/n}=(a^m)^{1/n},
    \end{equation}
    ce qui définit la fonction puissance sur \( \eQ^+\). Enfin nous posons
    \begin{equation}
        a^{-q}=\frac{1}{ a^q }
    \end{equation}
    lorsque \( q\in \eQ^+\).

    Et avec tout ça, lorsque \( a>0\) nous avons défini \( a^q\) pour tout \( q\in \eQ\).
\end{definition}

Nous allons souvent noter la définition \eqref{EQooZFOAooTsMbub} sous la forme
\begin{equation}        \label{EQooZIKKooVfjkZo}
    f_{m/n}(x)^n=x^m.
\end{equation}

\begin{lemma}[\cite{MonCerveau}]        \label{LEMooIDLJooZALNaD}
    Pour \( a>0\) et \( p,q\in \eZ\) nous avons :
    \begin{equation}
        a^{p/q}=(a^p)^{1/q}=(a^{1/q})^p.
    \end{equation}
\end{lemma}

\begin{proof}
    Nous divisons la preuve en fonction de la positivité du numérateur et du dénominateur.
    \begin{subproof}
        \item[Numérateur et dénominateurs positifs]
                
            Nous commençons avec \( p,q\in \eN\). La première égalité est la définition \ref{DEFooJWQLooWkOBxQ}. Pour la seconde, la définition de \( (a^p)^{1/q}\) est d'être le \( x>0\) tel que
            \begin{equation}
                x^q=a^p.
            \end{equation}
            La définition de \( a^{1/q}\) est d'être le \( y>0\) tel que
            \begin{equation}
                y^q=a.
            \end{equation}
            Ce \( y\) vérifie donc aussi \( y^{pq}=a^p\) et donc \( (y^p)^q=a^p\). Autrement dit, \( y^p=x\), c'est à dire exactement
            \begin{equation}
                (a^{1/q})^p=(a^p)^{1/q}.
            \end{equation}
            Le lemme est prouvé dans le cas où \( p,q\in \eN\).

        \item[Numérateur et dénominateur négatifs]

            Si \( p\) et \( q\) sont tous les deux négatifs, nous remarquons que \( p/q=(-p)/(-q)\) et nous sommes dans le même cas qu'avant.

        \item[Numérateur négatif, dénominateur positif]

            Pour simplifier les notations nous supposons toujours \( p,q\in \eN\) mais nous considérons \( a^{(-p)/q}\). Nous avons d'une part :
            \begin{equation}
                a^{(-p)/q}=a^{-(p/q)}=\frac{1}{ a^{p/q} }=\frac{1}{ (a^{1/q})^p }=(a^{1/q})^{-p}.   
            \end{equation}
            Dans ce calcul, nous avons utilisé au dénominateur le résultat dans le cas positif. 

            Et d'autre part nous avons :
            \begin{equation}
                (a^{-p})^{1/q}=\left( \left( \frac{1}{ a } \right)^p \right)^{1/q}=\left( \left( \frac{1}{ a } \right)^{1/q} \right)^p=\left( \frac{1}{ a^{1/q} } \right)^p=\frac{1}{ (a^{1/q})^p }=(a^{1/q})^{-p}
            \end{equation}
            où nous avons utilisé le résultat avec \( 1/a\) en guise de \( a\).

        \item[Numérateur positif, dénominateur négatif]

            Nous traitons maintenant \( a^{p/(-q)}\). Nous avons d'une part
            \begin{equation}
                a^{p/(-q)}=a^{-(p/q)}=\frac{1}{ a^{p/q} }=\frac{1}{ (a^p)^{1/q} }=(a^p)^{-(1/q)}=(a^p)^{1/(-q)}.
            \end{equation}
            Et d'autre part :
            \begin{equation}
                a^{p/(-q)}=\frac{1}{ a^{p/q} }=\frac{1}{ (a^{1/q})^p }=\left( \frac{1}{ a^{1/q} } \right)^p=\left( a^{-(1/q)} \right)^p=(a^{1/(-q)})^p.
            \end{equation}
    \end{subproof}
\end{proof}

Le lemme suivant montre que la définition sur \( \eQ^-\) est cohérente avec celle sur \( \eQ^+\), au sens où finalement nous retrouvons que \( a^{m/n}\) vérifie \( x^n=a^m \) quel que soient les signes de \( m\) et \( n\).
\begin{lemma}[\cite{MonCerveau}]
    Le nombre \( y=a^{-m/n}\) vérifie l'équation \( y^{-n}=a^m\)
\end{lemma}

\begin{proof}
    Nous posons \( x=a^{m/n}\), c'est à dire \( x^n=a^m\). Nous avons, par définition \( y=a^{-m/n}=\frac{1}{ x }\). Alors
    \begin{equation}
        y^{-n}=\frac{1}{ \left( \frac{1}{ x } \right)^n }=x^n=a^m,
    \end{equation}
    donc c'est bon.
\end{proof}

\begin{lemma}[\cite{MonCerveau}]        \label{LEMooJYGUooHhLASp}
    Pour \( a>0\) et \( q,q'\in \eQ\) nous avons
    \begin{equation}
        a^qa^{q'}=a^{q+q'}.
    \end{equation}
\end{lemma}

\begin{proof}
    Nous mettons \( q\) et \( q'\) au même dénominateur. Soient \( q=s/c\) et \( q'=r/c\) avec \( s,r\in \eZ\) et \( c\in \eN\). En utilisant les égalités du lemme \ref{LEMooIDLJooZALNaD} nous trouvons
    \begin{equation}
        a^{s/c}a^{r/c}=(a^{1/c})^s(a^{1/c})^r=(a^{1/c})^{s+r}=a^{(s+r)/c}=a^{q+q'}.
    \end{equation}
\end{proof}

\begin{lemma}[\cite{MonCerveau}]        \label{LEMooXJXUooLoiTMo}
    La fonction puissance prend les valeurs suivantes.
    \begin{enumerate}
        \item
            Si \( a=1\) alors \( a^q=1\) pour tout \( q\in \eQ\).
        \item       \label{ITEMooKZCGooKskUQx}
            Si \( a>1\) alors 
            \begin{itemize}
                \item \( a^q>1\) si \( q>0\)
                \item \( a^q<1\) si \( q<0\)
                \item \( a^0=1\).
            \end{itemize}
        \item
            Si \( a<1\) alors 
            \begin{itemize}
                \item \( a^q<1\) si \( q>0\)
                \item \( a^q>1\) si \( q<0\)
                \item \( a^0=1\).
            \end{itemize}
    \end{enumerate}
\end{lemma}

\begin{proof}
    Si \( a=1\) alors \( a^k=1\) pour tout \( k\in \eN\). Ensuite, pour \( m,n\in \eN\), \( a^{n/m}\) est solution de \( x^m=a^n=1\), donc \( x=1\). En ce qui concerne les puissances négatives, \( 1/1=1\).

    Si \( a>1\) alors \( a^k>1\) pour tout \( k\in \eN\). De plus pour \( q>0\) nous avons \( q=m/n\) avec \( m,n\in \eN\). Alors \( a^{m/n}\) est solution de \( x^m=a^n>1\). Or pour \( x\leq 1\) nous avons \( x^m\leq 1\), donc la solution à \( x^m=a^n\) vérifie forcément \( x>1\).

    Toujours avec \( a>1\), si \( q<0\) nous posons \( q=-q'\) avec \( q'>0\). Alors
    \begin{equation}
        a^q=q^{-q'}=\frac{1}{ a^{q'} }.
    \end{equation}
    Mais \( a^{q'}>1\), donc l'inverse est inférieur à \( 1\).

    En ce qui concerne les cas \( a<1\), ils sont obtenus en posant \( b=1/a\) et en calculant
    \begin{equation}
        a^q=\left( \frac{1}{ b } \right)^q=\frac{1}{ b^q }=b^{-q}.
    \end{equation}
\end{proof}

\begin{proposition}[\cite{MonCerveau}]\label{PROPooVXKBooQPPjMn}
    Si \( a>1\) et si \( M>0\), il existe \( n\in \eN\) tel que \( a^n>M\).
\end{proposition}

\begin{proof}
    Soit \( a=1+h\). Alors en utilisant la formule du binôme, 
    \begin{equation}
        a^n=(1+h)^n=\sum_{k=0}^n{n\choose k}h^{n-k}.
    \end{equation}
    Tous les termes de la somme sont strictement positifs. Prenons le terme \( k=n-1\). Il vaut
    \begin{equation}
        {n\choose n-1}h=nh.
    \end{equation}
    Donc \( a^n\geq nh\), donc oui, cela peut être rendu arbitrairement grand avec \( n\) sans toucher à \( a\).
\end{proof}

\begin{proposition}[\cite{MonCerveau}]      \label{PROPooGCBZooTcyGtO}
    Pour \( a>0\) nous considérons la fonction
    \begin{equation}
        \begin{aligned}
            g_a\colon \eQ&\to \eR \\
            q&\mapsto a^q.
        \end{aligned}
    \end{equation}
    \begin{enumerate}
        \item
            Si \( a\in \mathopen] 0 , 1 \mathclose[\) alors \( g_a\) est décroissante et
            \begin{subequations}
                \begin{align}
                    \lim_{q\to \infty} g_a(q)=0,&& \lim_{q\to -\infty} g_a(q)=\infty.
                \end{align}
           \end{subequations}
         \item
            Si \( a>1\)  alors \( g_a\) est croissante et
            \begin{subequations}
                \begin{align}
                    \lim_{q\to \infty} g_a(q)=\infty,&& \lim_{q\to -\infty} g_a(q)=0.
                \end{align}
           \end{subequations}
    \end{enumerate}
\end{proposition}

\begin{proof}
    Nous prouvons le cas \( a>1\). L'autre cas s'en déduit en posant \( b=1/a\). Pour la croissance, soient \( q\in \eQ\) et \( r>0\) dans \( \eQ\). En utilisant le lemme \ref{LEMooJYGUooHhLASp}, nous avons
    \begin{equation}
        a^{q+r}=a^qa^r>a^q
    \end{equation}
    parce que \( a^r>1\) par le lemme \ref{LEMooXJXUooLoiTMo}. 

    En ce qui concerne la limite \( q\to \infty\), la fonction \( g_a\) est croissante et non bornée par la proposition \ref{PROPooVXKBooQPPjMn}. Donc sa limite est \( \infty\).

    Pour la limite \( q\to -\infty\), nous avons
    \begin{equation}
        \lim_{q\to -\infty} a^q=\lim_{q\to \infty} a^{-q}=\lim_{q\to \infty} \frac{1}{ a^q }=0.
    \end{equation}
\end{proof}

\begin{proposition}[\cite{MonCerveau}]      \label{PROPooIIDGooTRtlUD}
    Soit \( a>0\). Nous avons
    \begin{equation}
        \lim_{x\to 0} a^x=1.
    \end{equation}
    Notons que dans cette limite, \( x\in \eQ\) parce que nous n'avons même pas encore défini \( a^x\) lorsque \( x\) est irrationnel.
\end{proposition}

\begin{proof}
    Nous notons, comme à l'accoutumée, \( g_a(x)=a^x\). Soit une suite \( x_k\to 0\) (avec \( x_k\neq 0\) pour tout \( k\)). En définissant \( y_k\) par \( x_k=1/y_k\) nous savons que \( a^{1/y_k}\) est la solution de \( x^{y_k}=a\).

    Nous posons \( t_k=a^{x_k}\) et notre but est de prouver que \( t_k\to 1\). Pour tout \( k\) nous avons la relation
    \begin{equation}
        t_k^{y_k}=a.
    \end{equation}
    Soit \( s>1\). Il existe un \( M>0\) tel que \( y_k>M\) implique \( s^{y_k}>a\) (proposition \ref{PROPooVXKBooQPPjMn}). Donc dès que \( y_k>M\) nous avons \( t_k<s\).

    De la même manière, si \( r<1\), il existe un \( R>0\) tel que \( y_k>R\) implique \( r^{y_k}<a\). Donc dès que \( y_k>R\) nous avons \( t_k>r\).

    Soit donc un voisinage \( \mathopen] r , s \mathclose[\) de \( 1\) (avec \( r<1\) et \( s>1\)). Nous avons les nombres \( M\) et \( R\) correspondant et nous posons \( L=\max\{ M,R \}\). Soit \( K\) tel que \( k>K\) implique \( y_k>L\). Alors pour \( k>K\) nous avons aussi \( t_k<s\) et \( t_k>r\), c'est à dire \( t_k\in \mathopen] r , s \mathclose[\).

        Cela prouve que \( t_k\to 1\).

        Donc pour toute suite \( x_k\to 0\) nous avons \( g_a(x_k)\to 1\). Par le critère séquentiel de la limite (proposition \ref{PROPooJYOOooZWocoq}) nous avons \( \lim_{x\to 0} g_a(x)=1\).
\end{proof}

\begin{lemma}       \label{LEMooKDBPooLQwxMD}
    Soit \( a>0\). La fonction
    \begin{equation}
        \begin{aligned}
            g_a\colon \eQ&\to \eR \\
            x&\mapsto a^x 
        \end{aligned}
    \end{equation}
    est continue.
\end{lemma}

\begin{proof}
    Soient \( x\in \eQ\) et une suite \( x_k\to 0\) (toujours dans \( \eQ\)) et utilisons le lemme \ref{LEMooJYGUooHhLASp} :
    \begin{equation}
        a^{x+x_k}=a^xa^{x_k}.
    \end{equation}
    Cela est, dans \( \eR\), le produit entre une constante (\( a^x\)) et une suite. La limite est donc le produit de cette constante et la limite de la suite (si elle existe). Par la proposition \ref{PROPooIIDGooTRtlUD} nous avons la limite \( a^{x_k}\to 1\), et donc
    \begin{equation}
        \lim_{k\to \infty} a^{x+x_k}=a^x,
    \end{equation}
    ce qui prouve la continuité (caractérisation séquentielle, proposition \ref{PropFnContParSuite}) de \( g_a\).
\end{proof}

\begin{definition}[Fonction puissance\cite{MonCerveau}]  \label{DEFooOJMKooJgcCtq}
    Si \( a>0\) et \( x\in \eR\), la fonction
    \begin{equation}
        \begin{aligned}
            g_a\colon \eQ&\to \eR \\
            x&\mapsto a^x. 
        \end{aligned}
    \end{equation}
    est continue par le lemme \ref{LEMooKDBPooLQwxMD}. Si \( x\in \eR\) nous définissons
    \begin{equation}
        a^x=\tilde g_a(x)
    \end{equation}
    où \( \tilde g_a\) est l'extension de \( g_a\) donnée par le lemme \ref{LEMooUAFBooAwiXxj}.

    Nous allons la noter \( g_a\) également, et écrire \( a^x\) la valeur de \( g_a\) même lorsque \( x\) n'est pas un rationnel.
\end{definition}


\begin{proposition}[\cite{MonCerveau}]      \label{PROPooVADRooLCLOzP}
    Quelque propriétés de la fonction puissance. 
    \begin{enumerate}
        \item       \label{ITEMooQHYRooJIewyp}
            Pour \( a>0\), la fonction \( g_a\colon x\mapsto a^x\) est continue sur \( \eR\).
        \item       \label{ITEMooIZBLooSGtWIp}
            Pour \( a>1\), la fonction \( g_a\colon x\mapsto a^x\) est croissante.
        \item       \label{ITEMooSCJBooNVJZah}
            Pour \( a>0\) et \( x,y\in \eR\) nous avons
            \begin{equation}        \label{EQooEWIHooDRAQGR}
                a^xa^y=a^{x+y}.
            \end{equation}
            En particulier, 
            \begin{equation}
                a^{-x}=\frac{1}{ a^x }.
            \end{equation}
    \end{enumerate}
\end{proposition}

\begin{proof}
    La continuité de \( x\mapsto a^x\) est par construction. Le point \ref{ITEMooQHYRooJIewyp} est fait.

    Pour le point \ref{ITEMooIZBLooSGtWIp}, lorsque \( a>1\), la fonction \( f_a\colon \eQ\to \eR\) est croissante (proposition \ref{PROPooGCBZooTcyGtO}). Donc par la proposition \ref{PROPooTNIAooNAJDzL}, la fonction \( x\mapsto a^x\) est croissante sur \( \eR\).

    Et enfin pour le point \ref{ITEMooSCJBooNVJZah}, il faut faire un peu plus attention. Soient des suites \( x_i\to x\) et \( y_i\to y\) dans \( \eQ\). Calculons :
    \begin{subequations}        \label{SUBEQSooMPNLooPoyjwJ}
        \begin{align}
            a^xa^y&=(\lim_ia^{x_i})a^y      \label{SUBEQooOCIOooZcewMo} \\
            &=\lim_i\big( a^{x_i}a^y \big)   \label{SUBEQooEKQXooPLqzcG}\\
            &=\lim_i\big( \lim_ka^{x_i}a^{y_k} \big)    \label{SUBEQooZEXDooRytDvS}\\
            &=\lim_i\big( \lim_k a^{x_i+y_k} \big)     \label{SUBEQooSYNBooIQZJzl}\\
            &=\lim_ia^{x_i+y}                           \label{SUBEQooKHKCooGwaPDQ}\\
            &=a^{x+y}.                                  \label{SUBEQooMZBFooSoSgKU}
        \end{align}
    \end{subequations}
    Justifications :
    \begin{itemize}
        \item Pour \ref{SUBEQooOCIOooZcewMo}. Définition de \( a^x\) lorsque \( x\in \eR\).
        \item Pour \ref{SUBEQooEKQXooPLqzcG}. Nous entrons le nombre \( a^y\) dans la limite. Entrer un facteur dans une limite convergente dans \( \eR\) est un acte anodin.
        \item Pour \ref{SUBEQooZEXDooRytDvS}. Définition de \( a^y\), et renter le nombre réel \( a^{x_i}\) dans la limite sur \( k\).
        \item Pour \ref{SUBEQooSYNBooIQZJzl}. Utilisation du lemme \ref{LEMooJYGUooHhLASp}, valable pour \( x_i,y_k\in \eQ\).
        \item Pour \ref{SUBEQooKHKCooGwaPDQ}. Pour \( i\) fixé, la suite \( k\mapsto x_i+x_k\) est une suite de rationnels qui converge vers le réel \( x_i+y\). Par définition \ref{DEFooOJMKooJgcCtq} de la fonction puissance nous avons alors \( \lim_ka^{x_i+y_k}=a^{x_i+y}\).
        \item Pour \ref{SUBEQooMZBFooSoSgKU}. La suite de réels \( i\mapsto x_i+y\) converge dans \( \eR\) vers le réel \( x+y\). Par la continuité de \( t\mapsto a^t\) (ça fait partie du lemme \ref{LEMooUAFBooAwiXxj} définissant la fonction puissance sur \( \eR\)) nous avons \( \lim_ia^{x_i+y}=a^{x+y}\).
    \end{itemize}

    Vous remarquerez que les limites sur \( k\) et sur \( i\) ne s'enlèvent pas tout à fait avec la même justification. Nous aurions pu invoquer la continuité sur \( \eR\) de \( t\mapsto a^t\) pour les deux limites. Mais cette continuité, dans le cas d'une suite purement constituée de rationnels, est la définition de la prolongation vers \( \eR\).
\end{proof}

\begin{lemma}       \label{LEMooIPLLooCgpCIn}
    Soient \( a,b>0\). Si \( 1<x<y\) alors
    \begin{equation}
        a-b<ay-bx.
    \end{equation}
\end{lemma}

\begin{proof}
    Nous posons \( y=x+s\) avec \( s>0\). Alors
    \begin{equation}
        ay-bx=a(x+s)-bx=(a-b)x+as>(a-b)x>a-b
    \end{equation}
    parce que \( as>0\) et \( x>1\).
\end{proof}

\begin{proposition}[\cite{MonCerveau}]      \label{PROPooJXHFooCDwxCS}
    Pour \( q>0\) dans \( \eQ\), la fonction
    \begin{equation}
        \begin{aligned}
            f_{q}\colon \eQ^+&\to \eR \\
                x&\mapsto x^{q} 
        \end{aligned}
    \end{equation}
    est strictement croissante.
\end{proposition}

\begin{proof}
    Division selon la généralité de \(q\).
    \begin{subproof}
        \item[Si \( q\) est entier positif]
            Soit \( q=n\in \eN\). Si \( s>0\) alors l'inégalité \( (x+s)^n>x^n\) découle du binôme de Newton de la proposition \ref{PropBinomFExOiL}.
        \item[Si \( q\) est rationnel]
            Soient un rationnel \( q=m/n\) et un nombre strictement positif \( s\). Nous avons, par la définition \ref{DEFooJWQLooWkOBxQ} sous la forme \eqref{EQooZIKKooVfjkZo} :
            \begin{equation}
                f_{m/n}(x+s)^n=(x+s)^m>x^m=f_{m/n}(x)^n.
            \end{equation}
            Nous avons utilisé la stricte croissance de \( x\mapsto x^m\). Cela donne
            \begin{equation}
                f_{m/n}(x+s)^n>f_{m/n}(x)^n.
            \end{equation}
            En utilisant encore la stricte croissance de \( x\mapsto x^n\), nous avons le résultat.
    \end{subproof}
\end{proof}

\begin{corollary}       \label{CORooYWNNooLwKmiD}
    Soient \( 1<b<a\) dans \( \eR\) et des rationnels strictement positifs \( p<q\). Alors
    \begin{equation}
        a^p-b^p<a^q-b^q
    \end{equation}
\end{corollary}

\begin{proof}
    Nous notons \( q=p+r\) avec \( r>0\) dans \( \eQ\). Par la proposition \ref{PROPooJXHFooCDwxCS},
    \begin{equation}
        a^r>b^r.
    \end{equation}
    Cela nous permet d'utilise le lemme \ref{LEMooIPLLooCgpCIn} pour écrire
    \begin{equation}
        a^p-b^p<a^pa^r-b^pb^r=a^q-b^q.
    \end{equation}
\end{proof}

\begin{proposition}[\cite{MonCerveau}]      \label{PROPooKWRGooMTbRdU}
    Soient \( a,b>0\) et \( \alpha\in \eR\). Nous avons :
    \begin{equation}
        a^{\alpha}b^{\alpha}=(ab)^{\alpha}.
    \end{equation}
\end{proposition}

\begin{proof}
    Nous supposons que c'est bon pour \( \alpha\in \eN\) et \( \alpha\in \eZ\). Pour les autres, nous donnons plus de détails.
    \begin{subproof}
        \item[\( \eQ^+\)]
            Soit \( q=m/n\) avec \( m,n\in \eN\). Si \( a^{m/n}=x\) et \( b^{m/n}=y\), alors
            \begin{subequations}
                \begin{align}
                    x^n&=a^m    \label{EQooGNMAooQJMNsL}\\
                    y^n&=b^m
                \end{align}
            \end{subequations}
            par \eqref{EQooZIKKooVfjkZo}. Nous multiplions \eqref{EQooGNMAooQJMNsL} par \( y^n\) à gauche et par \( b^m\) à droite : \( x^ny^n=a^mb^m\). En tenant compte du résultat pour \( \eN\), nous avons
            \begin{equation}
                (xy)^n=(ab)^m,
            \end{equation}
            ce qui signifie que le nombre \( xy\) est \( (ab)^{m/n}\).
        \item[Pour \( \eQ^-\)]
            Soit \( q\in \eQ^+ \), nous avons le calcul
            \begin{equation}
                a^{-q}b^{-q}=\frac{1}{ a^qb^q }=\frac{1}{ (ab)^q }=(ab)^{-q}.
            \end{equation}
        \item[Pour \( \eR\)]
            Soit une suite de rationnels \( \alpha_i\to \alpha\). Nous avons
            \begin{equation}
                a^{\alpha}b^{\alpha}=\big( \lim_ia^{\alpha_i} \big)\big( \lim_j b^{\alpha_j} \big)=\lim_i\big( a^{\alpha_i}b^{\alpha_i}\big)=\lim_i(ab)^{\alpha_i}=(ab)^{\alpha}.
            \end{equation}
            Justifications :
            \begin{itemize}
                \item la proposition \ref{PROPooIQOAooJPMoDD} pour le produit des limites,
                \item le résultat dans \( \eQ\) que nous venons de prouver,
                \item la définition de \( (ab)^{\alpha}\) comme limite de \( (ab)^{\alpha_i}\).
            \end{itemize}
    \end{subproof}
\end{proof}
    
Pour rappel, la proposition suivantes, dans le cas de \( \alpha\in \eQ^+\) est la proposition \ref{PROPooJXHFooCDwxCS}.
\begin{proposition}[\cite{MonCerveau}]      \label{PROPooRXLNooWkPGsO}
    Pour \( \alpha>0\), la fonction
    \begin{equation}
        \begin{aligned}
            f_{\alpha}\colon\mathopen] 0 , \infty \mathclose[&\to \eR \\
                x&\mapsto x^{\alpha} 
        \end{aligned}
    \end{equation}
    est strictement croissante.

    Aussi, la fonction
    \begin{equation}
        \begin{aligned}
        f_{\alpha}\colon  \mathopen] -\infty , 0 \mathclose[  &\to \eR \\
                x&\mapsto x^{\alpha} 
        \end{aligned}
    \end{equation}
    est strictement décroissante.
\end{proposition}

\begin{proof}
    Nous rappellons que le cas \( \alpha\in \eQ^+\) est déjà traité par la proposition \ref{PROPooJXHFooCDwxCS}. Soient \( x\in \mathopen] 0 , \infty \mathclose[\) et \( s>0\). Nous allons montrer que \( f_{\alpha}(x+s)-f_{\alpha}(x)>0\). Pour cela nous décomposons en plusieurs cas.
    \begin{subproof}
        \item[\( x>1\)]
            Par la proposition \ref{PROPooFGBOooHiZqbs}, nous considérons une suite strictement croissante de rationnels strictement positifs \( \alpha_i\to \alpha\). Pour tout \( i\) nous avons \( \alpha_i>\alpha_0\).

            En utilisant la stricte croissance de \( f_{\alpha_0}\) et le lemme \ref{LEMooXJXUooLoiTMo}\ref{ITEMooKZCGooKskUQx}, nous avons les inégalités \( 1<x^{\alpha_0}<(x+s)^{\alpha_0}\), et en particulier
            \begin{equation}
                0<(x+s)^{\alpha_0}-x^{\alpha_0}.
            \end{equation}
            De plus nous avons \( 1<x<x+s\) et \( \alpha_0<\alpha_i\) pour tout \( i\). Donc le corollaire \ref{CORooYWNNooLwKmiD} s'applique et nous avons, pour tout \( i\) :
            \begin{equation}
                0<(x+s)^{\alpha_0}-x^{\alpha_0}<(x+s)^{\alpha_i}-x^{\alpha_i}.
            \end{equation}
            C'est le moment de passer à la limite \( i\to \infty\). La seconde inégalité devient non stricte, mais la première reste :
            \begin{equation}
                0<(x+s)^{\alpha_0}-x^{\alpha_0}\leq(x+s)^{\alpha}-x^{\alpha}.
            \end{equation}
            Nous avons donc bien la stricte croissance de \( f_{\alpha}\) sur \( \mathopen] 1 , \infty \mathclose[\).
        \item[\( x\leq 1\)]
            Nous choisissons encore \( \alpha_i\to \alpha\) strictement croissante dans \( \eQ\). Pour chaque \( i\), nous avons encore
            \begin{equation}
                (x+s)^{\alpha_i}-x^{\alpha_i}>0.
            \end{equation}
            Le passage à la limite change l'inégalité stricte en inégalité large, et ne permet donc pas de conclure immédiatement. Nous devons donc ruser. Soit \( k\in \eN\) tel que \( k(x+s)>1\) et \( kx>1\) (existence parce que \( \eR\) est archimédien, proposition \ref{ThoooKJTTooCaxEny}). Nous avons :
            \begin{equation}
                \big( k(x+s) \big)^{\alpha}-(kx)^{\alpha}>0
            \end{equation}
            par la partie «\( x>1\)» que nous venons de prouver. Grâce à la proposition \ref{PROPooKWRGooMTbRdU} nous pouvons factoriser \( k^{\alpha}\) :
            \begin{equation}
                0<\big( k(x+s) \big)^{\alpha}-(kx)^{\alpha}=k^{\alpha}\big( (x+s)^{\alpha}-x^{\alpha} \big).
            \end{equation}
            Vu que \( k^{\alpha}>0\), cela implique \( (x+s)^{\alpha}-x^{\alpha}>0\), ce qu'il fallait.
    \end{subproof}
Nous avons fini de prouver que la fonction \( f_{\alpha}\) était strictement croissante sur \( \mathopen] 0 , \infty \mathclose[\). En ce qui concerne la fonction \( f_{\alpha}\) sur \( \mathopen] -\infty , 0 \mathclose[\), nous avons, pour \( x>0\) que
    \begin{equation}
        f_{\alpha}(-x)=\frac{1}{ f_{\alpha}(x) },
    \end{equation}
    et donc stricte décroissance.
\end{proof}

Nous prouvons à présent que \( f_{\alpha}\) est localement injective; nous en avons besoin pour prouver la continuité. Or cette continuité est nécéssait à prouver que \( f_{\alpha}\) est localement bijective. Donc nous ne pouvons pas énoncer la bijectivité ici.
\begin{proposition}     \label{PROPooHKTKooCUEBjh}
    Soient \( \alpha\in \eR\) et \( x\in \eR\setminus\{ 0 \}\). Il existe un voisinage \( V\) de \( x\) sur lequel
    \begin{equation}
            f_{\alpha}\colon V \to f_{\alpha}(V) 
    \end{equation}
    est injective.
\end{proposition}

\begin{proof}
Soit \( x>0 \); nous considérons un voisinage \( V\) de \( x\) inclu à \( \mathopen] 0 , \infty \mathclose[\). Soit \( y\in V\); pour fixer les idées nous supposons \( y<x\). Par la stricte croissance de \( f_{\alpha}\) sur \( \mathopen] 0 , \infty \mathclose[\) (proposition \ref{PROPooRXLNooWkPGsO}), nous avons \( f_{\alpha}(y)<f_{\alpha}(x)\) et en particulier \( f_{\alpha}(x)\neq f_{\alpha}(y)\).

Le cas \( x<0\) se traite de façon analogue, avec la stricte décroissance de \( f_{\alpha}\) sur \( \mathopen] -\infty , 0 \mathclose[\).
\end{proof}
Notons que les voisinages sur lesquels \( f_{\alpha}\) est injective sont assez grands. Ils peuvent être toute une demi-droite, si l'on veut.

\begin{lemma}   \label{LEMooQTNKooLVEytN}
    Soient \( \alpha>0\), une suite de rationnels strictement décroissante \( \alpha_i\to \alpha\) ainsi que les fonctions
    \begin{equation}
        \begin{aligned}
        f_{\alpha_i}\colon \mathopen] 1 , \infty \mathclose[&\to \eR \\
            x&\mapsto x^{\alpha_i}. 
        \end{aligned}
    \end{equation}
    La famille \( \{ f_{\alpha_i} \}_{i\in \eN}\) est équicontinue\footnote{Définition \ref{DEFooSGMVooASNbxo}.}.
\end{lemma}

\begin{proof}
    Soient \( x>1\), et \( \alpha>0\). Nous allons montrer que \( \{ f_{\alpha_i} \}\) est équicontinue en \( x\). Soit \( s\) tel que \( 1<x<x+s\); le corollaire \ref{CORooYWNNooLwKmiD} nous enseigne que 
    \begin{equation}
        (x+s)^p-x^p<(x+s)^q-x^q
    \end{equation}
    dès que \( p<q\). En particulier, \( f_p\) étant croissante par la proposition \ref{PROPooRXLNooWkPGsO},
    \begin{equation}
        0<(x+s)^{\alpha_i}-x^{\alpha_i}<(x+s)^{\alpha_0}-x^{\alpha_0}.
    \end{equation}
    Soit \( \epsilon>0\) et \( \delta\) tel que \( s<\delta\) implique \( | (x+s)^{\alpha_0}-x^{\alpha_0} |<\epsilon\). Alors nous avons aussi, pour de tels \( \sigma\) et \( s\) :
    \begin{equation}
        |(x+s)^{\alpha_i}-x^{\alpha_i}|<|(x+s)^{\alpha_0}-x^{\alpha_0}|<\epsilon.
    \end{equation}
    En procédant de même\ pour \( s<0\), nous trouvons bien que
    \begin{equation}
        | y^{\alpha_i}-x^{\alpha_i} |\leq \epsilon
    \end{equation}
    pour tout \( y\in B(x,\delta)\).

    Cela signifie que \( \{ f_i \}\) est équicontinue.
\end{proof}


\begin{proposition}[\cite{MonCerveau}]      \label{PROPooUQNZooSSHLqr}
    Soit \( a>0\) dans \( \eR\). La fonction
    \begin{equation}
        \begin{aligned}
            f_{\alpha}\colon \eR&\to \eR \\
            x&\mapsto x^{\alpha} 
        \end{aligned}
    \end{equation}
    est continue (sauf pour \( x=0\) si \( \alpha<0\)).
\end{proposition}

\begin{proof}
    Nous allons subdiviser quelque cas.
    \begin{subproof}
        \item[Pour \( \alpha\in \eN\)]
            Nous supposons que ce cas va bien.
        \item[Pour \( \alpha\in \eQ^+\)]
            Soit \( q=m/n\) avec \( m,n\in \eN\). Soit aussi \( \epsilon>0\). Nous avons :
            \begin{subequations}
                \begin{align}
                    f_{m/n}(x)^n&=x^m\\
                    f_{m/n}(x+\epsilon)^n&=(x+\epsilon)^m.      \label{SUBEQooGNCSooWAeRcL}
                \end{align}
            \end{subequations}
            L'équation \eqref{SUBEQooGNCSooWAeRcL} s'écrit aussi bien sous la forme
            \begin{equation}
                f_n\big( f_{m/n}(x+\epsilon) \big)=(x+\epsilon)^m.
            \end{equation}
            En prenant la limite,
            \begin{equation}
                \lim_{\epsilon\to 0}\big[ f_n\big( f_{m/n}(x+\epsilon) \big) \big]=x^m=f_{m/n}(x)^n.
            \end{equation}
            Vu que \( f_n\) est continue, nous pouvons la permuter avec la limite dans le membre de gauche tout en écrivant \( f_{m/n}(x)^n=f_n\big( f_{m/n}(x) \big)\) dans le membre de droite :
            \begin{equation}
                f_n\big[ \lim_{\epsilon\to 0}f_{m/n}(x+\epsilon) \big]=f_n\big( f_{m/n}(x) \big).
            \end{equation}
            La fonction \( f_n\) étant injective dans un voisinage autour de \( x\) (proposition \ref{PROPooHKTKooCUEBjh}),
            \begin{equation}
                \lim_{\epsilon\to 0}f_{m/n}(x+\epsilon)=f_{m/n}(x),
            \end{equation}
            ce qui est la continuité de \( f_{m/n}\) en \( x\).

        \item[Pour \( \alpha\in \eR^+\)]

        Nous prouvons séparément le cas \( x<1\) et le cas \( x\geq 1\). Commençons par \( x\in \mathopen] 1 , \infty \mathclose[\).

            Soit une suite \( \alpha_i\to \alpha\) strictement décroissante dans \( \eQ^+\). Le lemme \ref{LEMooQTNKooLVEytN} nous dit que l'ensemble de fonctions  \( \{ f_{\alpha_i}\colon \mathopen] 1 , \infty \mathclose[\to R \}_{i\in \eN}\) est équicontinu. La convergence simple \( f_{\alpha_i}\to f_{\alpha}\) étant par définition, la proposition \ref{PROPooICNNooAMjcut} nous dit que la fonction \( f_{\alpha}\colon \mathopen] 1 , \infty \mathclose[\to \eR\) est continue.

            Soit maintenant \( x\in \mathopen] 0 , 1 \mathclose]\). Il existe \( k\in \eN\) tel que \( kx>1\), \( k(x/2)>1\) et \( k^{\alpha}>1\) (si vous pensez bien, seule la première condition est utile).

            Nous considérons \( \epsilon\) tel que \( x+\epsilon>x/2\); de toutes façons nous comptions faire \( \epsilon\to 0\). Nous avons :
            \begin{equation}
                \big| (x+\epsilon)^{\alpha}-x^{\alpha} \big|\leq k^{\alpha}\big| (x+\epsilon)^{\alpha}-x^{\alpha} \big|=\big| [k(x+\epsilon)]^{\alpha}-(kx)^{\alpha} \big|.
            \end{equation}
            Nous prenons le \( \delta\) qui correspond à \( \epsilon\) en \( kx\) dans la continuité de \( f_{\alpha}\) déjà démontée pour \( kx>1\). Alors si \( \epsilon<\delta\) nous avons
            \begin{equation}
                \big| (x+\epsilon)^{\alpha}-x^{\alpha} \big|\leq\epsilon.
            \end{equation}
        \item[Pour \( \alpha\in \eR^{-}\)]

            Si \( \alpha>0\), la fonction \( f_{-\alpha}\) est donnée par
            \begin{equation}
                f_{-\alpha}(x)=\frac{1}{  f_{\alpha}(x) }
            \end{equation}
            et est donc continue (sauf en \( x=0\) où elle n'existe pas).
    \end{subproof}
\end{proof}

\begin{proposition}[\cite{MonCerveau}]     \label{PROPooDWZKooNwXsdV}
    Soient \( a>0\) ainsi que \( x,y\in \eR\). Alors
    \begin{equation}
        (a^x)^y=(a^y)^x=a^{xy}.
    \end{equation}
\end{proposition}

\begin{proof}
    Nous découpons en fonction de la nature de \( x\) et \( y\). 

    \begin{subproof}
        \item[\( x\) rationnel, \( y\) naturel]
            Si \( q\in \eQ\) et \( n\in \eN\) alors la formule
            \begin{equation}
                (a^q)^n=a^{nq}
            \end{equation}
            découle seulement d'une récurrence sur la formule \ref{EQooEWIHooDRAQGR}.

        \item[ \( x,y\in \eQ\)]
            Soient \( y=m/n\) avec \( n\in \eZ\), \( m\in \eN\) et \( q\in \eQ\). Nous avons, en utilisant la partie déjà démontrée et le lemme \ref{LEMooIDLJooZALNaD},
            \begin{equation}
                (a^q)^{p}=(a)^{m/n}=\big( (a^q)^m \big)^{1/n}=(a^{mq})^{1/n}=a^{mq/n}=a^{pq}.
            \end{equation}
        \item[\( x,y\) irrationnels]

            Soient des suites des rationnels \( x_i\to x\) et \( y_i\to y\). En utilisant les définitions,
            \begin{equation}        \label{EQooXITUooHYNSPU}
                (a^x)^y=\lim_i(a^x)^{y_i}=\lim_i\big( \lim_j a^{x_j} \big)^{y_i}.
            \end{equation}
            Fixons un \(i\) pour commencer. Nous avons, par la continuité de \( f_{y_i}\) (proposition \ref{PROPooUQNZooSSHLqr})
            \begin{equation}
                \big( \lim_ja^{x_j} \big)^{y_i}=f_{y_i}\big( \lim_ja^{x_j} \big)=\lim_j\big( f_{y_i}(a^{x_j}) \big)=\lim_ja^{x_jy_i}.
            \end{equation}
            Nous avons utilisé le résultats déjà démontré dans le cas des rationnels. La suite \( j\mapsto x_jy_i\) est une suite dans \( \eQ\) qui converge vers le réel \( xy_i\), donc la limite sur \( j\) redonne la fonction puissance :
            \begin{equation}        \label{EQooWORSooFoRBod}
                \big( \lim_ja^{x_j} \big)^{y_i}=\lim_ja^{x_jy_i}=a^{xy_i}.
            \end{equation}
            Le résultat découle maintenant de la prise de limite dans \eqref{EQooXITUooHYNSPU} qui revient à prendre la limite \( i\to \infty\) de l'expression dans \eqref{EQooWORSooFoRBod} :
            \begin{equation}
                (a^x)^y=\lim_i\big( \lim_j a^{x_j} \big)^{y_i}=\lim_ia^{xy_i}=a^{xy}.
            \end{equation}
    \end{subproof}
\end{proof}

Le lemme suivant montre en gros que \( x^y\) croît plus rapidement en \( y\) qu'en \( x\). 
\begin{lemma}       \label{LemLJOSooEiNtTs}
     Pour tout \( \alpha>0\) et \( a<1\) nous avons la limite
     \begin{equation}
         \lim_{n\to \infty} n^{\alpha}a^n=0
     \end{equation}
\end{lemma}

\begin{proof}
    Soit \( k\in \eN\) plus grand que \( \alpha\).
    Soit la suite numérique \( s_n=n^ka^n\). Tous ses termes sont positifs et
    \begin{equation}
        \frac{ s_n }{ s_{n+1} }=\left( \frac{ n }{ n+1 } \right)^k\frac{1}{ a }.
    \end{equation}
    Étant donné que \( n/n+1\to 1\) et que \( a<1\), il existe un certain rang à partir duquel la suite \( (s_n)\) est décroissante. Deux conclusions :
    \begin{itemize}
        \item Elle est majorée par une constante \( M\).
        \item Elle est convergente par le lemme~\ref{LemSuiteCrBorncv}.
    \end{itemize}
    Soit \( l\) tel que \( ka^l<1\) et \( n>l\) alors
    \begin{equation}
        s_{n+l}=(n+l)^ka^{n+l}\leq kn^ka^na^l=ka^ls_n\leq ka^lM.
    \end{equation}
    La majoration est due au fait que dans \( (n+l)^k\) nous avons \( k\) termes tous plus petits que \( n^k\). De la même façon,
    \begin{equation}
        s_{2n+2l}\leq ka^{2l}s_{2n}\leq ka^{2l}M.
    \end{equation}
    En posant \( \varphi(i)=in+il\) nous avons
    \begin{equation}
        s_{\varphi(i)}\leq ka^iM,
    \end{equation}
    qui est une sous-suite convergente vers \( 0\). Or si une suite est convergente (ce qui est le cas de \( (s_n)\)), toutes les sous-suites convergent vers la même limite. Nous en concluons que \( s_n\to 0\).
\end{proof}

\begin{normaltext}
    Une conséquence est que si vous voulez choisir un mot de passe fort, la longueur du mot est plus importante que la taille de l'alphabet choisit : il est plus efficace de choisir une combinaison longue qu'une combinaisons mélangeant des lettres, chiffres et symboles spéciaux.
    
    Exemple : si vous choisissez un mot de passe contenant majuscules, minuscules, chiffres et symboles spéciaux complètement mélangés (ne mentez pas, vous ne le faites pas), mais que vous ne le choisissez que de taille \( 6\), vous avez \( 72^6\) possibilités (en supposant un jeu de 10 symboles spéciaux).

    Eh bien, en seulement \( 8\) lettres minuscules, vous avez plus de possibilités : \( 26^8>72^6\).
\end{normaltext}

%--------------------------------------------------------------------------------------------------------------------------- 
\subsection{La fonction puissance : remarques pour la suite}
%---------------------------------------------------------------------------------------------------------------------------

Il y a encore de nombreuses choses à dire sur la fonction puissance. Pour savoir lesquelles, voir le thème \ref{THEMEooBSBLooWcaQnR}.

