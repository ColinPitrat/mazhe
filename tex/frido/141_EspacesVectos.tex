% This is part of Mes notes de mathématique
% Copyright (c) 2011-2018
%   Laurent Claessens, Carlotta Donadello
% See the file fdl-1.3.txt for copying conditions.


%+++++++++++++++++++++++++++++++++++++++++++++++++++++++++++++++++++++++++++++++++++++++++++++++++++++++++++++++++++++++++++
\section{Théorème de Burnside}
%+++++++++++++++++++++++++++++++++++++++++++++++++++++++++++++++++++++++++++++++++++++++++++++++++++++++++++++++++++++++++++

\begin{lemma}       \label{LemwXXzIt}
    Soit \( P\), un polynôme sur \( \eK\). Une racine de \( P\) est une racine simple si et seulement si elle n'est pas racine de \( P'\).
\end{lemma}

\begin{theorem}     \label{ThoBurnsideoPuCtS}
    Toute représentation d'un groupe abélien d'exposant fini sur \( \eC^n\) a une image finie.
\end{theorem}

\begin{proof}
    Dans le cas d'un groupe abélien, la démonstration est facile. Étant donné que \( G\) est d'exposant fini, il existe \( \alpha\in \eN^*\) tel que \( g^{\alpha}=e\) pour tout \( g\in G\). Le polynôme \( P(X)=X^{\alpha}-1\) est scindé à racines simples. En effet tout polynôme sur \( \eC\) est scindé. Le fait qu'il soit à racines simples provient du lemme~\ref{LemwXXzIt} parce que si \( a^{\alpha}=1\), alors il n'est pas possible d'avoir \( \alpha a^{\alpha-1}=0\).

    Par ailleurs \( P(g)=0\). Le fait que nous ayons un polynôme annulateur de \( g\) scindé à racines simples implique que \( g\) est diagonalisable (théorème~\ref{ThoDigLEQEXR}). Le fait que \( G\) soit abélien montre qu'il existe une base de \( \eC^n\) dans laquelle tous les éléments de \( G\) sont diagonaux. Nous devons par conséquent montrer qu'il existe un nombre fini de matrices de la forme
    \begin{equation}
        \begin{pmatrix}
            \lambda_1    &       &       \\
                &   \ddots    &       \\
                &       &   \lambda_n
        \end{pmatrix}.
    \end{equation}
    Nous savons que \( \lambda_i^{\alpha}=1\) parce que \( g^{\alpha}=\mtu\), par conséquent chacun des \( \lambda_i\) est une racine de l'unité dont il n'existe qu'un nombre fini.
\end{proof}

\begin{theorem}[Burnside\cite{fJhCTE,ooFBZQooXyHIWK}]\label{ThooJLTit}
    Un sous-groupe de \( \GL(n,\eC)\) est fini si et seulement s'il est d'exposant\footnote{Définition~\ref{DefvtSAyb}.} fini.
\end{theorem}
\index{exposant}
\index{racine!de l'unité}
\index{endomorphisme!diagonalisable}

\begin{proof}
    Soit \( G\) un sous-groupe de \( \GL(n,\eC)\). Si \( G\) est fini, l'ordre de ses éléments divise \( | G |\) (corollaire~\ref{CorpZItFX}) au théorème de Lagrange et l'exposant est le PPCM qui est donc fini également.

    Nous supposons maintenant que l'ordre de \( G\) est fini. Nous notons \( e\) l'exposant de \( G\). En particulier, tous les éléments de \( G\) sont des racines du polynôme \( X^e-1\).

    \begin{subproof}
        \item[Générateurs]

            Le groupe \( G\) est une partie de \( \eM(n,\eC)\) dont nous considérons l'algèbre engendrée (définition~\ref{DefkAXaWY}) \( \mG\). Soit \( C_1,\ldots, C_r\) une famille génératrice de \( \mG\) constituée d'éléments de \( G\) et la fonction
            \begin{equation}
                \begin{aligned}
                    \tau\colon G&\to \eC^r \\
                    A&\mapsto \big( \tr(AC_1),\ldots, \tr(AC_r) \big).
                \end{aligned}
            \end{equation}

        \item[\( \tau\) est injective] Supposons que \( \tau(A)=\tau(B)\). Alors pour tout générateur \( C_i\) nous avons \( \tr(AC_i)=\tr(BC_i)\) et par linéarité de la trace, nous avons
            \begin{equation}    \label{EqnCYmKW}
                \tr(AM)=\tr(BM)
            \end{equation}
            pour tout \( M\in G\). Notons par ailleurs
            \begin{equation}
                N=AB^{-1}-\mtu,
            \end{equation}
            qui est diagonalisable parce que \( AB^{-1}\in G\) et donc est annulé par le polynôme \( X^e-1\) qui est scindé à racines simples. Du coup \( AB^{-1}\) est diagonalisable; posons \( PAB^{-1}P^{-1}=D\), alors \( P\big( AB^{-1}-\mtu \big)P^{-1}=D-\mtu\) qui est encore diagonale. Donc \( N\) est diagonalisable.

            Par ailleurs nous avons
            \begin{subequations}
                \begin{align}
                    \tr\big( (AB^{-1})^p \big)&=\tr\big( AB^{-1}(AB^{-1})^{p-1} \big)\\
                    &=\tr\big( BB^{-1}(AB^{-1})^{p-1} \big) &\text{\eqref{EqnCYmKW}}\\
                    &=\tr\big( (AB^{-1})^{p-1} \big).
                \end{align}
            \end{subequations}
            En continuant nous obtenons
            \begin{equation}
                \tr\big(  (AB^{-1})^p \big)=\tr(\mtu)=n.
            \end{equation}

            D'autre part,
            \begin{equation}
                N^k=(AB^{-1}-\mtu)^k=\sum_{p=0}^k{p\choose k}(-1)^{k-p}(AB^{-1})^p
            \end{equation}
            En prenant la trace, et en tenant compte du fait que \( \tr\big( (AB^{-1})^p \big)=n\),
            \begin{equation}
                \tr(N^k)=\sum_{p=0}^k{p\choose k}(-1)^{k-p}n=n(1-1)^k=0.
            \end{equation}
            Donc la trace de \( N^k\) est nulle et le lemme~\ref{LemzgNOjY} nous enseigne que \( N\) est alors nilpotente. Étant donné qu'elle est aussi diagonalisable, elle est nulle. Nous en concluons que \( AB^{-1}=\mtu\) et donc que \( A=B\). La fonction \( \tau\) est donc injective.

        \item[Nombre fini de valeurs]

            Les éléments de \( G\) sont annulés par \( X^e-1\) qui est un polynôme scindé à racines simples. Dons le polynôme minimal d'un élément de \( G\) est (a fortiori) scindé à racines simples et le théorème~\ref{ThoDigLEQEXR} nous assure alors que ces éléments sont diagonalisables. Du coup les valeurs propres des matrices de \( G\) sont des racines \( e\)ièmes de l'unité. Par conséquent les traces des éléments de \( G\) ne peuvent prendre qu'un nombre fini de valeurs : toutes les sommes de \( n\) racines \( e\)ièmes de l'unité. Mais vu que les \( C_i\) sont dans \( G\), nous avons
            \begin{equation}
                \Image(\tau)=\{ \tr(A)\tq A\in G \}^r,
            \end{equation}
            qui est un ensemble fini. Par conséquent \( G\) est fini parce que \( \tau\) est injective.
    \end{subproof}
\end{proof}

%---------------------------------------------------------------------------------------------------------------------------
\subsection{Théorème de Lie-Kolchin}
%---------------------------------------------------------------------------------------------------------------------------

Contrairement à ce que l'on peut parfois croire, il n'est pas vrai que toute matrice à coefficient réel est diagonalisable, même pas sur \( \eC\). La raison est qu'une telle matrice peut très bien avoir des valeurs propres multiples.

\begin{example} \label{ExBRXUooIlUnSx}
    Le théorème~\ref{ThoDigLEQEXR} nous donne une façon simple de trouver des matrices non diagonalisables sur \( \eC\) : il suffit que le polynôme minimal ne soit pas scindé à racines simples. Par exemple
    \begin{equation}
        A=\begin{pmatrix}
            1    &   1    \\
            0    &   1
        \end{pmatrix},
    \end{equation}
    dont le polynôme caractéristique est \( \chi_A=(1-X)^2\). Ce polynôme n'a manifestement pas des racines simples. Nous pouvons faire le calcul explicite pour montrer que \( A\) n'est pas diagonalisable. D'abord l'unique valeur propre de \( A\) est \( 1\) et nous pouvons sans peine résoudre
    \begin{equation}
        \begin{pmatrix}
            1    &   1    \\
            0    &   1
        \end{pmatrix}\begin{pmatrix}
            x    \\
            y
        \end{pmatrix}=\begin{pmatrix}
            x    \\
            y
        \end{pmatrix}
    \end{equation}
    qui revient au système
    \begin{subequations}
        \begin{numcases}{}
            x+y=x\\
            y=y.
        \end{numcases}
    \end{subequations}
    La première équation donne directement \( y=0\). Le seul espace propre est de dimension \( 1\) et est engendré par \( \begin{pmatrix}
        1    \\
        0
    \end{pmatrix}\).
\end{example}

La remarque~\ref{RemBOGooCLMwyb} donne un exemple un peu plus avancé, qui montre la multiplicité algébrique et géométrique d'une racine d'un polynôme caractéristique.

\begin{lemma}[Trigonalisation simultanée]   \label{LemSLGPooIghEPI}
    Une famille de matrices de \( \GL(n,\eC)\) commutant deux à deux est simultanément trigonalisable.
\end{lemma}
\index{trigonalisation!simultanée}

\begin{proof}
    Commençons par enfoncer une porte ouverte par la proposition~\ref{PropKNVFooQflQsJ} : sur \( \GL(n,\eC)\) toutes les matrices sont trigonalisables parce que tous les polynômes sont scindés.

    Nous effectuons la démonstration par récurrence sur la dimension. Si \( n=1\) alors toutes les matrices sont triangulaires et nous ne nous posons pas de questions. Nous supposons donc \( n>1\).

    Soit la famille \( (A_i)_{i\in I}\) dans \( \GL(n,\eC)\) et \( A_0\) un de ses éléments. Nous nommons \( \lambda_1,\ldots, \lambda_r\) les valeurs propres distinctes de \( A_0\). Le théorème de décomposition primaire~\ref{ThoSpectraluRMLok} nous donne la somme directe d'espaces caractéristiques\footnote{Définition~\ref{DefFBNIooCGbIix}.}
    \begin{equation}
        E=F_{\lambda_1}(A_0)\oplus\ldots\oplus F_{\lambda_r}(A_0).
    \end{equation}
    Nous pouvons supposer que cette somme n'est pas réduite à un seul terme. En effet si tel était le cas, \( A_0\) serait un multiple de l'identité parce que \( A_0\) n'aurait qu'une seule valeur propre et les sommes dans la décomposition de Dunford~\ref{ThoRURcpW}\ref{ItemThoRURcpWiii} se réduisent à un seul terme (et \( p_i=\id\)). En particulier les dimensions des espaces \( F_{\lambda}(A_0)\) sont strictement plus petites que \( n\).

    Vu que tous les \( A_i\) commutent avec \( A_0\), les espaces \( F_{\lambda}(A_0)\) sont stables par les \( A_i\) et nous pouvons trigonaliser les \( A_i\) simultanément sur chacun des \( F_{\lambda}(A_0)\) en utilisant l'hypothèse de récurrence.
\end{proof}

\begin{theorem}[Lie-Kolchin\cite{PAXrsMn}]  \label{ThoUWQBooCvutTO}
    Tout sous-groupe connexe et résoluble de \( \GL(n,\eC)\) est conjugué à un groupe de matrices triangulaires.
\end{theorem}
\index{trigonalisation!simultanée}
\index{théorème!Lie-Kolchin}

\begin{proof}
    Soit \( G\) un sous-groupe connexe et résoluble de \( \GL(n,\eC)\).

    \begin{subproof}
        \item[Si sous-espace non trivial stable par \( G\)]

    Nous commençons par voir ce qu'il se passe s'il existe un sous-espace vectoriel non trivial \( V\) de \( \eC^n\) stabilisé par \( G\). Pour cela nous considérons une base de \( \eC^n\) dont les premiers éléments forment une base de \( V\) (base incomplète, théorème~\ref{ThonmnWKs}). Les éléments de \( G\) s'écrivent, dans cette base,
    \begin{equation}    \label{EqGOKTooEaGACG}
        \begin{pmatrix}
            g_1    &   *    \\
            0    &   g_2
        \end{pmatrix}.
    \end{equation}
    Les matrices \( g_1\) et \( g_2\) sont carrés. Nous considérons alors l'application \( \psi\) définie par
    \begin{equation}
        \begin{aligned}
            \psi\colon G&\to \GL(V) \\
            g&\mapsto g_1.
        \end{aligned}
    \end{equation}
    Cela est un morphisme de groupes parce que
    \begin{equation}
        \begin{pmatrix}
            g_1    &   *    \\
            0    &   g_2
        \end{pmatrix}\begin{pmatrix}
            h_1    &   *    \\
            0    &   h_2
        \end{pmatrix}=
        \begin{pmatrix}
            g_1h_1    &   *    \\
            0    &   g_2h_2
        \end{pmatrix},
    \end{equation}
    de telle sorte que \( \psi(gh)=\psi(g)\psi(h)\).

    Le groupe \( \psi(G)\) est connexe et résoluble. En effet \( \psi(G)\) est connexe en tant qu'image d'un connexe par une application continue (proposition~\ref{PropGWMVzqb}). Et il est résoluble en tant qu'image d'un groupe résoluble par un homomorphisme par la proposition~\ref{PropBNEZooJMDFIB}. Vu que \( \psi(G)\) est un sous-groupe résoluble et connexe de \( \GL(V)\) et que la dimension de \( V\) est strictement plis petite que celle de \( \eC^n\), une récurrence sur la dimension indique que \( \psi(G)\) est conjugué à un groupe de matrices triangulaires. C'est à dire qu'il existe une base de \( V\) dans laquelle toutes les matrices \( g_1\) (avec \( g\in G\)) sont triangulaires supérieures.

    On fait de même avec l'application \( g\mapsto g_2\), ce qui donne une base du supplémentaire de \( V\) dans laquelle les matrices \( g_2\) sont triangulaires.

    En couplant ces deux bases, nous obtenons une base de \( \eC^n\) dans laquelle toutes les matrices \eqref{EqGOKTooEaGACG} (c'est à dire toutes les matrices de \( G\)) sont triangulaires supérieures.

    \item[Sinon]

    Nous supposons à présent que \( \eC^n\) n'a pas de sous-espaces non triviaux stables sous \( G\). Nous posons \( m=\min\{ k\tq D^k(G)=\{ e \} \}\), qui existe parce que \( G\) et résoluble et que sa suite dérivée termine sur \( {e}\) (proposition~\ref{PropRWYZooTarnmm}).

\item[Si \( m=1\)]

    Si \( m=1\) alors \( G\) est abélien et il existe une base de \( G\) dans laquelle toutes les matrices de \( G\) sont triangulaires (lemme~\ref{LemSLGPooIghEPI}). Le premier vecteur d'une telle base serait stable par \( G\), mais comme nous avons supposé qu'il n'y avait pas de sous-espaces non triviaux stabilisés par \( G\), il faut déduire que ce vecteur stable est à lui tout seul non trivial, c'est à dire que \( n=1\). Dans ce cas, le théorème est démontré.

\item[Si \( m>1\)]

    Nous devons maintenant traiter le cas où \( m>1\). Nous posons \( H=D^{m-1}(G)\); cela est un sous-groupe normal et abélien de \( G\). Encore une fois le résultat de trigonalisation simultanée~\ref{LemSLGPooIghEPI} donne une base dans laquelle tous les éléments de \( H\) sont triangulaires. En particulier le premier élément de cette base est un vecteur propre commun à toutes les matrices de \( H\).

    Soit \( V\) le sous-espace engendré par tous les vecteurs propres communs de \( H\). Nous venons de voir que \( V\) n'est pas vide. Nous allons montrer que \( V\) est stable par \( G\). Soient \( h\in H\), \( v\in V\) et \( g\in G\) :
    \begin{equation}    \label{EqPMOBooVLIhrJ}
        h\big( g(v) \big)=g\underbrace{g^{-1}hg}_{\in H}(v)=g(\lambda v)=\lambda g(v)
    \end{equation}
    parce que \( v\) est vecteur propre de \( g^{-1} hg\). Ce que le calcul \eqref{EqPMOBooVLIhrJ} montre est que \( g(v)\) est vecteur propre de \( h\) pour la valeur propre \( \lambda\). Donc \( g(v)\in V\) et \( V\) est stabilisé par \( G\). Mais comme il n'existe pas d'espaces non triviaux stabilisés par \( G\), nous en déduisons que \( V=\eC^n\). Donc tous les vecteurs de \( \eC^n\) sont vecteurs propres communs de \( H\). Autrement dit on a une base de diagonalisation simultanée de \( H\).

\item[\( H\) est dans le centre de \( G\)]

    Montrons à présent que \( H\) est dans le centre de \( G\), c'est à dire que pour tout \( g\in G\) et \( h\in H\) il faut \( ghg^{-1}=h\). D'abord \( ghg^{-1}\) est une matrice diagonale (parce que elle est dans \( H\)) ayant les mêmes valeurs propres que \( h\). En effet si \( \lambda\) est valeur propre de \( ghg^{-1}\) pour le vecteur propre \( v\), alors
    \begin{subequations}
        \begin{align}
            (ghg^{-1})(v)&=\lambda v\\
            h\big( g^{-1} v \big)&=\lambda \big( g^{-1}v \big),
        \end{align}
    \end{subequations}
    c'est à dire que \( \lambda\) est également valeur propre de \( h\), pour le vecteur propre \( g^{-1} v\). Mais comme \( h\) a un nombre fini de valeurs propres, il n'y a qu'un nombre fini de matrices diagonales ayant les mêmes valeurs propres que \( h\). L'ensemble \( \AD(G)h\) est donc un ensemble fini. D'autre part, l'application \( g\mapsto g^{-1}hg\) est continue, et \( G\) est connexe, donc l'ensemble \( \AD(G)h\) est connexe. Un ensemble fini et connexe dans \( \GL(n,\eC)\) est nécessairement réduit à un seul point. Cela prouve que \( ghg^{-1}=h\) pour tout \( g\in G\) et \( h\in H\).

\item[Espaces propres stables pour tout \( G\)]

        Soit \( h\in H\) et \( W\) un espace propre de \( h\) (ça existe non vide parce que \( H\) est triangularisé, voir plus haut). Alors nous allons prouver que \( W\) est stable pour tous les éléments de \( G\). En effet si \( w\in W\) avec \( h(w)=\lambda w\) alors en permutant \( g\) et \( h\),
        \begin{equation}
            hg(w)=g(hw)=\lambda g(w),
        \end{equation}
        donc \( g(w)\) est aussi vecteur propre de \( h\) pour la valeurs propre \( \lambda\), c'est à dire que \( g(w)\in W\). Vu que nous supposons que \( \eC^n\) n'a pas d'espaces invariants non triviaux, nous devons conclure que \( W=\eC^n\), c'est à dire que \( H\) est composé d'homothéties. C'est à dire que pour tout \( h\in H\) nous avons \( h=\lambda_h\mtu\).

    \item[Contradiction sur la minimalité de \( m\)]

        Les éléments d'un groupe dérivé sont de déterminant \( 1\) parce que \( \det(g_1g_2g_1^{-1}g_2^{-1})=1\). Par conséquent pour tout \( h\), le nombre \( \lambda_h\) est une racine \( n\)\ieme de l'unité. Vu qu'il n'y a qu'une quantité finie de racines \( n\)\ieme de l'unité, le groupe \( H\) est fini et connexe et donc une fois de plus réduit à un élément, c'est à dire \( H=\{ e \}\). Cela contredit la minimalité de \( m\) et donc produit une contradiction. Nous devons donc avoir \( m=1\).

    \item[Conclusion]

        Nous avons vu que si \( \eC^n\) avait un sous-espace non trivial fixé par \( G\) alors le théorème était démontré. Par ailleurs si \( \eC^n\) n'a pas un tel sous-espace, soit \( m=1\) (et alors le théorème est également prouvé), soit \( m>1\) et alors on a une contradiction.

        Bref, le théorème est prouvé sous peine de contradiction.
    \end{subproof}
\end{proof}

%+++++++++++++++++++++++++++++++++++++++++++++++++++++++++++++++++++++++++++++++++++++++++++++++++++++++++++++++++++++++++++
\section{Retour sur les formes bilinéaires et quadratiques}
%+++++++++++++++++++++++++++++++++++++++++++++++++++++++++++++++++++++++++++++++++++++++++++++++++++++++++++++++++++++++++++

%---------------------------------------------------------------------------------------------------------------------------
\subsection{Topologie}
%---------------------------------------------------------------------------------------------------------------------------

La topologie considérée sur \( Q(E)\) est celle de la norme
\begin{equation}    \label{EqZYBooZysmVh}
    N(q)=\sup_{\| x \|_E=1}| q(x) |,
\end{equation}
qui du point de vue de \( S(n,\eR)\) est
\begin{equation}    \label{EQooJETQooIjxRWu}
    N(A)=\sup_{\| x \|_E=1}| x^tAx |.
\end{equation}
Notons que à droite, c'est la valeur absolue usuelle sur \( \eR\).

\begin{proposition} \label{PropFSXooRUMzdb}
    Soit \( \{ e_i \}\) une base de \( E\). L'application
    \begin{equation}
        \begin{aligned}
            \phi\colon Q(E)&\to S(n,\eR) \\
            q&\mapsto \big(   b(e_i,e_j)   \big)_{i,j}
        \end{aligned}
    \end{equation}
    où \( b\) est forme bilinéaire associée à \( q\) est une bijection linéaire et continue\footnote{Pour les topologies des normes \eqref{EqZYBooZysmVh} et \eqref{EQooJETQooIjxRWu}.}.
\end{proposition}

\begin{proof}
    Si \( \phi(q)=\phi(q')\); alors
    \begin{equation}
        q(x)=\sum_{i,j}\phi(q)_{ij}x_ix_j=\sum_{i,j}\phi(q')_{ij}x_ix_j=q'(x).
    \end{equation}
    Donc \( q=q'\). L'application \( \phi\) est donc injective

    De plus elle est surjective parce que si \( B\in S(n,\eR)\) alors la forme quadratique
    \begin{equation}
        q(x)=\sum_{i,j}B_{ij}x_ix_j
    \end{equation}
    a évidemment \( B\) comme matrice associée. L'application \( \phi\) est donc surjective.

    Notre application \( \phi\) est de plus linéaire parce que l'association d'une forme quadratique à la forme bilinéaire associée est linéaire.

    En ce qui concerne la continuité, nous la prouvons en zéro en considérant une suite convergente \( q_n\stackrel{Q(E)}{\longrightarrow}0\). C'est à dire que
    \begin{equation}
        \sup_{\| x \|=1}| q_n(x) |\to 0.
    \end{equation}
    Nous rappelons l'identité de polarisation :
    \begin{equation}
        b_n(x,y)=\frac{ 1 }{2}\big( q_n(x-y)-q(x)-q(y) \big).
    \end{equation}
    En ce qui concerne deux des trois termes, il n'y a pas de problèmes :
    \begin{equation}
        \big| \phi(q_n)_{ij} \big|=\big| b_n(e_i,e_j) \big|\leq\frac{ 1 }{2}\big| b_n(e_i-e_j) \big|+\frac{ 1 }{2}\big| q_n(e_i) \big|+\frac{ 1 }{2}\big| q_n(e_j) \big|.
    \end{equation}
    Si \( n\) est assez grand, nous avons tout de suite
    \begin{equation}
        \big| \phi(q_n)_{ij} \big|\leq \frac{ 1 }{2}\big| q_n(e_i-e_j) \big|+\epsilon.
    \end{equation}
    Nous définissons \( e_{ij}\) et \( \alpha_{ij}\) de telle sorte que \( e_i-e_j=\alpha_{ij}e_{ij}\) avec \( \| e_{ij} \|=1\). Si \( \alpha=\max\{ \alpha_{ij},1 \}\) alors nous avons
    \begin{equation}
        q_n(e_i-e_j)=\alpha_{ij}^2q_n(e_{ij})\leq \alpha^2q_n(e_{ij}).
    \end{equation}
    Il suffit maintenant de prendre \( n\) assez grand pour avoir \( \sup_{\| x \|=1}| q_n(x) |\leq \frac{ \epsilon }{ \alpha^2 }\) pour avoir
    \begin{equation}
        \big| \phi(q_n)_{ij} \big|\leq \frac{ \epsilon }{2}+\frac{ \epsilon }{ \alpha^2 }.
    \end{equation}
\end{proof}

%---------------------------------------------------------------------------------------------------------------------------
\subsection{Diagonalisation}
%---------------------------------------------------------------------------------------------------------------------------

\begin{proposition}\label{PropFWYooQXfcVY}
    Dans la base de diagonalisation de sa matrice associée, une forme quadratique a la forme
    \begin{equation}
        q(x)=\sum_i\lambda_ix_i^2
    \end{equation}
    où les \( \lambda_i\) sont les valeurs propres de la matrice associée à \( q\).
\end{proposition}

\begin{proof}
    Soit \( q\) une forme quadratique et \( b\) la forme bilinéaire associée. Si \( \{ f_i \}\) est une base de diagonalisation\footnote{Qui existe parce que la matrice est symétrique, théorème~\ref{ThoeTMXla}.} de la matrice de \( b\) alors dans cette base nous avons
\begin{equation}
    q(x)=b(x,x)=\sum_{ij}x_ix_jb(f_i,f_j)=\sum_i\lambda_ix_i^2
\end{equation}
où les \( \lambda_i\) sont les valeurs propres de la matrice de \( b\).
\end{proof}
Notons que si nous choisissons une autre base de diagonalisation, les \( \lambda_i\) ne changement pas (à part l'ordre éventuellement). Cela pour dire que nous nous permettrons de parler des \defe{valeurs propres}{valeur propre!d'une forme quadratique} d'une forme quadratique comme étant les valeurs propres de la matrice associée.

%---------------------------------------------------------------------------------------------------------------------------
\subsection{Dégénérescence}
%---------------------------------------------------------------------------------------------------------------------------

Soit \( b\), une forme bilinéaire symétrique non dégénérée  sur l'espace vectoriel \( E\) de dimension \( n\) sur \( \eK\) où \( \eK\) est un corps de caractéristique différente de \( 2\). Nous notons \( q\) la forme quadratique associée.

\begin{definition}
    Une forme bilinéaire est \defe{non dégénérée}{forme!bilinéaire!non dégénérée} \( b(x,z)=0\) pour tout \( z\) implique \( x=0\).
\end{definition}

\begin{lemma}   \label{LemyKJpVP}
    Soit \( b\) une forme bilinéaire non dégénérée. Si \( x\) et \( y\) sont tels que \( b(x,z)=b(y,z)\) pour tout \( z\), alors \( x=y\).
\end{lemma}

\begin{proof}
    C'est immédiat du fait de la linéarité en le premier argument et de la non-dégénérescence : si \( b(x,z)-b(y,z)=0\) alors
    \begin{equation}
        b(x-y,z)=0
    \end{equation}
    pour tout \( z\), ce qui implique \( x-y=0\).
\end{proof}

\begin{proposition}
    La forme bilinéaire \( b\) est non-dénénérée si et seulement si sa matrice associée est inversible.
\end{proposition}

\begin{proof}
    Nous savons que la matrice associée est symétrique et qu'elle peut donc être diagonalisée (théorème~\ref{ThoeTMXla}). En nous plaçant dans une base de diagonalisation, nous devons prouver que la forme est non-dégénérée si et seulement si les éléments diagonaux de la matrice sont tous non nuls.

    Écrivons \( b(x,z)\) en choisissant pour \( z\) le vecteur de base \( e_k\) de composantes \( (e_k)_j=\delta_{kj}\) :
    \begin{equation}
            b(x,e_k)=\sum_{ij}x_i(e_k)_j
            =\sum_i b_{ik}x_i
            =b_{kk}x_k.
    \end{equation}
    Si \( b\) est dégénérée et si \( x\) est un vecteur non nul (disons que la composante \( x_i\) est non nulle) de \( E\) tel que \( b(x,z)=0\) pour tout \( z\in E\), alors \( b_{ii}=0\), ce qui montre que la matrice de \( b\) n'est pas inversible.

    Réciproquement si la matrice de \( b\) est inversible, alors tous les \( b_{kk}\) sont différents de zéro, et le seul vecteur \( x\) tel que \( b_{kk}x_k=0\) pour tout \( k\) est le vecteur nul.
\end{proof}


\begin{definition}[Isotropie]   \label{DefVKMnUEM}
    Un vecteur est \defe{isotrope}{isotrope (vecteur)} pour \( b\) s'il est perpendiculaire à lui-même; en d'autres termes, \( x\) est isotrope si et seulement si \( b(x,x)=0\). Un sous-espace \( W\subset E\) est \defe{totalement isotrope}{isotrope!totalement} si pour tout \( x,y\in W\), nous avons \( b(x,y)=0\).

    Le \defe{cône isotrope}{isotrope!cône} de \( b\) est l'ensemble de ses vecteurs isotropes :
    \begin{equation}
        C(b)=\{ x\in E\tq b(x,x)=0 \}.
    \end{equation}
\end{definition}
Nous introduisons quelque notations. D'abord pour \( y\in E\) nous notons
\begin{equation}
    \begin{aligned}
        \Phi_y\colon E&\to \eR \\
        x&\mapsto b(x,y)
    \end{aligned}
\end{equation}
et ensuite
\begin{equation}
    \begin{aligned}
        \Phi\colon E&\to E^* \\
        y&\mapsto \Phi_y.
    \end{aligned}
\end{equation}
\begin{definition}
    Le fait pour une forme bilinéaire \( b\) d'être dégénérée signifie que l'application \( \Phi\) n'est pas injective. Le \defe{noyau}{noyau!d'une forme bilinéaire} de la forme bilinéaire est celui de \( \Phi\), c'est à dire
    \begin{equation}
        \ker(b)=\{ z\in E\tq b(z,y)=0\,\forall y\in E \}.
    \end{equation}
    Autrement dit, \( \ker(b)=E^{\perp}\) où le perpendiculaire est pris par rapport à \( b\).
\end{definition}
Notons tout de même que nous utilisons la notation \( \perp\) même si \( b\) est dégénérée et éventuellement pas positive; c'est à dire même si la formule \( (x,y)\mapsto b(x,y)\) ne fournit pas un produit scalaire.

\begin{proposition}[\cite{RTzQrdx}]     \label{PropHIWjdMX}
    Soit \( b\) une forme bilinéaire et symétrique. Alors
    \begin{enumerate}
        \item
            \( \ker(b)\subset C(b)\) (cône d'isotropie, définition~\ref{DefVKMnUEM})
        \item
            si \( b\) est positive alors \( \ker(b)=C(b)\).
    \end{enumerate}
\end{proposition}

\begin{proof}
    \begin{enumerate}
        \item
            Si \( z\in\ker(b)\) alors pour tout \( y\in E\) nous avons \( b(z,y)=0\). En particulier pour \( y=z\) nous avons \( b(z,z,)=0\) et donc \( z\in C(b)\).
        \item
            Soit \( b\) positive et \( x\in C(b)\). Par l'inégalité de Cauchy-Schwarz (proposition~\ref{ThoAYfEHG}) nous avons
            \begin{equation}
                | b(x,y) |\leq \sqrt{   b(x,x)b(y,y) }=0.
            \end{equation}
            Donc pour tout \( y\) nous avons \( b(x,y)=0\).
    \end{enumerate}
\end{proof}

%---------------------------------------------------------------------------------------------------------------------------
\subsection{Inégalité de Minkowski}
%---------------------------------------------------------------------------------------------------------------------------

Ce qui est couramment nommé «inégalité de Minkowski» est la proposition~\ref{PropInegMinkKUpRHg} dans les espaces \( L^p\). Nous allons en donner ici un cas très particulier.

\begin{proposition} \label{PropACHooLtsMUL}
    Si \( q\) est une forme quadratique sur \( \eR^n\) et si \( x,y\in \eR^n\) alors
    \begin{equation}
        \sqrt{q(x+y)}\leq\sqrt{q(x)}+\sqrt{q(y)}.
    \end{equation}
\end{proposition}

\begin{proof}
    La proposition~\ref{PropFWYooQXfcVY} nous permet de «diagonaliser» la forme quadratique \( q\). Quitte à ne plus avoir une base orthonormale, nous pouvons renormaliser les vecteurs de base pour avoir
    \begin{equation}
        q(x)=\sum_ix_i^2.
    \end{equation}
    Le résultat n'est donc rien d'autre que l'inégalité triangulaire pour la norme euclidienne usuelle, laquelle est démontrée dans la proposition~\ref{PropEQRooQXazLz}.
\end{proof}

%---------------------------------------------------------------------------------------------------------------------------
\subsection{Ellipsoïde}
%---------------------------------------------------------------------------------------------------------------------------

\begin{lemma}   \label{LemYVWoohcjIX}
    Toute matrice peut être décomposée de façon unique en une partie symétrique et une partie antisymétrique. Cette décomposition est donnée par
\begin{equation}\label{subEqHIQooyhiWM}
    \begin{aligned}[]
            S&=\frac{ M+M^t }{ 2 },&A&=\frac{ M-M^t }{ 2 }
    \end{aligned}
\end{equation}
\end{lemma}

\begin{proof}
    L'existence est une vérification immédiate de \( S+A=M\) en utilisant \eqref{subEqHIQooyhiWM}. Pour l'unicité, si \( S+A=S'+A'\) alors \( S-S'=A-A'\). Mais \( S-S'\) est symétrique et \( A-A'\) est antisymétrique; l'égalité implique l'annulation des deux membres, c'est à dire \( S=S'\) et \( A=A'\).
\end{proof}

\begin{definition}  \label{DefOEPooqfXsE}
    Un \defe{ellipsoïde}{ellipsoïde} dans \( \eR^n\) centré en \( v\) est le lieu des points \( x\) vérifiant l'équation
    \begin{equation}\label{EqSNWooXfbTH}
        (x-v)^t M(x-v)=1
    \end{equation}
    où \( M\) est une matrice symétrique strictement définie positive\footnote{Définition~\ref{DefAWAooCMPuVM}.}.

    Lorsque nous parlons d'ellipsoïde \emph{plein}, il suffit de changer l'égalité en une inégalité.
\end{definition}
Une autre façon d'écrire la relation \eqref{EqSNWooXfbTH} est d'écrire \( \langle (x-v),M(x,v)\rangle\) en utilisant le produit scalaire.

\begin{remark}
    Le fait que \( M\) soit symétrique n'est pas tout à fait obligatoire; la chose important est que toutes les valeurs propres soient strictement positives. En effet si \( M\) a toutes ses valeurs propres strictement positives, nous nommons \( S\) la partie symétrique de \( M\) et \( A\) la partie antisymétrique (lemme~\ref{LemYVWoohcjIX}). Alors pour tout \( x\in \eR^n\) nous avons
    \begin{equation}
        x^tAx=\langle x, Ax\rangle =\langle A^tx,x \rangle =-\langle Ax, x\rangle =-\langle x,Ax\rangle ,
    \end{equation}
    donc \( x^tAx=0\). L'équation \( x^tMx=1\) est donc équivalente à \( x^tSx=1\) (elles ont les mêmes solutions).

    De plus \( S\) reste strictement définie positive parce que pour tout \( x\in \eR^n\) nous avons
    \begin{equation}
        0<x^tMx=x^tSx.
    \end{equation}
\end{remark}

\begin{proposition}\label{PropWDRooQdJiIr}
    Si \( \ellE\) est un ellipsoïde centrée à l'origine, il existe une base de \( \eR^n\) dans laquelle son équation est :
    \begin{equation}
        \sum_{i=1}^n\frac{ x_i^2 }{ a_i^2 }=1.
    \end{equation}
\end{proposition}

\begin{proof}
    Nous avons une matrice symétrique strictement définie positive \( S\) telle que l'équation soit \( \langle x, Sx\rangle =1\). Le théorème spectral~\ref{ThoeTMXla} nous fournit une base orthonormale \( \{ e_i \}\) dans laquelle \( Se_i=\lambda_ie_i\) avec \( \lambda_i>0\). En substituant dans l'équation \( \langle x, Sx\rangle =1\) nous trouvons l'équation
    \begin{equation}
        \sum_i\lambda_ix_i^2=1.
    \end{equation}
    En posant \( a_i=\frac{1}{ \sqrt{\lambda_i} }\), nous trouvons le résultat.  Cette définition des \( a_i\) est toujours possible parce que \( \lambda_i>0\).
\end{proof}

\begin{corollary}   \label{CorKGJooOmcBzh}
    Un ellipsoïde plein centré en l'origine admet une équation de la forme \( q(x)\leq 1\) où \( q\) est une forme quadratique strictement définie positive.
\end{corollary}
Pour rappel de notation, l'ensemble des formes quadratiques strictement définies positives sur l'espace vectoriel \( E\) est noté \( Q^{++}(E)\).

\begin{proof}
    Soit \( \{ e_i \}\) une base de \( \eR^n\) telle que l'ellipsoïde \( \ellE\) ait pour équation
    \begin{equation}
        \sum_{i=1}^n\frac{ x_i^2 }{ a_i^2 }\leq 1.
    \end{equation}
    Nous considérons la forme quadratique
    \begin{equation}
        \begin{aligned}
            q\colon \eR^n&\to \eR \\
            x&\mapsto \sum_{i=1}^n\frac{ \langle x, e_i\rangle^2 }{ a_i^2 }.
        \end{aligned}
    \end{equation}
    Nous avons évidemment \( \ellE=\{ x\in \eR^n\tq q(x)\leq 1 \}\). De plus la forme \( q\) est strictement définie positive parce que dès que \( x\neq 0\), au moins un des produits scalaires \( \langle x, e_i\rangle \) est non nul et \( q(x)> 0\).
\end{proof}

%+++++++++++++++++++++++++++++++++++++++++++++++++++++++++++++++++++++++++++++++++++++++++++++++++++++++++++++++++++++++++++
\section{Théorème spectral auto-adjoint}
%+++++++++++++++++++++++++++++++++++++++++++++++++++++++++++++++++++++++++++++++++++++++++++++++++++++++++++++++++++++++++++

\begin{definition}      \label{DEFooYNEQooGQgbCf}
    Si \( E\) est un espace euclidien, un endomorphisme \( f\colon E\to E\) est \defe{auto-adjoint}{endomorphisme!auto-adjoint} si pour tout \( x,y\in E\) nous avons \( \langle x, f(y)\rangle=\langle f(x), y\rangle  \).
\end{definition}
L'ensemble des opérateurs auto-adjoints de \( E\) est noté \( \gS(E)\)\nomenclature[A]{\( \gS(E)\)}{Les opérateurs auto-adjoints de $E$}. Cette notation provient du fait que dans \( \eR^n\) muni du produit scalaire usuel, les opérateurs auto-adjoints sont les matrices symétriques.

\begin{theorem}[Théorème spectral auto-adjoint] \label{ThoRSBahHH}
    Un endomorphisme auto-adjoint d'un espace euclidien
    \begin{enumerate}
        \item
            est diagonalisable dans une base orthonormée,
        \item
            a son spectre réel.
    \end{enumerate}
\end{theorem}
\index{théorème!spectral!autoadjoint}
\index{diagonalisation!endomorphisme auto-adjoint}

\begin{proof}
    Nous procédons par récurrence sur la dimension de \( E\), et nous commençons par \( n=1\)\footnote{Dans \cite{KXjFWKA}, l'auteur commence avec \( n=0\) mais moi je n'en ai \wikipedia{en}{Vacuous_truth}{pas le courage.}.}. Soit donc \( f\colon E\to E\) avec \( \langle f(x), y\rangle =\langle x, f(y)\rangle \). Étant donné que \( f\) est également linéaire, il existe \( \lambda\in \eR\) tel que \( f(x)=\lambda x\) pour tout \( x\in E\). Tous les vecteurs de \( E\) sont donc vecteurs propres de \( f\).

    Passons à la récurrence. Nous considérons \( \dim(E)=n+1\) et \( f\in\gS(E)\). Nous considérons la forme bilinéaire symétrique \( \Phi_f\) et la forme quadratique associée \( \phi_f\). Pour rappel,
    \begin{subequations}
        \begin{align}
        \Phi_f(x,y)=\langle x, f(y)\rangle \\
        \phi_f(x)=\Phi_f(x,x).
        \end{align}
    \end{subequations}
    Et nous allons laisser tomber les indices \( f\) pour noter simplement \( \Phi\) et \( \phi\). Étant donné que \( \overline{ B(0,1) }\) est compacte et que \( \phi\) est continue, il existe \( x_0\in\overline{ B(0,1) }\) tel que
    \begin{equation}
        \lambda=\phi(x_0)=\sup_{x\in\overline{ B(0,1) }}\phi(x).
    \end{equation}
    Notons aussi que \( \| x_0 \|=1\) : le maximum est pris sur le bord. Nous posons
    \begin{equation}
        g=\lambda\id-f
    \end{equation}
    ainsi que
    \begin{equation}
        \Phi_1(x,y)=\langle x, g(y)\rangle .
    \end{equation}
    Cela est une forme bilinéaire et symétrique parce que
    \begin{equation}
        \Phi_1(y,x)=\langle y, g(x)\rangle =\langle g(y), x\rangle =\langle x, g(y)\rangle =\Phi_1(x,y)
    \end{equation}
    où nous avons utilisé le fait que \( g\) était auto-adjoint et la symétrie du produit scalaire. De plus \( \Phi_1\) est semi-définie positive parce que
    \begin{equation}
        \Phi_1(x,x)=\langle x, \lambda x-f(x)\rangle =\lambda\| x \|^2-\phi(x).
    \end{equation}
    Vu que \( \lambda\) est le maximum, nous avons tout de suite \( \Phi_1(x)\geq 0\) tant que \( \| x \|=1\). Et si \( x\) n'est pas de norme \( 1\), c'est le même prix parce qu'on se ramène à \( \| x \|=1\) en multipliant par un nombre positif. Attention cependant :
    \begin{equation}
        \Phi_1(x_0,x_0)=\lambda\| x_0 \|^2-\phi(x_0)=0.
    \end{equation}
    Donc \( \Phi_1\) a un noyau contenant \( x_0\) par la proposition~\ref{PropHIWjdMX}. Nous en déduisons que \( \Image(g)\neq E\) en effet, \( x_0\in\Image(g)^{\perp}\), mais nous avons la proposition~\ref{PropXrTDIi} sur les dimensions :
    \begin{equation}
        \dim E=\dim(\Image(g))+\dim( \Image(g)^{\perp}).
    \end{equation}
    Vu que \( \Image(g)^{\perp}\) est un espace vectoriel non réduit à \( \{ 0 \}\), la dimension de \( \Image(g)\) ne peut pas être celle de \( E\). L'endomorphisme \( g\) n'étant pas surjectif, il ne peut pas être injectif non plus parce que nous sommes en dimension finie; il existe donc \( e_1\in E\) tel que \( g(e_1)=0\) et tant qu'à faire nous choisissons \( \| e_1 \|=1\) (ici la norme est bien celle de l'espace euclidien considéré). Par définition,
    \begin{equation}
        f(e_1)=\lambda e_1,
    \end{equation}
    c'est à dire que \( \lambda\in\Spec(f)\). Et \( \phi\) étant une forme quadratique réelle nous avons \( \lambda\in \eR\).

    Nous posons à présent \( H=\Span\{ e_1 \}^{\perp}\). C'est un sous-espace stable par \( f\) parce que si \( x\in H\) alors
    \begin{equation}
        \langle e_1, f(x)\rangle =\langle f(e_1j),x\rangle =\lambda\langle e_1, x\rangle =0.
    \end{equation}
    Nous pouvons donc considérer la restriction de \( f\) à \( H\) : \( f_H\colon H\to H\). Cet endomorphisme est bilinéaire et symétrique sur l'espace \( H\) de dimension inférieure à celle de \( E\), donc la récurrence nous donne une base orthonormée
    \begin{equation}
        \{ e_2,\ldots, e_n \}
    \end{equation}
    de vecteurs propres de \( f_H\). De plus les valeurs propres sont réelles, toujours par récurrence. Donc
    \begin{equation}
        \Spec(f)=\{ \lambda \}\cup\Spec(f_H)\subset \eR.
    \end{equation}
    Notons pour être complet que si \( i\geq 2\) alors
    \begin{equation}
        \langle e_1, e_i\rangle =0
    \end{equation}
    parce que le vecteur \( e_i\) est par construction choisi dans l'espace \( H=e_1^{\perp}\). Nous avons donc bien une base orthonormée de \( E\) construite sur des vecteurs propres de \( f\).
\end{proof}

\begin{corollary}   \label{CorSMHpoVK}
    Soit \( E\) un espace vectoriel ainsi que \( \phi\) et \( \psi\) des formes quadratiques sur \( E\) avec \( \psi\) définie positive. Alors il existe une base \( \psi\)-orthonormale dans laquelle \( \phi\) est diagonale.
\end{corollary}

\begin{proof}
    Il suffit de considérer l'espace euclidien \( E\) muni du produit scalaire \( \langle x, y\rangle =\psi(x,y)\). Ensuite nous diagonalisons la matrice (symétrique) de \( \phi\) pour ce produit scalaire à l'aide du théorème~\ref{ThoRSBahHH}.
\end{proof}

\begin{definition}      \label{DefYNWUFc}
    Dans le cas de \( V=\eR^m\) nous avons un produit scalaire canonique. Soient $u$ et $v$, deux vecteurs de $\eR^m$. Le \defe{produit scalaire}{produit!scalaire!sur \( \eR^n\)} de $u$ et $v$, noté $\langle u, v\rangle $ ou $u\cdot v$ est le réel
	\begin{equation}		\label{EqDefProdScalsumii}
		\langle u, v\rangle =\sum_{k=1}^m u_kv_k=u_1v_1+u_2v_2+\cdots+u_mv_n.
	\end{equation}
\end{definition}

Calculons par exemple le produit scalaire de deux vecteurs de la base canonique : $\langle e_i, e_j\rangle $. En utilisant la formule de définition et le fait que $(e_i)_k=\delta_{ik}$, nous avons
\begin{equation}
	\langle e_i, e_j\rangle =\sum_{k=1}^m\delta_{ik}\delta_{jk}.
\end{equation}
Nous pouvons effectuer la somme sur $k$ en remarquant qu'à cause du $\delta_{ik}$, seul le terme avec $k=i$ n'est pas nul. Effectuer la somme revient donc à remplacer tous les $k$ par des $i$ :
\begin{equation}
	\langle e_i, e_j\rangle =\delta_{ii}\delta_{ji}=\delta_{ji}.
\end{equation}

Une des propriétés intéressantes du produit scalaire est qu'il permet de décomposer un vecteur dans une base, comme nous le montre la proposition suivante.

\begin{proposition}		\label{PropScalCompDec}
	Si nous notons $v_i$ les composantes du vecteur $v$, c'est à dire si $v=\sum_{i=1}^m v_ie_i$, alors nous avons $v_j=\langle v, e_j\rangle $.
\end{proposition}

\begin{proof}
	\begin{equation}		\label{Eqvejscalcomp}
		v\cdot e_j=\sum_{i=1}^m\langle v_ie_i, e_j\rangle =\sum_{i=1}^mv_i\langle e_i, e_j\rangle =\sum_{i=1}^mv_i\delta_{ij}
	\end{equation}
	En effectuant la somme sur $i$ dans le membre de droite de l'équation \eqref{Eqvejscalcomp}, tous les termes sont nuls sauf celui où $i=j$; il reste donc
	\begin{equation}
		v\cdot e_j=v_j.
	\end{equation}
\end{proof}

Le produit scalaire ne dépend en réalité pas de la base orthogonale choisie.

\begin{lemma}
	Si $\{ e_i \}$ est la base canonique, et si $\{ f_i \}$ est une autre base orthonormale, alors si $u$ et $v$ sont deux vecteurs de $\eR^m$, nous avons
	\begin{equation}
		\sum_i u_iv_j=\sum_iu'_iv'_j
	\end{equation}
	où $u_i$ sont les composantes de $u$ dans la base $\{ e_i \}$ et $u'_i$ sont celles dans la base $\{ f_i \}$.
\end{lemma}

\begin{proof}
	La preuve demande un peu d'algèbre linéaire. Étant donné que $\{ f_i \}$ est une base orthonormale, il existe une matrice $A$ orthogonale ($AA^t=\mtu$) telle que $u'_i=\sum_jA_{ij}u_j$ et idem pour $v$. Nous avons alors
	\begin{equation}
		\begin{aligned}[]
			\sum_iu'_iv'_j&=\sum_i\left( \sum_jA_{ij} u_j\right)\left( \sum_k A_{ik}v_k \right)\\
			&=\sum_{ijk}A_{ij}A_{ik}u_jv_k\\
			&=\sum_{jk}\underbrace{\sum_i(A^t)_{ji}A_{ik}}_{=\delta_{jk}}u_jv_k\\
			&=\sum_{jk}\delta_{jk}u_jv_k\\
			&=\sum_ku_jv_k.
		\end{aligned}
	\end{equation}
\end{proof}

Cette proposition nous permet de réellement parler du produit scalaire entre deux vecteurs de façon intrinsèque sans nous soucier de la base dans laquelle nous regardons les vecteurs.

Nous dirons que deux vecteurs sont \defe{orthogonaux}{orthogonal} lorsque leur produit scalaire est nul. Nous écrivons que $u\perp v$ lorsque $\langle u, v\rangle =0$.
\begin{definition}	\label{DefNormeEucleApp}
	La \defe{norme euclidienne}{norme!euclidienne!dans $\eR^m$} d'un élément de $\eR^m$ est définie par $\| u \|=\sqrt{u\cdot u}$.
\end{definition}

Cette définition est motivée par le fait que le produit scalaire $u\cdot u$ donne exactement la norme usuelle donnée par le théorème de Pythagore :
\begin{equation}
	u\cdot u=\sum_{i=1}^mu_iu_i=\sum_{i=1}^m u_i^2=u_1^2+u_2^2+\cdots+u_m^2.
\end{equation}

Le fait que $e_i\cdot e_j=\delta_{ij}$ signifie que la base canonique est \defe{orthonormée}{orthonormé}, c'est à dire que les vecteurs de la base canonique sont orthogonaux deux à deux et qu'ils ont tout $1$ comme norme.

\begin{lemma}\label{LemSclNormeXi}
	Pour tout $u\in\eR^m$, il existe un $\xi\in\eR^m$ tel que $\| u \|=\xi\cdot u$ et $\| \xi \|=1$.
\end{lemma}

\begin{proof}
	Vérifions que le vecteur $\xi=u/\| u \|$ ait les propriétés requises. D'abord $\| \xi \|=1$ parce que $u\cdot u=\| u \|^2$. Ensuite
	\begin{equation}
		\xi\cdot u=\frac{ u\cdot u }{ \| u \| }=\frac{ \| u \|^2 }{ \| u \| }=\| u \|.
	\end{equation}
\end{proof}

%+++++++++++++++++++++++++++++++++++++++++++++++++++++++++++++++++++++++++++++++++++++++++++++++++++++++++++++++++++++++++++
\section{Mini introduction au produit tensoriel}
%+++++++++++++++++++++++++++++++++++++++++++++++++++++++++++++++++++++++++++++++++++++++++++++++++++++++++++++++++++++++++++
\label{SeOOpHsn}

%---------------------------------------------------------------------------------------------------------------------------
\subsection{Définitions}
%---------------------------------------------------------------------------------------------------------------------------

Soit \( E\), un espace vectoriel de dimension finie. Si \( \alpha\) et \( \beta\) sont deux formes linéaires sur un espace vectoriel \( E\), nous définissons \( \alpha\otimes \beta\) comme étant la \( 2\)-forme donnée par
\begin{equation}
    (\alpha\otimes \beta)(u,v)=\alpha(u)\beta(v).
\end{equation}
Si \( a\) et \( b\) sont des vecteurs de \( E\), ils sont vus comme des formes sur \( E\) via le produit scalaire et nous avons
\begin{equation}
    (a\otimes b)(u,v)=(a\cdot u)(b\cdot v).
\end{equation}
Cette dernière équation nous incite à pousser un peu plus loin la définition de \( a\otimes b\) et de simplement voir cela comme la matrice de composantes
\begin{equation}
    (a\otimes b)_{ij}=a_ib_j.
\end{equation}
Cette façon d'écrire a l'avantage de ne pas demander de se souvenir qui est une vecteur ligne, qui est un vecteur colonne et où il faut mettre la transposée. Évidemment \( (a\otimes b)\) est soit \( ab^t\) soit \( a^tb\) suivant que \( a\) et \( b\) soient ligne ou colonne.

%---------------------------------------------------------------------------------------------------------------------------
\subsection{Application d'opérateurs}
%---------------------------------------------------------------------------------------------------------------------------

\begin{lemma}   \label{LemMyKPzY}
    Soient \( x,y\in E\) et \( A,B\) deux opérateurs linéaires sur \( E\) vus comme matrices. Alors
    \begin{equation}        \label{EqXdxvSu}
        (Ax\otimes By)=A(x\otimes y)B^t.
    \end{equation}
\end{lemma}

\begin{proof}
    Calculons la composante \( ij\) de la matrice \( (Ax\otimes By)\). Nous avons
    \begin{subequations}
        \begin{align}
            (Ax\otimes By)_{ij}&=(Ax)_i(By)_j\\
            &=\sum_{kl}A_{ik}x_kB_{jl}y_l\\
            &=A_{ik}(x\otimes y)_{kl}B_{jl}\\
            &=\big( A(x\otimes y)B^t \big)_{ij}.
        \end{align}
    \end{subequations}
\end{proof}

% TODO: Ajouter un texte sur les équations de plan, et pourquoi ax+by+cz+d=0 est perpendiculaire au vecteur (a,b,c).

%+++++++++++++++++++++++++++++++++++++++++++++++++++++++++++++++++++++++++++++++++++++++++++++++++++++++++++++++++++++++++++
\section{Méthode de Gauss pour résoudre des systèmes d'équations linéaires}
%+++++++++++++++++++++++++++++++++++++++++++++++++++++++++++++++++++++++++++++++++++++++++++++++++++++++++++++++++++++++++++

Pour résoudre un système d'équations linéaires, on procède comme suit:
\begin{enumerate}
\item Écrire le système sous forme matricielle. \[\text{p.ex. } \begin{cases} 2x+3y &= 5 \\ x+2y &= 4 \end{cases} \Leftrightarrow \left(\begin{array}{cc|c} 2 & 3 & 5 \\ 1 & 2 & 4 \end{array}\right) \]
\item Se ramener à une matrice avec un maximum de $0$ dans la partie de gauche en utilisant les transformations admissibles:
\begin{enumerate}
\item Remplacer une ligne par elle-même + un multiple d'une autre;
\[\text{p.ex. } \left(\begin{array}{cc|c} 2 & 3 & 5 \\ 1 & 2 & 4 \end{array}\right)  \stackrel{L_1  - 2. L_2 \mapsto L_1'}{\Longrightarrow} \left(\begin{array}{cc|c} 0 & -1 & -3 \\ 1 & 2 & 4 \end{array}\right) \]
\item Remplacer une ligne par un multiple d'elle-même;
\[\text{p.ex. } \left(\begin{array}{cc|c} 0 & -1 & -3 \\ 1 & 2 & 4 \end{array}\right)  \stackrel{-L_1  \mapsto L_1'}{\Longrightarrow} \left(\begin{array}{cc|c} 0 & 1 & 3 \\ 1 & 2 & 4 \end{array}\right) \]
\item Permuter des lignes.
\[\text{p.ex. } \left(\begin{array}{cc|c} 0 & 1 & 3 \\ 1 & 0 & -2 \end{array}\right)  \stackrel{L_1  \mapsto L_2' \text{ et } L_2  \mapsto L_1'}{\Longrightarrow} \left(\begin{array}{cc|c} 1 & 0 & -2 \\ 0 & 1 & 3  \end{array}\right) \]
\end{enumerate}
\item Retransformer la matrice obtenue en système d'équations.
\[\text{p.ex. }  \left(\begin{array}{cc|c} 1 & 0 & -2 \\ 0 & 1 & 3  \end{array}\right) \Leftrightarrow \begin{cases} x &= -2 \\ y &= 3 \end{cases}  \]
\end{enumerate}

\begin{remark}
\begin{itemize}
\item Si on obtient une ligne de zéros, on peut l'enlever:
\[\text{p.ex. }  \left(\begin{array}{ccc|c} 3 & 4 & -2 & 2 \\ 4 & -1 & 3 & 0 \\ 0 & 0 & 0 & 0 \end{array}\right) \Leftrightarrow  \left(\begin{array}{ccc|c} 3 & 4 & -2 & 2 \\ 4 & -1 & 3 & 0 \end{array}\right) \]
\item Si on obtient une ligne de zéros suivie d'un nombre non-nul, le système d'équations n'a pas de solution:
\[\text{p.ex. }  \left(\begin{array}{ccc|c} 3 & 4 & -2 & 2 \\ 4 & -1 & 3 & 0 \\ 0 & 0 & 0 & 7 \end{array}\right) \Leftrightarrow  \begin{cases} \cdots \\ \cdots \\ 0x + 0y + 0z = 7 \end{cases} \Rightarrow \textbf{Impossible} \]
\item Si on moins d'équations que d'inconnues, alors il y a une infinité de solutions qui dépendent d'un ou plusieurs paramètres:
\[\text{p.ex. }  \left(\begin{array}{ccc|c} 1 & 0 & -2 & 2 \\ 0 & 1 & 3 & 0 \end{array}\right) \Leftrightarrow  \begin{cases} x - 2z = 2 \\ y + 3z = 0 \end{cases} \Leftrightarrow  \begin{cases} x = 2 + 2\lambda \\ y = -3\lambda \\ z = \lambda \end{cases} \]
\end{itemize}
\end{remark}
