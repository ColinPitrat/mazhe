% This is part of Mes notes de mathématique
% Copyright (c) 2011-2017
%   Laurent Claessens
% See the file fdl-1.3.txt for copying conditions.

%+++++++++++++++++++++++++++++++++++++++++++++++++++++++++++++++++++++++++++++++++++++++++++++++++++++++++++++++++++++++++++
\section{Anneau euclidien}
%+++++++++++++++++++++++++++++++++++++++++++++++++++++++++++++++++++++++++++++++++++++++++++++++++++++++++++++++++++++++++++

\begin{definition}[\wikipedia{fr}{Anneau_euclidien}{Wikipédia}] \label{DefAXitWRL}
    Soit \( A\) un anneau intègre. Un \defe{stathme euclidien}{stathme euclidien} sur \( A\) est une application \( \alpha\colon A\setminus\{ 0 \}\to \eN\) tel que
    \begin{enumerate}
        \item       \label{ITEMooLVJAooLpjgEz}
            \( \forall a,b\in A\setminus\{ 0 \}\), il existe \( q,r\in A\) tel que
            \begin{equation}
                a=qb+r
            \end{equation}
            et \( \alpha(r)<\alpha(b)\).
        \item
            Pour tout \( a,b\in A\setminus\{ 0 \}\), \( \alpha(b)\leq \alpha(ab)\).
    \end{enumerate}
    Un anneau est \defe{euclidien}{euclidien!anneau} s'il accepte un stathme euclidien.
\end{definition}
Le stathme est la fonction qui donne le «degré» à utiliser dans la division euclidienne. La contrainte est que le degré du reste soit plus petit que le degré du dividende.

\begin{example} \label{ExwqlCwvV}
    Le stathme de \( \eN\) pour la division euclidienne usuelle est \( \alpha(n)=n\). Si \( a,b\in \eN\) nous écrivons
    \begin{equation}
        a=qb+r
    \end{equation}
    où \( q\) est l'entier le plus proche \emph{inférieur} à \( a/b\) (on veut que le reste soit positif) et \( r=a-qb\). Nous avons donc
    \begin{equation}
        r-b=a-b(q+1)<a-b\frac{ a }{ b }=0,
    \end{equation}
    ce qui montre que \( r<b\).
\end{example}

Cet exemple ne fonctionne pas avec \( \eZ\) au lieu de \( \eN\) parce que le stathme doit avoir des valeurs dans \( \eN\). Cela ne veut cependant pas dire qu'il n'existe pas de stathme sur \( \eZ\); cela veut seulement dire que \( \alpha(x)=x\) ne fonctionne pas. 

\begin{proposition}[\cite{ooELVSooZIZCRn}]\label{Propkllxnv}
    Un anneau euclidien est principal.
\end{proposition}

\begin{proof}
    Soit \( A\) un anneau principal et \( \alpha\) un stathme sur \( A\). Nous considérons un idéal \( I\) non nul de \( A\). Nous devons montrer que \( I\) est généré par un élément. En l'occurrence nous allons montrer qu'un élément \( a\in I\setminus\{ 0 \}\) qui minimise \( \alpha(a)\) va générer\footnote{Un tel élément existe\dots}. Soit \( x\in I\). Par construction, il existe \( q,r\in A\) tels que \( x=aq+r\) avec \( r=0\) ou \( \alpha(r)<\alpha(a)\). Étant donné que \( x,a\in I\), \( r\in I\). Si \( r\neq 0\), alors \( r\) contredirait la minimalité de \( \alpha(a)\). Donc \( r=0\) et \( x=aq\), ce qui signifie que \( I\) est principal.
\end{proof}

\begin{proposition}     \label{PROPooPJGLooQSrJTU}
    L'anneau \( \eZ\) est principal et euclidien.
\end{proposition}

\begin{proof}
    Nous allons seulement montrer que \( \alpha(x)=| x |\) est un stathme euclidien. Ainsi \( \eZ\) sera euclidien et donc principal par la proposition \ref{Propkllxnv}.

    D'abord \( \eZ\) est intègre, c'est l'exemple \ref{EXooLDXRooSxUAXs}.

    La condition \( \alpha(b)\leq \alpha(ab)\) est immédiate : \( | a |\leq | ab |\) pour tout \( a,b\in \eZ\).

    Soient maintenant \( a,b\in \eZ\). Nous définissons \( q_0,r_0\in \eN\) tels que
    \begin{equation}
        | a |=q_0| b |+r_0
    \end{equation}
    avec \( r_0<| b |\). Cela existe parce que \( \alpha(x)=x\) est un stathme sur \( \eN\) par l'exemple \ref{ExwqlCwvV}.

    \begin{subproof}
        \item[Si \( a>0\) et \( b>0\)]

            Alors \( a=q_0b+r_0\) et le couple \( (q_0,r_0)\) vérifie les conditions de la définition \ref{DefAXitWRL}\ref{ITEMooLVJAooLpjgEz}.

        \item[Si \( a>0\) et \( b<0\)]

            Alors \( a=-q_0b+r_0\), et le couple \( (-q_0,r_0)\) vérifie les conditions de la définition \ref{DefAXitWRL}\ref{ITEMooLVJAooLpjgEz}.


        \item[Si \( a<0\) et \( b>0\)]
            Alors \( a=-q_0b-r_0\), et le couple \( (-q_0,-r_0)\) vérifie les conditions de la définition \ref{DefAXitWRL}\ref{ITEMooLVJAooLpjgEz} parce que
            \begin{equation}
                \alpha(-r_0)=r_0<| b |=\alpha(b).
            \end{equation}
        \item[Si \( a<0\) et \( b<0\)]
            Alors \( a=q_0b-r_0\), et le couple \( (q_0,-r_0)\) vérifie les conditions de la définition \ref{DefAXitWRL}\ref{ITEMooLVJAooLpjgEz}.

    \end{subproof}
\end{proof}

Nous venons de voir que \( \eZ\) est principal; le lemme suivant nous dit que $\eZ[X]$ n'est pas principal, lui.
\begin{lemma}[\cite{ooRQHSooEBZpKe}]        \label{LEMooDJSUooJWyxCL}
    Si $A$ est un anneau intègre qui n'est pas un corps, alors \( A[X]\) n'est pas principal.
\end{lemma}

\begin{proof}
    Soit un élément non nul \( a\in A\).
    \begin{subproof}
        \newcommand{\foo}{A[X]}
    \item[Un idéal principal contenant \( a\) et \( X\) est $\foo$]
            
            Soit \( (P)\) un idéal principal contenant \( a\) et \( X\). Vu que \( a\in(P)\), il existe \( Q\) tel que \( a=QP\). Donc \( P\) divise \( a\) dans \( \eZ[X]\). Les degrés font que \( P\) est un polynôme constant, c'est à dire en réalité un élément de \( A\). Soit \( P=k\in A\).

            Vu que \( P\) divise \( X\), nous avons aussi \( X=kQ\) pour un certain \( Q\in \eZ[X]\). Les degrés disent qu'il existe \( k'\in A\) tel que \( Q=k'X\) et donc \( Q=k'X=k'kQ\), ce qui implique que \( kk'=1\). L'idéal engendré par \( k\) contient donc en particulier \( kk'=1\) et donc contient \( A[X]\) en entier :
            \begin{equation}
                1=k'k\in k'(P)=(P).
            \end{equation}

        \item[Si \( (a,X)=\foo\) alors \( a\) est inversible]

            Si \( (a,X)=A[X]\), en particulier, \( 1\in (a,X)\), ce qui signifie qu'il existe des polynômes \( U,V\in A[X]\) tels que
            \begin{equation}
                1=UX+Va.
            \end{equation}
            Nous évaluons cette égalité en \( 0\) : vu que \( (UX)(0)=0\) nous avons \( 1=V(0)a\), ce qui signifie que \( V(0)\) est un inverse de \( A\). Donc \( a\) est inversible.


        \item[Si \( a\) n'est pas inversible alors \( (a,X)\) n'est pas principal]

            Si \( (a,X)\) était principal, alors nous aurions, par ce qui est dit plus haut, \( (a,X)=A[X]\). Mais cette dernière égalité impliquerait que \( a\) est inversible.
    \end{subproof}
    En conclusion, si \( A\) n'est pas un corps, il possède un élément ni nul ni inversible. Dans ce cas, l'idéal \( (a,X)\) n'est pas principal dans \( A[X]\) et nous en déduisons que \( A[X]\) n'est pas un anneau principal.
\end{proof}

Nous verrons dans le lemme \ref{LEMooIDSKooQfkeKp} que si $\eK$ est un corps, alors \( \eK[X]\) est principal.

\begin{example} \label{ExeDufyZI}
    Prouvons que \( \eZ[i\sqrt{2}]\) est une anneau euclidien. Pour cela nous démontrons que
    \begin{equation}    \label{EqOZUIooZGmHWl}
        \begin{aligned}
            N\colon \eZ[i\sqrt{2}]&\to \eN \\
            a+bi\sqrt{2}&\mapsto a^2+2b^2 
        \end{aligned}
    \end{equation}
    est un stathme euclidien.    

    Soient \( z=a+bi\sqrt{2}\), \( t=a'+b'i\sqrt{2}\). Nous cherchons \( q\) et \( r\) tels que la division euclidienne s'écrive \( z=qt+r\). Soient \( \alpha,\beta\in \eQ\) tels que 
    \begin{equation}
        \frac{ z }{ t }=\alpha+\beta i\sqrt{2}.
    \end{equation}
    Nous désignons par \( \alpha+\epsilon_1\) et \( \beta+\epsilon_2\) les entiers les plus proches de \( \alpha\) et \( b\). Nous avons \( | \epsilon_1 |,| \epsilon_2 |\leq \frac{ 1 }{2}\). Nous posons alors naturellement 
    \begin{equation}
        q=(\alpha+\epsilon_1)+(\beta+\epsilon_2)i\sqrt{2}
    \end{equation}
    et nous calculons \( r=z-qt\) :
    \begin{equation}
        2b'\epsilon_2-a'\epsilon_1+i\sqrt{2}\big( \epsilon_1b'-a'\epsilon_2 \big).
    \end{equation}
    Nous trouvons 
    \begin{equation}
        N(r)=a'^2\epsilon_1^2+4b'^2\epsilon_2^2+2a'^2\epsilon_1^2+2b'^2\epsilon_2^2\leq \frac{ 3 }{ 4 }a'^2+\frac{ 3 }{2}b'^2.
    \end{equation}
    D'autre part \( N(t)=a'^2+2b'^2\), et nous avons donc bien \( N(r)<N(t)\).

    En ce qui concerne la seconde propriété du stathme, un petit calcul montre que
    \begin{equation}
        N(zt)=(a^2+2b^2)(a'^2+2b'^2),
    \end{equation}
    et tant que \( t\neq 0\) nous avons bien \( N(zt)>N(z)\).
\end{example}

Notons en particulier que \( \eZ[i\sqrt{2}]\) est factoriel et principal.

\begin{example} \label{ExluqIkE}
    Décomposition en facteurs irréductibles dans \( \eZ[i\sqrt{2}]\). Les éléments inversibles de \( \eZ[i\sqrt{2}]\) sont \( \pm 1\), donc deux éléments \( a\) et \( b\) sont associés (définition \ref{DefrXUixs}) si et seulement si \( a=\pm b\).

    De plus si \( p\) est irréductible, alors \( -p\) est irréductible. Les éléments irréductibles de \( \eZ[i\sqrt{2}]\) arrivent donc par pairs d'éléments associés. Soit \( \{ p_i \}\) une sélection de un élément irréductible parmi chaque paire. Tout élément \( x\) de \( \eZ[i\sqrt{2}]\) peut alors être écrit \( x=\pm p_1^{\alpha_1}\ldots p_n^{\alpha_n}\). Ce fait va être pratique pour comparer des décomposition en facteurs irréductibles d'éléments.
\end{example}

Le lemme suivant fait en pratique partie de l'exemple \ref{ExmuQisZU}, mais nous l'isolons pour plus de clarté\footnote{Merci à \href{http://fr.wikipedia.org/wiki/Utilisateur:Marvoir}{Marvoir} pour m'avoir souligné le manque.}.
\begin{lemma}       \label{LemTScCIv}
    Si \( a\) et \( b\) sont deux éléments premiers entre eux de \( \eZ[i\sqrt{2}]\), et s'il existe \( y \in  \eZ[i\sqrt{2}\) tel que \( ab=y^3\), alors \( a\) et \( b\) sont des cubes (dans \( \eZ[i\sqrt{2}]\)).
\end{lemma}

\begin{proof}
    D'après l'exemple \ref{ExluqIkE} nous pouvons écrire
    \begin{subequations}
        \begin{align}
            y&=\pm p_1^{\sigma_1}\ldots p_n^{\sigma_n}\\
            a&=\pm p_1^{\alpha_1}\ldots p_n^{\alpha_n}\\
            b&=\pm p_1^{\beta_1}\ldots p_n^{\beta_n}
        \end{align}
    \end{subequations}
    où les \( p_i\) sont les irréductibles de \( \eZ[i\sqrt{2}]\) «modulo \( \pm 1\)» au sens où la liste des irréductibles est \( \{ p_i \}\cup\{ -p_i \}\) (union disjointe). Étant donné que \( a\) et \( b\) sont premiers entre eux, \( \alpha_i\) et \( \beta_i\) ne peuvent pas être non nuls en même temps alors que leur somme doit faire \( 3\sigma_i\). Nous avons donc pour chaque \( i\) soit \( \alpha_i=3\sigma_i\) soit \( \beta=3\sigma_i\) (et bien entendu si \( \sigma_i=0\) alors \( \alpha_i=\beta_i=0\)).

    Étant donné que \( \pm 1\) sont également deux cubes, \( a\) et \( b\) sont bien des cubes.

    Notons que nous avons utilisé de façon capitale le fait que \( \eZ[i\sqrt{2}]\) était factoriel.
\end{proof}

%---------------------------------------------------------------------------------------------------------------------------
\subsection{Équations diophantiennes}
%---------------------------------------------------------------------------------------------------------------------------
%TODO : il y a une équation diophantienne qui semple pas mal ici : http://fr.wikipedia.org/wiki/Entier_quadratique#x2_.2B_5.y2_.3D_p

\begin{example} \label{ExZPVFooPpdKJc}
    L'équation diophantienne
    \begin{equation}
        x^2=3y^2+8
    \end{equation}
    n'a pas de solutions. En effet si nous prenons l'équation modulo \( 3\) nous obtenons
    \begin{equation}
        [x^2]_3=[3y^2+8]_3=[8]_3=[2]_3.
    \end{equation}
    Or dans \( \eZ/3\eZ\), aucun carré n'est égal à deux : \( 0^2=0\neq 2\), \( 1^2=1\neq 2\) et \( 2^2=4=1\neq 2\).
\end{example}

\begin{example}     \label{ExmuQisZU}
    Résolvons l'équation diophantienne\index{équation!diophantienne} 
    \begin{equation}
        x^2+2=y^3.
    \end{equation}
    Une première remarque est que \( x\) doit être impair. En effet si \( x=2k\), nous devons avoir \( y^3\) pair. Mais si un cube pair est divisible par \( 8\), donc \( y^3=8l\). L'équation devient \( 4k^2+2=8l^3\), c'est à dire \( 2k^2+1=4l^3\). Le membre de gauche est impair tandis que celui de droite est pair. Impossible.

    Nous pouvons écrire l'équation sous la forme \( x^2+2=(x+i\sqrt{2})(x-i\sqrt{2})\). Et nous considérons \( \eZ[i\sqrt{2}]\) muni de son stathme \( N\) donné par \eqref{EqOZUIooZGmHWl}.

    L'élément \( i\sqrt{2}\) est irréductible parce que \( N(i\sqrt{2})=2\), et si nous avions \( i\sqrt{2}=pq\), alors nous aurions \( N(p)N(q)=2\), ce qui n'est possible que si \( N(p)\) ou \( N(q)\) vaut \( 1\).

    Nous prouvons maintenant que les éléments \( x+i\sqrt{2}\) et \( x-i\sqrt{2}\) sont premiers entre eux. Supposons que \( d\) soit un diviseur commun; alors il divise aussi la somme et la différence. Donc \( d\) divise à la fois \( 2x\) et \( 2i\sqrt{2}\).

    Étant donné que \( i\sqrt{2}\) est irréductible et que \( 2i\sqrt{2}=(-i\sqrt{2})^3\), les diviseurs de \( 2i\sqrt{2}\) sont les puissances de \( (-i\sqrt{2})\). Du coup nous devrions avoir \( d=(i\sqrt{2})^{\beta}\) et donc
    \begin{equation}
        x=(i\sqrt{2})^{\beta}q
    \end{equation}
    pour un certain \( q\in\eZ[i\sqrt{2}]\). Dans ce cas nous avons \( N(x)=2^{\beta}N(q)\), mais nous avons déjà précisé que \( x\) ne pouvait pas être pair, donc \( \beta=0\) et nous avons \( d=1\).

    Vu que les nombres \( x\pm i\sqrt{2}\) sont premiers entre eux et que leur produit doit être un cube, ils doivent être séparément des cubes (lemme \ref{LemTScCIv}). Nous devons donc résoudre séparément \( x\pm i\sqrt{2}=y^3\).

    Cherchons les \( x\) et \( y\) entiers tels que \( x+i\sqrt{2}=y^3\). Si nous posons \( z=a+bi\sqrt{2}\), il suffit de calculer \( z^3\) :
    \begin{verbatim}
----------------------------------------------------------------------
| Sage Version 4.8, Release Date: 2012-01-20                         |
| Type notebook() for the GUI, and license() for information.        |
----------------------------------------------------------------------
sage: var('a,b')
(a, b)
sage: z=a+I*sqrt(2)*b
sage: (z**3).expand()
3*I*sqrt(2)*a^2*b - 2*I*sqrt(2)*b^3 + a^3 - 6*a*b^2
    \end{verbatim}
    En identifiant cela à \( x+i\sqrt{2}\) nous trouvons le système
    \begin{subequations}
        \begin{numcases}{}
            x=a^3-6ab^2\\
            1=3a^2b-2b^3
        \end{numcases}
    \end{subequations}
    où, nous le rappelons, \( x\), \( a\) et \( b\) sont des entiers. La seconde équation montre que \( b\) doit être inversible : \( b(3a^2-2b^2)=1\). Il y a donc les possibilités \( b=\pm 1\). Pour \( b=1\) l'équation devient \( 3a^2-2=1\), c'est à dire \( a=\pm 1\). Pour \( b=-1\) l'équation devient \( 3a^2-2=-1\), impossible. En conclusion les possibilités sont
    \begin{subequations}
        \begin{align}
            (x,z)=(-5,1+i\sqrt{2})\\
            (x,z)=(5,-1+i\sqrt{2})\\
        \end{align}
    \end{subequations}
    Le travail avec \( x-i\sqrt{2}\) donne les mêmes résultats.

    Les deux solutions de l'équation \( x^2+2=y^3\) sont alors \( (5,3)\) et \( (-5,3)\).
\end{example}

%--------------------------------------------------------------------------------------------------------------------------- 
\subsection{Triplets pythagoriciens et équation de Fermat pour \texorpdfstring{$ n=4$}{n=4}}
%---------------------------------------------------------------------------------------------------------------------------

\begin{definition}
    Les solutions entières (positives) de l'équation \( x^2+y^2=z^2\) sont appelés \defe{triplets pythagoriciens}{triplet!pythagoricien}. 
\end{definition}

Ils donnent toutes les possibilités de triangles rectangles dont les côtés ont des longueurs entières.

\begin{definition}
  On dit qu'un triplet pythagoricien est \defe{primitif}{primitif!triplet pythagoricien} si les trois nombres sont premiers dans leur ensemble\footnote{Définition
    \ref{DefZHRXooNeWIcB}.}.
\end{definition}

Remarquons que cela est équivalent à montrer que les trois nombres sont premiers deux à deux: en effet, si deux parmi \( x\), \( y\) et \( z\) sont divisibles par un nombre, alors tous les trois sont divisibles par ce nombre\footnote{Parce que \( k\) et \( k^2\) ont les mêmes facteurs premiers.}, donc les nombres \( x\), \( y\) et \( z\) sont premiers deux à deux.

\begin{lemma}    \label{LemTripletsPythagoriciensPrimitifs}
  Dans un triplet pythagoricien primitif \( (x, y, z) \), on a toujours $z$ impair et:
  \begin{itemize}
  \item
    soit $x$ impair et $y$ pair;
  \item
    soit $x$ pair et $y$ impair.
  \end{itemize}
\end{lemma}

\begin{proof}
  Remarquons que le fait d'imposer que le triplet soit primitif, interdit aux nombres $x$ et $y$ d'être pairs en même temps. En effet, si c'était le cas, alors \( x^2 \) et \( y^2 \) seraient aussi pairs, donc leur somme \( z^2 \) aussi, d'ou $z$ serait pair et les trois nombres ne seraient pas premiers entre eux.

  Nous montrons à présent que les nombres \( x\) et \( y\) ne sont pas tous les deux impairs. Par l'absurde, si \( x=2a+1\), nous avons \( x^2=4a^2+4a+1\in [1]_4\); de la même manière,  \( y^2 \in [1]_4\). On en déduit alors que \( z^2=x^2+y^2\in [2]_4\). Le nombre \(  z^2\) est donc pair, donc \( z\) est pair : disons \( z=2c\). Alors, \( z^2=4c^2\in [0]_4\). Comme les classes modulo 4 sont disjointes, nous aboutissons à une contradiction.
\end{proof}

\begin{proposition}[Triplets pythagoriciens\cite{fJhCTE,HARRooBvzbXo}]  \label{PropXHMLooRnJKRi}
    Un triplet \( (x,y,z)\in(\eN^*)^3\) est solution de \( x^2+y^2=z^2\) si et seulement s'il existe \( d\in \eN\) et \( u,v\in \eN^*\) premiers entre eux tels que
    \begin{subequations}        \label{subeqLVHFooVgWsFx}
        \begin{numcases}{}
            x=d(u^2-v^2)\\
            y=2duv\\
            z=d(u^2+v^2)
        \end{numcases}
    \end{subequations}
    ou
    \begin{subequations}
        \begin{numcases}{}
            x=2duv\\
            y=d(u^2-v^2)\\
            z=d(u^2+v^2)
        \end{numcases}
    \end{subequations}
    La différence entre les deux est seulement d'inverser les rôles de \( x\) et \( y\).
\end{proposition}

\begin{proof}
    Montrons d'abord que les formules proposées sont bien des solutions; nous vérifions \eqref{subeqLVHFooVgWsFx} :
    \begin{equation}
        x^2+y^2=d^2(u^2-v^2)+4d^2u^2v^2=d^2(u^2+v^2)^2,
    \end{equation}
    qui correspond bien au \( z^2\) proposé.

    Nous allons maintenant prouver la réciproque : toute solution est d'une des deux formes proposées. Déterminer les triplets primitifs suffira parce que si \( (x,y,z)\) n'est pas une solution primitive, alors en posant \( k=\pgcd(x,y,z)\), le triplet \( \big( \frac{ x }{ k },\frac{ y }{ k },\frac{ z }{ k } \big)\) est primitif. Connaissant les triplets primitifs, nous obtenons tous les autres par simple multiplication.

    Soit donc \( (x,y,z)\) un triplet pythagoricien primitif. Sans
    perte de généralité\footnote{En échangeant les rôles de $x$ et $y$
      ici, nous obtenons à la fin la seconde forme des solutions.},
    grâce au lemme \ref{LemTripletsPythagoriciensPrimitifs}, on peut
    supposer \( x\) pair, \( y\) impair, \( z\) impair. Comme \(
    x^2=(z+y)(z-y)\), nous avons
    \begin{equation}
        \left( \frac{ x }{2} \right)^2=\left( \frac{ z+y }{2} \right)^2\left( \frac{ z-y }{ 2 } \right)^2.
    \end{equation}
    Vu que \( z\) et \( y\) sont premiers entre eux, les nombres \( z-y\) et \( z+y\) sont également premiers entre eux\footnote{Si \( z-y=kn\) et \( z+y=km\), faisant la somme et la différence on voit que \( y\) et \( z\) sont divisibles par \( k\).}. Donc les facteurs premiers (qui arrivent au moins au carré) de \( (x/2)^2\) sont chacun soit dans \( (z+y)/2\) soit dans \( (z-y)/2\). Nous en déduisons que ces derniers sont des carrés d'entiers. Nous posons
    \begin{equation}
        \begin{aligned}[]
            \frac{ z-y }{2}=u^2&&\frac{ z+y }{2}=v^2.
        \end{aligned}
    \end{equation}
    Bien entendu \( u\) et \( v\) sont non nuls parce que nous avons exclu la possibilité de triplets dont un élément serait nul. Avec tout cela nous avons \( (x/2)^2=u^2v^2\) et donc \( x=2uv\) puis par somme et différence :
    \begin{subequations}
        \begin{numcases}{}
            x=2uv\\
            y=v^2-u^2\\
            z=u^2+v^2,
        \end{numcases}
    \end{subequations}
    ce qu'il fallait.
\end{proof}

\begin{remark}
    Les solutions dans lesquelles \( x\), \( y\) ou \( z\) sont nuls sont faciles à classer. La solution \( (1,0,1)\) n'est pas dans les formes proposées. En effet elle ne peut pas être de la première forme : avoir \( y=0\) demanderait qu'un nombre parmi \( d\), \( u\) et \( v\) soit nul, ce qui est interdit. La solution \( (1,0,1) \) ne peut pas non plus être de la seconde forme parce que \( x\) y est pair.
\end{remark}

\begin{proposition}[\cite{fJhCTE}]      \label{propFKKKooFYQcxE}
    Les équations \( x^4+y^4=z^2\) et \( x^2+y^4=z^4\) n'ont pas de solutions dans \( (\eN^*)^3\).
\end{proposition}
\index{équation!diophantienne}

\begin{proof}
  Si la première équation n'a pas de solutions, alors la seconde n'en
  n'a pas non plus parce que \( z^4\) est un carré. Nous nous
  concentrons donc sur l'équation \( x^4+y^4=z^2\) et nous supposons
  qu'il existe au moins une solution dans \( (\eN^*)^3\). Nous en choisissons une \( (x,y,z)\) avec \( z\) minimum (les \( z\) dans différentes solutions étant dans \( \eN\), il en existe forcément un minimum\footnote{Voir quelque chose comme le lemme \ref{PropQEPoozLqOQ}.}). Du coup, les trois nombres \( x\), \( y\) et \( z\) sont premiers dans leur ensembles parce que une
  division par leur \( \pgcd\) donnerait une nouvelle solution qui
  contredirait la minimalité de \( z\).

    Nous posons \( x^4=\bar x^2\) et \( y^4=\bar y^2\). Ils vérifient
    l'équation \( \bar x^2+\bar y^2=z^2\) et par la proposition
    \ref{PropXHMLooRnJKRi}, il existe \( u,v\in \eN^*\) premiers entre
    eux tels que, sans perte de généralité\footnote{En inversant les
      rôles de $x$ et $y$ au besoin.}, on ait
    \begin{subequations}
        \begin{numcases}{}
            \bar x=2uv\\
            \bar y=u^2-v^2\label{eqnFKKKooFYQcxE1}\\
            z=u^2+v^2.\label{eqnFKKKooFYQcxE2}
        \end{numcases}
    \end{subequations}
    Si \( u\) est pair, alors \( v\) est impair (et inversement) parce
    que \( \pgcd(u,v)=1\) Si \( u\) est pair, alors \( u=2a\) et \(
    v=2b+1\), ce qui donne \( \bar y=4a^2-4b^2-4b-1\in[-1]_4\). Or
    nous avons déjà vu qu'un carré est dans \( [0]_4\) ou dans \(
    [1]_4\). Il faut donc que \( u\) soit impair. Le lemme
    \ref{LemTripletsPythagoriciensPrimitifs} implique alors que \( v\)
    soit pair.

    De l'équation \ref{eqnFKKKooFYQcxE1}, nous en déduisons que \(
    v^2+\bar y=u^2\); de plus \( u^2\), \( v^2\) et \( \bar y\) sont
    premiers dans leur ensemble: en effet, $u$ et $v$ sont premiers
    entre eux, et si l'un parmi \( u^2\) et \( v^2\) a un facteur
    commun avec \( \bar y\), alors l'autre doit l'avoir aussi (dans
    une égalité \( a+b=c\), si deux des nombres ont un diviseur
    commun, le troisième l'a aussi). Comme \( \bar y=y^2\), le triplet
    \( (v,y,u)\) est un triplet pythagoricien primitif. Nous
    réappliquons la proposition \ref{PropXHMLooRnJKRi}, en se
    souvenant que $v$ est pair: il existe donc deux nombres \( r\) et
    \( s\) premiers entre eux tels que
    \begin{subequations} \label{eqnFKKKooFYQcxE3}
        \begin{numcases}{}
            v=2rs\\
            y=r^2-s^2\\
            u=r^2+s^2.
        \end{numcases}
    \end{subequations}
    Avec cela, \( \bar x=2uv=4rs(r^2+s^2)\). Remarquons que les trois
    nombres \( r\), \( s\) et \( r^2+s^2\) sont premiers entre
    eux dans leur ensemble; or, comme \( \bar x\) est un
    carré ces nombres doivent séparément être des carrés :
    \begin{subequations}
        \begin{numcases}{}
            r=\alpha^2\\
            s=\beta^2\\
            r^2+s^2=\gamma^2.
        \end{numcases}
    \end{subequations}
    En mettant les deux premiers dans la troisième, nous avons \( \alpha^4+\beta^4=\gamma^2\). Donc \( (\alpha^2,\beta^2,\gamma)\) est une solution. Nous allons prouver que \( \gamma<z\), ce qui terminera la preuve, puisque $z$ était supposé minimal. Nous avons :
    \begin{align*}
        z&=u^2+v^2&&\text{par \ref{eqnFKKKooFYQcxE2}}\\
          &=r^2+s^2+4r^2s^2&&\text{par \ref{eqnFKKKooFYQcxE3}}\\
          &=\gamma^2+4r^2s^2\\
          &> \gamma^2,
    \end{align*}
    et a fortiori \( \gamma<z\).
\end{proof}

%--------------------------------------------------------------------------------------------------------------------------- 
\subsection{Lignes et colonnes de matrices}
%---------------------------------------------------------------------------------------------------------------------------

Nous nommons \( E_{ij}\) la matrice remplie de zéros sauf à la case \( ij\) qui vaut \( 1\). Autrement dit
\begin{equation}
    (E_{ij})_{kl}=\delta_{ik}\delta_{jl}.
\end{equation}
\begin{definition}
    Une \defe{matrice de transvection}{transvection (matrice)}\index{matrice!de transvection} est une matrice de la forme
    \begin{equation}
        T_{ij}(\lambda)=\id+\lambda E_{ij}
    \end{equation}
    avec \( i\neq j\).

    Une \defe{matrice de dilatation}{matrice!de dilatation}\index{dilatation (matrice)} est une matrice de la forme
    \begin{equation}
        D_i(\lambda)=\id+(\lambda-1)E_{ii}.
    \end{equation}
    Ici le \( (\lambda-1)\) sert à avoir \( \lambda\) et non \( 1+\lambda\). C'est donc une matrice qui dilate d'un facteur \( \lambda\) la direction \( i\) tout en laissant le reste inchangé.

    Si \( \sigma\) est une permutation (un élément du groupe symétrique \( S_n\)) alors la \defe{matrice de permutation}{matrice!de permutation}\index{permutation!matrice} associée est la matrice d'entrées
    \begin{equation}
        (P_{\sigma})_{ij}=\delta_{i\sigma(j)}.
    \end{equation}
\end{definition}

\begin{lemma}   \label{LemyrAXQs}
    La matrice \( T_{ij}(\lambda)A=(\mtu+\lambda E_{ij})A\) est la matrice \( A\) à qui on a effectué la substitution
    \begin{equation}
        L_i\to L_i+\lambda L_j.
    \end{equation}
    La matrice \( AT_{ij}(\lambda)\) est la substitution 
    \begin{equation}
        C_j\to C_j+\lambda C_i.
    \end{equation}

    La matrice \( AP_{\sigma}\) est la matrice \( A\) dans laquelle nous avons permuté les colonnes avec \( \sigma\).

    La matrice \( P_{\sigma}A\) est la matrice \( A\) dans laquelle nous avons permuté les lignes avec \( \sigma^{-1}\).
\end{lemma}

\begin{proof}
    Calculons la composante \( kl\) de la matrice \( E_{ij}A\) :
    \begin{subequations}
        \begin{align}
            (E_{ij}A)_{kl}&=\sum_m(E_{ij})_{km}A_{ml}\\
            &=\sum_m\delta_{ik}\delta_{jm}A_{ml}\\
            &=\delta_{ik}A_{jl}.
        \end{align}
    \end{subequations}
    C'est donc la matrice pleine de zéros, sauf la ligne \( i\) qui est donnée par la ligne \( j\) de \( A\). Donc effectivement la matrice
    \begin{equation}
        A+\lambda E_{ij}A
    \end{equation}
    est la matrice \( A\) à laquelle on a substitué la ligne \( i\) par la ligne \( i\) plus \( \lambda\) fois la ligne \( j\).

    En ce qui concerne l'autre assertion sur les transvections, le calcul est le même et nous obtenons
    \begin{equation}
        (AE_{ij})=A_{ki}\delta_{jl}.
    \end{equation}

    Pour les matrices de permutation, nous avons 
    \begin{equation}
        (AP_{\sigma})_{kl}=A_{k\sigma(l)}
    \end{equation}
    et
    \begin{equation}
        (P_{\sigma}A)_{kl}=\sum_m\delta_{k\sigma(m)}A_{ml}=\sum_m\delta_{\sigma^{-1}(k)m}A_{ml}=A_{\sigma^{-1}(k)l}.
    \end{equation}
\end{proof}


%--------------------------------------------------------------------------------------------------------------------------- 
\subsection{Algorithme des facteurs invariants}
%---------------------------------------------------------------------------------------------------------------------------

\begin{probleme}
Définir les anneaux de matrices, et en particulier \(GL_n(A) \).
\end{probleme}

\begin{proposition}[Algorithme des facteurs invariants\cite{KXjFWKA}]   \label{PropPDfCqee}
    Soit \( (A,\delta)\) un anneau euclidien muni de son stathme  et \( U\in \eM(n,m,A)\). Alors il existe \( d_1,\ldots, d_s\in A^*\) et des matrices \( P\in\GL(m,A)\), \( Q\in \GL(n,A)\) tels que nous ayons
    \begin{equation}
        U=P \begin{pmatrix}
            \begin{matrix}
                d_1    &       &       \\
                    &   \ddots    &       \\
                    &       &   d_s
            \end{matrix}&   0    \\ 
            0    &   0    
        \end{pmatrix}Q
    \end{equation}
    avec \( d_i\divides d_{i+1}\) pour tout \( i\).
\end{proposition}
\index{anneau!euclidien!facteurs invariants}
\index{algorithme!facteurs invariants}

\begin{proof}
    Nous allons donner la preuve plus ou moins sous forme d'algorithme.

    D'abord si \( U=0\) c'est bon, on a la réponse. Sinon, nous prenons l'élément \( (i_0,j_0)\) dont le stathme est le plus petit et nous l'amenons en \( (1,1)\) par les permutations
    \begin{equation}
        \begin{aligned}[]
            C_1&\leftrightarrow C_{j_0}\\
            L_1&\leftrightarrow L_{i_0}
        \end{aligned}
    \end{equation}
    Ensuite nous traitons la première colonne jusqu'à amener des zéros partout en dessous de \( u_{11}\) de la façon suivante : pour chaque ligne successivement nous calculons la division euclidienne
    \begin{equation}
        u_{i1}=qu_{11}+r_i,
    \end{equation}
    et nous faisons
    \begin{equation}
        L_i\to L_i-qL_1,
    \end{equation}
    c'est à dire que nous enlevons le maximum possible et il reste seulement \( r_i\) en \( u_{i1}\). Vu que le but est de ne laisser que des zéros dans la première colonne, si le reste n'est pas zéro nous ne sommes pas content\footnote{S'il est zéro, nous passons à la ligne suivante}. Dans ce cas nous permutons \( L_1\leftrightarrow L_i\), ce qui aura pour effet de strictement diminuer le stathme de \( u_{11}\) parce qu'on va mettre en \( u_{11}\) le nombre \( r_i\) dont le stathme est strictement plus petit que celui de \( u_{11}\).

    En faisant ce jeu de division euclidienne puis échange, on diminue toujours le stathme de \( u_{11}\), donc ça finit par s'arrêter, c'est à dire qu'à un certain moment la division euclidienne de \( u_{i1}\) par \( u_{11}\) va donner un reste zéro et nous serons content.

    Une fois la première colonne ramenée à la forme
    \begin{equation}
        C_1=\begin{pmatrix}
            u_{11}    \\ 
            0    \\ 
            \vdots    \\ 
            0    
        \end{pmatrix},
    \end{equation}
    nous faisons tout le même jeu avec la première ligne en faisant maintenant des sommes divisions et permutations de colonnes. Notons que ce faisant nous ne changeons plus la première colonne.

    En fin de compte nous trouvons une matrice\footnote{Nous nommons toujours par la même lettre \( U\) la matrice originale et la modifiée, comme il est d'usage en informatique.}
    \begin{equation}
        U=\begin{pmatrix}
            u_{11}   &   0    &   \ldots    &   0    \\
             0   &       &       &       \\
             \vdots   &       &   A    &       \\ 
             0   &       &       &        
         \end{pmatrix}
    \end{equation}
    Si l'élément \( u_{11}\) ne divise pas un des éléments de \( A\), disons \( a_{ij}\), alors nous faisons 
    \begin{equation}
        C_1\to C_1-C_j.
    \end{equation}
    Cela nous détruit un peu la première colonne, mais ne change pas \( u_{11}\). Nous avons maintnant
    \begin{equation}
        U=\begin{pmatrix}
            u_{11}   &   0    &   \ldots    &   0    \\
             0   &       &       &       \\
             *   &       &       &       \\ 
             u_{ij}   &       &   A    &       \\ 
             *   &       &       &       \\ 
             0   &       &       &        
         \end{pmatrix}
    \end{equation}
    Et nous refaisons tout le jeu depuis le début. Cependant lorsque nous allons nous attaquer à la ligne \( i\), \( u_{11}\) ne divisera pas \( u_{ij}\), ce qui donnera lieu à une division euclidienne et un échange \( L_1\leftrightarrow L_i\). L'échange consistant à mettre \( r_i\) à la place de \( u_{11}\) et inversement  diminuera encore strictement le stathme. Encore une fois nous allons travailler jusqu'à avoir la matrice sous la forme
    \begin{equation}    \label{EqADcNVgI}
        U=\begin{pmatrix}
            u_{11}   &   0    &   \ldots    &   0    \\
             0   &       &       &       \\
             \vdots   &       &   A    &       \\ 
             0   &       &       &        
         \end{pmatrix},
    \end{equation}
    sauf que cette fois le stathme de \( u_{11}\) est strictement plus petit que la fois précédente. Si \( u_{11}\) ne divise toujours pas tous les éléments de \( A\), nous recommençons encore et encore. En fin de compte nous finissons par avoir une matrice de la forme \eqref{EqADcNVgI} avec \( u_{11}\) qui divise tous les éléments de \( A\).

    Une fois que cela est fait, il faut continuer en recommençant tout sur la matrice \( A\). Nous avons maintenant
    \begin{equation}
        U=\begin{pmatrix}
            \begin{matrix}
                u_{11}  &       \\ 
                &   u_{22}    
            \end{matrix}&   0    \\ 
            0    &   B    
        \end{pmatrix}.
    \end{equation}
    Sous cette forme nous avons \( u_{11}\divides u_{22}\) et \( u_{11}\) divise tous les éléments de \( B\). En effet \( u_{11}\) divisant tous les éléments de \( A\), il divise toutes les combinaisons de ces éléments. Or tout l'algorithme ne consiste qu'à prendre des combinaisons d'éléments.

    Nous finissons donc bien sûr une matrice comme annoncée. De plus n'ayant effectué que des combinaisons de lignes, nous avons seulement multiplié par des matrices inversibles (lemme \ref{LemyrAXQs}).
\end{proof}

%+++++++++++++++++++++++++++++++++++++++++++++++++++++++++++++++++++++++++++++++++++++++++++++++++++++++++++++++++++++++++++
\section{Polynômes à coefficients dans un anneau commutatif}
%+++++++++++++++++++++++++++++++++++++++++++++++++++++++++++++++++++++++++++++++++++++++++++++++++++++++++++++++++++++++++++
\label{SECooVMABooVdhbPo}

Nous définissons ici \( A[X]\) où \( A\) est un anneau commutatif. Pour la définition de \( \eK(X)\) où \( \eK\) est un corps, voir \ref{DEFooQPZIooQYiNVh}.

%--------------------------------------------------------------------------------------------------------------------------- 
\subsection{Défintions}
%---------------------------------------------------------------------------------------------------------------------------

Soit \( A\) un anneau commutatif. Nous considérons \( \polyP\) l'ensemble des suites presque nulles d'éléments de \( A\), ce sont les suites \( (a_n)_{n\in\eN}\) qui ne possèdent qu'un nombre fini d'éléments non nuls.

Cela est un \( A\)-module libre de base\footnote{Définition \ref{DefBasePouyKj}.}
\begin{equation}
    (e_n)_k=\delta_{nk}.
\end{equation}
Si \( a,b\in\polyP\), nous définissons le produit \( ab\) par
\begin{equation}
    (ab)_n=\sum_{k=0}^na_kb_{n-k},
\end{equation}
et la somme par
\begin{equation}
    (a+b)_n=a_n+b_n.
\end{equation}
Cela est bien un élément de \( \polyP\) parce qu'il existe \( N\in\eN\) tel que \( a_n=b_n=0\) pour tout \( n\geq N\). Avec la somme et le produit par un scalaire (élément de \( A\)), le module \( \polyP\) devient une \( A\)-algèbre commutative unitaire. L'unité est 
\begin{equation}
    e_0=(1,0,\ldots).
\end{equation}

\begin{definition}  \label{DefRGOooGIVzkx}
    En tant que \( A\)-algèbre, l'ensemble \( \polyP\) est l'\defe{algèbre des polynômes en une indéterminée}{algèbre!polynômes} à coefficients dans \( A\). Elle est notée \( A[X]\) pour des raisons que nous expliquons dans \ref{SUBSECooLEKVooFBPSJz}.
\end{definition}

\begin{definition}      \label{DEFooOSWQooHYYwVE}
    Si \( P\in A[X]\) est la suite \( (a_k)_{k\in \eN}\) et si \( \alpha\in A\), alors nous définissons
    \begin{equation}
        P(\alpha)=\sum_{k\in \eN}a_k\alpha^k.
    \end{equation}
    La somme est toujours finie.
\end{definition}

\begin{lemma}       \label{LEMooSFGGooGeVerf}
    Si \( A\) est un anneau et si \( \alpha\in A\), alors l'application
    \begin{equation}
        \begin{aligned}
            g\colon A[X]&\to A \\
            P&\mapsto P(\alpha) 
        \end{aligned}
    \end{equation}
    est un morphisme d'anneaux.
\end{lemma}

\begin{proof}
    Nous notons \( P_k\) les éléments de la suite définissant \( P\) et \( Q_k\) ceux de \( Q\). Alors nous avons
    \begin{equation}
        (P+Q)(\alpha)=\sum_k(P_k+Q_k)\alpha^k=\sum_kP_k\alpha^k+\sum_kQ_k\alpha^k=P(\alpha)+Q(\alpha)
    \end{equation}
    et
    \begin{subequations}
        \begin{align}
            P(\alpha)Q(\alpha)&=\big( \sum_nP_n\alpha^n \big)\big( \sum_kQ_k\alpha^k \big)=\sum_kQ_k\big( \sum_nP_n\alpha^n \big)\alpha^k=\sum_k\sum_nQ_kP_n^{n+k}\\
            &=\sum_m\big( \sum_{l=0}^mP_lQ_{m-l} \big)\alpha^m=\sum_m(PQ)_m\alpha^m=(PQ)(\alpha).
        \end{align}
    \end{subequations}
\end{proof}


\begin{definition}  \label{DefDegrePoly}
    Soit \( P \in \polyP\), \( P \neq 0 \). On appelle \defe{degré}{degré!d'un polynôme} de $P$ le plus grand nombre naturel $n$ pour lequel le coefficient correspondant est non-nul. Ce naturel est noté \( \deg(P) \).
\end{definition}

%--------------------------------------------------------------------------------------------------------------------------- 
\subsection{Notations}
%---------------------------------------------------------------------------------------------------------------------------
\label{SUBSECooLEKVooFBPSJz}

Le polynôme donné par la suite \( (a_n)_{n\in \eN}\) est souvent notée
\begin{equation}
    \sum_ka_kX^k.
\end{equation}
Par exemple avec \( a=(4,2,8)\) nous avons \( a=8X^2+2X+4\). Nous utiliserons souvent cette notation, qui est très pratique parce qu'elle s'adapte bien aux règles de multiplication et d'addition, en particulier la distributivité.

Il y a (au moins) deux façons de comprendre ce que signifie réellement «\( X\)» dans cette notation. 

%///////////////////////////////////////////////////////////////////////////////////////////////////////////////////////////
\subsubsection{Première façon}
%///////////////////////////////////////////////////////////////////////////////////////////////////////////////////////////

La première est de dire qu'il n'a pas de significations, et que \( X^2\) est un simple abus de notations pour écrire \( (0,0,1,0,\cdots)\). Avec cette façon de voir, nous notons l'anneau des polynômes sur \( A\) par «\( A[X]\)» où le \( X\) n'a pas d'autres raisons d'être que d'avertir le lecteur que nous réservons la lettre «\( X\)» pour utiliser la notation pratique des polynômes.

%///////////////////////////////////////////////////////////////////////////////////////////////////////////////////////////
\subsubsection{Seconde façon}
%///////////////////////////////////////////////////////////////////////////////////////////////////////////////////////////
\label{SUBSUBSECooPNBYooWXEHrg}

La seconde façon de voir le «\( X\)» est de nous rappeler que \( A[X]\) a une base en tant de que module : les \( e_k\) dont nous avons parlé plus haut. Nous posons \( X=e_1\), et nous prenons la convention \( X^0=1\). Alors nous avons \( e_k=X^k\) et nous notons \( A[X]\)\nomenclature[A]{\( A[X]\)}{tous les polynômes de degré fini à coefficients dans \( A\)} l'anneau \( \polyP\) exprimé avec \( X\). i

Dans les deux cas, il n'est pas vraiment légitime d'écrire des égalités comme « \( P(X)=X^2+2X-3\) », et encore moins de dire «Le polynôme \( P\), \emph{évalué} en \( X\) vaut \( X^2+2X-3\)»  : il est plus correct d'écrire « \( P=X^2+2X-3\) ». 

Le lemme suivant montre que ces notations tombent vraiment à point. La véritable difficulté de l'énoncé est de comprendre qu'il n'est pas trivial.
\begin{lemma}       \label{LEMooGKWQooVOyeDX}
    Nous avons
    \begin{equation}
        P(X)=P
    \end{equation}
    pour tout \( P\in A[X]\).
\end{lemma}

\begin{proof}
    Un polynôme \( P\in A[X]\) peut s'évaluer sur n'importe quel élément d'un anneau qui étend \( A\) : si \( P=(a_k)_{k\in \eN}\) alors par définition \( P(\alpha)=\sum_ka_k\alpha^k\). Or \( A[X]\) est lui-même un anneau qui étend \( A\); donc si \( Q\) est un polynôme, ça a un sens d'écrire \( P(Q)\) et le résultat sera un élément de \( A[X]\). Avec en particulier \( Q=X\), c'est à dire \( Q=(0,1,0,\ldots)\), l'élément \( P(X)\) de \( A[X]\) vaut
    \begin{equation}        \label{EQooABULooFCEasf}
        \sum_ka_kX^k,
    \end{equation}
    qui est exactement \( P\) lui-même.
\end{proof}

Mais il faut bien comprendre que si \( P\) est le polynôme \( (-3,2,1,0,\ldots)\), noté \( X^2+2X-3\), écrire \( P(X)=X^2+2X-3\) est une pirouette de notations que rien ne justifie par rapport à simplement écrire \( P=X^2+2X-3\).

\begin{normaltext}
    Dans la suite, nous considérons cette seconde façon de comprendre la notation \( X\).
\end{normaltext}

\begin{lemma}       \label{LEMooXFMAooMBgIrN}
    Nous considérons un polynôme \( P\in A[X]\), et le quotient \( A[X]/(P)\). Pour tout polynôme \( Q\in A[X]\) nous avons les égalités
    \begin{equation}
        Q(\bar X)=\overline{ Q(X) }=\bar Q.
    \end{equation}
\end{lemma}

\begin{proof}
    Si \( Q=\sum_ka_kX^k\), alors par la linéarité de la prise de classes,
    \begin{equation}        \label{EQooXQRMooIPGFVM}
        \bar Q=\sum_ka_k\overline{ X^k }.
    \end{equation}
    Nous insistons sur le fait que cette égalité n'est rien d'autre que l'itération de la définition de la somme dans l'espace quotient : \( \bar x+\bar y=\overline{ x+y }\) ainsi que du produit \( k\bar x=\overline{ kx }\) (définition \ref{PROPooGXMRooTcUGbi}). Toujours par définition du produit appliqué à l'élément \( \bar X\) nous avons \( (\bar X)^2=\overline{ X^2 }\); par récurrence \( \overline{ X^k }=\bar X^k\), et
    \begin{equation}
        \bar Q=\sum_ka_k\bar X^k=Q(\bar X).
    \end{equation}

    Le fait que \( \bar Q=\overline{ Q(X) }\) n'est rien d'autre que le fait que dans \( A[X]\) nous avons \( Q=Q(X)\), comme expliqué dans le lemme \ref{LEMooGKWQooVOyeDX}.
\end{proof}

%--------------------------------------------------------------------------------------------------------------------------- 
\subsection{Monômes}
%---------------------------------------------------------------------------------------------------------------------------

\begin{normaltext}
Les éléments de la forme \( \lambda X^k\) avec \( \lambda\in A\) et \( k\in\eN\) sont des \defe{monômes}{monôme}. 

Nous allons aussi considérer\nomenclature[A]{\( A_n[X]\)}{les polynômes à coefficients dans \( A\) et de degré inférieur à \( n\)}
\begin{equation}
    A_n[X]=\{ P\in A[X]\tq \deg(P)\leq n \}.
\end{equation}
Cela est un sous-module libre.
\end{normaltext}

%--------------------------------------------------------------------------------------------------------------------------- 
\subsection{Évaluation}
%---------------------------------------------------------------------------------------------------------------------------

Soit \( P\in A[X]\). A priori, \( P\) n'est qu'une suite dans \( A\) indexée par \( \eN\). Nous définissons son évaluation sur un élément \( \alpha\in A\) par
\begin{equation}
    P(\alpha)=\sum_ka_k\alpha^k.
\end{equation}
Cette somme est toujours finie.

\begin{normaltext}      \label{NORMooQFTJooLBcPxl}
    L'ensemble \( A[X]\) est une algèbre et donc un espace vectoriel. Il possède un unique élément nul qui est celui dont tous les coefficients sont nuls; cela est immédiat par la construction en tant que suites presque nulles.
\end{normaltext}

Il n'y a a priori pas équivalence entre le fait d'être un polynôme nul et le fait de s'évaluer à zéro sur tous les éléments de \( A\). Cela sera discuté dans le théorème \ref{ThoLXTooNaUAKR} et l'exemple \ref{exVQBooBMPLkD}.

\begin{definition}      \label{DEFooRFBFooKCXQsv}
    Soient un anneau \( A\) et un anneau \( B\) qui contient \( A\) (comme sous-anneau). Pour \( \alpha\in B\) nous définissons \( A[\alpha]_B\) comme étant l'intersection de tous les sous-anneaux de \( B\) contenant \( A\).
\end{definition}
Comme dit plus haut, nous nous permettons d'écrire \( A[\alpha]\) sans préciser \( B\) lorsque ce dernier sera clair dans le contexte.

\begin{proposition}     \label{PROPooPMNSooOkHOxJ}
    Soient un anneau \( A\) et un anneau \( B\) qui contient \( A\) (comme sous-anneau). Pour tout \( \alpha\in B\) nous avons
    \begin{equation}
        A[\alpha]=\{ P(\alpha)\tq P\in A[X] \}
    \end{equation}
    où encore une fois, \( P(\alpha)\) est calculé dans \( B\); le contexte est clair là-dessus.
\end{proposition}

\begin{proof}
    Si \( A'\) est un sous-anneau de \( B\) contenant \( A\) et \( \alpha\), alors \( A'\) contient tous les \( P(\alpha)\) avec \( P\in A[X]\). Nous avons donc
    \begin{equation}
        \{ P(\alpha)\tq P\in A[X] \}\subset A[\alpha].
    \end{equation}
    Par ailleurs, \( \{ P(\alpha)\tq P\in A[X] \}\) est un sous-anneau de \( B\) contenant \( A\) et \( \alpha\). Donc \( A[\alpha]\) y est inclus.
\end{proof}

%--------------------------------------------------------------------------------------------------------------------------- 
\subsection{Polynômes sur un anneau intègre}
%---------------------------------------------------------------------------------------------------------------------------

\begin{theorem}     \label{ThoBUEDrJ}
    L'anneau \( A\) est intègre si et seulement si \( A[X]\) est intègre.
\end{theorem}

\begin{proof}
    Soient \( P\) et \( Q\) des éléments non nuls de \( A[X]\). Vu que l'anneau \( A\) est intègre, nous avons
    \begin{equation}
        \deg(PQ)=\deg(P)+\deg(Q)
    \end{equation}
    et le produit ne peut pas être nul. L'anneau \( A[X]\) est donc intègre.

    Si \( A[X]\) est intègre, \( A\) est intègre parce qu'il peut être vu comme sous anneau.
\end{proof}

\begin{normaltext}
    Si \( A\) n'est pas intègre, soient \( \alpha,\beta\in A\) non nuls tels que \( \alpha\beta=0\). Le produit est polynômes \( X\mapsto \alpha X\) et \( X\mapsto \beta\) est \( (\alpha X)(\beta)=0\); le degré du produit n'est pas la somme des degrés.

    Les personnes qui ont tout compris jusqu'ici remarqueront que la notation «\( X\mapsto P(X)\)» n'est pas correcte parce que du point de vue que nous adoptons ici, un polynôme n'est pas une application.
\end{normaltext}

\begin{corollary}
    Si \( A\) est intègre, les inversibles de \( A[X]\) sont les éléments de \( U(A)\).
\end{corollary}

\begin{proof}
    Pour que \( Q\) soit inversible, il faut un \( P\) tel que \( PQ=1\). Mais l'anneau \( A\) étant intègre, les degrés s'additionnent. Par conséquent ils doivent être de degrés zéro et il faut que \( P,Q\in A\). Enfin pour qu'ils soient inversibles, ils doivent être dans \( U(A)\).
\end{proof}

La \defe{valuation}{valuation!d'un polynôme} du polynôme \( P=\sum_n a_nX^n\), notée \( \val(P)\), est 
\begin{equation}
    \val(P)=\min\{ n\tq a_n\neq 0 \}.
\end{equation}
Nous avons \( \val(P)\leq \deg(P)\) et \( \val(P)=\deg(P)\) si et seulement si \( P\) est un monôme. Si \( P=0\), nous convenons que \( \val(0)=\infty\) et \( \deg(0)=-\infty\).

%---------------------------------------------------------------------------------------------------------------------------
\subsection{Division euclidienne}
%---------------------------------------------------------------------------------------------------------------------------

Le théorème suivant établit la \defe{division euclidienne}{division!euclidienne} dans \( A[X]\) du polynôme \( P\) par un polynôme \( D\).
\begin{theorem}     \label{ThodivEuclPsFexf}
    Soit \( D\neq 0\) dans \( A[X]\) de coefficient dominant inversible dans \( A\). Pour tout \( P\in A[X]\), il existe \( Q,R\in A[X]\) tels que
    \begin{equation}
        P=QD+R
    \end{equation}
    avec \( \deg(R)<\deg(D)\).

    Les polynômes \( Q\) et \( R\) sont déterminés de façon univoque par cette condition. 
\end{theorem}

\begin{definition}\label{DefMPZooMmMymG}
    Le polynôme \( Q\) est le \defe{quotient}{quotient} et \( R\) est le \defe{reste}{reste} de la division euclidienne de \( P\) par \( D\). Si le reste de la division de \( P\) par $D$ est nul on dit que \( D\) \defe{divise}{diviseur!polynôme} \( P\) et on note \( D\divides P\)\nomenclature[A]{\( D\divides P\)}{\( D\) divise \( P\)}. Autrement dit \( D\) divise \( P\) s'il existe \( Q\) tel que \( P=QD\).\footnote{Ceci se rapproche tout naturellement des notions de divisibilité dans un anneau intègre général, vues en sous-section \ref{DivisibiliteAnneauxIntegres}.}
\end{definition}

\begin{normaltext}
    Le théorème \ref{ThodivEuclPsFexf} nous incite à utiliser le degré comme stathme euclidien sur \( A[X]\) dès que \( A\) est un anneau intègre. Or cela ne fonctionne en général pas parce que très peu de polynômes ont a priori un coefficient dominant inversible.
\end{normaltext}

\begin{lemma}       \label{LEMooIDSKooQfkeKp}
    Si \( \eK\) est un corps\footnote{Définition \ref{DefTMNooKXHUd}.}, alors l'anneau \( \eK[X]\) est euclidien et principal.
\end{lemma}

\begin{proof}
    Vu que \( \eK\) est un corps, tous les éléments sont inversibles et le degré donne un stathme par le théorème \ref{ThodivEuclPsFexf}. L'anneau \( \eK[X]\) est donc euclidien et par conséquent principal (proposition \ref{Propkllxnv}). 
\end{proof}

Dans le théorème \ref{ThoCCHkoU} nous donnerons une preuve directe du fait que \( \eK[X]\) est principal en montrant que tous ses idéaux sont principaux. Nous y démontrerons donc un peu moins pour un peu plus cher, mais avec le plaisir de ne pas devoir passer par un stathme.

\begin{definition}[\cite{ooSXFEooEehobn}]  \label{DefDSFooZVbNAX}
    Soit un anneau \( A\). Deux polynômes \( P\) et \( Q\) dans \( A[X]\) sont dits \defe{étrangers}{etranger@étrangers!polynômes} entre eux si \( 1\) est un pgcd\footnote{Définition \ref{DEFooTCUOooWHlbee}.} de \( P\) et \( Q\). Un ensemble de polynômes \( (P_i)_{i\in I}\) est étranger \defe{dans leur ensemble}{étranger!dans leur ensemble} si \( 1\) est un \( \pgcd\) des \( P_i\).
    
Les polynômes \( P\) et \( Q\) sont \defe{premiers entre eux}{premier!deux polynômes entre eux} si les seuls diviseurs communs de \( P\) et \( Q\) sont les inversibles.
\end{definition}

Les notions de polynômes étrangers entre eux ou de polynômes premiers entre eux ne sont pas identiques, comme le montre l'exemple suivant.

\begin{example}[\cite{MonCerveau}]
    Soient dans \( \eZ[X]\) les polynômes \( P(X)=2X+2\) et \( Q(X)=2X^2+2\). Le nombre \( 2\) est diviseur commun et n'est pas un diviseur de \( 1\). Donc \( 1\) n'est pas un pgcd de \( P\) et \( Q\). Ils ne sont pas étrangers.

    Mais ils sont premiers entre eux parce qu'ils n'ont pas d'autres diviseurs communs que les inversibles (\( 1\) et \( -1\)).
\end{example}

%--------------------------------------------------------------------------------------------------------------------------- 
\subsection{Polynôme primitif}
%---------------------------------------------------------------------------------------------------------------------------

\begin{definition}\label{DefContenuPolynome}
    Le \defe{contenu}{contenu}\index{polynôme!contenu} du polynôme \( P=\sum_ia_iX^i\in\eK[X]\) est le pgcd de ses coefficients : $c(P)=\pgcd(a_i)$.
\end{definition}

\begin{definition}[Ordre d'un polynôme]
    Soit \( P\) un polynôme irréductible de degré \( n\) sur \( \eF_p[X]\). L'\defe{ordre}{ordre!d'un polynôme} de \( P\) est
    \begin{equation}
        \min\{ k\tq P\divides X^k-1 \}.
    \end{equation}
\end{definition}

\begin{definition}[Polynôme primitif]           \label{DEFooDVOOooKaPZQC}
    Soit \( p\), un nombre premier et \( P\) un polynôme de degré $n$ dans \( \eF_p[X]\). Nous disons que \( P\) est \defe{primitif}{primitif!polynôme} si 
    \begin{enumerate}
        \item
            \( P\) est unitaire et irréductible,
        \item
            les racines de \( P\) sont d'ordre \( p^n-1\) dans \( \eF_p[X]/P\).
    \end{enumerate}
\end{definition}

\begin{definition}[Polynôme primitif au sens du pgcd]       \label{DEFooAIYGooRAEfHU}
    Soit un anneau \( A\). Un polynôme \( P\in A[X]\) est \defe{primitif au sens du pgcd}{primitif!polynôme!au sens du pgcd} si ses coefficients sont premiers entre eux.
\end{definition}

\begin{normaltext}
    Pour rappel, il y a plusieurs façons de périphraser le fait que les coefficients soient premiers entre eux. Nous pouvons dire \ldots
    \begin{enumerate}
        \item
            Le pgcd de ses coefficients est \( 1\) parce que c'est la définition \ref{DEFooXSPFooPumQSy} d'avoir des nombres premiers entre eux.
        \item
            Le contenu de ses coefficients est \( 1\). Parce que le contenu est précisément le pgcd, définition \ref{DefContenuPolynome}.
    \end{enumerate}
\end{normaltext}

La notion de polynôme primitif au sens du pgcd est particulière aux polynôme à coefficients dans un anneau comme le montre le lemme suivant.

\begin{lemma}
    Si \( \eK\) est un corps, tout polynôme unitaire dans \( \eK[X]\) non nul est primitif au sens du pgcd.
\end{lemma}

\begin{proof}
    Un polynôme unitaire a un \( 1\) parmi ses coefficients, donc le pgcd est forcément \( 1\). 
\end{proof}

Lorsque nous utiliserons la notion de polynôme primitif au sens du \( \pgcd\), nous le mentionnerons explicitement. C'est pas exemple le cas pour le corollaire \ref{CORooZCSOooHQVAOV}.

%--------------------------------------------------------------------------------------------------------------------------- 
\subsection{Racines des polynômes}
%---------------------------------------------------------------------------------------------------------------------------

\begin{definition}
  Soient \( A \) un anneau et \( P \in A[X] \). On appelle
  \defe{racine}{racine!d'un polynôme} un élément \( \alpha \in A \)
  tel que \( P(\alpha) = 0 \); c'est-à-dire que, en remplaçant toutes
  les occurrences de $X$ par $\alpha$ dans l'expression de $P$, on
  obtient $0$.
\end{definition}

\begin{proposition} \label{PropHSQooASRbeA}
    Soient \( A\) un anneau et \( P\) un polynôme non nul dans \( A[X]\). Si \( \alpha\in A\) est une racine de \( P\) alors \( X-\alpha\) divise \( P\), et réciproquement.
\end{proposition}

\begin{proof}
  Nous notons le polynôme \( \mu=X-\alpha\) par analogie avec le polynôme minimal dont il sera question dans la très semblable proposition \ref{PropXULooPCusvE}. Le sens réciproque est clair: si $\mu$ divise $P$, alors $\alpha$ est racine de $P$.

  Pour le sens direct, remarquons que si $\alpha$ est racine de $P$, alors $P$ est de degré au moins égal à \( 1\), et nous pouvons donc effectuer la division euclidienne\footnote{Théorème \ref{ThodivEuclPsFexf}.} de \( P\) par \( \mu\) : il existe des polynômes \( Q\) et \( R\) tels que
    \begin{equation} \label{PropHSQooASRbeA1}
        P=Q\mu+R
    \end{equation}
    avec \( \deg(R)<\deg(\mu)\). Donc \( R\) est une constante,
    élément de $A$: appelons-le $a$. En évaluant
    \eqref{PropHSQooASRbeA1} en \( \alpha\), il vient
    \begin{equation}
        0 = P(\alpha)=Q(\alpha)\mu(\alpha)+a,
    \end{equation}
    et nous en déduisons que \( a=0\), ce qui montre que \( P=Q\mu\) et que \( \mu\) divise \( P\).
\end{proof}

\begin{definition}[Racine simple et multiple d'un polynôme]
  Soit \( A\) un anneau ainsi qu'un polynôme \( P\in A[X]\) et \( \alpha\in A\) racine de $P$. La \defe{multiplicité}{multiplicité!racine d'un polynôme} de \( \alpha\) par rapport à \( P\) est l'entier \( h\) tel que \( P\) est divisible par \( (X-\alpha)^h\) mais pas divisible par \( (X-\alpha)^{h+1}\).  Nous noterons \( \theta_{\alpha}(P)\)\nomenclature[A]{\( \theta_{\alpha}(P)\)}{la multiplicité de \( \alpha\) par rapport à \( P\)} la multiplicité de \( \alpha\) par rapport à \( P\).
\end{definition}

Pour une définition générale d'une racine simple de l'équation \( f(x)=0\), voir la définition \ref{DEFooXSOQooAnWqKM}.

La proposition \ref{PropHSQooASRbeA} nous indique que toute racine est de multiplicité au moins \( 1\).

\begin{proposition} \label{PropahQQpA}
  L'élément \( \alpha\in A\) est une racine de multiplicité \( h\) du
  polynôme \( P\) si et seulement s'il existe \( Q\in A[X]\) tel que
  \( P=(X-\alpha)^hQ\) avec \( Q(\alpha)\neq 0\).
\end{proposition}

\begin{lemma}       \label{LemIeLhpc}
    Soient \( P\) et \( Q\) des polynômes non nuls de \( A[X]\) et \( \alpha\in A\). Alors
    \begin{enumerate}
        \item
            \( \theta_{\alpha}(P+Q)\leq\min\{
            \theta_{\alpha}(P),\theta_{\alpha}(Q) \}\), et l'égalité a
            lieu si \( \theta_{\alpha}(P)\neq \theta_{\alpha}(Q)\);
        \item     \label{ItemIeLhpciv}
            \( \theta_{\alpha}(PQ)\geq
            \theta_{\alpha(P)}+\theta_{\alpha}(Q)\), et l'égalité a
            lieu si \( A \) est intègre.
    \end{enumerate}
\end{lemma}

\begin{theorem} \label{ThoSVZooMpNANi} Soit \( A\) un anneau intègre
  et \( P\in A[X]\setminus\{ 0 \}\), un polynôme de degré \( n\). Si
  \( \alpha_1,\ldots, \alpha_p\in A\) sont des racines deux à deux
  distinctes de multiplicités \( k_1,\ldots, k_p\), alors il existe \(
  Q\in A[X]\), de degré \( n-p\), tel que \(
  P=Q\prod_{i=1}^p(X-\alpha_i)^{k_i}\) et \( Q(\alpha_i)\neq 0\) pour
  tout $i$.
    De plus la somme des multiplicités des racines de \( P\) est au plus \( \deg(P)\).
\end{theorem}
\index{factorisation!de polynôme}

\begin{proof}
    Si \( p=1\), soit \( \alpha\) une racine de multiplicité \( k\) de \( P\). La définition de la multiplicité d'une racine nous dit que \( P\) est divisible par \( (X-\alpha)^k\) mais pas par \( (X-\alpha)^{k+1}\). Donc il existe \( Q\in \eA[X]\) tel que \( P=Q(X-\alpha)^k\). Il reste à voir que \( Q(\alpha)\neq 0\). Cela est une conséquence de la proposition \ref{PropHSQooASRbeA} : si \( Q(\alpha)\) était nul, on pourrait lui factoriser \( (X-\alpha)\) et donc avoir \( (X-\alpha)^{k+1}\) qui se factorise dans \( P\), ce qui n'est pas possible.

    Nous supposons que \( p\geq 2\) et nous effectuons une récurrence sur \( p\). Nous considérons donc les \( p-1\) premières racines \( \alpha_1,\ldots, \alpha_{p-1}\) et un polynôme \( R\in\eA[X]\) tel que \( R(\alpha_i)\neq 0\) pour \( i=1,\ldots, p-1\) et
    \begin{equation}
        P=\underbrace{(X-\alpha_1)^{k_1}\ldots (X-\alpha_{p-1})^{k_{p-1}}}_SR.
    \end{equation}
    Par hypothèse \( P(\alpha_p)=S(\alpha_p)R(\alpha_p)=0\). L'anneau \( \eA\) étant intègre, \( S(\alpha_p)\neq 0\) parce que \( \alpha_i\neq \alpha_p\) pour \( i\neq p\). Par conséquent, \( R(\alpha_p)=0\).
    
    Nous devons encore vérifier que la multiplicité \( \alpha_p\) est \( k_p\) par rapport à \( R\). Pour cela nous utilisons le point \ref{ItemIeLhpciv} du lemme \ref{LemIeLhpc} afin de dire que le degré de \( \alpha_p\) pour \( P=SR\) est \( k_p\). Par conséquent
    \begin{equation}
        R=(X-\alpha_p)^{k_p}T
    \end{equation}
    avec \( T(\alpha_p)\neq 0\) et enfin
    \begin{equation}
        P=\prod_{i=1}^p(X-\alpha_i)T.
    \end{equation}
    De plus \( T(\alpha_i)\neq 0\), sinon \( R(\alpha_i)\) serait nul.
\end{proof}

\begin{corollary}[Conséquence du lemme de Gauss\cite{ooCDLEooEQGSvn}]       \label{CORooZCSOooHQVAOV}
    Soient \( A\) un anneau factoriel et \( \Frac(A)\) son corps des fractions. Un polynôme non constant \( P\in A[X]\) est irréductible (sur \( A\)) si et seulement s'il est irréductible et primitif au sens du pgcd\footnote{Définition \ref{DEFooAIYGooRAEfHU}.} sur \( \Frac(A)[X]\). 
\end{corollary}

%--------------------------------------------------------------------------------------------------------------------------- 
\subsection{Quelques identités}
%---------------------------------------------------------------------------------------------------------------------------

\begin{lemma}   \label{LemISPooHIKJBU}
    Quelques identités de polynômes.
    \begin{enumerate}
        \item   \label{ItemLTBooAcyMtN}
            Si \( n\) est impair, alors \( 1+X\) divise \( 1+X^n\).
        \item\label{ItemLTBooAcyMtNii}
            Pour tout \( n\) nous avons \( X^n-1=(X-1)(1+X+\cdots +X^{n-1})\).
        \item
            \( X^n-a^n=(X-a)\sum_{i=0}^{n-1}a^iX^{n-1-i}\).
    \end{enumerate}
\end{lemma}

\begin{proof}
  Nous démontrons uniquement le point \ref{ItemLTBooAcyMtNii}, puisque
  les autres ont été vus en début de chapitre\footnote{Voir l'égalité
    \eqref{Eqarpurmkbk}.}. Le cas \( n=1\) est évident. Procédons
  alors par récurrence en considérant un nombre entier impair \( n\) :
    \begin{subequations}
        \begin{align}
            1+X^{n+2}&=1+X^n+X^{n+2}-X^n\\
                    &=(1+X)P+X^n(X^2-1)\\
                    &=(1+X)P+X^n(X+1)(X-1)\\
                    &=(1+X)\big( P+X^n(X-1) \big).
        \end{align}
    \end{subequations}
\end{proof}

%--------------------------------------------------------------------------------------------------------------------------- 
\subsection{Polynômes en plus de variables}
%---------------------------------------------------------------------------------------------------------------------------

\begin{definition}      \label{DEFooZNHOooCruuwI}
    Nous définissons les polynômes en \( n\) variables, dont l'ensemble est noté \( A[X_1,\ldots, X_n]\) comme étant l'ensemble des suites indexées par \( \eN^n\) et dont seulement une quantité finie de coefficients sont non nuls.
\end{definition}

Je vous laisse écrire la loi de multiplication et les suites auxquelles correspondent les polynômes \( X_1\),\ldots,  \( X_n\).


