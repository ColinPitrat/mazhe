% This is part of Mes notes de mathématique
% Copyright (c) 2011-2016
%   Laurent Claessens
% See the file fdl-1.3.txt for copying conditions.

%+++++++++++++++++++++++++++++++++++++++++++++++++++++++++++++++++++++++++++++++++++++++++++++++++++++++++++++++++++++++++++ 
\section{Théorème de la moyenne}
%+++++++++++++++++++++++++++++++++++++++++++++++++++++++++++++++++++++++++++++++++++++++++++++++++++++++++++++++++++++++++++

\begin{theorem}[\cite{MonCerveau}]      \label{ThoooEZLGooMChwLT}
    Soit \( Q\) un compact connexe par arcs et une fonction continue \( f\colon Q\to \eR\). Si \( \lambda\) est la mesure de Lebesgue, alors il existe \( a\in Q\) tel que
    \begin{equation}
        f(a)=\frac{1}{ \lambda(Q) }\int_Qfd\lambda
    \end{equation}
\end{theorem}

\begin{proof}
    En posant \( I=\int_Qfd\lambda\) nous avons immédiatement
    \begin{equation}        \label{EqooTYQCooVxdazW}
        \min(f)\lambda(Q)\leq I\leq \max(f)\lambda(Q)
    \end{equation}
    où le minimum et le maximum existent parce que \( f\) est continue sur un compact. Si une des deux inégalités est une égalité alors la fonction est constante. En effet supposons que la première inégalité soit une égalité; si la fonction n'était pas constante, il existerait une boule sur laquelle \( f\) serait strictement supérieure à \( \min(f)\). En intégrant d'abord sur cette boule et ensuite sur le complémentaire nous obtenons une intégrale plus grande que \( \min(f)\lambda(Q)\).

    Soit \( \epsilon>0\). Il existe \( \alpha,\beta\in Q\) tels que \( f(\alpha)\leq\min(f)+\epsilon\) et \( f(\beta)\geq\max(f)-\epsilon\). Soit \( \gamma\colon \mathopen[ 0 , 1 \mathclose]\to Q\) un chemin continu tel que \( \gamma(0)=\alpha\) et \( \gamma(1)=\beta\). La fonction \( f\circ \gamma\colon \mathopen[ 0 , 1 \mathclose]\to \eR\) est alors continue et vérifie \( (f\circ\gamma)(0)\leq \min(f)+\epsilon\) et \( (f\circ\gamma)(1)\geq \max(f)-\epsilon\).

    Si \( \epsilon\) est assez petit et vu que les inégalités \eqref{EqooTYQCooVxdazW} sont strictes,
    \begin{equation}
        \lambda(Q)(f\circ\gamma)(0)\leq \min(f)\lambda(Q)+\epsilon\lambda(Q)<I<\max(f)\lambda(Q)-\epsilon\lambda(Q)\leq\lambda(Q)(f\circ \gamma)(1).
    \end{equation}
    Par le théorème des valeurs intermédiaires \ref{ThoValInter}, il existe \( t_0\in\mathopen[ 0 , 1 \mathclose]\) tel que \( \lambda(Q)(f\circ\gamma)(t_0)=I\). Le point \( a=\gamma(t_0)\) vérifie
    \begin{equation}
        f(a)=\frac{1}{ \lambda(Q) }\int_Qfd\lambda.
    \end{equation}
\end{proof}

%+++++++++++++++++++++++++++++++++++++++++++++++++++++++++++++++++++++++++++++++++++++++++++++++++++++++++++++++++++++++++++ 
\section{Mesure à densité}
%+++++++++++++++++++++++++++++++++++++++++++++++++++++++++++++++++++++++++++++++++++++++++++++++++++++++++++++++++++++++++++

%--------------------------------------------------------------------------------------------------------------------------- 
\subsection{Théorème de Radon-Nikodym}
%---------------------------------------------------------------------------------------------------------------------------

\begin{proposition}[Produit d'une mesure par une fonction]
    Si \( (S,\tribF,m_1)\) est un espace mesuré, et si \( f\colon S\to \eR\) est intégrable, et si \( B\) est un ensemble mesurable, nous définissons \( fm_1\) par
    \begin{equation}
        m_2(B)=(fm_1)(B)=\int_Bf(t)dm_1(t).
    \end{equation}
    Cela est une mesure positive sur \( (S,\tribF)\).
\end{proposition}

\begin{proof}
    D'abord pour l'ensemble vide : \( m_2(\emptyset)=\int_{\emptyset}fdm_1=0\).

    Si \( A_n\) sont des éléments disjoints de \( \tribF\) tels que \( \bigcup_nA_n\in\tribF\). Alors en utilisant la proposition \ref{PropOPSCooVpzaBt}, nous avons le calcul suivant :
    \begin{equation}
        m_2\big( \bigcup_nA_n \big)=\int_{\bigcup_nA_n}f(t)dm_1(t)=\sum_{n}\int_{A_n}f(t)dm_1(t)=\sum_nm_2(A_n).
    \end{equation}
\end{proof}

\begin{definition}[\cite{PersoFeng}]
    Soient \( \mu\) et \( \nu\) deux mesures sur l'espace mesurable \( (\Omega,\tribA)\). Nous disons que la mesure \( \mu\) est \defe{dominée}{dominée!mesure} par \( \nu\) si pour tout ensemble mesurable \( A\), \( \nu(A)=0\) implique \( \mu(A)=0\).

    Si \( \nu\) est une mesure positive et \( \mu\) une mesure, nous disons que \( \mu\) est \defe{absolument continue}{mesure!absolument continue} par rapport à \( \nu\) si \( \nu(A)=0\) implique \( \mu(A)=0\). On note aussi \( \mu\ll\nu\)\nomenclature[Y]{$\mu\ll\nu$}{La mesure \( \mu\) est absolument continue par rapport à la mesure \( \nu\).}.
\end{definition}

La mesure \( \mu\) est \defe{portée}{portée!mesure} par l'ensemble \( E\in\tribA\) si pour tout \( A\in\tribA\), 
\begin{equation}
    \mu(A)=\mu(A\cap E).
\end{equation}

Nous écrivons que \( \mu\perp\nu\)\nomenclature[Y]{\( \mu\perp\nu\)}{mesures perpendiculaires} s'il existe un ensemble \( E\in\tribA\) tel que \( \mu\) soit porté par \( E\) et \( \nu\) soit porté par \( \complement E\).

\begin{theorem}[Radon-Nikodym\cite{NikoLi}]
    Soient \( \mu\) et \( \nu\) deux mesures \( \sigma\)-finies sur un espace métrisable \( (\Omega,\tribA)\).
    \begin{enumerate}
        \item
            Il existe un unique couple de mesures \( \mu_1\) et \( \mu_2\) telles que
            \begin{enumerate}
                \item
                    \( \mu=\mu_1+\mu_2\)
                \item
                    \( \mu_1\) est dominé par \( \nu\)
                \item
                    \( \mu_2\perp \nu\).
            \end{enumerate}
            Dans ce cas, les mesures \( \mu_1\) et \( \mu_2\) sont positives et \( \sigma\)-finies.
        \item
            À égalité \(  \nu\)-presque partout près, il existe une unique fonction mesurable positive \( f\) telle que pour tout mesurable \( A\),
            \begin{equation}
                \mu_1(A)=\int_Ad\mu_1=\int_{\Omega}\mtu_Afd \nu.
            \end{equation}
        \item
            À égalité \( \nu\)-presque partout près, il existe une unique fonction positive mesurable \( h\) telle que \( \mu_1=h\nu\).
    \end{enumerate}
\end{theorem}
\index{théorème!Radon-Nikodym}
%TODO : une preuve

\begin{corollary}   \label{CorZDkhwS}
    Si \( \mu\) es une mesure \( \sigma\)-finie dominée par la mesure \( \sigma\)-finie \( m\), alors \( \mu\) possède une unique fonction de densité.
\end{corollary}

\begin{corollary}       \label{CorDomDens}
    Soient \( \mu\) et \( m\), deux mesures positives \( \sigma\)-finies sur \( (\Omega,\tribA)\). Alors \( m\) domine \( \mu\) si et seulement si \( \mu\) possède une densité par rapport à \( m\).
\end{corollary}
 
\begin{proof}
    Si \( \mu\) est dominée par \( m\), alors la décomposition \( \mu=\mu+0\) satisfait le théorème de Radon-Nikodym. Par conséquent il existe une fonction \( f\) telle que
    \begin{equation}
        \mu(A)=\int_Afdm.
    \end{equation}
    Cette fonction est alors une densité pour \( \mu\) par rapport à \( m\).

    Pour la réciproque, nous supposons que \( \mu\) a une densité \( f\) par rapport à \( m\), et que \( A\) est une ensemble de \( m\)-mesure nulle :
    \begin{equation}
        m(A)=\int_{\Omega}\mtu_Adm=0.
    \end{equation}
    Cela signifie que la fonction \( \mtu_A\) est \( m\)-presque partout nulle. La fonction produit \( \mtu_Af\) est également nulle \( m\)-presque partout, et par conséquent
    \begin{equation}
        \mu(A)=\int_{\Omega}\mtu_Afdm=0.
    \end{equation}
\end{proof}

\begin{probleme}
    Est-ce que la démonstration de cela ne demande pas la convergence monotone d'une façon ou d'une autre ?
\end{probleme}

%--------------------------------------------------------------------------------------------------------------------------- 
\subsection{Mesure complexe}
%---------------------------------------------------------------------------------------------------------------------------

\begin{definition}[Mesure complexe\cite{TLRRooOjxpTp}] \label{DefGKHLooYjocEt}
    Si \( (\Omega,\tribA)\) est un espace mesurable, une \defe{mesure complexe}{mesure!complexe} est une application \( \mu\colon \tribA\to \eC\) telle que
    \begin{enumerate}
        \item
            $\mu(\emptyset)=0$,
        \item
            \( \nu\) est sous-additive : si les ensembles \( A_i\in\tribA\), alors \( \sum_i\mu(A_i)=\mu(\bigcup_iA_i)\).
    \end{enumerate}
\end{definition}
Notons que la série $\sum_i\mu(A_i)$ est alors nécessairement absolument convergente. En effet changer l'ordre de la somme ne change pas l'union, et donc ne change pas la valeur de la somme. Si \( \sigma\colon \eN\to \eN\) est une permutation, 
\begin{equation}
    \sum_i\mu(A_{\sigma(i)})=\mu\big( \bigcup_iA_{\sigma(i)} \big)=\mu\big( \bigcup_iA_i \big)=\sum_i\mu(A_i).
\end{equation}
Le théorème \ref{PopriXWvIY} dit alors que la somme doit être absolument convergente.


\begin{theorem}[Radon-NikoDym complexe\footnote{L'histoire du nom de ce théorème est intéressante. Lorsque monsieur et madame Rèmederdonnukodym apprirent que leurs amis, les Rèmedelaboulechevelue avaient appelé leur fils Théo, ils décidèrent d'en faire autant. C'est en souvenir de ces circonstances que monsieur Nikodym (prénommé Radon) décida de faire des math.}]\label{ThoZZMGooKhRYaO}
    Soit \( \mu\) une mesure positive sur \( (\Omega,\tribA)\) et \( \nu\) une mesure complexe. Alors
    \begin{enumerate}
        \item
            Il existe un unique couple de mesures complexes \( \nu_a\), \( \nu_s\) sur \( (\Omega,\tribA)\) tel que
            \begin{enumerate}
                \item
                    \( \nu=\nu_a+\nu_s\)
                \item
                    \( \nu_a\ll\mu\)
                \item
                    \( \nu_s\perp \mu\).
            \end{enumerate}
        \item
            Ces mesures satisfont alors \( \nu_a\perp\nu_s\).
        \item
            Il existe une fonction intégrable \( h\colon \Omega\to \eC\) telle que \( \nu_a=h\mu\).
        \item
            La fonction \( h\) est unique à \( \mu\)-équivalence près.
        \item   \label{ItemDIXOooFqOkgGv}
            Si de plus \( \nu\ll \mu\) alors \( \nu=h\mu\).
    \end{enumerate}
\end{theorem}
\index{théorème!Radon-Nikodym!complexe}
\begin{proof}
    No proof.
\end{proof}

\begin{remark}  \label{RemSYRMooZPBhbQ}
    Le point \ref{ItemDIXOooFqOkgGv} est souvent utilisé sous la forme
    \begin{equation}
        \nu(A)=\int_{\Omega}\mtu_A(\omega)h(\omega)d\mu(\omega)=\int_{A}h(\omega)d\mu(\omega).
    \end{equation}
\end{remark}

%--------------------------------------------------------------------------------------------------------------------------- 
\subsection{Théorème d'approximation}
%---------------------------------------------------------------------------------------------------------------------------

\begin{theorem}[Théorème d'approximation\cite{YHRSDGc}]     \label{ThoAFXXcVa}
    Soit \( (X,\tribB,\mu)\) un espace mesuré où \( \tribB\) sont les boréliens de \( X\). Soit \( A\in \tribB\) tel que \( A\subset W\) où \( W\) est un ouvert avec \( \mu(W)<\infty\). Soit aussi \( \epsilon>0\).
    \begin{enumerate}
        \item
            Il existe un fermé \( F\) et un ouvert \( V\) tels que \( \mu(V)<\infty\) et
            \begin{equation}
                F\subset A\subset V
            \end{equation}
            et \( \mu(V\setminus F)<\epsilon\).
        \item
            Il existe \( f\in C^0(X,\eR)\) nulle hors de \( W\) vérifiant \( 0\leq f\leq 1\) et
            \begin{equation}
                \int_X| \mtu_A-f |^pd\mu(x)<\epsilon.
            \end{equation}
    \end{enumerate}
\end{theorem}
% TODO : la preuve est dans la référence. Il faut replacer ce théorème après la définition de l'intégrale.

