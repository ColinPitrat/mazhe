% This is part of Le Frido
% Copyright (c) 2008-2018
%   Laurent Claessens
% See the file fdl-1.3.txt for copying conditions.

Plusieurs choses sur les espaces vectoriels normés (dont la définition \ref{DefNorme}) ont déjà été vues dans la section \ref{SECooWKJNooKOqpsx}. Voir aussi le thème \ref{THEMEooUJVXooZdlmHj}.

On fixe maintenant une définition largement utilisée dans la suite.
\begin{definition}      \label{DefAQIQooYqZdya}
	 Soient $U$ et $V$, deux ouverts d'un espace vectoriel normé. Une application $f$ de $U$ dans $V$ est un \defe{difféomorphisme}{difféomorphisme} si elle est bijective, différentiable et dont l'inverse $f^{-1}:V\to U $ est aussi différentiable.
\end{definition}

\begin{remark}
	Il n'est pas possible d'avoir une application inversible d'un ouvert de $\eR^m$ vers un ouvert de $\eR^n$ si $m\neq n$. Il n'y a donc pas de notion de difféomorphismes entre ouverts de dimensions différentes.
\end{remark}

%+++++++++++++++++++++++++++++++++++++++++++++++++++++++++++++++++++++++++++++++++++++++++++++++++++++++++++++++++++++++++++
\section{Suites}
%+++++++++++++++++++++++++++++++++++++++++++++++++++++++++++++++++++++++++++++++++++++++++++++++++++++++++++++++++++++++++++
\label{SECooLLUGooOwZRyI}

%---------------------------------------------------------------------------------------------------------------------------
\subsection{Limites, convergence}
%---------------------------------------------------------------------------------------------------------------------------

Nous disons qu'une suite réelle $(x_n)$ converge\footnote{Voir la définition~\ref{PropLimiteSuiteNum} pour plus de détail.} vers $\ell$ lorsque pour tout $\varepsilon$, il existe un $N$ tel que
\begin{equation}
	n>N\Rightarrow | x_n-\ell |\leq\varepsilon.
\end{equation}
Le concept fondamental de cette définition est la notion de valeur absolue qui permet de donner la «distance» entre deux réels. Dans un espace vectoriel normé quelconque, cette notion est généralisée par la distance associée à la norme (définition~\ref{DefEVNetDistance}). Nous pouvons donc facilement définir le concept de convergence d'une suite dans un espace vectoriel normé.

\begin{definition}		\label{DefCvSuiteEGVN}
	Soit une suite $(x_n)$ dans un espace vectoriel normé $V$. Nous disons qu'elle est
    \defe{convergente}{convergence!dans un espace vectoriel normé} s'il existe un élément $\ell\in V$ tel que
	\begin{equation}
		\forall \varepsilon>0,\,\exists N\in\eN\tq n\geq N\Rightarrow \| x_n-l \|<\varepsilon.
	\end{equation}
	Dans ce cas, $\ell$ est appelé la \defe{limite}{limite!suite} de la suite $(x_n)$.
\end{definition}


\begin{lemma}		\label{LemLimAbarA}
	Soit $(x_n)$ une suite convergente contenue dans un ensemble $A\subset V$. Alors la limite $x_n$ appartient à $\bar A$.
\end{lemma}

\begin{proof}
	Supposons que nous ayons une partie $A$ de $V$, et une suite $(x_n)$ dont la limite $\ell$ se trouve hors de $\bar A$. Dans ce cas, il existe un $r>0$ tel que\footnote{Une autre manière de dire la même chose : si $\ell\notin\bar A$, alors $d(\ell,A)>0$.} $B(\ell,r)\cap A=\emptyset$. Si tous les éléments $x_n$ de la suite sont dans $A$, il n'y en a donc aucun tel que $d(x_n,\ell)=\| x_n-\ell \|<r$. Cela contredit la notion de convergence $x_n\to \ell$.
\end{proof}

Nous avons déjà mentionné dans l'exemple~\ref{ParlerEncoredeF} que zéro était un point adhérent à l'ensemble $F=\{ (-1)^n/n\tq n\in\eN_0 \}$. Nous savons maintenant que $0$ étant la limite de la suite, il est automatiquement adhérent à l'ensemble des éléments de la suite.

\begin{corollary}		\label{CorAdhEstLim}
	Soit $a$ un point de l'adhérence d'une partie $A$ de $V$. Alors il existe une suite d'éléments dans $A$ qui converge vers $a$.
\end{corollary}

\begin{proof}
	Si $a\in A$, alors nous pouvons prendre la suite constante $x_n=a$. Si $a$ n'est pas dans $A$, alors $a$ est dans $\partial A$, et pour tout $n$, il existe un point de $A$ dans la boule $B(a,\frac{1}{ n })$. Si nous nommons $x_n$ ce point, la suite ainsi construite est une suite contenue dans $A$ et qui converge vers $a$ (ce dernier point est laissé à la sagacité du lecteur ou de la lectrice).
\end{proof}

En termes savants, ce corollaire signifie que la fermeture $\bar A$ est composé de $A$ plus de toutes les limites de toutes les suites contenues dans $A$.

%---------------------------------------------------------------------------------------------------------------------------
\subsection{Critère de Cauchy}
%---------------------------------------------------------------------------------------------------------------------------


\begin{lemma}
    Une suite de Cauchy dans un espace vectoriel normé admettant une sous-suite convergente est elle-même convergente vers la même limite.
\end{lemma}

\begin{proof}
    Soit \( (a_n)\) une suite de Cauchy dans un espace vectoriel normé \( E\) et \( \ell\) la limite d'une sous-suite de \( (a_n)\). Soit \( \epsilon>0\) et \( N\in \eN\) tel que \( \| a_m-a_p \|<\epsilon\) dès que \( m,p\geq N\). Nous allons montrer que si \( k>N\) alors \( \| a_k-\ell \|<2\epsilon\). Pour cela nous considérons un \( n>N\) tel que \( \| a_n-\ell \|\leq \epsilon\) et nous calculons
    \begin{equation}
        \| a_k-\ell \|\leq \| a_k-a_n \|+\| a_n-\ell \|\leq 2\epsilon.
    \end{equation}
\end{proof}

\begin{definition}
    Nous disons que deux suites \( (u_n)\) et \( (v_n)\) sont \defe{équivalentes}{equivalence@équivalence!de suites} s'il existe une fonction \( \alpha\colon \eN\to \eR\) telle que
    \begin{enumerate}
        \item
            pour tout \( n\) à partir d'un certain rang, \( u_n=v_n\alpha(n)\)
        \item
            \( \alpha(n)\to 1\).
    \end{enumerate}
\end{definition}

\begin{lemma}
    Si les suites \( (u_n)\) et \( (v_n)\) sont équivalentes et si \( (v_n)\) admet une limite \( l\) différente de \( 1\), alors les suites \( (\ln u_n)\) et \( (\ln v_n)\) sont équivalentes.
\end{lemma}

\begin{proof}
    En effet si \( u_n=v_n\alpha(n)\) alors
    \begin{equation}
        \ln(u_n)=\ln(v_n)+\ln\big( \alpha(n) \big)=\ln(v_n)\left( 1+\frac{ \ln\big( \alpha(n) \big) }{ \ln(v_n) } \right),
    \end{equation}
    et comme \( \alpha(n)\to 1\), la parenthèse tend vers \( 1\).
\end{proof}

\begin{lemma}[Formule de Stirling\cite{MEHuVnb}]        \label{LemCEoBqrP}
    Nous avons l'équivalence de suites
    \begin{equation}
        n!\sim \left( \frac{ n }{ e } \right)^n\sqrt{2\pi n}.
    \end{equation}
\end{lemma}
\index{formule!Stirling}

Dans le cas des espaces de dimension finie, le fait d'être complet est automatique, comme le montre la proposition suivante.
\begin{proposition}     \label{PROPooGJDTooXOoYfw}
    Soit \( \big( E,\| . \| \big)\) un espace vectoriel normé de dimension finie sur un corps \( \eK\) qui est complet\footnote{La définition est~\ref{DefKCGBooLRNdJf}, mais si vous n'avez pas envie de vous embarquer trop loin, dites juste «toutes les suites de Cauchy convergent». Typiquement c'est \( \eR\) ou \( \eC\).}. Alors \( E\) est complet\footnote{Définition~\ref{DEFooHBAVooKmqerL}.}.
\end{proposition}
Pour rappel, la complétude de l'espace métrique \( \eR\) est la proposition~\ref{PROPooTFVOooFoSHPg}.

\begin{proof}
    Nous considérons une suite de Cauchy \( (f_n)\) dans \( E\) et si \( \{ e_{\alpha} \} \) est une base orthonormée de \( E\) nous définissons les coefficients \( f_n=\sum_{\alpha}a_{n\alpha}e_{\alpha} \). La somme sur \( \alpha\) est finie par hypothèse sur la dimension de \( E\).

    Nous avons
    \begin{equation}
        \| f_n-f_m \|=\| \sum_{\alpha}(a_{n\alpha}-a_{m\alpha})e_{\alpha} \|=\sum_{\alpha}| a_{n\alpha}-a_{m\alpha} |^2.
    \end{equation}
    Pour tout \( \epsilon\), il existe \( N\) tel que si \( m,n>N\) alors \( | a_{n\alpha}-a_{m\alpha} |<\sqrt{ \epsilon }\). Autrement dit, pour chaque \( \alpha\), la suite \( (a_{n\alpha})_{\alpha\in \eN}\) est de Cauchy dans \( \eK\) et converge donc dans \( \eK\). Soit \( a_{\alpha}\) la limite et définissons \( f=\sum_{\alpha}a_{\alpha}e_{\alpha}\). Nous avons alors
    \begin{equation}
        \| f_n-f \|=\| \sum_{\alpha}(a_{n\alpha}-a_{\alpha})e_{\alpha} \|,
    \end{equation}
    dont la limite \( n\to \infty\) est bien zéro. Donc la suite \( (f_n)\) converge vers \( f\in E\). L'espace \( E\) est alors complet.
\end{proof}

%---------------------------------------------------------------------------------------------------------------------------
\subsection{Approximation}
%---------------------------------------------------------------------------------------------------------------------------

Le lemme suivant est surtout intéressant en dimension infinie.
\begin{lemma}
    Soit un espace vectoriel normé \( V\) et un sous-espace vectoriel dense \( A\). Soit \( v\in V\); il existe une suite \( (v_n)\) dans \( A\) telle que \( v_n\stackrel{V}{\longrightarrow}v\) et \( \| v_n \|\leq \| v \|\) pour tout \( n\).
\end{lemma}

\begin{proof}
    Vu que \( A\) est dense, il existe une suite \( a_n\) dans \( A\) telle que \( a_n\to v\). Ensuite il suffit de poser
    \begin{equation}
        v_n=\frac{ n }{ n+1 }\frac{ \| v \| }{ \| a_n \| }a_n.
    \end{equation}
    Par construction nous avons toujours
    \begin{equation}
        \| v_n \|=\frac{ n }{ n+1 }\| v \|\leq \| v \|.
    \end{equation}
    Et de plus, la norme étant continue\footnote{Où dans le calcul suivant nous utilisons la continuité de la norme ? Posez-vous la question.},
    \begin{equation}
        \lim_{n\to \infty} v_n=\lim_{n\to \infty} \frac{ n }{ n+1 }\lim_{n\to \infty} \frac{ \| v \| }{ \| v_n \| }\lim_{n\to \infty} v_n=v.
    \end{equation}

    Le fait que \( v_n\) soit dans \( A\) est dû au fait que \( A\) soit vectoriel.
\end{proof}

\begin{proposition}     \label{PROPooVEMGooYKhMFy}
    Soit un espace vectoriel normé \( V\) et un sous-espace vectoriel dense \( A\). Soit \( v\in V\); pour tout \( a\in \eR\) nous avons
    \begin{equation}
        \sup\{ | v\cdot a |\tq a\in A\text{ et }\| a \|\leq \lambda \}=\lambda\| v \|.
    \end{equation}
\end{proposition}

\begin{proof}
    D'abord pour tout \( a\in A\) vérifiant \( \| a \|\leq \lambda\) l'inégalité de Cauchy-Schwarz~\ref{ThoAYfEHG} donne
    \begin{equation}
        | v\cdot a |\leq \| v \|\| a \|\leq \lambda\| v \|.
    \end{equation}
    Donc le supremum dont on parle est majoré par \( \lambda\| v \|\).

    Il nous faut l'inégalité dans l'autre sens. Par densité nous pouvons choisir une suite \( v_n\in A\) tel que \( v_n\to v\). Ensuite nous posons
    \begin{equation}
        a_n=\frac{ \lambda }{ \| v_n \| }v_n.
    \end{equation}
    Nous avons \( \| a_n \|=\lambda\) pour tout \( n\) et
    \begin{equation}
        | v\cdot a_n |=\frac{ \lambda }{ \| v_n \| }| v\cdot v_n |,
    \end{equation}
    et en passant à la limite,
    \begin{equation}
        \lim_{n\to \infty} | v\cdot a_n |=\frac{ \lambda }{ \| v \| }\| v\cdot v \|=\lambda\| v \|.
    \end{equation}
    Donc l'ensemble sur lequel nous prenons le supremum contient une suite convergente vers \( \lambda\| v \|\). Le supremum est donc au moins aussi grand que cela.
\end{proof}


%+++++++++++++++++++++++++++++++++++++++++++++++++++++++++++++++++++++++++++++++++++++++++++++++++++++++++++++++++++++++++++
\section{Équivalence des normes}
%+++++++++++++++++++++++++++++++++++++++++++++++++++++++++++++++++++++++++++++++++++++++++++++++++++++++++++++++++++++++++++
\label{normes_equiv}

Au premier coup d'œil, les notions dont nous parlons dans ce chapitre ont l'air très générales. Nous prenons en effet n'importe quel espace vectoriel $V$ de dimension finie, et nous le munissons de n'importe quelle norme (rien que dans $\eR^m$ nous en avons défini une infinité par l'équation \eqref{EqDeformeLp}). À partir de ces données, nous définissons les boules, la topologie, l'adhérence, etc.

%---------------------------------------------------------------------------------------------------------------------------
\subsection{En dimension finie}
%---------------------------------------------------------------------------------------------------------------------------

Dans $\eR^n$, les normes $\| . \|_{L^1}$, $\| . \|_{L^2}$ et $\| . \|_{\infty}$ ne sont pas égales. Cependant elles ne sont pas complètement indépendantes au sens où l'on sent bien que si un vecteur sera grand pour une norme, il sera également grand pour les autres normes; les normes «vont dans le même sens». Cette notion est précisée par le concept de norme équivalente.

\begin{definition}		\label{DefEquivNorm}
    Deux normes $N_1$ et $N_2$ sur $\eR^m$ sont \defe{\wikipedia{fr}{Norme_équivalente}{équivalentes}}{equivalence@équivalence!norme}\index{norme!équivalence}\index{équivalence!de norme} s'il existe deux nombres réels strictement positifs $k_1$ et $k_2$ tels que
	\begin{equation}
		k_1N_1(x)\leq N_2(x)\leq k_2 N_1(x),
	\end{equation}
	pour tout $x$ dans $\eR^m$. Dans ce cas nous écrivons que $N_1\sim N_2$.
\end{definition}
Il est possible de démontrer que cette notion est une relation d'équivalence (définition~\ref{DefHoJzMp}) sur l'ensemble des normes existantes sur $\eR^m$.

\begin{proposition} \label{PropLJEJooMOWPNi}
    Pour \( \eR^N\), nous avons les équivalences de normes $\| . \|_{L^1}\sim\| . \|_{L^2}$, $\| . \|_{L^1}\sim\| . \|_{\infty}$ et $\| . \|_{L^2}\sim\| . \|_{\infty}$. Plus précisément nous avons les inégalités
    \begin{enumerate}
        \item\label{ItemABSGooQODmLNi}
           $ \| x \|_2\leq \| x \|_1\leq\sqrt{n}\| x \|_2$
        \item\label{ItemABSGooQODmLNii}
            $\| x \|_{\infty}\leq \| x \|_1\leq n \| x \|_{\infty}$
        \item\label{ItemABSGooQODmLNiii}
            $\| x \|_{\infty}\leq \| x \|_2\leq \sqrt{n}\| x \|_{\infty}$
    \end{enumerate}
\end{proposition}


\begin{proof}
    En mettant au carré la première inégalité nous voyons que nous devons vérifier l'inégalité
    \begin{equation}
        | x_1 |^2+\cdots+| x_n |^2\leq\big( | x_1 |+\cdots+| x_n | \big)^2
    \end{equation}
    qui est vraie parce que le membre de droite est égal au carré de chaque terme plus les double produits. La seconde inégalité provient de l'inégalité de Cauchy-Schwarz (théorème~\ref{ThoAYfEHG}) sur les vecteurs
    \begin{equation}
        \begin{aligned}[]
            v&=\begin{pmatrix}
                1/n    \\
                \vdots    \\
                1/n
            \end{pmatrix},
            &w&=\begin{pmatrix}
                | x_1 |    \\
                \vdots    \\
                | x_n |
            \end{pmatrix}.
        \end{aligned}
    \end{equation}
    Nous trouvons
    \begin{equation}
        \frac{1}{ n }\sum_i| x_i |\leq\sqrt{n\cdot\frac{1}{ n^2 }}\sqrt{\sum_i| x_i |^2},
    \end{equation}
    et par conséquent
    \begin{equation}
        \sum_i| x_i |\leq\sqrt{n}\| x \|_2.
    \end{equation}

    La première inégalité de~\ref{ItemABSGooQODmLNiii} se démontre en remarquant que si \( a\) et \( b\) sont positifs, \( a\leq\sqrt{a^2+b}\). En appliquant cela à \( a=\max_i| x_i |\), nous avons
    \begin{equation}
        \max_i| x_i |\leq\sqrt{ | x_1 |^2+\cdots+| x_n |^2  }
    \end{equation}
    parce que \( \max_i| x_i |\) est évidemment un des termes de la somme. Pour la seconde inégalité de~\ref{ItemABSGooQODmLNiii}, nous avons
    \begin{equation}
        \sqrt{\sum_k| x_k |^2}\leq\left( \sum_k\max_i| x_i |^2 \right)^{1/2}=\sqrt{n}\| x \|_{\infty}.
    \end{equation}
    Pour obtenir cette inégalité, nous avons remplacé tous les termes \( | x_k |\) par le maximum.
\end{proof}

En réalité, toutes les normes \( \| . \|_{L^p}\) et \( \| . \|_{\infty}\) sont équivalentes et, plus généralement, nous avons le résultat suivant, très étonnant à première vue, et en réalité assez difficile à prouver :
\begin{theorem}[\cite{TrenchRealAnalisys}]		\label{ThoNormesEquiv}
	Sur un espace vectoriel de dimension finie, toutes les normes (pas seulement les normes $L^p$ que nous avons définies sur $\eR^m$) sont équivalentes.
\end{theorem}
% TODO : la preuve est à la page 583 de Trench.

\begin{corollary}       \label{CORooBRDYooLmGJDE}
    Soit \( V\) un espace vectoriel de dimension finie et \( \| . \|_1\), \( \| . \|_2\) deux normes sur \( V\). Alors l'identité \( \id\colon V\to V\) est un isomorphisme d'espace topologique \( (V,\| . \|_1)\to (V,\| . \|_2)\).

    De plus les ouverts sont les mêmes : une partie de \( V\) est ouverte dans \( (V,\| . \|_1)\) si et seulement si elle est ouverte dans \( (V,\| . \|_2)\).
\end{corollary}

\begin{normaltext}
    L'exemple~\ref{EXooCAPYooMgOSyH} donne une norme sur \( \eR^2\) qui ne dérive pas d'un produit scalaire. Vu que toutes les normes sur \( \eR^2\) produisent la même topologie (c'est le corollaire~\ref{CORooBRDYooLmGJDE}), il y a parfaitement moyen pour deux espaces vectoriels topologiques d'être isomorphes alors que l'un a une norme dérivant d'un produit scalaire et l'autre non.
\end{normaltext}

Le théorème d'équivalence de norme sera utilisé pour montrer que l'ensemble des formes quadratiques non dégénérées de signature \( (p,q)\) est ouvert dans l'ensemble des formes quadratiques, proposition~\ref{PropNPbnsMd}. Plus généralement il est utilisé à chaque fois que l'on fait de la topologie sur les espaces de matrices en identifiant \( \eM(n,\eR)\) à \( \eR^{n^2}\), pour se rassurer en se disant que ce qu'on fait ne dépend pas de la norme choisie.

%---------------------------------------------------------------------------------------------------------------------------
\subsection{Contre-exemple en dimension infinie}
%---------------------------------------------------------------------------------------------------------------------------
\label{SubSecPOlynomesCE}

Lorsque nous considérons des espaces vectoriels de dimension infinie, les choses ne sons plus aussi simples. Nous voyons ici sur l'exemple de l'espace des polynômes que le théorème~\ref{ThoNormesEquiv} n'est plus valable si on enlève l'hypothèse de dimension finie.

On considère l'ensemble des fonctions polynomiales à coefficients réels sur  l'intervalle $[0,1]$.
\begin{equation}
\mathcal{P}_\eR([0,1])=\{p:[0,1]\to \eR\,|\, p : x\mapsto a_0+a_1 x +a_2 x^2 + \ldots, \, a_i\in\eR,\,\forall i\in \eN\}.
\end{equation}
Cet ensemble, muni des opérations usuelles de somme entre polynômes et multiplications par les scalaires, est un espace vectoriel.

Sur $\mathcal{P}(\eR)$ on définit les normes suivantes
\begin{equation}
\begin{aligned}
&\|p\|_\infty=\sup_{x\in[0,1]}\{p(x)\},\\
&\|p\|_1 =\int_0^1|p(x)|\, dx,\\
&\|p\|_2 =\left(\int_0^1|p(x)|^2\, dx\right)^{1/2}.\\
\end{aligned}
\end{equation}
Les inégalités suivantes sont  immédiates
\begin{equation}
\begin{aligned}
&\|p\|_1 =\int_0^1|p(x)|\, dx\leq \|p\|_\infty,\\
&\|p\|_2 =\left(\int_0^1|p(x)|^2\, dx\right)^{1/2}\leq \|p\|_\infty,\\
\end{aligned}
\end{equation}
mais la norme $\|\cdot\|_\infty$ n'est  équivalente ni à $\|\cdot\|_1$, ni à $\|\cdot\|_2$. Soit $p_k(x)= x^k$. Alors
\begin{equation}
\begin{aligned}
&\|p_k\|_\infty=1,\\
&\|p_k\|_1 =\int_0^1x^k\, dx=  \frac{1}{k+1},\\
&\|p_k\|_2 =\left(\int_0^1x^{2k}\, dx\right)^{1/2}=\sqrt{\frac{1}{2k+1}}.
\end{aligned}
\end{equation}
Pour $k\to \infty$ les normes $\|p_k\|_1$, $\|p_k\|_2$ tendent vers zéro, alors que la norme $\|p_k\|_\infty$ est constante, donc les normes ne sont pas équivalentes parce que il n'existe pas un nombre positif $m$ tel que
\begin{equation}
\begin{aligned}
& m \|p_k\|_\infty\leq \|p_k\|_1 ,\\
& m \|p_k\|_\infty\leq \|p_k\|_2 ,\\
\end{aligned}
\end{equation}
uniformément pour tout $k$ dans $\eN$.

%+++++++++++++++++++++++++++++++++++++++++++++++++++++++++++++++++++++++++++++++++++++++++++++++++++++++++++++++++++++++++++
\section{Norme opérateur}
%+++++++++++++++++++++++++++++++++++++++++++++++++++++++++++++++++++++++++++++++++++++++++++++++++++++++++++++++++++++++++++

La proposition suivante donne une norme (au sens de la définition~\ref{DefNorme}) sur $\aL(V,W)$ dès que \( V\) et \( W\) sont des espaces vectoriels normés.
\begin{propositionDef}[Norme opérateur\cite{ooTZRDooWmjBJi}, thème~\ref{THEMEooHSLLooBQpFAr}]          \label{DefNFYUooBZCPTr}
    Soit une application linéaire \( T\colon V\to W\), et le nombre
	\begin{equation}
        \|T\|_{\aL}=\sup_{\substack{x\in V\\x\neq 0}}\frac{\|T(x)\|_{W}}{\|x\|_{V}}.
	\end{equation}
    \begin{enumerate}
        \item
            Si \( V\) est de dimension finie, alors \( \| T \|_{\aL}<\infty\).
        \item
            L'application \( T\mapsto\| T \|_{\aL}\) est une norme sur l'espace vectoriel des applications linéaires \( V\to W\).
        \item       \label{ITEMooUQPRooYQGZzu}
            Nous avons la formule
            \begin{equation}    \label{EqFZPooIoecGH}
                \| T \|_{\aL}=\sup_{x\in V}\frac{\|T(x)\|_{W}}{\|x\|_{V}} =\sup_{\|x\|_{V}=1}\|T(x)\|_{W}
            \end{equation}
    \end{enumerate}
    Le nombre \( \| T \|_{\aL}\) est la \defe{norme opérateur}{norme!d'application linéaire} de $T$. Nous disons que cette norme est \defe{subordonnée}{subordonnée!norme} aux normes choisies sur \( V\) et \( W\).
\end{propositionDef}
\index{norme!d'une application linéaire}

\begin{proof}
    Si \( V\) est de dimension finie alors l'ensemble $\{ \| x \|= 1 \}$ est compact par le théorème de Borel-Lebesgue~\ref{ThoXTEooxFmdI}. Alors la fonction
    \begin{equation}
        x\mapsto \frac{ \| T(x) \| }{ \| x \| }
    \end{equation}
    est une fonction continue sur un compact. Le corollaire~\ref{CorFnContinueCompactBorne} nous dit alors qu'elle est bornée. Le supremum est donc un nombre réel fini.

    Nous vérifions que l'application $\| . \|$ de $\aL(V,W)$ dans $\eR$ ainsi définie est effectivement une norme.
    \begin{enumerate}
        \item
            $\|T\|_{\aL}=0$ signifie que $\|T(x)\|=0$ pour tout $x$ dans $V$. Comme  $\|\cdot\|_W$ est une norme nous concluons que $T(x)=0_{n}$ pour tout $x$ dans $V$, donc $T$ est l'application nulle.
    \item
        Pour tout $a$ dans $\eR$ et tout  $T$ dans $\aL(V,W)$ nous avons
        \begin{equation}
            \|aT\|_{\mathcal{L}}=\sup_{\|x\|_{V}\leq 1}\|aT(x)\|_{W}=|a|\sup_{\|x\|_{V}\leq 1}\|T(x)\|_{W}=|a|\|T\|_{\mathcal{L}}.
        \end{equation}
    \item
        Pour tous $T_1$ et $T_2$ dans $\aL(V,W)$ nous avons
      \begin{equation}\nonumber
        \begin{aligned}
           \|T_1+ T_2\|_{\mathcal{L}}&=\sup_{\|x\|\leq 1}\|T_1(x)+T_2(x)\|\leq\\
     &\leq\sup_{\|x\|\leq 1}\|T_1(x)\| +\sup_{\|x\|\leq 1}\|T_2(x)\|\\
     &=\|T_1\|\|T_2\|.
        \end{aligned}
      \end{equation}
    \end{enumerate}


    Enfin nous prouvons la formule alternative \eqref{EqFZPooIoecGH}. Nous allons montrer que les ensembles sur lesquels ont prend le supremum sont en réalité les mêmes :
    \begin{equation}
        \underbrace{\left\{ \frac{ \| Ax \| }{ \| x \| }\right\}_{x\neq 0}}_{A}=\underbrace{\left\{ \| Ax \|\tq \| x \|=1 \right\}}_{B}.
    \end{equation}
    Attention : ce sont des sous-ensembles de réels; pas de sous-ensembles de \( \eM(\eR)\) ou des sous-ensembles de \( \eR^n\).

    Pour la première inclusion, prenons un élément de \( A\), et prouvons qu'il est dans \( B\). C'est à dire que nous prenons \( x\in V\) et nous considérons le nombre \( \| Ax \|/\| x \|\). Le vecteur \( y=x/\| x \|\) est un vecteur de norme $1$, donc la norme de \( Ay\) est un élément de \( B\), mais
    \begin{equation}
        \| Ay \|=\frac{ \| Ax \| }{ \| x \| }.
    \end{equation}
    Nous avons donc \( A\subset B\).

    L'inclusion \( B\subset A\) est immédiate.
\end{proof}

En d'autres termes, il y a autant de normes opérateur sur \( \aL(E,F)\) qu'il y a de paires de choix de normes sur \( E\) et \( F\). En particulier, cela donne lieu à toutes les normes \( \| A \|_p\) qui correspondent aux normes \( \| . \|_p\) sur \( \eR^n\).

\begin{example}     \label{EXooXPXAooYyBwMX}
    Voyons la norme opérateur subordonnée à la norme \( \| x \|_{\infty}=\max_i| x_i |\) sur \( \eC^n\). Par définition (et surtout par la propriété~\ref{DefNFYUooBZCPTr}\ref{ITEMooUQPRooYQGZzu}),
    \begin{equation}
        \| A \|_{\infty}=\sup_{\| x \|_{\infty}=1}=\| Ax \|_{\infty}.
    \end{equation}
    Vu que \( (Ax)_i=\sum_kA_{ik}x_k\), lorsque \( \| x \|_{\infty}\leq 1\) nous avons \( | (Ax)_i |\leq \sum_k| A_{ik} |\). Donc nous avons toujours
    \begin{equation}        \label{EQooPLCIooVghasD}
        \| A \|_{\infty}\leq \max_i\sum_{k}| A_{ik} |.
    \end{equation}
\end{example}

\begin{definition}
    La \defe{topologie forte}{topologie!forte} sur l'espace des opérateurs est la topologie de la norme opérateur.
\end{definition}
Lorsque nous considérons un espace vectoriel d'applications linéaires, nous considérons toujours\footnote{Sauf lorsque les événements nous forceront à trahir.} dessus la topologie liée à cette norme.

Il existe aussi la \defe{topologie faible}{topologie!faible} donnée par la notion de convergence\quext{Est-ce qu'on peut décrire cette topologie à partir de ses ouverts ? Facilement ?} \( A_i\to A\) si et seulement si \( A_ix\to Ax\) pour tout \( x\in E\).
    %TODO : il faut mettre au clair quelle est vraiment la topologie faible à partir des ouverts.

\begin{probleme}
    Je crois, mais demande confirmation, que la topologie faible est celle des semi-normes \( \{ p_v \}_{v\in E}\) données par \( p_v(A)=\| A \|\). En effet la notion de convergence associée par la proposition~\ref{PropQPzGKVk} est \( A_i\to A\) si et seulement si \( p_v(A_i-A)\to 0\). Cette condition signifie \( \| A_i(v)-A(v) \|\to 0\), c'est à dire \( A_i(v)\to A(v)\).

    Si le lecteur veut parler de cela au jury d'un concours, il est évident qu'il devra être capable d'ajouter des petits symboles au-dessus de toutes les flèches «\( \to\)» du paragraphe précédent pour indiquer pour quelles topologies sont les convergences dont on parle.
\end{probleme}

\begin{remark}
    Il faut noter que la topologie faible n'est pas une topologie métrique. Cela même si la condition \( A_ix\to Ax\), elle, est métrique vu qu'elle est écrite dans \( E\).

    Dans le cas où \( E\) est de dimension infinie, la topologie faible est réellement différente de la topologie forte. Nous verrons à la sous-section~\ref{subsecaeSywF} que dans le cas des projections sur un espaces de Hilbert, l'égalité
    \begin{equation}
        \sum_{i=1}^{\infty}\pr_{u_i}=\id
    \end{equation}
    est vraie pour la topologie faible, mais pas pour la topologie forte.
\end{remark}

%---------------------------------------------------------------------------------------------------------------------------
\subsection{Norme d'algèbre}
%---------------------------------------------------------------------------------------------------------------------------

\begin{definition}[Norme d'algèbre\cite{ooTZRDooWmjBJi}]  \label{DefJWRWQue}
    Si \( A\) est une algèbre\footnote{Définition~\ref{DefAEbnJqI}.}, une \defe{norme d'algèbre}{norme!d'algèbre} sur \( A\) est une norme telle que pour toute \( x,y\in A\),
    \begin{equation}
        \| xy \|\leq \| x \|\| y \|.
    \end{equation}
\end{definition}
Un des intérêts d'utiliser une norme d'algèbre est que l'on a l'inégalité \( \| x^k \|\leq \| x \|^k \). Cela sera particulièrement utile lors de l'étude des séries entières, voir par exemple~\ref{secEVnZXgf}.

\begin{definition}[\cite{ooYLHAooCzQvoa}]      \label{DEFooEAUKooSsjqaL}
    Le \defe{rayon spectral}{rayon!spectral} d'une matrice carrée $A$, noté $\rho(A)$, est défini de la manière suivante :
    \begin{equation}    \label{EQooNVNOooNjJhSS}
        \rho(A)=\max_i|\lambda_i|
    \end{equation}
    où les $\lambda_i$ sont les valeurs propres de $A$.
\end{definition}

\begin{normaltext}
    Quelques remarques sur la définition du rayon spectral.
    \begin{itemize}
        \item
             Même si \( A\) est une matrice réelle, les valeurs propres sont dans \( \eC\). Donc dans \eqref{EQooNVNOooNjJhSS}, \( | \lambda_i |\) est le module dans \( \eC\) de \( \lambda_i\).
        \item
            Vu que les valeurs propres de \( A\) sont les racines de son polynôme caractéristique (théorème~\ref{ThoWDGooQUGSTL}), il y en a un nombre fini et le maximum est bien défini.
        \item
            La définition s'applique uniquement pour les espaces de dimension finie.
    \end{itemize}
\end{normaltext}

\begin{lemma}       \label{LEMooIBLEooLJczmu}
    Soient des espaces vectoriels normés \( E\) et \( F\), sur les corps \( \eR\) ou \( \eC\). Pour tout \( A\in \aL(E,F)\), et pour tout \( u\in E\) nous avons la majoration
    \begin{equation}
        \| Au \|\leq \| A \|\| u \|
    \end{equation}
    où la norme sur \( A\) est la norme opérateur subordonnée à la norme sur \( u\).
\end{lemma}

\begin{proof}
    Si \( u\in E\) alors, étant donné que le supremum d'un ensemble est plus grand ou égal à chacun de éléments qui le compose,
    \begin{equation}
        \| A \|=\sup_{x\in E}\frac{ \| Ax \| }{ \| x \| }\geq \frac{ \| Au \| }{ \| u \| },
    \end{equation}
    donc le résultat annoncé : \( \| Au \|\leq \| A \|\| u \|\).
\end{proof}

\begin{proposition}[\cite{ooYLHAooCzQvoa}]      \label{PROPooKLFKooSVnDzr}
    Soit une matrice \( A\in \eM(n,\eC)\) de rayon spectral \( \rho(A)\). Soit une norme \( \| . \|\) sur \( \eC^n\) et la norme opérateur correspondante. Alors
    \begin{equation}
        \rho(A)\leq \| A^k \|^{1/k}
    \end{equation}
    pour tout \( k\in \eN\).
\end{proposition}

\begin{proof}
    Soit \( v\in \eC^n\) et \( \lambda\in \eC\) un couple vecteur-valeur propre. Nous avons \( \| Av \|=| \lambda |\| v \|\) et aussi
    \begin{equation}
        | \lambda |^k\| v \|=\| \lambda^kv \|=\| A^kv \|\leq \| A^k \|\| v \|.
    \end{equation}
    La dernière inégalité est due au fait que nous avons choisi sur \( \eM(n,\eC)\) la norme subordonnée à celle choisie sur \( \eC^n\), via le lemme~\ref{LEMooIBLEooLJczmu}. Nous simplifions par \( \| v \|\) et obtenons \( | \lambda |\leq \| A^k \|^{1/k}\). Étant donné que \( \rho(A)\) est la maximum de tous les \( \lambda\) possibles, la majoration passe au maximum :
    \begin{equation}
        \rho(A)\leq \| A^k \|^{1/k}.
    \end{equation}
\end{proof}

\begin{lemma}[La norme opérateur est une norme d'algèbre\cite{MonCerveau}]   \label{LEMooFITMooBBBWGI}
    Soient des espaces vectoriels normés \( E\), \( F\) et \( G\). Soient des opérateurs linéaires bornés \( B\colon E\to F\), \( A\colon F\to G\). Alors
    \begin{equation}
        \| AB \|\leq \| A \|\| B \|.
    \end{equation}
\end{lemma}

\begin{proof}

    Nous avons les (in)égalités suivantes :
    \begin{subequations}
        \begin{align}
            \| AB \|&=\sup_{x\in E}\frac{ \| ABx \|_G }{ \| x \|_E }\\
            &=\sup_{\substack{x\in E\\Bx\neq 0}}\frac{ \| ABx \| }{ \| x \| }\frac{ \| Bx \|_F }{ \| Bx \|_F }\\
            &=\sup_{\substack{x\in E\\Bx\neq 0}}\frac{ \| ABx \| }{ \| Bx \| }\frac{ \| Bx \| }{ \| x \| }\\
            &\leq\underbrace{\sup_{\substack{x\in E\\Bx\neq 0}}\frac{ \| ABx \| }{ \| Bx \| }}_{\leq\| A \|}\underbrace{\sup_{\substack{y\in E\\By\neq 0}}\frac{ \| Bx \| }{ \| y \| }}_{=\| B \|}\\
            &\leq \| A \|\| B \|.
        \end{align}
    \end{subequations}
    La dernière inégalité provient que dans \( \sup_{\substack{x\in E\\Bx\neq 0}}\| ABx \|/\| x \|\), le supremum est pris sur un ensemble plus petit que celui sur lequel porte la définition de la norme de \( A\) : seulement l'image de \( B\) au lieu de tout l'espace de départ de \( A\).
\end{proof}

La chose impressionnante dans la proposition suivante est que \( \rho(A)\) est définit indépendamment du choix de la norme sur \( \eM(n,\eK)\) ou sur \( \eK\). Lorsque nous écrivons \( \| A \|\), nous disons implicitement qu'une norme a été choisie sur \( \eK\) et que nous avons pris la norme subordonnée sur \( \eM(n,\eK)\).
\begin{proposition}[\cite{ooETMNooSrtWet}]      \label{PROPooWZJBooTPLSZp}
    Soit \( A\) une matrice de \( \eM(n,\eR)\) ou \( \eM(n,\eC)\). Alors
    \begin{equation}
        \rho(A)\leq \| A \|.
    \end{equation}
\end{proposition}

\begin{proof}
    Nous devons séparer les cas suivant que le corps de base soit \( \eR\) ou \( \eC\).

    \begin{subproof}
        \item[Pour \( A\in \eM(n,\eC)\)]
            Soit \( \lambda\) une valeur propre de \( A\) telle que \( | \lambda |\) soit la plus grande. Nous avons donc \( \rho(A)=| \lambda |\). Soit un vecteur propre \( u\in \eC^n\) pour la valeur propre \( \lambda\). En prenant la norme sur l'égalité \( Au=\lambda u\), et en utilisant le lemme~\ref{LEMooIBLEooLJczmu},
            \begin{equation}
                | \lambda |\| u \|=\| Au \|\leq \| A \|\| u \|.
            \end{equation}
            Donc \( | \lambda |\leq \| A \|\) et \( \rho(A)\leq\| A \|\).

        \item[Pour \( A\in \eM(n,\eR)\)]

            L'endroit qui coince dans le raisonnement fait pour \( \eM(n,\eC)\) est que certes \( A\in \eM(n,\eR)\) possède une plus grande valeur propre en module et qu'un vecteur propre lui est associé. Mais ce vecteur propre est a priori dans \( \eC^n\), et non dans \( \eR^n\). Nous pouvons donc écrire \( Au=\lambda u\), mais pas \( \| Au \|=| \lambda |\| u \|\) parce que nous ne savons pas quelle norme prendre sur \( \eC^n\).

            Il n'est pas certain que nous ayons une norme sur \( \eC^n\) qui se réduit sur \( \eR^n\) à celle choisie implicitement dans l'énoncé. Nous allons donc ruser un peu.

            Soit une norme \( N\) sur \( \eC^n\)\footnote{Il y en a plein, par exemple celle du produit scalaire \( \langle x, y\rangle =\sum_kx_k\bar y_k\).}. Nous nommons également \( N\) la norme subordonnée sur \( \eM(n,\eC)\) et la norme restreinte sur \( \eM(n,\eR)\). Vu que \( N\) est une norme sur \( \eM(n,\eR)\) et que ce dernier est de dimension finie, le théorème~\ref{ThoNormesEquiv} nous indique que \( N\) est équivalente à \( \| . \|\). Il existe donc \( C>0\) tel que
            \begin{equation}        \label{EQooBNWMooNgnMxC}
                 N(B)\leq C\| B \|
            \end{equation}
            pour tout \( B\in \eM(n,\eR)\). Nous avons maintenant
            \begin{equation}
                \rho(A)^m\leq N(A^m)\leq C\| A^m \|\leq C\| A \|^m.
            \end{equation}
            Justifications
            \begin{itemize}
                \item Par la proposition~\ref{PROPooKLFKooSVnDzr}.
                \item Parce que \( A^m\in \eM(n,\eR)\) et la relation \eqref{EQooBNWMooNgnMxC}.
                \item Par itération du lemme~\ref{LEMooFITMooBBBWGI}.
            \end{itemize}

            Nous avons donc \( \rho(A)\leq C^{1/m}\| A \|\) pour tout \( m\in\eN\). En prenant \( m\to \infty\) et en tenant compte de \( C^{1/m}\to 1\) nous trouvons \( \rho(A)\leq \| A \|\).
    \end{subproof}
\end{proof}

\begin{lemma}[\cite{ooETMNooSrtWet}]        \label{LEMooGBLJooCPvxNl}
    Soit \( A\in \eM(n,\eK)\) avec \( \eK=\eR\) ou \( \eC\). Soit \( \epsilon>0\). Il existe une norme algébrique sur \( \eM(n,\eK)\) telle que
    \begin{equation}
        N(A)\leq \rho(A)+\epsilon.
    \end{equation}
\end{lemma}

\begin{proof}
    Soit par le lemme~\ref{LemSchurComplHAftTq} une matrice inversible \( U\) telle que \( T=UAU^{-1}\) soit triangulaire supérieure, avec les valeurs propres sur la diagonale. Notons que même si \( A\in \eM(n,\eR)\), les matrices \( U\) et \( T\) sont a priori complexes.

    Soit \( s\in \eR\) ainsi que les matrices
    \begin{equation}
        D_s=\diag(1,s^{-1},s^{-2},\ldots, s^{1-n})
    \end{equation}
    et \( T_s=D_sTD_s^{-1}\). Nous fixerons un choix de \( s\) plus tard.

    La norme que nous considérons est :
    \begin{equation}
        N(B)=\| (D_sU)B(D_sU)^{-1} \|_{\infty}
    \end{equation}
    où \( \| . \|_{\infty}\) est la norme sur \( \eM(,n\eK)\) subordonnée à la norme \( \| . \|_{\infty}\) sur \( \eK^n\) dont nous avons déjà parlé dans l'exemple~\ref{EXooXPXAooYyBwMX}. Cela est bien une norme parce que
    \begin{itemize}
        \item Nous avons \( \| B \|_{\infty}=0\) si et seulement si \( B=0\), et vu que \( (D_sU)\) est inversible nous avons \( (D_sU)B(D_sU)^{-1}=0\) si et seulement si \( B=0\).
        \item \( N(\lambda B)=| \lambda |N(B)\).
        \item Pour l'inégalité triangulaire :
            \begin{subequations}
                \begin{align}
             N(B+C)&=\| (D_sU)B(D_sU)^{-1}+(D_sU)C(D_sU)^{-1} \|_{\infty}\\
             &\leq  \| (D_sU)B(D_sU)^{-1}\|_{\infty} +\| (D_sU)C(D_sU)^{-1} \|_{\infty} \\
             &=N(B)+N(C).
                \end{align}
            \end{subequations}
    \end{itemize}

    En ce qui concerne la matrice \( A\) elle-même, nous avons
    \begin{equation}
        N(A)=\| (D_sU)A(D_sU)^{-1} \|_{\infty}=\| T_s \|_{\infty}.
    \end{equation}
    C'est le moment de se demander comment se présente la matrice \( T_s\). En tenant compte du fait que \( (D_s)_{ik}=\delta_{ik}s^{1-i}\) nous avons
    \begin{equation}
        (T_s)_{ij}=\sum_{kl}(D_s)_{ik}T_{kl}(D^{-1}_s)_{lj}=T_{ij}s^{j-i}.
    \end{equation}
    La matrice \( T\) est encore triangulaire supérieure avec les valeurs propres de \( A\) sur la diagonale. Les éléments au-dessus de la diagonale sont tous multipliés par au moins \( s\). Il est donc possible de choisir \( s\) suffisamment petit pour avoir\quext{Il me semble qu'il manque un module dans \cite{ooETMNooSrtWet}.}
    \begin{equation}        \label{EQooSIEIooTWAXQD}
        \sum_{j=i+1}^n| (T_s)_{ij} |<\epsilon
    \end{equation}
    Avec ce choix, la formule~\ref{EQooPLCIooVghasD} donne
    \begin{equation}
        N(T_s)\leq\max_i\sum_k| (T_s)_{ik} |\leq \epsilon+\rho(A).
    \end{equation}
    En effet le \( \epsilon\) vient de la somme sur toute la ligne sauf la diagonale (c'est à dire la partie \( k\neq i\)) et du choix \eqref{EQooSIEIooTWAXQD} pour \( s\). Le \( \rho(A)\) provient du dernier terme de la somme (le terme sur la diagonale) qui est une valeur propre de \( A\), donc majorable par \( \rho(A)\).

    Nous devons encore prouver que \( N\) est une norme algébrique. Pour cela nous allons montrer qu'elle est subordonnée à la norme
    \begin{equation}
        \begin{aligned}
            n\colon \eK^n&\to \eR^+ \\
            v&\mapsto \| (UD_s)v \|_{\infty}.
        \end{aligned}
    \end{equation}
    Cela sera suffisant pour avoir une norme algébrique par le lemme~\ref{LEMooFITMooBBBWGI}. La norme \( n\) sur \( \eK^n\) produit la norme suivante sur \( \eM(n,\eK)\) :
    \begin{equation}
        n(B)=\sup_{v\neq 0}\frac{ n(B) }{ n(v) }=\sup_{v\neq 0}\frac{ \| (UD_s)Bv \|_{\infty} }{ \| UD_sv \|_{\infty} }.
    \end{equation}
    Vu que \( UD_s\) est inversible nous pouvons effectuer le changement de variables \( v\mapsto (UD_s)^{-1} v\) pour écrire
    \begin{equation}
        n(B)=\sup_{v\neq 0}  \frac{  \| (UD_s)B(UD_s)^{-1}v \|_{\infty} }{ \| (UD_s)(UD_s)^{-1}v \|_{\infty} }=\sup_{v\neq 0}\frac{  \| (UD_s)B(UD_s)^{-1}v \|_{\infty} }{ \| v \|_{\infty} }=\| (UD_s)B(UD_s)^{-1} \|_{\infty}=N(B).
    \end{equation}
\end{proof}

\begin{proposition}     \label{PROPooYPLGooWKLbPA}
    Si \( A\in \eM(n,\eR)\) alors \( \rho(A)^m=\rho(A^m)\) pour tout \( m\in \eN\).
\end{proposition}

\begin{proof}
    La matrice \( A\) peut être vue dans \( \eM(n,\eC)\) et nous pouvons lui appliquer le corollaire~\ref{CORooTPDHooXazTuZ} :
    \begin{equation}        \label{EQooJJIYooDBacjn}
        \Spec(A^k)=\{ \lambda^k\tq \lambda\in\Spec(A) \}.
    \end{equation}
    À noter qu'il n'y a pas de magie : le spectre de la matrice réelle \( A\) est déjà défini en voyant \( A\) comme matrice complexe. Le spectre dont il est question dans \eqref{EQooJJIYooDBacjn} est bien celui dont on parle dans la définition du rayon spectral.

    Nous avons ensuite :
    \begin{subequations}
        \begin{align}
            \rho(A^k)&=\max\{ | \lambda |\tq \lambda\in\Spec(A^k) \}\\
            &=\max\{ | \lambda^k |\tq \lambda\in\Spec(A) \}\\
            &=\max\{ | \lambda |^k\tq\lambda\in\Spec(A) \}\\
            &=\rho(A)^k.
        \end{align}
    \end{subequations}
\end{proof}

\begin{proposition}[Bornée si et seulement si continue\cite{GKPYTMb}]       \label{PROPooQZYVooYJVlBd}
    Soient \( E\) et \( F\) des espaces vectoriels normés. Une application linéaire \( E\to F\) est bornée si et seulement si elle est continue.
\end{proposition}

\begin{proof}
    Nous commençons par supposer que \( A\) est bornée. Par le lemme~\ref{LEMooFITMooBBBWGI}, pour tout \( x,y\in E\), nous avons
    \begin{equation}
        \| A(x)-A(y) \|=\| A(x-y) \|\leq \| A \|\| x-y \|.
    \end{equation}
    En particulier si \( x_n\stackrel{E}{\longrightarrow}x\) alors
    \begin{equation}
        0\leq \| A(x_n)-A(x) \|\leq \| A \|\| x_n-x \|\to 0
    \end{equation}
    et \( A\) est continue en vertu de la caractérisation séquentielle de la continuité, proposition~\ref{PropFnContParSuite}.

    Supposons maintenant que \( \| A \|\) ne soit pas borné, c'est à dire que l'ensemble \( \{ \| A(x) \|\tq \| x \|=1 \}\) ne soit pas borné. Alors pour tout \( k\geq 1\) il existe \( x_k\in B(0,1)\) tel que \( \| A(x_k) \|>k\). La suite \( x_k/k\) tend vers zéro parce que \( \| x_k \|=1\), mais \( \| A(x_k) \|\geq 1\) pour tout \( k\). Cela montre que \( A\) n'est pas continue.
\end{proof}

\begin{definition}[\cite{ooAISYooXtUafT}]      \label{DEFooTLQUooJvknvi}
    Soient \( E\) et \( F\) deux espaces vectoriels normés.
    \begin{itemize}
        \item
            L'ensemble des applications linéaires \( E\to F\) est noté \( \aL(E,F)\).
        \item Un \defe{morphisme}{morphisme!espace vectoriel normé} est une application linéaire \( E\to F\) continue pour la topologie de la norme opérateur. Nous avons vu dans la proposition~\ref{PROPooQZYVooYJVlBd} que la continuité était équivalente à être bornée. L'ensemble des morphismes est noté \( \cL(E,F)\)\nomenclature[B]{\( \cL(E,F)\)}{applications linéaires bornées (continues)}.
        \item
            Un \defe{isomorphisme}{isomorphisme!espace vectoriel normé} est un morphisme continu inversible dont l'inverse est continu. Nous notons \( \GL(E,F)\) l'ensemble des isomorphismes entre \( E\) et \( F\).
    \end{itemize}
\end{definition}

Le point important de la définition~\ref{DEFooTLQUooJvknvi} est la continuité. En dimension infine, la continuité n'est par exemple pas équivalente à l'inversibilité (penser à \( e_k\mapsto ke_k\)).

%---------------------------------------------------------------------------------------------------------------------------
\subsection{Application linéaire continue et bornée}
%---------------------------------------------------------------------------------------------------------------------------

Nous avons vu dans la proposition~\ref{PROPooQZYVooYJVlBd} que pour une application linéaire, être bornée est équivalent à être continue. Nous allons maintenant voir un certain nombre d'exemples illustrant ce fait.

\begin{example}[Une application linéaire non continue]  \label{ExHKsIelG}
    Soit \( V\) l'espace vectoriel normé des suites \emph{finies} de réels muni de la norme usuelle $\| c \|=\sqrt{\sum_{i=0}^{\infty}| c_i |^2}$ où la somme est finie. Nous nommons \( \{ e_k \}_{k\in \eN}\) la base usuelle de cet espace, et nous considérons l'opérateur \( f\colon V\to V\) donnée par \( f(e_k)=ke_k\). C'est évidemment linéaire, mais ce n'est pas continu en zéro. En effet la suite \( u_k=e_k/k\) converge vers \( 0\) alors que \( f(u_k)=e_k\) ne converge pas.
\end{example}

Cet exemple aurait pu également être donnée dans un espace de Hilbert, mais il aurait fallu parler de domaine.
%TODO : le faire, et regarder si Hilbet n'est pas la complétion de cet espace. Référencer à l'endroit qui définit l'espace vectoriel librement engendré. Ici ce serait par N.

%TODO : dire qu'une application bilinéaire sur RxR n'est pas une application linéaire sur R^2

\begin{example}[Une autre application linéaire non continue\cite{GTkeGni}]      \label{EXooDMVJooAJywMU}
    En dimension infinie, une application linéaire n'est pas toujours continue. Soit \( E\) l'espace des polynômes à coefficients réels sur \( \mathopen[ 0 , 1 \mathclose]\) muni de la norme uniforme. L'application de dérivation \( \varphi\colon E\to E\), \( \varphi(P)=P'\) n'est pas continue.

    Pour la voir nous considérons la suite \( P_n=\frac{1}{ n }X^n\). D'une part nous avons \( P_n\to 0\) dans \( E\) parce que \( P_n(x)=\frac{ x^n }{ n }\) avec \( x\in \mathopen[ 0 , 1 \mathclose]\). Mais en même temps nous avons \( \varphi(P_n)=X^{n-1}\) et donc \( \| \varphi(P_n) \|=1\).

    Nous n'avons donc pas \( \lim_{n\to \infty} \varphi(P_n)=\varphi(\lim_{n\to \infty} P_n)\) et l'application \( \varphi\) n'est pas continue en \( 0\). Elle n'est donc continue nulle part par linéarité.

    Nous avons utilisé le critère séquentiel de la continuité, voir la définition~\ref{DefENioICV} et la proposition~\ref{PropFnContParSuite}.
\end{example}

\begin{remark}  \label{RemOAXNooSMTDuN}
Cette proposition permet de retrouver l'exemple~\ref{ExHKsIelG} plus simplement. Si \( \{ e_k \}_{k\in \eN}\) est une base d'un espace vectoriel normé formée de vecteurs de norme \( 1\), alors l'opérateur linéaire donné par \( u(e_k)=ke_k\) n'est pas borné et donc pas continu.
\end{remark}

C'est également ce résultat qui montre que le produit scalaire est continu sur un espace de Hilbert par exemple.

\begin{definition}      \label{DEFooKSDFooGIBtrG}
    Soit un espace vectoriel \( E\) sur le corps \( \eK\). Son \defe{dual topologique}{dual topologique}, noté \( E'\), est l'ensemble des formes linéaires continues de \( E\) vers \( \eK\).
\end{definition}

\begin{lemma}   \label{LemWWXVSae}
Soit \( F\) un espace de Banach et deux suites \( A_k\to A\) et \( B_k\to B\) dans \( \aL(F,F)\). Alors \( A_k\circ B_k\to A\circ B\) dans \( \aL(F,F)\), c'est à dire
\begin{equation}
    \lim_{n\to \infty} (A_kB_k)=\left( \lim_{n\to \infty} A_k \right)\left( \lim_{n\to \infty} B_k \right).
\end{equation}
\end{lemma}

\begin{proof}
    Il suffit d'écrire
    \begin{equation}
        \| A_kB_k-AB \|\leq \| A_kB_k-A_kB \|+\| A_kB-AB \|.
    \end{equation}
    Le premier terme tend vers zéro pour \( k\to\infty\) parce que
    \begin{subequations}
        \begin{align}
            \| A_kB_k-A_kB \|&=\| A_k(B_k-B) \|\\
            &\leq \| A_k \|\| B_k-B \|\to \| A \|\cdot 0\\
            &=0
        \end{align}
    \end{subequations}
    où nous avons utilisé la propriété fondamentale de la norme opérateur : la proposition~\ref{PROPooQZYVooYJVlBd}. Le second terme tend également vers zéro pour la même raison.
\end{proof}

\begin{proposition}[Distributivité de la somme infinie] \label{PropQXqEPuG}
    Soient \( E\) un espace normé, une suite \( (u_k)\) dans \( \GL(E)\) ainsi que \( a\in\GL(E)\). Pourvu que la série \( \sum_{n=0}^{\infty}u_k\) converge nous avons
    \begin{equation}
        \left( \sum_{k=0}^{\infty}u_k \right)a=\sum_{k=0}^{\infty}(u_ka).
    \end{equation}
\end{proposition}

\begin{proof}
    Par définition de la somme infinie,
    \begin{equation}
        \spadesuit=\left( \sum_{k=0}^{\infty}u_k \right)a=\left( \lim_{n\to \infty} \sum_{k=0}^nu_k \right)a.
    \end{equation}
    Le lemme~\ref{LemWWXVSae} appliqué à \( n\mapsto\sum_{k=0}^nu_k\) et à la suite constante \( a\) nous donne
    \begin{equation}    \label{EqOAoopjz}
        \spadesuit=\lim_{n\to \infty} \left( \sum_{k=0}u_ka \right),
    \end{equation}
    ce que nous voulions par distributivité de la somme finie : dans \eqref{EqOAoopjz}, le \( a\) est dans ou hors de la somme, au choix. L'important est qu'il soit dans la limite.
\end{proof}

%---------------------------------------------------------------------------------------------------------------------------
\subsection{Normes de matrices et d'applications linéaires}
%---------------------------------------------------------------------------------------------------------------------------
\label{subsecNomrApplLin}

\begin{theorem}[Norme matricielle et rayon spectral\cite{ooBCKVooVunKyT}]       \label{THOooNDQSooOUWQrK}
    La norme $2$ d'une matrice est liée au rayon spectral de la façon suivante :
    \begin{equation}
        \|A\|_2=\sqrt{\rho(A{^t}A)}
    \end{equation}
    ou plus généralement par \( \| A \|_2=\sqrt{\rho(A^*A)}\).
\end{theorem}

\begin{lemma}       \label{LEMooNESTooVvUEOv}
    Soit une matrice \( A\in \eM(n,\eR)\) qui est symétrique, strictement définie positive. Soient \( \lambda_{min}\) et \( \lambda_{max}\) les plus petites et plus grandes valeurs propres. Alors
    \begin{subequations}
        \begin{align}
            \| A \|_2=\lambda_{max}&&\text{ et }&&\|A^{-1}  \|_2=\frac{1}{ \lambda_{min} }.
        \end{align}
    \end{subequations}
\end{lemma}

\begin{proof}
    Soient les vecteurs \( v_1,\ldots, v_n\) formant une base orthonormée de vecteurs propres\footnote{Possible par le théorème spectral~\ref{ThoeTMXla}.} de \( A\). Nous notons \( v_{max}\) celui de \( \lambda_{max}\). Nous avons :
    \begin{equation}
        \| A \|_2\geq \| Av_{max} \|=| \lambda_{max} |\| v_{max} \|=| \lambda_{max} |=\lambda_{max}.
    \end{equation}
    Voilà l'inégalité dans un sens. Montrons l'inégalité dans l'autre sens. Soit \( x=\sum_ix_iv_i\) avec \( \| x \|_2=1\). Alors
    \begin{equation}
        \| Ax \|=\| \sum_ix_i\lambda_iv_i \|\leq\sqrt{ \sum_ix_i^2\lambda_i^2 }\leq \lambda_{max}\sqrt{ \sum_ix_i^2}=\lambda_{max}.
    \end{equation}

    En ce qui concerne l'affirmation pour la norme de \( A^{-1}\), il suffit de remarquer que ses valeurs propres sont les inverses des valeurs propres de \( A\).
\end{proof}

\begin{proposition} \label{PropMAQoKAg}
    La fonction
    \begin{equation}
        \begin{aligned}
            f\colon \eM(n,\eR)\times \eM(n,\eR)&\to \eR \\
            (X,Y)&\mapsto \tr(X^tY)
        \end{aligned}
    \end{equation}
    est un produit scalaire sur \( \eM(n,\eR)\).
\end{proposition}
\index{trace!produit scalaire sur \( \eM(n,\eR)\)}
\index{produit!scalaire!sur \( \eM(n,\eR)\)}

\begin{proof}
    Il faut vérifier la définition~\ref{DefVJIeTFj}.
    \begin{itemize}
        \item La bilinéarité est la linéarité de la trace.
        \item La symétrie de \( f\) est le fait que \( \tr(A^t)=\tr(A)\).
        \item L'application \( f\) est définie positive parce que si \( X\in \eM\), alors \( X^tX\) est symétrique définie positive, donc diagonalisable avec des nombres positifs sur la diagonale. La trace étant un invariant de similitude, nous avons \( f(X,X)=\tr(X^tX)\geq 0\). De plus si \( \tr(X^tX)=0\), alors \( X^tX=0\) (pour la même raison de diagonalisation). Mais alors \( \| Xu \|=0\) pour tout \( u\in E\), ce qui signifie que \( X=0\).
    \end{itemize}
\end{proof}

\begin{example}
	Soit $m=n$, un point $\lambda$ dans $\eR$ et $T_{\lambda}$ l'application linéaire définie par $T_{\lambda}(x)=\lambda x$. La norme de $T_{\lambda}$ est alors
\[
\|T_{\lambda}\|_{\mathcal{L}}=\sup_{\|x\|_{\eR^m}\leq 1}\|\lambda x\|_{\eR^n}= |\lambda|.
\]
Notez que $T_{\lambda}$ n'est rien d'autre que l'homothétie de rapport $\lambda$ dans $\eR^m$.
\end{example}

\begin{example}
	Considérons la rotation $T_{\alpha}$ d'angle $\alpha$ dans $\eR^2$. Elle est donnée par l'équation matricielle
	\begin{equation}
		T_{\alpha}\begin{pmatrix}
			x	\\
			y
		\end{pmatrix}=\begin{pmatrix}
			\cos\alpha	&	\sin\alpha	\\
			-\sin\alpha	&	\cos\alpha
		\end{pmatrix}\begin{pmatrix}
			x	\\
			y
		\end{pmatrix}=\begin{pmatrix}
			\cos(\alpha)x+\sin(\alpha)y	\\
			-\sin(\alpha)x+\cos(\alpha)y
		\end{pmatrix}
	\end{equation}
	Étant donné que cela est une rotation, c'est une isométrie : $\| T_{\alpha}x \|=\| x \|$. En ce qui concerne la norme de $T_{\alpha}$ nous avons
	\begin{equation}
		\| T_{\alpha} \|=\sup_{x\in\eR^2}\frac{ \| T_{\alpha}(x) \| }{ \| x \| }=\sup_{x\in\eR^2}\frac{ \| x \| }{ \| x \| }=1.
	\end{equation}
	Toutes les rotations dans le plan ont donc une norme $1$. La même preuve tient pour toutes les rotations en dimension quelconque.
\end{example}

%TODO : le théorème de fuite des compacts qui dit qu'une solution de y'=f(y,t) cesse d'exister seulement si elle tend vers +- infini.

\begin{example}
  Soit $m=n$, un point $b$ dans $\eR^m$ et $T_b$ l'application linéaire définie par $T_b(x)=b\cdot x$ (petit exercice : vérifiez qu'il s'agit vraiment d'une application linéaire).  La norme de $T_b$ satisfait les inégalités suivantes
 \[
\|T_b\|_{\mathcal{L}}=\sup_{\|x\|_{\eR^m}\leq 1}\|b\cdot x\|_{\eR^n}\leq \sup_{\|x\|_{\eR^m}\leq 1}\|b \|_{\eR^n}\|x\cdot x\|_{\eR^n}\leq\|b \|_{\eR^n},
\]
\[
\|T_b\|_{\mathcal{L}}=\sup_{\|x\|_{\eR^m}\leq 1}\|b\cdot x\|_{\eR^n}\geq \left\|b\cdot \frac{b}{\|b \|_{\eR^n}}\right\|_{\eR^n}=\|b \|_{\eR^n},
\]
donc $\|T_b\|_{\mathcal{L}}=\|b \|_{\eR^n}$.
\end{example}

\begin{proposition}
    Une application linéaire de \( \eR^m\) dans \( \eR^n\) est continue.
\end{proposition}

\begin{proof}
      Soit $x$ un point dans $\eR^m$. Nous devons vérifier l'égalité
      \begin{equation}
       \lim_{h\to 0_m}T(x+h)=T(x).
      \end{equation}
      Cela revient à prouver que $\lim_{h\to 0_m}T(h)=0$, parce que $T(x+h)=T(x)+T(h)$. Nous pouvons toujours majorer $\|T(h)\|_n$ par $\|T\|_{\mathcal{L}(\eR^m,\eR^n)}\| h \|_{\eR^m}$ (lemme~\ref{LEMooIBLEooLJczmu}). Quand $h$ s'approche de $ 0_m $ sa norme $\|h\|_m$ tend vers $0$, ce que nous permet de conclure parce que nous savons que de toutes façons, $\| T \|_{\aL}$ est fini.
\end{proof}

Note : dans un espace de dimension infinie, la linéarité ne suffit pas pour avoir la continuité : il faut de plus être borné (ce que sont toutes les applications linéaires \( \eR^m\to\eR^n\)). Voir la proposition~\ref{PROPooQZYVooYJVlBd}.

%+++++++++++++++++++++++++++++++++++++++++++++++++++++++++++++++++++++++++++++++++++++++++++++++++++++++++++++++++++++++++++
\section{Produit fini d'espaces vectoriels normés}
%+++++++++++++++++++++++++++++++++++++++++++++++++++++++++++++++++++++++++++++++++++++++++++++++++++++++++++++++++++++++++++
\label{sec_prod}

Dans cette sections nous parlons de produits finis d'espaces. Cela ne signifie pas que chacun des espaces soient séparément de dimension finie.

%---------------------------------------------------------------------------------------------------------------------------
\subsection{Norme}
%---------------------------------------------------------------------------------------------------------------------------

La définition de la norme sur un produit d'espaces vectoriels normés découle immédiatement de la définition de la distance~\ref{DefZTHxrHA} :
\begin{definition}  \label{DefFAJgTCE}
    Soient $V$ et $W$ deux espaces vectoriels normés. On appelle \defe{espace produit}{produit!d'espaces vectoriels normés} de $V$ et $W$ le produit cartésien $V\times W$
    \begin{equation}
    V\times W=\{(v,w)\,|\, v\in V,\, w\in W\},
    \end{equation}
    muni de la norme $\|\cdot \|_{V\times W}$
    \begin{equation}	\label{EqNormeVxWmax}
        \|(v,w) \|_{V\times W}=\max\{\|v\|_{V},\|w\|_W\}.
    \end{equation}
\end{definition}
Il est presque immédiat de vérifier que le produit cartésien $V\times W$ est un espace vectoriel pour les opération de somme et multiplication par les scalaires définies composante par composante. C'est à dire,  si $(v_1,w_1)$, $(v_2,w_2)$ sont dans $V\times W$ et $a$, $b$ sont des scalaires, alors
\begin{equation}
 a (v_1,w_1)+ b(v_2,w_2)=(av_1,aw_1)+ (bv_2,bw_2)=(av_1+bv_2,aw_1+bw_2).
\end{equation}

\begin{lemma}
	L'opération $\|\cdot \|_{V\times W}\colon V\times W\to \eR$ est une norme.
\end{lemma}

\begin{proof}
	On doit vérifier les trois conditions de la définition~\ref{DefNorme}.
	\begin{itemize}
		\item Soit $(v,w)$ dans $V\times W$ tel que $\|(v,w)\|_{V\times W}=\max\{\|v\|_{V},\|w\|_W\}=0$. Alors $\|v\|_V=0$ et $\|w\|_W=0$, donc $v=0_V$ et $w=0_W$. Cela implique $(v,w)=(0_v,0_w)=0_{V\times W}$.
		\item Pour tout $a$ dans $\eR$ et $(v,w)$ dans $V\times W$,  la norme $\|a (v,w)\|_{V\times W}$ est donnée par  $\max\{\|av\|_{V},\|aw\|_W\}$. On peut factoriser $\|av\|_{V}=|a|\|v\|_{V}$ et $\|aw\|_W=|a|\|w\|_W$ et donc $\|a (v,w)\|_{V\times W}=|a|\max\{\|v\|_{V},\|w\|_W\}=|a|\|(v,w)\|_{V\times W}$.
		\item Soient $(v_1,w_1)$ et $(v_2,w_2)$ dans $V\times W$.
		\begin{equation}
			\begin{aligned}
				\|(v_1,w_1)+(v_2,w_2)\|_{V\times W}&=\max\{\|v_1+v_2\|_{V},\|w_1+w_2\|_W\}\\
				&\leq \max\{\|v_1\|_V+\|v_2\|_{V},\|w_1\|_W+\|w_2\|_W\}\\
				&\leq\max\{\|v_1\|_V,\|w_1\|_W\}+ \max\{\|v_2\|_{V},\|w_2\|_W\}\\
				&=\|(v_1,w_1)\|_{V\times W}+\|(v_2,w_2)\|_{V\times W}.
			\end{aligned}
		\end{equation}
	\end{itemize}
\end{proof}

Toutes ces définitions se généralisent à un produit fini d'espaces vectoriels normés. Si les espaces \( V_i\) sont des espaces vectoriels normés, nous pouvons mettre sur le produit une topologie et une norme :
\begin{itemize}
    \item La topologie produit donnée en~\ref{DefIINHooAAjTdY}
    \item La norme maximum \( \| v_1,\ldots, v_n \|_{max}=\max\{ \| v_1 \|,\ldots, \| v_n \| \}\). Dans le membre de droites, toutes les normes sont différentes.
\end{itemize}
Une question qui vient est la compatibilité entre ces deux constructions. Est-ce que la topologie associée à la norme maximum est le topologie produit ? Oui.

\begin{lemma}[\cite{ooALKGooMAzKpz}]       \label{LEMooWVVCooIGgAdJ}
    La topologie de la norme maximum est la topologie produit\footnote{Définition~\ref{DefIINHooAAjTdY}.}.
\end{lemma}

En particulier, pour la topologie de la norme maximum, la convergence d'une suite implique la convergence «composante par composante» par la proposition~\ref{PROPooNRRIooCPesgO}.

\begin{proposition}[\cite{ooCUHNooNYIeGt}]      \label{PROPooQFTSooPFfbCc}
    Soient des espaces vectoriels normés \( V\) et \( W\) ainsi qu'une forme sesquilinéaire \( \phi\colon V\times W\to \eC\). Il y a équivalence des faits suivants.
    \begin{enumerate}
        \item
            \( \phi\) est continue.
        \item
            \( \phi\) est continue en \( (0,0)\)
        \item
            \( \phi\) est bornée
        \item
            Il existe \( C\geq 0\) telle que \( | \phi(x,y) |\leq C\| x \|\| y \|  \) pour tout \( (x,y)\in V\times W\).
    \end{enumerate}
    De plus la norme de \( \phi\) est alors donnée par
    \begin{equation}
        \| \phi \|=\min\{  C\geq 0\tq | \phi(x,y) |\leq C\| x \|\| y \|\forall (x,y)\in V\times W  \}.
    \end{equation}
\end{proposition}

On remarque tout de suite que la norme $\|.\|_\infty$ sur $\eR^2$ est la norme de l'espace produit $\eR\times\eR$. En outre cette définition nous permet de trouver plusieurs nouvelles normes dans les espaces $\eR^p$. Par exemple, si nous écrivons $\eR^4$ comme $\eR^2\times \eR^2$ on peut munir $\eR^4$ de la norme produit
\[
\|(x_1,x_2,x_3,x_4)\|_{\infty, 2}=\max\{\|(x_1,x_2)\|_\infty, \|(x_3,x_4)\|_2\}.
\]
Les applications de projection de l'espace produit $V\times W$ vers les espaces <<facteurs>>, $V$ $W$ sont notées $\pr_V$ et $\pr_W$ et sont définies par
\begin{equation}
	\begin{aligned}
		\pr_V\colon V\times W&\to V \\
		(v,w)&\mapsto v
	\end{aligned}
\end{equation}
et
\begin{equation}
	\begin{aligned}
		\pr_W\colon V\times W &\to W \\
		(v,w)&\mapsto w.
	\end{aligned}
\end{equation}
Les inégalités suivantes sont évidentes
\begin{equation}
	\begin{aligned}[]
		\|\pr_V(v,w)\|_V&\leq \|(v,w)\|_{V\times W} \\
		\|\pr_W(v,w)\|_W&\leq \|(v,w)\|_{V\times W}.
	\end{aligned}
\end{equation}
La topologie de l'espace produit est induite par les topologies des espaces <<facteurs>>. La construction est faite en deux passages : d'abord nous disons que une partie $A\times B$ de $V\times W$ est ouverte si $A$ et $B$ sont des parties ouvertes de $V$ et de $W$ respectivement.  Ensuite nous définissons que une partie quelconque de $V\times W$ est ouverte si elle est une intersection finie ou une réunion de parties ouvertes de $V\times W$ de la forme $A\times B$.

Ce choix de topologie donne deux propriétés utiles de l'espace produit
\begin{enumerate}
	\item
		Les projections sont des \defe{applications ouvertes}{application!ouverte}. Cela veut dire que l'image par $\pr_V$ (respectivement $\pr_W$) de toute partie ouverte de $V\times W$ est une partie ouverte de $V$ (respectivement $W$).
	\item
		Pour toute partir $A$ de $V$ et $B$ de $W$, nous avons $\Int (A\times B)=\Int A\times \Int B$.\label{PgovlABeqbAbB}
\end{enumerate}
Une propriété moins facile a prouver est que pour toute partie $A$ de $V$ et $B$ de $W$ nous avons  $\overline{A\times B}=\bar{A}\times \bar{B}$. Voir le lemme~\ref{LemCvVxWcvVW}.
% position 26329
%et l'exercice~\ref{exoGeomAnal-0009}.

Ce que nous avons dit jusqu'ici est valable pour tout produit d'un nombre fini d'espaces vectoriels normés. En particulier, pour tout $m>0$  l'espace  $\eR^m$ peut être considéré comme le produit de $m$ copies de $\eR$.

\begin{example}
	Si $V$ et $W$ sont deux espaces vectoriels, nous pouvons considérer le produit $E=V\times W$. Les projections $\pr_V$ et $\pr_W$\nomenclature{$\pr_V$}{projection de $V\times W$ sur $V$}, définies dans la section~\ref{sec_prod}, sont des applications linéaires.

	En effet, la projection $\pr_V\colon V\times W\to V$ est donnée par $\pr_V(v,w)=v$. Alors,
	\begin{equation}
		\begin{aligned}[]
			\pr_V\big( (v,w)+(v',w') \big)&=\pr_V\big( (v+v'),(w+w') \big)\\
			&=v+v'\\
			&=\pr_V(v,w)+\pr_V(v',w'),
		\end{aligned}
	\end{equation}
	et
	\begin{equation}
		\pr_V\big( \lambda(v,w) \big)=\pr_V\big( (\lambda v,\lambda w) \big)=\lambda v=\lambda\pr_V(v,w).
	\end{equation}
	Nous laissons en exercice le soin d'adapter ces calculs pour montrer que $\pr_W$ est également une projection.
\end{example}

\begin{proposition} \label{PropDXR_KbaLC}
    Si \( \mO\) est un voisinage de \( (a,b)\) dans \( V\times W\) alors \( \mO\) contient un ouvert de la forme \( B(a,r)\times B(b,r)\).
\end{proposition}

\begin{proof}
    Vu que \( \mO\) est un voisinage, il contient un ouvert et donc une boule
    \begin{equation}
        B\big( (a,b),r \big)=\{ (v,w)\in V\times W\tq \max\{ \| v-a \|,\| w-b \| \}< r \}.
    \end{equation}
    Évidemment l'ensemble \( B(a,r)\times B(b,r)\) est dedans.
\end{proof}

%---------------------------------------------------------------------------------------------------------------------------
\subsection{Suites}
%---------------------------------------------------------------------------------------------------------------------------

Nous allons maintenant parler de suites dans $V\times W$. Nous noterons $(v_n,w_n)$ la suite dans $V\times W$ dont l'élément numéro $n$ est le couple $(v_n,w_n)$ avec $v_n\in V$ et $w_n\in W$. La notions de convergence de suite découle de la définition de la norme via la définition usuelle~\ref{DefCvSuiteEGVN}. Il se fait que dans le cas des produits d'espaces, la convergence d'une suite est équivalente à la convergence des composantes. Plus précisément, nous avons le lemme suivant.
\begin{lemma}		\label{LemCvVxWcvVW}
	La suite $(v_n,w_n)$ converge vers $(v,w)$ dans $V\times W$ si et seulement les suites $(v_n)$ et $(w_n)$ convergent séparément vers $v$ et $w$ respectivement dans $V$ et $W$.
\end{lemma}

\begin{proof}
	Pour le sens direct, nous devons étudier le comportement de la norme de $(v_n,w_n)-(v,w)$ lorsque $n$ devient grand. En vertu de la définition de la norme dans $V\times W$ nous avons
	\begin{equation}
		\Big\| (v_n,w_n)-(v,w) \Big\|_{V\times W}=\max\big\{ \| v_n-v \|_V,\| w_n-w \|_W \big\}.
	\end{equation}
	Soit $\varepsilon>0$. Par définition de la convergence de la suite $(v_n,w_n)$, il existe un $N\in\eN$ tel que $n>N$ implique
	\begin{equation}
		\max\big\{ \| v_n-v \|_V,\| w_n-w \|_W \big\}<\varepsilon,
	\end{equation}
	et donc en particulier les deux inéquations
	\begin{subequations}
		\begin{align}
			\| v_n-v \|&<\varepsilon\\
			\| w_n-w \|&<\varepsilon.
		\end{align}
	\end{subequations}
	De la première, il ressort que $(v_n)\to v$, et de la seconde que $(w_n)\to w$.

	Pour le sens inverse, nous avons pour tout $\varepsilon$ un $N_1$ tel que $\| v_n-v \|_V\leq\varepsilon$ pour tout $n>N_1$ et un $N_2$ tel que $\| w_n-w \|_W\leq\varepsilon$ pour tout $n>N_2$. Si nous posons $N=\max\{ N_1,N_2 \}$ nous avons les deux inégalités simultanément, et donc
	\begin{equation}
		\max\big\{ \| v_n-v \|_V,\| w_n-w \|_W \big\}<\varepsilon,
	\end{equation}
	ce qui signifie que la suite $(v_n,w_n)$ converge vers $(v,w)$ dans $V\times W$.
\end{proof}

\begin{proposition}[\cite{MonCerveau}]          \label{PROPooKDGOooDjWQct}
    Soit un espace \( E\) muni d'un produit scalaire à valeurs dans \( \eK\) (si \( \eK=\eC\) nous supposons le produit hermitien, mais ce n'est pas très important ici). Alors l'application
    \begin{equation}
        \begin{aligned}
            a\colon E\times E&\to \eK \\
            (x,y)&\mapsto \langle x, y\rangle
        \end{aligned}
    \end{equation}
    est continue.
\end{proposition}

\begin{proof}
    Nous ne disons pas que l'espace \( V\times V\) est muni d'un produit scalaire. Mais en tout cas c'est un espace métrique, et \( \eK\) l'est aussi. Donc \( a\) est une application entre deux espaces métriques et elle sera continue si et seulement si elle est séquentiellement continue (proposition~\ref{PropFnContParSuite}\ref{ItemWJHIooMdugfu}).

    Soit donc une suite convergente dans \( E\times E\), c'est à dire \( (x_k,y_k)\stackrel{E\times E}{\longrightarrow}(x,y)\). Nous devons démontrer que \( \langle x_k, y_k\rangle \stackrel{\eR}{\longrightarrow}\langle x, y\rangle \). Les majorations usuelles donnent
    \begin{subequations}
        \begin{align}
            \big| \langle x_k, y_k\rangle -\langle x, y\rangle  \big|&\leq \big| \langle x_k, y_k\rangle -\langle x, y_k\rangle  \big|+\big| \langle x, y_k\rangle -\langle x, y\rangle  \big|\\
            &=\big| \langle x_k-x, y_k\rangle  \big|+\big| \langle x, y_k-y\rangle  \big|.
        \end{align}
    \end{subequations}
    Nous savons du lemme~\ref{LemCvVxWcvVW} que les suites \( (x_k)\) et \( (y_k)\) sont séparément convergentes : \( x_k\stackrel{E}{\longrightarrow}x\) et \( y_k\stackrel{E}{\longrightarrow}y\). En utilisant l'inégalité de Cauchy-Schwarz~\ref{EQooZDSHooWPcryG} nous trouvons
    \begin{equation}
        \big| \langle x_k-x, y_k\rangle  \big|\leq \| x_k-x \|\| y_k \|.
    \end{equation}
    Nous avons \( \| x_k-x \|\to 0\) et \( \| y_k \|\to \| y \|\), et par la règle du produit de limites dans \( \eR\) nous avons que \( \big| \langle x_k-x, y_k\rangle  \big|\to 0\).
\end{proof}

\begin{remark}		\label{RemTopoProdPasRm}
	Il faut remarquer que la norme \eqref{EqNormeVxWmax} est une norme \emph{par défaut}. C'est la norme qu'on met quand on ne sait pas quoi mettre. Or il y a au moins un cas d'espace produit dans lequel on sait très bien quelle norme prendre : les espaces $\eR^m$. La norme qu'on met sur $\eR^2$ est
	\begin{equation}
		\| (x,y) \|=\sqrt{x^2+y^2},
	\end{equation}
	et non la norme «par défaut» de $\eR^2=\eR\times\eR$ qui serait
	\begin{equation}
		\| (x,y) \|=\max\{ | x |,| y | \}.
	\end{equation}
	Les théorèmes que nous avons donc démontré à propos de $V\times W$ ne sont donc pas immédiatement applicables au cas de $\eR^2$.

	Cette remarque est valables pour tous les espaces $\eR^m$. À moins de mention contraire explicite, nous ne considérons jamais la norme par défaut \eqref{EqNormeVxWmax} sur un espace $\eR^m$.
\end{remark}

Étant donné la remarque~\ref{RemTopoProdPasRm}, nous ne savons pas comment calculer par exemple la fermeture du produit d'intervalle $\mathopen] 0,1 ,  \mathclose[\times\mathopen[ 4 , 5 [$. Il se fait que, dans $\eR^m$, les fermetures de produits sont quand même les produits de fermetures.

\begin{proposition}		\label{PropovlAxBbarAbraB}
	Soit $A\subset\eR^m$ et $B\subset\eR^m$. Alors dans $\eR^{m+n}$ nous avons $\overline{ A\times B }=\bar A\times \bar B$.
\end{proposition}

La démonstration risque d'être longue; nous ne la faisons pas ici.

%---------------------------------------------------------------------------------------------------------------------------
\subsection{Continuité du produit de matrices}
%---------------------------------------------------------------------------------------------------------------------------
\label{SUBSECooOAWAooFcyUfI}

Nous avons introduit des normes sur \( \eM(n,\eK)\), entre autres la norme opérateur de la définition~\ref{DefNFYUooBZCPTr}. Qui dit norme dit topologie. Il advient alors la question évidente : est-ce que des opérations aussi élémentaires que le produit de matrices sont continues pour ces topologies ?

Une façon simple de répondre à cela est d'introduire sur \( \eM(n,\eK)\) une nouvelle norme très simple : celle de \( \eK^n\). C'est la topologie par composante. Pour cette topologie, il est simple de voir que le produit matriciel est continu parce que les éléments de \( AB\) sont des polynômes en les éléments de \( A\) \( B\). Ensuite il suffit d'invoquer l'équivalence de toutes les normes (théorème~\ref{ThoNormesEquiv}).

Voyons comment montrer cela de façon plus directe (bien que le raisonnement précédent soit une démonstration qui devrait déjà avoir convaincu les plus sceptiques). La preuve suivante va donc s'amuser à bien préciser les topologies et caractérisations utilisées.

\begin{lemma}
    Si \( \| . \|\) est une norme algébrique sur \( \eM(n,\eK)\) (\( \eK\) est \( \eR\) ou \( \eC\)) alors l'application
    \begin{equation}
        \begin{aligned}
            p\colon \eM(n,\eK)\times \eM(n,\eK)&\to \eM(n,\eK) \\
            (A,B)&\mapsto AB
        \end{aligned}
    \end{equation}
    est continue.
\end{lemma}

\begin{proof}
    L'espace \( \eM(n,\eK)\times \eM(n,\eK)\) est métrique (définition~\ref{DefFAJgTCE}), donc la caractérisation séquentielle de la continuité (proposition~\ref{PropXIAQSXr}) s'applique. Nous considérons donc une suite \( (A_k,B_k)\) dans \( \eM(n,\eK)\times \eM(n,\eK)\) convergente vers \( AB\).

    Nous savons que la topologie sur \( \eM(n,\eK)\times \eM(n,\eK)\) est la topologie produit (lemme~\ref{LEMooWVVCooIGgAdJ}) et que celle-ci donne la convergence composante par composante dès que nous avons convergence d'une suite; c'est la proposition~\ref{PROPooNRRIooCPesgO}. Nous avons donc \( A_k\stackrel{\eM(n,\eK)}{\longrightarrow}A\) et \( B_k\stackrel{\eM(n,\eK)}{\longrightarrow}B\).

    Voilà pour le contexte. Maintenant, la preuve de la continuité. Nous effectuons les majorations suivantes :
    \begin{subequations}
        \begin{align}
            \| p(A_k,B_K)-AB \|&\leq \| p(A_k,B_k)-p(A_k,B) \|+\| p(A_k,B)-AB \|\\
            &=\| A_Kb_k-A_kB \|+\| A_kB-AB \|\\
            &=\| A_k(B_k-B) \|+\| (A_k-A)B \|\\
            &\leq \underbrace{\| A_k \|}_{\to \| A \|}\underbrace{\| B_k-B \|}_{\to 0}+\underbrace{\| A_k-A \|}_{\to 0}\| B \|.
        \end{align}
    \end{subequations}
\end{proof}

%+++++++++++++++++++++++++++++++++++++++++++++++++++++++++++++++++++++++++++++++++++++++++++++++++++++++++++++++++++++++++++
\section{Calcul différentiel dans un espace vectoriel normé}
%+++++++++++++++++++++++++++++++++++++++++++++++++++++++++++++++++++++++++++++++++++++++++++++++++++++++++++++++++++++++++++
\label{SecLStKEmc}

%---------------------------------------------------------------------------------------------------------------------------
\subsection{Différentielle}
%---------------------------------------------------------------------------------------------------------------------------

\begin{propositionDef}  \label{DefKZXtcIT}
    Soit une application \( f\colon E\to F\) entre deux espaces vectoriels normés. Si \( T_1,T_2\colon E\to F\) sont deux applications linéaires continues telles que
    \begin{equation}\label{EqIQuRGmO}
        \lim_{h\to 0} \frac{ f(a+h)-f(a)-T_1(h) }{ \| h \| }= \lim_{h\to 0} \frac{ f(a+h)-f(a)-T_2(h) }{ \| h \| }  =0.
    \end{equation}
    Alors \( T_1=T_2\).

    Dans ce cas nous disons que \( f\) est \defe{différentiable}{différentiable!dans un Banach} en \( a\in E\) et l'application \( T_1=T_2\) est nommée \defe{différentielle}{différentielle} de \( f\) en \( a\).
\end{propositionDef}

\begin{proof}
    Nous avons
    \begin{equation}
        \| T_1(h)-T_2(h) \|\leq \| T_1(h)-f(x+h)+f(x) \|+\| f(x+h)-f(x)-T_2(h) \|,
    \end{equation}
    donc
    \begin{equation}
        \lim_{h\to 0} \frac{ \| T_1(h)-T_2(h) \| }{ \| h \| }\leq 0.
    \end{equation}

    Soit \( \epsilon>0\) et \( r>0\) tel que
    \begin{equation}
        \frac{ \| T_1(h)-T_2(h) \| }{ \| h \| }<\epsilon
    \end{equation}
    dès que \( h\in B(0,r)\). Alors en considérant \( \| h \|=1\) et \( t<r\) nous avons :
    \begin{equation}
        \frac{ \| T_1(th)-T_2(th) \| }{ \| th \| }<\epsilon.
    \end{equation}
    Nous avons donc
    \begin{equation}
        \lim_{t\to 0} \frac{ | t |\| T_1(h)-T_2(h) \| }{ | t |\| h \| }=0,
    \end{equation}
    et en nous rappelant que nous avions choisi \( \| h \|=1\),
    \begin{equation}
        \lim_{t\to 0} \| (T_1-T_2)(h) \|=0,
    \end{equation}
    autrement dit \( T_1(h)=T_2(h)\) pour tout \( h\) tel que \( \| h \|=1\). Par linéarité, \( T_1=T_2\) partout.
\end{proof}

L'application différentielle
\begin{equation}
    \begin{aligned}
        df\colon E&\to \aL(E,F) \\
        a&\mapsto df_a
    \end{aligned}
\end{equation}
est également très importante.

\begin{definition}      \label{DefJYBZooPTsfZx}
Une application \( f\colon E\to F\) est de \defe{classe \( C^1\)}{classe $C^1$} lorsque l'application différentielle \( df\colon E\to \aL(E,F)\) est continue. Voir aussi les définitions~\ref{DefPNjMGqy} pour les applications de classe \( C^k\).
\end{definition}

\begin{remark}      \label{RemATQVooDnZBbs}
    L'application norme étant continue, le critère du théorème~\ref{ThoWeirstrassRn} est en réalité assez général. Par exemple à partir d'une application différentiable\footnote{Définition~\ref{DefKZXtcIT}.} \( f\colon X\to Y\)  nous pouvons considérer la fonction réelle
    \begin{equation}
        a\mapsto \|  df_a   \|
    \end{equation}
    où la norme est la norme opérateur\footnote{Définition~\ref{DefNFYUooBZCPTr}.}. Si \( f\) est de classe \( C^1\) alors cette application est continue et donc bornée sur un compact \( K\) de \( X\).
\end{remark}

%---------------------------------------------------------------------------------------------------------------------------
\subsection{(non ?) Différentiabilité des applications linéaires}
%---------------------------------------------------------------------------------------------------------------------------

Si \( E\) et \( F\) sont deux espaces vectoriels nous notons \( \aL(E,F)\)\nomenclature[Y]{\( \aL(E,F)\)}{Les applications linéaires de \( E\) vers \( F\)} l'ensemble des applications linéaires de \( E\) vers \( F\) et \( \cL(E,F)\)\nomenclature[Y]{\( \cL\)}{Les applications linéaires continues de \( E\) vers \( F\)} l'ensemble des applications linéaires continues de \( E\) vers \( F\). Ces espaces seront bien entendu, sauf mention du contraire, toujours munis de la norme opérateur de la définition~\ref{DefNFYUooBZCPTr}.

\begin{lemma}       \label{LemooXXUGooUqCjmp}
    Soit une application linéaire \( f\).
    \begin{enumerate}
        \item
            Si \( f\) est continue, alors elle est différentiable et \( df_a(u)=f(u)\) pour tout \( a\) et \( u\).
        \item
            Si \( f\) n'est pas continue, alors elle n'est pas différentiable.
    \end{enumerate}
\end{lemma}

\begin{proof}
    La linéarité de \( f\) donne :
    \begin{equation}
        f(a+h)-f(a)-f(h)=0,
    \end{equation}
    et donc prendre \( T=f\) dans la définition~\ref{DefKZXtcIT} fait fonctionner la limite. De plus \( T\) est alors continue par hypothèse; elle est donc bien la différentielle de \( f\).

    Supposons que \( f\) ne soit pas continue, prenons une application linéaire continue \( T\), et calculons
    \begin{equation}        \label{EQooFLYMooEKTeOC}
        \frac{ f(a+h)-f(a)-T(h) }{ \| h \| }=\frac{ (f-T)(h) }{ \| h \| }=(f-T)(e_h)
    \end{equation}
    où \( e_h\) est le vecteur unitaire dans la direction de \( h\). Vu que \( f\) n'est pas continue et que \( T\) l'est, l'application \( f-T\) n'est pas continue. Elle n'est pas pas bornée par la proposition~\ref{PROPooQZYVooYJVlBd}. Il existe alors un vecteur \( h\) tel que \( \| (f-T)(e_h) \|>1\) (et même plus grand que ce qu'on veut).

    Donc la limite de \eqref{EQooFLYMooEKTeOC} pour \( h\to 0\) ne peut pas être nulle.
\end{proof}

\begin{lemma}   \label{LemLLvgPQW}
    Une application linéaire continue est de classe \(  C^{\infty}\).
\end{lemma}

\begin{proof}
    Soit \( a\in E\). Étant donné que \( f\) est linéaire et continue, elle est différentiable et
    \begin{equation}
        \begin{aligned}
            df\colon E&\to \cL(E,F) \\
            a&\mapsto f
        \end{aligned}
    \end{equation}
    est une fonction constante et en particulier continue; nous avons donc \( f\in C^1\). Pour la différentielle seconde nous avons \( d(df)_a=0\) parce que \( df(a+h)-df(a)=f-f=0\). Toutes les différentielles suivantes sont nulles.
\end{proof}

%---------------------------------------------------------------------------------------------------------------------------
\subsection{Dérivation en chaine et formule de Leibnitz}
%---------------------------------------------------------------------------------------------------------------------------

\begin{proposition} \label{PropOYtgIua}
    Soient \( f_i\colon U\to F_i\), des fonctions de classe \( C^r\) où \( U\) est ouvert dans l'espace vectoriel normé \( E\) et les \( F_i\) sont des espaces vectoriels normés. Alors l'application
    \begin{equation}
        \begin{aligned}
        f=f_1\times \cdots\times f_n\colon U&\to F_1\times \cdots\times F_n \\
    x&\mapsto \big( f_1(x),\ldots, f_n(x) \big)
        \end{aligned}
    \end{equation}
    est de classe \( C^r\) et
    \begin{equation}
    d^rf=d^rf_1\times\ldots d^rf_n.
    \end{equation}
\end{proposition}

\begin{proof}
    Soit \( x\in U\) et \( h\in E\). La différentiabilité des fonctions \( f_i\) donne
    \begin{equation}
        f_i(x+h)=f_i(x)+(df_i)_x(h)+\alpha_i(h)
    \end{equation}
    avec \( \lim_{h\to 0} \alpha_i(h)/\| h \|=0\). Par conséquent
    \begin{equation}
        f(x+h)=\big( \ldots, f_i(x)+(df_i)_x(h)+\alpha_i(h),\ldots \big)= \big( \ldots,f_i(x),\ldots \big)+ \big( \ldots,(df_i)_x(h),\ldots \big)+ \big( \ldots,\alpha_i(h),\ldots \big).
    \end{equation}
    Mais la définition~\ref{DefFAJgTCE} de la norme dans un espace produit donne
    \begin{equation}
        \lim_{h\to 0} \frac{ \| \big( \alpha_1(h),\ldots, \alpha_n(h) \big) \| }{ \| h \| }=0,
    \end{equation}
    ce qui nous permet de noter \( \alpha(h)=\big( \alpha_1(h),\ldots, \alpha_n(h) \big)\) et avoir \( \lim_{h\to 0} \alpha(h)/\| h \|=0\). Avec tout ça nous avons bien
    \begin{equation}
        f(x+h)=f(x)+\big( (df_1)_x(h)+\cdots +(df_n)_x(h) \big)+\alpha(h),
    \end{equation}
    ce qui signifie que \( f\) est différentiable et
    \begin{equation}
        df_x=\big( df_1,\ldots, df_n \big).
    \end{equation}
\end{proof}

\begin{theorem}[Différentielle de fonctions composées\cite{SNPdukn}]    \label{ThoAGXGuEt}
    Soient \( E\), \( F\) et \( G\) des espaces vectoriels normés, \( U\) ouvert dans \( E\) et \( V\) ouvert dans \( F\). Soient des applications de classe \( C^r\) (\( r\geq 1\))
    \begin{subequations}
        \begin{align}
            f\colon U\to V\\
            g\colon V\to G.
        \end{align}
    \end{subequations}
    Alors l'application \( g\circ f\colon V\to G\) est de classe \( C^r\) et
    \begin{equation}\label{EqHFmezmr}
        d(g\circ f)_x=dg_{f(x)}\circ df_x.
    \end{equation}
\end{theorem}

\begin{proof}
    Nous nous fixons \( x\in U\). La fonction \( f\) est différentiable en \( x\in U\) et \( g\) en \( f(x)\), donc nous pouvons écrire
    \begin{equation}
        f(x+h)=f(x)+df_x(h)+\alpha(h)
    \end{equation}
    et
    \begin{equation}
        g\big( f(x)+u \big)=g\big( f(x) \big)+dg_{f(x)}(u)+\beta(u)
    \end{equation}
    où la fonction \( \alpha\) a la propriété que
    \begin{equation}
        \lim_{h\to 0} \frac{ \| \alpha(h) \| }{ \| h \| }=0;
    \end{equation}
    et la même chose pour \( \beta\). La fonction composée en \( x+h\) s'écrit donc
    \begin{equation}    \label{EqCXcfhfH}
        (g\circ f)(x+h)=g\big( f(x)+df_x(h)+\alpha(h) \big)=g\big( f(x) \big)+dg_{f(x)}\big( df_x(h)+\alpha(h) \big)+\beta\big( df_x(h)+\alpha(h) \big).
    \end{equation}
    Nous montrons que tous les «petits» termes de cette formule peuvent être groupés. D'abord si \( h\) est proche de \( 0\), nous avons
    \begin{equation}
        \frac{ \| df_x(h)+\alpha(h) \| }{ \| h \| }\leq\frac{ \| df_x \|\| h \| }{ \| h \| }+\frac{ \| \alpha(h) \| }{ \| h \| }.
    \end{equation}
    Si \( h\) est petit, le second terme est arbitrairement petit, donc en prenant n'importe que \( M>\| df_x \|\) nous avons
    \begin{equation}
        \frac{ \| df_x(h)+\alpha(h) \| }{ \| h \| }\leq M.
    \end{equation}
    Par ailleurs, nous avons
    \begin{equation}
        \frac{ \| \beta\big( df_x(h)+\alpha(h) \big) \| }{ \| h \| }=\frac{  \| \beta\big( df_x(h)+\alpha(h) \big) \|  }{ \| df_x(h)+\alpha(h) \| }\frac{  \| df_x(h)+\alpha(h) \|  }{ \| h \| }\leq M\frac{  \| \beta\big( df_x(h)+\alpha(h) \big) \|  }{   \| df_x(h)+\alpha(h) \| }.
    \end{equation}
    Vu que la fraction est du type \( \frac{ \beta( f(h)) }{ f(h) }\) avec \( \lim_{h\to 0} f(h)=0\), la fraction tend vers zéro lorsque \( h\to 0\). En posant
    \begin{equation}
        \gamma_1(h)=\beta\big( df_x(h)+\alpha(h) \big)
    \end{equation}
    nous avons \( \lim_{h\to 0} \gamma_1(h)/\| h \|=0\).

    L'autre candidat à être un petit terme dans \eqref{EqCXcfhfH} est traité en utilisant le lemme~\ref{LEMooFITMooBBBWGI} :
    \begin{equation}
        \| dg_{f(x)}\big( \alpha(h) \big) \|\leq \| dg_{f(x)} \|\| \alpha(h) \|.
    \end{equation}
    Donc
    \begin{equation}
        \frac{ \| dg_{f(x)}\big( \alpha(h) \big) \| }{ \| h \| }\leq \| dg_{f(x)} \|\frac{ \| \alpha(h) \| }{ \| h \| },
    \end{equation}
    ce qui nous permet de poser
    \begin{equation}
        \gamma_2(h)=dg_{f(x)}\big( \alpha(h) \big)
    \end{equation}
    avec \( \gamma_2\) qui a la même propriété que \( \gamma_1\). Avec tout cela, en posant \( \gamma=\gamma_1+\gamma_2\) nous récrivons
    \begin{equation}
        (g\circ f)(x+h)=g\big( f(x) \big)+dg_{f(x)}\big( df_x(h) \big)+\gamma(h)
    \end{equation}
    avec \( \lim_{h\to 0} \frac{ \gamma(h) }{ \| h \| }=0\). Tout cela pour dire que
    \begin{equation}
        \lim_{h\to 0} \frac{ (g\circ f)(x+h)-(g\circ f)(x)-\big( dg_{f(x)}\circ df_x \big)(h) }{ \| h \| }=0,
    \end{equation}
    ce qui signifie que
    \begin{equation}
        d(g\circ f)_x=dg_{f(x)}\circ df_x.
    \end{equation}
    Nous avons donc montré que si \( f\) et \( g\) sont différentiables, alors \( g\circ f\) est différentiable avec différentielle donnée par \eqref{EqHFmezmr}.

    Nous passons à la régularité. Nous supposons maintenant que \( f\) et \( g\) sont de classe \( C^r\) et nous considérons l'application
    \begin{equation}
        \begin{aligned}
            \varphi\colon L(F,G)\times L(E,F)&\to L(E,G) \\
            (A,B)&\mapsto A\circ B.
        \end{aligned}
    \end{equation}
    Montrons que l'application \( \varphi\) est continue en montrant qu'elle est bornée\footnote{Proposition~\ref{PROPooQZYVooYJVlBd}.}. Pour cela nous écrivons la norme opérateur
    \begin{equation}
        \| \varphi \|=\sup_{\| (A,B) \|=1}\| \varphi(A,B) \|=\sup_{\| (A,B) \|=1}\| A\circ B \|\leq\sup_{\| (A,B) \|=1}\| A \|\| B \|\leq 1.
    \end{equation}
    Pour ce calcul nous avons utilisé le fait que la norme opérateur soit une norme algébrique (lemme~\ref{LEMooFITMooBBBWGI}) ainsi que la définition~\ref{DefFAJgTCE} de la norme sur un espace produit pour la dernière majoration. L'application \( \varphi\) est donc continue et donc \(  C^{\infty}\) par le lemme~\ref{LemLLvgPQW}. Nous considérons également l'application
    \begin{equation}
        \begin{aligned}
        \psi\colon U&\to L(F,G)\times L(E,F) \\
        x&\mapsto \big( dg_{f(x)},df_x \big).
        \end{aligned}
    \end{equation}
    Vu que \( f\) et \( g\) sont \( C^1\), l'application \( \psi\) est continue. Ces deux applications \( \varphi\) et \( \psi\) sont choisies pour avoir
    \begin{equation}
        (\varphi\circ\psi)(x)=\varphi\big( dg_{f(x)},df_x \big)=dg_{f(x)}\circ df_x,
    \end{equation}
    c'est à dire \( \varphi\circ\psi=d(g\circ f)\). Les applications \( \varphi\) et \( \psi\) étant continues, l'application \( d(g\circ f)\) est continue, ce qui prouve que \( g\circ f\) est \( C^1\).

    Si \( f\) et \( g\) sont \( C^r\) alors \( dg\in C^{r-1}\) et \( dg\circ f\in C^{r-1}\) où il ne faut pas se tromper : \( dg\colon F\to L(F,G)\) et \( f\colon U\to F\); la composée est \( dg\circ f\colon x\mapsto dg_{f(x)}\in L(F,G)\).

    Pour la récurrence nous supposons que \( f,g\in C^{r-1}\) implique \( g\circ f\in C^{r-1}\) pour un certain \( r\geq 2\) (parce que nous venons de prouver cela avec \( r=1\) et \( r=2\)). Soient \( f,g\in C^r\) et montrons que \( g\circ f\in C^r\). Par la proposition~\ref{PropOYtgIua} nous avons
    \begin{equation}
        \psi=dg\circ f\times df\in C^{r-1},
    \end{equation}
    et donc \( d(g\circ f)=\varphi\circ\psi\in C^{r-1}\), ce qui signifie que \( g\circ f\in C^r\).
\end{proof}

\begin{lemma}       \label{LemooTJSZooWkuSzv}
    Si \( f\colon U\to V\) est une difféomorphisme\footnote{Définition~\ref{DefAQIQooYqZdya}} alors pour tout \( a\in U\), l'application \( df_a\) est inversible et
    \begin{equation}
        (df_a)^{-1}=(df^{-1})_{f(a)}.
    \end{equation}
\end{lemma}

\begin{proof}
    Il suffit d'apercevoir qu'en vertu de la règle de différentiation en chaîne~\ref{EqHFmezmr},
    \begin{equation}
        (df_a)(df^{-1})_{f(a)}=f(f\circ f^{-1})_{f(a)}=\id.
    \end{equation}
\end{proof}
