% This is part of Mes notes de mathématique
% Copyright (c) 2012-2015,2017
%   Laurent Claessens
% See the file fdl-1.3.txt for copying conditions.

%+++++++++++++++++++++++++++++++++++++++++++++++++++++++++++++++++++++++++++++++++++++++++++++++++++++++++++++++++++++++++++ 
\section{Anciennes leçons}
%+++++++++++++++++++++++++++++++++++++++++++++++++++++++++++++++++++++++++++++++++++++++++++++++++++++++++++++++++++++++++++

\paragraph{Opérations élémentaires sur les lignes et les colonnes d’une matrice. Exemples et applications.}
%\index{matrice!lignes et colonnes}
\begin{itemize}
    \item Décomposition de Bruhat, théorème \ref{ThoizlYJO}.
    \item Algorithme des facteurs invariants \ref{PropPDfCqee}.
\end{itemize}

\paragraph{Exemples de décompositions remarquables dans le groupe linéaire. Applications}
\begin{itemize}
    \item Décomposition polaire \ref{ThoLHebUAU}.
    \item Décomposition de Dunford, théorème \ref{ThoRURcpW}. 
\end{itemize}
%---------------------------------------------------------------------------------------------------------------------------------------------
\paragraph{Résultant. Applications.}
\begin{itemize}
    \item Théorème de Rothstein-Trager \ref{ThoXJFatfu}.
    \item Théorème de Kronecker \ref{ThoOWMNAVp}.
\end{itemize}
%---------------------------------------------------------------------------------------------------------------------------------------------
\paragraph{Matrices équivalentes. Matrices semblables. Applications.}
\begin{itemize}
    \item Racine carré d'une matrice hermitienne positive, proposition \ref{PropVZvCWn}.
    \item Sous-groupes compacts de \( \GL(n,\eR)\), lemme \ref{LemOCtdiaE} ou proposition \ref{PropQZkeHeG}.
        %Ici on peut mettre le théorème de Sylvester.
\end{itemize}
\paragraph{Exemples d'utilisation de la notion de dimension d'un espace vectoriel.}
%\index{espace!vectoriel!dimension}
\begin{itemize}
    \item Forme alternées de degré maximum, proposition \ref{ProprbjihK}, parce que c'est ce théorème qui donne l'unicité du déterminant du fait que l'espace est de dimension un.
    \item Théorème de la dimension \ref{ThonmnWKs}, bien que ce soit plutôt dans la définition de la dimension que dans l'utilisation.
    \item Théorème de Carathéodory \ref{ThoJLDjXLe}.
\end{itemize}

\paragraph{Corps des fractions rationnelles à une indéterminée sur un corps commutatif. Applications.}
\begin{itemize}
    \item Théorème de Rothstein-Trager \ref{ThoXJFatfu}.
    \item Partitions d'un entier en parts fixes, proposition \ref{PropWUFpuBR}.
\end{itemize}

\paragraph{Anneau de séries formelles. Applications.}
\begin{itemize}
    \item Nombres de Bell, théorème \ref{ThoYFAzwSg}.
    \item Partitions d'un entier en parts fixes, proposition \ref{PropWUFpuBR}.
\end{itemize}
%---------------------------------------------------------------------------------------------------------------------------------------------
\paragraph{Formes linéaires et hyperplans en dimension finie. Exemples et applications.}
\begin{itemize}
    \item Extrema liés, théorème \ref{ThoRGJosS}.
    \item Enveloppe convexe du groupe orthogonal \ref{ThoVBzqUpy}.
    \item Une forme canonique pour les transvections et dilatations, théorème \ref{ThoooAZKDooNDcznv}.
\end{itemize}
%---------------------------------------------------------------------------------------------------------------------------------------------
\paragraph{Algèbre des polynômes d'un endomorphisme en dimension finie. Applications.}
\begin{itemize}
    \item Racine carré d'une matrice hermitienne positive, proposition \ref{PropVZvCWn}.
\end{itemize}
\paragraph{Formes quadratiques réelles. Exemples et applications.}
\begin{itemize}
    \item Le lemme au lemme de Morse, lemme \ref{LemWLCvLXe}.
    \item Connexité des formes quadratiques de signature donnée, proposition \ref{PropNPbnsMd}.
    \item Sous-groupes compacts de \( \GL(n,\eR)\), lemme \ref{LemOCtdiaE} ou proposition \ref{PropQZkeHeG}.
    \item Ellipsoïde de John-Loewner, proposition \ref{PropJYVooRMaPok}.
% On pourra mettre le théorème de Sylvester.
\end{itemize}
\paragraph{Exemples et représentations de groupes finis de petit cardinal}
\paragraph{Coniques. Applications.}
\paragraph{Polynômes d'endomorphisme en dimension finie. Applications à la réduction d'un endomorphisme en dimension finie.}
\begin{itemize}
    \item Endomorphismes cycliques et commutant dans le cas diagonalisable, proposition \ref{PropooQALUooTluDif}.
\end{itemize}
\paragraph{Applications des nombres complexes à la géométrie. Homographies.}
\begin{itemize}
    \item Action du groupe modulaire sur le demi-plan de Poincaré, théorème \ref{ThoItqXCm}. Parce que l'action est avec des homographies.
\end{itemize}

\paragraph{Anneaux principaux. Applications}
\begin{itemize}
    \item Polynôme minimal d'endomorphisme semi-simple, théorème \ref{ThoFgsxCE}.
    \item Théorème de Bézout, corollaire \ref{CorimHyXy}.
    \item Théorème des deux carrés, théorème \ref{ThospaAEI}.
    \item Algorithme des facteurs invariants \ref{PropPDfCqee}.
\end{itemize}
\paragraph{Représentations de groupes finis de petit cardinal.}
\begin{itemize}
    \item Table des caractères du groupe diédral, section \ref{SecWMzheKf}.
    \item Table des caractères du groupe symétrique \( S_4\), section \ref{SecUMIgTmO}.
\end{itemize}
\paragraph{Algèbre des polynômes à \( n\) indéterminées (\( n\geq 2\)). Polynômes symétriques. Applications.}
\begin{itemize}
    \item À propos d'extensions de \( \eQ\), le lemme \ref{LemSoXCQH}.
    \item Polynômes semi-symétriques, proposition \ref{PropUDqXax}.
    \item Théorème de Chevalley-Warning \ref{ThoLTcYKk}.
    \item Théorème de Kronecker \ref{ThoOWMNAVp}.
\end{itemize}

\paragraph{Racines d’un polynômes. Fonctions symétriques élémentaires. Localisation des racines dans les cas réel et complexe.}
\begin{itemize}
    \item À propos d'extensions de \( \eQ\), le lemme \ref{LemSoXCQH}.
\end{itemize}

\paragraph{135 - Isométries d'un espace affine euclidien de dimension finie. Forme réduite. Applications en dimensions $2$ et $3$.}
\begin{itemize}
    \item Points extrémaux de la boule unité dans \( \aL(E)\), théorème \ref{ThoBALmoQw}.
    \item Générateurs du groupe diédral, proposition \ref{PropLDIPoZ}
    \item Isométries du cube, section \ref{SecPVCmkxM}.
\end{itemize}

\paragraph{248 - Approximation des fonctions numériques par des fonctions polynomiales. Exemples.}
\begin{itemize}
    \item Théorème taubérien de Hardy-Littlewood \ref{ThoPdDxgP}.
    \item Théorème de Runge \ref{ThoMvMCci}.
    \item Densité des polynômes dans \( C^0\big( \mathopen[ 0 , 1 \mathclose] \big)\), théorème de Bernstein \ref{ThoDJIvrty}.
\end{itemize}
%---------------------------------------------------------------------------------------------------------------------------------------------
\paragraph{250 - Loi des grands nombres. Théorème central limite. Applications.}
\begin{itemize}
    \item Presque tous les nombres sont normaux, proposition \ref{PropEEOXLae}.
    \item Estimation des grands écarts, théorème \ref{ThoYYaBXkU}.
\end{itemize}
%---------------------------------------------------------------------------------------------------------------------------------------------
\paragraph{251 - Indépendance d’événements et de variables aléatoires. Exemples.}
\begin{itemize}
    \item Presque tous les nombres sont normaux, proposition \ref{PropEEOXLae}.
    \item Estimation des grands écarts, théorème \ref{ThoYYaBXkU}.
    \item Densité des polynômes dans \( C^0\big( \mathopen[ 0 , 1 \mathclose] \big)\), théorème de Bernstein \ref{ThoDJIvrty}.
    \item Problème de la ruine du joueur, section \ref{SecMSOjfgM}.
\end{itemize}
%---------------------------------------------------------------------------------------------------------------------------------------------
\paragraph{252 - Loi binomiale. Loi de Poisson. Applications.}
\begin{itemize}
        % Cette leçon est classée dans les non couvertes parce qu'il faudrait un développement sur la loi de Poisson.
    \item Estimation des grands écarts, théorème \ref{ThoYYaBXkU}.
    \item Densité des polynômes dans \( C^0\big( \mathopen[ 0 , 1 \mathclose] \big)\), théorème de Bernstein \ref{ThoDJIvrty}.
    \item Problème de la ruine du joueur, section \ref{SecMSOjfgM}.
\end{itemize}
%---------------------------------------------------------------------------------------------------------------------------------------------
\paragraph{219 - Problèmes d'extremums.}
\begin{itemize}
    \item Extrema liés, théorème \ref{ThoRGJosS}.
    \item Lemme de Morse, lemme \ref{LemNQAmCLo}.
    \item Ellipsoïde de John-Loewner, proposition \ref{PropJYVooRMaPok}.
\end{itemize}
%---------------------------------------------------------------------------------------------------------------------------------------------
\paragraph{Équations différentielles $X' = f (t , X )$. Exemples d'étude des solutions en dimension $1$ et $2$.}
\begin{itemize}
    \item Théorème de Cauchy-Lipschitz \ref{ThokUUlgU}.
    \item Théorème de stabilité de Lyapunov \ref{ThoBSEJooIcdHYp}.
    \item Équation de Hill \( y''+qy=0\), proposition \ref{PropGJCZcjR}.
    \item Le système proie-prédateur de Lotka-Volterra \ref{ThoJHCLooHjeCvT}
\end{itemize}
%---------------------------------------------------------------------------------------------------------------------------------------------
\paragraph{223 - Convergence des suites numériques. Exemples et applications.}
\begin{itemize}
    \item Calcul d'intégrale par suite équirépartie \ref{PropDMvPDc}.
    \item Théorème taubérien de Hardy-Littlewood \ref{ThoPdDxgP}.
    \item Méthode de Newton, théorème \ref{ThoHGpGwXk}
    \item Théorème d'Abel angulaire \ref{ThoTGjmeen}.
\end{itemize}
%---------------------------------------------------------------------------------------------------------------------------------------------
\paragraph{224 - Comportement asymptotique de suites numériques. Rapidité de convergence. Exemples.}
\begin{itemize}
    \item Le dénombrement des solutions de l'équation \( \alpha_1 n_1+\ldots \alpha_pn_p=n\) utilise des séries entières et des décomposition de fractions en éléments simples, théorème \ref{ThoDIDNooUrFFei}.
    \item Divergence de la somme des inverses des nombres premiers, théorème \ref{ThonfVruT}.
    \item Formule sommatoire de Poisson, proposition \ref{ProprPbkoQ}, grâce à l'exemple \ref{ExDLjesf}.
    \item Méthode de Newton, théorème \ref{ThoHGpGwXk}
    \item Estimation des grands écarts, théorème \ref{ThoYYaBXkU}.
    \item Le dénombrement des solutions de l'équation \( \alpha_1 n_1+\ldots \alpha_pn_p=n\) utilise des séries entières et des décomposition de fractions en éléments simples, théorème \ref{ThoDIDNooUrFFei}.
\end{itemize}
%---------------------------------------------------------------------------------------------------------------------------------------------
\paragraph{226 - Comportement d’une suite réelle ou vectorielle définie par une itération \( u_{n+1}=f(u_n)\). Exemples.}
\begin{itemize}
    \item Processus de Galton-Watson, section \ref{SecBPmrPdtGalton}.
    \item Méthode de Newton, théorème \ref{ThoHGpGwXk}
    \item Méthode du gradient à pas optimal \ref{PropSOOooGoMOxG}.
\end{itemize}
%---------------------------------------------------------------------------------------------------------------------------------------------
\paragraph{245 - Fonctions holomorphes et méromorphes sur un ouvert de \( \eC\). Exemples et applications.}
\begin{itemize}
    \item Le théorème de Weierstrass sur la limite uniforme de fonctions holomorphes, théorème \ref{ThoArYtQO}.
    \item Théorème de Montel \ref{ThoXLyCzol}.
    \item Prolongement méromorphe de la fonction \( \Gamma\) d'Euler.
\end{itemize}
%---------------------------------------------------------------------------------------------------------------------------------------------
\paragraph{238 - Méthodes de calcul approché d'intégrales et de solutions d’équations différentielles.}
\begin{itemize}
    \item Calcul d'intégrale par suite équirépartie \ref{PropDMvPDc}.
\end{itemize}
%---------------------------------------------------------------------------------------------------------------------------------------------
\paragraph{240 - Transformation de Fourier. Applications.}
\begin{itemize}
    \item Formule sommatoire de Poisson, proposition \ref{ProprPbkoQ}.
    \item Équation de Schrödinger, théorème \ref{ThoLDmNnBR}.
\end{itemize}
%---------------------------------------------------------------------------------------------------------------------------------------------
\paragraph{222 - Exemples d’équations différentielles. Solutions exactes ou approchées.}
\begin{itemize}
    \item Équation \( y''+qy=0\), \ref{subsecSyTwyM}.
\end{itemize}
%---------------------------------------------------------------------------------------------------------------------------------------------
\paragraph{225 - Étude locale de surfaces. Exemples.}
\begin{itemize}
    \item Lemme de Morse, lemme \ref{LemNQAmCLo}.
\end{itemize}
%---------------------------------------------------------------------------------------------------------------------------------------------
\paragraph{Exemples de problèmes d’interversion de limites.}
%\index{limite!inversion}
\begin{itemize}
    \item Théorème taubérien de Hardy-Littlewood \ref{ThoPdDxgP} parce que l'énoncé revient à dire que \( \lim_{x\to 1^-} \sum_{n\in \eN}a_nx^n=\sum_{n\in \eN}a_n\).
    \item Le théorème de Weierstrass sur la limite uniforme de fonctions holomorphes, théorème \ref{ThoArYtQO}.
    \item La proposition \ref{PropWoywYG} qui donne des indications sur la notion de classes dans \( L^p\). Ça utilise la convergence monotone pour  pour permuter une somme et une intégrale.
    \item Les théorèmes sur les fonctions définies par des intégrales, section \ref{SecCHwnBDj}.
    \item Nombres de Bell, théorème \ref{ThoYFAzwSg}.
\end{itemize}
\paragraph{254 - Espaces de Schwartz et distributions tempérées.}
\begin{itemize}
    \item Formule sommatoire de Poisson, proposition \ref{ProprPbkoQ}.
    \item Équation de Schrödinger, théorème \ref{ThoLDmNnBR}.
\end{itemize}

\paragraph{Théorème d’inversion locale, théorème des fonctions implicites. Exemples et applications.}
\begin{itemize}
    \item Extrema liés, théorème \ref{ThoRGJosS}.
    \item Théorème d'inversion locale, théorème \ref{ThoXWpzqCn}.
    \item Lemme de Morse, lemme \ref{LemNQAmCLo}.
    \item Théorème de Von Neumann \ref{ThoOBriEoe}.
\end{itemize}
\paragraph{Continuité et dérivabilité des fonctions réelles d'une variable réelle. Exemples et contre-exemples.}
\begin{itemize}
    \item Les théorèmes sur les fonctions définies par des intégrales, section \ref{SecCHwnBDj}.
    \item Lemme de Borel \ref{LemRENlIEL}.
\end{itemize}
\paragraph{232 - Méthodes d'approximation des solutions d’une équation $F(X)=0$. Exemples.}
\begin{itemize}
    \item Méthode de Newton, théorème \ref{ThoHGpGwXk}
    \item Méthode du gradient à pas optimal \ref{PropSOOooGoMOxG}.
\end{itemize}
\paragraph{Illustrer par des exemples quelques méthodes de calcul d'intégrales de fonctions d’une ou plusieurs variables réelles.}
\begin{itemize}
    \item Calcul d'intégrale par suite équirépartie \ref{PropDMvPDc}.
    \item Théorème de Rothstein-Trager \ref{ThoXJFatfu}.
\end{itemize}
\paragraph{256 - Transformation de Fourier dans \( \swS(\eR^d)\) et \( \swS'(\eR^d)\).}
\begin{itemize}
    \item Formule sommatoire de Poisson, proposition \ref{ProprPbkoQ}.
    \item Équation de Schrödinger, théorème \ref{ThoLDmNnBR}.
\end{itemize}
%---------------------------------------------------------------------------------------------------------------------------------------------
\paragraph{Espaces de Schwartz \( \swS(\eR^d)\) et distributions tempérées. Transformation de Fourier dans \( \swS(\eR^d)\) et \( \swS'(\eR^d)\)}
\begin{itemize}
    \item Formule sommatoire de Poisson, proposition \ref{ProprPbkoQ}.
    \item Équation de Schrödinger, théorème \ref{ThoLDmNnBR}.
\end{itemize}
%---------------------------------------------------------------------------------------------------------------------------------------------
\paragraph{Espaces de Schwartz. Distributions. Dérivation au sens des distributions.}
\begin{itemize}
    \item L'équation \( (x-x_0)^{\alpha}u=0\) pour \( u\in\swD'(\eR)\), théorème \ref{ThoRDUXooQBlLNb}.
    \item Espace de Sobolev \( H^1(I)\), théorème \ref{ThoESIyxfU}.
    \item Équation de Schrödinger, théorème \ref{ThoLDmNnBR}.
\end{itemize}

\paragraph{Analyse numérique matricielle : résolution approchée de systèmes linéaires, recherche de vecteurs propres, exemples.}
\paragraph{Fonctions développables en série entière, fonctions analytiques. Exemples.}
\begin{itemize}
    \item Le dénombrement des solutions de l'équation \( \alpha_1 n_1+\ldots \alpha_pn_p=n\) utilise des séries entières et des décomposition de fractions en éléments simples, théorème \ref{ThoDIDNooUrFFei}.
\end{itemize}
\paragraph{Fonction caractéristique et transformée de Laplace d'une variable aléatoire. Exemples et applications.}
\paragraph{Modes de convergence d’une suite de variables aléatoires. Exemples et applications.}
\paragraph{Variables aléatoires à densité. Exemples et applications.}
\paragraph{Variables aléatoires discrètes. Exemples et applications.}
\paragraph{Produit de convolution, transformation de Fourier. Applications.}
\paragraph{Étude métrique des courbes. Exemples.}
\paragraph{Théorèmes de point fixe. Exemples et applications.}
\begin{itemize}
    \item Processus de Galton-Watson, théorème \ref{ThoJZnAOA}.
    \item Théorème d'inversion locale, théorème \ref{ThoXWpzqCn}.
    \item Théorème de Brouwer en dimension \( 2\) via l'homotopie \ref{ThoLVViheK}.
    \item Théorème de Picard \ref{ThoEPVkCL} et l'inséparable théorème de Cauchy-Lipschitz \ref{ThokUUlgU}
\end{itemize}
%---------------------------------------------------------------------------------------------------------------------------------------------
\begin{itemize}
    \item Inégalité isopérimétrique, théorème \ref{ThoIXyctPo}.
    \item Théorème des quatre sommets, théorème \ref{THOooFRBBooWKZcfY}.
\end{itemize}
%---------------------------------------------------------------------------------------------------------------------------------------------
\begin{itemize}
    \item Formule sommatoire de Poisson, proposition \ref{ProprPbkoQ}.
\end{itemize}
%---------------------------------------------------------------------------------------------------------------------------------------------
\paragraph{Sous-variétés de \( \eR^n\). Exemples.}
\begin{itemize}
    \item Extrema liés, théorème \ref{ThoRGJosS}.
    \item Théorème de Von Neumann \ref{ThoOBriEoe}.
    \item Lemme de Morse, lemme \ref{LemNQAmCLo}.
\end{itemize}
%---------------------------------------------------------------------------------------------------------------------------------------------
\paragraph{Suites et séries de fonctions intégrables. Exemples et applications.}
\begin{itemize}
    \item La proposition \ref{PropWoywYG} qui donne des indications sur la notion de classes dans \( L^p\).
    \item Le théorème de Weierstrass sur la limite uniforme de fonctions holomorphes, théorème \ref{ThoArYtQO}.
    \item Les théorèmes sur les fonctions définies par des intégrales, section \ref{SecCHwnBDj}.
    \item Théorème de Fischer-Riesz \ref{ThoGVmqOro}.
    \item Prolongement méromorphe de la fonction \( \Gamma\) d'Euler.
    \item Problème de la ruine du joueur, section \ref{SecMSOjfgM}.
\end{itemize}
%---------------------------------------------------------------------------------------------------------------------------------------------
\paragraph{Utilisation en probabilités du produit de convolution et de la transformation de Fourier ou de Laplace.}
\begin{itemize}
    \item Processus de Galton-Watson, lemme \ref{LemezrOiI} et théorème \ref{ThoJZnAOA}.
    \item Fonction caractéristique \ref{PropDerFnCaract}.
    \item Théorème central limite \ref{ThoOWodAi}.
\end{itemize}
\paragraph{Suites de variables de Bernoulli indépendantes.}
\begin{itemize}
    \item Processus de Galton-Watson, section \ref{SecBPmrPdtGalton}.
    \item Estimation des grands écarts, théorème \ref{ThoYYaBXkU}.
    \item Densité des polynômes dans \( C^0\big( \mathopen[ 0 , 1 \mathclose] \big)\), théorème de Bernstein \ref{ThoDJIvrty}.
    \item Problème de la ruine du joueur, section \ref{SecMSOjfgM}.
\end{itemize}
