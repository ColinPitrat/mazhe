% This is part of Mes notes de mathématique
% Copyright (c) 2011-2016
%   Laurent Claessens, Carlotta Donadello
% See the file fdl-1.3.txt for copying conditions.

%+++++++++++++++++++++++++++++++++++++++++++++++++++++++++++++++++++++++++++++++++++++++++++++++++++++++++++++++++++++++++++ 
\section{Intégrale par rapport à une mesure}
%+++++++++++++++++++++++++++++++++++++++++++++++++++++++++++++++++++++++++++++++++++++++++++++++++++++++++++++++++++++++++++

Nous avons besoin d'un peu de théorie de l'intégration parce que la définition de la mesure sur un espace mesurable\footnote{Théorème \ref{ThoWWAjXzi}.} produit passe par une intégrale.

Une mesure \( \mu\) sur un espace mesurable \( (\Omega,\tribA)\) permet de définir une fonctionnelle linéaire sur l'ensemble des fonctions mesurables \( \Omega\to \eR\). Cette fonctionnelle linéaire est l'intégrale que nous allons définir à présent.

\begin{definition}  \label{DefTVOooleEst}
    Soit \( (\Omega,\tribA,\mu)\) un espace mesuré. Si \( Y\in \tribA\) et si \( f\) est une fonction étagée\footnote{Définition \ref{DefBPCxdel}.}, si sa forme canonique est \( f=\sum_{i=1}^n\alpha_i\mtu_{A_i}\) alors nous définissons
    \begin{equation}        \label{EqooGAFMooZLzjPs}
        \int_Yfd\mu=\sum_i\alpha_i\mu(Y\cap A_i).
    \end{equation}
    Pour une fonction \( \tribA\)-mesurable générale \( f\colon \Omega\to \mathopen[ 0 , \infty \mathclose]\) nous définissons l'intégrale de \( f\) sur \( Y\) par
    \begin{equation}        \label{EqDefintYfdmu}
        \int_Yfd\mu=\sup\Big\{ \int_Yhd\mu\,\text{où } h\text{ est une fonction étagée telle que } 0\leq h\leq f \Big\}.
    \end{equation}

    Si $f$ est mesurable à valeurs dans \( \bar \eR\) ou \( \eC\), l'intégrale se définit séparément pour les parties positives, négatives, réelles et imaginaires.
\end{definition}

\begin{remark}
    Toute fonction mesurable à valeurs dans \( \bar \eR\) est intégrable (l'intégrale vaut éventuellement \( +\infty\)). Au moment où une fonction commence à prendre des valeurs positives et négatives, nous demandons à pouvoir intégrer séparément les parties positive et négative. C'est pour cela que nous disons qu'une fonction \( f\) à valeurs dans \( \eR\) est intégrable si \( | f |\) l'est.

    Cela est indépendant du fait que \( \int_0^{\infty}f\) en tant que limite de \( \int_0^{M}f\) peut très bien exister grâce à des compensations, alors que \( \int_{\mathopen[ 0 , \infty \mathclose[}| f |\) n'existe pas.
\end{remark}

\begin{lemma}       \label{LemooPJLNooVKrBhN}
    Si \( (\Omega,\tribA,\mu)\) est un espace mesuré et si \( B\in \tribA\) alors
    \begin{equation}
        \mu(B)=\int_B1d\mu=\int_{\Omega}\mtu_B.
    \end{equation}
\end{lemma}

\begin{proof}
    La fonction caractéristique d'une partie mesurable est une fonction étagée dont la forme canonique est \( \mtu_B=1\cdot \mtu_B+0\times \mtu_{B^c}\). Son intégrale est donc
    \begin{equation}
        \int\mtu_Bd\mu=1\times \mu(B)+0\times \mu(B^c)=\mu(B)
    \end{equation}
    parce que \( 0\times \mu(B^c)=0\), même si \( \mu(B^c)=\infty\), comme nous l'avons convenu en \ref{normooGAAJooUPCbzG}.
\end{proof}

\begin{definition}  \label{DefTCXooAstMYl}
    Une fonction \( f\colon \Omega\to \eR\) est dite \defe{intégrable}{intégrable} au sens de Lebesgue si \( \int_{\Omega}| f |<\infty\). Dans ce cas nous définissons
    \begin{equation}    \label{EqUHSooWfgUty}
        \int_{\Omega}f=\int_{\Omega}f^+-\int_{\Omega}f^-
    \end{equation}
    où \( f^+\) et \( f^-\) sont les parties positives et négatives de \( f\); les deux intégrales à droite dans \eqref{EqUHSooWfgUty} sont finies dès que \( f\) est intégrables.
\end{definition}
Nous verrons comment donner un sens à \( \int_{\Omega}f\) dans certains cas où \( f\) n'est pas intégrable sur \( \Omega\) dans la section \ref{SecGAVooBOQddU} sur les intégrales impropres.

Nous définissons aussi
\begin{equation}
    \mu(f)=\int_{\Omega}f
\end{equation}
si \( f\) est une fonction mesurable sur \( \Omega\).

\begin{remark}
    Dans \( \eR^d\), quasiment toutes les fonctions et ensembles sont mesurables. En effet la construction d'ensembles non mesurables demande obligatoirement l'utilisation de l'axiome du choix; de tels ensembles doivent être construits «exprès pour». Il y a très peu de chances pour que vous tombiez sur un ensemble non mesurable de \( \eR^d\) sans que vous ne vous en rendiez compte.
\end{remark}

\begin{remark}
    «Mesurable» ne signifie pas «intégrable». Par exemple la fonction 
    \begin{equation}
        \begin{aligned}
            f\colon \eR&\to \bar\eR \\
            \omega&\mapsto\begin{cases}
            \frac{1}{ \omega }    &   \text{si } \omega\neq 0\\
            \infty    &    \text{if}\omega=0.
            \end{cases}
        \end{aligned}
    \end{equation}
    est mesurable, mais non intégrable.
\end{remark}

%--------------------------------------------------------------------------------------------------------------------------- 
\subsection{Quelque propriétés}
%---------------------------------------------------------------------------------------------------------------------------

\begin{theorem}[\cite{MesureLebesgueLi}]        \label{ThoooCZCXooVvNcFD}
    Soient \( f,g\) des fonctions étagées positives sur \( (S,\tribF,\mu)\). Alors si \( \alpha\in\mathopen[ 0 , \infty \mathclose]\) nous avons
    \begin{enumerate}
        \item
            \begin{equation}
                \int_S(\alpha f)d\mu=\alpha\int_Sfd\mu.
            \end{equation}
        \item       \label{ITEMooBLEVooDznQTY}
            \begin{equation}
                \int_S(f+g)d\mu=\int_Sfd\mu+\int_Sgd\mu.
            \end{equation}
        \item\label{ITEMooOJRAooQkoQyD}
    Si \( a_k\in \eR^+\) et si les \( f_k\) sont étagées positives,
    \begin{equation}
        \int_S\left( \sum_{k=1}^na_kf_k \right)=\sum_{k=1}^na_k\left( \int_S f_kd\mu \right).
    \end{equation}
    \end{enumerate}
\end{theorem}

\begin{proof}
    En ce qui concerne le produit par un nombre, tout repose sur le fait que
    \begin{equation}
        (\alpha f)^{-1}(\alpha a_i)=f^{-1}(a_i), 
    \end{equation}
    ce qui fait que si la représentation canonique de \( f\) est \( f=\sum_ia_i\mtu_{A_i}\) alors la représentation canonique de \( \alpha f\) est \( \alpha f=\sum_i(\alpha a_i)\mtu_{A_i}\). Donc
    \begin{equation}
        \int_S\alpha fd\mu=\sum_i\alpha a_i\mu(A_i)=\alpha \sum_ia_i\mu(A_i)=\alpha\int_Sfd\mu.
    \end{equation}
    
    Pour la somme c'est plus lourd. Soient les formes canoniques
    \begin{subequations}
        \begin{align}
            f&=\sum_ia_i\mtu_{A_i}\\
            g&=\sum_jb_j\mtu_{B_i}.
        \end{align}
    \end{subequations}
    Vu que l'union des \( B_j\) est \( S\) nous avons l'union disjointe \( A_i=\bigcup_jA_i\cap B_j\) et donc \( \mu(A_i)=\sum_j\mu(A_i\cap B_j)\). Nous avons donc pour les intégrales :
    \begin{subequations}
        \begin{align}
            \int_Sfd\mu&=\sum_ia_i\sum_j\mu(A_i\cap B_j)\\
            \int_Sgd\mu&=\sum_ib_k\sum_l\mu(B_k\cap A_l).
        \end{align}
    \end{subequations}
    Pour la somme :
    \begin{equation}
        \int_Sfd\mu+\int_Sgd\mu=\sum_{k,l}(a_k+b_l)\mu(A_k\cap B_l).
    \end{equation}

    Nous devons maintenant évaluer \( \int_S(f+g)d\mu\). Pour cela nous remarquons que si \( c\in (f+g)(S)\) (l'ensemble des valeurs atteintes pas \( f+g\)), alors nous notons
    \begin{equation}
        I_c=\{ (k,l)\tq a_k+b_l=c \}
    \end{equation}
    et nous avons
    \begin{equation}
        \{ f+g=c \}=\bigcup_{(k,l)\in I_c}(A_k\cap B_l),
    \end{equation}
    et comme cette union est disjointe, nous pouvons faire la somme des mesures :
    \begin{equation}
        \mu(f+g=c)=\sum_{(k,l)\in I_c}\mu(A_k\cap B_l).
    \end{equation}
    Cela nous permet de faire le calcul suivant :
    \begin{subequations}
        \begin{align}
            \int_S(f+g)d\mu&=\sum_{c\in (f+g)(S)}c\mu(f+g=c)\\
            &=\sum_{c\in(f+g)(S)}c\sum_{(k,l)\in I_c}\mu(A_k\cap B_l)\\
            &=\sum_{c\in(f+g)(S)}\sum_{(k,l)\in I_c} (a_k+b_l) \mu(A_k\cap B_l)
        \end{align}
    \end{subequations}
    Dans cette double somme, tous les couples \( (k,l)\) sont tirés une et une seule fois parce qu'ils sont tous dans un et un seul des \( I_c\), donc
    \begin{subequations}
        \begin{align}
            \int_S(f+g)d\mu&= \sum_{c\in(f+g)(S)}\sum_{(k,l)\in I_c} (a_k+b_l) \mu(A_k\cap B_l)\\
            &=\sum_{(k,l)}(a_k+b_l)\mu(A_k\cap B_l)\\
            &=\int_Sfd\mu+\int_Sgd\mu.
        \end{align}
    \end{subequations}
\end{proof}

\begin{remark}
    Si \( f=\sum_ka_k\mtu_{A_k}\) n'est pas une décomposition canonique, il n'en reste pas moins que chacun des \( \mtu_{A_k}\) est la forme canonique de lui-même. Donc le théorème \ref{ThoooCZCXooVvNcFD} s'applique et nous avons quand même
    \begin{equation}
        \int_Sfd\mu=\sum_ka_k\mu(A_k).
    \end{equation}
\end{remark}

\begin{proposition}
    Si \( f,g\) sont des fonctions étagées et si \( A,B\subset S\) sont disjoints, alors
    \begin{enumerate}
        \item
            \( \int_Afd\mu=\int_Sf\mtu_Ad\mu\)
        \item
            \( \int_{A\cup B}fd\mu=\int_Afd\mu+\int_Bfd\mu\).
    \end{enumerate}
\end{proposition}

\begin{proof}
    Si \( f=\sum_ka_k\mtu_{B_k}\) alors d'une part
    \begin{equation}
        \int_Afd\mu=\sum_ka_k\mu(B_k\cap A)
    \end{equation}
    et d'autre part,
    \begin{equation}
        f\mtu_A)d=\sum_ka_k\mtu_A\mtu_{B_k}=\sum_ka_k\mtu_{B_k\cap A},
    \end{equation}
    ce qui donne
    \begin{equation}
        \int f\mtu_Ad\mu=\sum_ka_k\mu(B_k\cap A).
    \end{equation}

    En ce qui concerne la seconde égalité à prouver, tout repose sur le fait que \( \mtu_{A\cup B=\mtu_A+\mtu_B}\). Du coup nous avons, en utilisant le théorème \ref{ThoooCZCXooVvNcFD} :
    \begin{subequations}
        \begin{align}
            \int_{A\cup B}fd\mu&=\int_Sf\mtu_{A\cup B}d\mu\\
            &=\int_Sf(\mtu_A+\mtu_B)d\mu\\
            &=\int_Sf\mtu_A+\int_Sf\mtu_B\\
            &=\int_Af+\int_Bf.
        \end{align}
    \end{subequations}
\end{proof}

\begin{proposition}     \label{PropOPSCooVpzaBt}
    Si \( A,B\subset \Omega\) sont des ensembles disjoints et si \( f\) est intégrable sur \( A\cup B\) alors
    \begin{equation}
        \int_{A\cup B}f=\int_Af+\int_Bf.
    \end{equation}
\end{proposition}

Le lemme suivant nous aide à détecter des fonctions presque partout nulles.
\begin{lemma}   \label{Lemfobnwt}
    Soit \( f\) une fonction mesurable positive ou nulle telle que
    \begin{equation}
        \int_{\Omega}fd\mu=0.
    \end{equation}
    Alors \( f=0\) \( \mu\)-presque partout.
\end{lemma}

\begin{proof}
    L'ensemble des points \( x\in\Omega\) tels que \( f(x)\neq 0\) peut s'écrire comme une union dénombrable disjointe :
    \begin{equation}
        \{ x\in\Omega\tq f(x)\neq 0 \}=\bigcup_{i=0}^{\infty}E_i
    \end{equation}
    avec
    \begin{subequations}
        \begin{align}
            E_0&=\{ x\in\Omega\tq f(x)>1 \}\\
            E_i&=\{ x\in\Omega\tq \frac{1}{ i+1 }\leq f(x)<\frac{1}{ i } \}.
        \end{align}
    \end{subequations}
    Si un des ensembles \( E_i\) est de mesure non nulle, alors nous pouvons considérer la fonction simple \( h(x)=\frac{1}{ i+1 }\mtu_{E_i}\) dont l'intégrale sur \( \Omega\) est strictement positive. Par conséquent le supremum de la définition \eqref{EqDefintYfdmu} est strictement positif.

    Nous savons donc que \( \mu(E_i)=0\) pour tout \( i\). Étant donné que la mesure d'une union disjointe dénombrable est égale à la somme des mesures, nous avons
    \begin{equation}
        \mu\{ x\in\Omega\tq f(x)\neq 0 \}=0,
    \end{equation}
    ce qui signifie que \( f\) est nulle \( \mu\)-presque partout.
\end{proof}

\begin{corollary}   \label{CorjLYiSm}
    Soit \( f\) une fonction mesurable sur l'espace mesuré \( (\Omega,\tribA,\mu)\) telle que
    \begin{equation}
        \int_{\Omega}f\mtu_{f>0}d\mu=0.
    \end{equation}
    Alors \( f\leq 0\) presque partout.
\end{corollary}

\begin{proof}
    Nous avons l'égalité d'ensembles
    \begin{equation}
        \{ f\mtu_{f>0}\neq 0 \}=\{ \mtu_{f>0}\neq 0 \}.
    \end{equation}
    Mais lemme \ref{Lemfobnwt} implique que \( f\mtu_{f>0}\) est nulle presque partout, c'est à dire que la mesure de l'ensemble du membre de gauche est nulle par conséquent
    \begin{equation}
        \mu\{ \mtu_{f>0}\neq 0 \}=0.
    \end{equation}
    Cela signifie que la fonction \( f\) est presque partout négative ou nulle.
\end{proof}

\begin{lemma}   \label{LemPfHgal}
    Soit \( f\) une fonction telle que \( | f(x)|\leq g(x) \) pour tout \( x\in\Omega\). Si \( g\) est intégrable, alors \( f\) est intégrable.
\end{lemma}

\begin{proof}
    Nous décomposons \( f\) en parties positives et négatives :
    \begin{subequations}
        \begin{align}
            A_+&=\{ x\in\Omega\tq f(x)>0 \}\\
            A_-&=\{ x\in\Omega\tq f(x)<0 \}.
        \end{align}
    \end{subequations}
    Nous posons \( f_+(x)=f(x)\mtu_{A_+}\) et \( f_-(x)=f(x)\mtu_{A_-}\). Nous avons \( f=f_+-f_-\) et
    \begin{equation}
        \int_{\Omega}f=\int_{A_+}f+\int_{A_-}f
    \end{equation}
    parce que \( \Omega=A_+\cup A_-\cup\{ x\in\Omega\tq f(x)=0 \}\). Si \( \varphi\) est une fonction simple qui majore \( f_+\) nous avons
    \begin{equation}
        \varphi(x)=\sum_{k}a_k\mtu_{E_k}(x)\leq f(x)\mtu_{A_+}(x)\leq g(x).
    \end{equation}
    Par conséquent le supremum qui définit \( \int f_+\) est inférieur au supremum qui définit \( \int g\). La fonction \( f_+\) est donc intégrable. La même chose est valable pour la fonction \( f_-\).
\end{proof}

%--------------------------------------------------------------------------------------------------------------------------- 
\subsection{Permuter limite et intégrale}
%---------------------------------------------------------------------------------------------------------------------------

%--------------------------------------------------------------------------------------------------------------------------- 
\subsubsection{Convergence uniforme}
%---------------------------------------------------------------------------------------------------------------------------

\begin{proposition}[Permuter limite et intégrale]       \label{PropbhKnth}
    Soit \( f_n\to f\) uniformément sur un ensemble mesuré \( A\) de mesure finie. Alors si les fonctions \( f_n\) et \( f\) sont intégrables sur \( A\), nous avons
    \begin{equation}
        \lim_{n\to \infty} \int_A f_n=\int_A \lim_{n\to \infty} f_n.
    \end{equation}
\end{proposition}

\begin{proof}
    Notons \( f\) la limite de la suite \( (f_n)\). Pour tout \( n\) nous avons les majorations
    \begin{subequations}
        \begin{align}
            \left| \int_A f_n d\mu-\int_A fd\mu \right| &\leq \int_A| f_n-f |d\mu\\
            &\leq \int_A \| f_n-f \|_{\infty}d\mu\\
            &=\mu(A)\| f_n-f \|_{\infty}
        \end{align}
    \end{subequations}
    où \( \mu(A)\) est la mesure de \( A\). Le résultat découle maintenant du fait que \( \| f_n-f \|_{\infty}\to 0\).
\end{proof}
Il existe un résultat considérablement plus intéressant que cette proposition. En effet, l'intégrabilité de \( f\) n'est pas nécessaire. Cette hypothèse peut être remplacée soit par l'uniforme convergence de la suite (théorème \ref{ThoUnifCvIntRiem}), soit par le fait que les normes des \( f_n\) sont uniformément bornées (théorème de la convergence dominée de Lebesgue \ref{ThoConvDomLebVdhsTf}).

\begin{theorem}[\cite{BJblWiS}]			\label{ThoUnifCvIntRiem}
    La limite uniforme d'une suite de fonctions intégrables sur un borné est intégrable, et on peut permuter la limite et l'intégrale. 
    
    Plus précisément, soit \( A\) un ensemble de \( \mu\)-mesure finie et \( f_n\colon A\to \eR\) des fonctions intégrables sur \( A\). Si la limite \( f_n\to f\) est uniforme, alors \( f\) est intégrable sur \( A\) et nous pouvons inverser la limite et l'intégrale :
    \begin{equation}
        \lim_{n\to \infty} \int_A f_n=\int_A\lim_{n\to \infty} f_n.
    \end{equation}
\end{theorem}

\begin{proof}
    Soit \( \epsilon>0\) et \( n\) tel que \( \| f_n-f \|_{\infty}\leq \epsilon\) (ici la norme uniforme est prise sur \( A\)). Étant donné que \( f_n\) est intégrable sur \( A\), il existe une fonction simple \( \varphi_n\) qui minore \( f_n\) telle que
    \begin{equation}
        \left| \int_{A}\varphi_n-\int_A f_n \right| <\epsilon.
    \end{equation}
    La fonction \( \varphi_n+\epsilon\) est une fonction simple qui majore la fonction \( f\). Si \( \psi\) est une fonction simple qui minore \( f\), alors
    \begin{equation}
        \int_A\psi\leq\int_A\varphi_n+\epsilon\leq\int_A f_n+\epsilon\mu(A).
    \end{equation}
    Par conséquent le supremum qui définit \( \int_A f\) existe, ce qui montre que \( f\) est intégrable. Le fait qu'on puisse inverser la limite et l'intégrale est maintenant une conséquence de la proposition \ref{PropbhKnth}.
\end{proof}

\begin{remark}
    L'hypothèse sur le fait que \( A\) soit de mesure finie est importante. Il n'est pas vrai qu'une suite uniformément convergente de fonctions intégrables est intégrables. En effet nous avons par exemple la suite
    \begin{equation}
        f_n(x)=\begin{cases}
            1/x    &   \text{si } x<n\\
            0    &    \text{sinon}
        \end{cases}
    \end{equation}
    qui converge uniformément vers \( f(x)=1/x\) sur \( A=\mathopen[ 1 , \infty [\). Le limite n'est cependant guerre intégrable sur \( A\).
\end{remark}

%---------------------------------------------------------------------------------------------------------------------------
\subsubsection{Convergence monotone}
%---------------------------------------------------------------------------------------------------------------------------

\begin{theorem}[Théorème de la convergence monotone ou de Beppo-Levi\cite{mathmecaChoi}] \label{ThoRRDooFUvEAN}
    Soit un espace mesuré \( (\Omega,\tribA,\mu)\) et \( (f_n)\) une suite croissante de fonctions mesurables à valeurs dans \( \mathopen[ 0 , \infty \mathclose]\). Alors la limite ponctuelle \( \lim_{n\to \infty} f_n\) existe, est mesurable et
    \begin{equation}    \label{EqFHqCmLV}
        \lim_{n\to \infty} \int_{\Omega}f_nd\mu= \int_{\Omega}\lim_{n\to \infty} f_nd\mu,
    \end{equation}
    cette intégrable valant éventuellement \( \infty\).
\end{theorem}
\index{théorème!convergence!monotone}
\index{théorème!Beppo-Levi}
\index{permuter!limite et intégrale!convergence monotone}

\begin{proof}
    La limite ponctuelle de la suite est la fonction à valeurs dans \( \mathopen[ 0 , \infty \mathclose]\) donnée par
    \begin{equation}
        f(x)=\lim_{n\to \infty} f_n(x).
    \end{equation}
    Ces limites existent parce que pour chaque \( x\) la suite \( f_n(x)\) est une suite numérique croissante. Nous notons
    \begin{equation}
        I_0=\int_{\Omega}fd\mu.
    \end{equation}
    Nous posons par ailleurs
    \begin{equation}
        I_n=\int_{\Omega}f_n.
    \end{equation}
    Cela est une suite numérique croissante qui a par conséquent une limite que nous notons \( I=\lim_{n\to \infty} I_n\). Notre objectif est de montrer que \( I=I_0\). D'abord par croissance de la suite, pour tous $n$ nous avons \( I_n\leq I_0\), par conséquent \( I\leq I_0\).

    Nous prouvons maintenant l'inégalité dans l'autre sens en nous servant de la définition \eqref{EqDefintYfdmu}. Soit une fonction simple \( h\) telle que \( h\leq f\), et une constante \( 0<C<1\). Nous considérons les ensembles
    \begin{equation}
        E_n=\{ x\in\Omega\tq f_n(x)\geq Ch(x) \}.
    \end{equation}
    Ces ensembles vérifient les propriétés \( E_n\subset E_{n+1}\) et \( \bigcup_{n=1}^{\infty}E_n=\Omega\). Pour chaque \( n\) nous avons les inégalités
    \begin{equation}
        \int_{\Omega}f_n\geq\int_{E_n}f_n\geq C\int_{E_n}h.
    \end{equation}
    Si nous prenons la limite \( n\to\infty\) dans ces inégalités,
    \begin{equation}
        \lim_{n\to \infty} \int_{\Omega}f_n\geq C\lim_{n\to \infty} \int_{E_n}h=C\int_{\Omega}h.
    \end{equation}
    Par conséquent \( \lim_{n\to \infty} \int f_n\geq C\int_{\Omega}h\). Mais étant donné que cette inégalité est valable pour tout \( C\) entre \( 0\) et \( 1\), nous pouvons l'écrire sans le \( C\) :
    \begin{equation}        \label{EqzAKEaU}
        \lim_{n\to \infty} \int_{\Omega}f_n\geq \int_{\Omega}h.
    \end{equation}
    Par définition, l'intégrale de \( f\) est donné par le supremum des intégrales de \( h\) où \( h\) est une fonction simple dominée par \( f\). En prenant le supremum sur \( h\) dans l'équation \eqref{EqzAKEaU} nous avons
    \begin{equation}
        \lim_{n\to \infty} \int_{\Omega}f_n\geq\int_{\Omega}f,
    \end{equation}
    ce qu'il nous fallait.
\end{proof}

\begin{remark}
    La proposition \ref{THOooXHIVooKUddLi} ainsi que le lemme \ref{LemYFoWqmS} montrent qu'une fonction mesurable peut-être écrite comme limite croissante de fonctions simples. Cela permet de démontrer des théorèmes en commençant par prouver sur les fonctions simples et en utilisant Beppo-Levi pour généraliser.
\end{remark}

\begin{remark}
    Une des raisons de demander la positivité des fonctions \( f_n\) est de n'avoir pas d'ambiguïté à parler d'intégrales qui valent \( \infty\). Si par exemple nous prenons \( \Omega=\mathopen[ 0 , 1 \mathclose]\) et que nous considérons
    \begin{equation}
        f_n(x)=\begin{cases}
            0    &   \text{si } x\leq \frac{1}{ n }\\
            \frac{1}{ x }    &    \text{sinon}.
        \end{cases}
    \end{equation}
    Ce sont des fonctions intégrables, mais la limite étant la fonction \( 1/x\), l'égalité \eqref{EqFHqCmLV} est une égalité entre deux intégrales valant \( \infty\).
\end{remark}

\begin{corollary}[Inversion de somme et intégrales] \label{CorNKXwhdz}
    Si \( (u_n)\) est une suite de fonctions mesurables positives ou nulles, alors
    \begin{equation}
        \sum_{i=0}^{\infty}\int u_i=\int\sum_{i=0}^{\infty}u_i.
    \end{equation}
\end{corollary}
\index{permuter!somme et intégrale}

\begin{proof}
    Nous considérons la suite des sommes partielles de \( (u_n)\) : \( f_n(x)=\sum_{i=0}^nu_n(x)\). Le théorème de la convergence monotone (théorème \ref{ThoRRDooFUvEAN}) implique que
    \begin{equation}
        \lim_{n\to \infty} \int f_n=\int\lim_{n\to \infty} f_n.
    \end{equation}
    Nous remplaçons maintenant \( f_n\) par sa valeur en termes des \( u_i\) et dans le membre de gauche nous permutons l'intégrale avec la somme finie :
    \begin{equation}
        \lim_{n\to \infty} \sum_{i=0}^{\infty}\int u_n=\int\sum_{i=0}^{\infty}u_n,
    \end{equation}
    ce qu'il fallait démontrer.
\end{proof}

\begin{lemma}[Lemme de Fatou]\index{lemme!Fatou}\index{Fatou}   \label{LemFatouUOQqyk}
    Soit \( (\Omega,\tribA,\mu)\) un espace mesuré et \( f_n\colon \Omega\to \mathopen[ 0 , \infty \mathclose]  \) une suite de fonctions mesurables. Alors la fonction \( f(x)=\liminf f_n(x)\) est mesurable et
    \begin{equation}
        \int_{\Omega}\liminf f_nd\mu\leq\liminf\int_{\Omega}fd\mu.
    \end{equation}
\end{lemma}
%TODO : pour la mesurabilité, il faudra citer un théorème du genre de celui fait avec le sup.

\begin{proof}
    Nous posons 
    \begin{equation}
        g_n(x)=\inf_{i\geq n}f_i(x).
    \end{equation}
    Cela est une suite croissance de fonctions positives mesurables telles que, par définition, 
    \begin{equation}
        \lim_{n\to \infty}g_n(x)=\liminf f_n(x).
    \end{equation}
    Nous pouvons y appliquer le théorème de la convergence monotone,
    \begin{equation}
        \lim_{n\to \infty} \int g_n(x)=\int\liminf f_n(x).
    \end{equation}
    Par ailleurs, pour chaque \( i\geq n\) nous avons
    \begin{equation}
        \int g_n\leq \int f_i,
    \end{equation}
    en passant à l'infimum nous avons
    \begin{equation}
        \int g_n\leq \inf_{i\geq n}\int f_i,
    \end{equation}
    et en passant à la limite nous avons
    \begin{equation}
        \int\liminf f_n=\lim_{n\to \infty} \int g_n\leq \lim_{n\to \infty} \inf_{i\geq n}\int f_i=\liminf_{i\to\infty}\inf f_i.
    \end{equation}
\end{proof}

L'inégalité donnée dans ce lemme n'est en général pas une égalité, comme le montre l'exemple suivant :
\begin{equation}
    f_i=\begin{cases}
        \mtu_{\mathopen[ 0 , 1 \mathclose]}    &   \text{si } i\text{ est pair}\\
        \mtu_{\mathopen[ 1 , 2 \mathclose]}    &    \text{si } i\text{ est impair}.
    \end{cases}
\end{equation}
Nous avons évidemment \( g_n(x)=0\) tandis que \( \int_{\mathopen[ 0 , 2 \mathclose]}f_i=1\) pour tout \( i\).

%---------------------------------------------------------------------------------------------------------------------------
\subsubsection{Convergence dominée de Lebesgue}
%---------------------------------------------------------------------------------------------------------------------------

\begin{theorem}[Convergence dominée de Lebesgue]        \label{ThoConvDomLebVdhsTf}
    Soit \( (f_n)_{n\in\eN}\) une suite de fonctions intégrables sur \( (\Omega,\tribA,\mu)\) à valeurs dans \( \eC\) ou \( \eR\). Nous supposons que  \( f_n\to f\) simplement sur \( \Omega\) presque partout et qu'il existe une fonction intégrable \( g\) telle que
    \begin{equation}
        | f_n(x) | \leq g(x) 
    \end{equation}
    pour presque\footnote{S'il n'y avait pas le «presque» ici, ce théorème serait à peu près inutilisable en probabilité ou en théorie des espaces \( L^p\), comme dans la démonstration du théorème de Fischer-Riesz \ref{ThoGVmqOro} par exemple.} tout \( x\in\Omega\) et pour tout \( n\in \eN\). Alors
    \begin{enumerate}
        \item
            \( f\) est intégrable,
        \item
           $\lim_{n\to \infty} \int_{\Omega}f_n=\int_\Omega f$,
        \item
            $\lim_{n\to \infty} \int_{\Omega}| f_n-f |=0$.
    \end{enumerate}
\end{theorem}
\index{théorème!convergence!dominée de Lebesgue}
\index{dominée!convergence (Lebesgue)}
\index{permuter!limite et intégrale!convergence dominée}

\begin{proof}

    La fonction limite \( f\) est intégrable parce que \( | f |\leq g\) et \( g\) est intégrable (lemme \ref{LemPfHgal}). Par hypothèse nous avons
    \begin{equation}
        -g(x)\leq f_n(x)\leq g(x).
    \end{equation}
    En particulier la fonction \( g_n=f_n+g\) est positive et mesurable si bien que le lemme de Fatou \ref{LemFatouUOQqyk} implique
    \begin{equation}
        \int_{\Omega}\liminf g_n\leq\liminf\int_{\Omega}g_n.
    \end{equation}
    Évidement nous avons \( \liminf g_n=f+g\), de telle sorte que
    \begin{equation}
        \int f+\int g\leq \liminf\int g_n=\liminf\int f_n+\int g,
    \end{equation}
    et le nombre \( \int g\) étant fini, nous pouvons le retrancher des deux côtés de l'inégalité :
    \begin{equation}
        \int f\leq\liminf\int f_n.
    \end{equation}
    Afin d'obtenir une minoration de \( \int f\) nous refaisons exactement le même raisonnement en utilisant la suite de fonctions \( k_n=-f_n\to k=-f\). Nous obtenons que
    \begin{equation}
        \int k\geq\liminf\int k_n=-\limsup\int f_n,
    \end{equation}
    et par conséquent
    \begin{equation}    \label{IneqsndMYTO}
        \liminf\int f_n\leq\int f\leq\limsup\int f_n.
    \end{equation}
    La limite supérieure étant plus grande ou égale à la limite inférieure, les trois quantités dans les inégalités \eqref{IneqsndMYTO} sont égales.

    Nous prouvons maintenant le troisième point. Soit la suite de fonctions
    \begin{equation}
        h_n(x)=| f_n(x)-f(x) |
    \end{equation}
    qui tend ponctuellement vers zéro. De plus
    \begin{equation}
    h_n(x)\leq | f_n(x) |+| f(x) |\leq 2g(x),
    \end{equation}
    ce qui prouve que les \( h_n\) majorés par une fonction intégrable. Donc
    \begin{equation}
        \lim_{n\to \infty} \int_{\Omega}| f_n-f |= \lim_{n\to \infty} \int_{\Omega}h_n(x)dx=\int_{\Omega}\lim_{n\to \infty} | f_n(x)-f(x) |=0
    \end{equation}
\end{proof}

\begin{remark}
    Lorsque nous travaillons sur des problèmes de probabilités, la fonction \( g\) peut être une constante parce que les constantes sont intégrables sur un espace de probabilité.
\end{remark}

\begin{corollary}       \label{CorCvAbsNormwEZdRc}
    Soit \( (a_i)_{i\in \eN}\) une suite numérique absolument convergente. Alors elle est convergente. Il en est de même pour les séries de fonctions si on considère la convergence ponctuelle.
\end{corollary}

\begin{proof}
    L'hypothèse est la convergence de l'intégrale \( \int_{\eN}| a_i |dm(i)\) où \( dm\) est la mesure de comptage. Étant donné que \( | a_i |\leq | a_i |\), la fonction \( a_i\) (fonction de \( i\)) peut jouer le rôle de \( g\) dans le théorème de la convergence dominée de Lebesgue (théorème \ref{ThoConvDomLebVdhsTf}).
\end{proof}

%--------------------------------------------------------------------------------------------------------------------------- 
\subsection{Produit d'une mesure par une fonction (mesure à densité)}
%---------------------------------------------------------------------------------------------------------------------------

\begin{propositionDef}[\cite{MonCerveau,ooGMNAooSLnIio}]\label{PropooVXPMooGSkyBo}
    Soit un espace mesuré \( (S,\tribF,\mu)\) et une fonction mesurable positive \( w\colon S\to \bar\eR^+\). Alors la formule
    \begin{equation}
        (w\cdot \mu)(A)=\int_Awd\mu
    \end{equation}
    pour tout \( A\in \tribF\) définit une mesure positive sur \( (S,\tribF)\) appelée \defe{produit}{produit!d'une mesure par une fonction} de la mesure \( \mu\) par la fonction \( w\). La fonction \( w\) est la \defe{densité}{densité!mesure} de la mesure \( w\cdot \mu\) par rapport à la mesure \( \mu\).
\end{propositionDef}

\begin{proof}
    D'abord \( (w\cdot \mu)(\emptyset)=0\) parce que le lemme \ref{LemooPJLNooVKrBhN} donne
    \begin{equation}
        (w\cdot \mu)(\emptyset)=\int_Sw\mtu_{\emptyset}d\mu=\int_S0d\mu=0\times \mu(S)=0
    \end{equation}
    où nous avons (éventuellement) utilisé deux fois la convention \( 0\times \infty=0\).


    Ensuite si les ensembles \( A_i\) sont des éléments deux à deux disjoints de \( \tribF\) alors nous avons \( \mtu_{\bigcup_{i=1}^{\infty}}=\sum_{i=1}^{\infty}\mtu_{A_i}\), et donc
    \begin{equation}
        (w\cdot \mu)(\bigcup_iA_i)=\int_Sw\mtu_{\bigcup_iA_i}d\mu=\int_S\big( \sum_{i=1}^{\infty}w\mtu_{A_i} \big)d\mu=\sum_i\int_Sw\mtu_{A_i}d\mu=\sum_i(w\cdot\mu)(A_i).
    \end{equation}
    Dans ce calcul nous avons utilisé le fait que \( f\) était positive pour justifier l'application du théorème de la convergence monotone \ref{ThoRRDooFUvEAN}.
\end{proof}

En particulier nous parlons souvent de mesure à densité par rapport à la mesure de Lebesgue. C'est alors la construction suivante. Si \( \mu\) est une mesure sur \( \eR^d\), une fonction \( f\colon \eR^d\to \eR\) est une \defe{densité}{densité d'une mesure} si pour tout \( A\subset\eR^d\) nous avons
\begin{equation}
    \mu(A)=\int_Af(x)dx
\end{equation}
où \( dx\) est la mesure de Lebesgue.

\begin{proposition}[\cite{ooGMNAooSLnIio}]  \label{PropooJMWAooDzfpmB}
    Soit une fonction mesurable \( w\colon (S,\tribF,\mu)\to \bar \eR^+\).
    \begin{enumerate}
        \item
            Si $f\colon S\to \bar\eR^+$ est mesurable, alors \( f\cdot(w\cdot \mu)=(fg)\cdot \mu\).
        \item
            Si \( f\colon S\to \bar \eR\) ou \( \eC\) est mesurable, elle est \( w\cdot\mu\)-intégrable si et seulement si \( fw\) est \( \mu\)-intégrable. Dans ce cas, nous avons encore \( f\cdot(w\cdot \mu)=(fg)\cdot\mu\).
    \end{enumerate}
    Attention : dans le cas où \( f\) est à valeurs dans \( \eC\), alors il faut que \( w\) soit à valeurs finies dans \( \eR\) parce que nous n'avons pas définit \( \infty\times z\) lorsque \( z\in \eC\).
\end{proposition}

\begin{proof}
    Nous commençons par prouver le résultat pour la fonction caractéristique de l'ensemble mesurable \( A\). Nous avons : $\mtu_A\cdot(w\cdot \mu)(B)=\int_B\mtu_Ad(w\cdot \mu)$. Mais par définition, l'intégrale d'une fonction indicatrice est la mesure de l'ensemble indiqué. En passant sur le fait que \( \mtu_A\mtu_B=\mtu_{A\cap B}\), 
    \begin{equation}
        \int_B\mtu_Ad(w\cdot \mu)=   (w\cdot\mu)(A\cap B)=\int_S\mtu_{A\cap B}wd\mu=\int_S\mtu_A\mtu_Bwd\mu=\int_B\mtu_Awd\mu=(\mtu_Aw)\cdot\mu(B).
    \end{equation}

    Supposons maintenant que \( f\) soit une fonction étagées qui s'écrit \( f=\sum_ka_k\mtu_{A_k}\) où les \( A_k\) sont des ensembles mesurables disjoints. Alors le calcul est le suivant, en utilisant le fait que sur \( A_k\), on a \( a_k=f(x)\) :
    \begin{subequations}
        \begin{align}
            f\cdot(g\cdot \mu)B&=\int_Bfd(g\cdot \mu)\\
            &=\sum_ka_k(g\cdot\mu)(A_k\cap B)\\
            &=\sum_ka_k\int_{A_k\cap B}gf\mu\\
            &=\int_{A_k\cap}f(x)g(x)d\mu(x)\\
            &=\sum_k(fg\cdot\mu)(A_k\cap B)\\
            &=(fg\cdot\mu)(B)
        \end{align}
    \end{subequations}
    parce que les \( A_k\cap B\) forment une partition de l'ensemble \( B\) (voir le point \ref{ItemQFjtOjXiii} de la définition \ref{DefBTsgznn}).

    Si \( f\colon S\to \bar\eR^+\) est mesurable, le théorème \ref{THOooXHIVooKUddLi} donne une suite croissante \( f_n\) de fonctions étagées positives convergeant (ponctuellement) vers \( f\). Vu que la fonction \( w\) est positive, nous avons aussi la limite positive et croissante \( wf_n\to wf\). Ainsi l'utilisation du théorème de la convergence monotone est justifié dans le calcul suivant :
    \begin{equation}
        \int_Sfd(w\cdot \mu)=\lim_{n\to \infty} \int_Sf_nd(w\cdot\mu)=\lim_{n\to \infty} \int_S(wf_n)d\mu=\int_Swfd\mu.
    \end{equation}
    
    Nous passons maintenant au cas général où \( f\) est une fonction à valeurs dans \( \bar\eR\) ou \( \eC\) (avec \( w\) finie dans ce dernier cas). Nous avons la chaîne d'équivalences 
    %\begin{itemize}{$\Leftrightarrow$}
    \begin{itemize}
            \renewcommand{\labelitemi}{$\Leftrightarrow$}
        \item \( f\) est \( (w\cdot\mu)\) intégrable
        \item \( | f |\) est \( (w\cdot\mu)\)-intégrable
        \item \( | f |w\) est \( \mu\)-intégrable
        \item \( | fw |\) est \( \mu\)-intégrable.
    \end{itemize}

    Si cela est le cas, la formule se démontre en se ramenant au cas déjà prouvé des fonctions positives en utilisant les \( (fw)^+=f^+w\), \( (fw)^-=f^-w\) etc.
\end{proof}

\begin{example}[Un compact n'est pas toujours de mesure finie]      \label{EXooKQDRooVMWaEC}
    Soit l'espace mesurable \( (\eR,\Borelien(\eR))\) réel avec ses boréliens et la fonction
    \begin{equation}
        \begin{aligned}
            w\colon \big( \eR,\Borelien(\eR) \big)&\to\big( \bar \eR,\Borelien(\bar \eR) \big) \\
            x&\mapsto \begin{cases}
                \frac{1}{ | x | }    &   \text{si } x\neq 0\\
                +\infty    &    \text{si }x=0.
            \end{cases}
        \end{aligned}
    \end{equation}
    Essayons d'étudier la mesure de densité \( w\) par rapport à la mesure de Lebesgue.
    \begin{subproof}
    \item[\( w\) est mesurable]
    Soit un borélien \( B\) de \( \bar \eR\). Si \( B\) ne contient pas \( \infty\) alors \( w^{-1}(B)\) est un borélien de \( \eR\) par continuité de l'application restreinte \( w\colon \eR\setminus\{ 0 \}\to \eR \). Ici nous avons par exemple appliqué la proposition \ref{PropooLNBHooBHAWiD} à chacun des deux intervalles \( \mathopen] -\infty , 0 \mathclose[\) et \( \mathopen] 0 , \infty \mathclose[\). Si \( +\infty\in B\) alors
        \begin{equation}
            w^{-1}(B)=w^{-1}\big( B\setminus\{ 0 \} \big)\cup w^{-1}(\{ \infty \})=  w^{-1}\big( B\setminus\{ 0 \} \big)\cup \{ 0 \},
        \end{equation}
        qui est borélien par union de boréliens.
    \item[Mesure produit]
    La proposition \ref{PropooVXPMooGSkyBo} nous assure alors qu'en posant\footnote{Avec un mini abus de notation : si \( 0\in B\), cette notation n'est pas tout à fait correcte.}
    \begin{equation}
        \mu(B)=\int_B\frac{1}{ | x | }d\lambda(x)
    \end{equation}
    où \(  \lambda \) est la mesure de Lebesgue, nous avons une mesure.

\item[Mesure du singleton]

    Pour avoir les idées claires, nous pouvons nous demander la mesure \( \mu\big( \{ 0 \} \big)\). Nous cela nous devons calculer
    \begin{equation}
        \int_{\{ 0 \}}\frac{1}{ | x | }d\lambda(x)=\int_{\{ 0 \}}w(x)d\lambda(x)
    \end{equation}
    où là, l'abus de notation n'est plus possible. Mais quelle que soit la fonction étagée \( h=\sum_i\alpha_i\caract_{A_i}\) considérée, 
    \begin{equation}
        \int_{\{ 0 \}}h(x)d\lambda(x)=\sum_i\alpha_i\lambda\big( A_i\cap\{ 0 \} \big)=0.
    \end{equation}

    Attention : ceci n'a rien de particulier à la fonction \( x\mapsto 1/| x |\). Lorsqu'une mesure a une densité par rapport à Lebesgue, la mesure d'un singleton sera toujours nulle.
    
\item[Mesure de la boule compacte]

    Il n'en reste pas moins que \( \mu\big( \mathopen[ -1 , 1 \mathclose] \big)=\infty\).
        
    \end{subproof}

    Tout ceci est un exemple du fait qu'un compact n'est pas toujours de mesure finie. En réalité, il n'y a pas de liens forts entre mesure et topologie. Un espace topologique est une chose, et y mettre une mesure en est une autre.

    Notons toutefois que la topologie de \( \eR\) n'est pas tout à fait sans lien avec notre mesure. En effet pour avoir une mesure à densité par rapport à Lebesgue, nous avons dû prendre une application mesurable par rapport à la tribu des boréliens, laquelle est éminemment liée à la topologie.

\end{example}


%+++++++++++++++++++++++++++++++++++++++++++++++++++++++++++++++++++++++++++++++++++++++++++++++++++++++++++++++++++++++++++ 
\section{Tribu produit, mesure produit}
%+++++++++++++++++++++++++++++++++++++++++++++++++++++++++++++++++++++++++++++++++++++++++++++++++++++++++++++++++++++++++++

%--------------------------------------------------------------------------------------------------------------------------- 
\subsection{Produit d'espaces mesurables}
%---------------------------------------------------------------------------------------------------------------------------

\begin{definition}      \label{DefTribProfGfYTuR}
    Si \( \tribA_1\) et \( \tribA_2\) sont deux tribus sur deux ensembles \( \Omega_1\) et \( \Omega_2\), nous définissons la \defe{tribu produit}{tribu!produit} \( \tribA_1\otimes\tribA_2\) comme étant la tribu engendrée par 
    \begin{equation}
        \{ X\times Y\tq X\in\tribA_1,Y\in\tribA_2 \}.
    \end{equation}
    Ces ensembles sont appelés \defe{rectangles}{rectangle!produit de tribus} de \( (\Omega_1,\tribA_1)\otimes (\Omega_2,\tribA_2)\).
\end{definition}

\begin{proposition}[\cite{KEQWooJsCGiw}]        \label{PropLJJWooKqWlTr}
    Soient deux espaces mesurables \( (S_1,\tribF_1)\) et \( (S_2,\tribF_2)\). Si \( \tribC_i\) est une classe de parties de \( S_i\) avec \( \tribF_i=\sigma(\tribC_i)\) et \( S_i\in\tribC_i\). Alors 
    \begin{equation}
        \tribF_1\otimes \tribF_2=\sigma(\tribC_1\times \tribC_2).
    \end{equation}
\end{proposition}

\begin{proof}
    Nous notons \( p_1\) et \( p_2\) les projections de \( S_1\times S_2\) vers \( S_1\) et \( S_2\). Nous commençons par prouver que
    \begin{equation}    \label{eqSGPBooLpQHfq}
        \tribF_1\otimes \tribF_2=\sigma\big( \p_1^{-1}(\tribF_1)\cup p_2^{-1}(\tribF_2) \big).
    \end{equation}
    En effet cette union est dans \( \tribF_1\otimes \tribF_2\) parce que ce sont tous des produits de la forme \( A_1\times S_2\) et \( S_1\times A_2\) où \( A_i\in \tribF_i\). Inversement, tous les produits de la forme \( A_1\times A_2\) sont dans la tribu engendrée par l'union parce que
    \begin{equation}
        A_1\cup A_2=(A_1\times S_2)\cap(S_1\times A_2).
    \end{equation}
    Par conséquent, la partie \( p_1^{-1}(\tribF_1)\cup p_2^{-1}(\tribF_2)\) engendre tous les produits qui \href{https://fr.wikisource.org/wiki/Bible_Crampon_1923/Matthieu}{ engendrent } la tribu \( \tribF_1\otimes\tribF_2\). L'égalité \eqref{eqSGPBooLpQHfq} est donc correcte.
    
    Si \( C_1\in\tribC_1\) alors
    \begin{equation}
        p_1^{-1}(C_1)=C_1\times S_2\in\tribC_1\times \tribC_2
    \end{equation}
    et donc \( p_1^{-1}(\tribC_1)\subset \tribC_1\times \tribC_2\). En utilisant le lemme de transfert \ref{LemOQTBooWGYuDU} nous avons alors
    \begin{equation}        \label{EqDQLYooVOLqMZ}
        p_1^{-1}(\tribF_1)=p_1^{-1}\big( \sigma(\tribC_1) \big)=\sigma\big( p_1^{-1}\tribC_1 \big)\subset\sigma(\tribC_1\times \tribC_1)
    \end{equation}
    et au bout de la même façon,
    \begin{equation}        \label{EqMTRCooVHNTHJ}
        p_2^{-1}(\tribF_1)\subset\sigma(\tribC_1\times \tribC_2).
    \end{equation}

    Vu les relations \eqref{EqDQLYooVOLqMZ}, \eqref{EqMTRCooVHNTHJ} et \eqref{eqSGPBooLpQHfq} nous avons
    \begin{equation}
        \tribF_1\otimes\tribF_2=\sigma\big( \p_1^{-1}(\tribF_1)\cup p_2^{-1}(\tribF_2) \big)\subset\sigma(\tribC_1\times \tribC_2).
    \end{equation}

    Réciproquement, si \( C_1\in \tribC_1\) et \( C_2\in \tribC_2\) alors
    \begin{equation}
        C_1\times C_2=(C_1\times S_1)\cap(S_1\times C_2)=p_1^{-1}(C_1)\cap p_2^{-1}(C_2)\in\tribF_1\otimes\tribF_2.
    \end{equation}
\end{proof}

%--------------------------------------------------------------------------------------------------------------------------- 
\subsection{Le cas des boréliens}
%---------------------------------------------------------------------------------------------------------------------------

Si \( X_1\) et  \( X_2\) sont des espaces topologiques et si nous notons \( \mO_i\) l'ensemble de leurs ouverts, par définition \( \Borelien(X_i)=\sigma(\mO_i)\). De plus par la proposition \ref{PropLJJWooKqWlTr} nous savons que
\begin{equation}        \label{EqOHMSooRSLrDk}
    \sigma(\mO_1\times \mO_2)=\Borelien(X_1)\otimes \Borelien(X_2).
\end{equation}

\begin{lemma}       \label{LemDEDQooJyzXgC}
    Si \( (X_i,\mO_i)\) sont des espaces topologiques, alors
    \begin{equation}
        \Borelien(X_1)\otimes \Borelien(X_2)\subset \Borelien(X_1\times X_2)
    \end{equation}
\end{lemma}

\begin{proof}
    Si \( A_i\in \mO_i\) alors \( A_1\times A_2\) est un ouvert de \( X_1\times X_2\) (voir la définition \ref{DefIINHooAAjTdY}). Par conséquent, \( \mO_1\times \mO_2\) est contenu dans l'ensemble des ouverts de \( X_1\times X_2\) ou encore
    \begin{equation}
        \mO_1\times \mO_2\subset\Borelien(X_1\times X_2),
    \end{equation}
    et donc
    \begin{equation}
        \sigma(\mO_1\times \mO_2)\subset\sigma\big( \Borelien(X_1\times X_2) \big)
    \end{equation}
    finalement, par \eqref{EqOHMSooRSLrDk}
    \begin{equation}
        \Borelien(X_1)\otimes\Borelien(X_2)\subset\Borelien(X_1\times X_2).
    \end{equation}
\end{proof}

Il n'y a en général pas égalité, mais nous allons immédiatement voir que dans (presque) tous les cas raisonnables, les boréliens sur un produit sont le produit des boréliens.

\begin{proposition}[\cite{KEQWooJsCGiw}]        \label{PropNAAJooBPbjkX}
    Soient \( (X_1,d_1)\) et \( (X_2,d_2)\) des espace métriques séparables. Alors
    \begin{equation}
        \Borelien(X_1\times X_2)=\Borelien(X_1)\otimes \Borelien(X_2).
    \end{equation}
\end{proposition}

\begin{proof}
    Nous savons par le lemme \ref{LemDUJXooWsnmpL} que tout ouvert de \( X_1\times X_2\) est une réunion dénombrable d'éléments de \( \mO_1\times\mO_2\). Donc tout ouvert de \( X_1\times X_2\) est dans \( \Borelien(X_1)\otimes \Borelien(X_2)\). Par conséquent
    \begin{equation}
        \Borelien(X_1\times X_2)\subset \Borelien(X_1)\otimes \Borelien(X_2).
    \end{equation}
    L'inclusion inverse étant déjà acquise par le lemme \ref{LemDEDQooJyzXgC}, nous avons l'égalité.
\end{proof}

\begin{proposition}     \label{CorWOOOooHcoEEF}
    Les boréliens sur \( \eR^N\) sont ceux qu'on croit.
    \begin{enumerate}
        \item
            \( \Borelien(\eR^2)=\Borelien(\eR)\otimes \Borelien(\eR)\)
        \item
            \( \Borelien(\eR^{N+1})=\Borelien(\eR^N)\otimes \Borelien(\eR)\)
    \end{enumerate}
\end{proposition}

\begin{proof}
    Cela n'est rien d'autre que la proposition \ref{PropNAAJooBPbjkX}.
\end{proof}

\begin{proposition}
    Soit un espace mesurable \( (S,\tribF)\) et des applications \( f_k\colon S\to \eR\) (\( k=1,\ldots, N\)). Alors l'application
    \begin{equation}
        \begin{aligned}
            f\colon (S,\tribF)&\to (\eR^n,\Borelien(\eR^N)) \\
            x&\mapsto \big( f_1(x),\ldots, f_N(x) \big) 
        \end{aligned}
    \end{equation}
    est mesurable si et seulement si chacun des \( f_i\) est mesurable.
\end{proposition}

\begin{proof}
    Division en deux.
    \begin{subproof}
    \item[Condition nécessaire]
        Nous supposons que les \( f_i\) sont mesurables. Nous avons
        \begin{subequations}
            \begin{align}
            f^{-1}\big( \prod_{k=1}^N\mathopen] a_k , b_k \mathclose[ \big)&=\{ x\in S\tq f_1(x)\in\mathopen] a_1 , b_1 \mathclose[ ,\cdots f_N(x)\in\mathopen] a_N , b_N \mathclose[\}\\
            &=\bigcap_{k=1}^Nf_k^{-1}\big( \mathopen] a_k , b_k \mathclose[ \big).
            \end{align}
        \end{subequations}
        Cela est une intersection finie d'éléments de \( \tribF\) et est donc un élément de \( \tribF\). Mais les pavés ouverts engendrent \( \Borelien(\eR^N)\) parce qu'ils sont une base dénombrable de la topologie (proposition \ref{PROPooYEkvbWBz}). Le théorème \ref{ThoECVAooDUxZrE} nous assure alors que \( f\) est mesurable parce que l'image inverse d'une base de la tribu est mesurable.
    \item[Condition suffisante]
        Si \( f\) est mesurable alors en particulier
        \begin{equation}
            f_k^{-1}\big( \mathopen] a , b \mathclose[ \big)=f^{-1}\big( \eR\times\ldots\times \mathopen] a , b \mathclose[\times \eR\times\ldots\times \eR \big)\in\tribF.
        \end{equation}
        Pour cela nous avons utilisé la proposition \ref{CorWOOOooHcoEEF} qui nous indique que le produit dans la parenthèse est un borélien de \( \eR^N\) en tant que produit de boréliens de \( \eR\).

        Encore une fois \( f_k^{-1}\) tombe dans \( \tribF\) pour une base dénombrable de la topologie de \( \eR\) et est donc mesurable.
    \end{subproof}
\end{proof}

%--------------------------------------------------------------------------------------------------------------------------- 
\subsection{Produit de mesures}
%---------------------------------------------------------------------------------------------------------------------------

\begin{lemma}[Propriété des sections\cite{NBoIEXO}] \label{LemAQmWEmN}
    Soient \( \tribA_1\) et \( \tribA_2\) des tribus sur les ensembles \( \Omega_1\) et \( \Omega_2\). Si \( A\in\tribA_1\otimes\tribA_2\) alors pour tout \( x\in \Omega_1\) et \( y\in\Omega_2\), les ensembles
    \begin{subequations}    \label{subEqCTtPccK}
        \begin{align}
            A_1(y)=\{ x\in\Omega_1\tq (x,y)\in A \}\\
            A_2(x)=\{ y\in\Omega_2\tq (x,y)\in A \}
        \end{align}
    \end{subequations}
    sont mesurables.
\end{lemma}
\index{section!propriété des}

\begin{proof}
    Soit \( y\in\Omega_2\); nous allons prouver le résultat pour \( A_1(y)\). Pour cela nous notons 
    \begin{equation}
        S=\{ A\in \tribA_1\otimes\tribA_2\tq \forall y\in\Omega_2, A_1(y)\in\tribA_1 \},
    \end{equation}
    et nous allons noter que \( S\) est une tribu contenant les rectangles. Par conséquent, \( S\) sera égal à \( \tribA_1\otimes \tribA_2\).

    \begin{subproof}
        \item[Les rectangles]

            Considérons le rectangle \( A=X\times Y\) et si \( y\in \Omega_2\) alors
            \begin{equation}
                A_1(y)=\{ x\in \Omega_1\tq (x,y)\in X\times Y \}.  
            \end{equation}
            Donc soit \( y\in Y\) alors \( A_1(y)=X\in\tribA_1\), soit \( y\notin Y\) et alors \( A_1(y)=\emptyset\in\tribA_1\).

        \item[Tribu : ensemble complet]

            Nous avons \( \Omega_1\times \Omega_2\in S\) parce que c'est un rectangle.

        \item[Tribu : complémentaire] Soit \( A\in S\) et montrons que \( A^c\in S\). Nous avons d'abord
            \begin{equation}
                (A^c)_1(y)=\{ x\in \Omega_1\tq (x,y)\in A^c \}.
            \end{equation}
            D'autre part
            \begin{equation}
                A_1(y)^c=\{ x\in\Omega_1\tq (x,y)\notin A \}=\{ x\in \Omega_1\tq (x,y)\in A^c \}=(A^c)_1(y).
            \end{equation}
            Vu que \( \tribA_1\) est une tribu et que par hypothèse \( A_1(y)\in\tribA_1\), nous avons aussi \( A_1(y)^c\in S\), et donc \( (A^c)_1(y)\in \tribA_1\), ce qui prouve que \( A^c\in S\).

        \item[Tribu : union dénombrable] Soit une suite \( A_n\in S\). Nous avons
            \begin{equation}
                (\bigcup_nA_n)_1(y)=\{ x\in\Omega_1\tq (x,y)\in \bigcup_nA_n \}=\bigcup_n\{ x\in\Omega_1\tq (x,y)\in A_n \}=\bigcup_n(A_n)_1(y),
            \end{equation}
            et ce dernier ensemble est dans \( \tribA_1\) parce que c'est une union dénombrable d'éléments de \( \tribA_1\).
        
    \end{subproof}
    Nous avons donc prouvé que \( S\) est une tribu contenant les rectangles, donc \( S\) contient au moins \( \tribA_1\otimes \tribA_2\).
\end{proof}

\begin{corollary}
    Si \( f\colon \Omega_1\times \Omega_2\to \eR\) est une fonction mesurable\footnote{Définition \ref{DefQKjDSeC}.} sur \( X\times Y\) alors pour chaque \( y\) dans \( \Omega_2\), la fonction
    \begin{equation}
        \begin{aligned}
            f_y\colon X&\to \eR \\
            x&\mapsto f(x,y) 
        \end{aligned}
    \end{equation}
    est mesurable.
\end{corollary}

\begin{proof}
    Soit \( \mO\) un ensemble mesurable de \( \eR\) (i.e. un borélien), et \( y\in \Omega_2\). Nous avons
    \begin{equation}
        f_y^{-1}(\mO)=\{ x\in X\tq f(x,y)\in \mO \}=A_1(y)
    \end{equation}
    où
    \begin{equation}
        A=\{ (x,y)\in \Omega_1\times \Omega_2\tq f(x,y)\in \mO \}=f^{-1}(\mO).
    \end{equation}
    Ce dernier est mesurable parce que \( f\) l'est.
\end{proof}

\begin{theorem}[\cite{NBoIEXO}\footnote{Modèle non contractuel : des notations et la définition de \( \lambda\)-système peuvent varier entre la référence et le présent texte.}]    \label{ThoCCIsLhO}
    Soient \( (\Omega_i,\tribA_i,\mu_i)\) (\( i=1,2\)) deux espaces mesurés \( \sigma\)-finie. Soit \( A\in\tribA_1\otimes \tribA_2\). Alors les fonctions\footnote{Voir la notation du lemme \ref{subEqCTtPccK}.}
    \begin{subequations}
        \begin{align}
            x\mapsto\mu_2\big( A_2(x) \big)\\
            y\mapsto\mu_1\big( A_1(y) \big)
        \end{align}
    \end{subequations}
    sont mesurables et
    \begin{equation}    \label{EqRKXwsQJ}
        \int_{\Omega_1}\mu_2\big( A_2(x) \big)d\mu_1(x)=\int_{\Omega_2}\mu_2\big( A_1(y) \big)d\mu_2(y).
    \end{equation}
\end{theorem}

\begin{proof}
    Nous supposons d'abord que \( \mu_1\) et \( \mu_2\) sont finies et nous notons \( \tribD\) le sous-ensemble de \( \tribA_1\otimes \tribA_2\) sur lequel le théorème est correct. Nous allons commencer par prouver que \( \tribD\) est un \( \lambda\)-système.

    \begin{subproof}
        \item[\( \lambda\)-système : différence ensembliste]
            Soient \( A,B\in\tribD\) avec \( A\subset B\). Nous avons
            \begin{subequations}
                \begin{align}
                    (B\setminus A)_1(y)&=\{ x\in \Omega_1\tq(x,y)\in B\setminus A \}\\
                    &=\{ x\in \Omega_1\tq(x,y)\in B\}\setminus\{ x\in \Omega_1\tq(x,y)\in  A \}\\
                    &=B_1(y)\setminus A_1(y).
                \end{align}
            \end{subequations}
            Vu que \( A_1(y)\subset B_1(y)\) et que les mesure sont finies le lemme \ref{LemPMprYuC} nous donne
            \begin{equation}
                \mu_1\big( (B\setminus A)_1(y) \big)=\mu_1\big( B_1(y) \big)-\mu_1\big( A_1(y) \big),
            \end{equation}
            et similairement pour \( 1\leftrightarrow 2\). Les deux fonctions (de \( y\)) à droite étant mesurables, nous avons la mesurabilité de la fonction \( y\mapsto \mu_1\big( (B\setminus A)_1(y) \big)\).

            Prouvons la formule intégrale en nous rappelant que la formule \eqref{EqRKXwsQJ} est supposée correcte pour \( A\) et \( B\) séparément :
            \begin{subequations}
                \begin{align}
                    \int_{\Omega_2}\mu_1\big( (B\setminus A)_1(y) \big)d\mu_2(y)&=\int_{\Omega_2}\mu_1\big( B_1(y) \big)d\mu_2(y)-\int_{\Omega_2}\mu_1\big( A_1(y) \big)d\mu_2(y)\\
                    &=\int_{\Omega_1}\mu_2\big( B_2(x) \big)d\mu_1(x)-\int_{\Omega_1}\mu_2\big( A_2(x) \big)d\mu_1(x)\\
                    &=\int_{\Omega_1}\mu_2\big( (B\setminus A)_2(x) \big)d\mu_1(x).
                \end{align}
            \end{subequations}
            
    
        \item[\( \lambda\)-système : limite de suite croissante]

            Soit \( (A_n)\) une suite croissante dans \( \tribD\); nous posons \( B_n=A_n\setminus A_{n-1}\) et \( A_0=\emptyset\) de telle sorte à travailler avec une suite d'ensembles disjoints qui satisfait \( \bigcup_nA_n=\bigcup_nB_n\). Vu que la suite est croissante nous avons \( A_{n-1}\subset A_n\) et donc \( B_n\in\tribD\) par le point déjà fait sur la différence ensembliste. Nous avons :
            \begin{subequations}
                \begin{align}
                    \mu_1\big( (\bigcup_nB_n)_1(y) \big)&=\{ x\in \Omega_1\tq (x,y)\in\bigcup_nB_n \}\\
                    &=\bigcup_n\{ x\in\Omega_1\tq (x,y)\in B_n \}\\
                    &=\bigcup_n (B_n)_1(y).
                \end{align}
            \end{subequations}
            Par conséquent, par la propriété \ref{ItemQFjtOjXiii} d'une mesure nous avons
            \begin{equation}
                \mu_1\big( (\bigcup_nB_n)_1(y) \big)=\sum_n\mu_1\big( (B_n)_1(y) \big).
            \end{equation}
            En tant que somme de fonctions positives et mesurables, la fonction
            \begin{equation}
                y\mapsto\sum_n\mu_1\big( (B_n)_1(y) \big)
            \end{equation}
            est mesurable par la proposition \ref{PropFYPEOIJ}. Il faut encore vérifier la formule intégrale. Le gros du boulot est de permuter une somme et une intégrale par le corollaire \ref{CorNKXwhdz} :
            \begin{subequations}
                \begin{align}
                    \int_{\Omega_2}\sum_n\mu_1\big( (B_n)_1(y) \big)d\mu_2(y)&=\sum_n\int_{\Omega_2}\mu_1\big( (B_n)_1(y) \big)d\mu_2(y)\\
                    &=\sum_n\int_{\Omega_1}\mu_2\big( (B_n)_2(x) \big)d\mu_1(x)\\
                    &=\int_{\Omega_1}\sum_n\mu_2\big( (B_n)_2(x) \big)d\mu_1(x)\\
                    &=\int_{\Omega_1}\mu_2\big( (\bigcup_nB_n)_1(y) \big)d\mu_1(x).
                \end{align}
            \end{subequations}
    \end{subproof}
    Maintenant que \( \tribD\) est un $\lambda$-système contenant les rectangles, le lemme \ref{LemLUmopaZ} dit que la tribu engendrée par \( \tribD\) (c'est à dire \( \tribA_1\otimes \tribA_2\)) est le $\lambda$-système \( \tribD\) lui-même.

    La preuve est finie dans le cas de mesures finies. Nous commençons maintenant à prouver dans le cas où les mesures \( \mu_1\) et \( \mu_2\) sont seulement \( \sigma\)-finies. Nous considérons des suites croissantes \( \Omega_{i,n}\to\Omega_i\) d'ensembles mesurables et de mesure finie : \( \mu_i(\Omega_{i,n})<\infty\). D'abord remarquons que
    \begin{equation}\label{EqNFuBzBF}
        \mu_2\Big( (A\cap \Omega_{1,j}\times E_{2,j})_2(x) \Big)=\mu_2\Big( A_2(x)\cap \Omega_{2,j} \Big)\mtu_{\Omega_{1,j}}.
    \end{equation}
    En effet,
    \begin{subequations}
        \begin{align}
            \heartsuit&=(A\cap\Omega_{1,j}\times E_{2,j})_2(x)\\
            &=\{ y\in\Omega_2\tq (x,y)\in A\cap \Omega_{1,j}\times E_{2,j} \}\\
            &=\{ y\in \Omega_2\tq (x,y)\in A\times \Omega_{2,j} \}\cap\{ y\in\Omega_2\tq (x,y)\in \Omega_{1,j}\times \Omega_{2,j} \}.
        \end{align}
    \end{subequations}
    Si \( y\in \Omega_{1,j}\) alors \( \{ y\in \Omega_2\tq (x,y)\in \Omega_{1,j}\times \Omega_{2,j} \}=\Omega_{2,j}\) et dans ce cas
    \begin{equation}
        \heartsuit=\{ y\in \Omega_2\tq (x,y)\in A\times \Omega_{2,j} \}\cap \Omega_{2,j}=A_2(x)\cap E_{2,j}.
    \end{equation}
    Et inversement, si \( x\notin \Omega_{1,j}\) alors \( \heartsuit=\emptyset\). Dans les deux cas nous avons \eqref{EqNFuBzBF}.

    Les ensembles \( A\cap \Omega_{1,j}\times \Omega_{2,j}\) étant de mesure finie, nous pouvons leur appliquer la première partie :
    \begin{equation}
        \int_{\Omega_1}\mu_2\Big( (A\cap\Omega_{1,j}\times \Omega_{2,j})_2(x) \Big)d\mu_1(x)=\int_{\Omega_2}\mu_1\Big( (A\cap\Omega_{1,j}\times \Omega_{2,j})_1(y) \Big)d\mu_2(u),
    \end{equation}
    ou encore
    \begin{equation}
        \int_{\Omega_1}\mu_2\Big( A_2(x)\cap \Omega_{2,j} \Big)\mtu_{\Omega_{1,j}}(x)d\mu_1(x)=\int_{\Omega_2}\mu_1\Big( A_1(y)\cap \Omega_{1,j} \Big)\mtu_{\Omega_{2,j}}(y)d\mu_2(y).
    \end{equation}
    Ce que nous avons dans ces intégrales sont (par rapport à \( j\)) des suites croissantes de fonction positives; nous pouvons donc permuter une limite et une intégrale. En sachant que si \( k\to \infty\), alors
    \begin{subequations}
        \begin{align}
            \mtu_{1,j}(x)\to 1\\
            \mu_2\big( A_2(x)\cap \Omega_2,j \big)\to\mu_2\big( A_2(x) \big),
        \end{align}
    \end{subequations}
    nous trouvons le résultat demandé.
\end{proof}

\begin{theorem}[\cite{FubiniBMauray,MesIntProbb}]   \label{ThoWWAjXzi}
    Soient \( \mu_i\) des mesures $\sigma$-finies sur \( (\Omega_i,\tribA_i)\) (\( i=1,2\)). 
    \begin{enumerate}
        \item
            
    Il existe une et une seule mesure, notée \( \mu_1\otimes \mu_2\), sur \( (\Omega_1\times\Omega_2,\tribA_1\otimes\tribA_2)\) telle que
    \begin{equation}    \label{EqOIuWLQU}
        (\mu_1\otimes\mu_2)(A_1\times A_2)=\mu_1(A_1)\mu_2(A_2)
    \end{equation}
    pour tout \( A_1\in \tribA_1\) et \( A_2\in\tribA_2\). 
\item
    Cette mesure est donnée par la formule\footnote{Voir les notations du lemme \ref{LemAQmWEmN}.}
    \begin{equation}   \label{EqDFxuGtH}
        (\mu_1\otimes \mu_2)(A)=\int_{\Omega_1}\mu_2\big( A_2(x) \big)d\mu_1(x)=\int_{\Omega_2}\mu_1\big( A_1(y) \big)d\mu_2(y).
    \end{equation}
    Cette mesure est la \defe{mesure produit}{mesure!produit} de \( \mu_1\) par \( \mu_2\).
\item
    La mesure \( \mu_1\otimes \mu_2\) ainsi définie est \( \sigma\)-finie.
    \end{enumerate}
\end{theorem}
\index{mesure!produit}

\begin{proof}
    La partie «existence» sera divisée en deux parties : l'une pour prouver que les formules \eqref{EqDFxuGtH} donnent une mesure et une pour montrer que cette mesure vérifie la condition \eqref{EqOIuWLQU}.
    \begin{subproof}
    \item[Unicité]
        
    L'ensemble des rectangles de \( \Omega_1\times \Omega_2\) engendre la tribu \( \tribA_1\otimes\tribA_2\), est fermé par intersection et contient une suite croissante d'ensembles \( P_n\times R_n\) de mesure finie (\( \mu(P_n\times R_n)<\infty\)) telle que \( P_n\times R_n\to \Omega_1\times \Omega_2\). Cette suite est donné par le fait que \( \mu_1\) et \( \mu_2\) sont \( \sigma\)-finies. En effet si \( (X_n)\) et \( (Y_n)\) sont des recouvrements dénombrables de \( \Omega_1\) et \( \Omega_2\) par des ensembles de mesure finie, en posant \( P_n=\bigcup_{k=1}^nX_n\) et \( R_n=\bigcup_{k=1}^nY_n\) nous avons bien une suite croissante de rectangles qui tendent vers \( \Omega_1\times \Omega_2\). Avec ces rectangles en main, le théorème \ref{ThoJDYlsXu} donne l'unicité.

\item[Les formules définissent une mesure]
    Le théorème \ref{ThoCCIsLhO} dit que ces formules ont un sens et que l'égalité entre les deux intégrales est correcte. Nous prouvons à présent qu'elles déterminent effectivement une mesure sur \( (\Omega_1\times\Omega_2,\tribA_1\otimes \tribA_2)\).

    Pour tout \( A\in \tribA_1\otimes \tribA_2\), \( \mu(A)\geq 0\) parce que \( \mu\) est donnée par l'intégrale d'une fonction positive.

    En ce qui concerne la condition d'unions dénombrable disjointe, soient \( A^{(i)}\) des éléments disjoints de \( \tribA_1\otimes \tribA_2\); nous commençons par remarquer que
    \begin{subequations}
        \begin{align}
            \left( \bigcup_{i=1}^{\infty}A^{(i)} \right)_2(x)&=\{ y\in\Omega_2\tq (x,y)\in\bigcup_{i=1}^{\infty}A^{(i)} \}\\
            &=\bigcup_{i=1}^{\infty}\{ y\in\Omega_2\tq (x,y)\in A^{(i)} \}\\
            &=\bigcup_{i=1}^{\infty}A^{(i)}_2(x).
        \end{align}
    \end{subequations}
    Par conséquent,
    \begin{subequations}
        \begin{align}
            \mu\left( \bigcup_{i=1}^{\infty}A^{(i)} \right)&=\int_{\Omega_1}\mu_2\left(    \Big( \bigcup_{i=1}^{\infty}A^{(i)} \Big)_2(x)     \right)d\mu_1(x)\\
            &=\int_{\Omega_1}\sum_{i=1}^{\infty}\mu_2\big( A^{(i)}_2(x) \big)d\mu_1(x)\\
            &=\int_{\Omega_1}\lim_{n\to \infty} \sum_{i=1}^{n}\mu_2\big( A^{(i)}_2(x) \big)d\mu_1(x).
        \end{align}
    \end{subequations}
    où nous avons utilisé l'additivité de la mesure \( \mu_2\). À ce niveau, il serait commode de permuter la somme et l'intégrale. Pour ce faire nous considérons la suite (croissante) de fonctions
    \begin{equation}
        f_n(x)=\sum_{i=1}^n\mu_2\big( A_2^{(i)}(x) \big).
    \end{equation}
    Nous pouvons permuter la limite et l'intégrale grâce au théorème de la convergence monotone \ref{ThoRRDooFUvEAN}; ensuite la somme se permute avec l'intégrale en tant que somme finie :
    \begin{subequations}
        \begin{align}
            \mu\left( \bigcup_{i=1}^{\infty}A^{(i)} \right)&=\lim_{n\to \infty} \sum_{i=1}^n\int_{\Omega_1}\big( A_2^{(i)}(x) \big)d\mu_1(x)\\
            &=\lim_{n\to \infty} \sum_{i=1}^n\mu(A^{(i)})\\
            &=\sum_{i=1}^{\infty}\mu( A^{(i)} ).
        \end{align}
    \end{subequations}

\item[Elles vérifient la condition]
    Prouvons que les formules \eqref{EqDFxuGtH} se réduisent à \eqref{EqOIuWLQU} dans le cas des rectangles. Soit donc \( A=X_1\times X_2\) avec \( X_i\in\tribA_i\). Alors
    \begin{equation}
        A_1(y)=\{ x\in\Omega_1\tq (x,y)\in X_1\times X_2 \}
    \end{equation}
    et
    \begin{equation}
        \mu_1\big( A_1(y) \big)=\mtu_{X_2}(y)\mu_1(X_1),
    \end{equation}
    donc
    \begin{subequations}
        \begin{align}
            (\mu_1\otimes\mu_2)(A)&=\int_{\Omega_2}\mu_1\big( A_1(y) \big)d\mu_2(y)\\
            &=\int_{\Omega_2}\mu_1(X_1)\mtu_{X_2}(y)d\mu_2(y)\\
            &=\mu_1(X_1)\int_{\Omega_2}\mtu_{X_2}(y)d\mu_2(y)\\
            &=\mu_1(X_1)\mu_2(X_2).
        \end{align}
    \end{subequations}
    Pour cela nous avons utilisé le fait que l'intégrale de la fonction caractéristique d'un ensemble mesurable est la mesure de cet ensemble.
    \end{subproof}
\end{proof}

\begin{definition}[Produit d'espaces mesurés]  \label{DefUMlBCAO}
    Si \( (\Omega_i,\tribA_i,\mu_i)\) sont deux espaces mesurés, l'\defe{espace produit}{produit!espaces mesurés} est l'ensemble \( \Omega_1\times \Omega_2\) muni de la tribu produit \( \tribA_1\otimes \tribA_2\) de la définition \ref{DefTribProfGfYTuR} et de la mesure produit \( \mu_1\otimes \mu_2\) définie par le théorème \ref{ThoWWAjXzi}.
\end{definition}

\begin{remark}
    Il n'est pas garantit que la tribu \( \tribA_1\otimes\tribA_2\) soit la tribu la plus adaptée à l'ensemble \( S_1\times S_2\). Dans le cas de \( \eR^N\), il se fait que c'est le cas : en prenant des produits des boréliens sur \( \eR\) on obtient bien les boréliens sur \( \eR^N\), voir proposition \ref{CorWOOOooHcoEEF}.
\end{remark}
