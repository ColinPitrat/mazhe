% This is part of Mes notes de mathématique
% Copyright (c) 2011-2019
%   Laurent Claessens, Carlotta Donadello
% See the file fdl-1.3.txt for copying conditions.

%+++++++++++++++++++++++++++++++++++++++++++++++++++++++++++++++++++++++++++++++++++++++++++++++++++++++++++++++++++++++++++
\section{Tribu produit, mesure produit}
%+++++++++++++++++++++++++++++++++++++++++++++++++++++++++++++++++++++++++++++++++++++++++++++++++++++++++++++++++++++++++++

%---------------------------------------------------------------------------------------------------------------------------
\subsection{Produit d'espaces mesurables}
%---------------------------------------------------------------------------------------------------------------------------

\begin{definition}      \label{DefTribProfGfYTuR}
    Si \( \tribA_1\) et \( \tribA_2\) sont deux tribus sur deux ensembles \( \Omega_1\) et \( \Omega_2\), nous définissons la \defe{tribu produit}{tribu!produit} \( \tribA_1\otimes\tribA_2\) comme étant la tribu engendrée par
    \begin{equation}
        \{ X\times Y\tq X\in\tribA_1,Y\in\tribA_2 \}.
    \end{equation}
    Ces ensembles sont appelés \defe{rectangles}{rectangle!produit de tribus} de \( (\Omega_1,\tribA_1)\otimes (\Omega_2,\tribA_2)\).
\end{definition}

\begin{proposition}[\cite{KEQWooJsCGiw}]        \label{PropLJJWooKqWlTr}
    Soient deux espaces mesurables \( (S_1,\tribF_1)\) et \( (S_2,\tribF_2)\). Si \( \tribC_i\) est une classe de parties de \( S_i\) avec \( \tribF_i=\sigma(\tribC_i)\) et \( S_i\in\tribC_i\). Alors
    \begin{equation}
        \tribF_1\otimes \tribF_2=\sigma(\tribC_1\times \tribC_2).
    \end{equation}
\end{proposition}

\begin{proof}
    Nous notons \( p_1\) et \( p_2\) les projections de \( S_1\times S_2\) vers \( S_1\) et \( S_2\). Nous commençons par prouver que
    \begin{equation}    \label{eqSGPBooLpQHfq}
        \tribF_1\otimes \tribF_2=\sigma\big( \p_1^{-1}(\tribF_1)\cup p_2^{-1}(\tribF_2) \big).
    \end{equation}
    En effet cette union est dans \( \tribF_1\otimes \tribF_2\) parce que ce sont tous des produits de la forme \( A_1\times S_2\) et \( S_1\times A_2\) où \( A_i\in \tribF_i\). Inversement, tous les produits de la forme \( A_1\times A_2\) sont dans la tribu engendrée par l'union parce que
    \begin{equation}
        A_1\cup A_2=(A_1\times S_2)\cap(S_1\times A_2).
    \end{equation}
    Par conséquent, la partie \( p_1^{-1}(\tribF_1)\cup p_2^{-1}(\tribF_2)\) engendre tous les produits qui \href{https://fr.wikisource.org/wiki/Bible_Crampon_1923/Matthieu}{ engendrent } la tribu \( \tribF_1\otimes\tribF_2\). L'égalité \eqref{eqSGPBooLpQHfq} est donc correcte.

    Si \( C_1\in\tribC_1\) alors
    \begin{equation}
        p_1^{-1}(C_1)=C_1\times S_2\in\tribC_1\times \tribC_2
    \end{equation}
    et donc \( p_1^{-1}(\tribC_1)\subset \tribC_1\times \tribC_2\). En utilisant le lemme de transfert~\ref{LemOQTBooWGYuDU} nous avons alors
    \begin{equation}        \label{EqDQLYooVOLqMZ}
        p_1^{-1}(\tribF_1)=p_1^{-1}\big( \sigma(\tribC_1) \big)=\sigma\big( p_1^{-1}\tribC_1 \big)\subset\sigma(\tribC_1\times \tribC_1)
    \end{equation}
    et au bout de la même façon,
    \begin{equation}        \label{EqMTRCooVHNTHJ}
        p_2^{-1}(\tribF_1)\subset\sigma(\tribC_1\times \tribC_2).
    \end{equation}

    Vu les relations \eqref{EqDQLYooVOLqMZ}, \eqref{EqMTRCooVHNTHJ} et \eqref{eqSGPBooLpQHfq} nous avons
    \begin{equation}
        \tribF_1\otimes\tribF_2=\sigma\big( \p_1^{-1}(\tribF_1)\cup p_2^{-1}(\tribF_2) \big)\subset\sigma(\tribC_1\times \tribC_2).
    \end{equation}

    Réciproquement, si \( C_1\in \tribC_1\) et \( C_2\in \tribC_2\) alors
    \begin{equation}
        C_1\times C_2=(C_1\times S_1)\cap(S_1\times C_2)=p_1^{-1}(C_1)\cap p_2^{-1}(C_2)\in\tribF_1\otimes\tribF_2.
    \end{equation}
\end{proof}

%---------------------------------------------------------------------------------------------------------------------------
\subsection{Le cas des boréliens}
%---------------------------------------------------------------------------------------------------------------------------

Si \( X_1\) et  \( X_2\) sont des espaces topologiques et si nous notons \( \mO_i\) l'ensemble de leurs ouverts, par définition \( \Borelien(X_i)=\sigma(\mO_i)\). De plus par la proposition~\ref{PropLJJWooKqWlTr} nous savons que
\begin{equation}        \label{EqOHMSooRSLrDk}
    \sigma(\mO_1\times \mO_2)=\Borelien(X_1)\otimes \Borelien(X_2).
\end{equation}

\begin{lemma}       \label{LemDEDQooJyzXgC}
    Si \( (X_i,\mO_i)\) sont des espaces topologiques, alors
    \begin{equation}
        \Borelien(X_1)\otimes \Borelien(X_2)\subset \Borelien(X_1\times X_2)
    \end{equation}
\end{lemma}

\begin{proof}
    Si \( A_i\in \mO_i\) alors \( A_1\times A_2\) est un ouvert de \( X_1\times X_2\) (voir la définition~\ref{DefIINHooAAjTdY}). Par conséquent, \( \mO_1\times \mO_2\) est contenu dans l'ensemble des ouverts de \( X_1\times X_2\) ou encore
    \begin{equation}
        \mO_1\times \mO_2\subset\Borelien(X_1\times X_2),
    \end{equation}
    et donc
    \begin{equation}
        \sigma(\mO_1\times \mO_2)\subset\sigma\big( \Borelien(X_1\times X_2) \big)
    \end{equation}
    finalement, par \eqref{EqOHMSooRSLrDk}
    \begin{equation}
        \Borelien(X_1)\otimes\Borelien(X_2)\subset\Borelien(X_1\times X_2).
    \end{equation}
\end{proof}

Il n'y a en général pas égalité, mais nous allons immédiatement voir que dans (presque) tous les cas raisonnables, les boréliens sur un produit sont le produit des boréliens.

\begin{proposition}[\cite{KEQWooJsCGiw}]        \label{PropNAAJooBPbjkX}
    Soient \( (X_1,d_1)\) et \( (X_2,d_2)\) des espaces métriques séparables. Alors
    \begin{equation}
        \Borelien(X_1\times X_2)=\Borelien(X_1)\otimes \Borelien(X_2).
    \end{equation}
\end{proposition}

\begin{proof}
    Nous savons par le lemme~\ref{LemDUJXooWsnmpL} que tout ouvert de \( X_1\times X_2\) est une réunion dénombrable d'éléments de \( \mO_1\times\mO_2\). Donc tout ouvert de \( X_1\times X_2\) est dans \( \Borelien(X_1)\otimes \Borelien(X_2)\). Par conséquent
    \begin{equation}
        \Borelien(X_1\times X_2)\subset \Borelien(X_1)\otimes \Borelien(X_2).
    \end{equation}
    L'inclusion inverse étant déjà acquise par le lemme~\ref{LemDEDQooJyzXgC}, nous avons l'égalité.
\end{proof}

\begin{proposition}     \label{CorWOOOooHcoEEF}
    Les boréliens sur \( \eR^N\) sont ceux qu'on croit.
    \begin{enumerate}
        \item
            \( \Borelien(\eR^2)=\Borelien(\eR)\otimes \Borelien(\eR)\)
        \item
            \( \Borelien(\eR^{N+1})=\Borelien(\eR^N)\otimes \Borelien(\eR)\)
    \end{enumerate}
\end{proposition}

\begin{proof}
    Cela n'est rien d'autre que la proposition~\ref{PropNAAJooBPbjkX}.
\end{proof}

\begin{proposition}
    Soit un espace mesurable \( (S,\tribF)\) et des applications \( f_k\colon S\to \eR\) (\( k=1,\ldots, N\)). Alors l'application
    \begin{equation}
        \begin{aligned}
            f\colon (S,\tribF)&\to (\eR^N,\Borelien(\eR^N)) \\
            x&\mapsto \big( f_1(x),\ldots, f_N(x) \big)
        \end{aligned}
    \end{equation}
    est mesurable si et seulement si chacun des \( f_i\) est mesurable.
\end{proposition}

\begin{proof}
    Division en deux.
    \begin{subproof}
    \item[Condition nécessaire]
        Nous supposons que les \( f_i\) sont mesurables. Nous avons
        \begin{subequations}
            \begin{align}
            f^{-1}\big( \prod_{k=1}^N\mathopen] a_k , b_k \mathclose[ \big)&=\{ x\in S\tq f_1(x)\in\mathopen] a_1 , b_1 \mathclose[ ,\cdots f_N(x)\in\mathopen] a_N , b_N \mathclose[\}\\
            &=\bigcap_{k=1}^Nf_k^{-1}\big( \mathopen] a_k , b_k \mathclose[ \big).
            \end{align}
        \end{subequations}
        Cela est une intersection finie d'éléments de \( \tribF\) et est donc un élément de \( \tribF\). Mais les pavés ouverts engendrent \( \Borelien(\eR^N)\) parce qu'ils sont une base dénombrable de la topologie (proposition~\ref{PROPooYEkvbWBz}). Le théorème~\ref{ThoECVAooDUxZrE} nous assure alors que \( f\) est mesurable parce que l'image inverse d'une base de la tribu est mesurable.
    \item[Condition suffisante]
        Si \( f\) est mesurable alors en particulier
        \begin{equation}
            f_k^{-1}\big( \mathopen] a , b \mathclose[ \big)=f^{-1}\big( \eR\times\ldots\times \mathopen] a , b \mathclose[\times \eR\times\ldots\times \eR \big)\in\tribF.
        \end{equation}
        Pour cela nous avons utilisé la proposition~\ref{CorWOOOooHcoEEF} qui nous indique que le produit dans la parenthèse est un borélien de \( \eR^N\) en tant que produit de boréliens de \( \eR\).

        Encore une fois \( f_k^{-1}\) tombe dans \( \tribF\) pour une base dénombrable de la topologie de \( \eR\) et est donc mesurable.
    \end{subproof}
\end{proof}

%---------------------------------------------------------------------------------------------------------------------------
\subsection{Produit de mesures}
%---------------------------------------------------------------------------------------------------------------------------

\begin{lemma}[Propriété des sections\cite{NBoIEXO}] \label{LemAQmWEmN}
    Soient \( \tribA_1\) et \( \tribA_2\) des tribus sur les ensembles \( \Omega_1\) et \( \Omega_2\). Si \( A\in\tribA_1\otimes\tribA_2\) alors pour tout \( x\in \Omega_1\) et \( y\in\Omega_2\), les ensembles
    \begin{subequations}    \label{subEqCTtPccK}
        \begin{align}
            A_1(y)=\{ x\in\Omega_1\tq (x,y)\in A \}\\
            A_2(x)=\{ y\in\Omega_2\tq (x,y)\in A \}
        \end{align}
    \end{subequations}
    sont mesurables.
\end{lemma}
\index{section!propriété des}

\begin{proof}
    Soit \( y\in\Omega_2\); nous allons prouver le résultat pour \( A_1(y)\). Pour cela nous notons
    \begin{equation}
        S=\{ A\in \tribA_1\otimes\tribA_2\tq \forall y\in\Omega_2, A_1(y)\in\tribA_1 \},
    \end{equation}
    et nous allons noter que \( S\) est une tribu contenant les rectangles. Par conséquent, \( S\) sera égal à \( \tribA_1\otimes \tribA_2\).

    \begin{subproof}
        \item[Les rectangles]

            Considérons le rectangle \( A=X\times Y\) et si \( y\in \Omega_2\) alors
            \begin{equation}
                A_1(y)=\{ x\in \Omega_1\tq (x,y)\in X\times Y \}.
            \end{equation}
            Donc soit \( y\in Y\) alors \( A_1(y)=X\in\tribA_1\), soit \( y\notin Y\) et alors \( A_1(y)=\emptyset\in\tribA_1\).

        \item[Tribu : ensemble complet]

            Nous avons \( \Omega_1\times \Omega_2\in S\) parce que c'est un rectangle.

        \item[Tribu : complémentaire] Soit \( A\in S\). Montrons que \( A^c\in S\). Nous avons d'abord
            \begin{equation}
                (A^c)_1(y)=\{ x\in \Omega_1\tq (x,y)\in A^c \}.
            \end{equation}
            D'autre part
            \begin{equation}
                A_1(y)^c=\{ x\in\Omega_1\tq (x,y)\notin A \}=\{ x\in \Omega_1\tq (x,y)\in A^c \}=(A^c)_1(y).
            \end{equation}
            Vu que \( \tribA_1\) est une tribu et que par hypothèse \( A_1(y)\in\tribA_1\), nous avons aussi \( A_1(y)^c\in S\), et donc \( (A^c)_1(y)\in \tribA_1\), ce qui prouve que \( A^c\in S\).

        \item[Tribu : union dénombrable] Soit une suite \( A_n\in S\). Nous avons
            \begin{subequations}
                \begin{align}
                (\bigcup_nA_n)_1(y)&=\{ x\in\Omega_1\tq (x,y)\in \bigcup_nA_n \}\\
                &=\bigcup_n\{ x\in\Omega_1\tq (x,y)\in A_n \}\\
                &=\bigcup_n(A_n)_1(y),
                \end{align}
            \end{subequations}
            et ce dernier ensemble est dans \( \tribA_1\) parce que c'est une union dénombrable d'éléments de \( \tribA_1\).

    \end{subproof}
    Nous avons donc prouvé que \( S\) est une tribu contenant les rectangles, donc \( S\) contient au moins \( \tribA_1\otimes \tribA_2\).
\end{proof}

\begin{corollary}
    Si \( f\colon \Omega_1\times \Omega_2\to \eR\) est une fonction mesurable\footnote{Définition~\ref{DefQKjDSeC}.} sur \( X\times Y\) alors pour chaque \( y\) dans \( \Omega_2\), la fonction
    \begin{equation}
        \begin{aligned}
            f_y\colon X&\to \eR \\
            x&\mapsto f(x,y)
        \end{aligned}
    \end{equation}
    est mesurable.
\end{corollary}

\begin{proof}
    Soit \( \mO\) un ensemble mesurable de \( \eR\) (i.e. un borélien), et \( y\in \Omega_2\). Nous avons
    \begin{equation}
        f_y^{-1}(\mO)=\{ x\in X\tq f(x,y)\in \mO \}=A_1(y)
    \end{equation}
    où
    \begin{equation}
        A=\{ (x,y)\in \Omega_1\times \Omega_2\tq f(x,y)\in \mO \}=f^{-1}(\mO).
    \end{equation}
    Ce dernier est mesurable parce que \( f\) l'est.
\end{proof}

\begin{theorem}[\cite{NBoIEXO}\footnote{Modèle non contractuel : des notations et la définition de \( \lambda\)-système peuvent varier entre la référence et le présent texte.}]    \label{ThoCCIsLhO}
    Soient \( (\Omega_i,\tribA_i,\mu_i)\) (\( i=1,2\)) deux espaces mesurés \( \sigma\)-finie. Soit \( A\in\tribA_1\otimes \tribA_2\). Alors les fonctions\footnote{Voir la notation du lemme~\ref{subEqCTtPccK}.}
    \begin{subequations}
        \begin{align}
            x\mapsto\mu_2\big( A_2(x) \big)\\
            y\mapsto\mu_1\big( A_1(y) \big)
        \end{align}
    \end{subequations}
    sont mesurables et
    \begin{equation}    \label{EqRKXwsQJ}
        \int_{\Omega_1}\mu_2\big( A_2(x) \big)d\mu_1(x)=\int_{\Omega_2}\mu_2\big( A_1(y) \big)d\mu_2(y).
    \end{equation}
\end{theorem}

\begin{proof}
    Nous supposons d'abord que \( \mu_1\) et \( \mu_2\) sont finies et nous notons \( \tribD\) le sous-ensemble de \( \tribA_1\otimes \tribA_2\) sur lequel le théorème est correct. Nous allons commencer par prouver que \( \tribD\) est un \( \lambda\)-système.

    \begin{subproof}
        \item[\( \lambda\)-système : différence ensembliste]
            Soient \( A,B\in\tribD\) avec \( A\subset B\). Nous avons
            \begin{subequations}
                \begin{align}
                    (B\setminus A)_1(y)&=\{ x\in \Omega_1\tq(x,y)\in B\setminus A \}\\
                    &=\{ x\in \Omega_1\tq(x,y)\in B\}\setminus\{ x\in \Omega_1\tq(x,y)\in  A \}\\
                    &=B_1(y)\setminus A_1(y).
                \end{align}
            \end{subequations}
            Vu que \( A_1(y)\subset B_1(y)\) et que les mesures sont finies le lemme~\ref{LemPMprYuC} nous donne
            \begin{equation}
                \mu_1\big( (B\setminus A)_1(y) \big)=\mu_1\big( B_1(y) \big)-\mu_1\big( A_1(y) \big),
            \end{equation}
            et similairement pour \( 1\leftrightarrow 2\). Les deux fonctions (de \( y\)) à droite étant mesurables, nous avons la mesurabilité de la fonction \( y\mapsto \mu_1\big( (B\setminus A)_1(y) \big)\).

            Prouvons la formule intégrale en nous rappelant que la formule \eqref{EqRKXwsQJ} est supposée correcte pour \( A\) et \( B\) séparément :
            \begin{subequations}
                \begin{align}
                    \int_{\Omega_2}\mu_1\big( (B\setminus A)_1(y) \big)d\mu_2(y)&=\int_{\Omega_2}\mu_1\big( B_1(y) \big)d\mu_2(y)-\int_{\Omega_2}\mu_1\big( A_1(y) \big)d\mu_2(y)\\
                    &=\int_{\Omega_1}\mu_2\big( B_2(x) \big)d\mu_1(x)-\int_{\Omega_1}\mu_2\big( A_2(x) \big)d\mu_1(x)\\
                    &=\int_{\Omega_1}\mu_2\big( (B\setminus A)_2(x) \big)d\mu_1(x).
                \end{align}
            \end{subequations}


        \item[\( \lambda\)-système : limite de suite croissante]

            Soit \( (A_n)\) une suite croissante dans \( \tribD\); nous posons \( B_n=A_n\setminus A_{n-1}\) et \( A_0=\emptyset\) de telle sorte à travailler avec une suite d'ensembles disjoints qui satisfait \( \bigcup_nA_n=\bigcup_nB_n\). Vu que la suite est croissante nous avons \( A_{n-1}\subset A_n\) et donc \( B_n\in\tribD\) par le point déjà fait sur la différence ensembliste. Nous avons :
            \begin{subequations}
                \begin{align}
                    \mu_1\big( (\bigcup_nB_n)_1(y) \big)&=\{ x\in \Omega_1\tq (x,y)\in\bigcup_nB_n \}\\
                    &=\bigcup_n\{ x\in\Omega_1\tq (x,y)\in B_n \}\\
                    &=\bigcup_n (B_n)_1(y).
                \end{align}
            \end{subequations}
            Par conséquent, par la propriété~\ref{ItemQFjtOjXiii} d'une mesure nous avons
            \begin{equation}
                \mu_1\big( (\bigcup_nB_n)_1(y) \big)=\sum_n\mu_1\big( (B_n)_1(y) \big).
            \end{equation}
            En tant que somme de fonctions positives et mesurables, la fonction
            \begin{equation}
                y\mapsto\sum_n\mu_1\big( (B_n)_1(y) \big)
            \end{equation}
            est mesurable par la proposition~\ref{PropFYPEOIJ}. Il faut encore vérifier la formule intégrale. Le gros du boulot est de permuter une somme et une intégrale par le corollaire~\ref{CorNKXwhdz} :
            \begin{subequations}
                \begin{align}
                    \int_{\Omega_2}\sum_n\mu_1\big( (B_n)_1(y) \big)d\mu_2(y)&=\sum_n\int_{\Omega_2}\mu_1\big( (B_n)_1(y) \big)d\mu_2(y)\\
                    &=\sum_n\int_{\Omega_1}\mu_2\big( (B_n)_2(x) \big)d\mu_1(x)\\
                    &=\int_{\Omega_1}\sum_n\mu_2\big( (B_n)_2(x) \big)d\mu_1(x)\\
                    &=\int_{\Omega_1}\mu_2\big( (\bigcup_nB_n)_1(y) \big)d\mu_1(x).
                \end{align}
            \end{subequations}
    \end{subproof}
    Maintenant que \( \tribD\) est un $\lambda$-système contenant les rectangles, le lemme~\ref{LemLUmopaZ} dit que la tribu engendrée par \( \tribD\) (c'est-à-dire \( \tribA_1\otimes \tribA_2\)) est le $\lambda$-système \( \tribD\) lui-même.

    La preuve est finie dans le cas de mesures finies. Nous commençons maintenant à prouver dans le cas où les mesures \( \mu_1\) et \( \mu_2\) sont seulement \( \sigma\)-finies. Nous considérons des suites croissantes \( \Omega_{i,n}\to\Omega_i\) d'ensembles mesurables et de mesure finie : \( \mu_i(\Omega_{i,n})<\infty\). D'abord remarquons que
    \begin{equation}\label{EqNFuBzBF}
        \mu_2\Big( (A\cap \Omega_{1,j}\times E_{2,j})_2(x) \Big)=\mu_2\Big( A_2(x)\cap \Omega_{2,j} \Big)\mtu_{\Omega_{1,j}}.
    \end{equation}
    En effet,
    \begin{subequations}
        \begin{align}
            \heartsuit&=(A\cap\Omega_{1,j}\times E_{2,j})_2(x)\\
            &=\{ y\in\Omega_2\tq (x,y)\in A\cap \Omega_{1,j}\times E_{2,j} \}\\
            &=\{ y\in \Omega_2\tq (x,y)\in A\times \Omega_{2,j} \}\cap\{ y\in\Omega_2\tq (x,y)\in \Omega_{1,j}\times \Omega_{2,j} \}.
        \end{align}
    \end{subequations}
    Si \( y\in \Omega_{1,j}\) alors \( \{ y\in \Omega_2\tq (x,y)\in \Omega_{1,j}\times \Omega_{2,j} \}=\Omega_{2,j}\) et dans ce cas
    \begin{equation}
        \heartsuit=\{ y\in \Omega_2\tq (x,y)\in A\times \Omega_{2,j} \}\cap \Omega_{2,j}=A_2(x)\cap E_{2,j}.
    \end{equation}
    Et inversement, si \( x\notin \Omega_{1,j}\) alors \( \heartsuit=\emptyset\). Dans les deux cas nous avons \eqref{EqNFuBzBF}.

    Les ensembles \( A\cap \Omega_{1,j}\times \Omega_{2,j}\) étant de mesure finie, nous pouvons leur appliquer la première partie :
    \begin{equation}
        \int_{\Omega_1}\mu_2\Big( (A\cap\Omega_{1,j}\times \Omega_{2,j})_2(x) \Big)d\mu_1(x)=\int_{\Omega_2}\mu_1\Big( (A\cap\Omega_{1,j}\times \Omega_{2,j})_1(y) \Big)d\mu_2(u),
    \end{equation}
    ou encore
    \begin{equation}
        \int_{\Omega_1}\mu_2\Big( A_2(x)\cap \Omega_{2,j} \Big)\mtu_{\Omega_{1,j}}(x)d\mu_1(x)=\int_{\Omega_2}\mu_1\Big( A_1(y)\cap \Omega_{1,j} \Big)\mtu_{\Omega_{2,j}}(y)d\mu_2(y).
    \end{equation}
    Ce que nous avons dans ces intégrales sont (par rapport à \( j\)) des suites croissantes de fonction positives; nous pouvons donc permuter une limite et une intégrale. En sachant que si \( k\to \infty\), alors
    \begin{subequations}
        \begin{align}
            \mtu_{1,j}(x)\to 1\\
            \mu_2\big( A_2(x)\cap \Omega_2,j \big)\to\mu_2\big( A_2(x) \big),
        \end{align}
    \end{subequations}
    nous trouvons le résultat demandé.
\end{proof}

\begin{theoremDef}[\cite{FubiniBMauray,MesIntProbb}]   \label{ThoWWAjXzi}
    Soient \( \mu_i\) des mesures $\sigma$-finies sur \( (\Omega_i,\tribA_i)\) (\( i=1,2\)).
    \begin{enumerate}
        \item

    Il existe une et une seule mesure, notée \( \mu_1\otimes \mu_2\), sur \( (\Omega_1\times\Omega_2,\tribA_1\otimes\tribA_2)\) telle que
    \begin{equation}    \label{EqOIuWLQU}
        (\mu_1\otimes\mu_2)(A_1\times A_2)=\mu_1(A_1)\mu_2(A_2)
    \end{equation}
    pour tout \( A_1\in \tribA_1\) et \( A_2\in\tribA_2\).
\item
    Cette mesure est donnée par la formule\footnote{Voir les notations du lemme~\ref{LemAQmWEmN}.}
    \begin{equation}   \label{EqDFxuGtH}
        (\mu_1\otimes \mu_2)(A)=\int_{\Omega_1}\mu_2\big( A_2(x) \big)d\mu_1(x)=\int_{\Omega_2}\mu_1\big( A_1(y) \big)d\mu_2(y).
    \end{equation}
    Cette mesure est la \defe{mesure produit}{mesure!produit} de \( \mu_1\) par \( \mu_2\).
\item
    La mesure \( \mu_1\otimes \mu_2\) ainsi définie est \( \sigma\)-finie.
    \end{enumerate}
\end{theoremDef}
\index{mesure!produit}

\begin{proof}
    La partie «existence» sera divisée en deux parties : l'une pour prouver que les formules \eqref{EqDFxuGtH} donnent une mesure et une pour montrer que cette mesure vérifie la condition \eqref{EqOIuWLQU}.
    \begin{subproof}
    \item[Unicité]

    L'ensemble des rectangles de \( \Omega_1\times \Omega_2\) engendre la tribu \( \tribA_1\otimes\tribA_2\), est fermé par intersection et contient une suite croissante d'ensembles \( P_n\times R_n\) de mesure finie (\( \mu(P_n\times R_n)<\infty\)) telle que \( P_n\times R_n\to \Omega_1\times \Omega_2\). Cette suite est donné par le fait que \( \mu_1\) et \( \mu_2\) sont \( \sigma\)-finies. En effet si \( (X_n)\) et \( (Y_n)\) sont des recouvrements dénombrables de \( \Omega_1\) et \( \Omega_2\) par des ensembles de mesure finie, en posant \( P_n=\bigcup_{k=1}^nX_n\) et \( R_n=\bigcup_{k=1}^nY_n\) nous avons bien une suite croissante de rectangles qui tendent vers \( \Omega_1\times \Omega_2\). Avec ces rectangles en main, le théorème~\ref{ThoJDYlsXu} donne l'unicité.

\item[Les formules définissent une mesure]
    Le théorème~\ref{ThoCCIsLhO} dit que ces formules ont un sens et que l'égalité entre les deux intégrales est correcte. Nous prouvons à présent qu'elles déterminent effectivement une mesure sur \( (\Omega_1\times\Omega_2,\tribA_1\otimes \tribA_2)\).

    Pour tout \( A\in \tribA_1\otimes \tribA_2\), \( \mu(A)\geq 0\) parce que \( \mu\) est donnée par l'intégrale d'une fonction positive.

    En ce qui concerne la condition d'unions dénombrable disjointe, soient \( A^{(i)}\) des éléments disjoints de \( \tribA_1\otimes \tribA_2\); nous commençons par remarquer que
    \begin{subequations}
        \begin{align}
            \left( \bigcup_{i=1}^{\infty}A^{(i)} \right)_2(x)&=\{ y\in\Omega_2\tq (x,y)\in\bigcup_{i=1}^{\infty}A^{(i)} \}\\
            &=\bigcup_{i=1}^{\infty}\{ y\in\Omega_2\tq (x,y)\in A^{(i)} \}\\
            &=\bigcup_{i=1}^{\infty}A^{(i)}_2(x).
        \end{align}
    \end{subequations}
    Par conséquent,
    \begin{subequations}
        \begin{align}
            \mu\left( \bigcup_{i=1}^{\infty}A^{(i)} \right)&=\int_{\Omega_1}\mu_2\left(    \Big( \bigcup_{i=1}^{\infty}A^{(i)} \Big)_2(x)     \right)d\mu_1(x)\\
            &=\int_{\Omega_1}\sum_{i=1}^{\infty}\mu_2\big( A^{(i)}_2(x) \big)d\mu_1(x)\\
            &=\int_{\Omega_1}\lim_{n\to \infty} \sum_{i=1}^{n}\mu_2\big( A^{(i)}_2(x) \big)d\mu_1(x).
        \end{align}
    \end{subequations}
    où nous avons utilisé l'additivité de la mesure \( \mu_2\). À ce niveau, il serait commode de permuter la somme et l'intégrale. Pour ce faire nous considérons la suite (croissante) de fonctions
    \begin{equation}
        f_n(x)=\sum_{i=1}^n\mu_2\big( A_2^{(i)}(x) \big).
    \end{equation}
    Nous pouvons permuter la limite et l'intégrale grâce au théorème de la convergence monotone~\ref{ThoRRDooFUvEAN}; ensuite la somme se permute avec l'intégrale en tant que somme finie :
    \begin{subequations}
        \begin{align}
            \mu\left( \bigcup_{i=1}^{\infty}A^{(i)} \right)&=\lim_{n\to \infty} \sum_{i=1}^n\int_{\Omega_1}\big( A_2^{(i)}(x) \big)d\mu_1(x)\\
            &=\lim_{n\to \infty} \sum_{i=1}^n\mu(A^{(i)})\\
            &=\sum_{i=1}^{\infty}\mu( A^{(i)} ).
        \end{align}
    \end{subequations}

\item[Elles vérifient la condition]
    Prouvons que les formules \eqref{EqDFxuGtH} se réduisent à \eqref{EqOIuWLQU} dans le cas des rectangles. Soit donc \( A=X_1\times X_2\) avec \( X_i\in\tribA_i\). Alors
    \begin{equation}
        A_1(y)=\{ x\in\Omega_1\tq (x,y)\in X_1\times X_2 \}
    \end{equation}
    et
    \begin{equation}
        \mu_1\big( A_1(y) \big)=\mtu_{X_2}(y)\mu_1(X_1),
    \end{equation}
    donc
    \begin{subequations}
        \begin{align}
            (\mu_1\otimes\mu_2)(A)&=\int_{\Omega_2}\mu_1\big( A_1(y) \big)d\mu_2(y)\\
            &=\int_{\Omega_2}\mu_1(X_1)\mtu_{X_2}(y)d\mu_2(y)\\
            &=\mu_1(X_1)\int_{\Omega_2}\mtu_{X_2}(y)d\mu_2(y)\\
            &=\mu_1(X_1)\mu_2(X_2).
        \end{align}
    \end{subequations}
    Pour cela nous avons utilisé le fait que l'intégrale de la fonction caractéristique d'un ensemble mesurable est la mesure de cet ensemble.
    \end{subproof}
\end{proof}

\begin{definition}[Produit d'espaces mesurés]  \label{DefUMlBCAO}
    Si \( (\Omega_i,\tribA_i,\mu_i)\) sont deux espaces mesurés, l'\defe{espace produit}{produit!espaces mesurés} est l'ensemble \( \Omega_1\times \Omega_2\) muni de la tribu produit \( \tribA_1\otimes \tribA_2\) de la définition~\ref{DefTribProfGfYTuR} et de la mesure produit \( \mu_1\otimes \mu_2\) définie par le théorème~\ref{ThoWWAjXzi}.
\end{definition}

\begin{remark}
    Il n'est pas garantit que la tribu \( \tribA_1\otimes\tribA_2\) soit la tribu la plus adaptée à l'ensemble \( S_1\times S_2\). Dans le cas de \( \eR^N\), il se fait que c'est le cas : en prenant des produits des boréliens sur \( \eR\) on obtient bien les boréliens sur \( \eR^N\), voir proposition~\ref{CorWOOOooHcoEEF}.
\end{remark}
