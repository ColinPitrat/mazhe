%+++++++++++++++++++++++++++++++++++++++++++++++++++++++++++++++++++++++++++++++++++++++++++++++++++++++++++++++++++++++++++
\section{Langages}
%+++++++++++++++++++++++++++++++++++++++++++++++++++++++++++++++++++++++++++++++++++++++++++++++++++++++++++++++++++++++++++

%---------------------------------------------------------------------------------------------------------------------------
\subsection{Alphabets et mots}
%---------------------------------------------------------------------------------------------------------------------------

\begin{definition}
    Un \defe{alphabet}{alphabet} est un ensemble fini de symboles appelés \defe{lettres}{lettres}.
\end{definition}

On utilise aussi parfois le terme \defe{vocabulaire}{alphabet} pour désigner un alphabet.

\begin{definition}
    Un \defe{mot}{mot} sur l'alphabet \( \Sigma \) est une suite finie et ordonnée, éventuellement vide, de lettres de \( \Sigma \). Le \defe{mot vide}{mot!mot vide} est toujours noté $\varepsilon$.
\end{definition}

% TODO: mettre la notation du mot vide à part ?
% TODO: utiliser une commande pour le mot vide ?

\begin{definition}
    La \defe{longueur d'un mot}{mot!longueur d'un mot} \( w \), noté \( |w| \), est le nombre de lettres constituant le mot \( w \). Le mot vide a une longueur de 0.
\end{definition}

Soit \( w \) un mot de longueur \( k \), on peut désormais noter \( w = w_1 \cdots w_k \), où chacun des \( w_i, 1 \leq i \leq k \) représente une lettre de \( w \). Par convention, si \( k = 0 \), alors le mot \( w \) est le mot vide.

\begin{definition}
    Soient \( w \) un mot sur l'alphabet \( \Sigma \) et \( a \in \Sigma \) une lettre, le \defe{nombre d'occurrences}{mot!nombre d'occurrences} de la lettre \( a \) dans le mot \( w \), noté \( |w|_a \), est le nombre de fois où apparaît la lettre \( a \) dans le mot \( w \), c'est-à-dire le cardinal de l'ensemble \( \{ i \mid w_i = a, 1 \leq i \leq |w| \} \).
\end{definition}

% TODO: je crois que la définition de «cardinal d'un ensemble» (éventuellement fini) n'apparaît nulle part, à moins que ça ne fasse partie de la théorie des ensembles qui est considérée comme acquise

\begin{definition}
    Soit \( \Sigma \) un alphabet, l'\defe{ensemble des mots non-vides}\ sur l'alphabet \( \Sigma \), noté \( \Sigma^+ \), est l'ensemble:
    \begin{equation}
        \Sigma^+ = \{ w = w_1 \ldots w_n, n > 0 \}
    \end{equation}
\end{definition}

\begin{definition}
    Soit \( \Sigma \) un alphabet, l'\defe{ensemble des mots}\ sur l'alphabet \( \Sigma \), noté \( \Sigma^* \), est l'ensemble:
    \begin{equation}
        \Sigma^* = \{ w = w_1 \ldots w_n, n \geq 0 \}
    \end{equation}
\end{definition}

Des deux définitions précédentes, on tire l'égalité suivante:

\begin{equation}
  \Sigma^* = \Sigma^+ \cup \{ \varepsilon \}
\end{equation}

\begin{definition}
    Soient \( \Sigma \) un alphabet et \( x, y \in \Sigma^* \) deux mots sur l'alphabet \( \Sigma \) de longueur respective \( n \) et \( m \), le \defe{produit}{produit!de mots} \( w \) de \( x \) et \( y \), noté \( x \cdot y \) est défini par \( w = x_1 \ldots x_n y_1 \ldots y_m \).
\end{definition}

Le produit est également appelé \defe{concaténation}{concaténation!de mots}.

\begin{proposition}[Longueur du produit de deux mots]
    % TODO: il y a peut-être moins fort qu'une proposition pour cette propriété

    La longueur du produit de deux mots \( x \) et \( y \) est la somme des longueurs des mots \( x \) et \( y \).

    \begin{equation}
        |x \cdot y| = |x| + |y|
    \end{equation}
\end{proposition}


\begin{proposition}[Monoïde \( (\Sigma^*, \cdot, \varepsilon) \)]
    L'ensemble \( \Sigma^* \) munie de l'opération produit d'élément neutre \( \varepsilon \) est un monoïde.
\end{proposition}

\begin{proof}
    Soient \( x, y, z \in \Sigma^* \), avec les définitions précédentes, on peut vérifier facilement que:
    \begin{itemize}
    \item
      le produit est une loi interne: $x \cdot y \in \Sigma^*$;
    \item
      le produit est associatif: $x \cdot (y \cdot z) = (x \cdot y) \cdot z$;
    \item
      $\varepsilon$ est l'élément neutre du produit: $x \cdot \varepsilon = \varepsilon \cdot x = x$.
    \end{itemize}
\end{proof}

Le produit n'est pas commutatif.

% TODO: à partir de là, on peut aussi parler de monoïde libre
% Voir https://fr.wikipedia.org/wiki/Mono%C3%AFde#Bases_et_mono%C3%AFdes_libres et aussi (surtout) https://en.wikipedia.org/wiki/Free_monoid
% La notion de monoïde libre est très liée aux définitions précédentes, donc si jamais on souhaite développer la notion de monoïde libre, il semble naturel de le faire ici-même plutôt que dans 42_nombres (où doit être défini monoïde). La page wikipedia anglophone est un bon point d'appui pour développer cette partie.

\begin{definition}
    Soient \( \Sigma \) un alphabet et \( w \in \Sigma^* \), la \defe{puissance}{puissance!d'un mot} \( n \)\ieme d'un mot \( w \), notée \( w^n \), est définie par:
    \[
    w^n =
    \begin{cases}
        \varepsilon     & \text{si $n = 0$} \\
        w \cdot w^{n-1} & \text{si $n > 0$}
    \end{cases}
    \]
\end{definition}


%---------------------------------------------------------------------------------------------------------------------------
\subsection{Langage}
%---------------------------------------------------------------------------------------------------------------------------

\begin{definition}
    Un \defe{langage}{langage} sur un alphabet \( \Sigma \) est un sous-ensemble de \( \Sigma^* \). C'est un ensemble de mots sur l'alphabet \( \Sigma \).
\end{definition}

Un langage étant défini comme un ensemble, on peut appliquer toutes les notions de la théorie des ensembles aux langages.

\begin{definition}
    Le \defe{langage vide}{langage!vide}, noté \( \varnothing \) est le langage qui ne contient aucun mot.
\end{definition}

\begin{definition}
    Le \defe{langage unité}{langage!unité} est le langage qui contient uniquement le mot vide: \( \{ \varepsilon \} \).
\end{definition}


\begin{definition}
    Soient \( \Sigma \) un alphabet et \( L_1, L_2 \subseteq \Sigma^* \) deux langages sur l'alphabet \( \Sigma \), on définit le \defe{produit}{produit!de langages} \( L \) de \( L_1 \) et \( L_2 \), noté \( L_1 . L_2 \) par:
    \begin{equation}
        L = L_1 . L_2 = \{ u_1 \cdot u_2, u_1 \in L_1, u_2 \in L_2 \}
    \end{equation}
\end{definition}

Le produit de langages est également appelé \defe{concaténation}{concaténation!de langages}. Il ne faut pas confondre le produit de langage avec le produit cartésien de deux ensembles. Le langage unité est l'élément neutre du produit de langages.

\begin{proposition}[Distributivité du produit de langage par rapport à l'union]
    Le produit de langage est distributif par rapport à l'union. Soient \( \Sigma \) un alphabet et \( L_1, L_2, L_3 \subseteq \Sigma^* \), alors:
    \begin{equation}
        L_1 . (L_2 \cup L_3) = (L_1 . L_2) \cup (L_1 . L_3) \text{ et } (L_1 \cup L_2) . L_3 = (L_1 . L_3) \cup (L_2 . L_3)
    \end{equation}
\end{proposition}

\begin{proof}
    Soit $w \in L_1 . (L_2 \cup L_3)$, montrons que $w \in (L_1 . L_2) \cup (L_1 . L_3)$.
    $\exists w_1 \in L_1, w' \in L_2 \cup L_3, w = w_1 \cdot w'$. Donc $w' \in L_2$ ou $w' \in L_3$.
    Si $w' \in L_2$, alors $w = w_1 \cdot w' \in L_1 . L_2$.
    Si $w' \in L_3$, alors $w = w_1 \cdot w' \in L_1 . L_3$.
    Donc, $w \in (L_1 . L_2) \cup (L_1 . L_3)$.
    Donc $L_1 . (L_2 \cup L_3) \subseteq (L_1 . L_2) \cup (L_1 . L_3)$.

    Soit $w \in (L_1 . L_2) \cup (L_1 . L_3)$, montrons que $w \in L_1 . (L_2 \cup L_3)$.
    $w \in L_1 . L_2$ ou $w \in L_1 . L_3$.
    Si $w \in L_1 . L_i, i \in \{ 2, 3\}$ alors $\exists w_1 \in L_1, w_i \in L_i, w = w_1 \cdot w_i$.
    Donc $w \in L_1 . (L_2 \cup L_3)$.
    Donc, $(L_1 . L_2) \cup (L_1 . L_3) \subseteq L_1 . (L_2 \cup L_3)$

    Donc $(L_1 . L_2) \cup (L_1 . L_3) = L_1 . (L_2 \cup L_3)$

    L'autre partie de la proposition se montre de manière analogue.
\end{proof}


\begin{definition}
    Soient \( \Sigma \) un alphabet et \( L \subseteq \Sigma^* \), la \defe{puissance}{puissance!d'un langage} \( n \)\ieme du langage \( L \), notée \( L^n \) est définie par:
    \[
      L^n =
      \begin{cases}
        \{ \varepsilon \} & \text{si $n = 0$} \\
        L . L^{n-1}       & \text{si $n > 0$}
      \end{cases}
    \]
\end{definition}


% https://fr.wikipedia.org/wiki/%C3%89toile_de_Kleene
\begin{definition}
    L'\defe{étoile de Kleene}{étoile de Kleene} est un opérateur unaire noté \( * \). L'\defe{itéré}{langage!itéré} d'un langage \( L \), noté \( L^* \), est l'application de l'étoile de Kleene à un langage \( L \) et est défini par:
    \begin{equation}
        L^* = \bigcup_{i \geq 0} L^i
    \end{equation}
\end{definition}

En particulier, on remarque que le mot vide fait toujours partie de l'itéré d'un langage, y compris quand ce même langage ne contient pas le mot vide.

\begin{definition}
  L'\defe{itéré strict}{langage!itéré strict} d'un langage \( L \), noté \( L^+ \), est défini par:
    \begin{equation}
        L^+ = \bigcup_{i > 0} L^i
    \end{equation}
\end{definition}

\begin{proposition}[Relations entre itéré et itéré strict]
  Soit \( L \) un langage, alors on a:
  \begin{equation}
    L^* = L^+ \cup \{ \varepsilon \}
  \end{equation}
  \begin{equation}
    L^+ = L.L^+ = L^+.L
  \end{equation}
\end{proposition}

