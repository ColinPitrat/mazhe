% This is part of Mes notes de mathématique
% Copyright (c) 2006-2018
%   Laurent Claessens, Carlotta Donadello
% See the file fdl-1.3.txt for copying conditions.

%++++++++++++++++++++++++++++++++++++++++++++++++++++++++++++++++++++++++++++++++++++++++
\section{Fonctions de classe $\mathcal{C}^1$}
%++++++++++++++++++++++++++++++++++++++++++++++++++++++++++++++++++++++++++++++++++++++++++++++++++++++++++++++++++++++++++++++

Soit $f$ une fonction différentiable de $U$, ouvert de $\eR^m$, dans $\eR^n$. L'application différentielle de $f$ est une application  de $\eR^m$ dans $\mathcal{L}(\eR^m, \eR^n)$
\begin{equation}
  \begin{array}{rccc}
    df : & \eR^m & \to & \mathcal{L}(\eR^m, \eR^n)\\
& a& \mapsto & df_a.
  \end{array}
\end{equation}
Nous savons que $\mathcal{L}(\eR^m, \eR^n)$ est un espace vectoriel normé avec la définition~\ref{DefDQRooVGbzSm}. Si $T$ est un élément dans $\mathcal{L}(\eR^m, \eR^n)$ alors la norme de $T$ est définie par
\[
\|T\|_{\mathcal{L}(\eR^m, \eR^n)}=\sup_{x\in\eR^m} \frac{\|T(x)\|_n}{\|x\|_m}=\sup_{\begin{subarray}{l}
    x\in\eR^m\\
\|x\|_m\leq 1
  \end{subarray}} \|T(x)\|_n.
\]

Lorsqu'il existe un $M>0$ tel que $\| df(a) \|_{\aL(\eR^m,\eR^n)}<M$ pour tout $a$ dans $U$, nous disons que la différentielle de $f$ est \defe{bornée}{bornée!différentielle} sur $U$.

\begin{definition}
	La fonction $f$ est dite \defe{de classe $\mathcal{C}^1$}{fonction!de classe  $\mathcal{C}^1$} de $U\subset\eR^m$  dans $\eR^n$ si son application différentielle $df$ est continue de $\eR^m$ dans $\mathcal{L}(\eR^m, \eR^n)$. Nous écrivons $f\in\mathcal{C}^1(U,\eR^n)$\nomenclature{$\aC^1(U,\eR^n)$}{Les applications une fois continument dérivables}.
\end{definition}

\begin{proposition}		\label{PropDerContCun}
	Une fonction \( f\colon U\to \eR^n\) où \( U\) est ouvert dans \( \eR^m\) est de classe \( C^1\) si et seulement si les dérivées partielles de $f$ existent et sont continues.
\end{proposition}

\begin{proof}
	Supposons que les dérivées partielles de $f$ existent et sont continues. Nous savons alors déjà par la proposition~\ref{Diff_totale} que la fonction $f$ est différentiable et qu'elle s'exprime sous la forme
	\[
		df_a(h)=\sum_{i=1}^{m}\partial_if (a)h_i, \qquad \forall a \in U,\,\forall h\in\eR^m.
	\]
	Pour montrer que $df$ est continue, nous devons montrer que la quantité $\| df(x)-df(a) \|_{\aL(\eR^m,\eR^n)}$ peut être rendue arbitrairement petite si $\| x-a \|_m$ est rendu petit. Nous avons
	\begin{equation}
		\begin{aligned}
			\| df_x-df_a \|_{\aL}&=\sup_{\| h \|=1}\| df_x(h)-df_a(h) \|\\
			&=\sup_{\| h \|_m=1}\left\|\sum_{i=1}^{m}\left(\partial_if (x)-\partial_if (a)\right)h_i\right\|_n\leq\\
			&\leq\sup_{\| h \|_m=1}\sum_{i=1}^{m} \left\|\left(\partial_if (x)-\partial_if (a)\right)\right\|_n|h_i|\leq\\
			&\leq\sup_{\| h \|_m=1} \|h\|_\infty\sum_{i=1}^{m} \left\|\left(\partial_if (x)-\partial_if (a)\right)\right\|_n\\
			&\leq \sum_{i=1}^m\| \partial_if(x)-\partial_if(a) \|.
		\end{aligned}
	\end{equation}
	Dans ce calcul, nous avons utilisé le fait que si $\| h \|_m\leq 1$, alors $\| h \|_{\infty}\leq 1$. Étant donné la continuité de $\partial_if$, la dernière ligne peut être rendue arbitrairement petite lorsque $x$ est proche e $a$.

Supposons maintenant que $f$ soit dans $\mathcal{C}^1(U,\eR^n)$. Alors
\[
\left\|\partial_if (x)-\partial_if (a)\right\|_n= \left\|df(x).e_i-df(a).e_i\right\|_n \leq  \left\|df(x)-df(a)\right\|_{\mathcal{L}(\eR^m,\eR^n)},
\]
la continuité de $df$ implique donc celle de $\partial_i f$ pour tout $i$ dans $\{1,\ldots,m\}$.
\end{proof}
\begin{proposition}
  Soient $U$ un ouvert de $\eR^m$ et $V$ un ouvert de $\eR^n$. Soient $f: U\to V$  dans $\mathcal{C}^1(U,V)$ et $g: V \to \eR^p$ dans $\mathcal{C}^1(V,\eR^n)$.  Alors la fonction composée $g\circ f: U\to \eR^p $ est dans $\mathcal{C}^1(U,\eR^p)$.
\end{proposition}
\begin{proof} On fixe $a$ dans $U$
  \begin{equation}
    \begin{aligned}
     \big\|d(g\circ f)(x)&-d(g\circ f)(a)\big\|_{\mathcal{L}(\eR^m,\eR^p)}\\
     &=\left\|dg(f(x))\circ df(x)-dg(f(a))\circ df(a)\right\|_{\mathcal{L}(\eR^m,\eR^p)}\leq\\
&\leq \left\|\left(dg(f(x))-dg(f(a))\right)\circ df(x)\right\|_{\mathcal{L}(\eR^m,\eR^p)}+\\
&\quad+ \left\|dg(f(a))\circ \left(df(x)-df(a)\right)\right\|_{\mathcal{L}(\eR^m,\eR^p)}\leq\\
&\leq \left\|dg(f(x))-dg(f(a))\right\|_{\mathcal{L}(\eR^n,\eR^p)}\left\| df(x)\right\|_{\mathcal{L}(\eR^m,\eR^n)}+\\
&\quad+ \left\|dg(f(a))\right\|_{\mathcal{L}(\eR^n,\eR^p)}\left\| df(x)-df(a)\right\|_{\mathcal{L}(\eR^n,\eR^p)}.\\
    \end{aligned}
  \end{equation}
On peut conclure en passant à la limite $x\to a$ parce que les fonctions $f$, $g$, $df$ et $dg$ sont continues, de telle sorte que
\begin{equation}
	\begin{aligned}[]
		\lim_{x\to a} dg\big( f(x) \big)=dg\big( f(a) \big)\\
		\lim_{x\to a} df(x)=df(a).
	\end{aligned}
\end{equation}
\end{proof}

\begin{remark}
  On peut prouver le même résultat en utilisant la continuité de l'application bilinéaire
\begin{equation}
  \begin{array}{rccc}
    \circ : & \mathcal{C}^1(U,V)\times\mathcal{C}^1(V,\eR^p)  & \to & \mathcal{L}(U, \eR^p)\\
& (T,S)& \mapsto & T\circ S.
  \end{array}
\end{equation}
\end{remark}


%++++++++++++++++++++++++++++++++++++++++++++++++++++++++++++++++++++++++++++++++++++++++++++++++++++++++++++++++++++++
\section{Théorèmes des accroissements finis}		\label{SecThoAccrsFinis}
%++++++++++++++++++++++++++++++++++++++++++++++++++++++++++++++++++++++++++++++++++++++++++++++++++++++++++++++++++++++

Nous avons déjà démontré (lemme~\ref{LemdfaSurLesPartielles}) que si $f$ est différentiable au point $x$ alors  $df_x(u)=\partial_uf(x)$. Une importante conséquence est le théorème des accroissements finis
\begin{theorem}[Accroissements finis, inégalité de la moyenne]\label{val_medio_2}
   Soit $U$ un ouvert dans $\eR^m$ et soit $f:U\to\eR^n$ une fonction différentiable. Soient $a$ et $b$ deux points dans $U$, $a\neq b$, tels que le segment $[a,b]$ soit contenu dans $U$. Alors
   \begin{equation}
        \|f(b)-f(a)\|_n\leq \sup_{x\in[a,b]}\|df(x)\|_{\mathcal{L}(\eR^m,\eR^n)}\|b-a\|_m.
   \end{equation}
\end{theorem}
\index{application!différentiable}
\index{inégalité!de la moyenne}
\index{théorème!accroissements finis!forme générale}

\begin{proof}
 On utilise le théorème~\ref{val_medio_1} et le fait que
\[
\|\partial_u f(x)\|_n\leq \|df(x)\|_{\mathcal{L}(\eR^m,\eR^n)}\|u\|_m,
\]
pour tout $u$ dans $\eR^m$.
\end{proof}

La proposition suivante est une application fondamentale du théorème des accroissements finis~\ref{val_medio_2}.
\begin{proposition}		\label{PropAnnulationEtConstance}
	Soit $U$ un ouvert connexe par arcs de $\eR^m$ et une fonction $f\colon U\to \eR^n$. Les conditions suivantes sont équivalentes :
	\begin{enumerate}
		\item\label{ItemPropCstDiffZeroi}
			$f$ est constante;
		\item\label{ItemPropCstDiffZeroii}
			$f$ est différentiable et $df(a)=0$ pour tout $a\in U$;
		\item\label{ItemPropCstDiffZeroiii}
			les dérivées partielles $\partial_1f,\ldots,\partial_mf$ existent et sont nulles sur $U$.
	\end{enumerate}
\end{proposition}
\index{connexité!par arc!fonction différentiable}
\index{différentiabilité}

\begin{proof}
	Nous allons démonter les équivalences en plusieurs étapes. D'abord~\ref{ItemPropCstDiffZeroi} $\Rightarrow$~\ref{ItemPropCstDiffZeroii}, puis~\ref{ItemPropCstDiffZeroii} $\Rightarrow$~\ref{ItemPropCstDiffZeroiii}, ensuite~\ref{ItemPropCstDiffZeroiii} $\Rightarrow$~\ref{ItemPropCstDiffZeroii} et enfin~\ref{ItemPropCstDiffZeroii} $\Rightarrow$~\ref{ItemPropCstDiffZeroi}.

	Commençons par montrer que la condition~\ref{ItemPropCstDiffZeroi} implique la condition~\ref{ItemPropCstDiffZeroii}. Si $f(x)$ est constante, alors la condition \eqref{EqCritereDefDiff} est vite vérifiée en posant $T(h)=0$.

	Afin de voir que la condition~\ref{ItemPropCstDiffZeroii} implique la condition~\ref{ItemPropCstDiffZeroiii}, remarquons d'abord que la différentiabilité de $f$ implique que les dérivées partielles existent (proposition~\ref{diff1}) et que nous avons l'égalité $df(a).u=\sum_iu_i\partial_if(a)$ pour tout $u\in\eR^m$ (lemme~\ref{LemdfaSurLesPartielles}). L'annulation de $\sum_iu_i\partial_if(a)$ pour tout $u$ implique l'annulation des $\partial_if(a)$ pour tout $i$.

	Prouvons maintenant que la propriété~\ref{ItemPropCstDiffZeroiii} implique la propriété~\ref{ItemPropCstDiffZeroii}. D'abord, par la proposition~\ref{Diff_totale}, l'existence et la continuité des dérivées partielles $\partial_if(a)$ implique la différentiabilité de $f$. Ensuite, la formule $df(a).u=\sum_i u_i\partial_if(a)$ implique que $df(a)=0$.


	Il reste à montrer que~\ref{ItemPropCstDiffZeroii} implique la condition~\ref{ItemPropCstDiffZeroi}, c'est à dire que l'annulation de la différentielle implique la constance de la fonction. C'est ici que nous allons utiliser le théorème des accroissements finis. En effet, si $a$ et $b$ sont des points de $U$, le théorème~\ref{val_medio_2} nous dit que
	\begin{equation}
		\|f(b)-f(a)\|_n\leq \sup_{x\in[a,b]}\|df(x)\|_{\mathcal{L}(\eR^m,\eR^n)}\|b-a\|_m.
	\end{equation}
	Mais $\| df(x) \|=0$ pour tout $x\in U$, donc ce supremum est nul et $f(b)=f(a)$, ce qui signifie la constance de la fonction.
\end{proof}

%\begin{proof}
%  \begin{itemize}
%  \item Le théorème~\ref{val_medio_2} nous dit que si la différentielle de $f$ est nulle alors $f$ est constante sur chaque segment contenu dans $U$. Cela nous dit que $f$ est constante sur chaque boule contenue dans $U$, donc $f $ est localement constante. Il est possible de démontrer que toute fonction localement constante sur un connexe est constante.
%\item Si toutes les dérivées partielles $\partial_1 f, \ldots, \partial_m f $ existents et sont identiquement nulles sur $U$ alors $f$ est différentiable et sa différentielle est identiquement nulle. On utilise la première partie de la preuve pour conclure.
%  \end{itemize}
%\end{proof}

%+++++++++++++++++++++++++++++++++++++++++++++++++++++++++++++++++++++++++++++++++++++++++++++++++++++++++++++++++++++++++++
\section{Fonctions Lipschitziennes}
%+++++++++++++++++++++++++++++++++++++++++++++++++++++++++++++++++++++++++++++++++++++++++++++++++++++++++++++++++++++++++++


\begin{definition}      \label{DEFooQHVEooDbYKmz}
    Soient \( (E,d_E)\) et \( (F,d_F)\) deux espaces métriques\footnote{Pour rappel, les espaces métriques sont définis par la définition~\ref{DefMVNVFsX} et le théorème~\ref{ThoORdLYUu}; je précise que nous ne supposons pas que \( E\) soit vectoriel; en particulier il peut être un ouvert de \( \eR^n\).}, \( f\colon E\to F\) une application et un réel \( k\) strictement positif. Nous disons que \( f\) est \defe{Lipschitzienne}{Lipschitzienne} de constante $k$ sur \( E\) si pour tout \( x,y\in E\),
    \begin{equation}
        d_F\big( f(x)-f(y) \big)\leq kd_E(x,y).
    \end{equation}
\end{definition}
%TODO : faire la chasse aux endroits où cette définition devrait être référencée.
Soit \( f\) une fonction \( k\)-Lipschitzienne. Si \( y\in \overline{ B(x,\delta)}\) alors \( \| x-y \|\leq\delta\) et donc \( \big\| f(x)-f(y) \big\|\leq k\delta\). Cela signifie que la condition Lipschitz pour s'énoncer en termes de boules fermées par
\begin{equation}    \label{EqDZvtUbn}
    f\big( \overline{ B(x,\delta) } \big)\subset \overline{  B\big( f(x),k\delta \big) }
\end{equation}
tant que \( \overline{ B(x,\delta) } \) est contenue dans le domaine sur lequel \( f\) est Lipschitz.

\begin{proposition}
  Soit  $U$ un ouvert convexe  de $\eR^m$, et soit $f:U\to \eR^n$ une fonction différentiable. La fonction $f$ est Lipschitzienne sur $U$ si et seulement si $df$ est bornée sur $U$.
\end{proposition}
\begin{proof}
	Le fait que l'application différentielle $df$ soit bornée signifie qu'il existe un $M>0$ dans $\eR$ tel que $\|df_a\|_{\mathcal{L}(\eR^m,\eR^n)}\leq M$, pour tout $a$ dans $U$. Si cela est le cas, alors le théorème~\ref{val_medio_2} et la convexité\footnote{La convexité de $U$ sert à assurer que la droite reliant $a$ à $b$ est contenue dans $U$; c'est ce que nous utilisons dans la démonstration du théorème~\ref{val_medio_2}.} de $U$ impliquent évidemment que $f$ est de Lipschitz de constante plus petite ou égale à $M$.

	Inversement, si $f$ est Lipschitz de constante $k$, alors pour tout $a$ dans $U$ et $u$ dans $\eR^m$ on a
	\[
		\left\|\frac{f(a+tu)-f(a)}{t}\right\|_n\leq k \|u\|_m,
	\]
	En passant à la limite pour $t\to 0$ on a
	\[
		\|\partial_u f(a)\|_n=\|df_a(u)\|_n\leq k \|u\|_m,
	\]
	donc la norme de $df_a$ est majorée par $k$ pour tout $a$ dans $U$.
\end{proof}

Notez cependant qu'une fonction peut être Lipschitzienne sans être différentiable.

\begin{proposition} \label{PropFZgFTEW}
    Une fonction Lipschitzienne \( f\colon \eR\to \eR\) est continue.
\end{proposition}

\begin{proof}
    Nous utilisons la caractérisation de la continuité donnée par le théorème~\ref{ThoESCaraB}. Prouvons donc la continuité en \( a\in \eR\). Pour tout \( x\) nous avons
    \begin{equation}
        \big| f(x)-f(a) \big|\leq k| x-a |.
    \end{equation}
    Si \( \epsilon>0\) est donné, il suffit de prendre \( \delta<\frac{ \epsilon }{ k }\) pour avoir
    \begin{equation}
        \big| f(x)-f(a) \big|\leq k\frac{ \epsilon }{ k }=\epsilon.
    \end{equation}
    Donc \( f\) est continue en \( a\).
\end{proof}

\begin{definition}      \label{DefJSFFooEOCogV}
    Une fonction
    \begin{equation}
        \begin{aligned}
            f\colon \eR^n\times \eR^m&\to \eR^p \\
            (t,y)&\mapsto f(t,y)
        \end{aligned}
    \end{equation}
    est \defe{localement Lipschitz}{Lipschitz!localement} en \( y\) au point \( (t_0,y_0)\) s'il existe des voisinages \( V\) de \( t_0\) et \( W\) de \( y_0\) et un nombre \( k>0\) tels que pour tout \( (t,y)\in V\times W\) on ait
    \begin{equation}
        \big\| f(t_0,y_0)-f(t,y) \big\|\leq k\| y-y_0 \|.
    \end{equation}
    La fonction est localement Lipschitz sur un ouvert \( U\) de \( \eR^n\times \eR^m\) si elle est localement Lipschitz en chaque point de \( U\).
\end{definition}

\begin{normaltext}      \label{NORMooYNRAooBgobcK}
    Autrement dit, une fonction est localement Lipschitzienne en sa deuxième variable lorsque tout point admet un voisinage sur lequel elle est Lipschitzienne.
\end{normaltext}

\begin{proposition} \label{PROPooVZSAooUneOQK}
    Une application Lipschitz\footnote{Définition~\ref{DEFooQHVEooDbYKmz}.} est uniformément continue.
\end{proposition}

\begin{proposition}     \label{PropGIBZooVsIqfY}
    Si \( f\) et \( g\) sont deux fonctions localement Lipschitz alors \( f+g\) l'est.
\end{proposition}

\begin{proof}
    Il s'agit d'un simple calcul avec une majoration standard :
    \begin{subequations}
        \begin{align}
            \| (f+g)(t_0,y_0)-(f+g)(t,y) \|&\leq \|  f(t_0,y_0)-f(t,y)  \|+\| g(t_0,y_0)-g(t,y) \|\\
            &\leq k_f\| y-y_0 \|+k_g\| y-y_0 \|\\
            &=(k_f+k_g)\| y-y_0 \|.
        \end{align}
    \end{subequations}
\end{proof}

\begin{lemma}   \label{LemCFZUooVqZmpc}
    La fonction donné par
    \begin{equation}
        f(t, (x,y) )=xy
    \end{equation}
    est localement Lipschitz en tout point.
\end{lemma}

\begin{proof}
    Nous avons la majoration classique
    \begin{equation}
        | f\big(t,(x_0,y_0)\big)-f\big( t,(x,y) \big) |=| x_0y_0-xy |\leq| x_0y_0-x_0y |+| x_0y-xy |\leq | x_0 || y_0-y |+| y || x_0-x |.
    \end{equation}
    Vu que nous parlons de fonction \emph{localement Lipschitzienne}, nous pouvons majorer \( | y |\) et \( | x_0 |\) par un même nombre \( k\) dans un voisinage de \( (x_0,y_0)\). Cela donne
    \begin{equation}
        | f\big(t,(x_0,y_0)\big)-f\big( t,(x,y) \big) |\leq k\big( | y_0-y |+| x_0-x | \big)\leq \sqrt{2}k\| \begin{pmatrix}
            x_0-x    \\
            y_0-y
        \end{pmatrix}\|.
    \end{equation}
    Nous avons utilisé l'équivalence de norme de la proposition~\ref{PropLJEJooMOWPNi}\ref{ItemABSGooQODmLNi}.
\end{proof}



%++++++++++++++++++++++++++++++++++++++++++++++++++++++++++++++++++++++++++++++++++++++++
\section{Différentielles d'ordre supérieur}		\label{SecDiffOrdSup}
%++++++++++++++++++++++++++++++++++++++++++++++++++++++++++++++++++++++++++++++++++++++++++++++++++++++++++++++++++++++++++++++
\begin{definition}
	Soit $U$ un ouvert de $\eR^m$ et  $f:U\subset\eR^m\to \eR^n$ une fonction. La fonction $f$ est dite \defe{deux fois différentiable}{différentiable!deux fois} au point $a$ dans $U$,  si $f$ est différentiable dans un voisinage de $a$, et sa différentielle $df$ est différentiable au point $a$ en tant que application de $U$ dans $\mathcal{L}(\eR^m, \eR^n)$.

La fonction $f$ sera dite deux fois différentiable sur l'ensemble $U$ si elle est deux fois différentiable en chaque point de $U$.

\end{definition}

%---------------------------------------------------------------------------------------------------------------------------
\subsection{Identification des espaces d'applications multilinéaires}
%---------------------------------------------------------------------------------------------------------------------------

La différentielle de la différentielle de $f$ est notée
\[
d(df)(a)=d^2f(a),
\]
et est une application de $U$ dans $\mathcal{L}(\eR^m,\mathcal{L}(\eR^m, \eR^n) )$. Comme on a vu dans la proposition~\ref{isom_isom}, l'espace $\mathcal{L}(\eR^m,\mathcal{L}(\eR^m, \eR^n) )$ est isométriquement isomorphe à l'espace $\mathcal{L}(\eR^m\times\eR^m, \eR^n )$. On verra comment cette propriété  est utilisé dans l'exemple~\ref{bilin_2diff}.


Soient \( V\) et \( W\) deux espaces vectoriel normés de dimension finie et \( \mO\) un ouvert autour de \( x\in V\). D'une part l'espace des applications linéaires \( \aL(V,W)\) est lui-même un espace vectoriel normé de dimension finie, et on peut identifier \(  \aL\big( V,\aL^{(k)}(V,W) \big)\)\nomenclature[Y]{\( \aL^{(n)}(V,W)\)}{L'espace des applications \( n\)-linéaires \( V^n\to W\)} avec \( \aL^{(k+1)}(V,W)\), ce qui nous permet de dire que la \( k\)\ieme\ différentielle est une application
\begin{equation}
    d^kf\colon \mO\to \aL^{(k)}(V,W).
\end{equation}
Plus précisément, l'identification se fait de la façon suivante : si \( \omega\in \aL\big( V,\aL^{(k)}(V,W) \big)\), alors \( \omega\) vu dans \( \aL^{(k+1)}(V,W)\) est définie par
\begin{equation}
    \omega(u_1,\ldots, u_{k+1})=\omega(u_1)(u_2,\ldots, u_{k+1}).
\end{equation}

Cela étant posé nous pouvons donner les définitions.

%---------------------------------------------------------------------------------------------------------------------------
\subsection{Fonctions différentiables plusieurs fois}
%---------------------------------------------------------------------------------------------------------------------------

\begin{definition}[\cite{ZCKMFRg}]  \label{DefPNjMGqy}
    La fonction \( f\colon \mO\subset V\to W\) est
    \begin{enumerate}
        \item
            de classe \( C^0\) si elle est continue,
        \item
            de classe \( C^1\) si \( df\colon \mO\to \aL(V,W)\) est continue,
        \item
            de classe \( C^k\) si \( d^kf\colon \mO\to \aL^{(k)}(V,W)\) est continue,
        \item
            de classe \(  C^{\infty}\) si \( f\) est dans \( \bigcap_{k=0}^{\infty}C^k(V,W)\).
    \end{enumerate}
\end{definition}
\index{application!différentiable}
\index{application!de classe \( C^k\)}

\begin{definition}
    Un \defe{\( C^k\)-difféomorphisme}{difféomorphisme!de classe $C^k$} est une application inversible de classe \( C^k\) dont l'inverse est également de classe \( C^k\).
\end{definition}

\begin{example}\label{bilin_2diff}
	Soit $B:\eR^m\times \eR^m\to\eR^n$ une application bilinéaire. On définit $f:\eR^m\to\eR^n$ par $f(x)=B(x,x)$. Le lemme~\ref{bilin_diff} nous dit que $B$ est différentiable. Cela implique la différentiabilité de $f$. Pour trouver la différentielle de la fonction $f$, nous écrivons $f=B\circ s$ où $s\colon \eR^m\to \eR^m\times\eR^m$ est l'application $s(x)=(x,x)$. En utilisant la règle de différentiation de fonctions composées,
	\begin{equation}
		df(a)=dB\big( s(a) \big)\circ ds(a).
	\end{equation}
	Mais $ds(a).u=(u,u)$ parce que $s(a+h)-s(a)-(h,h)=0$. Par conséquent,
	\begin{equation}		\label{EqdBsaExp}
		df(a).u=dB\big( s(a) \big)(u,u)=B(u,a)+B(a,u)
	\end{equation}
	où nous avons utilisé la formule du lemme~\ref{bilin_diff}. La formule \eqref{EqdBsaExp} peut être écrite sous la forme compacte
	\begin{equation}
		df(a)=B(\cdot,a)+B(a,\cdot)
	\end{equation}
	La fonction $df(a)$ ainsi écrite est linéaire par rapport à $a$, donc différentiable. En outre elle coïncide avec sa différentielle, comme on a vu dans le remarque~\ref{rk_lin}, au sens que la différentielle de $df$ au point $a$ sera l'application que à chaque $x$ dans $\eR^m$ associe l'application linéaire $B(x,\cdot)+B(\cdot, x)$. On voit bien que $d^2f$ au point $a$ est une application de $\eR^m$ vers l'espace des applications linéaires $\mathcal{L}(\eR^m, \eR^n)$. On peut utiliser d'autre part l'isomorphisme des espaces $\mathcal{L}(\eR^m,\mathcal{L}(\eR^m, \eR^n) )$ et $\mathcal{L}(\eR^m\times\eR^m, \eR^n )$ et dire que, une fois que $a$ est fixé, l'application $d^2f(a)$ est une application bilinéaire sur $\eR^m\times\eR^m$. On écrit alors $d^2f(a)(x,y)=B(x,y)+B(y,x)$.
\end{example}

%---------------------------------------------------------------------------------------------------------------------------
\subsection{Différentielle seconde, fonction de classe \( C^2\)}
%---------------------------------------------------------------------------------------------------------------------------

Une condition nécessaire et suffisante pour l'existence de la différentielle seconde est la suivante
\begin{proposition}
   Soit $U$ un ouvert de $\eR^m$ et  $f:U\subset\eR^m\to \eR^n$ une fonction. La fonction $f$ est deux fois différentiable au point $a$ si et seulement si les dérivées partielles $\partial_1 f, \ldots, \partial_m f $ sont différentiables en $a$.
\end{proposition}
Cela veut dire, en particulier, que $f$ est deux fois différentiable si et seulement si ses dérivées partielles secondes, $\partial_i\partial_j f$, pour tout couple d'indices $i,j$  dans $\{1,\ldots, m\}$, existent et sont continues. Pour les différentielles d'ordre supérieur on a la proposition suivante.

La différentielle seconde dans l'exemple ~\ref{bilin_2diff} est symétrique, c'est à dire que $d^2f(a)(x_1,x_2)=d^2f(a)(x_2,x_1)$. En fait toute différentielle seconde est symétrique.


\begin{theorem}[Schwarz]\label{Schwarz}
 Soit $U$ un ouvert de $\eR^m$ et  $f:U\subset\eR^m\to \eR^n$ une fonction de classe $\mathcal{C}^2$. Alors, pour tout couple $i,j$ d'indices dans $\{1,\ldots, m\}$ et pour tout point $a$ dans $U$, on a
\[
\frac{\partial^2 f}{\partial  x_i\partial x_j}(a)=\frac{\partial^2 f}{\partial  x_j\partial x_i}(a).
\]
\end{theorem}
\begin{proof}
  Pour simplifier l'exposition nous nous limitons ici au cas $m=2$. Soit $(h,g)$ un vecteur fixé dans $\eR^2$. Pour tout  $v=(x,y)$ dans $\eR^2$ on note
  \begin{equation}
    \begin{array}{c}
      \Delta_h f(v)=f(v+he_1) -f(v) = f(x+h,y)-f(x,y),\\
      \Delta_g f(v)=f(v+ge_2) -f(v) = f(x,y+g)-f(x,y),\\
    \end{array}
  \end{equation}
Nous avons
\begin{equation}
  \begin{array}{c}
   \Delta_g   \Delta_h f(v)=\left(f(x+h,y+g)-f(x,y+g)\right)-\left(f(x+h,y)-f(x,y)\right),\\
   \Delta_h   \Delta_g f(v)=\left(f(x+h,y+g)-f(x+h,y)\right)-\left(f(x,y+g)-f(x,y)\right),
  \end{array}
\end{equation}
donc,
\begin{equation}
  \frac{1}{g} \Delta_g  \left(\frac{1}{h} \Delta_h f(v)\right) = \frac{1}{h} \Delta_h \left(\frac{1}{g} \Delta_g f(v)\right).
\end{equation}
On utilise alors le théorème des accroissements finis~\ref{ThoAccFinis}
\begin{equation}
\frac{1}{h} \Delta_h f(v)=\frac{1}{h}\big(f(x+h,y)-f(x,y)\big)=\frac{1}{h}\partial_1f(x+t_1h,y )h=\partial_1f(x+t_1h, y),
\end{equation}
pour un certain $t_1$ dans $]0,1[$. De même on obtient
\[
\frac{1}{g} \Delta_g f(v)= \partial_2 f(x, y+t_2g),
\]
pour un certain $t_2$ dans $]0,1[$. Alors
 \begin{equation}
  \frac{1}{g} \Delta_g  \big(\partial_1f(x+t_1h, y)\big) = \frac{1}{h} \Delta_h \big(\partial_2 f(x, y+t_2g)\big).
\end{equation}
En appliquant encore une fois le théorème des accroissements finis on a
 \begin{equation}
  \partial_2\partial_1f(x+t_1h, y+s_1g) = \partial_1\partial_2 f(x+s_2h, y+t_2g).
\end{equation}
Il suffit maintenant de passer à la limite pour $(h,g) \to (0,0)$ et de se souvenir du fait que $f$ est $\mathcal{C}^2$ seulement si ses dérivées partielles secondes sont continues pour avoir $\partial_2\partial_1f(v)=\partial_1\partial_2 f(v)$.
\end{proof}
Si $f$ est deux fois différentiable $d^2f(a)$ est l'application bilinéaire associée avec la matrice symétrique
\begin{equation}
 H_f(a)= \begin{pmatrix}
    \partial^2_1f(a)& \ldots& \partial_1\partial_m f(a)\\
    \vdots& \ddots& \vdots\\
    \partial_1\partial_m f(a)&\ldots&\partial^2_1f(a),
  \end{pmatrix}
\end{equation}
Cette matrice est dite la matrice \defe{hessienne}{hessienne} de $f$.

\begin{example}
  Montrons qu'il n'existe pas de fonctions $f$ de classe $\mathcal{C}^2$ telles que
  \begin{subequations}
      \begin{numcases}{}
  \partial_xf(x,y)= 5\sin x\\
  \partial_y(x,y)=6x+y.
      \end{numcases}
  \end{subequations}
  Ceci est vite fait en appliquant le théorème de Schwarz,~\ref{Schwarz}; ce que nous trouvons est
\[
\partial_y (\partial_xf)= 0\neq \partial_x(\partial_yf)= 6.
\]
Donc, l'existence d'une fonction $f$ de classe $\mathcal{C}^2$ telle que $\partial_x(x,y)= 5\sin x$ et $\partial_yf(x,y)=6x+y$ serait en contradiction avec le théorème.
\end{example}

Soit une fonction de classe \( C^2\) \( f\colon V\to \eR\) où \( V\) est un espace vectoriel de dimension \( n<\infty\). Nous avons
\begin{subequations}
    \begin{align}
        f&\colon V\to \eR\\
        df&\colon V\to \aL(V,\eR)\\
        d^2f&\colon V\to \aL\Big( V,\aL(V,\eR) \Big),
    \end{align}
\end{subequations}
avec, en suivant les différentes formules du lemme~\ref{LemdfaSurLesPartielles},
\begin{equation}
        df_a(u)=\Dsdd{ f(v+tu) }{t}{0}
\end{equation}
et
\begin{equation}
    (d^2f)_a(u)=\Dsdd{ df_{v+tu} }{t}{0}
\end{equation}
pour tout \( a,u\in V\). Notons que dans le deuxième cas, il s'agit d'une limite dans \( \aL(V,\eR)\). Si \( \dim(V)=n\), alors \( \dim\aL(V,\eR)=n\) et avec un choix de base, nous pouvons trouver une matrice \( n\times n\) pour \( (d^2f)_a\).

Soit une base \( \{ e_i \}\) de \( V\) et la base duale \( \{ e_i^* \}\) de \( \aL(V,\eR)\). Nous allons chercher la matrice de \( (d^2f)_a\) pour ces bases. L'élément de matrice
\begin{equation}
    \big[ (d^2f)_a \big]_{ij}
\end{equation}
est la composante \( e_j^*\) de \( (d^2f)_a\) appliqué à \( e_i\). Trouver cette composante \( e_j^*\) revient à appliquer l'élément \( (d^2f)_ae_i\) de \( \aL(V,\eR)\) à \( e_j\). Le calcul est donc :
\begin{subequations}
    \begin{align}
        \big[ (d^2f)_{a} \big]_{ij}&=\big( (d^2f)_ae_i \big)(e_j)\\
        &=\Dsdd{ df_{a+te_i}(e_j) }{t}{0}       \label{SUBEQooDRZFooAuuaad}\\
        &=\Dsdd{    \Dsdd{ f(a+te_i+se_j) }{s}{0}    }{t}{0}\\
        &=\frac{ \partial^2f }{ \partial x_i\partial x_j }(a).
    \end{align}
\end{subequations}
Attention : le passage à \eqref{SUBEQooDRZFooAuuaad} n'est pas une trivialité. Le fait est que si \( t\mapsto A(t)\) est une application continue \( \eR\to \aL(V,\eR)\) alors
\begin{equation}
    \lim_{t\to 0} \big( A(t)v \big)=\big( \lim_{t\to 0} A(t) \big)v.
\end{equation}

Donc la matrice de \( d^2f  \) est la matrice des dérivées secondes. Il s'agit d'une matrice symétrique par le théorème de Schwarz~\ref{Schwarz}.

\begin{normaltext}      \label{NORMooZAOEooGqjpLH}
    Si \( a\in v\), nous pouvons aussi voir \( (d^2f)_a\) comme une forme bilinéaire sur \( V\) grâce à la proposition~\ref{isom_isom}. Si \( u,v\in V\) nous notons
    \begin{equation}
        (d^2f)_a(u,v)=(d^2f)_a(u)v.
    \end{equation}
    À droite, il s'agit de la définition réelle de \( d^2f\) sans abus de notations, et à gauche, il s'agit d'une notation. Cette application bilinéaire \( (d^2f)_a\in \aL^{(2)}(V,\eR)\) a pour matrice symétrique la matrice des dérivées secondes calculées en \( a\).
\end{normaltext}

\begin{example} \label{ExZHZYcNH}
    Voyons comment la différentielle seconde fonctionne entre deux espaces vectoriels. Soient deux espaces vectoriels de dimension finie \( V\) et \( W\). Pour que les choses soient claires, nous avons :
    \begin{subequations}
        \begin{align}
            f&\colon V\to W\\
            df&\colon V\to \aL(V,W)\\
            d^2f&\colon V\to \aL\Big( V,\aL(V,W) \Big).
        \end{align}
    \end{subequations}
    Si \( a\in V\), alors \( (d^2f)_a\) est une application \( V\to \aL(V,W)\). Il faut donc l'appliquer à \( u\in V\) et ensuite à \( v\in V\) pour obtenir un élément de \( W\) :
    \begin{subequations}
        \begin{align}
            (d^2f)_a(u)v&=\Dsdd{ df_{a+tu} }{t}{0}v\\
            &=\Dsdd{ df_{a+tu}(v) }{t}{0}\\
            &=\Dsdd{ \Dsdd{ f(a+tu+sv) }{s}{0} }{t}{0}\\
            &=\frac{ \partial^2f }{ \partial u\partial v }(a).
        \end{align}
    \end{subequations}


    Par conséquent nous voyons
    \begin{equation}\label{EqQHINNtD}
        \begin{aligned}
            d^2f\colon V&\to \aL^{(2)}(V,W) \\
            d^2f_a(u,v)&=\frac{ \partial^2f  }{ \partial u\partial v }(a).
        \end{aligned}
    \end{equation}

    Dans le cas d'une fonction \( f\colon \eR\to \eR\), nous avons une seule direction et par linéarité de \eqref{EqQHINNtD} par rapport à \( u\) et \( v\), nous avons
    \begin{equation}        \label{EQooSOCGooIiNGmG}
        d^2f_a(u,v)=f''(a)uv
    \end{equation}
    où les produits sont des produits usuels dans \( \eR\) et \( f''\) est la dérivée seconde usuelle.
\end{example}

Tout ceci est un peu résumé dans la proposition suivante.
\begin{proposition}     \label{PROPooFWZYooUQwzjW}
    Soit une fonction \( f\colon \eR^n\to \eR\) de classe \( C^2\). Alors en désignant par \( H_af\) sa matrice hessienne au point \( a\) nous avons
    \begin{equation}
        (d^2f)_a(u,v)=\frac{ \partial^2f }{ \partial u\partial v }(a)=\langle (H_af)u, v\rangle
    \end{equation}
    pour tout \( u,v\in \eR^n\).
\end{proposition}

\begin{proof}
    La première égalité est l'équation \eqref{EQooSOCGooIiNGmG} déjà faite. Pour la seconde, il faut se rappeler du lien entre dérivée partielle et dérivée directionnelle, donné en le lemme~\ref{LemdfaSurLesPartielles}. En particulier ici nous avons
    \begin{equation}
        \frac{ \partial^2f }{ \partial u\partial v }=\sum_{kl}\frac{ \partial^2f }{ \partial x_k\partial x_l  }(a)u_kv_l=\langle (H_af)u, v\rangle .
    \end{equation}
\end{proof}

En particulier, la matrice hessienne \( H_af\) est symétrique et donc diagonalisable (théorème spectral~\ref{ThoeTMXla}). Si \( e_i\) est un vecteur propre unitaire pour la valeur propre \( \lambda_i\) nous avons
\begin{equation}
    (d^2f)_a(e_i,e_i)=\langle (H_af)e_i, e_i\rangle =\lambda_i\langle e_i, e_i\rangle =\lambda.
\end{equation}

Enfin pour celles qui aiment les notations matricielles de tout poil, il y a cette façon-ci d'écrire :
\begin{equation}
    (d^2f)_a(\alpha,\beta)=\begin{pmatrix}
        \alpha    &   \beta
    \end{pmatrix}\begin{pmatrix}
        \partial^2_xf(a)    &   \partial^2_{xy}f(a)    \\
        \partial^2_{xy}f(a)    &   \partial^2_yf(a)
    \end{pmatrix}\begin{pmatrix}
        \alpha    \\
        \beta
    \end{pmatrix}.
\end{equation}

%---------------------------------------------------------------------------------------------------------------------------
\subsection{Ordre supérieur}
%---------------------------------------------------------------------------------------------------------------------------

Soit une fonction \( f\colon \eR^n\to \eR\) différentiable \( l\) fois. L'application
\begin{equation}
    d^lf\colon \eR^n\to \aL\Big( \eR^n,\aL\big( \eR^n,\aL(\ldots \big) \Big)
\end{equation}
au point \( x\) appliquée à \( v^{(1)}\) appliquée au point \( v^{(2)}\), \ldots, appliquée à \( v^{(l)}\) est notée
\begin{equation}        \label{EQooITOLooQllUlJ}
    (d^lf)_x(v^{(1)},\ldots ,v^{(l)})\in \eR.
\end{equation}

\begin{proposition}     \label{PROPooQKZIooXTvkIr}
    Soit une fonction \( f\colon \eR^n\to \eR\) différentiable \( l\) fois. Avec la notation \eqref{EQooITOLooQllUlJ} nous avons
    \begin{equation}
        (d^lf)_x(v^{(1)},\ldots v^{(l)})=\sum_{k_1,\ldots, k_l}v^{(1)}_{k_1}\ldots v_{k_l}^{(l)}\frac{ \partial^lf }{ \partial x_{k_1}\ldots \partial x_{k_l} }(x).
    \end{equation}
\end{proposition}

\begin{proof}
    La preuve se fait par récurrence sur \( l\), en sachant que la formule est déjà vraie pour \( l=1\) et \( l=2\). Si la formule est valable pour \( l\), nous avons
    \begin{subequations}
        \begin{align}
            (d^{l+1}f)_x(v^{(1)},\ldots, v^{(l+1)})&=\Dsdd{ (d^l)_{x+tv^{(l+1)}}(v^{(1)},\ldots, v^{(l)}) }{t}{0}\\
            &=\sum_{k_1\ldots k_l}v_{k_1}^{(1)}\ldots v_{k_l}^{(l)}\Dsdd{   \frac{ \partial^lf }{ \partial x_1\ldots \partial x_l }(x+tv^{l+1})   }{t}{0}\\
            &=\sum_{k_1\ldots k_l}v_{k_1}^{(1)}\ldots v_{k_l}^{(l)}\sum_i\frac{ \partial  }{ \partial x_i }\frac{ \partial^lf }{ \partial x_{k_1}\ldots \partial x_k }(x).
        \end{align}
    \end{subequations}
    Cela donne le résultat attendu.
\end{proof}

\begin{normaltext}
    La formule de la proposition~\ref{PROPooQKZIooXTvkIr} nous permet d'écrire de jolies formules comme
    \begin{equation}        \label{EQooXRWWooMoKoOB}
        (d^3f)_x(h,h,h)=\sum_{ijk}h_ih_jh_k(\partial^3_{ijk}f)(x).
    \end{equation}
\end{normaltext}

\begin{proposition}[Dérivées partielles et fonctions \( C^k\)] \label{PropDYKooHvrfGw}
    Soit $U$ un ouvert de $\eR^m$ et  $f:U\subset\eR^m\to \eR^n$. La fonction $f$ est de classe $C^k$ si et seulement si les dérivées partielles $\partial_1 f, \ldots, \partial_m f $ existent et sont de classe $C^{k}$.
\end{proposition}
% TODO : une preuve serait importante.

%+++++++++++++++++++++++++++++++++++++++++++++++++++++++++++++++++++++++++++++++++++++++++++++++++++++++++++++++++++++++++++
\section{Développement asymptotique, théorème de Taylor}
%+++++++++++++++++++++++++++++++++++++++++++++++++++++++++++++++++++++++++++++++++++++++++++++++++++++++++++++++++++++++++++
\label{AppSecTaylorR}

\begin{theorem}[Théorème de Taylor\cite{TrenchRealAnalisys,ooCNZAooJEcgHZ}]		\label{ThoTaylor}
Soit $I\subset$ un intervalle non vide et non réduit à un point de $\eR$ ainsi que $a\in I$. Soit une fonction $f\colon I\to \eR$ telle que $f^{(n)}(a)$ existe. Alors il existe une fonction $\alpha$ définie sur $I$ et à valeurs dans $\eR$ vérifiant les deux conditions suivantes :
\begin{subequations}		\label{SubEqsDevTauil}
	\begin{align}
		f(x)&= \sum_{k=0}^n\frac{ f^{(k)}(a) }{ k! }(x-a)^k +\alpha(x)(x-a)^{n},	\\	\label{subeqfTepseqb}
		\lim_{t\to a}\alpha(t)&=0
	\end{align}
\end{subequations}
pour tout \( x\in I\). Ici $f^{(k)}$ dénote la $k$-ième dérivée de $f$ (en particulier, $f^{(0)}=f$, $f^{(1)}=f'$).\nomenclature{$f^{(n)}$}{La $n$-ième dérivée de la fonction $f$}
\end{theorem}

Nous insistons sur le fait que la formule \eqref{subeqfTepseqb} est une égalité, et non une approximation. Ce qui serait une approximation serait de récrire la formule dans le terme contenant $\alpha$.

Le polynôme $T^a_{f,n}$ est le \defe{polynôme de Taylor}{Taylor} de $f$ au point $a$ à l'ordre $n$.

Les conditions \eqref{SubEqsDevTauil} sont souvent aussi énoncées sous la forme qu'il existe une fonction \( \alpha\) telle que
\begin{subequations}    \label{SUBEQooPYABooKpDgdu}
    \begin{numcases}{}
        \lim_{t\to 0} \frac{ \alpha(t) }{ t^n }=0\\
        f(a+h)=f(a)+hf'(a)+\frac{ h^2 }{2}f''(a)+\cdots+ \frac{ h^n }{ n! }f^{(n)}(a)+\alpha(h).
    \end{numcases}
\end{subequations}

Voici un énoncé pour les fonctions à plusieurs variables.

\begin{theorem}[\cite{ooLZSZooIOILHY}]      \label{THOooTDFRooEkChgi}
    Si \( f\colon E\to \eR\) est une application \( n\) fois différentiable en \( a\in E\) alors il existe une fonction \( \epsilon\colon \eR\to \eR\) telle que
    \begin{subequations}
        \begin{numcases}{}
            f(a+h)=f(a)+df_a(h)+\frac{ 1 }{2}(d^2f)_a(h,h)+\ldots +\\
            \qquad+\ldots+\frac{1}{ n! }(d^nf)_a(h,\ldots, h)+\| h \|^n\epsilon(\| h \|)\\
            \lim_{t\to 0} \epsilon(t)=0.
        \end{numcases}
    \end{subequations}
\end{theorem}

%---------------------------------------------------------------------------------------------------------------------------
\subsection{Fonctions «petit o» }
%---------------------------------------------------------------------------------------------------------------------------

Nous voulons formaliser l'idée d'une fonction qui tend vers zéro « plus vite » qu'une autre. Nous disons que $f\in o\big(\varphi(x)\big)$ si
\begin{equation}
    \lim_{x\to 0} \frac{ f(x) }{ \varphi(x) }=0.
\end{equation}
En particulier, nous disons que $f\in o(x)$ lorsque $\lim_{x\to 0} f(x)/x=0$.

\begin{remark}
    À titre personnel, l'auteur de ces lignes déconseille d'utiliser cette notation qui est un peu casse-figure pour qui ne la maîtrise pas bien.
\end{remark}

En termes de notations, nous définissons l'ensemble $o(x)$\nomenclature{$o(x)$}{fonction tendant rapidement vers zéro} l'ensemble des fonctions $f$ telles que
\begin{equation}
	\lim_{x\to 0} \frac{ f(x) }{ x }=0.
\end{equation}
Plus généralement si $g$ est une fonction telle que $\lim_{x\to 0} g(x)=0$, nous disons $f\in o(g)$ si
\begin{equation}
	\lim_{x\to 0} \frac{ f(x) }{ g(x) }=0.
\end{equation}
De façon intuitive, l'ensemble $o(g)$ est l'ensemble des fonctions qui tendent vers zéro «plus vite» que $g$.

Nous pouvons donner un énoncé alternatif au théorème~\ref{ThoTaylor} en définissant $h(x)=\epsilon(x+a)x^n$. Cette fonction est définie exprès pour avoir
\begin{equation}
	h(x-a)=\epsilon(x)(x-a)^n,
\end{equation}
et donc
\begin{equation}
	\lim_{x\to 0} \frac{ h(x) }{ x^n }=\lim_{x\to 0} \epsilon(x-a)=\lim_{x\to a}\epsilon(x)=0.
\end{equation}
Donc $h\in o(x^n)$.

Le théorème dit donc qu'il existe une fonction $\alpha\in o(x^n)$ telle que
\begin{equation}
	f(x)=T^a_{f,n}(x)+\alpha(x-a).
\end{equation}
pour tout $x\in I$.

\begin{example}
    L'exemple~\ref{EXooXLYJooKVqhTE} du développement du cosinus sera traité quand les fonctions trigonométriques seront définies.
\end{example}

\begin{proposition}[Ordre deux sur \( \eR^n\)\cite{MonCerveau}]         \label{PROPooTOXIooMMlghF}
    Soit un ouvert \( \Omega\) de \( \eR^n\) et \( a\in \Omega\) ainsi qu'une fonction \( f\colon \Omega\to \eR\) de classe \( C^2\). Alors il existe une fonction \( \alpha\colon \eR^n\to \eR\) telle que
    \begin{subequations}
        \begin{numcases}{}
            f(a+h)=f(a)+df_a(h)+\frac{ 1 }{2}(d^2f)_a(h,h)+\| h \|^2\alpha(h)\\
            \lim_{h\to 0} \alpha(h)=0.
        \end{numcases}
    \end{subequations}
    Ici, la notation \( (d^2f)_a(h,h)\) réfère à ce qui est expliqué en~\ref{NORMooZAOEooGqjpLH}.
\end{proposition}

\begin{proof}
    Dans la suite nous considérons \( t\) et \( h\) tels que toutes les expressions suivantes aient un sens, c'est à dire que tous les trucs comme \( a+th\) restent dans \( \Omega\). Pour \( h\in \eR^n\) nous nommons \( e_h\) le vecteur unitaire dans la direction de \( h\), c'est à dire \( e_h=h/\| h \|\) et nous posons
    \begin{equation}
        k_h(t)=f(a+te_h).
    \end{equation}
    et nous lui appliquons Taylor~\ref{ThoTaylor} à l'ordre deux : il existe une fonction \( \beta_h\) telle que
    \begin{equation}        \label{EQooETDFooAmiRcV}
        k_h(x)=k_h(0)+xk_h'(0)+\frac{ x^2 }{2}k''_h(0)+x^2\beta_h(x).
    \end{equation}
    avec \( \lim_{x\to 0} \beta_h(x)=0\).

    En ce qui concerne les dérivées de \( k_h\) nous avons
    \begin{equation}
        k'_h(0)=df_a(e_h)
    \end{equation}
    et
    \begin{equation}
        k_h''(0)=(d^2f)_{a}(e_h,e_h).
    \end{equation}
    Il est maintenant temps d'écrire \( f(a+h)=k(\| h \|)\) et de substituer les dérivées de \( k\) par les différentielles de \( f\) dans \eqref{EQooETDFooAmiRcV} :
    \begin{equation}        \label{EQooUSUGooYPscxV}
            f(a+h)=k(\| h \|)=f(a)+df_a(h)+\frac{ 1 }{2}(d^2f)_a(h,h)+\| h^2 \|\beta_{h}(\| h \|).
    \end{equation}
    Il reste à voir que la fonction \( \alpha\colon h\mapsto \beta_h(\| h \|)\) tend vers zéro pour \( h\to 0\). En prenant la limite \( h\to 0\) dans \eqref{EQooUSUGooYPscxV}, il est manifeste que la limite du membre de gauche existe et vaut \( f(a)\). Donc la limite du membre de droite doit exister et valoir également \( f(a)\). Nous en déduisons que la limite de
    \begin{equation}
        df_a(h)+\frac{ 1 }{2}(d^2f)_a(h,h)+\| h \|^2\beta_h(\| h \|)
    \end{equation}
    existe et vaut zéro. La limite des deux premiers termes existe et vaut zéro, donc la limite du troisième existe et vaut zéro :
    \begin{equation}
        \lim_{h\to 0} \| h \|^2\beta_h(\| h \|)=0.
    \end{equation}
\end{proof}

\begin{proposition}     \label{PROPooWWMYooPOmSds}
Soit un ouvert \( \Omega\) de \( \eR^n\) et une fonction \( f\colon \Omega\to \eR\) de classe \( C^1\) et deux fois différentiable sur \( \mathopen] x , x+h \mathclose[\). Alors il existe \( \theta\in \mathopen] 0 , 1 \mathclose[\) tel que
    \begin{equation}
        f(x+h)=f(x)+df_x(h)+\frac{ 1 }{2}(d^2f)_{x+\theta h}(h,h).
    \end{equation}
\end{proposition}

%---------------------------------------------------------------------------------------------------------------------------
\subsection{Autres formulations}
%---------------------------------------------------------------------------------------------------------------------------

\begin{example}		\label{ExempleUtlDev}
	Une des façons les plus courantes d'utiliser les formules \eqref{SubEqsDevTauil} est de développer $f(a+t)$ pour des petits $t$ en posant $x=a+t$ dans la formule :
	\begin{equation}	\label{EqDevfautouraeps}
		f(a+t)=f(a)+f'(a)t+f''(a)\frac{ t^2 }{ 2 }+\epsilon(a+t)t^2
	\end{equation}
	avec $\lim_{t\to 0} \epsilon(a+t)=0$. Ici, la fonction $T$ dont on parle dans le théorème est $T_{f,2}^a(a+t)=f(a)+f'(a)t+f''(a)\frac{ t^2 }{2}$.

	Lorsque $x$ et $y$ sont deux nombres «proches\footnote{par exemple dans une limite $(x,y)\to(h,h)$.}», nous pouvons développer $f(y)$ autour de $f(x)$ :
	\begin{equation}		\label{Eqfydevfx}
		f(y)=f(x)+f'(x)(y-x)+f''(x)\frac{ (y-x)^2 }{ 2 }+\epsilon(y-x)(y-x)^2,
	\end{equation}
	et donc écrire
	\begin{equation}
		f(x)-f(y)=-f'(x)(y-x)-f''(x)\frac{ (y-x)^2 }{ 2 }-\epsilon(y-x)(y-x)^2.
	\end{equation}
	De cette manière nous obtenons une formule qui ne contient plus que $y$ dans la différence $y-x$.
\end{example}

%---------------------------------------------------------------------------------------------------------------------------
\subsection{Formule et reste}
%---------------------------------------------------------------------------------------------------------------------------

\begin{proposition}     \label{PropDevTaylorPol}
    Soient $f\colon I\subset\eR\to \eR$ et $a\in\Int(I)$. Soit un entier $k\geq 1$. Si $f$ est $k$ fois dérivable en $a$, alors il existe un et un seul polynôme $P$ de degré $\leq k$ tel que
    \begin{equation}
        f(x)-P(x-a)\in o\big( | x-a |^k \big)
    \end{equation}
    lorsque $x\to a$, $x\neq a$. Ce polynôme  est donné par
    \begin{equation}
        P(h)=f(a)+f'(a)h+\frac{ f''(a) }{ 2! }h^2+\cdots+\frac{ f^{(k)}(a) }{ k! }h^k.
    \end{equation}
    Notons encore deux façons alternatives d'écrire le résultat. Si \( f\in C^k\) il existe une fonction \( \alpha\) telle que \( \lim_{t\to 0} \alpha(t)=0\) et
    \begin{equation}
        f(x)=\sum_{n=0}^k\frac{ f^{(n)}(a) }{ n! }(x-a)^n+(x-a)^n\alpha(x-a).
    \end{equation}
    Si \( f\in C^{k+1}\) alors
    \begin{equation}        \label{EquQtpoN}
        f(x)=\sum_{n=0}^k\frac{ f^{(n)}(a) }{ n! }(x-a)^n+(x-a)^{n+1}\xi(x-a)
    \end{equation}
    où \( \xi\) est une fonction telle que \( \xi(t)\) tend vers une constante lorsque \( t\to 0\).
\end{proposition}

La proposition suivant donne une intéressante façon de trouver le reste d'un développement de Taylor.
\begin{proposition}     \label{PropResteTaylorc}
Soient $I$, un intervalle dans $\eR$ et $f\colon I\to \eR$ une fonction de classe $C^k$ sur $I$ telle que $f^{(k+1)}$ existe sur $I$. Soient $a\in\Int(I)$ et $x\in I$. Alors il existe $c\in\mathopen] x , a \mathclose[$ tel que
\begin{equation}
    f(x)=\sum_{n=0}^k\frac{ f^{(n)}(a) }{ n! }(x-a)^n+\frac{ f^{(k+1)}(c) }{ (k+1)! }(x-a)^{k+1}.
\end{equation}
\end{proposition}

%---------------------------------------------------------------------------------------------------------------------------
\subsection{Reste intégral}
%---------------------------------------------------------------------------------------------------------------------------

Comme son nom l'indique, le «reste intégral» demande de savoir les intégrales. La formule du reste intégral sera donc pour après la définition des intégrales, proposition~\ref{PropAXaSClx}.

%+++++++++++++++++++++++++++++++++++++++++++++++++++++++++++++++++++++++++++++++++++++++++++++++++++++++++++++++++++++++++++
\section{Développement limité autour de zéro}
%+++++++++++++++++++++++++++++++++++++++++++++++++++++++++++++++++++++++++++++++++++++++++++++++++++++++++++++++++++++++++++

Dans cette sections nous supposons toujours que les fonctions sont définies sur un intervalle ouvert de $\eR$, $I$, contenant \( 0\).

%---------------------------------------------------------------------------------------------------------------------------
\subsection{Généralités}
%---------------------------------------------------------------------------------------------------------------------------

\begin{definition}
    Soit \( f\colon I\to 0\) une fonction définie sur un ouvert \( I\) autour de zéro. Nous disons que \( f\) admet un \defe{développement limité}{développement!limité!en zéro} autour de \( 0\) à l'ordre \( n\) s'il existe une fonction \( \alpha\colon I\to \eR\) telle que
    \begin{subequations}
        \begin{numcases}{}
            f(x)=P_n(x)+x^n\alpha(x)\\
            \lim_{x\to 0} \alpha(x)=0
        \end{numcases}
    \end{subequations}
    où \( P(x)=a_0+a_1x+\cdots +a_nx^n\) est une polynôme de degré \( n\). Le polynôme \( P_n\) est appelé la \defe{partie régulière}{partie!régulière} du développement.
\end{definition}
La fonction \( \alpha\) est appelé le \defe{reste}{reste!d'un développement limité} du développement et sera parfois noté \( \alpha_f\). Lorsque \( P\) est la partie régulière d'un développement limité de \( f\) nous notons parfois \( f\sim P\).

\begin{proposition}[Troncature]
    Si \( f\) admet un développement limité d'ordre \( n\) alors il admet également un développement limité d'ordre \( n'\) pour tout \( n'<n\). Ce dernier s'obtient en tronquant le polynôme d'ordre \( n\) à l'ordre \( n'\).
\end{proposition}

\begin{proposition}[Unicité]
    Si \( f\) admet une développement limité alors ce dernier est unique : il existe un unique polynôme \( P_n\) d'ordre \( n\) et une unique fonction \( \alpha\) vérifiant simultanément les deux conditions
    \begin{subequations}
        \begin{numcases}{}
            f(x)=P_n(x)+x^n\alpha(x),\\
            \lim_{x\to 0} \alpha(x)=0.
        \end{numcases}
    \end{subequations}
\end{proposition}

\begin{example} \label{ExTHGooCBcnAy}
    En ce qui concerne les séries géométriques de raison \( x\) nous savons les formules
    \begin{equation}
        1+x+x^2+\cdots +x^n=\frac{ 1-x^{n+1} }{ 1-x }
    \end{equation}
    et
    \begin{equation}
        1+x+x^2+x^3+\cdots=\frac{ 1 }{ 1-x }
    \end{equation}
    pour tout \( x\in\mathopen] -\infty , 1 \mathclose[\). Comparant les deux, il est naturel d'essayer de prendre \( 1+x+x^2+\cdots +x^n\) comme développement limité de la fonction \( f(x)=\frac{1}{ 1-x }\). Pour voir si cela fonctionne, il faut vérifier si «le reste» est bien de la forme \( x^n\alpha(x)\) avec \( \lim_{x\to 0} \alpha(x)=0\).

    Le reste en question est donné par
    \begin{equation}
        \frac{1}{ 1-x }-1-x-x^2-\ldots-x^n=\frac{1}{ 1-x }-\frac{ 1-x^{n+1} }{ 1-x }=\frac{ x^{n+1} }{ 1-x }=x^n\frac{ x }{ 1-x }.
    \end{equation}
    En posant \( \alpha(x)=\frac{ x }{ 1-x }\) nous avons donc bien
    \begin{equation}
        f(x)=\frac{1}{ 1-x }=1+x+x^2+\cdots +x^n+x^n\frac{ x }{ 1-x }
    \end{equation}
    et \( \lim_{x\to 0} \frac{ x }{ 1-x }=0\). Cela est le développement limité de \( f\) à l'ordre \( n\) autour de \( 0\).
\end{example}

La formule des accroissements finis est un cas particulier de développement fini. Supposons que \( f\) soit dérivable en \( 0\). En effet nous pouvons facilement trouver la fonction \( \alpha\) qui convient. Sachant que \( f(0)+xf'(0)\) donne l'approximation affine de \( f\) autour de \( 0\), nous cherchons \( \alpha\) en écrivant
\begin{equation}
    f(x)=f(0)+xf'(0)+x\alpha(x).
\end{equation}
Cela nous pousse à définir
\begin{equation}    \label{EqDCFooKozKrt}
    \alpha(x)=\frac{ f(x)-f(0) }{ x }-f'(0).
\end{equation}
Notons que cette fonction n'est pas définie en \( x=0\), mais cela n'a pas d'importance : seule la limite \( \lim_{x\to 0} \alpha(x)\) nous intéresse. Par définition de la dérivée,
\begin{equation}
    \lim_{x\to 0} \alpha(x)=\lim_{x\to 0} \frac{ f(x)-f(0) }{ x }-f'(0)=0.
\end{equation}

En conclusion si \( f\) est dérivable, son développement limité à l'ordre \(  1\) est donné par
\begin{equation}
    f(x)=f(0)+xf'(0)+x\alpha(x)
\end{equation}
où \( \alpha(x)\) est donnée par la formule \eqref{EqDCFooKozKrt}.

%---------------------------------------------------------------------------------------------------------------------------
\subsection{Formule de Taylor-Young}
%---------------------------------------------------------------------------------------------------------------------------

Plus généralement nous avons la proposition suivante qui donne le développement limité de toute fonction dérivable \( n\) fois.

\begin{proposition}[Formule de Taylor-Young]    \label{PropVDGooCexFwy}
    Soit \( f\) une fonction \( n\) fois dérivable sur un intervalle \( I\) contenant \( 0\). Alors il existe une fonction \( \alpha\colon I\to \eR\) telle que
    \begin{equation}        \label{EQooBKZDooTqYyIB}
        f(x)=f(0)+f'(0)x+\frac{ f''(0) }{ 2 }x^2+\frac{ f^{(3)}(0) }{ 3! }x^3+\cdots +\frac{ f^{(n)}(0) }{ n! }x^n+x^n\alpha(x)
    \end{equation}
    et
    \begin{equation}
        \lim_{x\to 0} \alpha(x)=0.
    \end{equation}
\end{proposition}

Cette proposition nous permet de trouver facilement des développements limités. Dans l'exemple~\ref{ExTHGooCBcnAy} nous avons dû utiliser des astuces et des formules pour déterminer le développement limité de \( \frac{1}{ 1-x }\). Au contraire la formule \eqref{EQooBKZDooTqYyIB} nous permet de trouver le polynôme en calculant seulement de dérivées.

\begin{example}
    Utilisation de la formule \eqref{EQooBKZDooTqYyIB} pour déterminer le développement limité de la fonction
    \begin{equation}
        f(x)=\frac{1}{ 1-x }.
    \end{equation}
    Il faut calculer les dérivées successives de \( f\) :
    \begin{subequations}
        \begin{align}
            f(x)&=\frac{1}{ 1-x }\\
            f'(x)&=\frac{ 1 }{ (1-x)^2 }\\
            f''(x)&=\frac{ 2 }{ (1-x)^3 }
        \end{align}
    \end{subequations}
    Avec ces résultats, nous devinons que
    \begin{equation}
        f^{(n)}(x)=\frac{ n! }{ (1-x)^{n+1} }.
    \end{equation}
    Pour en être sûr nous le prouvons par récurrence. La dérivée de \(\frac{ n! }{ (1-x)^{n+1} } \) est donnée par
    \begin{equation}
        \frac{ n!(n+1)(1-x)^n }{ (1-x)^{2n+2} }=\frac{(n+1)! }{ (1-x)^{n+2} }.
    \end{equation}
    Évaluées en \( x=0\), les dérivées successives de \( f\) sont \( f(0)=0\), \( f'(0)=1\), \( f''(0)=2\),\ldots,\( f^{(n)}(0)=n!\). Utilisant la formule \eqref{EQooBKZDooTqYyIB} nous avons
    \begin{equation}
        f(x)=1+x+x^2+\cdots +x^n+x^n\alpha(x),
    \end{equation}
    conformément à ce que nous avions déjà trouvé.
\end{example}
\begin{remark}
  Pour alléger la notation et ne pas écrire \(\ldots +x^n\alpha(x)\) nous pouvons aussi écrire
    \begin{equation}
         f(x)\sim 1+x+x^2+\cdots +x^n,
    \end{equation}
    mais il est interdit d'écrire
    \begin{equation}
         f(x)= 1+x+x^2+\cdots +x^n
    \end{equation}
    en mettant un signe d'égalité entre une fonction et son développement limité\footnote{Il faut cependant être très prudents avec la notation abrégée. Elle pourrait nous faire oublier des informations importantes, voir les développements des fonctions trigonométriques pour un exemple.}.
\end{remark}

Notons cependant que la proposition~\ref{PropVDGooCexFwy} ne donne pas de moyen simple de trouver la fonction \( \alpha\). Si la fonction $f$ est très régulière dans l'intervalle $I$ on a le résultat suivant.

\begin{proposition}[Reste dans la forme de Lagrange]
    Si la fonction $f$ est dérivable $n+1$ fois dans $I$ alors il existe $\bar x$ dans l'intervalle \( \mathopen[ 0 , x \mathclose]\) tel que
  \begin{equation}
    f(x) = P_n(x) + \frac{1}{(n+1)!} f^{n+1}(\bar x) x^{n+1}.
  \end{equation}
\end{proposition}

Voici quelque développements limités à savoir. Ils sont calculables en utilisant la formule de Taylor-Young (proposition~\ref{PropVDGooCexFwy}).
\begin{framed}
    \begin{subequations}  \label{SUBEQSooIFXBooIOkveI}
    \begin{align}
      e^x&=\sum_{k=0}^n\frac{ x^k }{ k! }+x^n\alpha(x)&\text{ordre } n\\
      \cos(x)&=\sum_{k=0}^p\frac{ (-1)^kx^{2k} }{ (2k)! }+x^{2p+1}\alpha(x)&\text{ordre } 2p+1    \label{subeqCBQooTZGxQpcos}\\
      \sin(x)&=\sum_{k=0}^p\frac{ (-1)^kx^{2k+1} }{ (2k+1)! }+x^{p+2}\alpha(x)&\text{ordre } 2p+1 \label{subeqCBQooTZGxQpsin}\\
      \ln(1+x)&=\sum_{k=0}^{n-1}(-1)^k\frac{ x^{k+1} }{ k+1 }+x^n\alpha(x)&\text{ordre } n\\
      (1+x)^l&=\sum_{k=0}^l\binom{ l }{ k }x^k&\text{exact si } l\text{ est entier.}\\
      (1+x)^{\alpha}&=1+\sum_{k=1}^n\frac{ \alpha(\alpha-1)\ldots(\alpha-k+1) }{ k! }x^k+x^n\alpha(x)&\text{ordre } n.\label{subEqJFNooQQodFQ}
    \end{align}
  \end{subequations}
\end{framed}
Il est fortement suggéré au lecteur de vérifier ces développements par lui-même.

\begin{remark}
  Quelque remarques.
  \begin{enumerate}
  \item
    Les développements de sinus et de cosinus ont un terme sur deux qui est nul. C'est pour cela qu'en ayant une polynôme de degré \( 2p\), nous avons le développement d'ordre \( 2p+1\).
  \item
    En ce qui concerne le développement de \( \ln(1+x)\), il faut noter que la somme va jusqu'à \( k=n-1\) (et non \( k=n\)) parce que nous voulons aller jusqu'à \( x^n\) et que nous écrivons \( x^{k+1}\).
  \item
    Si \( l\) est entier, le développement de \( (1+x)^l\) est exact. Dans le développement de \( (1+x)^{\alpha}\), nous reconnaissons la formule de \( \binom{ k }{\alpha }\), sauf que nous ne pouvons pas l'écrire avec cette notation lorsque \( \alpha\) n'est pas entier.
  \item Le développement limité en $0$ d'une fonction paire ne contient que les puissances de $x$ d'exposant paire. Voir comme exemple le développement de la fonction cosinus.
  \end{enumerate}
\end{remark}

%---------------------------------------------------------------------------------------------------------------------------
\subsection{Règles de calcul}
%---------------------------------------------------------------------------------------------------------------------------

Les règles suivantes permettent de calculer les développements limités des fonctions qu'on peut écrire comme combinaison des fonctions dans le tableau des équation s\eqref{SUBEQSooIFXBooIOkveI} de la section précédente.
\begin{remark}
  Il est toujours possible de calculer le développement limité d'une fonction par la formule de Taylor-Young. Les règles suivantes peuvent nous economiser de l'effort et du temps.
\end{remark}

%///////////////////////////////////////////////////////////////////////////////////////////////////////////////////////////
\subsubsection{Linéarité des développements limités}
%///////////////////////////////////////////////////////////////////////////////////////////////////////////////////////////

L'opération qui consiste à prendre le développement limité d'une fonction est une opération linéaire : connaissant les développements limités de \( f\) et de \( g\), il suffit de les sommer pour obtenir celui de \( f+g\). De m\^eme, si $\lambda$ est une constante, le développement limité de $\lambda f$ est le développement limité de $f$ fois $\lambda$.
\begin{proposition}
Soient $\lambda$ et $\mu$ dans $\eR$.  Si \( f\) et \( g\) sont deux fonctions acceptant des développements limités d'ordre \( n\)
    \begin{subequations}    \label{EqJPPooCHihNn}
        \begin{align}
            f(x)&=P(x)+x^n\alpha_f(x)\\
            g(x)&=Q(x)+x^n\beta(x)
        \end{align}
    \end{subequations}
    avec \( \lim_{x\to 0} \alpha(x)=\lim_{x\to 0} \beta(x)=0\), alors la fonction \( \lambda f+\mu g\) admet le développement limité
    \begin{equation}    \label{EqCJFooVpyCtz}
        (f+g)(x)=(\lambda P+\mu Q)(x)+(\lambda \alpha+\mu\beta)(x).
    \end{equation}
\end{proposition}
\begin{remark}
  La forme explicite du reste ne nous interesse pas. Dans la pratique on écrira toujours $(f+g)(x)=(P+Q)(x)+\alpha(x)$, où on appelle $\alpha$ une fonction apportune telle que $\lim_{x\to 0} \alpha(x)=0$.
\end{remark}
\begin{proof}
    Vu les définitions \eqref{EqJPPooCHihNn} des polynômes \( P\), \( Q\) et des restes \( \alpha\) et \( \beta\), l'égalité \eqref{EqCJFooVpyCtz} est une conséquence de la linéarité de la dérivation et de la proposition~\ref{PropVDGooCexFwy}

    De plus \( P+Q\) est un polynôme de degré \( n\) dès que \( P\) et \( Q\) sont des polynômes de degré \( n\), et
    \begin{equation}
        \lim_{x\to 0} (\lambda \alpha+\mu\beta)(x)=\lim_{x\to 0} \lambda\alpha(x)+\lim_{x\to 0} \mu\beta(x)=0.
    \end{equation}
    Par conséquent \(\lambda \alpha+\mu\beta\) est la fonction de reste de \( \lambda f+\mu g\).
\end{proof}

\begin{example} \label{ExKPBooJmdFvY}
    Calculer le développement de la fonction
    \begin{equation}
        f(x)=3\sqrt[3]{1+x}+ e^{-2x}.
    \end{equation}
    Le développement de \( \sqrt[3]{1+x}\) est donné par la formule \eqref{subEqJFNooQQodFQ} avec \( \alpha=\frac{1}{ 3 }\). Nous avons donc dans un premier temps
    \begin{subequations}
        \begin{align}
            \sqrt[3]{1+x}&=1+\frac{ 1 }{ 3 }x+\frac{ \frac{1}{ 3 }\left( \frac{1}{ 3 }-1 \right) }{ 2 }x^2+\frac{ \frac{1}{ 3 }\left( \frac{1}{ 3 }-1 \right)\left( \frac{1}{ 3 }-2 \right) }{ 6 }x^3+x^3\alpha(x)\\
            &=1+\frac{1}{ 3 }x-\frac{1}{ 9 }x^2+\frac{ 5 }{ 81 }x^3+x^3\alpha(x).
        \end{align}
    \end{subequations}
    Nous avons alors
    \begin{subequations}
        \begin{align}
            3\sqrt[3]{1+x}+ e^{-2x}&=3\Big[  1+\frac{1}{ 3 }x-\frac{1}{ 9 }x^2+\frac{ 5 }{ 81 }x^3+x^3\alpha(x)\Big]+1-2x+2x^2-\frac{ 4 }{ 3 }x^3+x^3\beta(x)\\
            &=4-x+\frac{ 5 }{ 3 }x^2-\frac{ 31 }{ 27 }x^3+x^3\big( \alpha(x)+\beta(x) \big).
        \end{align}
    \end{subequations}

\end{example}

La condition \( \lim_{x\to 0} \alpha(x)=0\) signifie que l'approximation qui consiste à remplacer \( f(x) \) par le polynôme n'est pas une trop mauvaise approximation lorsque \( x\) est petit. Cela ne signifie rien de plus. En particulier si \( x\) est grand, l'approximation polynomiale peut-être (et est souvent) très mauvaise.

À ce propos, notez qu'un polynôme tend toujours vers \( \pm\infty\) lorsque \( x\) est grand. Une approximation polynomiale d'une fonction bornée est donc toujours (très) mauvaise pour les grandes valeurs de \( x\).

À titre d'exemple nous avons tracé sur la figure~\ref{LabelFigWUYooCISzeB} la fonction
\begin{equation}
    f(x)=3\sqrt[3]{x+1}+ e^{-2x}
\end{equation}
et ses développements limités d'ordre \( 1\) à \( 3\). Il est particulièrement visible que l'approximation est assez bonne pour la partie gauche du graphe sur laquelle la fonction est bien croissante, alors qu'elle est franchement mauvaise sur la droite où le graphe ressemble plutôt à une constante\footnote{Pouvez-vous cependant dire que vaut \( \lim_{x\to \infty} f(x)\) ?}.

\newcommand{\CaptionFigWUYooCISzeB}{Les développements limités d'ordre de plus en plus grand de la fonction de l'exemple~\ref{ExKPBooJmdFvY}. La fonction est en bleu et les «approximations» sont en rouge.}
\input{auto/pictures_tex/Fig_WUYooCISzeB.pstricks}

%///////////////////////////////////////////////////////////////////////////////////////////////////////////////////////////
\subsubsection{Développement limité d'un quotient}
%///////////////////////////////////////////////////////////////////////////////////////////////////////////////////////////

\begin{proposition}     \label{PROPooMANAooXhuanS}
    Si \( P_f\) est le polynôme du développement limité de \( f\) à l'ordre \( n\) et \( P_g\) celui de \( g\), alors nous obtenons le développement limité de \( f/g\) à l'ordre \( n\) en effectuant la division selon les puissances croissantes de \( P_f\) par \( P_g\).
\end{proposition}
Attention : il s'agit bien de faire une division selon les puissances croissantes, et non une divisions euclidienne. La division euclidienne de \( A\) par \( B\) consiste à écrire \( A=BQ+R\) avec le reste \( R\) de degré le plus \emph{petit} possible. Ici nous voulons avoir un reste de degré le plus \emph{grand} possible.

%///////////////////////////////////////////////////////////////////////////////////////////////////////////////////////////
\subsubsection{Développement limité d'une fonction composée}
%///////////////////////////////////////////////////////////////////////////////////////////////////////////////////////////


\begin{proposition}
    Soient \( f\) et \( g\) des fonctions admettant des développements limités d'ordre $n$ au voisinage de $0$. Nous supposons que \( \lim_{x\to 0} g(x)=0\). Alors la composée \( f\big( g(x) \big)\) admet un développement limité d'ordre $n$ au voisinage de $0$ qui s'obtient en substituant le développement de \( g\) à chaque <<\(x \)>> du développement de \( f\), et en supprimant tous les termes de degré plus élevé que $n$.
\end{proposition}

%+++++++++++++++++++++++++++++++++++++++++++++++++++++++++++++++++++++++++++++++++++++++++++++++++++++++++++++++++++++++++++
\section{Développement au voisinage de $x_0\neq 0$}
%+++++++++++++++++++++++++++++++++++++++++++++++++++++++++++++++++++++++++++++++++++++++++++++++++++++++++++++++++++++++++++

Il est intéressant de développer une fonction au voisinage de zéro lorsque nous nous intéressons à son comportement pour les \( x\) pas très grands. Il est toutefois souvent souhaitable de savoir le comportement d'une fonction au voisinage d'autres valeurs que zéro.

Pour développer la fonction \( f\) autour de \( x_0\), nous considérons la fonction \( h\mapsto f(x_0+h)\) que nous développons autour de zéro (pour \( h\)). L'objectif est de trouver une polynôme \( P\) et une fonction \( \alpha\) tels que
\begin{subequations}
    \begin{numcases}{}
        f(x)=P(x)+(x-x_0)^n\alpha(x)\\
        \lim_{x\to x_0} \alpha(x)=0.
    \end{numcases}
\end{subequations}
En pratique, le développement limité à l'ordre $n$ d'une fonction autour d'un point $x_0$ quelconque à l'intérieur de son domaine prend la forme suivante, qui généralise la formule de Taylor-Young vue dans la proposition~\ref{PropVDGooCexFwy}
\begin{proposition}[Formule de Taylor-Young, cas général]
    Soit \( f\) une fonction \( n\) fois dérivable sur un intervalle \( I\) contenant \(x_0\). Alors il existe une fonction \( \alpha\colon I\to \eR\) telle que
    \begin{equation}    \label{EqTJRooUbsyzJ}
      \begin{aligned}
        f(x)=f(x_0)+&f'(x_0)(x-x_0)+\frac{ f''(x_0) }{ 2 }(x-x_0)^2+\\
        &+\frac{ f^{(3)}(x_0) }{ 3! }(x-x_0)^3+\cdots +\frac{ f^{(n)}(x_0) }{ n! }(x-x_0)^n+(x-x_0)^n\alpha(x-x_0)
      \end{aligned}
    \end{equation}
    et
    \begin{equation}
        \lim_{t\to 0} \alpha(t)=0.
    \end{equation}

\end{proposition}
\begin{example}\label{developcosenpisur3}
    Développer la fonction \( \cos\) autour de \( x=\frac{ \pi }{ 3 }\). Utiliser les valeurs remarquables \eqref{}. Nous développons autour de \( h=0\) la fonction \( \cos(\frac{ \pi }{ 3 }+h)\) :
    \begin{equation}
        \cos\big( \frac{ \pi }{ 3 }+h \big)\sim \cos\big( \frac{ \pi }{ 3 } \big)+h\cos'(\frac{ \pi }{ 3 })+\frac{ h^2 }{2}\cos''\big( \frac{ \pi }{ 3 } \big)=\frac{ 1 }{2}-\frac{ \sqrt{3} }{2}h-\frac{1}{ 4 }h^2.
    \end{equation}
    Il est aussi possible d'écrire cela en notant \( x=x_0+h\), c'est à dire en remplaçant \( h\) par \( x-\frac{ \pi }{ 3 }\) :
    \begin{equation}
        \cos(x)\sim\frac{ 1 }{2}-\frac{ \sqrt{3} }{ 2 }(x-\frac{ \pi }{ 3 })-\frac{1}{ 4 }(x-\frac{ \pi }{ 3 })^2.
    \end{equation}
\end{example}

Pour donner une idée nous avons dessiné sur le graphe suivant la fonction sinus et ses développements d'ordre \( 4\) autour de zéro et autour de \( 3\pi/4\).
\begin{center}
   \input{auto/pictures_tex/Fig_WJBooMTAhtl.pstricks}
\end{center}


%+++++++++++++++++++++++++++++++++++++++++++++++++++++++++++++++++++++++++++++++++++++++++++++++++++++++++++++++++++++++++++
\section{Application au calcul de limites}
%+++++++++++++++++++++++++++++++++++++++++++++++++++++++++++++++++++++++++++++++++++++++++++++++++++++++++++++++++++++++++++

Lors d'un calcul de limite, développer une partie d'une expression peut être utile.

\begin{example}
    À calculer :
    \begin{equation}
        \lim_{x\to 0} \frac{ \ln(1+x) }{ x }.
    \end{equation}
    Cela est une indétermination de type \( \frac{ 0 }{ 0 }\). Le développement limité du numérateur nous donne une fonction \( \alpha(x)\) telle que \( \lim_{x\to 0} \alpha(x)=0\) et
    \begin{equation}
        \frac{ \ln(1+x) }{ x }=\frac{ x-\frac{ x^2 }{2}+x^2\alpha(x) }{ x }=1-\frac{ x }{ 2 }+x\alpha(x).
    \end{equation}
    Sur le membre de droite la limite est facile à calculer :
    \begin{equation}
        \lim_{x\to 0} \frac{ \ln(1+x) }{ x }=\lim_{x\to 0} \Big( 1-\frac{ x }{ 2 }+x\alpha(x) \Big) =1.
    \end{equation}
\end{example}

%+++++++++++++++++++++++++++++++++++++++++++++++++++++++++++++++++++++++++++++++++++++++++++++++++++++++++++++++++++++++++++
\section{Développement au voisinage de l'infini}
%+++++++++++++++++++++++++++++++++++++++++++++++++++++++++++++++++++++++++++++++++++++++++++++++++++++++++++++++++++++++++++

Il est souvent utile de connaître le comportement d'une fonction pour les grandes valeurs de \( x\) et de déterminer ses asymptotes éventuelles. La technique que nous allons utiliser consiste à poser \( x=\frac{1}{ h }\) et de développer la fonction ``auxiliaire'' $g(h) = f(1/h)$ autour de \( h=0\). La limite avec \( h\to 0^+\) donnera le comportement pour \( x\to \infty\) et la limite \( h\to 0^-\) donnera le comportement pour \( x\to -\infty\).

Dans le cas d'une développement autour de \( \pm\infty\) nous ne parlons plus de développement \emph{limité} mais de \defe{développement asymptotique}{développement!asymptotique}.
