% This is part of Mes notes de mathématique
% Copyright (c) 2011-2018
%   Laurent Claessens, Carlotta Donadello
% See the file fdl-1.3.txt for copying conditions.

%+++++++++++++++++++++++++++++++++++++++++++++++++++++++++++++++++++++++++++++++++++++++++++++++++++++++++++++++++++++++++++
\section{Espaces métriques}
%+++++++++++++++++++++++++++++++++++++++++++++++++++++++++++++++++++++++++++++++++++++++++++++++++++++++++++++++++++++++++++

%---------------------------------------------------------------------------------------------------------------------------
\subsection{Espaces métrisables}
%---------------------------------------------------------------------------------------------------------------------------

\begin{definition}
    Un espace topologique est \defe{métrisable}{espace!topologique!métrisable} s'il est homéomorphe à un espace métrique.
\end{definition}

\begin{proposition}     \label{PROPooMJEQooHtIyeX}
    Si \( X\) est un espace topologique dont la topologie est donnée par une famille dénombrable de semi-normes, alors il est métrisable.
\end{proposition}
%TODO : une preuve

%---------------------------------------------------------------------------------------------------------------------------
\subsection{Fonctions continues}
%---------------------------------------------------------------------------------------------------------------------------

La propriété suivante donne des caractérisations importantes de la continuité dans le cas des espaces métriques.
\begin{proposition}[Continuité, ouverts et voisinages et limite\cite{DHpsZoY}] \label{PropQZRNpMn}
    Soient \( f\colon E\to F\) une application entre espaces métriques et \( a\in E\). Alors nous avons équivalence entre les choses suivantes :
    \begin{enumerate}
        \item\label{ItemCBUoRWJi}
            \( f\) est continue en \( a\),
        \item\label{ItemCBUoRWJii}
            Pour tout voisinage ouvert \( W\) de \( f(a)\), il existe un voisinage ouvert \( V\) de \( a\) tel que \( f(V)\subset W\).
        \item\label{ItemCBUoRWJiii}
            Pour toute boule \( W'=B\big( f(a),\epsilon \big)\), il existe une boule \( V'=B(a,\delta)\) telle que \( f(V)\subset W\).
        \item\label{ItemCBUoRWJiv}
            $\forall \epsilon>0,\,\exists \delta>0\,\tq f\big( B(a,\delta) \big)\subset B\big( f(a),\epsilon \big)$.
        \item\label{ItemYNQpikrii}
            \( \lim_{x\to a}f(x)=f(a)\) où la limite est donnée par la définition~\ref{DefYNVoWBx},
        \item\label{ItemYNQpikriii}
            Pour tout \( \epsilon>0\), il existe \( \delta>0\) tel que \( \| x-a \|<\delta\) implique \( \| f(x)-f(a) \|<\epsilon\).
    \end{enumerate}
\end{proposition}
\index{continue!fonction entre espaces métriques}
La proposition~\ref{PropNGjQnqF} nous montrera que ces équivalences tiennent encore lorsque l'espace a une topologie de semi-normes.

\begin{proof}
    L'équivalence~\ref{ItemCBUoRWJi} \( \Leftrightarrow\)~\ref{ItemCBUoRWJii} est la définition~\ref{DefOLNtrxB}. L'équivalence~\ref{ItemCBUoRWJiii} \( \Leftrightarrow\)~\ref{ItemCBUoRWJiv} est une simple paraphrase.

    Montrons~\ref{ItemCBUoRWJii} \( \Rightarrow\)~\ref{ItemCBUoRWJiii}. Si \( W'=B\big( f(a),\delta \big)\), nous avons un voisinage \( V\) de \( a\) tel que \( f(V)\subset W\). L'ensemble \( V\) contenant une boule autour de chacun de ses points\footnote{Cela est le théorème-définition~\ref{ThoORdLYUu} des ouverts dans un espace métrique, à ne pas confondre avec le théorème~\ref{ThoPartieOUvpartouv}.}, il en contient un autour de \( a\) : \( V'=B(a,\delta)\subset V\). A fortiori nous avons \( f(V')\subset W\).

    Montrons~\ref{ItemCBUoRWJiii} \( \Rightarrow\)~\ref{ItemCBUoRWJii}. Si \( W\) est un ouvert autour de \( f(a)\), il contient une boule autour de \( f(a)\) : \( B\big( f(a),\epsilon \big)\subset W\). Il existe donc une boule \( V'=B(a,\delta)\) telle que \( f(V')\subset B\big( f(a),\epsilon \big)\subset W\).

    L'équivalence~\ref{ItemCBUoRWJi} \( \Leftrightarrow\)~\ref{ItemYNQpikrii} est la définition~\ref{DefOLNtrxB} de la continuité en un point couplée à l'unicité de la limite due à la proposition~\ref{PropFObayrf} parce qu'un espace métrique est séparé.

    Prouvons~\ref{ItemYNQpikrii} \( \Rightarrow\)~\ref{ItemYNQpikriii}. Soient \( \epsilon>0\) et \( V=B\big( f(a),\epsilon \big)\). Étant donné que \( f(a)\) est une limite de \( f\) pour \( x\to a\), il existe un voisinage \( W\) de \( a\) tel que \( f(W)\subset V\). Soit \( \delta>0\) tel que \( B(a,\delta)\subset W\); alors si \( \| x-a \|<\delta\) nous avons \( x\in B(x,\delta)\subset W\) et donc \( f(x)\in B\big( f(a),\epsilon \big)\), c'est à dire \( \| f(a)-f(x) \|<\epsilon\).

    Enfin l'implication~\ref{ItemCBUoRWJii} \( \Rightarrow\)~\ref{ItemYNQpikrii} est une réécriture de la définition de la limite en un point.
\end{proof}

\begin{proposition}\label{PropLYMgVMJ}
    Une isométrie entre deux espaces métriques est continue.
\end{proposition}

\begin{proof}
    Soient \( f\colon X\to Y\) une application isométrique et \( \mO\) un ouvert de \( Y\). Soit \( a\in f^{-1}(\mO)\); si \( d(a,b)<r\), alors \( d\big( f(a),f(b) \big)<r\) et donc \( b\in f^{-1}\big( B(f(a),r) \big)\). Donc autour de chaque point de \( f^{-1}(\mO)\) nous pouvons trouver une boule ouverte contenue dans \( f^{-1}(\mO)\), ce qui prouve que \( f^{-1}(\mO)\) est ouvert.
\end{proof}

\begin{theorem}[Théorème \wikipedia{fr}{Théorème_des_fermés_emboités}{des fermés emboîtés}\cite{OIywOjl}]   \label{ThoCQAcZxX}
    Soit \( (E,d)\) un espace métrique. Il est complet si et seulement si toute suite décroissante de fermés non vides dont le diamètre tend vers zéro a une intersection qui se réduit à un seul point.
\end{theorem}

\begin{proof}
    \begin{subproof}
    \item[Condition suffisante]

        Soit \( \{ F_n \}_{n\in \eN}\) une telle suite de fermés emboités. Si nous choisissons des points \( x_n\in F_n\), nous obtenons une suite \( (x_n)\) de Cauchy et qui est par conséquent convergente vu que l'espace est par hypothèse complet. De plus, pour chaque \( N\geq n\), la queue de suite \( (x_n)_{n\geq N}\) est contenue dans \( F_N\) et donc converge vers un élément de \( F_N\) (parce que ce dernier est fermé). Donc la limite de \( (x_n)\) est dans \( \bigcap_{n\in \eN}F_n\).

        De plus cette intersection a diamètre nul parce que le diamètre de \( \bigcap_{n\in \eN}F_n\) est majoré par tous les diamètres des \( F_n\), lesquels sont arbitrairement petits par hypothèse. Donc l'intersection est réduite a un point.

    \item[Condition nécessaire]

        Soit \( (x_n)\) un suite de Cauchy. Nous considérons les ensembles
        \begin{equation}
            F_n=\overline{ \{ x_i\tq i\geq n \} }.
        \end{equation}
        Le fait que la suite soit de Cauchy implique que \( \diam(F_n)\to 0\). Par hypothèse, nous avons alors
        \begin{equation}
            \bigcap_{n\in \eN}F_n=\{ a \}.
        \end{equation}
        Pour s'assurer que \( a\) est bien la limite de \( (x_n)\), il suffit de remarquer que
        \begin{equation}
            d(x_n,a)\leq \diam F_n\to 0.
        \end{equation}
    \end{subproof}
\end{proof}

%---------------------------------------------------------------------------------------------------------------------------
\subsection{Caractérisations séquentielles}
%---------------------------------------------------------------------------------------------------------------------------

\begin{definition}  \label{DefENioICV}
    Si \( X\) est un espace topologique, une fonction \( f\colon X\to \eR\) est \defe{séquentiellement continue}{continuité!séquentielle} en un point \( a\) si pour toute suite convergente \( x_n\to a\) dans \( X\) nous avons \( f(x_n)\to f(x)\) dans \( \eR\).
\end{definition}

%TODO : il y a un contre-exemple à faire à la page http://www.les-mathematiques.net/phorum/read.php?14,787368,787582

Une fonction continue est séquentiellement continue. Dans les espaces métriques la proposition suivante montre que la réciproque est également vraie et la continuité est équivalente à la continuité séquentielle. Cela n'est cependant pas vrai pour n'importe quel espace topologique.

\begin{proposition}[Caractérisation séquentielle de la continuité\cite{MonCerveau}]		\label{PropFnContParSuite}
    Le lien entre continuité et continuité séquentielle.

    \begin{enumerate}
        \item
    Une application continue est séquentiellement continue (quels que soient les espaces de départ et d'arrivée).

\item\label{ItemWJHIooMdugfu}

    Si \( X\) et \( Y\) sont des espaces métriques, alors une fonction \( f\colon X\to Y\) est continue en un point si et seulement si elle est séquentiellement continue.
    \end{enumerate}
\end{proposition}

\begin{proof}
    En deux parties.
    \begin{subproof}
    \item[Sens direct]
        Supposons que \( f\) soit continue en \( a\) et considérons une suite \( a_k\to a\). Soient \( V\) un voisinage de \( f(a)\) et \( W\) un voisinage de \( a\) tels que \( f(W)\subset V\) (définition~\ref{DefYNVoWBx} de la continuité en un point). Par la convergence \( a_k\to a\),  il existe \( N\) tel que pour tout \( k>N\), \( a_k\in W\), et donc tel que \( f(a_k)\in V\), ce qui donne la continuité séquentielle de \( f\).
    \item[Sens réciproque, espaces métriques]

        Nous supposons maintenant que \( X\) et \( Y\) soient métriques. Si \( f\) n'est pas continue en \( a\), il existe \( \epsilon>0\) tel que pour tout \( \delta>0\), il existe \( x\) tel que \( \| x-a \|\leq\delta\) et \( \| f(x)-f(a) \|>\epsilon\). Nous considérons un tel \( \epsilon\) et pour chaque \( n\geq1\in \eN\) nous considérons un \( x_n\) correspondant à \( \delta=\frac{1}{ n }\). Cela nous donne une suite \( x_n\to a\) dans \( X\) mais \( \| f(x_n) -f(a)\|\) reste plus grand que \( \epsilon\). Cela montre que \( f\) n'est pas non plus séquentiellement continue.
    \end{subproof}
\end{proof}

\begin{proposition}[Caractérisation séquentielle de la limite\cite{MonCerveau}]     \label{PROPooJYOOooZWocoq}
    Soient deux espaces métriques \( X\) et \( Y\) ainsi qu'une fonction \( f\colon X\to Y\). Soit \( a\in X\) et \( \ell\in Y\). Nous avons
    \begin{equation}
        \lim_{x\to a} f(x)=\ell
    \end{equation}
    si et seulement si
    \begin{equation}
        \lim f(x_k)=\ell
    \end{equation}
    pour toute suite \( (x_k)\) convergente vers \( a\).

    De plus, une des deux limites existe si et seulement si l'autre existe.
\end{proposition}

\begin{proof}
    En deux parties.
    \begin{subproof}
    \item[Sens direct]
        Nous considérons une suite \( (x_k)\) qui converge vers \( a\) dans \( X\). Soient \( \epsilon>0\), et \( \delta>0\) tel que 
        \begin{equation}
            0<| x-a |<\delta\Rightarrow\,\| f(x)-\ell \|<\epsilon.
        \end{equation}
        Soit encore \( N>0\) tel que pour tout \( k>N\) nous ayons \( \| x_k-a \|<\delta\). Pour un tel \( k\) nous avons
        \begin{equation}
            \| f(x_k)-\ell \|\leq \epsilon,
        \end{equation}
        ce qui signifie que \(f(x_k)\to \ell \).

    \item[Réciproque]
        Pour la réciproque, nous passons par la contraposée. C'est à dire que nous supposons que \( \ell\) n'est pas une limite de \( f\) pour \( x\to a\). Il existe un \( \epsilon\) tel que pour tout \( \delta\), il existe un \( x\) vérifiant \( \| x-a \|<\delta\) et \( \| f(x)-\ell \|>\epsilon\).

        Nous construisons à présent une suite de la manière suivante. Pour \( \delta=\frac{1}{ n }\) nous considérons \( x_n\) tel que \( \| x_n-a \|<\delta\) et \( \| f(x_n)-\ell \|>\epsilon\). Cette suite converge vers \( a\), mais la suite \( f(x_n)\) ne converge manifestement pas vers \( \ell\) : elle ne rendre jamais dans la boule \( B(\ell,\epsilon)\).
    \end{subproof}
\end{proof}

Les espaces métriques ont une propriété importante que la \wikipedia{fr}{Espace_séquentiel}{fermeture séquentielle} est équivalente à la fermeture.
\begin{proposition}[Caractérisation séquentielle d'un fermé]    \label{PropLFBXIjt}
    Soient \( X\) un espace métrique et \( F\subset X\). L'ensemble \( F\) est fermé si et seulement si toute suite contenue dans \( F\) et convergeant dans \( X\) converge vers un élément de \( F\).
\end{proposition}
\index{fermeture séquentielle}
\index{séquentiellement fermé}

\begin{proof}
    Si \( F\) est fermé alors \( X\setminus F\) est ouvert. Soit \( x_n\stackrel{X}{\longrightarrow}x\) une suite à valeurs dans \( F\). Si \( A\) est un ouvert autour de \( x\) contenu dans \( X\setminus F\) (existence par le théorème~\ref{ThoPartieOUvpartouv}), alors la suite ne peut pas entrer dans \( A\) et ne peut donc pas converger vers \( x\).

    Dans l'autre sens maintenant. Supposons que \( X\setminus F\) ne soit pas ouvert. Alors il existe \( x\in X\setminus F\) pour lequel tout voisinage intersecte \( F\). En prenant \( x_k\in B(x,\frac{1}{ k })\), nous construisons une suite contenue dans \( F\) qui converge vers \( x\).
\end{proof}

\begin{proposition} \label{PropXIAQSXr}
    Soient \( E\) et \( Y\) deux espaces métriques. Soit \( f\colon E\to Y\) une application séquentiellement continue. Alors \( f\) est continue.
\end{proposition}

\begin{proof}
    Soit \( \mO\) un ouvert de \( Y\); nous allons voir que le complémentaire de \( f^{-1}(\mO)\) est fermé dans \( E\). Pour cela nous considérons une suite convergente \( x_k\stackrel{E}{\longrightarrow} x\) avec \( x_k\in\complement f^{-1}(\mO)\) pour tout \( k\). Nous allons montrer que \( x\in \complement f^{-1}(\mO)\) et la caractérisation séquentielle\footnote{Proposition~\ref{PropLFBXIjt}.} de la fermeture conclura que \( \complement f^{-1}(\mO)\) est fermé.

    Pour tout \( k\), nous avons \( f(x_k)\in\complement \mO\), mais \( \mO\) est ouvert et \( f(x_k)\stackrel{Y}{\longrightarrow}f(x)\) parce que \( f\) est séquentiellement continue. Par conséquent \( f(x)\in\complement \mO\) et \( x\in\complement f^{-1}(\mO)\).
\end{proof}

%TODO : il y a ici trois théorèmes sur la continuité séquentielle. Il faut sans doute les fusionner.

\begin{proposition} \label{PROPooKNVUooMbLZoy}
    Une fonction séquentiellement continue sur un espace métrisable et à valeurs dans un espace métrique est continue.
\end{proposition}

\begin{proof}
    Soient \( E\) un espace métrique et \( \phi\colon X\to (E,d)\) un homéomorphisme. Nous supposons que \( f\colon X\to Y\) est séquentiellement continue. Nous considérons l'application \( \tilde f=f\circ\phi^{-1}\), c'est à dire
    \begin{equation}
        \begin{aligned}
            \tilde f\colon E&\to Y \\
            a&\mapsto f\big( \phi^{-1}(a) \big).
        \end{aligned}
    \end{equation}
    L'application \( \phi^{-1}\) est continue et donc séquentiellement continue. De plus \( \tilde f\) est séquentiellement continue. En effet si \( a_k\stackrel{E}{\longrightarrow}a\), alors
    \begin{equation}
        \tilde f(a_k)=f\big( \phi^{-1}(a_k) \big),
    \end{equation}
    mais \( \phi^{-1}\) est séquentiellement continue, donc \( \phi^{-1}(a_k)\stackrel{X}{\longrightarrow}\phi^{-1}(a)\), ce qui signifie que \( \phi^{-1}(a_k)\) est une suite convergente dans \( X\) et donc
    \begin{equation}
        \lim_{k\to \infty} \tilde f(a_k)=\lim_{k\to \infty} f\big( \phi^{-1}(a_k) \big)=f\big( \phi^{-1}(a) \big)=\tilde f(a).
    \end{equation}
    L'application \( \tilde f\) est donc séquentiellement continue. Mais étant donné que \( \tilde f\) est définie sur un espace métrique (\( E\)) et à valeurs dans un métrique, elle est continue par la proposition~\ref{PropXIAQSXr}. L'application \( f=\tilde f\circ\phi\) est donc continue en tant que composée d'applications continues.
\end{proof}

\begin{proposition} \label{PropCJGIooZNpnGF}
    Si \( X\) et \( Y\) sont deux espaces métriques et \( f,g\colon X\to Y\) sont deux fonctions continues égales sur une partie dense de \( X\) alors \( f=g\).
\end{proposition}
\index{fonction!continue!égales}

\begin{proof}
    Les fonctions \( f\) et \( g\) sont séquentiellement continues (proposition~\ref{PropFnContParSuite}\ref{ItemWJHIooMdugfu}). Soient \( A\) un ensemble dense dans \( X\) sur lequel \( f\) et \( g\) sont égales, et \( x\notin A\). Vu que \( A\) est dense, il existe une suite \( a_n\) dans \( A\) telle que \( a_n\to x\). La séquentielle continuité de \( f\) et \( g\) donnent
    \begin{subequations}
        \begin{align}
            f(a_n)\to f(x)\\
            g(a_n)\to g(x),
        \end{align}
    \end{subequations}
    mais pour tout \( n\), \( f(a_n)=g(a_n)\). Par unicité de la limite\footnote{Proposition~\ref{PropFObayrf}.} dans \( Y\), \( f(x)=g(x)\).
\end{proof}

%---------------------------------------------------------------------------------------------------------------------------
\subsection{Distance entre un point et un ensemble}
%---------------------------------------------------------------------------------------------------------------------------

\begin{definition}
	Si $A$ est une partie de l'espace métrique $(X,d)$ et si $x\in X$, nous disons que la \defe{distance}{distance!point et ensemble} entre $A$ et $x$ est le nombre
	\begin{equation}		\label{EqdefDistaA}
		d(x,A)=\inf_{a\in A}d(x,a).
	\end{equation}
\end{definition}
%The result is on the figure~\ref{LabelFigDistanceEnsemble}
\newcommand{\CaptionFigDistanceEnsemble}{La distance entre $x$ et $A$ est donnée par la distance entre $x$ et $p$. Les distances entre $x$ et les autres points de $A$ sont plus grandes que $d(x,p)$.}
\input{auto/pictures_tex/Fig_DistanceEnsemble.pstricks}

\begin{proposition}     \label{PropGULUooNzqZKj}
    Soient \( (X,d) \) un espace topologique métrique et \( F\) un fermé de \( X\). Nous avons \( d(x,F)=0\) si et seulement si \( x\in F\).
\end{proposition}

\begin{proof}
    Si \( x\in F\) alors \( d(x,F)=0\) parce que \( d(x,x)\) fait partie de l'ensemble sur lequel nous prenons l'infimum.

    Si réciproquement \( d(x,F)=0\), cela signifie que pour tout \( \epsilon\), il existe \( x_{\epsilon}\in F\) tel que \( d(x_{\epsilon},x)\leq \psi\). En prenant \(\epsilon=1/k\) nous construisons une suite \( (x_k)\) d'éléments dans \( F\) vérifiant \( d(x_k,x)=\frac{1}{ k }\). Cela signifie que \( \lim_{k\to \infty} x_k=x\) par la proposition~\ref{PropooUEEOooLeIImr}\ref{ItemooROYMooAQCXnj}.

    Par la caractérisation séquentielle des fermés (un fermé contient les limites de toutes ses suites, proposition~\ref{PropLFBXIjt}), la suite \( (x_k)\) étant dans \( F\), la limite est dans \( F\). Donc \( x\in F\).
\end{proof}

%---------------------------------------------------------------------------------------------------------------------------
\subsection{Compacité}
%---------------------------------------------------------------------------------------------------------------------------

\begin{lemma}[de Lebesgue\cite{AntoniniAndAl-EspacesMetriquesCompacts}]    \label{LemQFXOWyx}
    Soit \( (X,d)\) un espace métrique tel que toute suite ait une sous-suite convergente à l'intérieur de l'espace. Si \( \{ V_i \}\) est un recouvrement par des ouverts de \( X\), alors il existe \( \epsilon\) tel que pour tout \( x\in X\), nous ayons \( B(x,\epsilon)\subset V_i\) pour un certain \( i\).
\end{lemma}

\begin{proof}
    Par l'absurde, nous supposons que pour tout \( n\), il existe un \( x_n\in X\) tel que la boule \( B(x_n,\frac{1}{ n })\) n'est contenue dans aucun des \( V_i\). Ce des \( x_n\) nous extrayons une sous-suite convergente (que nous nommons encore \( (x_n)\)) et nous posons \( x_n\to x\). Pour \( n\) assez grand (\( \frac{1}{ n }<\epsilon\)) nous avons \( x_n\in B(x,\epsilon)\), donc tous les \( x_n\) suivants sont dans le \( V_i\) qui contient \( x\).
\end{proof}

\begin{lemma}[\cite{AntoniniAndAl-EspacesMetriquesCompacts}]   \label{LemMGQqgDG}
    Soit \( (X,d)\) un espace métrique tel que toute suite possède une sous-suite convergente. Pour tout \( \epsilon>0\), il existe un ensemble fini \( \{ x_i \}_{i\in I}\) tel que les boules \( B(x_i,\epsilon)\) recouvrent \( X\).
\end{lemma}

\begin{proof}
    Soit par l'absurde un \( \epsilon>0\) contredisant le lemme. Il n'existe pas d'ensemble finis autour des points duquel les boules de taille \( \epsilon\) recouvrent \( X\).

    Nous construisons par récurrence une suite ne possédant pas de sous-suites convergente. Le premier terme, \( x_0\) est pris arbitrairement dans \( X\). Ensuite si nous en avons \( N\) termes, nous savons que les boules de rayon \( \epsilon\) et centrées en les points \( \{ x_i \}_{i=1,\ldots, N}\) ne recouvrent pas \( X\). Donc nous prenons \( x_{N+1}\) hors de l'union de ces boules.

    Ainsi nous avons une suite \( (x_n)\) dont tous les termes sont à distance plus grande que \( \epsilon\) les uns des autres. Une telle suite ne peut pas contenir de sous-suite convergente. Contradiction.
\end{proof}

\begin{theorem}[Bolzano-Weierstrass\cite{AntoniniAndAl-EspacesMetriquesCompacts}]\label{ThoBWFTXAZNH}
    Un espace métrique est compact si et seulement si toute suite admet une sous-suite qui converge à l'intérieur de l'espace.
\end{theorem}
\index{théorème!Bolzano-Weierstrass}
\index{Bolzano-Weierstrass!espaces métriques}
\index{compacité}
Une version dédiée à \( \eR^n\) sera démontrée dans le théorème~\ref{ThoBolzanoWeierstrassRn}.

\begin{proof}
   Soient \( X\) un espace métrique compact et \( (x_n)\) une suite dans \( X\). Nous considérons la suite de fermés emboîtés
   \begin{equation}
       X_n=\overline{ \{ x_k\tq k>n \} }.
   \end{equation}
   Ce sont des fermés ayant la propriété d'intersection finie non vide, et donc la proposition~\ref{PropXKUMiCj} nous dit qu'ils ont une intersection non vide. Un élément de cette intersection est automatiquement un point d'accumulation de la suite.

   Nous passons à l'autre sens. Nous supposons que toute suite dans \( X\) contient une sous-suite convergente, et nous considérons \( \{ V_i \}_{i\in I}\), un recouvrement de \( X\) par des ouverts. Par le lemme~\ref{LemQFXOWyx}, nous considérons un \( \epsilon\) tel que pour tout \( x\), il existe un \( i\in I\) avec \( B(x,\epsilon)\subset V_i\). Par le lemme~\ref{LemMGQqgDG}, nous considérons un ensemble fini \( \{ y_i \}_{i\in A}\) tel que le boules \( B(y_i,\epsilon)\) recouvrent \( X\).

   Par construction, chacune de ces boules \( B(y_i,\epsilon)\) est contenue dans un des ouverts \( V_i\). Nous sélectionnons donc parmi les \( V_i\) le nombre fini qu'il faut pour recouvrir les \( B(y_i,\epsilon)\) et donc pour recouvrir \( X\).
\end{proof}

\begin{example}[Non compacité de la boule unité en dimension infinie]\label{ExEFYooTILPDk}
    Le théorème de Bolzano-Weierstrass permet de voir tout de suite que la boule unité n'est pas compacte dans un espace vectoriel de dimension infinie : la suite des vecteurs de base ne possède pas de sous-suites convergentes.
\end{example}

Le théorème de Bolzano–Weierstrass~\ref{ThoBWFTXAZNH} a l'importante conséquence suivante.
\begin{theorem}[Weierstrass]		\label{ThoWeirstrassRn}
	Une fonction continue à valeurs réelles définie sur un compact est bornée et atteint ses bornes.
\end{theorem}
\index{théorème!Weierstrass}
\index{compact!et fonction continue}

\begin{proof}
	Soient \( K\) un compact et $f\colon K\to \eR$ une fonction continue. Nous désignons par $A$ l'ensemble des valeurs prises par $f$ sur $K$ :
	\begin{equation}
		A=f(K)=\{ f(x)\tq x\in K \}.
	\end{equation}
	Nous considérons le supremum $M=\sup A=\sup_{x\in K}f(x)$ avec la convention comme quoi si $A$ n'est pas borné supérieurement, nous posons $M=\infty$ (voir définition~\ref{DefSupeA}).

	Nous allons maintenant construire une suite $(x_n)$ de deux façons différentes suivant que $M=\infty$ ou non.
	\begin{enumerate}
		\item
			Si $M=\infty$, nous choisissons, pour chaque $n\in\eN$, un $x_n\in K$ tel que $f(x_n)>n$. Cela est certainement possible parce que si $A$ n'est pas borné, nous pouvons y trouver des nombres aussi grands que nous voulons.
		\item
			Si $M<\infty$, nous savons que pour tout $\varepsilon$, il existe un $y\in A$ tel que $y>M-\varepsilon$. Pour chaque $n$, nous choisissons donc $x_n\in K$ tel que $f(x_n)>M-\frac{1}{ n }$.
	\end{enumerate}
    Quel que soit le cas dans lequel nous sommes, la suite $(x_n)$ est une suite dans $K$ qui est compact, et donc nous pouvons en extraire une sous-suite convergente à l'intérieur de \( K\) par le théorème de Bolzano-Weierstrass~\ref{ThoBWFTXAZNH}. Afin d'alléger la notation, nous allons noter $(x_n)$ la sous-suite convergente. Nous avons donc
	\begin{equation}
		x_n\to x\in K.
	\end{equation}
	Par la proposition~\ref{PropFnContParSuite}, nous avons que $f$ prend en \( x\) la valeur
	\begin{equation}
		f(x)=\lim_{n\to \infty} f(x_n).
	\end{equation}
	Donc $f(x)<\infty$. Évidemment, si nous avions été dans le cas où $M=\infty$, la suite $x_n$ aurait été choisie pour avoir $f(x_n)>n$ et donc il n'aurait pas été possible d'avoir $\lim_{n\to \infty} f(x_n)<\infty$. Nous en concluons que $M<\infty$, et donc que $f$ est bornée sur $K$.

	Afin de prouver que $f$ atteint sa borne, c'est à dire que $M\in A$, nous considérons les inégalités
	\begin{equation}
		M-\frac{1}{ n }<f(x_n)\leq M.
	\end{equation}
	En passant à la limite $n\to \infty$, ces inégalités deviennent
	\begin{equation}
		M\leq f(x)\leq M,
	\end{equation}
	et donc $f(x)=M$, ce qui prouve que $f$ atteint sa borne $M$ au point $x\in K$.
\end{proof}

\begin{lemma}       \label{LemooynkH}
    Soit \( A_n\) une suite décroissante de fermés dans un espace métrique\footnote{L'hypothèse métrique provient de l'utilisation de Bolzano-Weierstrass, lequel est vrai pour les espaces séquentiellement compacts, dont les espaces métriques.} compact \( K\). Alors
    \begin{equation}
        C=\bigcap_{n\in \eN}A_n
    \end{equation}
    est non vide.
\end{lemma}

\begin{proof}
    Soit \( (x_n)\) une suite dans \( K\) telle que \( x_n\in A_n\). La suite étant contenue dans \( A_1\), et \( A_1\) étant compact (lemme~\ref{LemnAeACf}), elle possède une sous-suite \( (y_n=x_{\sigma_1(n)})\) convergente dont la limite est dans \( A_1\) par le théorème de Bolzano-Weierstrass~\ref{ThoBWFTXAZNH}. Une queue de la suite \( y_n\) est dans \( A_2\) et nous considérons donc une sous-suite convergente dans \( A_2\) donnée par
    \begin{equation}
        z_n=y_{\sigma_2(n)}=x_{\sigma_1\sigma_2(n)}.
    \end{equation}
    En continuant ainsi nous construisons une suite convergente dans \( A_k\). Nous considérons enfin la suite
    \begin{equation}
        y_n=x_{\sigma_1\ldots \sigma_n(n)}.
    \end{equation}
    Pour tout \( k\), une queue de cette suite est une sous-suite de \( x_{\sigma_1\ldots \sigma_k(n)}\) et par conséquent cette suite converge dans \( A_k\). La limite de cette suite est donc dans l'intersection demandée.
\end{proof}

\begin{remark}
    Cette propriété est fausse pour les ouverts. Par exemple
    \begin{equation}
        \bigcap_{n>1}\mathopen] 0 , \frac{1}{ n } \mathclose[=\emptyset.
    \end{equation}
\end{remark}

\begin{lemma}   \label{LemKIcAbic}
    Si \( K\) est un compact dans un espace métrique et \( F\) un fermé disjoint de \( K\), alors \( d(K,F)>0\).
\end{lemma}

\begin{proof}
    Le fonction
    \begin{equation}
        \begin{aligned}
             K&\to \eR \\
            x&\mapsto d(x,F)
        \end{aligned}
    \end{equation}
    est une fonction continue sur \( K\), et donc atteint son minimum par le théorème de Weierstrass~\ref{ThoWeirstrassRn}. Soit \( x_0\in K\) un point de \( K\) qui réalise ce minimum. Si \( d(x_0,F)=0\), alors on aurait une suite \( (x_n)\) dans \( F\) qui convergerait vers \( x_0\), mais \( F\) étant fermé cela signifierait que \( x_0\) serait dans \( F\), ce qui contredirait l'hypothèse que \( F\) et \( K\) sont disjoints.
\end{proof}

\begin{proposition}[\cite{AntoniniAndAl-EspacesMetriquesCompacts}]
    Une isométrie d'un espace métrique compact sur lui-même est une bijection.
\end{proposition}

\begin{proof}
    Soient \( X\) un espace métrique compact et \( f\colon X\to X\) une isométrie. Le fait que \( f\) soit injective est obligatoire (sinon il y a des images dont la distance est nulle). Il faut montrer que \( f\) est surjective.

    Soit \( x\in X\) hors de \( f(X)\). Le lemme~\ref{LemKIcAbic} appliqué au fermé \( \{ x \}\) et au compact \( f(K)\) donne un \( r>0\) tel que
    \begin{equation}
        d\big( x,f(K)\big)>r.
    \end{equation}
    Soit la suite \( u_n=f^n(x)\); c'est une suite dans \( K\) et possède donc une sous-suite convergente (Bolzano-Weierstrass\ref{ThoBWFTXAZNH}) que l'on nomme \( (y_n)\). Vu que \( f\) est une isométrie,
    \begin{equation}
        d(y_{n},y_{n+1})=d(x,y_m)>r
    \end{equation}
    pour un certain \( m\leq n+1\). Cela signifie que pour tout \( n\), nous avons \( d(y_n,y_{n+1})>r\), ce qui contredit le fait que la suite \( (y_n)\) converge.
\end{proof}

\begin{proposition} \label{PropLHWACDU}
    Soient \( (X,d)\) un espace métrique compact et \( (u_n)\) une suite de \( X\) telle que
    \begin{equation}
        \lim_{n\to \infty} d(u_n,u_{n+1})=0.
    \end{equation}
    Alors l'ensemble des points d'accumulation de \( (u_n)\) est connexe.
\end{proposition}
\index{connexité!points d'accumulation}
\index{compacité}

\begin{proof}
    Nous notons \( \Gamma\) l'ensemble des points d'accumulation de la suite.
    \begin{subproof}
    \item[\( \Gamma\) est compact]
        Nous notons \( A_p=\{ u_n\tq n\geq p \}\) et nous avons
        \begin{equation}
            \Gamma=\bigcap_{p\in \eN}\overline{ A_p }
        \end{equation}
        parce que si \( x\in\Gamma\), alors pour tout \( n\), il existe \( m>n\) tel que \( x_m\in B(x,\epsilon)\), et donc tel que \( x\in B(x_m,\epsilon)\). Donc pour tout \( \epsilon\) et pour tout \( p\), l'intersection \( B(x,\epsilon)\cap A_p\) est non vide.

        En tant qu'intersection de fermés, \( \Gamma\) est fermé (lemme~\ref{LemQYUJwPC}). En tant que fermé dans un compact, \( \Gamma\) est compact (lemme~\ref{LemnAeACf}).

    \item[Recouvrement par deux compacts]

        Supposons que \( \Gamma\) ne soit\quext{est-ce qu'il faut vraiment un subjonctif ici ?} pas connexe. Nous pouvons alors considérer \( S\) et \( O\), deux ouverts disjoints recouvrant \( \Gamma\) et intersectant tout deux \( \Gamma\). Nous posons alors
        \begin{subequations}
            \begin{align}
                A&=S\cap\Gamma\\
                B&=O\cap\Gamma,
            \end{align}
        \end{subequations}
        et nous avons évidemment \( \Gamma=A\cup B\). Montrons que \( A\) est fermé (\( B\) le sera aussi par le même raisonnement). Soit une suite d'éléments de \( S\cap \Gamma\) convergent dans \( X\). Alors la limite est dans \( \bar\Gamma=\Gamma\) et donc elle est donc \( O\) ou \( S\), mais elle est certainement dans \( \bar S\). Cependant \( \bar S\) n'intersecte pas \( O\). En effet si \( x\in \bar S\cap O\), alors tout voisinage de \( x\) intersecterait \( S\), mais il y a des voisinages de \( x\) étant inclus dans \( O\) parce que \( O\) est ouvert; cela donnerait une intersection entre \( O\) et \( S\), ce qui est impossible. Donc la limite n'est pas dans \( O\) et donc elle est dans \( S\). Au final la limite est dans \( S\cap \Gamma\), ce qui prouve son caractère fermé.

        Comme d'habitude, \( \Gamma\cap S\) est compact parce que fermé dans un compact.

    \item[Décomposition en trois morceaux]

        Vu que \( A\) et \( B\) sont des compacts disjoints, nous avons \( d(A,B)=\alpha>0\) pour un certain \( \alpha\) par le lemme~\ref{LemKIcAbic}. Nous notons
        \begin{subequations}
            \begin{align}
                A'&=\{x\in X\tq d(x,A)<\frac{ \alpha }{ 3 }\}\\
                B'&=\{x\in X\tq d(x,B)<\frac{ \alpha }{ 3 }\}
            \end{align}
        \end{subequations}
        Nous avons \( A'=\bigcup_{x\in A}B(x,\frac{ \alpha }{ 3 })\) et donc en tant qu'union d'ouverts, \( A'\) est ouvert (définition de la topologie). Même chose pour \( B'\).

        Enfin nous notons
        \begin{equation}
            K=X\setminus(A'\cup B')
        \end{equation}
        qui est fermé en tant que complémentaire d'ouvert, et donc compact. Étant donné que \( A\subset A'\) et \( B\subset B' \), nous avons \( K\cap \Gamma=\emptyset\).

        L'idée est maintenant de montrer que \( K\) contient un point d'accumulation de \( (u_n)\).

    \item[Sous-suites de \( (u_n)\)]

        L'hypothèse sur la suite \( (u_n)\) nous indique qu'il existe un \( N_0\) tel que \( \forall n\geq N_0\),
        \begin{equation}    \label{EqIHioHjW}
            d(u_{n},u_{n+1})<\frac{ \alpha }{ 3 }.
        \end{equation}
        Soient \( N>N_0 \) et \( x_0\in A\). Étant donné que \( x_0\) est point d'accumulation de la suite, il existe \( n_1>N\) tel que \( d(x_0,u_{n_1})<\frac{ \alpha }{ 3 }\). Même chose dans \( B\) : nous prenons \( y_0\in B\) et un naturel \( n_2>n_1\) tel que \( d(y_0,u_{n_2})<\frac{ \alpha }{ 3 }\). Nous avons \( u_{n_1}\in A'\) et \( u_{n_2}\in B'\).

        Soit \( n_0\) le plus petit naturel supérieur à \( n_1\) tel que \( u_{n_0}\notin A'\). Cela existe parce que \( u_{n_2}\in B'\) et \( B'\cap A'=\emptyset\), mais \( n_0\) n'est pas \( n_2\) lui-même parce que \( d(A',B')\geq \frac{ \alpha }{ 3 }\) alors que nous considérons \( n_0,n_1,n_2>N_0\) et donc pour tous les \( i\) entre \( n_1\) et \( n_2\) (compris), \( d(u_i,u_{i+1})<\frac{ \alpha }{ 3 }\). Notons qu'ici le strict dans la condition \eqref{EqIHioHjW} est important. Nous avons donc \(N_0<n_1<n_0<n_2\).

        Nous allons maintenant montrer que \( u_{n_0}\) est dans \( K\). C'est fait pour : il est loin en même temps de \( A'\) et de \( B'\). En utilisant l'inégalité triangulaire à l'envers, nous avons
        \begin{equation}
            \begin{aligned}[]
            d(u_{n_0},B)&\geq d(u_{n_0-1},B)-d(u_{n_0-1},u_{n0})\\
            &\geq d(A,B)-d(u_{n_0-1},A)-d(u_{n_0-1},u_{n_0})\\
            &\geq \alpha-\frac{ \alpha }{ 3 }-\frac{ \alpha }{ 3 }\\
            &=\frac{ \alpha }{ 3 }.
            \end{aligned}
        \end{equation}
        Pour la dernière inégalité nous avons utilisé le fait que \( u_{n_0-1}\) n'est pas dans \( A'\). Bref, nous avons montré que \( u_{n_0}\) n'est pas dans \( B'\) (dans la définition de ce dernier nous avons bien une inégalité stricte). Vu que par définition \( u_{n_0}\) n'est pas non plus dans \( A'\), nous avons \( u_{n_0}\in K\).

        Nous avons montré jusqu'à présent que pour tout \( N\geq N_0\), il existe un \( n_0\geq N\) tel que \( u_{n_0}\in K\). Cela nous construit donc une sous-suite \( (v_n)\) de \( (u_n)\) contenue dans \( K\). En tant que suite dans le compact \( K\), la suite \( (v_n)\) admet un point d'accumulation dans \( K\). Ce point est également point d'accumulation de la suite \( (u_n)\) complète, ce qui donne un point d'accumulation de \( (u_n)\) dans \( K\) et donc une contradiction.

    \end{subproof}
    Nous concluons que \( \Gamma\) est connexe.
\end{proof}

Encore une petite conséquence sans ambition du théorème de Bolzano-Weierstrass.
\begin{proposition}\label{PropHNylIAW}
    Si \( (x_n)\) est une suite dans un compact telle que toute sous-suite convergente ait le même point \( x\) comme limite. Alors la suite entière converge vers \( x\).
\end{proposition}

\begin{proof}
    Supposons que ce ne soit pas le cas. Alors il existe un \( \epsilon\) tel que pour tout \( N>0\), il existe \( n>N\) avec \( d(x_n,x)>\epsilon\). Cela nous donne une sous-suite de \( (x_n)\) composée d'éléments tous à une distance de \( x\) supérieure à \( \epsilon\). Nous la nommons \( (y_n)\); c'est une suite dans un compact qui admet donc une sous-suite convergente (et une telle sous-suite est une sous-suite de \( (x_n)\)) dont la limite devrait être \( x\), mais c'est impossible par construction.
\end{proof}

\begin{lemmaDef}       \label{LemGDeZlOo}
    Soi \( \Omega\) un ouvert dans un espace métrique \( E\). Il existe une suite \( (K_n)\) de compacts tels que
    \begin{enumerate}
        \item
            \( K_n\subset \Omega\)
        \item
            \( \bigcup_{n=0}^{\infty}K_n=\Omega\)
        \item
            \( K_n\subset\Int(K_{n+1})\).
    \end{enumerate}
    Une telle suite de compacts vérifie alors
    \begin{enumerate}
        \item
            Il existe \( \delta_n\) tel que pour tout \( z\in K_n\), \( B(z,\delta_n)\subset K_{n+1}\).
        \item
            Tout compact de \( \Omega\) est inclus dans \( \Int(K_n)\) pour un certain \( n\).
    \end{enumerate}
    Une telle suite de compacts est une \defe{suite exhaustive}{compact!suite exhaustive}\index{exhaustive (suite de compacts)!} de compacts pour \( \Omega\).
\end{lemmaDef}

\begin{proof}
    Nous considérons les ensembles
    \begin{equation}
        V_n=\{ z\in E\tq | z | \}\cup\bigcup_{a\notin\Omega}B(a,\frac{1}{ n }),
    \end{equation}
    et nous définissons \( K_n=\complement V_n\). Vérifions que ces ensembles vérifient tout ce qu'il faut.
    \begin{enumerate}
        \item
            Si \( a\notin\Omega\) alors \( a\) est dans tous les \( V_n\) et donc dans aucun des \( K_n\); nous avons donc bien \( K_n\subset\Omega\).
        \item
            Si \( z\in \Omega\) alors nous prenons \( n_1>| z |\) puis \( n_2\) tel que \( B(z,\frac{1}{ n_2 })\subset \Omega\). Alors \( z\in K_n\) avec \( n>\max(n_1,n_2)\).
        \item
            Une chose à comprendre est que si \( z\in K_n\), alors \( d(z,\complement \Omega)\geq \frac{1}{ n }\). Du coup si nous prenons \( \delta\) tel que
            \begin{equation}
                \frac{1}{ n+1 }<\delta<\frac{1}{ n }
            \end{equation}
            alors \( B(z,\delta)\subset K_{n+1}\).
        \item
            Enfin, les \( K_n\) sont tous compacts. En effet ils sont bornés parce que \( K_n\subset B(0,n)\) et ensuite \( K_n\) est fermé en tant que complémentaire d'un ouvert (\( V_n\) est ouvert en tant qu'union d'ouverts).
    \end{enumerate}

    Nous passons maintenant aux propriétés, qui sont indépendantes de la façon dont nous avons construit les \( K_n\) vérifiant les conditions.
    \begin{enumerate}
        \item

            Nous pouvons considérer la fonction \( K_n\to \eR\) donnée par \( z\mapsto d(z,\complement K_{n+1})\). Vu que \( K_n\subset\Int(K_{n+1})\), c'est une fonction (continue sur le compact \( K_n\)) prenant des valeurs strictement positives. Elle a donc un minimum strictement positif. Si \( \delta_n\) est plus petit que ce minimum nous avons \( B(z,\delta_n)\subset K_{n+1}\) pour tout \( z\in K_n\).

        \item

            D'abord nous avons \( \Omega=\bigcup_{n=0}^{\infty}\Int(K_n)\). En effet nous avons
            \begin{equation}
                \Omega=\bigcup_{n=0}^{\infty}K_n\subset\bigcup_{n=0}^{\infty}\Int(K_{n+1})\subset\bigcup_{n=0}^{\infty}\Int(K_n).
            \end{equation}
            L'inclusion dans l'autre sens est facile.

            Soit \( K\) compact dans \( \Omega\). Vu que \( \Omega\) est l'union des \( \Int(K_n)\), nous avons
            \begin{equation}
                K\subset\bigcup_{n=0}^{\infty}\Int(K_n).
            \end{equation}
            Cela donne à \( K\) un recouvrement par des ouverts dont nous pouvons extraire un sous-recouvrement fini par compacité. Les \( K_n\) étant croissants, du recouvrement fini, il suffit de prendre le plus grand (disons \( K_m\)) et nous avons \( K\subset\Int(K_m)\).

    \end{enumerate}
\end{proof}
Notons qu'avec la suite de \( K_n\) telle que construite, le dernier point est réglé en prenant
\begin{equation}
    \frac{1}{ n+1 }<\delta_n<\frac{1}{ n }.
\end{equation}


\begin{theorem}[Tykhonov]\index{théorème!Tykhonov}\label{ThoFWXsQOZ}
    Un produit quelconque d'espaces métriques non vides est compact si et seulement si chacun de ses facteurs est compact.
\end{theorem}
Nous n'allons donner la preuve que dans le cas d'un produit fini dans le théorème~\ref{THOIYmxXuu}.

%---------------------------------------------------------------------------------------------------------------------------
\subsection{Ensembles enchaînés}
%---------------------------------------------------------------------------------------------------------------------------

Soit \( (x,d)\) un espace métrique.
\begin{definition}
    Une \defe{\( \epsilon\)-chaîne}{chaîne} joignant les points \( a\) et \( b\) de \( X\) est une suite finie \( (u_0,\ldots, u_n)\) dans \( X\) telle que \( u_0=a\), \( u_n=b\) et pour tout \( 0\leq i\leq n-1\) nous avons \( d(u_n,u_{n+1})\leq \epsilon\).

    Une partie \( A\) de \( X\) est \defe{bien enchaînée}{bien!enchaîné} si pour tout \( \epsilon>0\) et pour tout \( a,b\in A\), il existe une \( \epsilon\)-chaîne joignant \( a\) et \( b\) dans $A$.
\end{definition}
Les rationnels dans \( \eR\) sont bien enchaînés.

\begin{proposition}
    Un espace connexe est bien enchaîné.
\end{proposition}
%TODO: une preuve.

\begin{proposition}
    La fermeture d'un ensemble bien enchaîné dans un espace métrique compact \( (X,d)\) est connexe.
\end{proposition}
\index{connexité}
\index{compacité}

\begin{proof}
    Soit \( A\subset X\) un ensemble bien enchaîné, et soient \( a,b\in \bar A\). Nous construisons une suite \( (u_k)\) dans \( A\) de la façon suivante. Pour chaque \( n>0\) nous prenons \( a'\in B(a,\frac{1}{ n })\cap A\) et \( b'\in B(b,\frac{1}{ n })\cap A\). Ensuite nous considérons une \( \frac{1}{ n }\)-chaîne \( \{ v_i^{(n)} \}_{i\in I_n}\) dans \( A\) entre \( a'\) et \( b'\). Ici l'ensemble \( I_n\) est fini. La suite \( (u_k)\) est simplement construite en mettant bout à bout les éléments \( v_i^{(n)}\).

    La suite ainsi construite est une suite dans \( A\) admettant \( a\) et \( b\) comme points d'accumulation (les autres points d'accumulation sont également dans \( \bar A\)) et telle que \( \lim_{k\to \infty} d(u_k,u_{k+1})=0\). Par conséquent la proposition~\ref{PropLHWACDU} nous dit que l'ensemble des points d'accumulation de \( (u_k)\) est connexe dans \( X\). Nous le notons \( C_{a,b}\).

    Si nous fixons \( a\in \bar A\), alors nous avons
    \begin{equation}
        \bigcup_{x\in \bar A}C_{a,x}=\bar A.
    \end{equation}
    Vu que le membre de gauche est une union de connexes, c'est un connexe par la proposition~\ref{PropIWIDzzH}.
\end{proof}
En particulier, un espace métrique compact est connexe si et seulement s'il est bien enchaîné.

%---------------------------------------------------------------------------------------------------------------------------
\subsection{Produit fini d'espaces métriques}
%---------------------------------------------------------------------------------------------------------------------------

\begin{definition}\label{DefZTHxrHA}
    Si \( (E_1,d_1)\),\ldots, \( (E_n,d_n)\) sont des espaces métriques nous mettons la distance suivante sur le produit cartésien \( E=E_1\times\ldots\times E_n\) :
    \begin{equation}
        d(x,y)=\max_{i=1,\ldots, n}d_i(x_i,y_i).
    \end{equation}
\end{definition}

\begin{theorem}[\cite{MonCerveau}]\label{THOIYmxXuu}
    Un produit fini d'espaces métriques non vides est compact si et seulement si chacun de ses facteurs est compact.
\end{theorem}
\index{compact!produit fini}
\index{théorème!Tykhonov!fini}

\begin{proof}
    Soient \( K_1\),\ldots, \( K_n\) des compacts et \( K=K_1\times \ldots\times K_n\) le produit muni de sa métrique usuelle de la définition \eqref{DefZTHxrHA} (attention : chacun des \( K_i\) peut être de dimension infinie) :
    \begin{equation}
        d(\alpha,\beta)=\max\{ d_i(\alpha_i,\beta_i) \}
    \end{equation}
    où \( d_i\) est la distance sur \( K_i\). Si \( (\alpha_n)\) est une suite dans \( K\) alors la suite \( (\alpha_n)_1\) est une suite dans le compact \( K_1\) dont nous pouvons extraire une sous-suite convergente (Bolzano-Weierstrass~\ref{ThoBWFTXAZNH}). De la sous-suite de \( \alpha\) correspondante nous extrayons la sous-suite pour la seconde composante, etc.

    En fin de compte nous avons une sous-suite (que nous nommons \( \alpha\) également) donc chacune des composantes est convergente. Nous nommons \( \ell_k\) les limites correspondantes. Soit \( \epsilon>0\) pour chaque \( k=1,\ldots, n\), il existe \( N_k>0\) tel que si \( p>N_k\) alors
    \begin{equation}
        d\big( (\alpha_p)_k-\ell_p \big)\leq \epsilon.
    \end{equation}
    Ici \( \alpha_p\in K\) est le \( p\)\ieme élément de la suite \( \alpha\) et \( (\alpha_p)_i\in K_i\) est la \( i\)\ieme composante de \( \alpha_p\). En prenant \( N=\max_kN_k\) et \( n>N\) nous avons
    \begin{equation}
        d\big( \alpha_n,(\ell_1,\ldots, \ell_n) \big)\leq\epsilon.
    \end{equation}
    Par conséquent de la suite \( (\alpha)\) nous avons extrait une sous-suite convergente et la partie «réciproque» de Bolzano-Weierstrass nous assure alors que \( K\) est compact.

    À l'inverse si un des facteurs n'est pas compact (mettons \( K_1\)) alors nous prenons un recouvrement \( \{ \mO_i \}_{i\in I}\) de \( K_1\) par des ouverts duquel il est impossible d'extraire un sous-recouvrement fini. Ensuite nous posons
    \begin{equation}
        \mP_i=\mO_i\times K_2\times\ldots\times K_n,
    \end{equation}
    qui est un recouvrement de \( K\) par des ouverts (de \( K\)) d'où aucun sous-recouvrement fini ne peut être extrait.
\end{proof}

Pour la culture générale, il y a bien entendu moyen de faire des produits dénombrables et pire d'espaces métriques.
\begin{definition}[\cite{AntoniniAndAl-TheoremeTykhonov}]
    Soient \( (E_n,d_n)\) des espaces métriques. Sur l'ensemble produit \( E=\prod_{i=1}^{\infty}E_i\) nous définissons la métrique
    \begin{equation}
        d(x,y)=\sum_{k=1}^{\infty}\frac{1}{ 2^k }d'_k(x_i,y_i)
    \end{equation}
    où \( d'_i=\min(d_i,1)\).
\end{definition}
On peut montrer que ce \( d\) est bien une distance et que \( (E,d)\) devient un espace métrique.

\begin{theorem}[Tykhonov dénombrable\cite{AntoniniAndAl-TheoremeTykhonov}] \label{ThoKKBooNaZgoO}  % Ce résultat n'est pas censé être utilisé dans l'agrégation.
    Un produit dénombrable d'espaces métriques non vides est compact si et seulement si chacun de ses facteurs est compact.
\end{theorem}
\index{compact!produit dénombrable}
\index{théorème!Tykhonov!dénombrable}
Note : ce résultat est encore valable pour un produit quelconque, c'est le théorème de Tykhonov~\ref{ThoFWXsQOZ}.

%+++++++++++++++++++++++++++++++++++++++++++++++++++++++++++++++++++++++++++++++++++++++++++++++++++++++++++++++++++++++++++
\section{Ensembles nulle part denses}
%+++++++++++++++++++++++++++++++++++++++++++++++++++++++++++++++++++++++++++++++++++++++++++++++++++++++++++++++++++++++++++

Nous allons nous limiter au cas de \( \eR\), mais je crois que ça se généralise sans trop de peine aux espaces en tout cas métriques. Voir aussi la section~\ref{SecBDlaUrz} sur les espaces de Baire.

\begin{definition}
    Un ensemble est dit \defe{nulle part dense}{nulle part dense}\index{dense!nulle part} s'il n'est dense dans aucun intervalle.

    Un ensemble dans \( \eR\) est de \defe{première catégorie}{catégorie!ensemble de première} ou \defe{maigre}{maigre (ensemble)} s'il est une union dénombrable d'ensembles nulle part dense (c'est à dire d'ensembles denses sur aucun intervalle).
\end{definition}

\begin{theorem}[Baire\cite{BaireZied}]      \label{ThoQGalIO}
    Une réunion dénombrable d'ensembles nulle part denses est d'intérieur vide.
\end{theorem}
\index{Baire!théorème}
\index{théorème!Baire}

\begin{proof}
    Soient \( a\in S\) et \( \epsilon>0\). Nous allons trouver un élément dans \( B(a,\epsilon)\) qui n'est pas dans \( S\). Nous commençons par choisir \( x_1\in B(a,\epsilon)\) et \( r_1<\frac{ \epsilon }{2}\) tel que
    \begin{equation}
        B(x_1,r_1)\cap A_1=\emptyset.
    \end{equation}
    Ensuite nous choisissons \( x_2\in B(x_1,r_1)\) et \( r_2<\epsilon/4\) tel que \( B(x_2,r_2)\subset B(x_1,r_1)\) et \( B(x_2,r_2)\cap A_2=\emptyset\). Notons que \( B(x_2,r_2)\cap A_1=\emptyset\) aussi, par construction.

    Par récurrence nous construisons une suite d'éléments \( x_n\) et de rayons \( r_n<\epsilon/2^n\) tels que
    \begin{enumerate}
        \item
            \( B(x_n,r_n)\cap A_j=\emptyset\) pour tout \( j\leq n\),
        \item
            \( \overline{ B(x_n,r_n) }\subset B(x_{n-1},r_{r-1})\).
    \end{enumerate}
    Cette suite étant de Cauchy (parce que contenue dans des intervalles emboîtés de rayon décroissant vers zéro), elle converge\footnote{Par la proposition~\ref{PROPooTFVOooFoSHPg}} donc vers un point qui en particulier appartient à \( B(a,\epsilon)\). Mais la limite n'est dans aucun des \( A_n\) et donc pas dans \( S\).
\end{proof}

%+++++++++++++++++++++++++++++++++++++++++++++++++++++++++++++++++++++++++++++++++++++++++++++++++++++++++++++++++++++++++++
\section{Encore de la topologie réelle}
%+++++++++++++++++++++++++++++++++++++++++++++++++++++++++++++++++++++++++++++++++++++++++++++++++++++++++++++++++++++++++++

Dans cette section, nous travaillons dans l'espace $\eR^n$ pour un certain naturel $n$. Nous y définissons la notion d'ouvert et de fermé, qui sont la base de la topologie générale. Notons que ces définitions n'ont de sens que relativement à l'espace ambiant, aussi un ouvert de $\eR$ ne sera en général pas un ouvert de $\eR^2$~: d'une part, il n'y a pas d'inclusion canonique de $\eR$ dans $\eR^2$ (les ouverts du second ne sont même pas des sous-ensembles du premier) et, d'autre part, les définitions se basent sur la notion de boule de $\eR^n$ qui dépend évidemment de la valeur de $n$ (une boule dans $\eR$ est un intervalle, dans $\eR^2$ c'est un disque, etc.)

%---------------------------------------------------------------------------------------------------------------------------
\subsection{Ouverts et fermés}
%---------------------------------------------------------------------------------------------------------------------------

\begin{definition}  \label{DefZVuBbqp}
	La \defe{boule ouverte}{boule!ouverte} de centre $x_0 \in \eR^n$ et de rayon $r \in
	\eR^+$ est définie par
	\begin{equation}
		B(x_0,r) = \{ x \in \eR^n \tq \norme{x - x_0} < r \},
	\end{equation}
	tandis que la \defe{boule fermée}{boule!fermée} de centre $x_0$ et de rayon $r$ est
	\begin{equation}
		\bar B(x_0,r) = \{ x \in \eR^n \tq \norme{x - x_0} \leq r \};
	\end{equation}
	la différence est que l'inégalité dans la première est stricte.
\end{definition}

\begin{definition}  \label{DefUOyCQtW}
    Une partie \( A\) de \( \eR^n\) est \defe{ouverte}{ouvert!dans $\eR^n$} si pour tout \( a\in A\) il existe \( r>0\) tel que \( B(a,r)\subset A\). Une partie est donc ouverte lorsqu'elle contient une boule autour de chacun de ses éléments.
\end{definition}
Cette définition est évidemment à mettre en rapport avec le théorème~\ref{ThoPartieOUvpartouv}.

Le lemme suivant justifie le vocabulaire des définitions~\ref{DefZVuBbqp}.
\begin{lemma}   \label{LemMESSExh}
    Pour tout $x \in \eR^n$ et tout $r >0$ la boule \( B(x,r)\) est ouverte.
\end{lemma}

\begin{proof}
    Afin de prouver que la boule est ouverte, nous prenons un point $p\in B(x,r)$, et nous allons montrer qu'il existe une boule autour de $p$ qui est contenue dans $B(x,r)$.

    Étant donné que $p\in B(x,r)$, nous avons $d(p,x)<r$. Prouvons que la boule $B\big(p,r-d(p,x)\big)$ est contenue dans $B(x,r)$. Pour cela, nous prenons $p'\in B\big(p,r-d(p,x)\big)$, et nous essayons de prouver que $p'\in B(x,r)$. En effet, en utilisant l'inégalité triangulaire,
    \begin{equation}
	    d(x,p')\leq d(x,p)+d(p,p')\leq d(x,p)+r-d(p,x)=r.
    \end{equation}
\end{proof}

%---------------------------------------------------------------------------------------------------------------------------
					\subsection{Intérieur, adhérence et frontière}
%---------------------------------------------------------------------------------------------------------------------------

\begin{definition}
  Soient $A \subset \eR^n$ et $x \in \eR^n$. Le point $x$ est \defe{intérieur}{intérieur} à $A$ s'il existe une boule autour de $x$ complètement contenue dans $A$. L'ensemble des points intérieurs à $A$ est noté $\Int A$ ou $\mathring A$, de sorte qu'on a précisément
  \begin{equation*}
    x \in \Int A \iffdefn  \exists \epsilon > 0 \tq
    B(x,\epsilon) \subset A.
  \end{equation*}
\end{definition}

\begin{normaltext}

La notion d'adhérence a déjà été définie en~\ref{DEFooSVWMooLpAVZR}, et précisé par le lemme~\ref{LEMooILNCooOFZaTe}. Dans le cas de \( \eR^n\) dans lequel les boules forment une base de la topologie nous pouvons encore préciser de la façon suivante:
\begin{equation}
	x \in \Adh A \iffdefn \forall \epsilon > 0, B(x,\epsilon) \cap A \neq \emptyset
\end{equation}
\end{normaltext}

\begin{proposition}
Pour $A \subset \eR^n$, nous avons
\begin{equation*}
	\Int A \subseteq A  \subseteq \Adh A
\end{equation*}
\end{proposition}

\begin{definition}      \label{DEFooACVLooRwehTl}
  La \defe{frontière}{frontière} ou le \defe{bord}{bord} de $A$ est défini par $\partial A = \Adh A \setminus \Int A$. L'ensemble $A$ est un \defe{ouvert}{ouvert} si $A = \Int A$, et c'est un \defe{fermé}{fermé} si $A = \Adh A$.
\end{definition}

\begin{lemma}[Caractérisation équivalente de la frontière]      \label{LEMooEUYEooYcUfKr}
    Soient \( X\) un espace topologique et \( S\subset X\). Un point \( x\in X\) est dans \( \partial S\) si et seulement si tout voisinage de \( x\) contient un point de \( S\) et un point de \( S^c\).
\end{lemma}

\begin{proof}
    Supposons que tout voisinage de \( x\) contienne un point de \( S\) et un point de \( S^c\). Alors \( x\in Adh(S)\) (définition~\ref{DEFooSVWMooLpAVZR}), mais pas dans l'intérieur de \( S\) parce que \( x\) ne possède pas de voisinage contenu dans \( S\). Donc \( x\in \partial S\).

    À l'inverse, si \( x\in\partial S\) alors \( x\) est dans l'adhérence de \( S\) et tout voisinage de \( x\) contient un point de \( S\). Mais \( x\) n'est pas dans l'intérieur de \( S\) et tout voisinage de \( x\) contient un point qui n'est pas dans \( S\), aka un point de \( S^c\).
\end{proof}

\begin{corollary}
    Un ensemble et son complémentaire ont même frontière.
\end{corollary}

\begin{proof}
    Conséquence du lemme~\ref{LEMooEUYEooYcUfKr}. Les points de \( \partial(S^c)\) sont caractérisés par le fait que tout voisinage contient un point de \( S^c\) et un point de \( (S^c)^c=S\).
\end{proof}

\begin{example}
    Soit \( X=\mathopen[ 0 , 1 \mathclose]\) muni de la topologie de la distance \( | x-y |\) (définition~\ref{ThoORdLYUu}). Les points \( 0\) et \( 1\) \emph{ne sont pas} dans la frontière de $X$. En effet une boule ouverte autour de \( 1\) est un ensemble de la forme
    \begin{equation}
        B(1,r)=\{ x\in X\tq | x-1 |<r \}=\mathopen] 1-r , 1 \mathclose]
    \end{equation}
    où nous avons supposé \( r<1\).

    Les points \( 0\) et \( 1\) sont par contre sur la frontière de \( \mathopen[ 0 , 1 \mathclose]\) lorsque cet ensemble est vu comme partie de l'espace métrique \( \eR\).
\end{example}

\begin{lemma}[Passage de douane\cite{ooDKEWooFqlDyN,ooWBUCooAdPjMK}]        \label{LEMooLKWEooItGnkP}
    Dans un espace topologique, toute partie connexe qui rencontre à la fois une partie \( A\) et son complémentaire rencontre nécessairement la frontière de \( A\).
\end{lemma}

\begin{proof}
    Nommons \( \gamma\) la partie connexe qui intersecte \( A\) et \( A^c\). Les ouverts \( \Int(A)\) et \( X\setminus \bar A\) ne peuvent pas recouvrir \( \gamma\) parce que ce sont deux ouverts disjoints alors que \( \gamma\) est connexe (voir la définition~\ref{DefIRKNooJJlmiD} de la connexité). Donc \( \gamma\) doit contenir des points qui sont dans \( \bar A\) mais pas dans \( \Int(A)\). C'est à dire des points de \( \partial A\).
\end{proof}

On vérifiera que les notations et les dénominations sont cohérentes en prouvant la proposition suivante.
\begin{proposition}Pour $\epsilon > 0$,
  \begin{enumerate}
  \item l'adhérence de $B(x,\epsilon)$ est $\bar B(x,\epsilon)$,
  \item l'intérieur de $\bar B(x,\epsilon)$ est $B(x,\epsilon)$,
  \item la boule ouverte $B(x,\epsilon)$ est un ouvert,
  \item la boule fermée $\bar B(x,\epsilon)$ est un fermé.
  \end{enumerate}
\end{proposition}

Nous avons également les liens suivants entre intérieur, adhérence, ouvert, fermé et passage au complémentaire (noté ${}^c$)~:
\begin{proposition}
Si $A \subset \eR^n$ et $A^c = \eR^n\setminus A$, nous
  avons
  \begin{enumerate}
  \item $(\Int A)^c = \Adh (A^c)$ et $(\Adh A)^c = \Int
    (A^c)$,
  \item $A$ est ouvert si et seulement si $A^c$ est fermé,
  \item $\Int A$ est le plus grand ouvert contenu dans $A$,
  \item $\Adh A$ est le plus petit fermé contenant $A$,
  \end{enumerate}
\end{proposition}

\begin{example} \label{ExBFLooUNyvbw}
    Il n'est en général pas vrai que \( \overline{ A\cap B }=\bar A\cap \bar B\). Par exemple si \( A=\mathopen[ 0 , 1 [\) et \( B=\mathopen] 1 , 2 \mathclose]\). Dans ce cas, \( A\cap B=\emptyset\) alors que \( \bar A\cap\bar B=\{ 1 \}\).
\end{example}

%---------------------------------------------------------------------------------------------------------------------------
\subsection{Point d'accumulation, point isolé}
%---------------------------------------------------------------------------------------------------------------------------

Soit $D\subset\eR$. Un point $a\in D$ est \defe{isolé}{isolé!élément de $\eR$} dans $D$ (relativement à $\eR$) s'il existe $\varepsilon>0$ tel que
\begin{equation}
	\mathopen[ a-\varepsilon , a+\varepsilon \mathclose]\cap D=\{ a \}.
\end{equation}
Autrement dit, il existe un intervalle autour de $a$ dans lequel $a$ est le seul élément de $D$.

Un point $a\in \eR$ est un \defe{point d'accumulation}{accumulation!dans $\eR$} de $D$ si pour tout $\varepsilon>0$,
\begin{equation}
	\Big( \mathopen[ a-\varepsilon , a+\varepsilon \mathclose]\setminus\{ a \} \Big)\cap D\neq\emptyset.
\end{equation}
Autrement dit, quel que soit l'intervalle autour de  $a$ que l'on considère, le point $a$ n'est pas tout seul dans $D$.

\begin{example}
	Prenons $D=\mathopen[ 0 , 1 [\cup\mathopen] 2 , 3 \mathclose]$. Cet ensemble n'a pas de points isolés, et l'ensemble de ses points d'accumulation est $\mathopen[ 0 , 1 \mathclose]\cup\mathopen[ 2,3  \mathclose]$.

	Notez que les points $1$ et $2$ sont des points d'accumulation de $D$ qui ne font pas partie de $D$. Il est possible d'être un point d'accumulation de $D$ sans être dans $D$, mais pour être un point isolé dans $D$, il faut être dans $D$.
\end{example}

\begin{example}
	Soit $D=\{ \frac{1}{ n }\}_{n\in\eN}$. Tous les points de cet ensemble sont des points isolés (vérifier !).  Aucun point de $D$ n'est point d'accumulation. Cependant $0$ est un point d'accumulation.
\end{example}

% TODO: retrouver où je comptais mettre cette référence.
\cite{GGIibHE}

%---------------------------------------------------------------------------------------------------------------------------
\subsection{Limite de suite}
%---------------------------------------------------------------------------------------------------------------------------

\begin{definition}[Limite d'une suite dans $\eR^m$]
	Une suite de points $(x_n)$ dans $\eR^m$ est dite \defe{convergente}{convergence!suite dans $\eR^m$} s'il existe un élément $\ell\in\eR^m$ tel que
	\begin{equation}	\label{EqCondLimSuite}
		\forall\varepsilon>0,\,\exists N\in \eN\tq\,\forall n\geq N,\,\| x_n-\ell \|<\varepsilon.
	\end{equation}
	Dans ce cas, nous disons que $\ell$ est la \defe{limite}{limite!suite dans $\eR^m$} de la suite $(x_n)$ et nous écrivons $\lim x_n=\ell$ ou plus simplement $x_n\to \ell$.
\end{definition}
Notez aussi la similarité avec la définition~\ref{PropLimiteSuiteNum}.

\begin{remark}
	Nous n'écrivons pas «$\lim_{n\to\infty}x_n$» parce que, lorsqu'on parle de suites, la limite est \emph{toujours} lorsque $n$ tend vers l'infini. Il n'y a aucun intérêt à chercher par exemple $\lim_{n\to 4}x_n$ parce que cela vaudrait $x_4$ et rien d'autre.

	Ceci est une différence importante avec les limites de fonctions.
\end{remark}

\begin{lemma}[Unicité de la limite]
	Il ne peut pas y avoir deux nombres différents qui satisfont à la condition \eqref{EqCondLimSuite}. En d'autres termes, si $\ell$ et $\ell'$ sont deux limites de la suite $(x_n)$, alors $\ell=\ell'$.
\end{lemma}

\begin{proof}
	Soit $\varepsilon>0$. Nous considérons $N$ tel que
	\begin{equation}
		\| x_n-\ell \|<\varepsilon
	\end{equation}
	pour tout $n\geq N$, et $N'>0$ tel que
	\begin{equation}
		\| x_n-\ell' \|<\epsilon
	\end{equation}
	pour tout $n>N'$. Maintenant, nous prenons $n$ plus grand que $N$ et $N'$ de telle façon que les deux équations pour $x_n$ soient vérifiées en même temps. Alors
	\begin{equation}
		\| \ell-\ell' \|=\| \ell-x_n+x_n-\ell' \|\leq\| \ell-x_n \|+\| x_n-\ell' \|<2\varepsilon.
	\end{equation}
	Cela prouve que $\| \ell-\ell' \|=0$.
\end{proof}
Le théorème de Bolzano-Weierstrass~\ref{ThoBWFTXAZNH} dit que dans le cas métrique, la compacité séquentielle est équivalente à la compacité.

%TODO : le théorème sur l'équivalence des normes sur les espaces vectoriels normés devrait être énoncé comme le fait que si N1 et N2 sont deux normes sur V, alors
%       nous avons un isomorphisme d'espace topologique (V,N1) ~ (V,N2). L'isomorphisme étant donné par l'identité.

%+++++++++++++++++++++++++++++++++++++++++++++++++++++++++++++++++++++++++++++++++++++++++++++++++++++++++++++++++++++++++++
\section{Application réciproque}
%+++++++++++++++++++++++++++++++++++++++++++++++++++++++++++++++++++++++++++++++++++++++++++++++++++++++++++++++++++++++++++

\begin{definition}[injection, surjection, bijection]        \label{DEFooBFCQooPyKvRK}
    Soient des ensembles \( A\) et \( B\) ainsi qu'une application \( f\colon A\to B\).
    \begin{enumerate}
        \item
            La fonction \( f\) est \defe{injective}{injection} si \( f(x_1)=f(x_2)\), implique \( x_1=x_2\).
        \item
            La fonction \( f\) est \defe{surjective}{surjection} si tous les éléments de \( B\) sont atteints, c'est à dire si pour tout \( y\in B\) il existe \( x\in A\) tel que \( f(x)=y\).
        \item
            La fonction \( f\) est une \defe{bijection}{bijection} entre \( A\) et \( B\) si elle est injective et surjective, c'est à dire si pour tout \( y\in B\) il existe un unique \( x\in A\) tel que \( f(x)=y\).
    \end{enumerate}
\end{definition}
La surjection et l'injection sont des propriétés bien différentes qu'il convient de prouver séparément. De plus une même «formule» peut définir une application injective, surjective, bijective ou non selon le domaine sur laquelle nous la considérons.

\begin{definition}      \label{DEFooTRGYooRxORpY}
    Soit \( f\colon A\to B\) une bijection. L'\defe{application réciproque}{application réciproque} de \( f\) est la fonction
    \begin{equation}
        \begin{aligned}
            f^{-1}\colon B&\to A \\
            y&\mapsto \text{le } x\in A\text{ tel que } f(x)=y.
        \end{aligned}
    \end{equation}
\end{definition}

Plus généralement si \( f\colon X\to Y\) est une application quelconque et si \( S\subset Y\) nous notons
\begin{equation}
    f^{-1}(S)=\{ x\in X\tq f(x)\in S \},
\end{equation}
et dans le cas où \( S\) est réduit à un unique élément \( y\), nous notons \( f^{-1}(y)\) au lieu de \( f^{-1}\big( \{ y \} \big)\). Si de plus \( f^{-1}(S)\) est un singleton \( x\), nous noterons \( f^{-1}(S)=x\) et non \( f^{-1}(S)=\{ x \}\).

Les plus acharnés parmi les lecteurs se rendront compte de la différence ontologique fondamentale entre \( x\) et \( \{ x \}\).

\begin{proposition}	\label{PropoInvCompCont}
Soit $f\colon A\subset\eR^n\to B\subset\eR^m$ une bijection continue. Si $A$ est compact, alors $f^{-1}\colon B\to A$ est continue.
\end{proposition}
\index{réciproque!continuité}

\begin{proposition}		\label{PropIntContMOnIvCont}
Soient $I$ un intervalle dans $\eR$ et $f\colon I\to \eR$ une fonction continue strictement monotone. Alors la fonction réciproque $f^{-1}\colon f(I)\to \eR$ est continue sur l'intervalle $f(I)$.
\end{proposition}
\index{réciproque!continuité}

%+++++++++++++++++++++++++++++++++++++++++++++++++++++++++++++++++++++++++++++++++++++++++++++++++++++++++++++++++++++++++++
\section{Topologie des semi-normes}
%+++++++++++++++++++++++++++++++++++++++++++++++++++++++++++++++++++++++++++++++++++++++++++++++++++++++++++++++++++++++++++

Les principaux espaces topologiques construit avec des semi-normes seront les espaces de fonctions de la définition~\ref{DefFGGCooTYgmYf}. Nous verrons également la topologie \( *\)-faible sur \( \swD'(\Omega)\) en la définition~\ref{DefASmjVaT}.

\begin{definition}  \label{DefPNXlwmi}
    Si \( E\) est un espace vectoriel, une \defe{semi-norme}{semi-norme} sur \( E\) est une application \( p\colon E\to \eR\) telle que
    \begin{enumerate}
        \item
            \( p(x)\geq 0\),
        \item   \label{ItemSHnimhDii}
            \( p(\lambda x)=| \lambda |p(x)\)
        \item   \label{ItemSHnimhDiii}
            \( p(x+y)\leq p(x)+p(y)\).
    \end{enumerate}
\end{definition}

\begin{lemma}[\cite{DRcmzcB}]
    Si \( p\) est une semi-norme nous avons
    \begin{equation}
        | p(x)-p(y) |\leq p(x-y).
    \end{equation}
\end{lemma}

\begin{proof}
    Nous avons d'une part \( p(x+h)\leq p(x)+p(h)\) et d'autre part \( p(x)\leq p(x+h)+p(-h)=p(x+h)+p(h)\). En isolant \( p(x+h)-p(x)\) dans chacune ce des deux inégalités,
    \begin{equation}
        -p(h)\leq p(x+h)-p(x)\leq p(h)
    \end{equation}
    ou encore
    \begin{equation}
        |p(x+h)-p(x)|\leq p(h)
    \end{equation}
    qui donne le résultat demandé en posant \( h=y-x\).
\end{proof}

Soit \( (p_i)_{i\in I}\) une famille de semi-normes sur \( E\). Nous construisons alors une topologie sur \( E\) de la façon suivante.

\begin{definition}[Topologie et semi-normes\cite{SOdaAdx,MUbDonp}]
    Pour tout \( J\) fini dans \( I\) nous définissons les \defe{boules ouvertes}{boule!avec semi-normes}
    \begin{equation}
        B_J(x,r)=\{ y\in E\tq p_j(y-x)<r\,\forall j\in J \}.
    \end{equation}
    La \defe{topologie}{topologie!et semi-normes} sur \( E\) donnée par la famille de semi-norme est définie en disant que \( \mO\subset E\) est ouvert si et seulement si chaque point de \( \mO\) est dans une boule contenue dans \( \mO\).
\end{definition}

\begin{proposition} \label{PropQPzGKVk}
    Une suite \( (x_n)\) dans \( E\) converge vers \( x\) au sens de la topologie des semi-normes si et seulement si pour tout \( i\in I\),
    \begin{equation}
        p_i(x-x_n)\to 0.
    \end{equation}
\end{proposition}

\begin{proof}
    Si la suite \( (x_n)\) converge\footnote{Définition~\ref{DefXSnbhZX}.} vers \( x\), alors pour tout ouvert \( \mO\) autour de \( x\), il existe un \( N\) tel que si \( n\geq N\), alors \( x_n\in\mO\). En particulier pour tout \( j\) et pour tout \( \epsilon>0\), il doit exister un \( n\geq N_j\) tel que \( x_n\in B_j(x,\epsilon)\).

    Voyons l'implication inverse. Soit \( \epsilon>0\). Pour tout \( i\in I\), il existe un \( N_i\) tel que \( n\geq N_i\) implique \( p_i(x-x_n)\leq \epsilon\). Si \( \mO\) est un ouvert, il doit contenir une boule du type \( B_J(x,r)\) pour un certain ensemble fini \( J\subset I\).

    En prenant \( N=\max\{ N_j\tq j\in J \}\), nous avons \( p_j(x-x_n)\leq \epsilon\) pour tout \( j\) et donc \( x_n\in B_J(x,r)\).
\end{proof}

La proposition suivante est un vulgaire plagiat de la proposition~\ref{PropQZRNpMn}.
\begin{proposition} \label{PropNGjQnqF}
    Soit \( f\colon \eR\to (E,p_i)_{i\in I}\) une application. Nous avons équivalence entre
    \begin{enumerate}
        \item   \label{ItemHNxGMpCi}
            la fonction \( f\) est continue en \( t_0\in \eR\),
        \item\label{ItemHNxGMpCii}
            si \( W\) est un voisinage ouvert de \( f(t_0)\) il existe un voisinage ouvert \( V\) de \( t_0\) (dans \( \eR\)) tel que \( f(V)\subset W\),
        \item\label{ItemHNxGMpCiii}
            pour tout \( i\in I\) et \( \epsilon>0\) il existe \( \delta>0\) tel que
            \begin{equation}
                f\big( B(t_0,\delta) \big)\subset B_i\big( f(t_0),\epsilon \big).
            \end{equation}
    \end{enumerate}
\end{proposition}

\begin{proof}
    L'équivalence~\ref{ItemHNxGMpCi} \( \Leftrightarrow\)~\ref{ItemHNxGMpCii} est la définition~\ref{DefOLNtrxB}.

    Prouvons~\ref{ItemHNxGMpCii} \( \Rightarrow\)~\ref{ItemHNxGMpCiii}. Soient \( i\in I\) et \( \epsilon>0\). Considérons la boule \( B_i\big( f(t_0),\epsilon \big)\), qui est un ouvert de \( E\) contenant \( f(t_0)\). Il existe donc un ouvert \( V\) autour de \( t_0\) tel que \( f(V)\subset B_i\big( f(t_0),\epsilon \big)\). En particulier \( V\) contient une boule \( B(t_0,\delta)\) et nous avons
    \begin{equation}
        f\big( B(t_0,\delta) \big)\subset f(V)\subset B_i\big( f(t_0),\epsilon \big).
    \end{equation}

    Prouvons~\ref{ItemHNxGMpCiii} \( \Rightarrow\)~\ref{ItemHNxGMpCii}. Soit \( W\) un ouvert autour de \( f(t_0)\). Il existe un \( i\in I\) et \( \epsilon>0\) tel que \( B_i\big( f(t_0),\epsilon \big)\subset W\). Nous avons alors un \( \delta>0\) tel que
    \begin{equation}
        f\big( B(t_0,\delta) \big)\subset B_i\big( f(t_0),\epsilon \big)\subset W.
    \end{equation}
\end{proof}

Lorsqu'on a un espace \( E\) muni d'une quantité dénombrable de semi-normes \( \{ p_k \}_{k\in I}\) nous définissons l'écart\footnote{Dans le cas de \( E=\swD(K)\), la première semi-norme est numérotée à zéro, donc il faudra poser \( d(\varphi_1,\varphi_2)\) avec \( p_{k-1}\) au lieu de \( p_k\).}
\begin{equation}        \label{EqAAghiUR}
    d(x,y)=\sup_{k\geq 1}\min\big\{  \frac{1}{ k },p_k(x-y) \big\}.
\end{equation}
Notons que cette écart est invariant par translation au sens où pour tout \( x,y,h\) dans \( E\) nous avons
\begin{equation}
    d(x+h,y+h)=\sup_{k\geq 1}\min\big\{ \frac{1}{ k },p_k(x-y) \big\}=d(x,y).
\end{equation}

\begin{proposition}[\cite{DRcmzcB}] \label{PropLOwUvCO}
    La topologie donnée par les boules
    \begin{equation}    \label{EqGHfYIlQ}
        B_k(a,r)=\{ x\in E\tq \forall\,k\leq \frac{1}{ r },p_k(x-a)<r\}
    \end{equation}
    est la même que celle «usuelle» donnée par les semi-normes. En disant «la même» nous entendons le fait que les ouverts sont les mêmes : \( A\) est ouvert pour une des deux topologies si et seulement s'il est ouvert pour l'autre.
\end{proposition}

\begin{proof}
    Pour cette démonstration nous allons préfixer par \( d\) les notions topologiques issues des boules \eqref{EqGHfYIlQ} et par \( P\) celle des semi-normes : \( P\)-continue, \( d\)-ouvert, etc.

    D'abord nous avons
    \begin{equation}    \label{EqRIURpQo}
        B(a,r)=\bigcap_{k\leq \frac{1}{ k }}B_k(a,r).
    \end{equation}
    Si \( \mO\) est un \( d\)-ouvert, il contient une \( d\)-boule autour de chacun de ses points. Or d'après la formule \eqref{EqRIURpQo}, une \( d\)-boule est une intersection \emph{finie} de \( P\)-ouverts et donc est un \( P\)-ouvert par définition. Donc \( \mO\) contient un \( P\)-ouvert autour de tous ses points et est donc \( P\)-ouvert.

    Inversement nous supposons que \( \mO\) est un \( P\)-ouvert. Commençons par prouver que les semi-normes \( p_k\) sont \( d\)-continues. En effet soient \( k\in \eN\), \( \epsilon\leq \frac{1}{ k }\) et \( x,y\in E\) tels que \( d(x,y)\leq \epsilon\); nous avons
    \begin{subequations}
        \begin{align}
            | p_k(y)-p_k(x) |&\leq p_k(x-y)\\
            &=\min\{ \frac{1}{ k },p_k(x-y) \}\\
            &\leq d(x,y)\\
            &\leq \epsilon.
        \end{align}
    \end{subequations}
    Montrons à présent que \( \mO\) est \( d\)-ouverte. Si \( a\in\mO\), il existe \( k\) et \( r\) tels que \( B_k(a,r)\subset\mO\). Soit \( x\in B_k(a,r)\). Montrons que si \( \epsilon\) est suffisamment petit, la \( d\)-boule \( B(x,\epsilon)\) est inclue à \( B_k(a,r)\). Pour cela prenons \( y\in B(x,\epsilon)\); nous avons
    \begin{equation}
        \big| p_k(a-x)-p_k(a-y) \big|\leq d(x,y)\leq \epsilon.
    \end{equation}
    Par conséquent le nombre \( p_k(a-y)\) est dans l'intervalle
    \begin{equation}
        p_k(a-x)\pm\epsilon
    \end{equation}
    et il suffit de prendre \( \epsilon<\frac{ r-p_k(a-x) }{2}\).
\end{proof}

%---------------------------------------------------------------------------------------------------------------------------
\subsection{Espace dual}
%---------------------------------------------------------------------------------------------------------------------------

Nous parlerons plus en détail d'espace dual d'un espace normé en la section~\ref{SECooKOJNooQVawFY}.

\begin{definition}  \label{DefHUelCDD}
    Soient \( F\) un espace métrique et \( E\) un espace topologique vectoriel. Une topologie possible\footnote{C'est, dans l'idée, celle qui sera choisie pour les espaces de distributions, voir la définition~\ref{DefASmjVaT}.} sur l'espace des applications linéaires \( \aL(E,F)\) est la \defe{topologie \( *\)-faible}{topologie!$*$-faible} qui est la topologie des semi-normes
    \begin{equation}
        p_v(T)=\| T(v) \|_F.
    \end{equation}
\end{definition}
C'est une famille de semi-normes indicées par les éléments de \( E\). Si \( E\) est un espace métrique, c'est cette topologie qui sera considérée sur son dual topologique\index{topologie!sur dual topologique} \( E'\) des applications continues \( E\to \eR\).

La proposition suivante indique qu'elle est un peu la topologie de la convergence ponctuelle.
\begin{proposition}
    Soient \( E\) un espace muni de la topologie des semi-normes \( \{ p_i \}_{i\in I}\) et \( F\) un espace métrique. Soient une suite \( (T_n)\) dans \( \aL(E,F)\) et \( T\in \aL(E,F)\). Nous avons \( T_n\stackrel{*}{\longrightarrow}T\) si et seulement si \( T_n(v)\stackrel{F}{\longrightarrow}T(v)\) pour tout \( v\in E\).
\end{proposition}

\begin{proof}
    Nous avons équivalence entre les lignes suivantes :
    \begin{subequations}
        \begin{align}
            T_n\stackrel{*}{\longrightarrow}T\\
            p_v(T_n-T)\to 0\,\forall v\in E &&\text{proposition~\ref{PropQPzGKVk}}\\
            \| T_n(v)-T(v) \|_E\to 0\,\forall v\in E\\
            T_n(v)\stackrel{E}{\longrightarrow}T(v).
        \end{align}
    \end{subequations}
\end{proof}

%---------------------------------------------------------------------------------------------------------------------------
\subsection{Espace \texorpdfstring{$ C^k(\eR,E')$}{C(R,E')}}
%---------------------------------------------------------------------------------------------------------------------------

Nous revenons à nos histoires de limites de la définition~\ref{DefXSnbhZX}.
\begin{proposition}[Unicité de la limite dans un dual topologique] \label{PropRBCiHbz}
    Soient \( E\) un espace métrique et \( E'\) son dual topologique muni de sa topologie de la définition~\ref{DefHUelCDD}. Il y a unicité de l'élément de \( E'\) vers lequel une fonction \( u\colon \eR\to E' \) peut converger.
\end{proposition}

\begin{proof}
    Soit \( T\) un élément vers lequel \( u_t\) converge lorsque \( t\to t_0\). Soient \( \epsilon>0\) et \( x\in E\). La boule \( B_x(T,\epsilon)\) de \( E'\) subordonnée à la norme \( p_x\) et centrée en \( T\) est un ouvert de \( E'\). Étant donné que \( u\) converge vers \( T\) il existe \( \delta>0\) tel que \( u_t\in B_x(T,\epsilon)\) dès que \( | t-t_0 |\leq \delta\). Nous avons donc, pour tout \( x\in E\), la limite (dans \( \eR\)) :
    \begin{equation}
        \lim_{t\to t_0} u_t(x)=T(x).
    \end{equation}
    Cela prouve que la convergence de \( u\) vers \( T\) implique l'existence pour tout \( x\) de la limite de \( u_t(x)\) dans \( \eR\). Si \( T'\) est un autre élément vers lequel \( u_t\) converge, nous avons par le même raisonnement que
    \begin{equation}
        \lim_{t\to t_0} u_t(x)=T'(x).
    \end{equation}
    Par unicité de la limite dans \( \eR\)
    %TODO : prouver ça et mettre une référence.
    nous devons alors avoir \( T(x)=T'(x)\) pour tout \( x\), c'est à dire \( T=T'\).
\end{proof}

\begin{proposition} \label{PropVKSNflB}
    Soit \( u\colon \eR\to E'\) une fonction continue. Alors
    \begin{enumerate}
        \item   \label{ItemLSJjfZdi}
            pour tout \( x\in E\) la fonction \( t\mapsto u_t(x)\) est continue,
        \item\label{ItemLSJjfZdii}
            pour tout \( x\in E\) nous avons la limite dans \( \eR\)
            \begin{equation}    \label{EqWKdFPVO}
                \lim_{t\to t_0} u_t(x)=u_{t_0}(x),
            \end{equation}
        \item\label{ItemLSJjfZdiii}
            nous avons la limite dans \( E'\)
            \begin{equation}
                \lim_{t\to t_0} u_t=u_{t_0}.
            \end{equation}
    \end{enumerate}
\end{proposition}

\begin{proof}
    Soient \( x\in E\) et \( \epsilon> 0\). Par la proposition~\ref{PropNGjQnqF} la continuité de \( u\) donne un \( \delta>0\) tel que
    \begin{equation}
        u_{B(t_0,\delta)}\subset B_x(u_{t_0},\epsilon).
    \end{equation}
    C'est à dire que si \( | t-t_0 |\leq \delta\) nous avons
    \begin{equation}
        \big| u_{t_0}(x)-u_t(x) \big|<\epsilon,
    \end{equation}
    ce qui signifie bien que la fonction \( t\mapsto u_t(x)\) est continue en tant que fonction \( \eR\to \eR\). Cela est le point~\ref{ItemLSJjfZdi}. Le théorème de limite et continuité dans \( \eR\) nous donne immédiatement la limite \eqref{EqWKdFPVO}.
    % TODO : aussi fou que cela puisse paraître, le théorème limite et continuité dans R n'est pas prouvé ni énoncé. Le faire et lié vers ici.

    Nous passons à la preuve du point~\ref{ItemLSJjfZdiii}. Soit \( \mO\) un ouvert de \( E'\) contenant \( u_{t_0}\). Il existe donc un \( i\in I\) et \( \epsilon>0\) tel que \( B_i(u_{t_0},\epsilon)\subset \mO\). Étant donné que \( u\) est continue, il existe \( \delta>0\) tel que
    \begin{equation}
        u_{B(t_0,\delta)}\subset B_i(u_{t_0},\epsilon)\subset \mO.
    \end{equation}
    Cela signifie bien que
    \begin{equation}
        | t-t_0 |\leq \delta\Rightarrow u_t\in\mO,
    \end{equation}
    c'est à dire que nous avons la limite \( \lim_{t\to t_0} u_t=u_{t_0}\) dans \( E'\). Pour dire cela nous avons utilisé la définition~\ref{DefYNVoWBx} de la limite et le résultat d'unicité~\ref{PropRBCiHbz}.
\end{proof}

\begin{definition}  \label{DefDZsypWu}
    Si nous avons une application \( u\colon \eR\to E'\) nous considérons sa \defe{dérivée}{dérivée!fonction à valeurs dans $E'$} donnée par la limite
    \begin{equation}
        u'_{t_0}=\lim_{t\to t_0} \frac{ u_t-u_{t_0} }{ t-t_0 }.
    \end{equation}
    Cela est un nouvel élément de \( E'\) (pour peu que la limite existe). La fonction \( u'\colon \eR\to E'\) ainsi définie peut être continue ou non. Cela nous permet de définir les espaces \( C^k(\eR,E')\) et \( C^{\infty}(\eR,E')\).
\end{definition}
Une des principales utilisations que nous ferons de ces espaces seront les espaces de fonctions à valeurs dans les distributions tempérées dont nous parlerons dans la section~\ref{SecTEgDVWO}.

%+++++++++++++++++++++++++++++++++++++++++++++++++++++++++++++++++++++++++++++++++++++++++++++++++++++++++++++++++++++++++++
\section{Espaces de Baire}
%+++++++++++++++++++++++++++++++++++++++++++++++++++++++++++++++++++++++++++++++++++++++++++++++++++++++++++++++++++++++++++
\label{SecBDlaUrz}

\begin{definition}
    Un \defe{espace de Baire}{espace!de Baire}\index{Baire!espace} est un espace topologique dans lequel toute intersection dénombrable d'ouverts denses est dense.
\end{definition}

\begin{theorem}[Théorème de Baire\cite{SIdTHwW}]    \label{ThoBBIljNM}
    Les espaces suivants sont de Baire :
    \begin{enumerate}
        \item
            les espaces topologiques localement compacts,
        \item
            les espaces métriques complets (donc ceux de Banach en particulier),
        \item
            tout ouvert d'un espace de Baire.
    \end{enumerate}
\end{theorem}
\index{théorème!de Baire}
\index{Baire!théorème}
%TODO : une preuve c'est sans doute bien, et ça a l'air d'être pas trop dur et donné sur Wikipédia.

\begin{proof}
    \begin{subproof}
    \item[Espaces topologiques localement compacts]
        \item[Espaces métriques complets]
            Soit \( (E,d)\) un espace métrique complet. Soient \( V\) un ouvert quelconque de \( E\) et \( U_n\) une suite d'ouverts denses. Le but est de prouver que l'ensemble \( \bigcap_{n\in \eN}U_n\) intersecte \( V\). Vu que \( V\) est ouvert dans un espace métrique, il contient une boule ouverte et donc une boule fermée \( B_0\) de rayon strictement positif. L'ensemble \( U_1\) est dense et intersecte donc un ouvert contenu dans \( B_0\). L'intersection est un ouvert qui contient alors une boule fermée \( B_1\) de rayon strictement positif. Continuant ainsi nous construisons une suite de fermés emboités \( B_n\) telle que
            \begin{equation}
                \bigcap_{n\in \eN}U_n\cap V
            \end{equation}
            contient l'intersection des \( B_n\). Par le théorème~\ref{ThoCQAcZxX} des fermés emboités (que nous utilisons parce que \( E\) est métrique et complet), cette intersection est non vide.
        \item[Ouvert d'un espace de Baire]
    \end{subproof}
\end{proof}

Une des applications du théorème de Baire est le théorème de Banach-Steinhaus~\ref{ThoPFBMHBN}.
