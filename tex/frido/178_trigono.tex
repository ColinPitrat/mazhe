% This is part of (everything) I know in mathematics
% Copyright (c) 2011-2013,2016-2017
%   Laurent Claessens
% See the file fdl-1.3.txt for copying conditions.

%+++++++++++++++++++++++++++++++++++++++++++++++++++++++++++++++++++++++++++++++++++++++++++++++++++++++++++++++++++++++++++ 
\section{Trigonométrie}
%+++++++++++++++++++++++++++++++++++++++++++++++++++++++++++++++++++++++++++++++++++++++++++++++++++++++++++++++++++++++++++

%--------------------------------------------------------------------------------------------------------------------------- 
\subsection{Définitions, périodicité et quelque valeurs remarquables}
%---------------------------------------------------------------------------------------------------------------------------

\begin{propositionDef}[Défintion du cosinus et du sinus]        \label{PROPooZXPVooBjONka}
    La série
    \begin{equation}
        \cos(x)=\sum_{n=0}^{\infty}\frac{ (-1)^n }{ (2n)! }x^{2n}
    \end{equation}
    définit une fonction \( \cos\colon \eR\to \eR\) de classe \(  C^{\infty}\). Nous l'appelons \defe{cosinus}{cosinus}

    La série
    \begin{equation}        \label{EQooCMRFooCTtpge}
        \sin(x)=\sum_{n=0}^{\infty}\frac{ (-1)^n }{ (2n+1)! }x^{2n+1}
    \end{equation}
    définit une fonction \( \sin\colon \eR\to \eR\) de classe \(  C^{\infty}\). Nous l'appelons \defe{sinus}{sinus}
\end{propositionDef}

\begin{proof}
    La série entière définissant \( \cos(x)\) a pour coefficients
    \begin{equation}
        a_n=\begin{cases}
            0    &   \text{si } n\text{ est impair}\\
            \frac{ (-1)^{n/2} }{ n! }    &   \text{si } n\text{ est pair}.
        \end{cases}
    \end{equation}
    Nous pouvons la majorer par la série entière donnée par les coefficients
    \begin{equation}
        b_n=\begin{cases}
            1/n!    &   \text{si } n\text{ est impair}\\
            \frac{ (-1)^{n/2} }{ n! }    &   \text{si } n\text{ est pair}.
        \end{cases}
    \end{equation}
    Quelle que soit la parité de \( k\) nous avons toujours
    \begin{equation}
        | \frac{ b_{k+1} }{ b_k } |=\frac{1}{ k+1 },
    \end{equation}
    de telle sorte que la formule d'Hadamard \eqref{EqAlphaSerPuissAtern} nous donne \( R=\infty\) pour la série \( \sum_{k=0}^{\infty}b_kx^k\). A fortiori\footnote{Remarque \ref{REMooYOTEooKvxHSf}.} le rayon de convergence pour la série du cosinus est infini.

    L'assertion concernant le sinus se démontre de même.

    En ce qui concerne le fait que les fonctions \( \sin\) et \( \cos\) sont de classe \(  C^{\infty}\) sur \( \eR\), il faut invoquer le corollaire \ref{CorCBYHooQhgara}.
\end{proof}

\begin{lemma}
    En ce qui concerne la dérivation, nous avons
    \begin{subequations}
        \begin{align}
            \sin'&=\cos\\
            \cos'&=-\sin.
        \end{align}
    \end{subequations}
\end{lemma}

\begin{proof}
    Il s'agit de se permettre de dériver terme à terme (proposition \ref{ProptzOIuG}) les séries qui définissent le sinus et le cosinus.
\end{proof}

\begin{lemma}       \label{LEMooAEFPooGSgOkF}
    Les fonctions sinus et cosinus vérifient
    \begin{equation}
        \cos^2(x)+\sin^2(x)=1
    \end{equation}
    pour tout \( x\in \eR\).
\end{lemma}

\begin{proof}
    Posons \( f(x)=\sin^2(x)+\cos^2(x)\) et dérivons :
    \begin{equation}
        f'(x)=2\sin(x)\cos(x)+2\cos(x)(-)\sin(x)=0.
    \end{equation}
    La fonction \( f\) est donc constante par le corollaire \ref{CORooEOERooYprteX}. Nous avons donc pour tout \( x\) :
    \begin{equation}
        f(x)=f(0)=\sin^2(0)+\cos^2(0)=1.
    \end{equation}
    Le dernier calcul s'obtient en substituant directement \( x\) par zéro dans les séries : \( \sin(0)=0\) et \( \cos(0)=1\).
\end{proof}

\begin{lemma}       \label{LEMooHOYZooKQTsXW}
    Nous avons la formule
    \begin{equation}        \label{EQooRVPJooTMwNTU}
        e^{ix}=\cos(x)+i\sin(x)
    \end{equation}
    pour tout \( x\in \eR\).
\end{lemma}

\begin{proof}
    Il faut partir de la définition de l'exponentielle \eqref{EqEIGZooKWSvPS}, et remarquer que \( i^k\) vaut \( 1\), \( i\), \( -1\), \( -i\). Donc un terme sur deux est imaginaire pur et parmi ceux-là, un sur deux est positif. À bien y regarder, les termes imaginaires purs forment la série du sinus et ceux réels la série du cosinus.
\end{proof}

\begin{lemma}
    Nous avons les formules d'addition d'angles
    \begin{subequations}        \label{SUBEQSooFSSMooHcYwRc}
        \begin{align}
            \cos(a+b)=\cos(a)\cos(b)-\sin(a)\sin(b)\\
            \sin(a+b)=\cos(a)\sin(b)+\sin(a)\cos(b).
        \end{align}
    \end{subequations}
    pour tout \( a\), \( b\) réels.
\end{lemma}

\begin{proof}
    Nous utilisons la formule d'addition dans l'exponentielle, proposition \eqref{EQooVFXUooBfwjJY} et la formule \eqref{EQooRVPJooTMwNTU} avant de séparer les parties réelles et imaginaires :
    \begin{equation}
        e^{i(a+b)}= e^{ia} e^{ib}=\cos(a)\cos(b)-\sin(a)\sin(b)+i\big( \cos(a)\sin(b)+\sin(a)\cos(b) \big).
    \end{equation}
    Cela est également égal à
    \begin{equation}
        \cos(a+b)+i\sin(a+b).
    \end{equation}
    En identifiant les parties réelle et imaginaires, nous obtenons les formules annoncées.
\end{proof}

\begin{lemma}       \label{LEMooPQWWooMdPWUT}
    Un sous-groupe de \( (\eR,+)\) est soit dense dans \( \eR\) soit de la forme \( p\eZ\) pour un certain réel \( p\neq 0\).
\end{lemma}

\begin{proof}
    Soit \( A\), un sous groupe de \( (\eR,+)\) qui ne soit pas dense. Soit un intervalle \( \mathopen] a , b \mathclose[\) qui n'intersecte pas \( A\) (si vous voulez frimer, vous noterez ici que nous utilisons le fait que les intervalles ouverts forment une base de la topologie de \( \eR\)). Si \( d=| b-a |\), l'ensemble \( A\) ne contient pas deux éléments séparés par strictement moins de \( d\). Soit \( p\), le plus petit élément strictement positif de \( A\); nous avons \( p\geq d\) (parce que \( 0\in A\) de toutes façons).

        Vu que \( A\) est un groupe nous avons \( p\eZ\subset A\).

        Pour l'inclusion inverse, si \( x\in A\) est hors de \( p\eZ\), il existe un \( y\in p\eZ\) avec \( | x-y |<p\). Et donc le nombre \( | x-y |\) est dans \( A\) tout en étant plus petit que \( p\). Contradiction.
\end{proof}

\begin{proposition}[\cite{ooUMDHooHrJpfV}]      \label{PROPooFRVCooKSgYUM}
    La fonction \( \cos\) est périodique et le nombre \( T>0\) est une période si et seulement si \( \cos(T)=1\) et \( \sin(T)=0\).
\end{proposition}

\begin{proof}
    Plusieurs étapes.
    \begin{subproof}
        \item[La fonction cosinus n'est pas toujours positive]
    Supposons d'abord que \( \cos(x)>0\) pour tout \( x\in \eR\). Dans ce cas, la fonction \( \sin\) est strictement croissante. Mais les deux fonctions sont bornées par \( 1\) du fait de la formule \( \cos^2(x)+\sin^2(x)=1\). La fonction \( \sin\) étant croissante et bornée, elle est convergente vers un réel par la proposition \ref{PropMTmBYeU} :
    \begin{equation}
        \lim_{x\to \infty} \sin(x)=\ell
    \end{equation}
    pour un certain \( \ell>0\). Avec ça nous avons aussi (pour cause de dérivée) \( \lim_{x\to \infty} \sin'(x)=0\), c'est à dire \( \lim_{x\to \infty} \cos(x)=0\). Mais vu que \( \cos^2(x)+\sin^2(x)=1\) nous avons \( \lim_{x\to \infty} \sin(x)=1\). Mézalor \( \lim_{x\to \infty} \cos'(x)=-1\), ce qui donne que la fonction \( \cos\) n'est pas bornée. Cela est impossible. Nous en déduisons que \( \cos(x)\) n'est pas toujours positive.

\item[Il existe \( T>0\) tel que \( \cos(T)=1\) et \( \sin(T)=0\)]

    Par ce que nous venons de faire, il existe \( r>0\) tel que \( \cos(r)=0\). Pour cette valeur, nous avons aussi obligatoirement \( \sin(r)=\pm 1\). Nous avons aussi, en utilisant les formules \eqref{SUBEQSooFSSMooHcYwRc},
    \begin{subequations}
        \begin{align}
            \cos(2r)=\cos^2(r)-\sin^2(r)=-1\\
            \sin(2r)=2\cos(r)\sin(r)=0.
        \end{align}
    \end{subequations}
    et par conséquent
    \begin{subequations}
        \begin{align}
            \cos(4r)=\cos^2(2r)-\sin^2(2r)=1\\
            \sin(4r)=2\cos(2r)\sin(2r)=0.
        \end{align}
    \end{subequations}
    Donc \( T=4r\) fonctionne.

\item[Si \( T\) est une période]
    Nous entrons dans le vif de la preuve. Soit un \( T>0\) tel que \( \cos(x+T)=\cos(x)\) pour tout \( x\in \eR\). Avec la formule d'addition d'angle\footnote{Rien ne nous empêche de donner ce nom à ces formules, mais seriez vous capable de définir précisément le mot «angle» ?} dans le cosinus nous cherchons un \( T\) tel que
    \begin{equation}
        \cos(x+T)=\cos(x)\cos(T)-\sin(x)\sin(T)=\cos(x)
    \end{equation}
    et donc tel que
    \begin{equation}        \label{EQooELSAooLNtBnm}
        \cos(x)\big( \cos(T)-1 \big)=\sin(x)\sin(T).
    \end{equation}
    Nous dérivons cette équation :
    \begin{equation}        \label{EQooCECFooLpxXaw}
        -\sin(x)\big( \cos(T)-1 \big)=\cos(x)\sin(T).
    \end{equation}
    Nous multiplions chacune des deux équations \eqref{EQooELSAooLNtBnm} et \eqref{EQooCECFooLpxXaw} par \( \sin(x)\) et \( \cos(x)\) pour obtenir les quatre relations suivantes :
    \begin{subequations}
        \begin{align}
            \cos^2(x)\big( \cos(T)-1 \big)-\sin(x)\cos(x)\sin(T)=0   \label{SUBEQooLGQXooIrLMLW}\\
            -\sin(x)\cos(x)\big( \cos(T)-1 \big)-\cos^2(x)\sin(T)=0     \label{SUBEQooCHTDooKwvyZF}\\
            \sin(x)\cos(x)\big( \cos(T)-1 \big)-\sin^2(x)\sin(T)=0 \label{SUBEQooEWPTooTLCUMf}\\
            \sin^2(x)\big( \cos(T)-1 \big)-\sin(x)\cos(x)\sin(T)=0  \label{SUBEQooGBXTooCFekGJ}
        \end{align}
    \end{subequations}
    En faisant \eqref{SUBEQooLGQXooIrLMLW} moins \eqref{SUBEQooGBXTooCFekGJ} nous trouvons \( \cos(T)=1\). Et en sommant \eqref{SUBEQooCHTDooKwvyZF} avec \eqref{SUBEQooEWPTooTLCUMf} nous avons \( -\sin(T)=0\).

\item[Si \( T>0\) est tel que \( \sin(T)=0\) et \( \cos(T)=1\)]

    Alors la formule d'addition d'angle donne tout de suite
    \begin{equation}
        \cos(x+T)=\cos(x).
    \end{equation}

    \end{subproof}

    À de niveau nous croyons avoir prouvé que \( \cos\) était périodique et que la période est donnée par
    \begin{equation}
        \min\{ T>0\tq \sin(T)=0,\cos(T)=1 \}.
    \end{equation}
    Or rien n'est moins sûr parce qu'il pourrait arriver que ce minimum n'existe pas, c'est à dire que l'infimum soit zéro. Autrement dit, il peut arriver que l'ensemble des périodes soit dense. Plus précisément, soit \( P\subset \eR\) l'ensemble des périodes de \( \cos\). C'est un sous-groupe de \( (\eR,+)\) et le lemme \ref{LEMooPQWWooMdPWUT} nous dit que \( P\) est soit dense dans \( \eR\) soit de la forme \( p\eZ\) pour un \( p>0\).

    Si \( P\) est dense, soit \( t\in \eR\) et une suite \( (t_n)\) dans \( P\) telle que \( t_n\to t\). Pour tout \( x\) et tout \( n\) nous avons
    \begin{equation}
        \cos(x+t_n)=\cos(x),
    \end{equation}
    Vu que la fonction cosinus est continue, nous pouvons passer à la limite et écrire \( \cos(x+t)=\cos(x)\). Cela étant valable pour tout \( x\) et pour tout \( t\), la fonction cosinus est constante. Or nous savons que ce n'est pas le cas, donc \( P\) n'est pas dense. Donc cosinus est périodique.
\end{proof}

\begin{definition}[Le nombre \( \pi\)]
    Le nombre \( \pi>0\) est donné par
    \begin{equation}
        2\pi=\min\{ T>0\tq \cos(x+T)=\cos(x)\,\forall x \}.
    \end{equation}
\end{definition}
Par ce qui a été dit dans la démonstration nous avons aussi
\begin{equation}
    2\pi=\min\{ T>0\tq \sin(T)=0,\cos(T)=1 \}.
\end{equation}
Notons que tout ceci ne nous donne pas la plus petite indication d'ordre de grandeur de la valeur de \( \pi\). Cela peut encore être \( 0.1\) autant que \( 500\).

\begin{proposition}[\cite{ooUMDHooHrJpfV}]      \label{PROPooMWMDooJYIlis}
    Des propriétés à la chaîne à propos des sinus, cosinus et de leurs périodes.
    \begin{enumerate}
        \item
            Le nombre \( 2\pi\) est le plus petit tel que
            \begin{subequations}
                \begin{numcases}{}
                    \cos(2\pi)=1\\
                    \sin(2\pi)=0.
                \end{numcases}
            \end{subequations}
        \item
            Le nombre \( 2\pi\) est également la période de la fonction \( \sin\).
        \item
            Nous avons \( \cos(\pi)=- 1\) et \( \sin(\pi)=0\).
        \item
            Pour tout \( a\in \eR\) nous avons
            \begin{subequations}
                \begin{align}
                    \cos(a+\pi)&=-\cos(\pi)\\
                    \sin(a+\pi)&=-\sin(\pi).
                \end{align}
            \end{subequations}
        \item
            Le nombre \( \pi\) est le plus petit \( T>0\) tel que \( \cos(T)=-1\) et \( \sin(T)=0\).
        \item
            Nous avons
            \begin{subequations}
                \begin{numcases}{}
                    \cos(\pi/2)=0\\
                    \sin(\pi/2)=1.
                \end{numcases}
            \end{subequations}
        \item
            Nous avons les formules
            \begin{subequations}        \label{EQSooRJZGooCFVqbZ}
                \begin{numcases}{}
                    \cos(x+\pi/2)=-\sin(x)\\
                    \sin(x+\pi/2)=\cos(x)
                \end{numcases}
            \end{subequations}
            pour tout \( x\in \eR\).
        \item
            Le nombre \( \pi/2\) est le plus petit \( T\) vérifiant \( \sin(T)=1\), \( \cos(T)=0\).
        \item
            Nous avons les valeurs
            \begin{subequations}
                \begin{numcases}{}
                    \cos(\frac{ 3\pi }{ 2 })=0\\
                    \sin(\frac{ 3\pi }{ 2 })=-1.
                \end{numcases}
            \end{subequations}
        \item       \label{ITEMooQKPKooEPeHER}
            Le nombre \( \pi/2\) est le plus petit \( T>0\) tel que \( \cos(T)=0\).
    \end{enumerate}
\end{proposition}

\begin{proof}
    C'est parti.
    \begin{enumerate}
        \item
            Le fond de la proposition \ref{PROPooFRVCooKSgYUM} est que toutes les périodes \( T>0\) vérifient \( \cos(T)=1\) et \( \sin(T)=0\). La définition de \( \pi\) est que c'est la plus petite période.
        \item
            En utilisant le fait que l'une est la dérivée de l'autre, si \( T\) est une période de \( \cos\) nous avons
            \begin{subequations}
                \begin{align}
                    \sin(x+T)&=-\cos'(x+T)\\
                    &=-\lim_{\epsilon\to 0}\frac{ \cos(x+T+\epsilon)-\cos(x+T) }{\epsilon  }\\
                    &=-\lim_{\epsilon\to 0}\frac{ \cos(x+\epsilon)-\cos(x) }{ \epsilon }\\
                    &=-\cos'(x)\\
                    &=\sin(x).
                \end{align}
            \end{subequations}
            Nous déduisons que toute période de \( \cos\) est une période de \( \sin\). De la même façon, nous pouvons prouver le contraire : toute période de \( \sin\) est une période de \( \cos\).
        \item
            D'un côté nous avons
            \begin{equation}
                \cos(2\pi)=\cos^2(\pi)-\sin^2(\pi)=1
            \end{equation}
            parce que \( \cos(2\pi)=\cos(0)=1\). Vu que \( \cos(\pi)\) et \( \sin(\pi)\) sont bornés par \( -1\) et \( 1\), nous devons avoir \( \sin(\pi)=0\) et \( \cos(\pi)=\pm 1\).

            Mais d'un autre côté, le nombre \( 2\pi\) est le plus petit \( T\) vérifiant \( \cos(T)=1\), \( \sin(T)=0\). Donc avoir \( \cos(\pi)=1\) n'est pas possible. Nous concluons
            \begin{subequations}
                \begin{numcases}{}
                    \cos(\pi)=-1\\
                    \sin(\pi)=0.
                \end{numcases}
            \end{subequations}
        \item
            Il s'agit d'utiliser les formules d'addition d'angles pour calculer \( \cos(a+\pi)\) et \( \sin(a+\pi)\) en tenant compte du fait que \( \cos(\pi)=-1\) et \( \sin(\pi)=0\).
        \item
        Soit \( a\in\mathopen] 0 , \pi \mathclose[\) tel que \( \cos(a)=-1\) et \( \sin(a)=0\). Alors nous avons
            \begin{subequations}
                \begin{align}
                    \cos(a+\pi)=-\cos(\pi)=1\\
                    \sin(a+\pi)=-\sin(\pi)=0,
                \end{align}
            \end{subequations}
        ce qui donnerait \( a+\pi\in\mathopen] \pi , 2\pi \mathclose[\) dont le cosinus est \( 1\) et le sinus est zéro. Mais nous savons déjà que \( 2\pi\) est le minimum pour cette propriété.
        \item
            Nous avons
            \begin{equation}
                -1=\cos(\pi)=\cos^2(\pi/2)-\sin^2(\pi/2),
            \end{equation}
            donc \( \cos(\pi/2)=0\) et \( \sin^2(\pi/2)=1\), ce qui donne \( \sin(\pi/2)=\pm 1\).

        Nous devons départager le \( \pm\). Pour cela nous savons que \( \sin'(0)=\cos(0)=1\), donc il existe \( \epsilon>\epsilon\) tel que pour tout \( x\in\mathopen] 0 , \epsilon \mathclose[\) nous avons \( 0<\cos(x)<1\) et \( 0\sin(x)<1\). Nous choisissons \( \epsilon\) plus petit que \( \pi/2\) .  
            
        Supposons que \( \sin(\pi/2)=-1\). Le théorème des valeurs intermédiaires \ref{ThoValInter} dit qu'il existe \( x_0\in\mathopen] \epsilon , \pi/2 \mathclose[\) tel que \( \sin(x_0)=0\). Pour cette valeur de \( x_0\) nous devons aussi avoir \( \cos(x_0)=\pm 1\). Mais vu que \( 2\pi\) est minium pour avoir \( \cos=1\) et \( \sin=0\) nous devons avoir \( \cos(x_0)=-1\). Alors nous avons aussi
            \begin{subequations}
                \begin{align}
                    \cos(x_0+\pi)=\cos(x_0)\cos(\pi)-\sin(x_0)\sin(\pi)=-\cos(x_0)=1\\
                    \sin(x_0+\pi)=\cos(x_0)\sin(\pi)+\sin(x_0)\cos(\pi)=\sin(x_0)=0.
                \end{align}
            \end{subequations}
            Encore une fois par minimalité de \( 2\pi\), cela ne va pas. Conclusion : \( \sin(\pi/2)=1\).
        \item
            Il s'agit encore d'utiliser les formules d'addition d'angle en tenant compte du fait que \( \cos(\pi/2)=0\) et \( \sin(\pi/2)=1\).
        \item
        Supposons \( x_0\in\mathopen] 0 , \pi/2 \mathclose[\) tel que \( \sin(x_0)=1\) et \( \cos(x_0)=0\). En utilisant les formules \eqref{EQSooRJZGooCFVqbZ} nous avons
            \begin{subequations}
                \begin{align}
                    \cos(x_0+\pi/2)=-1\\
                    \sin(x_0+\pi/2)=0,
                \end{align}
            \end{subequations}
            avec \( x_0+\pi/2<\pi\). Cela contredirait la minimalité de \( \pi\).
        \item
            Il s'agit d'utiliser les formules \eqref{EQSooRJZGooCFVqbZ} :
            \begin{subequations}
                \begin{align}
                    \cos(\frac{ 3\pi }{ 2 })=\cos(\pi+\pi/2)=-\sin(\pi)=0\\
                    \sin(\frac{ 3\pi }{ 2 })=\sin(\pi+\pi/2)=\cos(\pi)=-1.
                \end{align}
            \end{subequations}
        \item
            Si \( \cos(x_0)=0\) alors \( \sin(x_0)=-1\) (parce que \( \sin(x_0)=1\) est déjà exclu). Alors \( \cos(x_0+\pi/2)=1\) et \( \sin(x_0+\pi/2)=0\), ce qui est également impossible.
    \end{enumerate}
\end{proof}
Tout cela nous permet d'écrire le tableau de variations de sinus et cosinus.

\begin{proposition}     \label{PROPooKSGXooOqGyZj}
    L'application
    \begin{equation}
        \begin{aligned}
            \gamma\colon \mathopen[ 0 , 2\pi \mathclose[&\to S^1\subset \eR^2 \\
            t&\mapsto \big( \cos(t),\sin(t) \big) 
        \end{aligned}
    \end{equation}
    est une bijection continue.
\end{proposition}

\begin{proof}
    La continuité découle de la continuité des composantes. Le fait que l'image de \( \gamma\) soit dans \( S^1\) découle immédiatement du fait que \( \sin^2+\cos^2=1\).

    Pour la bijection, il faut injectif et surjectif.
    \begin{subproof}
        \item[Injectif]
            Soient \( x_1<x_2\) tels que \( \sin(x_1)=\sin(x_2)\) et \( \cos(x_1)=\cos(x_2)\). Supposons pour fixer les idées que \( \sin(x_1)>0\) et \( \cos(x_1)>0\) : si ce n'est pas le cas, il faut traiter séparément les \( 4\) possibilités de combinaisons de signes.

            Nous avons obligatoirement \( x_1,x_2\in\mathopen[ 0 , \frac{ \pi }{ 2 } \mathclose[\). Vu que \( \sin(x_1)=\sin(x_2)\), il existe par le théorème de Rolle \ref{ThoRolle} un élément \( c\in \mathopen] x_1 , x_2 \mathclose[\) tel que \( \sin'(c)=0\), c'est à dire \( \cos(c)=0\). Cela contredirait la proposition \ref{PROPooMWMDooJYIlis}\ref{ITEMooQKPKooEPeHER} à moins que \( x_1=x_2\).

            \item[Surjectif]

                Soient \( x,y\) tels que \( x^2+y^2=1\). Supposons pour varier les plaisirs que \( x<0\) et \( y>0\). Vu que la fonction \( \cos\) va de \( 0\) à \( -1\) lorsque \( x\) va de \( \pi/2\) à \( \pi\), le théorème des valeurs intermédiaires donne \( t\in\mathopen[ \pi/2 , \pi \mathclose]\) tel que \( \cos(t)=x\). Pour cette valeur de \( x\) nous avons
                \begin{equation}
                    \cos^2(x)+\sin^2(x)=1,
                \end{equation}
                et donc \( \sin^2(x)=y^2\), ce qui donne \( \sin(x)=\pm y\). Mais pour \( x\in \mathopen[ \pi/2 , \pi \mathclose]\) nous avons \( \sin(t)>0\). Par conséquent \( \sin(t)=y\).
    \end{subproof}
\end{proof}

\begin{lemma}       \label{LEMooIGNPooPEctJy}
    Nous avons les valeurs remarquables
    \begin{equation}
        \sin(\frac{ \pi }{ 4 })=\cos(\frac{ \pi }{ 4 })=\frac{ \sqrt{ 2 } }{2}.
    \end{equation}
\end{lemma}
Je crois que vous devriez pouvoir faire la preuve tout seul. Si ça ne va pas, contactez-moi.

%--------------------------------------------------------------------------------------------------------------------------- 
\subsection{Exemples}
%---------------------------------------------------------------------------------------------------------------------------

Nous mettons ici quelque exemples concernant les fonctions trigonométriques, qui n'ont pas pu être mis dans les chapitres le plus adapté, parce que ces derniers sont plus haut dans la table des matière.

\begin{example}[Taylor]     \label{EXooXLYJooKVqhTE}
	Le développement du cosinus est donné par
	\begin{equation}
		\cos(x)=1-\frac{ x^2 }{ 2 }+\frac{ x^4 }{ 4! }-\frac{ x^6 }{ 6! }\cdots
	\end{equation}
	Nous avons donc l'existence d'une fonction $h_1\in o(x^2)$ telle que $\cos(x)=1-\frac{ x^2 }{ 2 }+h_1(x)$. Il existe aussi une autre fonction $h_2\in o(x^4)$ telle que $\cos(x)=1-\frac{ x^2 }{ 2 }+\frac{ x^4 }{ 4! }+h_2(x)$.
\end{example}

\begin{example}[Limite et prolongement par continuité] \label{ExQWHooGddTLE}
    La fonction 
    \begin{equation}
        f(x)=\frac{ \cos(x)-1 }{ x }
    \end{equation}
    n'est pas définie en \( x=0\), mais en la limite
    \begin{equation}
        \lim_{x\to 0} \frac{ \cos(x)-1 }{ x }
    \end{equation}
    nous reconnaissons la limite définissant la dérivée du cosinus en \( 0\), c'est à dire que
    \begin{equation}
        \lim_{x\to 0} \frac{ \cos(x)-1 }{ x }=\sin(0)=0.
    \end{equation}
    Nous avons donc le prolongement par continuité
    \begin{equation}
        \tilde f(x)=\begin{cases}
            \frac{ \cos(x)-1 }{ x }    &   \text{si } x\neq 0\\
            0    &    \text{sinon}.
        \end{cases}
    \end{equation}

    Encore une fois, le graphe de la fonction \(\tilde f\) ne présente aucune particularité autour de \( x=0\).
    \begin{center}
        \input{auto/pictures_tex/Fig_RPNooQXxpZZ.pstricks}
    \end{center}
\end{example}

\begin{example}[Un calcul heuristique de limite]        \label{EXooINLRooPzRWEA}
    Soit à calculer la limite suivante :
    \begin{equation}
        \lim_{x\to 0} \frac{  e^{-2\cos(x)+2}\sin(x) }{ \sqrt{ e^{2\cos(x)+2}}-1 }.
    \end{equation}
    La stratégie que nous allons suivre pour calculer cette limite est de développer certaines parties de l'expression en série de Taylor, afin de simplifier l'expression. La première chose à faire est de remplacer $ e^{y(x)}$ par $1+y(x)$ lorsque $y(x)\to 0$. La limite devient
    \begin{equation}
        \lim_{x\to 0} \frac{ \big( -2\cos(x)+3 \big)\sin(x) }{ \sqrt{-2\cos(x)+2} }.
    \end{equation}
    Nous allons maintenant remplacer $\cos(x)$ par $1$ au numérateur et par $1-x^2/2$ au dénominateur. Pourquoi ? Parce que le cosinus du dénominateur est dans une racine, donc nous nous attendons à ce que le terme de degré deux du cosinus donne un degré un en dehors de la racine, alors que du degré un est exactement ce que nous avons au numérateur : le développement du sinus commence par $x$.

    Nous calculons donc
    \begin{equation}
        \begin{aligned}[]
            \lim_{x\to 0} \frac{ \sin(x) }{ \sqrt{-2\left( 1-\frac{ x^2 }{ 2 } \right)+2} }=\lim_{x\to 0} \frac{ \sin(x) }{ x }=1.
        \end{aligned}
    \end{equation}
    Tout ceci n'est évidement pas très rigoureux, mais en principe vous avez tous les éléments en main pour justifier les étapes.
\end{example}
