% This is part of Mes notes de mathématique
% Copyright (c) 2012-2017, 2019
%   Laurent Claessens
% See the file fdl-1.3.txt for copying conditions.

{\bf Attention } : cette liste de développements est un document de travail pour ceux qui écrivent dans la Frido. Elle sera supprimée des versions commerciales et donc des versions disponibles pour l'agrégation.

\vspace{2cm}


Nous donnons ici quelques idées de développements associés aux leçons données dans le rapport du jury 2016\cite{ooJECQooJvIKEJ}. Parfois, il est bon d'ajouter quelques lemmes au développement proposé, s'il est trop court. Si l'un ou l'autre ne vous semble pas adapté à l'énoncé de la leçon, faites le moi savoir.

%+++++++++++++++++++++++++++++++++++++++++++++++++++++++++++++++++++++++++++++++++++++++++++++++++++++++++++++++++++++++++++
\section{Algèbre et géométrie}
%+++++++++++++++++++++++++++++++++++++++++++++++++++++++++++++++++++++++++++++++++++++++++++++++++++++++++++++++++++++++++++

\paragraph{Applications des nombres complexes à la géométrie.}
\begin{itemize}
    \item Action du groupe modulaire sur le demi-plan de Poincaré, théorème~\ref{ThoItqXCm}.
    \item Générateurs du groupe diédral, proposition~\ref{PropLDIPoZ}
    \item Le groupe circulaire, proposition~\ref{THOooKMKWooZPIDaK}.
\end{itemize}
%---------------------------------------------------------------------------------------------------------------------------------------------
\paragraph{Formes linéaires et dualité en dimension finie. Exemples et applications.}
%---------------------------------------------------------------------------------------------------------------------------------------------
\paragraph{Polynômes d'endomorphismes en dimension finie. Réduction d'un endomorphisme en dimension finie. Applications.}
%---------------------------------------------------------------------------------------------------------------------------------------------
\paragraph{Anneaux principaux. Applications.}
%---------------------------------------------------------------------------------------------------------------------------------------------
\paragraph{Caractères d'un groupe abélien fini et transformée de Fourier discrète. Applications}
%---------------------------------------------------------------------------------------------------------------------------------------------
\paragraph{Algèbre des polynômes à plusieurs indéterminées. Applications.}
%---------------------------------------------------------------------------------------------------------------------------------------------
\paragraph{Racines d'un polynôme. Fonctions symétriques élémentaires. Exemples et applications.}

Le rapport du jury mentionne les théorèmes de Gershgorin (\ref{THOooUJNFooHpvCCF} et~\ref{THOooTXAPooQqsBCj}) pour cette leçon, mais je ne sais pas si ils valent un développement.
% En tout cas, cette note est utile pour qu'il y ait quelque part un lien vers ces théorèmes.

%---------------------------------------------------------------------------------------------------------------------------------------------
\paragraph{Sous-groupes discrets de \( \eR^2\). Réseaux. Exemples}
\begin{itemize}
    \item Les groupes de pavage de \( \eR^2\), théorème \ref{THOooUPHQooYfeHAy}.
\end{itemize}
%---------------------------------------------------------------------------------------------------------------------------------------------
\paragraph{Isométries d'un espace affine euclidien de dimension finie. Applications en dimensions $2$ et $3$.}
\begin{itemize}
    \item Isométries du cube, section~\ref{SecPVCmkxM}.
\end{itemize}
%---------------------------------------------------------------------------------------------------------------------------------------------
\paragraph{Systèmes d'équations linéaires ; opérations élémentaires, aspects algorithmiques et conséquences théoriques.}
\begin{itemize}
    \item Algorithme des facteurs invariants~\ref{PropPDfCqee}.
    \item Méthode du gradient à pas optimal~\ref{PropSOOooGoMOxG}.
\end{itemize}
%---------------------------------------------------------------------------------------------------------------------------------------------
\paragraph{Exemples d’équations diophantiennes.}
\begin{itemize}
    \item Dans~\ref{subsecZVKNooXNjPSf}, nous résolvons \( ax+by=c\) en utilisant Bézout (théorème~\ref{ThoBuNjam}).
    \item L'exemple~\ref{ExmuQisZU} résout l'équation \( x^2+2=y^3\) en parlant de l'extension \( \eZ[i\sqrt{2}]\) et de stathme.
    \item Les propositions~\ref{PropXHMLooRnJKRi} et~\ref{propFKKKooFYQcxE} parlent de triplets pythagoriciens.
    \item Le dénombrement des solutions de l'équation \( \alpha_1 n_1+\ldots \alpha_pn_p=n\) utilise des séries entières et des décompositions de fractions en éléments simples, théorème~\ref{ThoDIDNooUrFFei}.
\end{itemize}
%---------------------------------------------------------------------------------------------------------------------------------------------
\paragraph{Groupe opérant sur un ensemble. Exemples et applications.}
\begin{itemize}
    \item Les groupes de pavage de \( \eR^2\), théorème \ref{THOooUPHQooYfeHAy}.
    \item Action du groupe modulaire sur le demi-plan de Poincaré, théorème~\ref{ThoItqXCm}.
    \item Polynômes semi-symétriques, proposition~\ref{PropUDqXax}.
    \item Lemme de Morse, lemme~\ref{LemNQAmCLo}.
    \item Générateurs du groupe diédral, proposition~\ref{PropLDIPoZ}.
    \item Sous-groupes compacts de \( \GL(n,\eR)\), lemme~\ref{LemOCtdiaE} ou proposition~\ref{PropQZkeHeG}.
    \item Théorème de Wedderburn~\ref{ThoMncIWA}.
    \item Isométries du cube, section~\ref{SecPVCmkxM}.
    \item Algorithme des facteurs invariants~\ref{PropPDfCqee}.
\end{itemize}
%---------------------------------------------------------------------------------------------------------------------------------------------
\paragraph{Exemples de sous-groupes distingués et de groupes quotients. Applications.}
\begin{itemize}
    \item Suites de décomposition et théorème de Jordan-Hölder~\ref{ThoLgxWIC}.
    \item Groupes d'ordre \( pq\), théorème~\ref{ThoLnTMBy}.
    \item Le groupe alterné est simple, théorème~\ref{ThoURfSUXP}.
    \item Théorème de Lie-Kolchin~\ref{ThoUWQBooCvutTO}.
\end{itemize}
%---------------------------------------------------------------------------------------------------------------------------------------------
\paragraph{Sous-groupes finis de \( \gO(2,\eR)\) et \( \gO(3,\eR)\). Applications}
\begin{itemize}
    \item Les groupes de pavage de \( \eR^2\), théorème \ref{THOooUPHQooYfeHAy}.
\end{itemize}
%---------------------------------------------------------------------------------------------------------------------------------------------
\paragraph{Groupes finis. Exemples et applications.}
\begin{itemize}
    \item RSA, section~\ref{SecEVaFYi}, plus l'exponentielle rapide, plus la recherche de couples de Bézout.
    \item Théorème de Wedderburn~\ref{ThoMncIWA}.
    \item Théorème de Sylow~\ref{ThoUkPDXf}. Tout le théorème, c'est un peu long. On peut se contenter de la partie qui dit que \( G\) contient un \( p\)-Sylow.
    \item Coloriage de roulette (\ref{pTqJLY}) et composition de colliers (\ref{siOQlG}).
    \item Suites de décomposition et théorème de Jordan-Hölder~\ref{ThoLgxWIC}.
    \item Théorème de Burnside sur les sous-groupes d'exposant fini de \( \GL(n,\eC)\), théorème~\ref{ThooJLTit}.
    \item \( (\eZ/p\eZ)^*\simeq \eZ/(p-1)\eZ\), corolaire~\ref{CorpRUndR}.
    \item Groupes d'ordre \( pq\), théorème~\ref{ThoLnTMBy}.
    \item Générateurs du groupe diédral, proposition~\ref{PropLDIPoZ}.
    \item Le groupe alterné est simple, théorème~\ref{ThoURfSUXP}.
    \item Isométries du cube, section~\ref{SecPVCmkxM}.
\end{itemize}
%---------------------------------------------------------------------------------------------------------------------------------------------
\paragraph{Groupe des permutations d'un ensemble fini. Applications.}
\begin{itemize}
    \item RSA, section~\ref{SecEVaFYi}, plus l'exponentielle rapide, plus la recherche de couples de Bézout.
    \item Coloriage de roulette (\ref{pTqJLY}) et composition de colliers (\ref{siOQlG}).
    \item Forme alternées de degré maximum, proposition~\ref{ProprbjihK}.
    \item Décomposition de Bruhat, théorème~\ref{ThoizlYJO}.
    \item Polynômes semi-symétriques, proposition~\ref{PropUDqXax}.
    \item Table des caractères du groupe diédral, section~\ref{SecWMzheKf}.
    \item Table des caractères du groupe symétrique \( S_4\), section~\ref{SecUMIgTmO}.
    \item Isométries du cube, section~\ref{SecPVCmkxM}.
    \item Le groupe alterné est simple, théorème~\ref{ThoURfSUXP}.
\end{itemize}
%---------------------------------------------------------------------------------------------------------------------------------------------
\paragraph{Groupe linéaire d’un espace vectoriel de dimension finie $E$ , sous-groupes de $\GL(E)$. Applications.}
%\index{groupe!linéaire}
\begin{itemize}
    \item Théorème de Burnside sur les sous-groupes d'exposant fini de \( \GL(n,\eC)\), théorème~\ref{ThooJLTit}.
    \item Décomposition de Bruhat, théorème~\ref{ThoizlYJO}.
    \item Le lemme au lemme de Morse, lemme~\ref{LemWLCvLXe}.
    \item Décomposition polaire~\ref{ThoLHebUAU}.
    \item Enveloppe convexe du groupe orthogonal~\ref{ThoVBzqUpy}.
    \item Sous-groupes compacts de \( \GL(n,\eR)\), lemme~\ref{LemOCtdiaE} ou proposition~\ref{PropQZkeHeG}.
    \item Théorème de Von Neumann~\ref{ThoOBriEoe}.
\end{itemize}
%---------------------------------------------------------------------------------------------------------------------------------------------
\paragraph{Représentations et caractères d'un groupe fini sur un \( \eC\)-espace vectoriel. Exemples}
\begin{itemize}
    \item Table des caractères du groupe diédral, section~\ref{SecWMzheKf}.
    \item Table des caractères du groupe symétrique \( S_4\), section~\ref{SecUMIgTmO}.
\end{itemize}
%---------------------------------------------------------------------------------------------------------------------------------------------
\paragraph{Exemples de parties génératrices d’un groupe. Applications.}
\begin{itemize}
    \item RSA, section~\ref{SecEVaFYi}. Assez indirect : la système RSA se base sur la formule \( \varphi(pq)=(p-1)(q-1)\), laquelle se base sur l'isomorphisme \( \eZ/p\eZ\times \eZ/q\eZ\simeq \eZ/pq\eZ\) et leurs générateurs.
    \item Action du groupe modulaire sur le demi-plan de Poincaré, théorème~\ref{ThoItqXCm}, parce que c'est avec lui qu'on montre les générateurs du groupe modulaire dans le corolaire~\ref{CorJQwgNp}.
    \item Générateurs du groupe diédral, proposition~\ref{PropLDIPoZ}
    \item Table des caractères du groupe diédral, section~\ref{SecWMzheKf}.
    \item Le groupe alterné est simple, théorème~\ref{ThoURfSUXP}.
    \item Les groupes de pavage de \( \eR^2\), théorème \ref{THOooUPHQooYfeHAy}.
\end{itemize}

%---------------------------------------------------------------------------------------------------------------------------------------------
\paragraph{Anneaux $\eZ/n\eZ$. Applications.}
\begin{itemize}
    \item RSA, section~\ref{SecEVaFYi}, plus l'exponentielle rapide, plus la recherche de couples de Bézout.
    \item Forme faible du théorème de Dirichlet (avec ses deux lemmes)~\ref{ThoxwTjcl}.
    \item \( (\eZ/p\eZ)^*\simeq \eZ/(p-1)\eZ\), corolaire~\ref{CorpRUndR}.
    \item Groupes d'ordre \( pq\), théorème~\ref{ThoLnTMBy}.
    \item Irréductibilité des polynômes cyclotomiques, proposition~\ref{PropoIeOVh}.
\end{itemize}

%---------------------------------------------------------------------------------------------------------------------------------------------
\paragraph{Nombres premiers. Applications.}
\begin{itemize}
    \item Structure des groupes d'ordre \( pq\), théorème~\ref{ThoLnTMBy}.
    \item Divergence de la somme des inverses des nombres premiers, théorème~\ref{ThonfVruT}.
    \item RSA, section~\ref{SecEVaFYi}, plus l'exponentielle rapide, plus la recherche de couples de Bézout.
    \item Forme faible du théorème de Dirichlet (avec ses deux lemmes)~\ref{ThoxwTjcl}.
    \item \( (\eZ/p\eZ)^*\simeq \eZ/(p-1)\eZ\), corolaire~\ref{CorpRUndR}, peut-être redondant avec les groupes d'ordre \( pq\).
    \item Irréductibilité des polynômes cyclotomiques, proposition~\ref{PropoIeOVh}.
    \item Théorème des deux carrés, théorème~\ref{ThospaAEI}.
\end{itemize}
%---------------------------------------------------------------------------------------------------------------------------------------------
\paragraph{Corps finis. Applications.}
\begin{itemize}
    \item Théorème de Chevalley-Warning~\ref{ThoLTcYKk}.
    \item Loi de réciprocité quadratique~\ref{ThoMiEiUm}.
    \item \( (\eZ/p\eZ)^*\simeq \eZ/(p-1)\eZ\), corolaire~\ref{CorpRUndR}.
    \item Polynômes irréductibles sur \( \eF_q\).
\end{itemize}

%---------------------------------------------------------------------------------------------------------------------------------------------
\paragraph{Groupe des nombres complexes de module 1. Sous-groupes des racines de l'unité. Applications.}
\begin{itemize}
    \item Théorème de Burnside sur les sous-groupes d'exposant fini de \( \GL(n,\eC)\), théorème~\ref{ThooJLTit}.
    \item Forme faible du théorème de Dirichlet (avec ses deux lemmes)~\ref{ThoxwTjcl} (parce qu'on parle de polynômes cyclotomiques qui sont basés sur les racines de l'unité).
    \item Action du groupe modulaire sur le demi-plan de Poincaré, théorème~\ref{ThoItqXCm}, parce qu'on y utilise un peu les propriétés des nombres du type \( | z |=1\).
    \item Générateurs du groupe diédral, proposition~\ref{PropLDIPoZ}.
    \item Irréductibilité des polynômes cyclotomiques, proposition~\ref{PropoIeOVh}.
    \item Théorème de Wedderburn~\ref{ThoMncIWA}.
\end{itemize}

%---------------------------------------------------------------------------------------------------------------------------------------------
\paragraph{Polynômes irréductibles à une indéterminée. Corps de rupture. Exemples et applications.}
\begin{itemize}
    \item Irréductibilité des polynômes cyclotomiques, proposition~\ref{PropoIeOVh}.
    \item Polynômes irréductibles sur \( \eF_q\).
\end{itemize}
%---------------------------------------------------------------------------------------------------------------------------------------------
\paragraph{Exemples d'actions de groupes sur les espaces de matrices.}
\begin{itemize}
    \item Action du groupe modulaire sur le demi-plan de Poincaré, théorème~\ref{ThoItqXCm}.
    \item Lemme de Morse, lemme~\ref{LemNQAmCLo}.
    \item Sous-groupes compacts de \( \GL(n,\eR)\), lemme~\ref{LemOCtdiaE} ou proposition~\ref{PropQZkeHeG}.
    \item Algorithme des facteurs invariants~\ref{PropPDfCqee}.
\end{itemize}
%---------------------------------------------------------------------------------------------------------------------------------------------
\paragraph{Dimension d'un espace vectoriel (on se limitera au cas de la dimension finie). Rang. Exemples et applications.}
%\index{rang}
\begin{itemize}
    \item Forme alternées de degré maximum, proposition~\ref{ProprbjihK}, parce que c'est ce théorème qui donne l'unicité du déterminant du fait que l'espace est de dimension un.
    \item Théorème de la dimension~\ref{ThonmnWKs}.
        %Ici on peut mettre le théorème de Sylvester.
    \item Extrema liés, théorème~\ref{ThoRGJosS}.
    \item Théorème~\ref{ThoeTMXla} sur la diagonalisation de matrices symétriques.
    \item Stabilité du rang par extension des scalaires, proposition~\ref{PROPooJFQDooZSsxMf}.
\end{itemize}
%---------------------------------------------------------------------------------------------------------------------------------------------
\paragraph{Déterminant. Exemples et applications.}
\begin{itemize}
    \item Forme alternées de degré maximum, proposition~\ref{ProprbjihK}, parce que c'est ce théorème qui donne l'unicité du déterminant du fait que l'espace est de dimension un.
    \item Théorème de Rothstein-Trager~\ref{ThoXJFatfu} parce que le résultant est est un.
    \item Ellipsoïde de John-Loewner, proposition~\ref{PropJYVooRMaPok}.
\end{itemize}
%---------------------------------------------------------------------------------------------------------------------------------------------
\paragraph{Polynômes d’endomorphisme en dimension finie. Réduction d’un endomorphisme en dimension finie. Applications.}
\begin{itemize}
    \item Racine carrée d'une matrice hermitienne positive, proposition~\ref{PropVZvCWn}.
    \item Théorème de Burnside sur les sous-groupes d'exposant fini de \( \GL(n,\eC)\), théorème~\ref{ThooJLTit}.
    \item Décomposition de Dunford, théorème~\ref{ThoRURcpW}.
    \item Algorithme des facteurs invariants~\ref{PropPDfCqee}.
\end{itemize}
%---------------------------------------------------------------------------------------------------------------------------------------------
\paragraph{Sous-espaces stables par un endomorphisme ou une famille d’endomorphismes d’un espace vectoriel de dimension finie. Applications.}
%\index{endomorphisme!sous-espace stable}
\begin{itemize}
    \item Équation de Hill \( y''+qy=0\), proposition~\ref{PropGJCZcjR}.
    \item Décomposition de Dunford, théorème~\ref{ThoRURcpW}.
    \item Théorème de Lie-Kolchin~\ref{ThoUWQBooCvutTO}.
\end{itemize}
%---------------------------------------------------------------------------------------------------------------------------------------------
\paragraph{Endomorphismes diagonalisables en dimension finie.}
\begin{itemize}
    \item Théorème de Burnside sur les sous-groupes d'exposant fini de \( \GL(n,\eC)\), théorème~\ref{ThooJLTit}.
    \item Racine carrée d'une matrice hermitienne positive, proposition~\ref{PropVZvCWn}, parce qu'un utilise le résultat de diagonalisation simultanée.
    \item Équation de Hill \( y''+qy=0\), proposition~\ref{PropGJCZcjR}.
    \item Décomposition de Dunford, théorème~\ref{ThoRURcpW}.
    \item Endomorphismes cycliques et commutant dans le cas diagonalisable, proposition~\ref{PropooQALUooTluDif}.
\end{itemize}
%---------------------------------------------------------------------------------------------------------------------------------------------
\paragraph{Exponentielle de matrices. Applications.}
\begin{itemize}
    \item Décomposition de Dunford, théorème~\ref{ThoRURcpW}.
    \item Théorème de Von Neumann~\ref{ThoOBriEoe}.
\end{itemize}
%---------------------------------------------------------------------------------------------------------------------------------------------
\paragraph{Endomorphismes trigonalisables. Endomorphismes nilpotents.}
\begin{itemize}
    \item Théorème de Burnside sur les sous-groupes d'exposant fini de \( \GL(n,\eC)\), théorème~\ref{ThooJLTit}.
    \item Décomposition de Dunford, théorème~\ref{ThoRURcpW}.
    \item Théorème de Lie-Kolchin~\ref{ThoUWQBooCvutTO}.
\end{itemize}
%---------------------------------------------------------------------------------------------------------------------------------------------
\paragraph{Matrices symétriques réelles, matrices hermitiennes.}
\begin{itemize}
    \item Racine carrée d'une matrice hermitienne positive, proposition~\ref{PropVZvCWn}.
    \item Le lemme au lemme de Morse, lemme~\ref{LemWLCvLXe}.
    \item Connexité des formes quadratiques de signature donnée, proposition~\ref{PropNPbnsMd}.
    \item Théorème~\ref{ThoeTMXla} sur la diagonalisation de matrices symétriques.
\end{itemize}
%---------------------------------------------------------------------------------------------------------------------------------------------
\paragraph{Formes quadratiques sur un espace vectoriel de dimension finie. Orthogonalité, isotropie. Applications.}
\begin{itemize}
    \item Le lemme au lemme de Morse, lemme~\ref{LemWLCvLXe}, voir le lemme de Morse lui-même~\ref{LemNQAmCLo}.
    \item Connexité des formes quadratiques de signature donnée, proposition~\ref{PropNPbnsMd}.
    \item Sous-groupes compacts de \( \GL(n,\eR)\), lemme~\ref{LemOCtdiaE} ou proposition~\ref{PropQZkeHeG}.
    \item Ellipsoïde de John-Loewner, proposition~\ref{PropJYVooRMaPok}.
\end{itemize}
%---------------------------------------------------------------------------------------------------------------------------------------------
\paragraph{Endomorphismes remarquables d’un espace vectoriel euclidien (de dimension finie).}
%\index{endomorphisme!décomposition!polaire}
\begin{itemize}
    \item Décomposition polaire~\ref{ThoLHebUAU}.
    \item Sous-groupes compacts de \( \GL(n,\eR)\), lemme~\ref{LemOCtdiaE} ou proposition~\ref{PropQZkeHeG}.
    \item Théorème~\ref{ThoeTMXla} sur la diagonalisation de matrices symétriques.
\end{itemize}
%---------------------------------------------------------------------------------------------------------------------------------------------
\paragraph{Angles : Définitions et utilisation en géométrie}
\begin{itemize}
    \item Les groupes de pavage de \( \eR^2\), théorème \ref{THOooUPHQooYfeHAy}.
\end{itemize}
%---------------------------------------------------------------------------------------------------------------------------------------------
\paragraph{Isométries d'un espace affine euclidien de dimension finie. Formes réduites. Applications}
\begin{itemize}
    \item Les groupes de pavage de \( \eR^2\), théorème \ref{THOooUPHQooYfeHAy}.
\end{itemize}
%---------------------------------------------------------------------------------------------------------------------------------------------
\paragraph{Applications affines}
\begin{itemize}
    \item Les groupes de pavage de \( \eR^2\), théorème \ref{THOooUPHQooYfeHAy}.
\end{itemize}
%---------------------------------------------------------------------------------------------------------------------------------------------
\paragraph{Barycentres dans un espace affine réel de dimension finie, convexité. Applications.}
\begin{itemize}
    \item Théorème de Carathéodory~\ref{ThoJLDjXLe}.
    \item Points extrémaux de la boule unité dans \( \aL(E)\), théorème~\ref{ThoBALmoQw}.
    \item Enveloppe convexe du groupe orthogonal~\ref{ThoVBzqUpy}.
\end{itemize}
%---------------------------------------------------------------------------------------------------------------------------------------------
\paragraph{Utilisation des groupes en géométrie.}
\begin{itemize}
    \item Coloriage de roulette (\ref{pTqJLY}) et composition de colliers (\ref{siOQlG}).
    \item Forme alternées de degré maximum, proposition~\ref{ProprbjihK}, parce que c'est ce théorème qui donne l'unicité du déterminant du fait que l'espace est de dimension un.
    \item Action du groupe modulaire sur le demi-plan de Poincaré, théorème~\ref{ThoItqXCm}.
    \item Générateurs du groupe diédral, proposition~\ref{PropLDIPoZ}
    \item Isométries du cube, section~\ref{SecPVCmkxM}.
    \item Les groupes de pavage de \( \eR^2\), théorème \ref{THOooUPHQooYfeHAy}.
\end{itemize}
%---------------------------------------------------------------------------------------------------------------------------------------------
\paragraph{Méthodes combinatoires, problèmes de dénombrement.}
\begin{itemize}
    \item Coloriage de roulette (\ref{pTqJLY}) et composition de colliers (\ref{siOQlG}).
    \item Nombres de Bell, théorème~\ref{ThoYFAzwSg}.
    \item Le dénombrement des solutions de l'équation \( \alpha_1 n_1+\ldots \alpha_pn_p=n\) utilise des séries entières et des décompositions de fractions en éléments simples, théorème~\ref{ThoDIDNooUrFFei}.
\end{itemize}
%---------------------------------------------------------------------------------------------------------------------------------------------
\paragraph{Formes quadratiques réelles. Coniques. Exemples et applications.}
\paragraph{Extensions de corps. Exemples et applications.}
\begin{itemize}
    \item Polynômes séparables, proposition~\ref{PropolyeZff}.
    \item Lien entre les racines (multiples) de \( P\) et \( P'\), proposition~\ref{PropolyeZff}.
    \item Théorème de l'élément primitif~\ref{ThoORxgBC}.
    \item À propos d'extensions de \( \eQ\), le lemme~\ref{LemSoXCQH}.
    \item Polynômes irréductibles sur \( \eF_q\).
    \item Polygones réguliers constructibles, théorème de Gauss-Wantzel,~\ref{ThoTWAooEsLjJu}.
\end{itemize}
%---------------------------------------------------------------------------------------------------------------------------------------------
%+++++++++++++++++++++++++++++++++++++++++++++++++++++++++++++++++++++++++++++++++++++++++++++++++++++++++++++++++++++++++++
\section{Analyse}
%+++++++++++++++++++++++++++++++++++++++++++++++++++++++++++++++++++++++++++++++++++++++++++++++++++++++++++++++++++++++++++

%---------------------------------------------------------------------------------------------------------------------------------------------
%---------------------------------------------------------------------------------------------------------------------------------------------
\paragraph{Fonctions holomorphes sur un ouvert de $\eC$. Exemples et applications.}
%---------------------------------------------------------------------------------------------------------------------------------------------
\paragraph{Problèmes d'interversion de limites et d'intégrales.}
%---------------------------------------------------------------------------------------------------------------------------------------------
\paragraph{Suites vectorielles et réelles définies par une relation de récurrence \( u_{n+1}=f(u_n)\). Exemples. Applications à la résolution approchée d'équations.}
%---------------------------------------------------------------------------------------------------------------------------------------------
\paragraph{Approximation d'une fonction par des polynômes et des polynômes trigonométriques. Exemples et applications.}
%---------------------------------------------------------------------------------------------------------------------------------------------
\paragraph{Espérance, variance et moments d'une variable aléatoire.}
%---------------------------------------------------------------------------------------------------------------------------------------------
\paragraph{Extremums : existence, caractérisation, recherche. Exemples et applications.}
%---------------------------------------------------------------------------------------------------------------------------------------------
\paragraph{Suites numériques. Convergence, valeurs d'adhérence. Exemples et applications.}
%---------------------------------------------------------------------------------------------------------------------------------------------
\paragraph{Exemples de développements asymptotiques de suites et de fonctions.}
%---------------------------------------------------------------------------------------------------------------------------------------------
\paragraph{Utilisation de la notion de convexité en analyse.}
\begin{itemize}
    \item Ellipsoïde de John-Loewner, proposition~\ref{PropJYVooRMaPok}.
    \item Peut-être la méthode de Newton, théorème~\ref{ThoHGpGwXk}, mais je ne sais pas très bien pourquoi.
\end{itemize}
%---------------------------------------------------------------------------------------------------------------------------------------------
\paragraph{Équations différentielles \( X'=f(t,X)\). Exemple d'étude des solutions en dimension \( 1\) et \( 2\).}
\begin{itemize}
    \item Théorème de Cauchy-Lipschitz global~\ref{THOooZIVRooPSWMxg}.
\end{itemize}
%---------------------------------------------------------------------------------------------------------------------------------------------
\paragraph{Exemples d'équations aux dérivées partielles linéaires}
%---------------------------------------------------------------------------------------------------------------------------------------------
\paragraph{Équations différentielles linéaires. Systèmes d’équations différentielles linéaires. Exemples et applications.}
\begin{itemize}
    \item Équation de Hill \( y''+qy=0\), proposition~\ref{PropGJCZcjR}.
    \item Théorème de stabilité de Lyapunov~\ref{ThoBSEJooIcdHYp}.
    \item Le système proie-prédateur de Lotka-Volterra~\ref{ThoJHCLooHjeCvT}
    \item Théorème de Cauchy-Lipschitz global~\ref{THOooZIVRooPSWMxg}, si on parvient à réexprimer le théorème dans le cas linéaire.
\end{itemize}
%--------------------------------------------------------------------------------------------------------------------------------------------
\paragraph{Méthodes itératives en analyse numérique matricielle.}
%---------------------------------------------------------------------------------------------------------------------------------------------
\paragraph{Transformation de Fourier. Applications.}
%---------------------------------------------------------------------------------------------------------------------------------------------
\paragraph{Fonctions monotones. Fonctions convexes. Exemples et applications.}
\begin{itemize}
    \item La proposition~\ref{PropMYskGa} donne un résultat sur \( y''+qy=0\) à partir d'une hypothèse de croissance.
    \item L'inégalité de Jensen, proposition~\ref{PropABtKbBo}.
    %\item Méthode de Newton, théorème~\ref{ThoHGpGwXk}      Arnaud Girand la mets, mais je ne vois pas pourquoi.
    \item Ellipsoïde de John-Loewner, proposition~\ref{PropJYVooRMaPok}.
\end{itemize}
%---------------------------------------------------------------------------------------------------------------------------------------------
\paragraph{Espaces de fonctions : exemples et applications.}
\begin{itemize}
    \item Théorème de Fischer-Riesz~\ref{ThoGVmqOro}.
    \item Espace de Sobolev \( H^1(I)\), théorème~\ref{ThoESIyxfU}.
    \item Théorème de Cauchy-Lipschitz~\ref{ThokUUlgU}.
    \item Dual de \( L^p\big( \mathopen[ 0 , 1 \mathclose] \big)\) pour \( 1<p<2\), proposition~\ref{PropOAVooYZSodR}.
\end{itemize}
%---------------------------------------------------------------------------------------------------------------------------------------------
\paragraph{Exemples de parties denses et applications.}
\begin{itemize}
    \item Prolongement de fonction définie sur une partie dense, théorème~\ref{ThoPVFQMi}
    \item Complétion d'un espace métrique, théorème~\ref{ThoKHTQJXZ}.
    \item Points extrémaux de la boule unité dans \( \aL(E)\), théorème~\ref{ThoBALmoQw}.
    \item Critère de Weyl, proposition~\ref{PropDMvPDc}.
    \item Densité des polynômes dans \( C^0\big( \mathopen[ 0 , 1 \mathclose] \big)\), théorème de Bernstein~\ref{ThoDJIvrty}.
    \item Enveloppe convexe du groupe orthogonal~\ref{ThoVBzqUpy}.
\end{itemize}
%---------------------------------------------------------------------------------------------------------------------------------------------
\paragraph{Utilisation de la notion de compacité.}
\begin{itemize}
    \item Le théorème de Weierstrass sur la limite uniforme de fonctions holomorphes, théorème~\ref{ThoArYtQO}.
    \item Suite telle que \( \lim_{k\to \infty} d(u_{k+1},u_k)=0\), théorème~\ref{PropLHWACDU}.
    \item Sous-groupes compacts de \( \GL(n,\eR)\), lemme~\ref{LemOCtdiaE} ou proposition~\ref{PropQZkeHeG}.
    \item Théorème de Montel~\ref{ThoXLyCzol}.
    \item Ellipsoïde de John-Loewner, proposition~\ref{PropJYVooRMaPok}.
\end{itemize}
%---------------------------------------------------------------------------------------------------------------------------------------------
\paragraph{Connexité. Exemples et applications.}
\begin{itemize}
    \item Théorème de Runge~\ref{ThoMvMCci}.
    \item Suite telle que \( \lim_{k\to \infty} d(u_{k+1},u_k)=0\), théorème~\ref{PropLHWACDU}.
    \item Théorème de Brouwer en dimension \( 2\) via l'homotopie~\ref{ThoLVViheK}.
    \item Théorème de Lie-Kolchin~\ref{ThoUWQBooCvutTO}.
\end{itemize}
%---------------------------------------------------------------------------------------------------------------------------------------------
\paragraph{Espaces complets. Exemples et applications.}
\begin{itemize}
    \item La proposition~\ref{PropWoywYG} qui donne des indications sur la notion de classes dans \( L^p\).
    \item Prolongement de fonction définie sur une partie dense, théorème~\ref{ThoPVFQMi}
    \item Complétion d'un espace métrique, théorème~\ref{ThoKHTQJXZ}.
    \item Théorème de Fischer-Riesz~\ref{ThoGVmqOro}.
    \item Théorème de Cauchy-Lipschitz global~\ref{THOooZIVRooPSWMxg}.
\end{itemize}
%---------------------------------------------------------------------------------------------------------------------------------------------
\paragraph{Prolongement de fonctions. Exemples et applications.}
\begin{itemize}
    \item Prolongement de fonction définie sur une partie dense, théorème~\ref{ThoPVFQMi}
    \item Lemme de Borel~\ref{LemRENlIEL}.
    \item Prolongement méromorphe de la fonction \( \Gamma\) d'Euler.
    \item Théorème de Tietze~\ref{ThoFFQooGvcLzJ}.
        % théorème de Hadamard
\end{itemize}
%---------------------------------------------------------------------------------------------------------------------------------------------
\paragraph{Espaces de Hilbert. Bases hilbertiennes. Exemples et applications.}
\begin{itemize}
    \item Espace de Sobolev \( H^1(I)\), théorème~\ref{ThoESIyxfU}.
    \item Inégalité isopérimétrique, théorème~\ref{ThoIXyctPo}.
    \item Dual de \( L^p\big( \mathopen[ 0 , 1 \mathclose] \big)\) pour \( 1<p<2\), proposition~\ref{PropOAVooYZSodR}.
        % Fonctions de Haar
\end{itemize}
%---------------------------------------------------------------------------------------------------------------------------------------------
\paragraph{Théorème d'inversion locale, théorème des fonctions implicites. Exemples et applications en analyse et en géométrie.}
%---------------------------------------------------------------------------------------------------------------------------------------------
\paragraph{Applications différentiables définies sur un ouvert de $\eR^n$ . Exemples et applications.}
\begin{itemize}
    \item Extrema liés, théorème~\ref{ThoRGJosS}.
    \item Théorème d'inversion locale, théorème~\ref{ThoXWpzqCn}.
    \item Lemme de Morse, lemme~\ref{LemNQAmCLo}.
\end{itemize}
%---------------------------------------------------------------------------------------------------------------------------------------------
\paragraph{Applications des formules de Taylor.}
\begin{itemize}
    \item Méthode de Newton, théorème~\ref{ThoHGpGwXk}
    \item Lemme de Morse, lemme~\ref{LemNQAmCLo}.
\end{itemize}
%---------------------------------------------------------------------------------------------------------------------------------------------
\paragraph{Continuité et dérivabilité des fonctions réelles d'une variable réelle. Exemples et applications.}
%---------------------------------------------------------------------------------------------------------------------------------------------
%\index{série!numérique}
\begin{itemize}
    \item Divergence de la somme des inverses des nombres premiers, théorème~\ref{ThonfVruT}.
    \item Formule sommatoire de Poisson, proposition~\ref{ProprPbkoQ}.
    \item Théorème taubérien de Hardy-Littlewood~\ref{ThoPdDxgP}.
    \item Nombres de Bell, théorème~\ref{ThoYFAzwSg}.
    \item Partitions d'un entier en parts fixes, proposition~\ref{PropWUFpuBR}.
    \item Théorème d'Abel angulaire~\ref{ThoTGjmeen}.
\end{itemize}
%---------------------------------------------------------------------------------------------------------------------------------------------
\paragraph{Espaces \( L^p\), \( 1\leq p\leq\infty\)}
\begin{itemize}
    \item La proposition~\ref{PropWoywYG} qui donne des indications sur la notion de classes dans \( L^p\).
    \item Théorème de Fischer-Riesz~\ref{ThoGVmqOro}.
    \item Espace de Sobolev \( H^1(I)\), théorème~\ref{ThoESIyxfU}.
    \item Dual de \( L^p\big( \mathopen[ 0 , 1 \mathclose] \big)\) pour \( 1<p<2\), proposition~\ref{PropOAVooYZSodR}.
\end{itemize}
%---------------------------------------------------------------------------------------------------------------------------------------------
\paragraph{Illustrer par des exemples quelques méthodes de calcul d'intégrales de fonctions d’une ou plusieurs variables.}
%---------------------------------------------------------------------------------------------------------------------------------------------
\paragraph{Fonctions définies par une intégrale dépendant d’un paramètre. Exemples et applications.}
\begin{itemize}
    \item Le théorème de Weierstrass sur la limite uniforme de fonctions holomorphes, théorème~\ref{ThoArYtQO}.
    \item Les théorèmes sur les fonctions définies par des intégrales, section~\ref{SecCHwnBDj}.
    \item Lemme de Morse, lemme~\ref{LemNQAmCLo}.
    \item Prolongement méromorphe de la fonction \( \Gamma\) d'Euler.
\end{itemize}
%---------------------------------------------------------------------------------------------------------------------------------------------
\paragraph{Suites et séries de fonctions. Exemples et contre-exemples.}
\begin{itemize}
    \item Formule sommatoire de Poisson, proposition~\ref{ProprPbkoQ}.
    \item Théorème taubérien de Hardy-Littlewood~\ref{ThoPdDxgP}.
    \item Le théorème de Weierstrass sur la limite uniforme de fonctions holomorphes, théorème~\ref{ThoArYtQO}.
    \item La proposition~\ref{PropWoywYG} qui donne des indications sur la notion de classes dans \( L^p\).
    \item Théorème de Montel~\ref{ThoXLyCzol}.
    \item Prolongement méromorphe de la fonction \( \Gamma\) d'Euler.
\end{itemize}
%---------------------------------------------------------------------------------------------------------------------------------------------
\paragraph{Convergence des séries entières, propriétés de la somme. Exemples et applications.}
\begin{itemize}
    \item Processus de Galton-Watson, théorème~\ref{ThoJZnAOA}.
    \item Formule sommatoire de Poisson, proposition~\ref{ProprPbkoQ}.
    \item Nombres de Bell, théorème~\ref{ThoYFAzwSg}.
    \item Théorème d'Abel angulaire~\ref{ThoTGjmeen}.
\end{itemize}
\paragraph{Séries de Fourier. Exemples et applications.}
\begin{itemize}
    \item Formule sommatoire de Poisson, proposition~\ref{ProprPbkoQ}.
    \item Inégalité isopérimétrique, théorème~\ref{ThoIXyctPo}.
    \item Fonction continue et périodique dont la série de Fourier ne converge pas, proposition~\ref{PropREkHdol}.
\end{itemize}
%---------------------------------------------------------------------------------------------------------------------------------------------
\paragraph{Espaces vectoriels normés, applications linéaires continues. Exemples.}
\begin{itemize}
    \item Théorème de Fischer-Riesz~\ref{ThoGVmqOro}.
    \item Théorème de Banach-Steinhaus~\ref{ThoPFBMHBN}.
    \item Dual de \( L^p\big( \mathopen[ 0 , 1 \mathclose] \big)\) pour \( 1<p<2\), proposition~\ref{PropOAVooYZSodR}.
\end{itemize}
