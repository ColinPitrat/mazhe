% This is part of (everything) I know in mathematics
% Copyright (c) 2011-2017
%   Laurent Claessens
% See the file fdl-1.3.txt for copying conditions.

%+++++++++++++++++++++++++++++++++++++++++++++++++++++++++++++++++++++++++++++++++++++++++++++++++++++++++++++++++++++++++++
\section{Isométries dans $\eR^n$}
%+++++++++++++++++++++++++++++++++++++++++++++++++++++++++++++++++++++++++++++++++++++++++++++++++++++++++++++++++++++++++++

\begin{definition}
    Un \defe{hyperplan}{hyperplan} de \( \eR^n\) est un sous-espace affine de dimension \( n-1\).
\end{definition}

\begin{lemmaDef}
    Si un hyperplan \( H\) de \( \eR^n\) est donné, et si \( x\in \eR^n\), il existe un unique point \( y\in \eR^n\) tel que
    \begin{enumerate}
        \item
            \( x-y\perp H\),
        \item
            Le segment \( [x,y]\) coupe \( H\) en son milieu.
    \end{enumerate}
    La \defe{réflexion}{réflexion!par rapport à un hyperplan} \( \sigma_H\) est l'application $\sigma_H\colon \eR^n\to \eR^n $ qui à \( x\) fait correspondre ce \( y\).
\end{lemmaDef}

\begin{proof}
    Il faut vérifier que les conditions données définissent effectivement un unique point de \( \eR^n\). Soit \( H_0\) le sous-espace vectoriel parallèle à \( H\) et une base orthonormée \( \{ e_1,\ldots, e_{n-1} \}\) de \( H_0\). Nous complétons cela en une base orthonormée de \( \eR^n\) avec un vecteur \( e_n\). Si \( H=H_0+v\), quitte à décomposer \( v\) en une partie parallèle et une partie perpendiculaire à \( H\), nous avons
    \begin{equation}
        H=H_0+\lambda e_n
    \end{equation}
    pour un certain \( \lambda\).

    Une droite passant par \( x\) et perpendiculaire à \( H\) est de la forme \( t\mapsto x+te_n\). Si \( x=\sum_{i=1}^{n}x_ie_i\) alors l'unique point de cette droit à être dans \( H\) est le point tel que \(   x_ne_n+te_n=\lambda e_n   \), c'est-à-dire \( t=-x_n\). L'unique point \( y\) sur cette droite à être tel que \( [x,y ]\) coupe \( H\) en son milieu est celui qui correspond à \( t=-2x_n\).
\end{proof}

Notons au passage que cette preuve donne une formule pour \( \sigma_H\) :
\begin{equation}        \label{EQooRTWLooLPsUpY}
    \sigma_H(x)=\sum_{i=1}^{n-1}x_ie_i-x_ne_n.
\end{equation}
Il s'agit donc de changer le signe de la composante perpendiculaire à \( H\).

\begin{lemma}       \label{LEMooWYVRooQmWqvM}
    Dans cette même base si \( H_0\) est l'hyperplan parallèle à \( H\) et passant par l'origine, nous écrivons \( H=H_0+\lambda e_n\) pour un certain \( \lambda\). Alors
    \begin{equation}
        \sigma_H=\sigma_{H_0}+2\lambda e_n.
    \end{equation}
\end{lemma}

\begin{proof}
    Un élément \( x\in \eR^n\) peut être décomposé dans la base adéquate en \( x=x_H+x_ne_n\). Nous savons de la formule \eqref{EQooRTWLooLPsUpY} que
    \begin{equation}
        \sigma_H(x)=x_H-x_ne_n.
    \end{equation}
    Mais vu que \( \sigma_{H_0}(x_H)=x_H-2\lambda e_n\) nous avons
    \begin{equation}
            \sigma_{H_0}(x)+2\lambda e_n=\sigma_{H_0}(x_H+x_ne_n)+2\lambda e_N=x_H-2\lambda e_n-x_ne_n+2\lambda e_n=x_H-x_ne_n.
    \end{equation}
\end{proof}

Le lemme suivant est une généralisation du fait que tous les points de la médiatrice d'un segment sont à égale distance des deux extrémités du segment (très utile lorsqu'on étudie les triangles isocèles).
\begin{lemma}[\cite{ooZYLAooXwWjLa}]        \label{LEMooDPLYooJKZxiM}
    Soient deux points distincts \( x_0,y_0\in \eR^n\) l'ensemble \( H\subset \eR^n\) donné par
    \begin{equation}
        H=\{ x\in \eR^n\tq d(x,x_0)=d(x,y_0) \}.
    \end{equation}
    Alors \( H\) est l'hyperplan orthogonal au vecteur \( v=y_0-x_0\) et \( H\) passe par le milieu du segment \( [x_0,y_0] \).
\end{lemma}

\begin{proof}
    Nous savons que
    \begin{equation}
        d(x,x_0)^2=\langle x-x_0, x-x_0\rangle =\| x \|^2+\| x_0 \|^2-2\langle x, x_0\rangle,
    \end{equation}
    ou encore
    \begin{equation}
        \| x_0 \|^2-\| y_0 \|^2=2\langle x, x_0-y_0\rangle .
    \end{equation}
    En posant \( v=y_0-x_0\) et en considérant la forme linéaire
    \begin{equation}
        \begin{aligned}
            \beta\colon \eR^n&\to \eR \\
            x&\mapsto \langle x, v\rangle ,
        \end{aligned}
    \end{equation}
    Nous avons \( x\in H\) si et seulement si \( \beta(x)=\frac{ 1 }{2}\big( \| y_0 \|^2-\| x_0 \|^2 \big)=\lambda\). En d'autres termes, \( H=\beta^{-1}(\lambda)\). Par la proposition~\ref{PROPooAKJBooMkmsiV} la partie \( H\) est un sous-espace affine. C'est même un translaté de \( \ker(\beta)\), et comme \( \ker(\beta)\) est l'espace vectoriel des vecteurs perpendiculaires à \( v\), nous avons \( \dim(H)=\dim\big( \ker(\beta) \big)=n-1\).

    Le fait que \( H\) contienne le milieu du segment \( [x_0,y_0]\) est par définition.
\end{proof}

Pour le lemme suivant, et pour que la récurrence se passe bien nous disons que l'ensemble vide est un espace vectoriel de dimension \( -1\).
\begin{lemma}[\cite{ooYVHDooLeexeT}]       \label{LEMooJCDRooGAmlwp}
    Soit un espace euclidien \( E\).
    \begin{enumerate}
        \item       \label{ITEMooFYEDooIJZBjP}
            Si \( f\) est une isométrie de \( E\) satisfaisant
            \begin{equation}
                \dim\big( \Fix(f) \big)=n-k
            \end{equation}
            alors \( f\) peut être écrit comme composition de \( k\) réflexions hyperplanes.
        \item       \label{ITEMooJTZVooWvyfDD}
            Une isométrie de \( E\) peut être écrite sous la forme de \( \rang(f-\id)\) réflexions, mais pas moins.
        \item       \label{ITEMooUCZWooSbyPwt}
            Toute isométrie de \( \eR^n\) peut être écrite comme composition de \( n+1\) réflexions.
    \end{enumerate}
\end{lemma}

\begin{proof}
    Les deux parties importantes à démontrer sont les points \ref{ITEMooFYEDooIJZBjP} et la partie «pas moins» de \ref{ITEMooJTZVooWvyfDD}. Le reste sont des reformulations.
        \begin{subproof}
        \item[Pour \ref{ITEMooFYEDooIJZBjP}]

    Nous faisons une récurrence sur \( k\geq 0\).

    Pour l'initialisation, si \( k=0\) alors \( \dim\big( \Fix(f) \big)=n\), c'est-à-dire que \( f\) fixe tout \( \eR^n\), autant dire que \( f\) est l'identité, une composition de zéro réflexions.

    Pour la récurrence, nous supposons que le lemme est démontré jusqu'à \( k\geq 0\). Soit donc \( f\in\Isom(\eR^n)\) tel que
    \begin{equation}
        \dim\big( \Fix(f) \big)=n-(k+1).
    \end{equation}
    Vu que \( k\geq 0\), la dimension de \( \Fix(f)\) est strictement plus petite que \( n\), donc il existe un \( x_0\in \eR^n\) tel que \( f(x_0)\neq x_0\). Nous posons
    \begin{equation}
        H=\{ x\in E\tq d(x,x_0)=d\big( x,f(x_0) \big)  \}.
    \end{equation}
    Par le lemme~\ref{LEMooDPLYooJKZxiM}, ce \( H\) est l'hyperplan orthogonal à \( v=f(x_0)-x_0\) et passant par le milieu du segment \( [x_0,f(x_0)]\).

    Nous posons \( g=\sigma_H\circ f\). Vu que \( g(x_0)=\sigma_H(f(x_0))=x_0\), ce \( x_0\) est un point fixe de \( g\). Le fait que \( \sigma_H\big( f(x_0) \big)=x_0\) est vraiment la définition de l'hyperplan \( H\).

    Nous avons donc
    \begin{equation}
        x_0\in\Fix(g)\setminus\Fix(f).
    \end{equation}
    Mais nous prouvons de plus que \( \Fix(f)\subset\Fix(g)\). En effet si \( y\in Fix(f)\) alors \( y\in H\) parce que
    \begin{equation}
        d(y,x_0)=d\big( f(y),f(x_0) \big)=d\big( y, f(x_0) \big).
    \end{equation}
    Vu que \( y\in H\) nous avons \( y\in \Fix(g)\) parce que
    \begin{equation}
        g(y)=\sigma_H\big( f(y) \big)=\sigma_H(y)=y.
    \end{equation}
    Tout cela pour dire que l'ensemble \( \Fix(g)\) est \emph{strictement} plus grand que \( \Fix(f)\). Et comme ce sont des espaces affines nous pouvons parler de dimension :
    \begin{equation}
        \dim\big( \Fix(g) \big)>\dim\big( \Fix(f) \big).
    \end{equation}
    Par hypothèse de récurrence, l'application \(  g\) peut être écrite comme composition de \( k\) réflexions. Donc l'application
    \begin{equation}
        f=\sigma_H\circ g
    \end{equation}
    est une composition de \( k+1\) réflexions.
        \item[Pour \ref{ITEMooJTZVooWvyfDD}, existence]

            Ce point est une reformulation du point \ref{ITEMooFYEDooIJZBjP}. Le fait est que \( \Fix(f)=\ker(f-\id)\) parce que \( x\in\Fix(f)\) si et seulement si \( f(x)=x\) si et seulement si \( (f-\id)x=0\). Nous utilisons le théorème du rang \ref{ThoGkkffA} à l'endomorphisme \( f-\id\) :
            \begin{equation}
                \dim\big( \Fix(f) \big)=\dim\big( \ker(f-\id) \big)=\dim(E)-\rang(f-\id).
            \end{equation}
            En remplaçant par les valeurs :
            \begin{equation}
                n-k=n-\rang(f-\id).
            \end{equation}
            Or le point \ref{ITEMooFYEDooIJZBjP} donnait \( f\) comme composée de \( n-k\) réflexions. Donc \( f\) est composée de \( \rang(f-\id)\) réflexions.
        \item[Pour \ref{ITEMooJTZVooWvyfDD}, «pas moins»]

            Supposons que \( f=\sigma_1\circ\ldots \circ \sigma_r\) où \( \sigma_i\) est la réflexion de l'hyperplan \( H_i\). Nous devons prouver que \( r\geq \rang(f-\id)\). Nous avons
            \begin{equation}
                \bigcap_{i=1}^rH_i\subset \ker(f-\id).
            \end{equation}
            D'autre part, la proposition \ref{PROPooRCLNooJpIMMl} nous donne \( \dim\bigcap_iH_i\geq n-r\). Donc
            \begin{equation}
                n-r\leq \dim\big( \bigcap_{i=1}^r \big)\leq\dim\big( \ker(f-\id) \big)=n-\rang(f-\id).
            \end{equation}
            Donc \( n-r\leq n-\rang(f-\id)\) ou encore
            \begin{equation}
                r\geq \rang(f-\id).
            \end{equation}

        \item[Pour \ref{ITEMooUCZWooSbyPwt}]
    La première partie de ce théorème n'est rien d'autre que le lemme~\ref{LEMooJCDRooGAmlwp} parce que le pire cas est celui où le fixateur de \( f\) est réduit à l'ensemble vide, et dans ce cas l'application \( f\) est une composition de \( n+1\) réflexions.
        \end{subproof}
\end{proof}

\begin{lemma}       \label{LEMooMCVKooKzmlAg}
    Soit un hyperplan \( H\) et un vecteur \( v\) de \( \eR^n\). Nous avons
    \begin{equation}
        \tau_v\circ \sigma_H\circ\tau_v^{-1}=\sigma_{\tau_v(H)}.
    \end{equation}
\end{lemma}

\begin{proof}
    Pour ce faire nous considérons une base adaptée. Les vecteurs \( \{ e_1,\ldots, e_{n-1} \}\) forment une base orthonormée de \( H_0\) et \( e_n\) complète en une base orthonormée de \( \eR^n\). Soit \( H_0\) l'hyperplan parallèle à \( H\) et passant par l'origine; nous avons, pour un certain \( \lambda\in \eR\),
    \begin{equation}
        H=H_0+\lambda e_n
    \end{equation}
    D'un autre côté, le vecteur \( v\) peut être décomposé en \( v=v_1+v_2\) où \( v_1\perp H\) et \( v_2\parallel H\). Alors
    \begin{equation}
        \tau_v(H)=H+v=H+v_2=H_0+\lambda e_n+v_2.
    \end{equation}
    Nous pouvons maintenant utiliser le lemme~\ref{LEMooWYVRooQmWqvM} pour exprimer la transformation \( \sigma_{\tau_v(H)}\) :
    \begin{equation}        \label{EQooNYKFooXprXav}
        \sigma_{\tau_v(H)}(x)=\sigma_{H_0}(x)+ 2\lambda e_n+2v_2
    \end{equation}

    Mais d'autre part,
    \begin{equation}
        (\tau_v\circ \sigma_H\circ\tau_{v}^{-1})(x)=v+\sigma_H(x-v)=v+\sigma_{H_0}(x-v)+2\lambda e_n.
    \end{equation}
    Vue la décomposition de \( v=v_1+v_2\) nous avons \( \sigma_{H_0}(v)=-v_1+v_2\) et donc
    \begin{equation}        \label{EQooGOHEooALPRFB}
        (\tau_v\circ \sigma_H\circ\tau_{v}^{-1})(x)= v+  \sigma_{H_0}(x)+v_1-v_2+2\lambda e_n=\sigma_{H_0}+2v_1+2\lambda e_n.
    \end{equation}
    Les expressions \eqref{EQooNYKFooXprXav} et \eqref{EQooGOHEooALPRFB} coïncident, d'où l'égalité recherchée.
\end{proof}

\begin{theorem}[\cite{ooZYLAooXwWjLa}]      \label{THOooWBIYooCtWoSq}
    Une isométrie de \( \eR^n\) préserve l'orientation si et seulement si est elle composition d'un nombre pair de réflexions.
\end{theorem}

\begin{proof}
    Nous définissons
    \begin{equation}
        \begin{aligned}
            \epsilon\colon \Isom(\eR^n)&\to \{ \pm 1 \} \\
            \tau_v\circ \alpha&\mapsto \det(\alpha)
        \end{aligned}
    \end{equation}
    où nous nous référons à la décomposition unique d'un élément de \( \Isom(\eR^n)\) sous la forme \( \tau_v\circ \alpha\) avec \( \alpha\in O(n)\) donnée par le théorème~\ref{THOooQJSRooMrqQct}\ref{ITEMooEWSIooNKzRxB}.

    Le noyau de \( \epsilon\) est alors la partie
    \begin{equation}
        \ker(\epsilon)=\eR^n\times_{\AD}\SO(n).
    \end{equation}
    Une isométrie \( f\) préserve l'orientation si et seulement si \( \epsilon(f)=1\). Vu que toutes les isométries sont des compositions de réflexions (première partie), il nous suffit de montrer que \( \epsilon(\epsilon_H)=-1\) pour qu'une isométrie préserve l'orientation si et seulement si elle est composition d'un nombre pair de réflexions.

    Nous commençons par prouver que pour tout vecteur \( v\), \( \epsilon\big( \sigma_H \big)=\epsilon\big( \sigma_{\tau_v(H)} \big)\). Pour cela nous utilisons le lemme~\ref{LEMooMCVKooKzmlAg} et le fait que \( \epsilon\) est un homomorphisme :
    \begin{equation}
        \epsilon(\sigma_{\tau_v(H)})=\epsilon(\tau_v)\epsilon(\sigma_H)\epsilon(\tau_v^{-1})=\epsilon(\sigma_H)
    \end{equation}
    parce que la partie linéaire d'une translation est l'identité (et donc \( \epsilon(\tau_v)=1\) pour tout \( v\)).

    Nous avons donc \( \epsilon(\sigma_H)=\epsilon(\sigma_{H_0})\). En ce qui concerne \( \sigma_{H_0}\), dans la base adaptée la matrice est
    \begin{equation}
        \sigma_{H_0}=\begin{pmatrix}
             1   &       &       &       \\
                &   \ddots    &       &       \\
                &       &   1    &       \\
                &       &       &   -1
         \end{pmatrix},
    \end{equation}
    dont le déterminant est \( -1\).
\end{proof}

Pour en savoir plus sur le groupe des isométries, il faut lire le théorème de Cartan-Dieudonné dans \cite{JGAdTA}.

%---------------------------------------------------------------------------------------------------------------------------
\subsection{Groupes finis d'isomorphismes}
%---------------------------------------------------------------------------------------------------------------------------

\begin{definition}
    Si \( X\) est une partie finie de \( \eR^n\), le \defe{barycentre}{barycentre!cas vectoriel} de \( X\) est le point
    \begin{equation}
        B_X=\frac{1}{ | X | }\sum_{x\in X}x
    \end{equation}
    où \( | X |\) est le cardinal de \( X\).
\end{definition}
Cela est à mettre en relation avec la définition dans le cadre affine~\ref{LemtEwnSH}.

\begin{lemma}[\cite{ooZYLAooXwWjLa}]
    Soit une partie finie \( X\) de \( \eR^n\) et une application affine \( f\in\Aff(\eR^n)\). Alors
    \begin{equation}
        f(B_X)=B_{f(X)}.
    \end{equation}
\end{lemma}

\begin{proof}
    Nous savons que toute application affine est une composée de translation et d'une application linéaire : \( f=\tau_v\circ g\) avec \( v\in \eR^n\) et \( g\in \GL(n,\eR)\). Nous vérifions le résultat séparément pour \( \tau_v\) et pour \( g\).

    D'une part,
    \begin{equation}
        B_{\tau_v(X)}=\frac{1}{ | \tau_v(X) | }\sum_{y\in \tau_v(X)}y=\frac{1}{ | X | }\sum_{x\in X}(x+v)=B_x+\frac{1}{ | X | }\sum_{x\in X}v=B_x+v=\tau_v(B_X).
    \end{equation}
    Nous avons utilisé le fait que \( X\) et \( \tau_v(X)\) possèdent le même nombre d'éléments, ainsi que le fait d'avoir une somme de \( | X |\) termes tous égaux à \( v\).

    D'autre part,
    \begin{equation}
        B_{g(X)}=\frac{1}{ | X | }\sum_{x\in X}g(x)=g\big( \frac{1}{ |X | }\sum_{x\in X}x \big)=g(B_X)
    \end{equation}
    où nous avons utilisé la linéarité de \( g\) dans tous ses retranchements.
\end{proof}

\begin{proposition}     \label{PROPooLAEBooWdcBoe}
    Points fixes d'un sous-groupe.
    \begin{enumerate}
        \item
            Soit \( H\) un sous-groupe finie de \( \Isom(\eR^n)\). Alors il existe \( v\in \eR^n\) tel que \( f(v)=v\) pour tout \( f\in H\).
        \item
            Si \( H\) est un sous-groupe de \( \Isom(\eR^n)\) n'acceptant pas de points fixes, alors il est infini.
    \end{enumerate}
\end{proposition}

\begin{proof}
    Le groupe \( H\) agit sur \( \eR^n\), et si \( x\in \eR^n\) nous pouvons considérer son orbite \( Hx\), qui est une partie finie de \( \eR^n\). Considérons son barycentre
    \begin{equation}
        v=B_{Hx}
    \end{equation}
    Soit \( f\in H\). Alors \( f(v)=f(B_{Hx})=B_{f(Hx)}=B_{Hx}=v\), donc \( v\) est fixé par \( H\).

    La seconde affirmation n'est rien d'autre que la contraposée de la première.
\end{proof}

\begin{proposition}     \label{PROPooEUFIooDUIYzi}
    À propos de groupes finis d'isométries.
    \begin{enumerate}
        \item
            Tout sous-groupe finie de \( \Isom(\eR^n)\) est isomorphe à un sous-groupe fini de \( \gO(n)\).
        \item
            Tout sous-groupe fini de \( \Isom^+(\eR^n)\) est isomorphe à un sous-groupe fini de \( \SO(n)\).
    \end{enumerate}
\end{proposition}

\begin{proof}
    Soit \( H\) un sous-groupe fini de \( \Isom(\eR^n)\) et \( v\in \eR^n\) un élément fixé par \( H\) (comme garantit par la proposition~\ref{PROPooLAEBooWdcBoe}). Nous posons
    \begin{equation}
        \begin{aligned}
            \phi\colon H&\to \Isom(\eR^n) \\
            f&\mapsto \tau_v^{-1}\circ f\circ \tau_v.
        \end{aligned}
    \end{equation}

    \begin{subproof}
        \item[\( \phi\) est un homomorphisme]
            Les opération du type \( \phi=\AD(\tau_v)\) sont toujours des homomorphismes.
        \item[\( \phi\) consiste à extraire la partie linéaire]
            Si \( f=\tau_w\circ g\) alors
            \begin{subequations}
                \begin{align}
                    \phi(f)(x)&=(\tau_{-v}\circ\tau_w\circ g\circ\tau_v)(x)\\
                    &=\tau_{w-v}(   g(x)+g(v)  )\\
                    &=g(x)+g(v)-v+w
                \end{align}
            \end{subequations}
            Mais \( g(v)+w=f(v)\) et nous savons que \( f(v)=v\). Donc il ne reste que \( \phi(f)(x)=g(x)\).
        \item[\( \phi\) est injective]
            Si \( f=\tau_w\circ g\) vérifie \( \phi(f)=\id\), il faut en particulier que \( g=\id\). Mais \( H\) est fini et ne peut donc pas contenir de translations non triviales. Donc \( w=0\) et \( f=\id\).
    \end{subproof}
    Donc \( \phi\) est une injection à valeur dans les transformations linéaires de \( \Isom(\eR^n)\). Autrement dit, \( \phi\) est un isomorphisme entre \( H\) et son image, laquelle image est dans \( \gO(n)\).

    En ce qui concerne la seconde partie, si \( f\in\Isom^+(\eR^n)\), alors \( \phi(f)\) y est aussi, tout en étant linéaire. Donc \( \phi(f)\in \SO(n)\).
\end{proof}

L'extraction de la partie linéaire est injective ? Certe c'est prouvé, mais on peut se demander ce qu'il se passe si \( H\) contient deux éléments qui ont la même partie linéaire. Cela n'est pas possible parce si \( f_1=\tau_{w_1}\circ g\) et \( f_2=\tau_{w_2}\circ g\) sont dans \( H\) alors \( f_1f_2^{-1}=\tau_{w_1+w_2}\) est également dans \( H\), ce qui n'est pas possible si \( H\) est fini.

\begin{definition}[Groupe de symétrie d'une partie de \( \eR^n\)\cite{ooZYLAooXwWjLa}]
    Si \( Y\) est une partie de \( \eR^n\), nous définissons le \defe{groupe des symétries}{groupe!des symétries} de \( Y\) par
    \begin{equation}
        \Sym(Y)=\{ f\in\Isom(\eR^n)\tq f(Y)=Y \}.
    \end{equation}
    Nous définissons aussi le groupe des symétries propres de \( Y\) par
    \begin{equation}
        \Sym^+(Y)=\{ f\in\Isom^+(\eR^n)\tq f(Y)=Y \}.
    \end{equation}
\end{definition}

\begin{theorem}[\cite{ooZYLAooXwWjLa}]
    Soit \( Y\subset \eR^2\) tel que le groupe \( \Sym^+(Y)\) soit fini d'ordre \( n\). Alors c'est un groupe cyclique d'ordre \( n\).

    Si \( \Sym^+(Y)\) est fini, alors \( \Sym(Y)\) est soit cyclique d'ordre \( n\) soit isomorphe au groupe diédral d'ordre \( 2n\).
\end{theorem}

\begin{proof}
    Nous savons déjà par la proposition~\ref{PROPooEUFIooDUIYzi} que \( \Sym^+(Y)\) est isomorphe à un sous-groupe \( H^+\) d'ordre \( n\) de \( \SO(2)\). Vérifions que ce groupe est cyclique. Si \( n=1\), c'est évident. Si \( n\geq 2\) alors nous savons que \( H^+\) est constitué de rotations d'angles dans \( \mathopen[ 0 , 2\pi \mathclose[\) et vu que c'est un ensemble fini, il possède une rotation d'angle minimal (à part zéro). Notons \( \alpha_0\) cet angle.

        Nous montrons que \( H^+\) est engendré par la rotation d'angle \( \alpha_0\). Soit une rotation d'angle \( \alpha\). Étant donné que \( \alpha_0<\alpha\) nous pouvons effectuer la division euclidienne\footnote{Théorème~\ref{ThoDivisEuclide}.} de \( \alpha\) par \( \alpha_0\) et obtenir
        \begin{equation}
            \alpha=k\alpha_0+\beta
        \end{equation}
        avec \( \beta<\alpha_0\). Mézalors \( R(\beta)=R(\alpha)R(\alpha_0)^{-k}\) est également un élément du groupe. Cela contredit la minimalité dès que \( \beta\neq 0\). Avoir \( \beta=0\) revient à dire que \( \alpha\) est un multiple de \( \alpha_0\), ce qui signifie que le groupe \( H^+\) est cyclique engendré par \( \alpha_0\).

        Notons au passage que nous avons automatiquement \( \alpha_0=\frac{ 2\pi }{ n }\) parce qu'il faut \( R(\alpha_0)^n=\id\). Nous avons prouvé que \( \Sym^+(Y) \) est cyclique d'ordre \( n\).

        Nous étudions maintenant le groupe \( \Sym(Y)\). Par la proposition~\ref{PROPooEUFIooDUIYzi} nous avons un homomorphisme injectif
        \begin{equation}
            \phi\colon \Sym(Y)\to \gO(2),
        \end{equation}
        et en posant \( H=\phi\big( \Sym(Y) \big)\) nous avons un isomorphisme de groupes \( \phi\colon \Sym(Y)\to H\). Nous savons aussi que ce \( \phi\) se restreint en
        \begin{equation}
            \phi\colon   \Sym^+(Y) \to H^+\subset\SO(2)
        \end{equation}
        où \( H^+=\phi\big( \Sym^+(Y) \big)=H\cap\SO(2)\). Le groupe \( H^+\) est cyclique et est engendré par la rotation \( R(2\pi/n)\).

        Supposons un instant que \( H\subset \SO(2)\). Alors nous avons \( H=H^+\) et \( \phi\) est un isomorphisme entre \( \Sym(Y)\) et le groupe cyclique engendré par \( R(2\pi/n)\).

        Nous supposons à présent que \( H\) n'est pas un sous-ensemble de \( \SO(2)\). Quelles sont les isométries de \( \eR^2\) qui ne sont pas de déterminant \( 1\) ? Il faut regarder dans le théorème~\ref{THOooVRNOooAgaVRN} quelles sont les isométries contenant un nombre impair de réflexions. Ce sont les réflexions et les réflexions glissées. Or il ne peut pas y avoir de réflexions glissées dans un groupe fini parce que si \( f\) est une réflexion glissée, tous les \( f^k\) sont différents.

        Nous en déduisons que si \( H\) n'est pas inclus dans \( \SO(2)\), il contient une réflexion que nous nommons \( \sigma\). Nous allons en déduire que \( H\simeq H^+\times_{\AD}C_2\) où \( C_2=\{ \id,\sigma \}\). Si \( h\in H\) nous pouvons écrire
        \begin{equation}
            h=(h\sigma^{\epsilon})\sigma^{\epsilon}
        \end{equation}
        pour n'importe quelle valeur de \( \epsilon\), et en particulier pour \( \epsilon=\pm 1\).

        Si \( h\in \SO(2)\) alors nous écrivons \( h=h\epsilon^{0}\) et si \( h\notin\SO(2)\) nous écrivons \( h=(h\sigma)\sigma\). Vu que \( h\sigma\in\SO(2)\), cette dernière écriture est encore de la forme \( \SO(2)\times C_2\). Quoi qu'il en soit tout élément de \( H\) s'écrit comme un produit
        \begin{equation}
            H=H^+C_2.
        \end{equation}
        Cette décomposition est unique parce que si \( h_1c_1=h_2c_2\) alors \( h_2^{-1}h_1=c_2c_1^{-1}\), et comme \( h_2^{-1}h_1\in H^+\) nous avons \( c_2c_1^{-1}\in H^+\) et donc \( c_1=c_2\). Partant nous avons aussi \( h_1=h_2\). Pour avoir le produit semi-direct il faut encore montrer que \( \AD(C_2)H^+\subset H^+\). Le seul cas à vérifier est \( \AD(\sigma)H^+\subset H^+\). Vu que les éléments de \( H^+\) sont caractérisés par le fait d'avoir un déterminant positif, nous avons
        \begin{equation}
            \AD(\sigma)R(\alpha)=\sigma R(\alpha)\sigma^{-1}\in H^+.
        \end{equation}
\end{proof}

\begin{remark}
    Tout ceci est cohérent avec le théorème de Burnside~\ref{ThooJLTit} parce que le sous-groupe fini de \( \SO(n)\) engendré par la rotation \( R(2\pi/n)\) est un groupe d'exposant fini, à savoir que si \( h\) est dans ce groupe, \( h^n=\id\).
\end{remark}

%---------------------------------------------------------------------------------------------------------------------------
\subsection{Groupe diédral}
%---------------------------------------------------------------------------------------------------------------------------
\label{subsecHibJId}

%///////////////////////////////////////////////////////////////////////////////////////////////////////////////////////////
\subsubsection{Définition et générateurs : vue géométrique}
%///////////////////////////////////////////////////////////////////////////////////////////////////////////////////////////

\begin{definition}  \label{DEFooIWZGooAinSOh}
    Le \defe{groupe diédral}{groupe!diédral} \( D_n\)\nomenclature[R]{\( D_n\)}{groupe diédral} est le groupe des isométries de \( \eR^2\) laissant invariant un polygone régulier à \( n\) côtés.
\end{definition}
Le groupe diédral peut être vu comme le stabilisateur de l'ensemble
\begin{equation}
    \{  e^{2ik\pi/n},k=0,\ldots, n-1 \}
\end{equation}
dans le groupe des isométries affines de \( \eC^*\).
\index{groupe!agissant sur un ensemble!diédral}
\index{groupe!en géométrie}
\index{groupe!fini!diédral}
\index{groupe!permutation!diédral}
% TODO : prouver que les racines de l'unité forment un polygone régulier.

Si \( f\in D_n\), alors \( f( e^{2ik\pi/n}) \) doit être l'un des \(  e^{2ik'\pi/n}\), et vu que \( f\) conserve les longueurs dans \( \eC\), nous devons avoir
\begin{equation}
    1=d(0, e^{2ik\pi/n})=d\big( f(0), e^{2ik'\pi/n} \big).
\end{equation}
Donc \( f(0)\) est à l'intersection de tous les cercles de rayon \( 1\) centrés en les \(  e^{2ik\pi/n}\), ce qui montre que \( f(0))0\) (dès que \( n\geq 3\)). Par conséquent notre étude du groupe diédral ne doit prendre en compte que les isométries vectorielles de \( \eR^2\). En d'autres termes
\begin{equation}
    D_n\subset O(2,\eR).
\end{equation}

\begin{proposition}[\cite{tzHydF}]
    Le groupe \( D_n\) contient un sous-groupe cyclique d'ordre \( 2\) et un sous-groupe cyclique d'ordre \( n\).
\end{proposition}

\begin{proof}
    Si \( s\) est la réflexion d'axe \( \eR\), alors \( s\) est d'ordre \( 2\). De plus \( s\) est bien dans \( D_n\) parce que
    \begin{equation}    \label{EqSUshknP}
        s\big(  e^{2ki\pi/n} \big)= e^{2(n-k)i\pi/n}.
    \end{equation}

    De la même façon, la rotations d'angle \(2\pi/n\), que l'on note \( r\), agit sur les racines de l'unité et engendre un le groupe d'ordre \( n\) des rotations d'angle \(2 k\pi/n\).
\end{proof}

Notons que la conjugaison complexe ne fait pas spécialement partie du groupe \( D_n\). En effet pour \( n=3\) par exemple les points fixes sont \( A_1=(1,0)\), \( A_2=(-\frac{ 1 }{2},\frac{ \sqrt{3} }{2})\) et \( A_3=(\frac{ 1 }{2},-\frac{ \sqrt{3} }{2})\). La conjugaison complexe envoie évidemment \( A_1\) sur \( A_1\), mais pas du tout \( A_2\) sur \( A_3\).
%TODO : un dessin du triangle équilatéral serait pas mal ici.

\begin{proposition}[\cite{tzHydF}]
    Nous avons \( (sr)^2=\id\).
\end{proposition}

\begin{proof}
    Si \( z^n=1\), alors
    \begin{equation}
        (srsr)z=srs e^{2 i\pi/n}z=sr\big( e^{-2\pi i/n\bar z}\big)=s\bar z=z.
    \end{equation}
\end{proof}

\begin{proposition}[\cite{tzHydF}] \label{PropLDIPoZ}
    Le groupe diédral \( D_n\) est engendré par \( s\) et \( r\). De plus tous les éléments de \( D_n\) s'écrivent sous la forme \( s\circ r^m\).
\end{proposition}
\index{groupe!diédral!générateurs (preuve)}
\index{racine!de l'unité}
\index{géométrie!avec nombres complexes}
\index{géométrie!avec des groupes}
\index{isométrie!de l'espace euclidien \( \eR^2\)}

\begin{proof}
    Nous considérons les points \( A_0=1\) et \( A_k= e^{2ki\pi/n}\) avec \( k\in\{ 1,\ldots, n-1 \}\). Par convention, \( A_n=A_0\). L'action des éléments \( s\) et \( r\) sur ces points est
    \begin{subequations}
        \begin{align}
            r(A_k)&=A_{k+1}\\
            s(A_k)&=A_{n-k}.
        \end{align}
    \end{subequations}
    Cette dernière est l'équation \eqref{EqSUshknP}.

    Soit \( f\in D_n\). Étant donné que c'est une isométrie de \( \eR^2\) avec un point fixe (le point \( 0\)), \( f\) est soit une rotation soit une réflexion.
    %TODO : il faut démontrer ce point et mettre un lien vers ici.

    Supposons pour commencer que un des \( A_k\) est fixé par \( f\). Dans ce cas \( f\) a deux points fixes : \( O\) et \( A_k\) et est donc la réflexion d'axe \( (OA_k)\). Dans ce cas, nous avons \( f=s\circ r^{n-2k}\). En effet
    \begin{equation}
        s\circ r^{n-2k}(A_k)=s(A_{k+n-2k})=s(A_{n-k})=A_k.
    \end{equation}
    Donc \( O\) et \( A_k\) sont deux points fixes de l'isométrie \( f\); donc \( f\) est bien la réflexion sur le bon axe.

    Nous passons à présent au cas où \( f\) ne fixe aucun des \( A_k\).
    \begin{enumerate}
        \item
            Supposons que \( f\) soit une rotation. Si \( f(A_k)=A_m\), alors l'angle de la rotation est
            \begin{equation}
                \frac{ 2(m-k)\pi }{ n },
            \end{equation}
            et donc \( f=r^{m-k}\), qui est de la forme demandée.
        \item
            Supposons à présent que \( f\) soit une réflexion d'axe \( \Delta\). Cette fois, \( \Delta\) ne passe par aucun des points \( A_k\), par contre \( \Delta\) passe par \( 0\). Nous commençons par montrer que \( \Delta\) doit être la médiatrice d'un des côtés \( [A_p,A_{p+1}]\) du polygone. Vu que \( \Delta\) passe par \( O\) et n'est aucune des droites \( (OA_k)\), cette droite passe par l'intérieur d'un des triangles \( OA_pA_{p+1}\) et intersecte donc le côté correspondant.

            Notre tâche est de montrer que \( \Delta\) coupe \( [A_p,A_{p+1}]\) en son milieu. Dans ce cas, \( \Delta\) sera automatiquement perpendiculaire parce que le triangle \( OA_pA_{p+1}\) est isocèle en \( O\). Nommons \( l\) la longueur des côtés du polygone, \( P=\Delta\cap[A_p,A_{p+1}]\), \( x=d(A_p,P)\) et \( \delta=d(A_p,\Delta)\). Vu que \( f\) est la symétrie d'axe \( \Delta\), nous avons aussi \( d\big( f(A_p),\Delta \big)=\delta\) et \( d\big( A_p,f(A_p) \big)=2\delta\). D'autre part, par la définition de la distance, \( \delta<x\). Si \( x<\frac{ l }{2}\), alors \( \delta<\frac{ \delta }{2}\) et donc \( d\big( A_p,f(A_p) \big)<l\). Or cela est impossible parce que le polygone ne possède aucun sommet à distance plus courte que \( l\) de \( A_p\).

            De la même manière si \( x>\frac{ l }{2}\), nous raisonnons avec \( A_{p+1}\) pour obtenir une contradiction. Nous en concluons que la seule possibilité est \( x=\frac{ l }{2}\), et donc \( f(A_p)=A_{p+1}\). Montrons alors que \( f=s\circ r^{n-2p-1}\). Il faut montrer que c'est une réflexion qui envoie \( A_p\) sur \( A_{p+1}\). D'abord c'est une réflexion parce que
            \begin{equation}
                \det(sr^{n-2p-1})=\det(s)\det(r^{n-2p-1})=-1
            \end{equation}
            parce que \( \det(s)=-1\) alors que \( \det(r^k)=1\) parce que \( r\) est une rotation dans \( \SO(2)\). Ensuite nous avons
            \begin{equation}
                s\circ r^{n-2p-1}(A_p)=s(A_{p+n-2p-1})=s(A_{n-p-1})=A_{n-(n-p-1)}=A_{p+1}.
            \end{equation}

            Donc \( s\circ r^{n-2p-1}\) est bien une réflexion qui envoie \( A_p\) sur \( A_{p+1}\).

    \end{enumerate}
\end{proof}

\begin{corollary}   \label{CorWYITsWW}
La liste des éléments de \( D_n\) est
\begin{equation}
    D_n=\{ 1,r,\ldots, r^{n-1},s,sr,\ldots, sr^{n-1} \}
\end{equation}
et \( | D_n |=2n\).
\end{corollary}

\begin{proof}
    Nous savons par la proposition~\ref{PropLDIPoZ} que tous les élément de \( D_n\) s'écrivent sous la forme \( r^k\) ou \( sr^k\). Vu que \( r\) est d'ordre \( n\), il ne faut considérer que \( k\in\{ 1,\ldots, n-1 \}\). Les éléments \( 1\), \( r\),\ldots, \( r^{n-1}\) sont tous différents, et sont (pour des raisons de déterminant) tous différents des \( sr^k\). Les isométries \( sr^k\) sont toutes différentes entre elles pour essentiellement la même raison :
    \begin{equation}
        sr^k(A_p)=s(A_{p+k})=A_{n-p+k}
    \end{equation}
    donc si \( k\neq k'\), \( sr^k(A_p)\neq sr^{k'}(A_p)\). La liste des éléments de \( D_n\) est donc
    \begin{equation}
        D_n=\{ 1,r,\ldots, r^{n-1},s,sr,\ldots, sr^{n-1} \}
    \end{equation}
    et donc \( | D_n |=2n\).
\end{proof}

\begin{example}     \label{EXooHNYYooUDsKnm}
    Nous considérons le carré \( ABCD\) dans \( \eR^2\) et nous cherchons les isométries de \( \eR^2\) qui laissent le carré invariant. Nous nommons les points comme sur la figure~\ref{LabelFigIsomCarre}. La symétrie d'axe vertical est nommée \( s\) et la rotation de \( 90\) degrés est notée \( r\).
    \newcommand{\CaptionFigIsomCarre}{Le carré dont nous étudions le groupe diédral.}
    \input{auto/pictures_tex/Fig_IsomCarre.pstricks}

    Il est facile de vérifier que toutes les symétries axiales peuvent être écrites sous la forme \( r^is\). De plus le groupe engendré par \( s\) agit sur le groupe engendré par \( r\) parce que
    \begin{equation}
        (srs^{-1})(A,B,C,D)=sr(B,A,D,C)=s(A,D,C,B)=(B,C,D,A),
    \end{equation}
    c'est-à-dire \( srs^{-1}=r^{-1}\). Nous sommes alors dans le cadre du corollaire~\ref{CoroGohOZ} et nous pouvons écrire que
    \begin{equation}
        D_4=\gr(r)\times_{\sigma}\gr(s).
    \end{equation}
\end{example}

%///////////////////////////////////////////////////////////////////////////////////////////////////////////////////////////
\subsubsection{Générateurs : vue abstraite}
%///////////////////////////////////////////////////////////////////////////////////////////////////////////////////////////

\begin{normaltext}      \label{NORMooCCUEooRRENed}
    Nous allons montrer que \( D_n\) peut être décrit de façon abstraite en ne parlant que de ses générateurs. Nous considérons un groupe \( G\) engendré par des éléments \( a\) et \( b\) tels que
    \begin{enumerate}
        \item
            \( a\) est d'ordre \( 2\),
        \item
            \( b\) est d'ordre \( n\) avec \( n\geq 3\),
        \item
            \( abab=e\).
    \end{enumerate}
    Nous allons prouver que ce groupe doit avoir la même liste d'éléments que celle du corollaire~\ref{CorWYITsWW}.
\end{normaltext}

\begin{proposition}[\cite{tzHydF}]
    Le groupe \( G\) n'est pas abélien.
\end{proposition}

\begin{proof}
    Nous savons que \( abab=e\), donc \( abab^{-1}=b^{-2}\), mais \( b^{-2}\neq e\) parce que \( b\) est d'ordre \( n>2\). Donc \( abab^{-1}\neq e\). En manipulant un peu :
    \begin{equation}
        e\neq abab^{-1}=(ab)(ba^{-1})^{-1}=(ab)(ba)^{-1}
    \end{equation}
    parce que \( a^{-1}=a\). Donc \( ab\neq ba\).
\end{proof}

\begin{lemma}[\cite{tzHydF}]        \label{LemKKXdqdL}
    Pour tout \( k\) entre \( 1\) et \( n-1\) nous avons
    \begin{equation}
        \AD(a)b^k=ab^ka^{-1}=ab^ka=b^{-k}.
    \end{equation}
\end{lemma}

\begin{proof}
    Nous faisons la démonstration par récurrence. D'abord pour \( k=1\), nous devons avoir \( aba=b^{-1}\), ce qui est correct parce que par construction de \( G\) nous avons \( abab=e\). Ensuite nous supposons que le lemme tient pour \( k\) et nous regardons ce qu'il se passe avec \( k+1\) :
    \begin{equation}
            ab^{k+1}ba=ab^kba=\underbrace{ab^ka}_{b^{-k}}\underbrace{aba}_{b^{-1}}=b^{-k}b^{-1}=b^{-(k+1)}.
    \end{equation}
\end{proof}

\begin{proposition}     \label{PROPooVQARooWuKHMZ}
    L'élément \( a\) n'est pas une puissance de \( b\).
\end{proposition}

\begin{proof}
    Supposons le contraire : \( a=b^k\). Dans ce cas nous aurions
    \begin{equation}
        e=(ab)(ab)=b^{k+1}b^{k+1}=b^{2k+2}=b^{2k}b^2=a^2b^2=b^2,
    \end{equation}
    ce qui signifierait que \( b\) est d'ordre \( 2\), ce qui est exclu par construction.
\end{proof}

\begin{proposition}[\cite{tzHydF}]      \label{PROPooEPVGooQjHRJp}
    La liste des éléments de \( G\) est donnée par
    \begin{equation}
        G=\{ 1,b,\cdots,b^{n-1},a,ab,\ldots, ab^{n-1} \}=\{ a^{\epsilon}b^k\}_{\substack{\epsilon=0,1\\k=0,\ldots, n-1}}
    \end{equation}
    Les éléments de ces listes sont distincts.
\end{proposition}

\begin{proof}
    Étant donné que \( a\) n'est pas une puissance de \( b\), les éléments \( 1\), \( a\), \( b\),\ldots, \( b^{n-1}\) sont distincts. De plus si \( k\) et \( m=k+p\) sont deux éléments distincts de \( \{ 1,\ldots, n-1 \}\), nous avons \( ab^k\neq ab^m\) parce que si \( ab^k=ab^{k+p}\), alors \( a=ab^p\) avec \( p<n\), ce qui est impossible. Pour la même raison, \( ab^k\neq e\), et \( ab^k\neq b^m\).

    Au final les éléments \( 1,a,b,\ldots, b^{n-1},ab,\ldots, ab^{n-1}\) sont tous différents. Nous devons encore voir qu'il n'y en a pas d'autres.

    Par définition le groupe \( G\) est engendré par \( a\) et \( b\), donc tout élément \( x\in G\) s'écrit $x=a^{m_1}b^{k_1}\ldots a^{m_r}b^{k_r}$ pour un certain \( r\) et avec pour tout \( i\), \( k_i\in\{ 1,\ldots, n-1 \}\) (sauf \( k_r\) qui peut être égal à zéro) et \( m_i=1\), sauf \( m_1\) qui peut être égal à zéro. Donc
    \begin{equation}
        x=a^mb^{k_1}ab^{k_2}a\ldots b^{k_{r-1}}ab^{k_r}
    \end{equation}
    où \( m\) et \( k_r\) peuvent éventuellement être zéro. En utilisant le lemme~\ref{LemKKXdqdL} sous la forme \( b^{k_i}a=ab^{-k_i}\), quitte à changer les valeurs des exposants, nous pouvons passer tous les \( a \) à gauche et tous les \( b\) à droite pour finir sous la forme \( x=a^kb^m\).

    Donc non, il n'existe pas d'autres éléments dans \( G\) que ceux déjà listés.
\end{proof}

\begin{lemma}[\cite{MonCerveau}]        \label{LemooNFRIooPWuikH}
    Tout élément de \( G\) s'écrit de façon unique sous la forme \( a^{\epsilon}b^k\) ou \( b^ka^{\epsilon}\) avec \( \epsilon=0,1\) et \( k=0,\ldots, n-1\).
\end{lemma}

\begin{proof}
    Nous commençons par la forme \( a^{\epsilon}b^k\). L'existence est la proposition~\ref{PROPooEPVGooQjHRJp}. Pour l'unicité nous supposons \( a^{\epsilon}b^k=a^{\sigma}b^l\) et nous décomposons en \( 4\).
    \begin{subproof}
        \item[\( \epsilon=0\), \( \sigma=0\)]
            Alors \( b^k=b^l\). Mais \( b\) étant d'ordre \( n\) et \( k,l\) étant égaux au maximum à \( n-1\), cette égalité implique \( k=l\).
        \item[\( \epsilon=0\), \( \sigma=1\)]
            Alors \( b^k=ab^l\), ce qui donne \( a=b^{k-l}\), ce qui est interdit par la proposition~\ref{PROPooVQARooWuKHMZ}.
        \item[\( \epsilon=1\), \( \sigma=0\)]
            Même problème.
        \item[\( \epsilon=1\), \( \sigma=1\)]
            Encore une fois \( b^k=b^l\) implique \( k=l\).
    \end{subproof}
    En ce qui concerne la forme \( b^ka^{\epsilon}\), l'existence est à montrer. Soit l'élément \( g=a^{\epsilon}b^k\) et cherchons à le mettre sous la forme \( b^la^{\sigma}\). Si \( \epsilon=0\) c'est évident. Sinon \( \epsilon=1\) et nous avons par le lemme~\ref{LemKKXdqdL}
    \begin{equation}
        ab^k=b^{-k}a^{-1}=b^{-k}b^na=b^{-k}a.
    \end{equation}
    En ce qui concerne l'unicité, nous refaisons \( 4\) cas pour \( b^ka^{\epsilon}=b^la^{\sigma}\) comme précédemment et ils se traitement exactement comme précédemment.
\end{proof}

\begin{theorem}     \label{THOooYITHooTNTBuG}
    Les groupes \( G\) et \( D_n\) sont isomorphes.
\end{theorem}

\begin{proof}
        Nous utilisons l'application
    \begin{equation}
        \begin{aligned}
            \psi\colon G&\to D_n \\
            a^kb^m&\mapsto s^kr^m.
        \end{aligned}
    \end{equation}
    C'est évidemment bien défini et bijectif, mais c'est également un homomorphisme parce que si nous calculons \( \psi\) sur un produit, nous devons comparer
    \begin{equation}        \label{EqBULPilp}
        \psi\big( a^{k_1}b^{m_1}a^{k_2}b^{m_2} \big)
    \end{equation}
    avec
    \begin{equation}        \label{EqIVEIphI}
        \psi\big( a^{k_1}b^{m_1}\big)\psi\big(a^{k_2}b^{m_2} \big)= s^{k_1}r^{m_1}s^{k_2}r^{m_2}.
    \end{equation}
    Vu que \( D_n\) et \( G\) ont les mêmes propriétés qui permettent de permuter \( a\) et \( b\) ou \( s\) et \( r\), l'expression à l'intérieur du \( \psi\) dans \eqref{EqBULPilp} se simplifie en \( a^kb^m\) avec les même \( k\) et \( n\) que l'expression à droite dans \eqref{EqIVEIphI} ne se simplifie en \( s^kr^m\).
\end{proof}

\begin{corollary}
    Toutes les propriétés démontrées pour \( G\) sont vraies pour \( D_n\). En particulier, avec quelques redites :
    \begin{enumerate}
        \item
            Le groupe \( D_n\) peut être défini comme étant le groupe engendré par un élément \( s\) d'ordre \( 2\) et un élément \( r\) d'ordre \( n-1\) assujettis à la relation \( srsr=e\).
        \item
            Le groupe \( D_n\) n'est pas abélien.
        \item
            Pour tout \( k\in\{ 1,\ldots, n-1 \}\) nous avons \( sr^ks=r^{-k}\).
        \item
            L'élément \( s\) ne peut pas être obtenu comme une puissance de \( r\).
        \item
            La liste des éléments de \( D_n\) est
            \begin{equation}
                D_n=\{ 1,r,\ldots, r^{n-1},s,sr,\ldots, sr^{n-1} \}
            \end{equation}
        \item
            Le groupe diédral \( D_n\) est d'ordre \( 2n\).
    \end{enumerate}
\end{corollary}

\begin{proposition}
    En posant \( C_n=\{ r^k \}_{k=0,\ldots, n-1}\) et \( C_2=\{ a^{\epsilon} \}_{\epsilon=0,1}\), nous pouvons exprimer \( D_n\) comme le produit semi-direct
    \begin{equation}
        D_n=C_n\times_{\rho}C_2
    \end{equation}
    où \( \rho\) désigne l'action adjointe.
\end{proposition}

\begin{proof}
    L'isomorphisme est :
    \begin{equation}
        \begin{aligned}
            \psi\colon C_n\times_{\rho}C_2&\to D_n \\
            (b^k,a^{\epsilon})&\mapsto b^ka^{\epsilon}.
        \end{aligned}
    \end{equation}
    \begin{subproof}
        \item[Action adjointe]
            L'application \( \rho_{a^{\epsilon}}=\AD(a^{\epsilon})\) est toujours un homomorphisme. Vu que \( a^{\epsilon}\) est soit \( e\) soit \( a\), nous allons nous restreindre à \( a\) et oublier l'exposant \( \epsilon\). Il faut montrer que\( \AD(a)\in\Aut(C_n)\). En utilisant le lemme~\ref{LemKKXdqdL},
            \begin{equation}
                \AD(a)b^k=ab^ka^{-1}=b^{-k}=b^{n-k}.
            \end{equation}
            L'application \( \AD(a)\colon C_n\to C_n\) est donc bijective et homomorphique. Ergo isomorphisme.
        \item[Injectif]
            Si \( \psi(b^k,a^{\epsilon})=\psi(b^l,a^{\sigma})\), alors par unicité du lemme~\ref{LemooNFRIooPWuikH} nous avons \( k=l\) et \( \epsilon=\sigma\).
        \item[Surjectif]
            Par la partie «existence»  du lemme~\ref{LemooNFRIooPWuikH}.
        \item[Homomorphisme]
            Lorqu'on prend deux sous-groupes d'un même groupe (ici le groupe des isométrique de \( \eR^2\)), et que l'on tente de faire un produit demi-direct en utilisant l'action adjointe, nous avons toujours un homomorphisme. Dans notre cas, le calcul est :
            \begin{equation}
                \psi\big( (b^k,a^{\epsilon})(b^l,a^{\sigma}) \big)=b^k\rho_{a^{\epsilon}}(b^l)a^{\epsilon+\sigma}=b^ka^{\epsilon}b^la^{-\epsilon}a^{\epsilon+\sigma}=b^ka^{\epsilon}b^la^{\sigma}=\psi(b^k,a^{\epsilon})\psi(b^l,a^{\sigma}).
            \end{equation}
    \end{subproof}
\end{proof}

%///////////////////////////////////////////////////////////////////////////////////////////////////////////////////////////
\subsubsection{Classes de conjugaison}
%///////////////////////////////////////////////////////////////////////////////////////////////////////////////////////////
\label{subsubsecZQnBcgo}

Pour les classes de conjugaison du groupe diédral nous suivons \cite{HRIMAJJ}.

D'abord pour des raisons de déterminants\footnote{Vous notez qu'ici nous utilisons un argument qui utilise la définition de \( D_n\) comme isométries de \( \eR^2\). Si nous avions voulu à tout prix nous limiter à la définition «abstraite» en termes de générateurs, il aurait fallu trouver autre chose.}, les classes des éléments de la forme \( r^k\) et de la forme \( sr^k\) ne se mélangent pas. Nous notons \( C(x)\) la classe de conjugaison de \( x\), et \( y\cdot x=yxy^{-1}\).

Les relations que nous allons utiliser sont
\begin{subequations}
    \begin{align}
        sr^ks=r^{-k}\\
        rs=sr^{-1}=sr^{n-1}.
    \end{align}
\end{subequations}

La classe de conjugaison qui ne rate jamais est bien entendu \( C(1)={1}\). Nous commençons les vraies festivités \( C(r^{m})\). D'abord \( r^k\cdot r^m=r^m\), ensuite
\begin{equation}
    (sr^k)\cdot r^m=sr^kr^mr^{-k}s^{-1}=sr^ms^{-1}=r^{-m}.
\end{equation}
Donc
\begin{equation}    \label{EqVFfFxgi}
    C(r^m)=\{ r^m,r^{-m} \}.
\end{equation}
À ce niveau il faut faire deux remarques. D'abord si \( m>\frac{ n }{2}\), alors \( C(r^m)\) est la classe de \( C^{n-m}\) avec \( n-m<\frac{ n }{2}\). Donc les classes que nous avons trouvées sont uniquement à lister avec \( m<\frac{ n }{2}\). Ensuite si \( m=\frac{ n }{2}\) alors \( r^m=r^{-m}\) et la classe est un singleton. Cela n'arrive que si \( n\) est pair.

Nous passons ensuite à \( C(s)\). Nous avons
\begin{equation}
    r^k\cdot s=r^ksr^{-k}=ssr^ksr^{-k}=sr^{-k}r^{-k}=sr^{n-2k},
\end{equation}
et
\begin{equation}
    (sr^k)\cdot s=\underbrace{sr^ks}_{r^{-k}}r^{-k}s^{-1}=r^{-2k}s=r^{n-2k}s=sr^{(n-1)(n-2k)}=sr^{n^2-2kn-n+2k}=sr^{2k}.
\end{equation}
donc
\begin{equation}
    C(s)=\{ sr^{n-2k},sr^{2k} \}_{k=0,\ldots, n-1}.
\end{equation}
Ici aussi l'écriture n'est pas optimale : peut-être que pour certains \( k\) il y a des doublons. Nous reportons l'écriture exacte à la discussion plus bas qui distinguera \( n\) pair de \( n\) impair. Notons juste que si \( n\) est pair, l'élément \( sr\) n'est pas dans la classe \( C(s)\).

Nous en faisons donc à présent le calcul en gardant en tête le fait qu'il n'a de sens que si \( n\) est pair. D'abord
\begin{equation}
    s\cdot (sr)=ssrs=rs=sr^{n-1}.
\end{equation}
Ensuite
\begin{equation}
    (sr^k)\cdot (sr)=sr^ksrr^{-k}s=r^{-2k+1}s=sr^{2k-1}.
\end{equation}
Avec \( k=\frac{ n }{2}\), cela rend \( s\cdot (sr)\), donc pas besoin de le recopier. Nous avons
\begin{equation}
    C(sr)=\{ sr^{2k-1} \}_{k=1,\ldots, n-1}.
\end{equation}

%///////////////////////////////////////////////////////////////////////////////////////////////////////////////////////////
\subsubsection{Le compte pour $ n$ pair}
%///////////////////////////////////////////////////////////////////////////////////////////////////////////////////////////
\label{SubsubsecROVmHuM}

Si \( n\) est pair, nous avons les classes
\begin{subequations}
    \begin{align}
        C(1)&=\{ 1 \}       &&& 1\text{ élément}\\
        C(r^m)&=\{ r^m,r^{m-1} \}&\text{ pour }&0<m<\frac{ n }{2}   & \frac{ n }{2}-1\text{ fois } 2\text{ éléments}\\
        C(r^{n/2})&=\{ r^{n/2} \}   &&&  1\text{ élément}\\
        C(s)&=\{ sr^{2k} \}_{k=0,\ldots, \frac{ n }{2}-1} &&&  \frac{ n }{2}\text{ éléments}\\
        C(sr)&=\{ sr^{2k+1} \}_{k=0,\ldots, \frac{ n }{2}-1} &&&  \frac{ n }{2}\text{ éléments}.
    \end{align}
\end{subequations}
Au total nous avons bien listé \( 2n\) éléments comme il se doit, dans \(  \frac{ n }{2}+3\) classes différentes.

%///////////////////////////////////////////////////////////////////////////////////////////////////////////////////////////
\subsubsection{Le compte pour $ n$ impair}
%///////////////////////////////////////////////////////////////////////////////////////////////////////////////////////////
\label{GJIzDEP}

Si \( n\) est impair, nous avons les classes
\begin{subequations}
    \begin{align}
        C(1)&=\{ 1 \}       &&& 1\text{ élément}\\
        C(r^m)&=\{ r^m,r^{m-1} \}&\text{ pour }&0<m<\frac{ n-1 }{2}   & \frac{ n-1 }{2}\text{ fois } 2\text{ éléments}\\
        C(s)&=\{ sr^k \}_{k=0,\ldots, n-1} &&&  n\text{ éléments}
    \end{align}
\end{subequations}
Au total nous avons bien listé \( 2n\) éléments comme il se doit, dans \(  \frac{ n+3 }{2}\) classes différentes.

%---------------------------------------------------------------------------------------------------------------------------
\subsection{Applications : du dénombrement}
%---------------------------------------------------------------------------------------------------------------------------

%///////////////////////////////////////////////////////////////////////////////////////////////////////////////////////////
\subsubsection{Le jeu de la roulette}
%///////////////////////////////////////////////////////////////////////////////////////////////////////////////////////////
\label{pTqJLY}
\index{groupe!fini}
\index{groupe!de permutations}
\index{groupe!et géométrie}
\index{combinatoire}
\index{dénombrement}

Soit une roulette à \( n\) secteurs que nous voulons colorier en \( q\) couleurs\cite{HEBOFl}. Nous voulons savoir le nombre de possibilités à rotations près. Soit d'abord \( E\) l'ensemble des coloriages possibles sans contraintes; il y a naturellement \( q^n\) possibilités. Sur l'ensemble \( E\), le groupe cyclique \( G\) des rotations d'angle \( 2\pi/n\) agit. Deux coloriages étant identiques si ils sont reliés par une rotation, la réponse à notre problème est donné par le nombre d'orbites de l'action de \( G\) sur \( E\) qui sera donnée par la formule du théorème de Burnside~\ref{THOooEFDMooDfosOw}.

Nous devons calculer \( \Card\big( \Fix(g) \big)\) pour tout \( g\in G\). Soit \( g\), un élément d'ordre \( d\) dans \( G\). Si \( g\) agit sur la roulette, chaque secteur a une orbite contenant \( d\) éléments. Autrement dit, \( g\) divise la roulette en \( n/d\) secteurs. Un élément de \( E\) appartenant à \( \Fix(g)\) doit colorier ces \( n/d\) secteurs de façon uniforme; il y a \( q^{n/d}\) possibilités.

Il reste à déterminer le nombre d'éléments d'ordre \( d\) dans \( G\). Un élément de \( G\) est donné par un nombre complexe de la forme \(  e^{2ik\pi/n}\). Les éléments d'ordre \( d\) sont les racines primitives\footnote{Une racine non primitive \( 8\)ième de l'unité est par exemple \( i\). Certes \( i^8=1\), mais \( i^4=1\) aussi. Le nombre \( i\) est d'ordre \( 4\).} \( d\)ièmes de l'unité. Nous savons que --par définition-- il y a \( \varphi(d)\) telles racines primitives de l'unité. Bref il y a \( \varphi(d)\) éléments d'ordre \( d\) dans \( G\).

La formule de Burnside nous donne maintenant le nombre d'orbites :
\begin{equation}
    \frac{1}{ n }\sum_{d|n}\varphi(d)q^{n/d}.
\end{equation}
Cela est le nombre de coloriage possibles de la roulette à \( n\) secteurs avec \( q\) couleurs.

%///////////////////////////////////////////////////////////////////////////////////////////////////////////////////////////
\subsubsection{L'affaire du collier}
%///////////////////////////////////////////////////////////////////////////////////////////////////////////////////////////
\label{siOQlG}

Nous avons maintenant des perles de \( q\) couleurs différentes et nous voulons en faire un collier à \( n\) perles. Cette fois non seulement les rotations donnent des colliers équivalents, mais en outre les symétries axiales (il est possible de retourner un collier, mais pas une roulette). Le groupe agissant sur \( E\) est maintenant le groupe diédral\footnote{Définition~\ref{DEFooIWZGooAinSOh}.}\index{diédral}\index{groupe!diédral} \( D_n\) conservant un polygone a \( n\) sommets.

Nous devons séparer le cas \( n\) impair du cas \( n\) pair.

Si \( n\) est impair, alors les axes de symétries passent par un sommet par le milieu du côté opposé. Le groupe \( D_n\) contient \( n\) symétries axiales. Nous avons donc maintenant
\begin{equation}
    | G |=2n.
\end{equation}
Nous écrivons la formule de Burnside
\begin{equation}
    \Card(\Omega)=\frac{1}{ 2n }\sum_{g\in G}\Card\big( \Fix(g) \big).
\end{equation}
Si \( g\) est une rotation, le travail est déjà fait. Si \( g\) est une symétrie, nous avons le choix de la couleur du sommet par lequel passe l'axe et le choix de la couleur des \( (n-1)/2\) paires de sommets. Cela fait
\begin{equation}
    qq^{(n-1)/2}=q^{\frac{ n+1 }{2}}
\end{equation}
possibilités. Nous avons donc
\begin{equation}
    \Card(\Omega)=\frac{1}{ 2n }\left( \sum_{d|n}q^{n/d}\varphi(d)+nq^{\frac{ n+1 }{2}} \right).
\end{equation}

Si \( n\) est pair, le choses se compliquent un tout petit peu. En plus de symétries axiales passant par un sommet et le milieu du côté opposé, il y a les axes passant par deux sommets opposés. Pour colorier un collier en tenant compte d'une telle symétrie, nous pouvons choisir la couleur des deux perles par lesquelles passe l'axe ainsi que la couleur des \( (n-2)/2\) paires de perles. Cela fait en tout
\begin{equation}
    q^2q^{\frac{ n-2 }{2}}=q^{\frac{ n+2 }{2}}.
\end{equation}
Le groupe \( G\) contient \( n/2\) tels axes.

Notons que cette fois \( G\) ne contient plus que \( n/2\) symétries passant par un sommet et un côté. L'ordre de $G$ est donc encore \( 2n\). La formule de Burnside donne
\begin{equation}
    \Card(\Omega)=\frac{1}{ 2n }\left( \sum_{d\divides n}\varphi(d)q^{n/d}+\frac{ n }{2}q^{(n+2)/2}+\frac{ n }{2}q^{n/2} \right).
\end{equation}


%+++++++++++++++++++++++++++++++++++++++++++++++++++++++++++++++++++++++++++++++++++++++++++++++++++++++++++++++++++++++++++
\section{Angles et rotations dans \( \eR^2\)}
%+++++++++++++++++++++++++++++++++++++++++++++++++++++++++++++++++++++++++++++++++++++++++++++++++++++++++++++++++++++++++++

%---------------------------------------------------------------------------------------------------------------------------
\subsection{Angles orientés, rotations}
%---------------------------------------------------------------------------------------------------------------------------

Nous avons défini les rotations planes en~\ref{DEFooFUBYooHGXphm}, et montré que pour les rotations vectorielles, nous avions en réalité le groupe \( \SO(2)\) en le corollaire~\ref{CORooVYUJooDbkIFY}. Il est temps de donner une forme matricielle aux rotations et de lier cela aux fonctions trigonométriques.

\begin{lemma}       \label{LEMooAJMAooXPSKtS}
    Si \( A\in \gO(2)\) alors il existe un unique \( \theta\in\mathopen[ 0 , 2\pi \mathclose[\) et un unique \( \epsilon=\pm 1\) tels que
    \begin{equation}
        A=\begin{pmatrix}
            \cos(\theta)    &   -\epsilon\sin(\theta)    \\
            \sin(\theta)    &   \epsilon\cos(\theta)
        \end{pmatrix}
    \end{equation}
\end{lemma}

\begin{proof}
    Soit une matrice \( A=\begin{pmatrix}
        a    &   b    \\
        c    &   d
    \end{pmatrix}\) et imposons qu'elle soit dans \( \gO(2)\). Le fait que \( A\) soit orthogonale impose
    \begin{equation}
        \begin{pmatrix}
            a    &   c    \\
            b    &   d
        \end{pmatrix}\begin{pmatrix}
            a    &   b    \\
            c    &   d
        \end{pmatrix}=\begin{pmatrix}
            a^2+c^2    &   ab+cd    \\
            ab+cd    &   b^2+d^2
        \end{pmatrix}=\begin{pmatrix}
            1    &   0    \\
            0    &   1
        \end{pmatrix}.
    \end{equation}
    Nous avons alors le système
    \begin{subequations}
        \begin{numcases}{}
            a^2+b^2=1\\
            b^2+d^2=1\\
            ab+cd=0
        \end{numcases}
    \end{subequations}
    La proposition~\ref{PROPooKSGXooOqGyZj} nous permet de déduire qu'il existe un unique \( \theta\in\mathopen[ 0 , 2\pi \mathclose[\) tel que \( a=\cos(\theta)\), \( c=\sin(\theta)\), ainsi que plusieurs \( \alpha\in \eR\) tel que \( b=\cos(\alpha)\), \( d=\sin(\alpha)\).

        Note : si nous voulons \( \alpha\in\mathopen[ 0 , 2\pi \mathclose[\), alors il y a unicité. Ici nous ne nous attachons pas à cette contrainte; nous savons qu'il en existe plusieurs, et nous allons en fixer un en fonction de \( \theta\). Le \( \alpha\) ainsi fixé ne sera peut-être pas dans \( \mathopen[ 0 , 2\pi \mathclose[\), mais ce ne sera pas grave.

        Les angles \( \theta\) et \( \alpha\) sont alors liés par la contrainte
        \begin{equation}
            \cos(\theta)\cos(\alpha)+\sin(\theta)\sin(\alpha)=0.
        \end{equation}
        Utilisant l'identité \eqref{EQooCVZAooQfocya} cela signifie que \( \cos(\theta-\alpha)=0\). Donc
        \begin{equation}
            \alpha\in\{ \theta+\frac{ \pi }{2}+k\pi \}_{k\in \eZ}.
        \end{equation}
        Si \( k\) est pair, ça donne
        \begin{subequations}
            \begin{align}
                \cos(\alpha)=-\sin(\theta)\\
                \sin(\alpha)=\cos(\theta)
            \end{align}
        \end{subequations}
        et alors
        \begin{equation}        \label{EQooNAMKooKACIfd}
            A=\begin{pmatrix}
                \cos(\theta)    &   -\sin(\theta)    \\
                \sin(\theta)    &   \cos(\theta)
            \end{pmatrix}.
        \end{equation}
        Si au contraire \( k\) est impair, alors
        \begin{subequations}
            \begin{align}
                \cos(\alpha)=\sin(\theta)\\
                \sin(\alpha)=-\cos(\theta),
            \end{align}
        \end{subequations}
        et
        \begin{equation}        \label{EQooJMYFooGgAiMJ}
            A=\begin{pmatrix}
                \cos(\theta)    &   \sin(\theta)    \\
                \sin(\theta)    &   -\cos(\theta)
            \end{pmatrix}.
        \end{equation}

        Nous avons démontré qu'une matrice de \( \gO(2)\) était forcément d'une des deux formes \eqref{EQooNAMKooKACIfd} ou \eqref{EQooJMYFooGgAiMJ}. Il est maintenant facile de vérifier que ces deux matrices sont effectivement dans \( \gO(2)\).
\end{proof}


\begin{lemma}       \label{LEMooHRESooQTrpMz}
    Toute rotation s'écrit (dans la base canonique) de façon unique sous la forme
    \begin{equation}
        \begin{pmatrix}
            \cos(\theta)    &   -\sin(\theta)    \\
            \sin(\theta)    &   \cos(\theta)
        \end{pmatrix}
    \end{equation}
    avec \( \theta\in\mathopen[ 0 , 2\pi \mathclose[\).
\end{lemma}

\begin{proof}
    Vu que \( \SO(2)\) est la partie de \( \gO(2)\) constitué des matrices de déterminant \( 1\), nous pouvons reprendre la forme donnée par le lemme~\ref{LEMooAJMAooXPSKtS} et fixer \( \epsilon\) par la contrainte sur le déterminant.

    Nous avons, en utilisant la relation du lemme~\ref{LEMooAEFPooGSgOkF},
    \begin{equation}
        \det\begin{pmatrix}
            \cos(\theta)    &   -\epsilon\sin(\theta)    \\
            \sin(\theta)    &   \epsilon\cos(\theta)
        \end{pmatrix}=\epsilon,
    \end{equation}
    et donc il faut et suffit de fixer \( \epsilon=1\).
\end{proof}

\begin{corollary}[\cite{MonCerveau}]        \label{CORooGGVUooLQYGET}
    Nous avons une bijection
    \begin{equation}
        \begin{aligned}
            \psi\colon \SO(2)&\to \mathopen[ 0 , 2\pi \mathclose[ \\
        \begin{pmatrix}
            \cos(\theta)    &   -\sin(\theta)    \\
            \sin(\theta)    &   \cos(\theta)
        \end{pmatrix}
              &\mapsto \theta,
        \end{aligned}
    \end{equation}
    et un isomorphisme de groupe
    \begin{equation}
        \begin{aligned}
            \varphi\colon \SO(2)&\to \gU(1)=\{  e^{i\theta} \}_{\theta\in \eR} \\
        \begin{pmatrix}
            \cos(\theta)    &   -\sin(\theta)    \\
            \sin(\theta)    &   \cos(\theta)
        \end{pmatrix}
        &\mapsto  e^{i\theta}.
        \end{aligned}
    \end{equation}
\end{corollary}

\begin{proof}
    La première assertion est une paraphrase du lemme~\ref{LEMooHRESooQTrpMz}. Pour la seconde, il faut vérifier que c'est bien un morphisme et une bijection.

    Pour le morphisme, ce sont les formules d'addition d'angle du lemme~\ref{LEMooJAWBooJGfZIL} qui jouent. En ce qui concerne la bijection\ldots

    \begin{subproof}
        \item[Surjection]
            Vu que \(  e^{i\theta+2ki\pi}= e^{i\theta}\), tout élément de \( \gU(1)\) est exponentielle de \( i\theta\) pour un \( \theta\in\mathopen[ 0 , 2\pi \mathclose[\).
        \item[Injection]
            Nous avons \( \varphi(A)=\varphi(B)\) lorsque les formes matricielles de \( A\) et \( B\) sous forme trigonométrique sont avec des angles différents d'un multiple de \( 2\pi\). Vu que les fonctions trigonométriques sont périodiques, nous avons \( A=B\) parce que leurs matrices sont égales.
    \end{subproof}
\end{proof}

\begin{proposition}[\cite{ooGEXYooMTrOdH}]      \label{PROPooDWIMooQPkobw}
    Si \( u\) et \( v\) sont des vecteurs unitaires\footnote{De norme \( 1\).} de \( \eR^2\) alors il existe une unique rotation\footnote{Définition~\ref{DEFooFUBYooHGXphm}.} \( f\) telle que \( f(u)=v\).
\end{proposition}

\begin{proof}
    C'est la proposition~\ref{PROPooNXJKooEDOczh} appliquée à \( O=(0,0)\).
\end{proof}

Notons l'unicité. Nous ne faisons pas de différences entre \( R_{\theta}\) et \( R_{\theta+2\pi}\) et les autres \( R_{\theta+2k\pi}\). En particulier si une rotation \( T\) est donnée, dire «\( T=R_{\theta}\)» ne définit pas un nombre \( \theta\) de façon univoque. Par contre ça définit une classe modulo \( 2\pi\), c'est-à-dire un élément \( \theta\in \eR/2\pi\).

Nous avons déjà défini le groupe \( \SO(2)\) en la définition~\ref{DEFooJLNQooBKTYUy} et nous avons déterminé ses matrices dans \( \eR^2\) en le lemme~\ref{LEMooHRESooQTrpMz}.

La proposition~\ref{PROPooDWIMooQPkobw} donne une application
\begin{equation}
    T\colon S^1\times S^1\to \SO(2).
\end{equation}
Et nous avons une relation d'équivalence sur \( S^1\times S^1\) donnée par \( (u,v)\sim(u',v')\) si et seulement si il existe \( g\in\SO(2)\) telle que \( g(u)=u'\) et \( g(v)=v'\).

\begin{definition}[Angle orienté\cite{ooGEXYooMTrOdH}]      \label{DEFooVBKIooWlHvod}
    Les classes de \( S^1\times S^1\) pour cette relation d'équivalence sont les \defe{angles orientés de vecteurs}{angle!orienté de vecteurs}. Nous notons \( [u,v]\) la classe de \( (u,v)\).
\end{definition}

\begin{proposition}     \label{PROPooIWJQooGQJBWR}
    Nous avons \( T(u,v)=T(u',v')\) si et seulement si \( (u,v)\sim(u',v')\).
\end{proposition}

\begin{proof}
    En utilisant la commutativité du groupe \( \SO(2)\) nous avons équivalence entre les affirmations suivantes :
    \begin{itemize}
        \item \( (u,v)\sim (u',v')\)
        \item \( T(u,u')=T(v',v')\)
        \item \( T(u,u')\circ T(u',v)=T(v,v')\circ T(u',v)\)
        \item
            \( T(u,v)=T(u',v')\).
    \end{itemize}
\end{proof}

\begin{proposition}
    Nous avons une bijection
    \begin{equation}
        \begin{aligned}
            S\colon \frac{ S^1\times S^1 }{ \sim }&\to \SO(2) \\
            [u,v]&\mapsto T(u,v).
        \end{aligned}
    \end{equation}
\end{proposition}

\begin{proof}
    En plusieurs points.
    \begin{subproof}
    \item[\( S\) est bien définie]
        En effet si \( [u,v]=[z,t]\) alors \( T(u,v)=T(z,t)\).
    \item[Injectif]
        Si \( S[u,v]=S[z,t]\) alors \( T(u,v)=T(z,t)\), qui implique \( (u,v)\sim (z,t)\) par la proposition~\ref{PROPooIWJQooGQJBWR}.
    \item[Surjectif]
        Nous avons \( R_{\theta}=T(u,R_{\theta}u)\).
    \end{subproof}
\end{proof}

\begin{definition}[Somme d'angles orientés\cite{ooGEXYooMTrOdH}]
    Si \( [u,v]\) et \( [z,t]\) sont des angles orientés, nous définissons la somme par
    \begin{equation}
        [u,v]+[z,t]=S^{-1}\Big( S[u,v]\circ S[z,t] \Big).
    \end{equation}
\end{definition}

\begin{lemma}       \label{LEMooWISVooYsStJp}
    Quelques propriétés des angles plats liées à la somme.
    \begin{enumerate}
        \item
            \( (S^1\times S^1)/\sim\) est un groupe commutatif.
        \item       \label{ITEMooBKTFooWbEvIU}
            Relations de Chasles :
            \begin{equation}
                [u,v]+[v,w]=[u,w].
            \end{equation}
        \item
            \( -[u,v]=[v,u]\).
    \end{enumerate}
\end{lemma}

\begin{proof}
    Pour la relation de Chasles, ça se base sur la propriété correspondante sur \( T\) :
    \begin{subequations}
        \begin{align}
            [u,v]+[v,w]&=S^{-1}\Big( T(u,v)\circ T(v,w) \Big)\\
            &=S^{-1}\big( T(u,w) \big)\\
            &=[u,w].
        \end{align}
    \end{subequations}
    Pour l'inverse, la vérification est que
    \begin{equation}
        [u,v]+[v,u]=[u,u]=0.
    \end{equation}
\end{proof}

\begin{definition}      \label{DEFooFLGNooCZUkHY}
    La \defe{mesure}{mesure!angle entre vecteurs} de l'angle orienté \( [u,v]\) est \( [\theta]_{2\pi}\) si \( T[u,v]=R_{\theta}\).
\end{definition}
Notons dans cette définition qu'écrire \( T[u,v]=R_{\theta}\) dans \( \SO(2)\) ne définit pas \( \theta\), mais seulement sa classe modulo \( 2\pi\). C'est pour cela que la mesure de l'angle orienté n'est également définie que modulo \( 2\pi\).

Pour la suite nous allons nous intéresser à des vecteurs qui ont, dans l'idée, un point de départ et un point d'arrivée. Si \( A,B\in \eR^2\) nous notons
\begin{equation}
    \vect{ AB }=\frac{ B-A }{ \| B-A \| }.
\end{equation}
C'est le vecteur unitaire dans la direction «de \( B\) vers $A$».

\begin{theorem}[Théorème de l'angle inscrit\cite{ooRGSCooNgALYH}]       \label{THOooQDNKooTlVmmj}
    Soit un cercle \( \Gamma\) de centre \( O\) et trois points distincts \( A,B,M\in \Gamma\). Alors
    \begin{equation}
        2(\vect{ MA },\vect{ MB })\in (\vect{ OA },\vect{ OB })_{2\pi}
    \end{equation}
    où l'indice \( 2\pi\) indique la classe modulo \( 2\pi\).
\end{theorem}

\begin{proof}
    Le triangle \( MOA\) est isocèle en \( O\), donc les angles à la base sont égaux. Et de plus la somme des angles est dans \( [\pi]_{2\pi}\). Bon, entre nous, nous savons que la somme des angles est exactement \( \pi\), mais comme nous n'avons pas défini les angles autrement que modulo \( \pi\), nous ne pouvons pas dire mieux. Donc
    \begin{equation}
        2(\vect{ AB },\vect{ AO })+(\vect{ OB },\vect{ OA })\in [\pi]_{2\pi}.
    \end{equation}
    Il faut être sûr de l'orientation de tout cela. Le nombre \( (\vect{ AB },\vect{ AO })\) est l'angle qui sert à amener \( \vect{ AB }\) sur \( \vect{ AO }\). Vu que nous l'avons choisi dans le sens trigonométrique, il faut bien prendre les autres dans le sens trigonométrique et utiliser \( (\vect{ OA }, \vect{ OB })\) et non \( (\vect{ OB },\vect{ OA })\).

\begin{center}
   \input{auto/pictures_tex/Fig_YQIDooBqpAdbIM.pstricks}
\end{center}

De la même manière sur le triangle \( MOB\) nous écrivons
\begin{equation}
    2(\vect{ MB },\vect{ MO })+(\vect{ OM },\vect{ OB })\in[\pi]_{2\pi}.
\end{equation}
Nous faisons la différence entre les deux équations en remaquant que la différence de deux représentants de \( [\pi]_{2\pi}\) est un représentant de \( [0]_{2\pi}\) et en en nous souvenant que \( -(\vect{ MB },\vect{ MO })=(\vect{ MO },\vect{ MB })\) et les relations de Chasles du lemme~\ref{LEMooWISVooYsStJp}\ref{ITEMooBKTFooWbEvIU} nous avons :
\begin{equation}
    2(\vect{ MA },\vect{ MB })+(\vect{ OB },\vect{ OA })\in[0]_{2\pi}.
\end{equation}
\end{proof}

\begin{normaltext}
    Comment exprimer le fait qu'un angle orienté soit égal à \( \theta\) modulo \( \pi\) alors que les angles orientés sont des classes modulo \( 2\pi\) ? Nous ne pouvons certainement pas écrire
    \begin{equation}
        (u,v)=[\theta]_{\pi}
    \end{equation}
    parce que \( (u,v)\) est un élément de \( S^1\times S^1\) alors que \( [\theta]_{\pi}\) est un ensemble de nombres. Nous pouvons écrire
    \begin{equation}
        [u,v]\subset [\theta]_{\pi}.
    \end{equation}
    C'est cohérent parce que nous avons des deux côtés des ensembles de nombres. Les opérations permises sont l'égalité ou l'inclusion. L'égalité entre les deux ensembles n'est pas possible parce que la différence minimale ente deux éléments dans \( [u,v]\) est \( 2\pi\) alors que celle dans \( [\theta]_{\pi}\) est \( \pi\).

    Si \( u\) et \( v\) forment un angle droit, nous avons
    \begin{equation}
        [u,v]=\{ \frac{ \pi }{2}+2k\pi \}_{k\in \eZ}.
    \end{equation}
    Et cela est bien un sous-ensemble de \( [\pi/2]_{\pi}\).

    Pour exprimer que la différence entre deux angles orientés diffèrent de \( \pi\) nous devrions écrire :
    \begin{equation}
        [u,v]\subset[a,b]_{\pi}
    \end{equation}
    où le membre de droite signifie la classe modulo \( \pi\) d'un représentant de \( [a,b]\).

    Nous allons cependant nous permettre d'écrire
    \begin{equation}
        [u,v]=[a,b]_{\pi}
    \end{equation}
    voire carrément
    \begin{equation}
        (u,v)=(a,b)_{\pi}.
    \end{equation}
    Cette dernière égalité devant être comprise comme voulant dire que l'angle pour passer de \( u\) à \( v\) est soit le même que celui pour alle de \( a\) à \( b\) soit ce dernier plus \( \pi\).
\end{normaltext}

\begin{theorem}[\cite{ooRGSCooNgALYH}]      \label{THOooUDUGooTJKDpO}
    Soient \( 4\) points distincts du plan \( A,B,C,D\). Ils sont alignés ou cocycliques\footnote{C'est-à-dire sur un même cercle.} si et seulement si
    \begin{equation}
        (\vect{ CA },\vect{ CB })=(\vect{ DA },\vect{ DB })_{\pi}.
    \end{equation}
\end{theorem}

Nous allons seulement démontrer l'implication directe.
\begin{proof}
    Si les quatre points sont alignés nous avons \( [\vect{ CA },\vect{ CB }]=[0]_{2\pi}\) et \( [\vect{ DA },\vect{ DB }]=[0]_{2\pi}\). En particulier nous avons
    \begin{equation}
        [\vect{ CA },\vect{ CB }]=[\vect{ DA },\vect{ DB }]
    \end{equation}
    et a fortiori l'égalité modulo \( \pi\) au lieu de \( 2\pi\).

    Nous nous relâchons en termes de notations. Si les quatre points sont cocycliques, nous pouvons utiliser le théorème de l'angle inscrit~\ref{THOooQDNKooTlVmmj} dans les triangles \( ABC\) et \( ADB\) :
    \begin{subequations}
        \begin{align}
            2(\vect{ CA },\vect{ CB })=(\vect{ OA },\vect{ OB })_{2\pi}\\
            2(\vect{ DA },\vect{ DB })=(\vect{ OA },\vect{ OB })_{2\pi},
        \end{align}
    \end{subequations}
    ce qui donne \(  2(\vect{ CA },\vect{ CB })=2(\vect{ DA },\vect{ DB })_{2\pi}  \) et donc
    \begin{equation}
        (\vect{ CA },\vect{ CB })=(\vect{ DA },\vect{ DB })_{\pi}.
    \end{equation}

    Comme annoncé, nous ne faisons pas la preuve dans l'autre sens; elle peut être trouvée dans~\cite{ooRGSCooNgALYH}.
\end{proof}

\begin{example}     \label{EXooOXAAooZMdDfP}
    À propos de groupe engendré et de générateur\footnote{Définition \ref{DEFooWMFVooLDqVxR} et \ref{DefHFJWooFxkzCF}}. Soit \( G\) le groupe des rotations d'angle\footnote{Voir la définition \ref{DEFooFLGNooCZUkHY}.} \( k\pi/5\) (avec \( k \) entier). Ce groupe est constitué des \og{} dixièmes de tour \fg{}, puisque \( \frac{k\pi} 5 = \frac{2k\pi}{10}.\)
    
     La rotation d'angle \( 2 \pi/5\)  n'est pas génératrice parce qu'elle n'engendre que des \og{} cinquièmes de tour \fg{} : \( 4 \pi/5\), \( 6 \pi/ 5\),\( 8\pi/5\) et l'identité.
     
     Par contre, la rotation d'angle \( \pi/5\) est génératrice.
\end{example}

%---------------------------------------------------------------------------------------------------------------------------
\subsection{Angles et nombres complexes}
%---------------------------------------------------------------------------------------------------------------------------
\label{SUBSECooKNUVooUBKaWm}

Les nombres complexes peuvent être repérés par une norme et un angle, ce qui en fait un terrain propice à l'utilisation des angles orientés. Nous en ferons d'ailleurs usage dans \( \hat\eC=\eC\cup\{ \infty \}\) pour parler d'alignement, de cocyclicité et de birapport dans la proposition~\ref{PROPooSGCJooLnOLCx}.

Soient deux éléments \( z_1,z_2\in \eC\). Nous les écrivons sous la forme \( z_1=r_1 e^{i\theta_1}\) et \( z_2=r_2 e^{i\theta_2}\); remarquons que cela ne définit \( \theta_i\) qu'à \( 2\pi\) près. Nous avons
\begin{equation}
    [z_1,z_2]=[\theta_2-\theta_1]_{2\pi}.
\end{equation}

Soient maintenant \( a,b,c,d\in \eC\). Nous écrivons \( \vect{ ab }\) le vecteur unitaire dans le sens «de \( a\) vers \( b\)», c'est-à-dire un multiple positif bien choisi du nombre \( b-a\). Nous notons \( \theta_{ab}\) l'argument du nombre complexe \( b-a\), et nous avons encore
\begin{equation}
    [\vect{ ab },\vect{ cd }]=[\theta_{ab}-\theta_{cd}].
\end{equation}

Avec toutes ces notations, ce qui est bien est que les produits et quotients de nombres complexes se comportent très bien par rapport aux angles : l'argument de \( a/b\) est \( \theta_a-\theta_b\) et en particulier l'argument de
\begin{equation}
    \frac{ a-b }{ c-d }
\end{equation}
est dans la classe de l'angle orienté
\begin{equation}
    [\vect{ ba },\vect{ dc }].
\end{equation}

%---------------------------------------------------------------------------------------------------------------------------
\subsection{Sous-groupes finis de $\SO(2)$}
%---------------------------------------------------------------------------------------------------------------------------

\begin{lemma}[\cite{MonCerveau}]        \label{LEMooUKEVooAEWvlM}
    Tout sous-groupe fini de \( \SO(2)\) est cyclique.
\end{lemma}

\begin{proof}
    Soit uns sous-groupe fini \(G\) de \( \SO(2)\).  Nous savons que \( \SO(2)\) est isomorphe à \( \gU(1)\) par le corollaire~\ref{CORooGGVUooLQYGET}, et en bijection avec \( \mathopen[ 0 , 2\pi \mathclose[\). Vu que \( G\) est fini, l'ensemble \( G\setminus\{ e \}\) il possède, dans \( \mathopen[ 0 , 2\pi \mathclose[\) un élément minimum non nul. Soit \( g_0\) ce minimum.

        Soit un élément \( g_1\) de \( G\) qui ne serait ni l'identité ni un multiple de \( g_0\). En particulier tous les nombres du type \( g_1-kg_0\) sont dans \( G\) (l'image de \( G\) dans \( \mathopen[ 0 , 2\pi \mathclose[\) en fait). Si \( g_1\) n'est pas un multiple de \( g_0\), il n'en reste pas moins que \( g_1=\lambda g_0\); alors en prenant pour \( k\) la partie entière de \( \lambda\), l'élément \( g_1-kg_0\) est plus petit que \( g_0\). Contradiction.
\end{proof}
