% This is part of Mes notes de mathématique
% Copyright (c) 2011-2017
%   Laurent Claessens
% See the file fdl-1.3.txt for copying conditions.

%+++++++++++++++++++++++++++++++++++++++++++++++++++++++++++++++++++++++++++++++++++++++++++++++++++++++++++++++++++++++++++ 
\section{Les choses qui doivent vous faire tiquer}
%+++++++++++++++++++++++++++++++++++++++++++++++++++++++++++++++++++++++++++++++++++++++++++++++++++++++++++++++++++++++++++

Un cours de math doit toujours être lu attentivement, surtout si vous avez l'intention de resservir à un jury le fruit de vos lectures. Dans ce livre, trois éléments doivent vous faire redoubler de prudent.

D'abord les références à \cite{MonCerveau} indiquent qu'une bonne partie de ce qui suit est de l'invention personnelle de l'auteur. Ensuite les notes en bas de pages écrites en \texttt{fonte tt} comme celle-ci\quext{Les notes comme celle-ci signifient qu'il y a quelque chose dont je ne suis pas sûr.} Et enfin certains problèmes sont indiqués plus longuement dans un environnement dédié en petit caractères comme ceci :
\begin{probleme}
    Les choses écrites comme ceci sont des questions ou des éléments sur lesquels j'ai un doute. Lisez-les attentivement. Ces notes mentionnent des points que personnellement je n'oserais pas affirmer plein d'aplomb à un jury d'agrégation.
\end{probleme}

%+++++++++++++++++++++++++++++++++++++++++++++++++++++++++++++++++++++++++++++++++++++++++++++++++++++++++++++++++++++++++++
\section{Les questions pour lesquelles je n'ai pas (encore) de réponse}
%+++++++++++++++++++++++++++++++++++++++++++++++++++++++++++++++++++++++++++++++++++++++++++++++++++++++++++++++++++++++++++
\label{SecooCKWWooBFgnea}

Ces notes ne sont pas relues de façon systématique. Aucune garantie. Merci de me signaler toute faute ou remarque : le relecteur c'est toi. Voici une petite liste de questions que je me pose ou de choses écrites dont je ne suis pas certain. Si vous avez un avis ou une réponse à un des points, merci de vous faire connaître.

Vous pouvez soit m'écrire directement, soit créer une \emph{issue} sur github pour répondre aux questions.

%---------------------------------------------------------------------------------------------------------------------------
\subsection{Mes questions d'analyse.}
%---------------------------------------------------------------------------------------------------------------------------

\begin{enumerate}
    \item
        À propos de suites de Cauchy dans un espace vectoriel topologique et dans un espace métrique, est-ce que le théorème \ref{THOooGQZSooAmQolf} est correct ?

        Soit \( V\) un espace vectoriel topologique métrisable\footnote{i.e. admet une base dénombrable de topologique, voir la proposition \ref{PROPooPRLBooGtsRjr}}, alors il admet une métrique \( d\) compatible avec la topologie telle que une suite dans \( V\) est \( d\)-Cauchy si et seulement si elle est \( \tau\)-Cauchy.

        Dans cet ordre d'idée, il faut des exemples de :
        \begin{itemize}
            \item un espace vectoriel topologique métrisable et une métrique \( d\) compatible avec la topologie, mais dont les suites \( d\)-Cauchy ne sont pas celles \( \tau\)-Cauchy. Et en particulier dont la complétude est différente que celle de la «bonne» métrique donnée par la théorème \ref{THOooAGBXooZnvQLK}.
            \item Et aussi un exemple pour la remarque \ref{REMooUFQYooUVCCjs}.
        \end{itemize}
    \item
        Est-ce que l'énoncé du théorème de Müntz \ref{ThoAEYDdHp} est correct ? Voir en particulier la remarque \ref{REMooGPYYooCQJwFa} à propos de la présence du monôme  \( 1\) dans la liste.
    \item
        Que penser de la remarque \ref{RemfdJcQF} qui dit qu'on doit avoir un théorème de complétion de partie orthonormale en une base orthonormale pour un espace de Hilbert ? C'est vrai ?
    \item
        À propos de formule sommatoire de Poisson, est-ce que l'exemple \ref{ExDLjesf} est bien fait ? En particulier la formule \eqref{EqjrNxLr} est-elle correcte et bien justifiée ?
    \item 
        L'exemple \ref{ExfYXeQg} parle d'inverser une intégrale et une dérivée au sens des distributions pour prouver que la dérivée de \( \int_0^xg(t)dt\) par rapport à \( x\) est \( g\). Rendre cela rigoureux.
    \item
        À propos du théorème de récurrence de Poincaré \ref{ThoYnLNEL}, l'application \( \phi\) doit être mesurable ? Répondre à la question posée sur la page de discussion de \wikipedia{fr}{Théorème_de_récurrence_de_Poincaré}{l'article sur wikipédia}.
    \item
        Toujours à propos du théorème de récurrence de Poincaré, il me semble qu'il y a un énoncé qui insiste sur la compacité de l'espace des phase et une démonstration utilisant la propriété de sous-recouvrement fini. Je serais content de retrouver cela. (ce serait sans doute mettable dans la leçon sur l'utilisation de la compacité)
    \item 
        À propos de complétude d'espace de Hiblert, il faut relire et me dire si la démonstration de \ref{ThoESIyxfU}\ref{ITEMooITQUooKWwMwu} est correcte.
        \item 
            Dans \cite{OEVAuEz}, on parle de la proposition \ref{PropZMKYMKI} à sa page \( 10\). Comment est-ce qu'on justifie le passage
            \begin{equation}
                \int_{\eR^d}T\big( y\mapsto \varphi(x)\psi(x-y) \big)dx=T\Big( y\mapsto\int_{\eR^d}\varphi(x)\psi(x-y)dx \Big).
            \end{equation}
            Sylvie Benzoni précise que «ceci demanderai quelque justification». Où trouver lesdites justifications ? Il s'agit de permuter une distribution et une intégrale.

        \item
            En ce qui concerne les questions «fines» de topologie concernant l'espace \( \swD(K)\) des fonctions \(  C^{\infty}\) à support dans le compact \( K\), il faut voir les énoncés, les preuves et l'enchaînement des théorèmes et propositions
            \begin{enumerate}
                \item Proposition \ref{PropQAEVcTi} pour dire que \( \swD(K)\) est métrique et complet (et donc de Baire).
                \item La proposition \ref{PropSYMEZGU} qui donne la complétude de \( \big( C(X,Y),\| . \|_{\infty} \big)\) lorsque \( X\) est compact et \( Y\) métrique complet.
                \item Le théorème \ref{ThoNBrmGIg} de Banach-Steinhaus avec les semi-normes. Le fait qu'il s'applique bien à \( \swD(K)\).
                \item L'utilisation de cette version du théorème de Banach-Steinhaus dans la proposition \ref{PropLKtBsVi}. En particulier l'utilisation de la famille de fonctionnelles \eqref{EqBEKoqMb} et la preuve qu'elles sont continues.
                \item
                    L'équivalence entre les deux topologies de la proposition \ref{PropLOwUvCO} et le fait que cela s'applique bien à \( \swD(K)\).
                \item
                    Et enfin l'utilisation de tout ça pour donner l'unique solution de l'équation de Schrödinger du théorème \ref{ThoLDmNnBR}. Est-ce que l'énoncé de ce théorème est déjà correct ?
                \item
                    En particulier la fonction \eqref{EqEVtJcnz} me semble être un petit bricolage. 
            \end{enumerate}
    \item
        Dans la démonstration du théorème de Baire (\ref{ThoQGalIO}), il manque peut-être quelques fermetures sur les boules.
    \item
        Préciser l'énoncé et donner une démonstration de la proposition \ref{PropMpBStL} qui traite de sommes dénombrables.
    \item
        La justification que \( f_n\) est borélienne dans la proposition \ref{PropfqvLOl} mérite plus de détails.
    \item
        Où trouver une preuve de la proposition \ref{PropKZDqTR} sur le supplémentaire topologique ?
    \item
        La partie «unicité» du théorème de Cauchy-Lipschitz \ref{ThokUUlgU}.
    \item
        L'inversion entre la somme et l'intégrale de l'équation \eqref{EqXSgZGw}.
    \item   \label{ItemLPrIWZhPg}
        À propos de différentiabilité pour une application entre deux espaces de Banach. Le théorème \ref{ThoOYwdeVt} dit que si
    \begin{equation}
        \Dsdd{ f(x+tu) }{t}{0}
    \end{equation}
    existe pour tout \( x\in B(a,r)\) et est continue (par rapport à \( x\)) en \( x=a\), et que de plus \( \frac{ \partial f }{ \partial u }(a)=0\) pour tout \( u\), alors \( f\) est différentiable en \( a\).

    Est-ce que cet énoncé est encore vrai lorsque \( \frac{ \partial f }{ \partial u }(a)\neq 0\) ? Est-ce qu'il existe un théorème comparable à la proposition \ref{Diff_totale} pour la dimension infinie ? Il me semble que non, mais je n'ai pas de contre-exemples en tête. Par contre, je vois bien où la preuve bloque : c'est le lien entre la linéarité et la différentiabilité.

    % Lorsque cette question aura une réponse, on pourra peut-être décommenter les choses à la position 229262367.

    \item

        À propos du théorème de la fonction implicite \ref{ThoAcaWho}. Est-ce que la partie unicité est correctement énoncée et démontrée ? En particulier il me semble qu'il manque la mention d'un voisinage de \( y_0\) dans \cite{SNPdukn}.

    \item

        Il me faudrait un exemple de partie de \( \eR\) qui soit non borélienne mais mesurable au sens de Lebesgue. Comme je suis exigeant, je le veux le plus constructif possible : pas d'axiome du choix. Pour l'instant le mieux que j'aie trouvé est une feuille de TD\cite{XSHoosgoQa} qui utilise des ordinaux.

    \item

        À propos de l'écriture décimale des nombres, la proposition \ref{PropSAOoofRlQR} et le théorème \ref{ThoRXBootpUpd} sont à relire attentivement parce que les démonstrations sont presque complètement des inventions personnelles.

    \item

        À propos de l'ensemble de Cantor, le lemme \ref{LemAZGoosKzEm} et la proposition \ref{PropTPPooDySbm} sont à relire attentivement : les démonstrations sont des inventions personnelles.

    \item

        La démonstration du théorème de Lyapunov \ref{ThoBSEJooIcdHYp} est en grande partie de l'invention personnelle interpolée de divers morceaux pris par-ci par-là. En particulier tout ce qu'il y a autour de \eqref{subeqsFNPJooERJkxO} et l'utilisation par le théorème d'explosion en temps fini pour déduire l'existence sur \( \eR\) de la solution mérite d'être relu, et plutôt deux fois qu'une.

    \item

        Dans la preuve de la dualité de \( L^p\), théorème \ref{ThoLPQPooPWBXuv}, il y a une partie dans laquelle on diffère le cas \( p= 1\) des autres. À quelle endroit la partie \( p=1\) ne fonctionne pas ? Est-ce seulement le fait que
        \begin{equation}
            \| \mtu_E \|_p=\mu(E)^{1/p} 
        \end{equation}
        et non \( \mu(E)\) ?

    \item
        Pour démontrer qu'une série entière est de classe \(  C^{\infty}\) dans son disque de convergence, on s'appuie sur la proposition \ref{PropSNMEooVgNqBP} qui donne la différentiabilité et ensuite sur le corollaire \ref{CorCBYHooQhgara} qui effectue une parieuse récurrence. Toutes ces preuves sont de moi. À relire.
    \item
        Est-ce que la proposition \ref{PropooUEEOooLeIImr} qui donne le critère \( d(x_p,x_q)\leq \epsilon\) pour être une suite de Cauchy est valide dans un espace topologique métrique au lieu de normé ?  Dans quels cas a-t-on
        \begin{equation}
            d(a,b)=d(a+u)+d(b+u) 
        \end{equation}
        lorsque \( d\) est une distance qui n'est pas spécialement induite d'une norme ?
    \item
        En ce qui concerne la limite de fonction étagées pour les fonctions mesurables, le lemme \ref{LemYFoWqmS} démontre ça beaucoup plus simplement que le théorème \ref{THOooXHIVooKUddLi}. Pourquoi ?
    \item
        Peut-on permuter une application linéaire et continue avec une somme pas spécialement dénombrable ? En supposant que \( \sum_{i\in I}f(v_i)\) existe, la proposition \ref{PROPooWLEDooJogXpQ} semble dire que oui. Est-ce correct ?

        Peut-on avoir un exemple de partie sommable \( \{ v_i \}_{i\in I}\) et d'application linéaire continue \( f\) telle que la partie \( \{ f(v_i) \}\) ne soit pas sommable ?
    \item
        Soit une partie orthonormale \emph{pas spécialement dénombrable} \( \{ u_i \}_{i\in I}\) d'un espace de Hilbert (pas spécialement séparable). Si
        \begin{equation}
            x=\sum_{i\in I}x_iu_i,
        \end{equation}
        puis-je prendre le produit scalaire avec \( u_{k}\) et le permuter avec la somme pour déduire que \( x_k=\langle x, u_k\rangle \) ?

        C'est ce que je fais dans la proposition \ref{PROPooWTOZooYZdlml}. 
    \item
        Un espace de Hilbert dont toutes les parties libres sont dénombrables est séparable. Pour démontrer cela, je passe par dire que toute base hilbertienne est dénombrable, et pour cela j'utilise le lemme de Zorn (voir le théorème \ref{THOooMKNFooVrCNGA} et le lemme \ref{LEMooXIECooCAQeJN}). Est-ce vraiment utile, ou bien peut-on démontrer l'existence de base dénombrable à partir du fait que toute partie libre est dénombrable, sans passer par Zorn ?
    \item
        Pour le théorème de Lax-Milgram \ref{THOooLLUXooHyqmVL}, il y a la majoration \( \| u \|\leq MC/\alpha\), mais je ne parviens pas à la prouver  : j'obtiens seulement \( \| u \|\leq C/\alpha\). Ici, \( \alpha\) est la constante de coercivité, \( M\) est tel que \( | a(u,v) |\leq M\| u \|\| v \|\) et \( C\) est la norme de \( L\).
    \item
        Explosion en temps fini. C'est le corollaire \ref{CorGDJQooNEIvpp}. Si nous sommes dans le cas où \( \lim_{t\to t_{max}} \| y(t) \|=\infty\) alors la dérivée de \( y\) n'est pas non plus bornée. Correct ?
    \item 
        Théorème de point fixe et équation différentielle. Que penser de l'exemple \ref{EXooJXIGooQtotMc} qui itère la contraction de Cauchy-Lipschitz pour résoudre \( y'(t)=y(t)\), \( y(0)=1\) ? Est-ce que c'est générique comme comportement ? Est-ce que la convergence est efficace dans des cas moins triviaux ?
\end{enumerate}

%---------------------------------------------------------------------------------------------------------------------------
\subsection{Mes questions d'algèbre, géométrie.}
%---------------------------------------------------------------------------------------------------------------------------

\begin{enumerate}
    \item
        Est-ce qu'il existe une structure raisonnable d'espace vectoriel sur \( \eZ\) ? Est-ce qu'il existe des corps discrets infinis ? 

        Dans cette question, j'ai derrière la tête que dans un espace vectoriel topologique nous avons une notion de suite de Cauchy, définition \ref{DefZSnlbPc}. Donc dans ce cas la notion d'espace complet et une notion topologique. Or il y a l'exemple \ref{EXooNMNVooXyJSDm} qui donne deux distances sur \( \eN\), qui donnent la même topologique, mais l'un étant complet, l'autre non.

        Si il y avait une structure vectorielle sur \( \eN\), cela créerait une contradiction. Au moins au sens où la définition \ref{DefZSnlbPc} de suite de Cauchy «topologique» ne redonne pas la même que la notion «métrique» de la définition usuelle \ref{DEFooXOYSooSPTRTn}.
    \item
        La «décomposition en facteurs premiers» dans \( \eZ[i\sqrt{2}]\) que je donne dans l'exemple \ref{ExluqIkE} est-elle correcte ? En particulier le lemme \ref{LemTScCIv} ?
    \item
        Est-ce que la fin de la démonstration \ref{ThojCJpFW} avec cette histoire d'ensemble \( \{ \xi_k^q\tq q\in \eN \}\) fini est compréhensible ?
    \item
        Les représentations \emph{irréductibles} sont les modules \emph{indécomposables}. Quid des modules irréductibles ? C'est pas un peu dingue de ne pas utiliser le mot «irréductible» pour désigner les mêmes choses dans le cas des modules et celui des représentations ?
    \item
        Rendre rigoureuse la remarque \eqref{RemmQjZOA} qui dit que les matrices ont le polynôme minimal est égal au polynôme caractéristique sont denses dans les matrices.
    \item
        Soit \( \eL\) une extension algébrique de corps de \( \eK\) et \( a\in \eL\). Est-ce que le polynôme minimal de \( a\) dans \( \eK[X]\) est l'unique polynôme irréductible unitaire \( P\in \eK[X]\) tel que \( P(a)=0\) ? D'ailleurs, est-ce qu'un tel polynôme existe ? est unique ? J'utilise cela dans la proposition \ref{PropUmxJVw}.
    \item
        La partie initiation de récurrence (\( r=2\)) de la preuve de la proposition \ref{PropSVvAQzi} à propos de convexe et de barycentre est-elle correcte ? Ce passage de l'espace affine à l'espace vectoriel sous-jacent me paraît un peu facile.
    \item
        Le lemme \ref{LemUELTuwK} à propos de PGCD de polynômes est-il correct ? J'imagine que le polynôme \( \pgcd(P,PU+R)\) ne dépend pas de \( U\). Est-ce exact ?
    \item
        Est-ce que parler de sous-groupe «normal» au lieu de «distingué» est un anglicisme ?
    \item
        Est-ce que l'énoncé et la démonstration de la proposition \ref{PropyMTEbH} sont corrects ? Si \( a\) et \( b\) sont des racines de \( P\), alors \( \mu_a\mu_b\) divise \( P\) (si \( \mu_a\neq \mu_b\)). Cette proposition est utilisée dans la démonstration de l'irréductibilité des polynômes cyclotomiques (proposition \ref{PropoIeOVh}).
    \item
        À quoi sert l'hypothèse «autre que \( \eF_2\)» dans le lemme \ref{LemcDOTzM} ? Peut-être dans la notion de déterminant parce qu'en caractéristique \( 2\), l'antisymétrie d'une forme linéaire n'implique le fait qu'elle soit alternée.
    \item
        L'inversibilité de la somme de Gauss (proposition \ref{PropciRUov}) est-elle bien démontrée ?
    \item
        Des commentaires sur l'exemple \ref{ExfUqQXQ} qui montre que \( X^p-X+1\) est irréductible sur \( \eF_p\).
    \item
        Les idéaux de \( A/I\) sont en bijection avec les idéaux de \( A\) contenant \( I\). Justification de l'équation \eqref{EqKbrizu}.
    \item
        L'énoncé et la démonstration d'une des multiples version du théorème de l'élément primitif, proposition \ref{PropNsLqWb}.
    \item
        Pourquoi la pseudo-réduction simultanée (corollaire \ref{CorNHKnLVA}) est-elle \emph{pseudo} ? Pourtant les matrices sont bel et bien simultanément diagonalisées.
    \item
        À propos d'extensions algébriques, est-ce que la proposition \ref{PropURZooVtwNXE} est correcte ? Est-ce qu'implicitement, il n'y a pas un sur-corps de \( \eK\) dans lequel il faut travailler ?
    \item
        Est-ce que la proposition \ref{PropRARooKavaIT} disant qu'un polynôme minimal est irréductible et premier avec tout polynôme non annulateur est correcte ?
    \item
        À propos de construction à la règle et au compas. Pour l'addition d'angles, l'exemple \ref{ExOVDooXnWPDl} explique comment on construit la somme de deux angles. Le problème est que cette construction se fait par intersection de deux cercles. Une des deux intersections donne \( \alpha+\beta\) et l'autre donne \( \alpha-\beta\). Comment par construction peut-on choisir le bon point ?
    \item
        À propos de chiffrement RSA, quelle est la probabilité que le message \( M\) ne soit pas premier avec \( p\) ? Est-ce que Alice (qui est celle qui chiffre avec la clef de Bob) peut le vérifier ? Que penser des points que j'énumère à la page \pageref{PageAKTBooMDeQxY} au dessus du problème \ref{ProbGAYFooZATuYy} ?
    \item
        Extension des scalaires. Si \( \{ e_i \}_i\) est une base du \( \eK\)-espace vectoriel \( E\), alors \( \{ 1\otimes e_i \}\) est une base de \( E_{\eL}=\eL\otimes_{\eL}E\) où \( \eL\) est une extension de \( KK\). C'est vrai ? Que penser de la preuve donné au lemme \ref{PROPooMHARooUycAts} ?
    \item
        Si \( f\colon V\to V\) est une application linéaire, on ne peut pas définir la transposée \( f^t\colon V\to V\) comme étant l'application linéaire dont la matrice est la transposée de la matrice de \( f\) parce que cette définition n'est pas invariante par changement de base. C'est vrai ? Voir le problème \ref{PROBooWNTKooNErOvt}.
    \item
        En géométrie projective, dans la sphère de Riemann \( \hat\eC=P_1(\eC)=\eC\cup\{ \infty \}\) est-ce qu'il existe une notion de cercle dont le centre est \( \infty\) ? Voir le point \ref{NORMooCXVJooMTMqEU}.
    \item
        Géométrie projective. Tout le monde semble définir le birapport en identifiant \( P(\eK^2)\) à \( \hat\eK=\eK\cup\{ \infty \}\). Bien entendu, personne ne semble s'être attribué la mission d'expliciter la dépendance du birapport en le choix de l'identification. Je le fais à la définition \ref{DEFooBFSKooDwzwmO}.

        Mais cette définition dépend du choix d'identification \( \varphi\colon P(\eK^2)\to \hat\eK\), comme le montre l'exemple \ref{EXooYCOYooWFSfUv}. J'ai donc défini des classes d'identifications possibles \( A(\varphi)\) en \ref{DEFooMLQUooGwvQMh}. Et je démontre la proposition \ref{PROPooTFMQooIOQGvs} que si \( \varphi_a\in A(\varphi)\) alors les birapports construits à partir de \( \varphi\) et \( \varphi_a\) sont identiques.

        Question : pourquoi personne ne semble faire ce travail ? En quoi l'identification \( \varphi_0\) que tout le monde utilise est plus canonique qu'une autre ? Est-ce que l'on peut décrire simplement les classes \( A(\varphi)\) ? Le groupe qui conserve le birapport associé à \( \varphi\) est-il isomorphe au groupe qui conserve le birapport associé à \( \varphi'\) ? Quels que soient \( \varphi\) et \( \varphi'\) ?

        Suis-je la seule personne au monde à m'être demandé si le birapport était un objet canonique ?
    \item
        En géométrie projective, dans \( P_1(\eC)=\hat\eC=\eC\cup\{ \infty \}\), si \( \ell\) est une droite dans \( \eC\), est-ce que la droite correspondante dans \( \hat\eC\) contient le point \( \infty\) ? Moi j'ai envie de dire que \( \infty\) est sur toutes les droites. Voir le problème \ref{PROBooZHHTooIFNwxR} et la remarque \ref{REMooBMAEooHDvNID}.

    \item
        Géométrie affine, barycentre. Les mauvaise langues diraient que tout cela est du snobisme autour de la paresse d'écrire \( \vect{ xy }\) au lieu de \( y-x\). Est-ce qu'il y a des cas où toute l'étude des espaces affines et des barycentres en particulier apportent \emph{réellement} plus qu'une facilité d'écriture par rapport à travailler dans le cadre vectoriel pur ?
    \item
        Isomorphisme du corps \( \eR\). Que penser de la remarque \ref{REMooGHEDooOYYUPk} ?
\end{enumerate}

%---------------------------------------------------------------------------------------------------------------------------
\subsection{Mes questions de probabilité et statistiques.}
%---------------------------------------------------------------------------------------------------------------------------

\begin{enumerate}

    \item
        Est-ce que le théorème d'arrêt de Doob \ref{ThoZTrdjtZ} est correctement énoncé ? En particulier la seconde condition. 
    \item
        À propos du problème de la ruine du joueur. Dans \cite{KXjFWKA}, l'équation \eqref{EqABPXmgr} vient avec une \( \limsup\) et non une limite normale. Je ne comprends pas pourquoi.

    \item
        Soit une variable aléatoire \( X\) à valeurs réelles. Est-ce que la tribu engendrée par \( X\) est d'une façon ou d'une autre engendrée par les «courbes de niveau» de \( X\) ? C'est à dire par les \( X^{-1}\big( \{ t \} \big)\) pour les \( t\in \eR\). 

        C'est ce qui semble ressortir de l'exemple de \ref{SUBSECooWOOGooVxflVZ}. Et intuitivement, je trouve que ça irait bien \ldots
\end{enumerate}

%--------------------------------------------------------------------------------------------------------------------------- 
\subsection{Numérique}
%---------------------------------------------------------------------------------------------------------------------------

\begin{enumerate}
    \item
        L'erreur de cancellation provoquée par la différence \( a-\tilde a\) lorsque \( a\) et \( \tilde a\) n'a pas de conséquences sur l'ordre de grandeur de la réponse. Seulement des conséquences sur la valeur des chiffres significatifs. Vrai ou faux ?

        Voir la remarque \ref{REMooRQIJooNLdAZE}.
\end{enumerate}
 

%TODO : dans l'index, le «é» doit aller avec le «e» et non au début de l'alphabet.

%--------------------------------------------------------------------------------------------------------------------------- 
\subsection{Les preuves à relire par des experts}
%---------------------------------------------------------------------------------------------------------------------------

Ici lorsque je parle d'«expert», j'entends quelqu'un qui est sûr de son coup dans un domaine particulier. Soit quelqu'un qui donne un cours dans le domaine, soit un étudiant qui a eu une bonne note dans un cours du domaine. Je m'adresse donc à des personnes qui ont déjà lu et compris une preuve des résultats \emph{avant} de les lire ici.

Les preuves des résultats suivants sont en tout ou en partie des inventions personnelles, et doivent donc être lues avec beaucoup de prudence. Si vous les lisez, que vous ne voyez pas de fautes et que vous êtes sûr de votre coup, écrivez moi. Si au contraire, vous y trouvez des fautes ou des raccourcis, écrivez moi même si vous n'êtes pas sûr de votre coup.
\begin{enumerate}
    \item
        La proposition \ref{PropDerrSSIntegraleDSD}.
        \begin{proposition}[Dérivation sous l'intégrale]
    Supposons $A\subset\eR^m$ ouvert et $B\subset\eR^n$ compact. Nous considérons une fonction \( f\colon A\times B\to \eR\). Si pour un $i\in\{ i,\ldots,n \}$, la dérivée partielle $\frac{ \partial f }{ \partial x_i }$ existe dans $A\times B$ et est continue, alors la fonction
    \begin{equation}
        F(x)=\int_Bf(x,t)dt
    \end{equation}
    admet une dérivée partielle dans la direction \( x_i\) sur \( A\). Cette dérivée partielle y est continue et
    \begin{equation}
        \frac{ \partial F }{ \partial x_i }(a)=\int_B\frac{ \partial f }{ \partial x_i }(a,t)dt,
    \end{equation}
    pour tout \( a\) dans l'ouvert \( A\).
\end{proposition}
        \item
            La proposition \ref{PropooVXPMooGSkyBo}
            \begin{proposition}
    Soit un espace mesuré \( (S,\tribF,\mu)\) et une fonction mesurable positive \( w\colon S\to \bar\eR^+\). Alors la formule
    \begin{equation}
        (w\cdot \mu)(A)=\int_Awd\mu
    \end{equation}
    pour tout \( A\in \tribF\) définit une mesure positive sur \( (S,\tribF)\) appelée \defe{produit}{produit!d'une mesure par une fonction} de la mesure \( \mu\) par la fonction \( w\). La fonction \( w\) est la \defe{densité}{densité!mesure} de la mesure \( w\cdot \mu\) par rapport à la mesure \( \mu\).
            \end{proposition}
        \item
            La proposition \ref{PROPooOEHTooHyjuZQ} et la remarque \ref{REMooBEXGooLgpHzg}.

Soit \( \{ e_i \}\) une base de \( E \) et \(\{ e_a \}\) une  de\( F\). Nous noterons également \( e_i\) et \( e_a\) les éléments \( \tau e_i\) et \( \tau e_a\) correspondants. Si la fonction \( f\colon E_{\eL}\to F_{\eL}\) s'écrit dans ce ces bases comme
\begin{equation}
    f(e_i)=\sum_af_{ai}e_a
\end{equation}
alors nous définissons \( \pr(f)\) par 
\begin{equation}        \label{EQooUWJLooTsbGZj}
    (\pr f)e_i=\sum_a\pr(f_{ai})e_a.
\end{equation}

            \begin{proposition}
    L'application \( \pr\) définie en \eqref{EQooUWJLooTsbGZj} est indépendante du choix des bases.
            \end{proposition}

        \item La proposition \ref{PROPooXVZMooXcJrsJ}
            \begin{proposition}
    Soit \( \eL\) une extension du corps \( \eK\) et une application linéaire \( f\colon E\to F\) entre deux \( \eK\)-espaces vectoriels. Alors \( \mu_f=\mu_{f_{\eL}}\).
\end{proposition}
\item
    Le lemme \ref{LemAZGByEs}

    \begin{lemma}
    Résultats sur les unions croissantes d'ensembles mesurables dans \( (S,\tribA,\mu)\).
    \begin{enumerate}
        \item
        Si \( (A_k)\) est une suite croissante d'ensembles \( \mu\)-mesurables dont l'union est mesurable, alors
        \begin{equation}
            \lim_{n\to \infty} \mu(A_k)=\mu(\bigcup_kA_k).
        \end{equation}

    \item
        Soit \( K_n\), une suite emboîtée d'éléments de \( \tribA\) tels que \( K_n\to S\). Si \( A\in\tribA\) alors
        \begin{equation}
            \lim_{n\to \infty} \mu(A\cap K_n)=\mu(A).
        \end{equation}
    \end{enumerate}
\end{lemma}

\item Le lemme \ref{LemYFoWqmS}

    \begin{lemma}
    Soit \( f\colon (S,\tribF)\to \bar\eR\) une fonction positive mesurable. Il existe une suite \( f_n\colon S\to \eR\) de fonctions étagées positives telles que \( f_n\to f\) ponctuellement et \( f_n \leq f\).
\end{lemma}

\item    La proposition \ref{PROPooAHMIooVpunrF}

    \begin{proposition}
    Soit la suite de rationnels \( (x_n)\) définie par \( x_0\in \eQ^+\) et 
    \begin{equation}
        x_{n+1}=x_n+\frac{ x_n^2-2 }{ 2x_n }.
    \end{equation}
    Alors :
    \begin{enumerate}
        \item
            En posant \( y_n=x_n^2\) nous avons \( y_n\to 2\).
        \item
            La suite \( (x_n)\) est de Cauchy.
        \item
            La suite \( (x_n)\) ne converge pas dans \( \eQ\).
    \end{enumerate}
\end{proposition}

\item La proposition \ref{PROPDEFooCWESooVbDven}

\begin{propositionDef}[Cercle circonscrit]
    Soit une courbe fermée simple et continue \( \gamma\colon \mathopen[ 0 , 1 \mathclose]\to \eR^2\). Soit \( \Gamma\) son image. Il existe un unique cercle de rayon minimum contenant \( \Gamma\). Ce cercle est le \defe{cercle circonscrit}{cercle!circonscrit à une courbe} à \( \gamma\).

    Il a les propriétés suivantes :
    \begin{enumerate}
        \item
            Le cercle circonscrit à \( \gamma\) coupe \( \Gamma\) en au moins deux points distincts.
        \item
            Tout arc du cercle circonscrit plus grand que le demi-cercle intersection \( \Gamma\).
    \end{enumerate}
\end{propositionDef}


\item La proposition \ref{PropFDDHooEyYxBC}
    \begin{proposition}
    Soient des entiers positifs \( \alpha_1,\ldots, \alpha_p\). Nous avons
    \begin{equation}
        \bigcup_{i=1}^pU_{\alpha_i}=\{ 1 \}
    \end{equation}
    si et seulement si \( \pgcd(\alpha_1,\ldots, \alpha_p)=1\) (c'est à dire que les \( \alpha_i\) sont premiers dans leur ensemble).
\end{proposition}

\item   La proposition \ref{PropRARooKavaIT}

    \begin{proposition}
    Soit \( \eL\) un extension de \( \eK\) et \( a\in \eL\) dont le polynôme minimal sur \( \eK\) est \( \mu_a\in\eK[X]\). Alors
    \begin{enumerate}
        \item   
            le polynôme \( \mu_a\) est irréductible sur \( \eK\);
        \item
            si \( Q\in \eK[X]\) est un polynôme non annulateur de \( a\) alors \( Q\) et \( \mu_a\) sont premiers entre eux.
    \end{enumerate}
\end{proposition}

\item La proposition \ref{PropURZooVtwNXE}


\begin{proposition}[Propriétés d'extensions algébriques\cite{MonCerveau}]  
    Soit \( \eK\) un corps commutatif et \( a\) un élément algébrique sur \( \eK\), de polynôme minimal \( \mu_a\) de degré \( n\). Alors
    \begin{enumerate}
        \item
            En considérant l'application d'évaluation
            \begin{equation}
                \begin{aligned}
                    \varphi_a\colon \eK[X]&\to \eL \\
                    Q&\mapsto Q(a), 
                \end{aligned}
            \end{equation}
            nous avons \( \eK[a]=\Image(\varphi_a)\).
        \item
            Une base de \( \eK[a]\) comme espace vectoriel sur \( \eK\) est donnée par \( \{ 1,a,a^2,\ldots, a^{n-1} \}\).
        \item
            Le degré de l'extension \( \eK[a]\) est égal au degré du polynôme minimal :
            \begin{equation}
                \big[ \eK[a]:\eK \big]=n.
            \end{equation}
         \item
            L'anneau \( \eK[a]\) est l'ensemble des polynômes en \( a\) de degré \( n-1\) à coefficient dans \( \eK\).
        \item
            \( \eK(a)=\eK[a]\).
        \item 
            \( \eK[a]\simeq\eK[X]/(\mu_a)\) (isomorphisme d'anneau).
    \end{enumerate}
\end{proposition}

\item  La proposition \ref{PropSLCUooUFgiSR}
\begin{proposition}[\cite{MonCerveau}] 
    Quel que soit le réel \( r\), il existe une suite croissante de rationnels convergente vers \( r\).
\end{proposition}

\item Le lemme \ref{LemBWSUooCCGvax}

\begin{lemma}
    Soit \( (X,\tau_X)\) un espace topologique et \( S\subset X\), un fermé de \( X\) sur lequel nous considérons la topologie induite \( \tau_S\). Si \( F\) est un fermé de \( (S,\tau_S)\) alors \( F\) est fermé de \( (X,\tau_X)\).
\end{lemma}
            
\item Le lemme \ref{LemCKBooXkwkte}
    \begin{lemma}
    Si \( K_1\) et \( K_2\) sont des compacts dans \( \eR\) alors \( K_1\times K_2\) est compact dans \( \eR^2\).
\end{lemma}

\item La proposition \ref{PropFnContParSuite}

    \begin{proposition}[Caractérisation séquentielle de la continuité]
    Le lien entre continuité et continuité séquentielle.

    \begin{enumerate}
        \item
    Une application continue est séquentiellement continue (quels que soient les espaces de départ et d'arrivée).

\item
    
    Si \( X\) et \( Y\) sont des espaces métriques, alors une fonction \( f\colon X\to Y\) est continue en un point si et seulement si elle est y séquentiellement continue.
    \end{enumerate}
\end{proposition}

\item Le théorème \ref{THOIYmxXuu}

\begin{theorem}
    Un produit fini d'espaces métriques non vides est compact si et seulement si chacun de ses facteurs est compact.
\end{theorem}

\item Le lemme \ref{LemooNFRIooPWuikH}
\begin{lemma}
    Tout élément de \( G\) s'écrit de façon unique sous la forme \( a^{\epsilon}b^k\) ou \( b^ka^{\epsilon}\) avec \( \epsilon=0,1\) et \( k=0,\ldots, n-1\).
\end{lemma}

\item La proposition \ref{PropMTmBYeU}


\begin{proposition}
    Si la fonction réelle \( f\colon I=\mathopen[ a , b [\to \eR\) est croissante et bornée, alors la limite
    \begin{equation}
        \lim_{x\to b} f(x)
    \end{equation}
    existe et est finie.
\end{proposition}

\item La proposition \ref{PropYKwTDPX}

\begin{proposition}
    Une fonction dérivable sur un intervalle \( I\) de \( \eR\) 
    \begin{enumerate}
        \item
            est convexe si et seulement si sa dérivée est croissante sur \( I\).
        \item
            est strictement convexe si et seulement si sa dérivée est strictement croissante sur \( I\)
    \end{enumerate}
\end{proposition}

\item La proposition \ref{PropNIBooSbXIKO}

\begin{proposition}
    Soit \( f\colon \eR\to \eR \) une fonction convexe et \( a\in \eR\). Il existe une constante \( c_a\in \eR\) telle que pour tout \( x\) nous ayons
    \begin{equation}
        f(x)-f(a)\geq c_a(x-a).
    \end{equation}
    Autrement dit, le graphe de la fonction \( f\) est toujours au dessus de la droite d'équation
    \begin{equation}
        y=f(a)+c_a(x-a).
    \end{equation}
\end{proposition}

Je me demande si ce ne serait pas un «si et seulement si».
\item La proposition \ref{PropPEJCgCH}
\begin{proposition}
    Si \( g\) est une fonction convexe, il existe deux suites réelles \( (a_n)\) et \( (b_n)\) telles que
    \begin{equation}
        g(x)=\sup_{n\in \eN}(a_nx+b_n).
    \end{equation}
\end{proposition}

\item

    Le théorème \ref{ThoooEZLGooMChwLT}

\begin{theorem}
    Soit \( Q\) un compact connexe par arcs et une fonction continue \( f\colon Q\to \eR\). Si \( \lambda\) est la mesure de Lebesgue, alors il existe \( a\in Q\) tel que
    \begin{equation}
        f(a)=\frac{1}{ \lambda(Q) }\int_Qfd\lambda
    \end{equation}
\end{theorem}

\item Le lemme \ref{LEMooGEVEooHxPTMO}


\begin{lemma}
    Soit une courbe simple, convexe \( \gamma\colon \mathopen[ 0 , 1 \mathclose]\to \eR^2\) que nous supposons être de classe \( C^2\). Nous notons \( \Gamma\) l'image de \( \gamma\). Alors \( \Gamma\) est localement le graphe d'une fonction convexe.
\end{lemma}

\item La proposition \ref{PROPooWZITooTFiWsi}.

\begin{proposition}
    Soit une courbe simple, fermée et convexe \( \gamma\colon \mathopen[ 0 , 1 \mathclose]\to \eR^2\) que nous supposons être de classe \( C^2\). Nous notons \( \Gamma\) l'image de \( \gamma\). Alors il existe un convexe \( D\) tel que \( \partial D=\Gamma\).
\end{proposition}

\item La proposition \ref{PropSNMEooVgNqBP}

\begin{proposition}
    Si la série entière \( \sum_{n\geq 0}a_nz^n\) a un rayon de convergence \( R\) alors
    \begin{enumerate}
        \item
            La somme est une fonction holomorphe dans le disque de convergence.
        \item      
            La somme est différentiable et
            \begin{equation}
                du_{z_0}(z)=\sum_{n=1}^{\infty}na_nz_0^{n-1}z.
            \end{equation}
        \item
    De plus pour tout \( z_0\in B(0,R)\), on pose\footnote{Pour rappel, dans tout ce texte, \( B(a,r)\) est une boule \emph{ouverte}.}
    \begin{subequations}
        \begin{align}
            S(z)&=\sum_{n\geq 0}a_nz^n\\
            T(z)&=\sum_{n\geq 1}na_nz^{n-1}=\sum_{n=0}^{\infty}(n+1)a_{n+1}z^n.
        \end{align}
    \end{subequations}
    Alors  nous avons
    \begin{equation}   
        \lim_{z\to z_0}\frac{ S(z)-S(z_0) }{ z-z_0 }=T(z_0).
    \end{equation}
    \end{enumerate}
\end{proposition}

\item Le corollaire \ref{CorCBYHooQhgara}

\begin{corollary}
    La somme d'une série entière est de classe \( C^{\infty}\) sur le disque ouvert de convergence.
\end{corollary}

\item La proposition \ref{PROPooWXBAooAEweSF}

\begin{proposition}
    Soit \( f\colon \eR^2\to \eR\) une application continue dont la variable \( y\) varie dans un compact \( I\) de \( \eR\). Alors la fonction
    \begin{equation}
        \begin{aligned}
            d\colon \eR&\to \eR \\
            x&\mapsto \sup_{y\in I} f(x,y) 
        \end{aligned}
    \end{equation}
    est continue.
\end{proposition}

À ce propos, est-il vrai que l'opération \( \sup\) vue comme application \( C(\eR^n)\to C(\eR^{n-1})\) est continue ? Pour quelle topologie sur les espaces de fonctions continues ?

\item La proposition \ref{PROPooCKTZooIPcUca}

\begin{proposition}   
    Soit \( \gamma\colon \mathopen[ 0 , L \mathclose]\to \eR^2\) une courbe fermée simple et convexe de classe \( C^1\). Si la tangente en \( p=\gamma(s_p)\) et la tangente en \( q=\gamma(s_q)\) sont identiques (pas seulement parallèles), alors soit 
    \begin{equation}
        \gamma\big( \mathopen[ s_p , s_q \mathclose] \big)=[p,q]
    \end{equation}
    soit
    \begin{equation}
        \gamma\big( \mathopen[ s_q , L \mathclose] \big)=[p,q]
    \end{equation}
\end{proposition}

\item
    Le corollaire \ref{CORooZFPSooHCFUSH}

    \begin{corollary}       \label{CORooPHJZooWDBuCQ}
    Soit \( \varphi\) une fonction Schwartz sur \( \eR^m\times \eR^n\). Alors la fonction
    \begin{equation}
        y\mapsto \sup_{x\in \eR^n}| \varphi(x,y) |
    \end{equation}
    est intégrable.
\end{corollary}

\item La proposition \ref{PROPooMVQMooGYAzSX}


\begin{proposition}
    Soit \( \varphi\in\swS(\eR^n\times \eR^m)\) et la transformée de Fourier partielle
    \begin{equation}
        \tilde \varphi(x,k)=\int_{\eR^m}  e^{-iky}\varphi(x,y)dy.
    \end{equation}
    Alors \( \tilde \varphi\in\swS(\eR^n\times \eR^m  )\).
\end{proposition}
Cette proposition utilise le corollaire \ref{CORooPHJZooWDBuCQ}.

\item Le théorème du flot d'une équation différentielle

    Le théorème \ref{THOooSTHXooXqLBoT} qui donne la régularité \( C^1\) du flot d'une équation de transport est à relire attentivement. Il y a d'inextricables problèmes de notations dont j'espère m'être extirpé, mais sans garanties.

\end{enumerate}

%+++++++++++++++++++++++++++++++++++++++++++++++++++++++++++++++++++++++++++++++++++++++++++++++++++++++++++++++++++++++++++ 
\section{Comment aider à rendre ces notes plus utiles ?}
%+++++++++++++++++++++++++++++++++++++++++++++++++++++++++++++++++++++++++++++++++++++++++++++++++++++++++++++++++++++++++++

%--------------------------------------------------------------------------------------------------------------------------- 
\subsection{Niveau facile}
%---------------------------------------------------------------------------------------------------------------------------

Niveau facile : un étudiant de licence devrait pouvoir le faire. Ne demande pas de toucher au codes \LaTeX.

\begin{enumerate}
    \item
        M'écrire pour me signaler toutes les fautes que vous voyez, même si vous n'êtes pas sûr.
    \item
        Si vous n'êtes pas expert, me signaler tous le endroits qui vous semblent obscurs. Vu que ces notes sont destinées à \emph{appendre}, les avis des non experts sont très importants.
    \item
        Mettre une copie de (ou un lien vers) ces notes sur votre site.
\end{enumerate}

%--------------------------------------------------------------------------------------------------------------------------- 
\subsection{Niveau moyen}
%---------------------------------------------------------------------------------------------------------------------------

Niveau moyen : un candidat à l'agrégation de mathématique devrait pouvoir le faire. Demande de toucher un peu au code \LaTeX.

\begin{enumerate}
    \item
        Donner une liste de résultats qui devraient être connue pour l'agrégation. En particulier, remplir le fichier \info{98\_liste\_developpements}
    \item
        Mettre de l'ordre dans les énoncés des développements de Taylor. En particulier, donner des énoncés corrects pour des fonctions \( \eR^n\to \eR^m\). Faire une référence vers le résultat qu'il faut là où se trouve la référence  \info{382218354}, proposition \ref{PropoExtreRn}\ref{ITEMooCBMYooQQMqQL}.
\end{enumerate}

%--------------------------------------------------------------------------------------------------------------------------- 
\subsection{Niveau difficile}
%---------------------------------------------------------------------------------------------------------------------------

Demande probablement de connaissances avancées en mathématique (au moins être tout à fait à l'aise avec le niveau d'agrégation), et de toucher assez bien \LaTeX.

\begin{enumerate}
    \item
        Si vous savez comment faire \info{pdf --> epub} pour créer un eBook, faites le moi savoir. Cahier des charges :
        \begin{itemize}
            \item libre, disponible sur Ubuntu
            \item en ligne de commande (en tout cas : exécutable depuis un script en python ou C++)
        \end{itemize}
        Attention : le Frido étant un truc assez compliqué, avant de répondre la première chose qui vous passe par la tête, assurez-vous que votre solution fait avancer les choses sur le Frido et non sur un petit document de test.
    \item
        Répondre aux questions ci-dessus.
    \item
        Répondre aux choses dites «problème et choses à faire»  un peu partout.
    \item
        Un peu partout des \verb+%TODO+ sont placés dans les sources \LaTeX. Ils décrivent des choses qu'il serait bon de faire. Si vous avez un avis sur l'un d'eux, n'hésitez pas à me le faire savoir.
    \item 
        L'environnement \info{example} est un sale hack pour placer le triangle. Faire mieux.
\end{enumerate}

