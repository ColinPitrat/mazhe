% This is part of Mes notes de mathématique
% Copyright (c) 2011-2018
%   Laurent Claessens
% See the file fdl-1.3.txt for copying conditions.

%+++++++++++++++++++++++++++++++++++++++++++++++++++++++++++++++++++++++++++++++++++++++++++++++++++++++++++++++++++++++++++ 
\section{Groupe des permutations}
%+++++++++++++++++++++++++++++++++++++++++++++++++++++++++++++++++++++++++++++++++++++++++++++++++++++++++++++++++++++++++++

\begin{definition}      \label{DEFooJNPIooMuzIXd}
    Soit un ensemble \( E\). Une \defe{permutation}{permutation} de l'ensemble \( E\) est une bijection \( E\to E\). Le \defe{groupe symétrique}{groupe!symétrique} de \( E\) le groupe des bijections \( E\mapsto E\); il est noté \( S_E\).

    Le \defe{groupe symétrique}{groupe!symétrique} \( S_n\)\nomenclature[R]{\( S_n\)}{le groupe symétrique} est le groupe des permutations de l'ensemble \( \{ 1,\ldots,n \}\). C'est donc l'ensemble des bijections \( \{ 1,\ldots, n \}\to\{ 1,\ldots, n \}\).
\end{definition}

\begin{lemma}[\cite{ooJFLYooKMbycW}]        \label{LEMooUPBOooWbwMTx}
    Deux résultats.
    \begin{enumerate}
        \item
            Tout groupe est isomorphe à une sous-groupe d'un groupe symétrique.
        \item
            Tout groupe fini d'ordre \( n\) est isomorphe à un sous-groupe de \( S_n\).
    \end{enumerate}
\end{lemma}

\begin{proof}
    Soit, pour \( g\in G\) donné, l'application
    \begin{equation}
        \begin{aligned}
            \tau_g\colon G&\to G \\
            x&\mapsto gx. 
        \end{aligned}
    \end{equation}
    Cela est une bijection de \( G\). En effet \( \tau_g(x)=y\) pour \( x=g^{-1} y\) (surjectif) et \( \tau_g(x)=\tau_g(y)\) impllique \( gx=gy\) et donc \( x=y\) (injection).

    Nous avons donc \( \tau_g\in S_G\). De plus l'application
    \begin{equation}
        \begin{aligned}
            \varphi\colon G&\to S_G \\
            g&\mapsto \tau_g 
        \end{aligned}
    \end{equation}
    est un morphisme de groupe. Il est injectif parce que si \( \tau_g=\tau_h\) alors \( gx=hx\) pour tout \( x\). En particulier \( g=h\).

    Donc \( \varphi\colon G\to \Image(\varphi)\) est un isomorphisme entre \( G\) et un sous-groupe de \( S_G\).

    Un groupe fini de cardinal \( n\) est isomorphe à un sous-groupe de \( S_G\); or \( S_G\) est isomorphe à un des \( S_n\).
\end{proof}

%+++++++++++++++++++++++++++++++++++++++++++++++++++++++++++++++++++++++++++++++++++++++++++++++++++++++++++++++++++++++++++
\section{Produit semi-direct de groupes}
%+++++++++++++++++++++++++++++++++++++++++++++++++++++++++++++++++++++++++++++++++++++++++++++++++++++++++++++++++++++++++++

\begin{definition}
    Une \defe{suite exacte}{suite!exacte} est une suite d'applications comme suit :
    \begin{equation}
        \xymatrix{%
        \cdots \ar[r]^{f_i}&A_i\ar[r]^{f_{i+1}}& A_{i+1}\ar[r]^{f_{i+2}}&\ldots
           }
    \end{equation}
    où pour chaque \( i\), les application \( f_i\) et \( f_{i+1}\) vérifient \( \ker(f_{i+1})=\Image(f_i)\). Lorsque les ensembles \( A_i\) sont des groupes, alors nous demandons de plus que les \( f_i\) soient de homomorphismes.
\end{definition}

Très souvent nous sommes confrontés à des suites exactes de la forme
\begin{equation}
    \xymatrix{%
    1 \ar[r]& A\ar[r]^f&G\ar[r]^g&B\ar[r]&1
       }
\end{equation}
où \( G\), \( A\) et \( B\) sont des groupes, \( 1\) est l'identité. La première flèche est l'application \( \{ 1 \}\to A\) qui à \( 1\) fait correspondre \( 1\). La dernière est l'application \( B\to 1\) qui à tous les éléments de \( B\) fait correspondre \( 1\). Le noyau de \( f\) étant l'image de la première flèche (c'est à dire \( \{ 1 \}\)), l'application \( f\) est injective. L'image de \( g\) étant le noyau de la dernière flèche (c'est à dire \( B\) en entier), l'application \( g\) est surjective.

\begin{definition}     \label{DEFooKWEHooISNQzi}
    Soient \( N\) et \( H\) deux groupes et un morphisme de groupes \( \phi\colon H\to \Aut(N)\). Le \defe{produit semi-direct}{produit!semi-direct} de \( N\) et \( H\) relativement à \( \phi\), noté \( N\times_{\phi}H\)\nomenclature[R]{\( N\times_{\phi}H\)}{produit semi-direct} est l'ensemble \( N\times H\) muni de la loi (que l'on vérifiera être de groupe)
    \begin{equation}\label{EqDRgbBI}
        (n,h)\cdot (n',h')=(n\phi_h(n'),hh').
    \end{equation}
\end{definition}
Attention à l'ordre quelque peu contre intuitif. Lorsque nous notons \( N\times_{\phi}H\), c'est bien \( \phi\colon H\to \Aut(N)\), c'est à dire \( H\) qui agit sur \( N\) et non le contraire.

Lorsque \( N\) et \( H\) sont des sous-groupes d'un même groupe, le plus souvent \( \phi\) est l'action adjointe définie en \ref{DEFooCORTooEeOLPT}.

Le théorème suivant permet de reconnaître un produit semi-direct lorsqu'on en voit un.
\begin{theorem}[\cite{MathAgreg}]       \label{THOooZNYTooPhnIdE}
    Soit une suite exacte de groupes
    \begin{equation}
    \xymatrix{%
    1 \ar[r]        & N\ar[r]^i&G\ar[r]^s&H\ar[r]&1
       }
    \end{equation}
    S'il existe un sous-groupe \( \tilde H\) de \( G\) à partir duquel \( s\) est un isomorphisme, alors
    \begin{equation}
        G\simeq i(N)\times_{\sigma}\tilde H
    \end{equation}
    où \( \sigma\) est l'action adjointe\footnote{Le fait que \( H\) agisse sur \( i(N)\) fait partie du théorème.} de \( \tilde H\) sur \( i(N)\).
\end{theorem}

\begin{proof}
    Nous posons \( \tilde N=i(N)\) et nous allons subdiviser la preuve en petits pas.

    \begin{enumerate}
        \item  \( \tilde N\) est normal dans \( G\). En effet étant donné que la suite est exacte nous avons \( \tilde N=\ker(s)\). Le noyau d'un morphisme est toujours un sous-groupe normal.

        \item \( \tilde N\cap\tilde H=\{ e \}\). L'application \( s\) étant un isomorphisme depuis $\tilde H$, il n'y a pas d'éléments de \( \tilde H\) dans \( \ker(s)\) autre que $e$.
    
        \item\label{ItemzIaXGM} \( G=\tilde N\tilde H\). Nous considérons \( g\in G\) et \( h\in \tilde H\) tel que \( s(g)=s(h)\). L'existence d'un tel \( h\) est assurée par le fait que \( s\) est surjective depuis \( \tilde H\). Du coup nous avons \( e=s(gh^{-1})\), c'est à dire \( gh^{-1}\in \ker (s)=\tilde N\). Nous avons donc bien la décomposition \( g=(gh^{-1})h\), et donc \( G=\tilde N\tilde H\).

        \item\label{ItemUGFjle} L'écriture \( g=nh\) avec \( n\in \tilde N\) et \( h\in \tilde H\) est unique. Si \( nh=n'h'\), alors \( n=n'h'h^{-1}\), ce qui signifierait que \( h'h^{-1}\in\tilde N\). Mais étant donné que \( \tilde H\cap\tilde N=\{ e \}\), nous obtenons \( h=h'\) et par suite \( n=n'\).

        \item   \label{ItemUZlrKo}
            L'application
            \begin{equation}
                \begin{aligned}
                    \phi\colon G&\to \tilde N\times \tilde H \\
                    nh&\mapsto (n,h) 
                \end{aligned}
            \end{equation}
            est une bijection. C'est une conséquence des points \ref{ItemzIaXGM} et \ref{ItemUGFjle}.

        \item
            Si sur \( \tilde N\times \tilde H\) nous mettons le produit
            \begin{equation}
                (n,h)\cdot(n',h')=(n\sigma_hn',hh')
            \end{equation}
            où \( \sigma\) est l'action adjointe du groupe sur lui-même, c'est à dire \( \sigma_x(y)=xyx^{-1}\), alors \( \phi\) est un isomorphisme. Si \( g,g'\in G\) s'écrivent (de façon unique par le point \ref{ItemUZlrKo}) \( g=nh\) et \( g'=n'h'\) alors
            \begin{subequations}
                \begin{align}
                    \phi(nhn'h')&=\phi(n\underbrace{hn'h^{-1}}_{\in \tilde N}hh')\\
                    &=\phi\big( (nhn'h^{-1})(hh') \big)\\
                    &=(nhn'h^{-1},hh')\\
                    &=(n,h)\cdot(n',h')\\
                    &=\phi(nh)\phi(n'h').
                \end{align}
            \end{subequations}
    \end{enumerate}
\end{proof}

\begin{corollary}\label{CoroGohOZ}
    Soit \( G\) un groupe, et \( N,H\) des sous-groupes de \( G\) tels que
    \begin{enumerate}
        \item
            \( H\) normalise \( N\) (c'est à dire que \( hnh^{-1}\in N\) pour tout \( h\in H\) et \( n\in N\)\footnote{Ou encore que \( H\) agit sur \( N\) par automorphismes internes.}),
        \item
            \( H\cap N=\{ e \}\),
        \item
            \( HN=G\).
    \end{enumerate}
    Alors l'application
    \begin{equation}
        \begin{aligned}
            \psi\colon N\times_{\sigma}H&\to G \\
            (n,h)&\mapsto nh 
        \end{aligned}
    \end{equation}
    est un isomorphisme de groupes.
\end{corollary}
Dans les hypothèses, l'ordre entre \( N\) et \( H\) est important lorsqu'on dit que c'est \( N\) qui agit sur \( H\); mais l'hypothèse \( NH=G\) est équivalente à \( HN=G\) (passer à l'inverse pour s'en assurer).

Insistons encore un peu sur la notation : dans \( N\times_{\sigma}H\), c'est \( H\) qui agit sur \( N\) par \( \sigma\).

%+++++++++++++++++++++++++++++++++++++++++++++++++++++++++++++++++++++++++++++++++++++++++++++++++++++++++++++++++++++++++++
\section{Groupe de torsion}
%+++++++++++++++++++++++++++++++++++++++++++++++++++++++++++++++++++++++++++++++++++++++++++++++++++++++++++++++++++++++++++

Soit \( G\) un groupe. Un élément \( g\in G\) est un \defe{élément de torsion}{element@élément!de torsion} s'il est d'ordre fini. La \defe{torsion}{torsion!d'un groupe} de \( G\) est l'ensemble de ses éléments de torsion. Nous disons qu'un groupe est un \defe{groupe de torsion}{groupe!de torsion} si tous ses éléments sont de torsion.

\begin{example}
    Le groupe additif \( \eQ/\eZ\) est un groupe de torsion parce que si \( [x]=[p/q]\), alors \( q[x]=[p]=[0]\).
\end{example}

\section{Famille presque nulle}
%+++++++++++++++++++++++++++++++++++++++++++++++++++++++++++++++++++++++++++++++++++++++++++++++++++++++++++++++++++++++++++

Soit \( (G,+)\) un groupe abélien et \( \mF=\{ g_i \}_{i\in I}\) une famille d'éléments de \( G\) indicés par un ensemble \( I\). Le \defe{support}{support!famille d'éléments} de \( \mF\) est l'ensemble \( \{ i\in I\tq g_i\neq 0 \}\). La famille est dite \defe{presque nulle}{presque!nulle} si le support est fini.

Nous disons que \( \mF\) est une \defe{suite}{suite} si \( I=\eN\).

%+++++++++++++++++++++++++++++++++++++++++++++++++++++++++++++++++++++++++++++++++++++++++++++++++++++++++++++++++++++++++++ 
\section{Le groupe des racines de l'unité dans les nombres complexes}
%+++++++++++++++++++++++++++++++++++++++++++++++++++++++++++++++++++++++++++++++++++++++++++++++++++++++++++++++++++++++++++
\label{SecGJOLooWdMYVl}

\begin{definition}
    Une \defe{racine \( n\)\ieme de l'unité}{racine!de l'unité} est une racine du polynôme \( X^n-1\).
\end{definition}

\begin{lemma}
    Les racines \( n\)\ieme de l'unité dans \( \eC\) sont les élément du groupe multiplicatif
    \begin{equation}        \label{EqIEAXooIpvFPe}
        \gU_n=\{  e^{2i\pi k/n}  \tq k=0,\ldots, n-1 \}
    \end{equation}
    \nomenclature[A]{\( \gU_n\)}{Le groupe des racines \( n\)\ieme de l'unité.}
\end{lemma}

\begin{proof}
    Il est vite vu que tous les éléments de \( \gU_n\) sont des racines de l'unité parce que
    \begin{equation}
        \big(  e^{2i\pi k/n} \big)^n= e^{2i\pi k}=1,
    \end{equation}
    entre autres à cause du lemme \ref{LEMooHOYZooKQTsXW}.

    Cela nous donne déjà \( n\) racines pour \( X^n-1\) dans \( \eC\). Le théorème \ref{ThoLXTooNaUAKR} nous indique qu'il ne peut pas y en avoir plus.
\end{proof}

Un des intérêts du groupe des racines est qu'il permet de factoriser \( X^n-1\), comme nous le verrons via les polynômes cyclotomiques dans le lemme \ref{LemKYGBooAwpOHD}.

\begin{lemma}       \label{LemWHQGooXyeJiw}
    L'ensemble \( U_n\) est un groupe cyclique\footnote{Définition \ref{DefHFJWooFxkzCF}.} d'ordre \( n\) généré par \( \xi= e^{2i\pi/n}\).
\end{lemma}

\begin{proof}
    Il y a les trois propriétés à vérifier pour que ce soit un groupe.
    \begin{subproof}
        \item[Neutre]
            Le nombre \( 1\) est une racine de l'unité.
    \item[Inverse]
        Si \( \omega\in U_n\) alors \( \omega^n=1\) et donc \( \omega\omega^{n-1}=1\), ce qui signifie que \( \omega^{n-1}\) est un inverse de \( \omega\). Il reste à voir que \( \omega^{n-1}\in U_n\). En effet \(  \big( \omega^{n-1} \big)^n=(\omega^n)^{n-1}=1^{n-1}=1  \).
    \item[Associativité]
        Cas particulier de l'associativité dans \( \eC\).
    \end{subproof}
    Le fait que ce soit un groupe cyclique contenant \( n\) éléments est la définition.
\end{proof}


Le lemme suivant donne les autres générateurs.
\begin{lemma}   \label{LemcFTNMa}
    Le nombre \( \xi^a\) est un générateur de \( \gU_n\) si et seulement si \( \pgcd(a,n)=1\).
\end{lemma}

\begin{proof}
    Si \( \pgcd(a,n)=1\) alors le théorème de Bézout \ref{ThoBuNjam} nous fournit des entiers \( u\) et \( v\) tels que \( ua+vn=1\). Alors nous avons
    \begin{equation}
        e^{2i\pi /n}= e^{2(ua+vn)i\pi/n}=( e^{2ai\pi/n})^u,
    \end{equation}
    ce qui signifie que \( \xi\) est dans le groupe engendré par \( \xi^a\), et par conséquent tout \( \gU_n\) est engendré.

    Pour l'implication inverse, nous utilisons Bézout dans le sens inverse. Soit \( \xi^a\) un générateur de \( \gU_n\). Alors il existe \( u\) tel que \( (\xi^a)^u=\xi\), donc \( \xi^{au-1}=1\), c'est à dire qu'il existe \( v\) tel que \( au-1=vn\). Cette dernière égalité implique que \( \pgcd(a,n)=1\).
\end{proof}

\begin{example}
Une conséquence tout à fait extraordinaire de ce lemme est que \( 7\) est générateur de \( \eZ/12\eZ\) (parce que \( \pgcd(7,12)=1\)). Or en solfège\index{solfège}, une quinte fait \( 7\) demi-tons, et une gamme en fait 12. Le cycle des quintes est donc générateur de la gamme\cite{YDXsAM}. Ce fait est connu des pianistes\footnote{Même ceux qui ignorent le théorème de Bézout.} depuis des siècles.
\end{example}

\begin{proposition}[Intersection par deux]      \label{PROPooIOQEooGMcCJm}
    Les ensembles \( U_{\alpha}\) et \( U_{\beta}\) ont une intersection réduite à \( \{ 1 \}\) si et seulement si \( \alpha\) et \( \beta\) sont premiers entre eux.
\end{proposition}

\begin{proof}
    Nous rappelons qu'une racine \( \alpha\)\ieme de l'unité peut s'écrire sous la forme \(  e^{2i\pi k/\alpha}\) avec \( 0\leq k<\alpha\).
    \begin{subproof}
    \item[Sens direct]
        Par contraposée, nous supposons que \( \alpha\) et \( \beta\) ne sont pas premiers entre eux, et nous notons \( d\) leur \( \pgcd\). Nous nommons \( \alpha=d\alpha'\) et \( \beta=d\beta'\). Pour trouver une intersection entre \( U_{\alpha}\) et \( U_{\beta}\) nous devons trouver une valeur de \( 0<k<\alpha\) telle que
        \begin{equation}
            ( e^{2i\pi k/\alpha})^{\beta}= e^{2i\pi k\beta/\alpha}=1,
        \end{equation}
        c'est à dire une valeur de \( k\) telle que \( k\beta/\alpha\) soit un entier. Mais \( k\beta/\alpha=k\beta'/\alpha'\) et par conséquent prendre \( k=\alpha'\) fonctionne. Surtout que par hypothèse \( d>1\) et donc \( k=\alpha'<\alpha\).
    \item[Sens réciproque]
        Supposons maintenant que \( \alpha\) et \( \beta \) soient premiers entre eux. Soit \( z\in U_{\alpha}\cap U_{\beta}\). Le fait que \( z\) soit une racine \( \alpha\)\ieme de l'unité implique qu'il existe un \( k<\alpha\) tel que \( z= e^{2i\pi k/\alpha}\). Mais si \( z\) est également une racine \( \beta\)\ieme de l'unité, alors \( z^{\beta}=1\), c'est à dire que \( k\beta/\alpha\) doit être un entier, soit \( l\) cet entier. Nous avons
        \begin{equation}
            k\beta=l\alpha.
        \end{equation}
		Si \( k>0\), comme le nombre \( \alpha\) divise \( k\beta\), cela conduirait via le lemme de Gauss \ref{LemPRuUrsD} à dire que \( \alpha\) divise \( k\). Mais \( \alpha\) ne peut pas diviser \( k\) parce que nous avions supposé que \( k\) était strictement plus petit que \( \alpha\). Donc \( k = 0\) et \( z = 1\).
    \end{subproof}
\end{proof}

\begin{proposition}[Intersection : le cas général\cite{MonCerveau}]  \label{PropFDDHooEyYxBC}
    Soient des entiers positifs \( \alpha_1,\ldots, \alpha_p\). Nous avons
    \begin{equation}
        \bigcup_{i=1}^pU_{\alpha_i}=\{ 1 \}
    \end{equation}
    si et seulement si \( \pgcd(\alpha_1,\ldots, \alpha_p)=1\) (c'est à dire que les \( \alpha_i\) sont premiers dans leur ensemble).
\end{proposition}

\begin{proof}
    Nous le décomposons les \( \alpha_i\) en facteurs premiers\footnote{Théorème \ref{ThoAJFJooAveRvY}.} de la façon suivante : \( \alpha_i=\prod_{k\in \eN}p_k^{\alpha_i^{(k)}}\) où les \( p_k\) sont les nombres premiers. 
    
    \begin{subproof}
    \item[Caractérisation par une décomposition en facteurs premiers]
        Les éléments \( z\) différents de \( 1\) dans \( U_{\alpha_1}\) s'écrivent sous la forme
        \begin{equation}
            z= e^{2i\pi k/\alpha_1}
        \end{equation}
        avec \( 0<k<\alpha_1\).

        Pour tout \( i\neq 1\), le fait que \( z\in U_{\alpha_i}\cap U_{\alpha_1}\) se traduit par le fait que \( \big(  e^{2i\pi k/\alpha_1} \big)^{\alpha_i}=1\), c'est à dire que \( \alpha_ik/\alpha_1\) est entier, donc que \( \alpha_1\) divise \( k\alpha_i\). Par conséquent il existera un élément différent de \( 1\) dans l'intersection des \( U_{\alpha_i}\) si et seulement s'il existe un entier \( k\) strictement compris entre \( 0\) et \( \alpha_1\) pour lequel \( \alpha_1\) divise tous les \( k\alpha_i\).

        Un entier \( 0<k<\alpha_1\) convient si et seulement si pour tout \( l\), la puissance de \( p_l\) dans la décomposition de \( k\) est au moins égale à
        \begin{equation}
            \alpha_1^{(l)}-\alpha_i^{(l)}
        \end{equation}
        pour tout \( l\).
    \item[Sens direct]
        L'hypothèse \( \pgcd(\alpha_1,\ldots, \alpha_p)\neq 1\) implique qu'il existe un \( l\) pour lequel tous les \( \alpha_i^{(l)}\) sont non nuls. Nous construisons le \( k\) voulu en prenant pour tout \( p_i\) la même puissance que celle dans \( \alpha_1\), sauf pour \( p_l\) pour lequel nous prenons la puissance \(  \alpha_1^{(l)}-\min_i\{   \alpha_i^{(l)} \} \). Le minimum en question est strictement positif, ce qui donne un \( k\) strictement inférieur à \( \alpha_1\).
    \item[Sens réciproque]
        Si \( \pgcd(\alpha_1,\ldots, \alpha_p)=1\) alors pour tout \( l\), il existe un \( i\) tel que \( \alpha_i^{(l)}=0\). Donc pour tout \( l\), la puissance de \( p_l\) dans la décomposition de \( k\) est au moins \( \alpha_1^{(l)}\). Cela implique que \( k\geq \alpha_1\), ce qui est impossible.
    \end{subproof}
\end{proof}

\begin{definition}\label{DefLYGTooFPOYGZ}
    Les générateurs de \( \gU_n\) sont les \defe{racines primitives}{racine!de l'unité!primitive}\footnote{parce qu'en prenant les puissances successives de l'une d'entre elles, nous retrouvons toutes les racines de l'unité, voir aussi la définition \ref{DefnPNCFO}.} de l'unité dans \( \eC\). Nous nommons \( \Delta_n\) leur ensemble :
\begin{equation}
    \Delta_n=\{  e^{2ki\pi/n}\tq 0\leq k\leq n-1,\pgcd(k,n)=1 \}.
\end{equation}
\end{definition}
Nous avons par exemple
\begin{subequations}
    \begin{align}
        \Delta_1&=\{ 1 \}\\
        \Delta_2&=\{  e^{\pi i} \}\\
        \Delta_4&=\{  e^{\pi i/2}, e^{3\pi i/2} \}.
    \end{align}
\end{subequations}
Notons que \( 1\in \Delta_d\) seulement avec \( d=1\).

Pour rappel, les exponentielles complexes sont définies en \ref{}.


%+++++++++++++++++++++++++++++++++++++++++++++++++++++++++++++++++++++++++++++++++++++++++++++++++++++++++++++++++++++++++++
\section{Fonction indicatrice d'Euler (première partie)}
%+++++++++++++++++++++++++++++++++++++++++++++++++++++++++++++++++++++++++++++++++++++++++++++++++++++++++++++++++++++++++++

Nous introduisons ici la fonction indicatrice d'Euler et ses liens basiques avec les racines de l'unité. Pour les propriétés plus avancées, voir \ref{subSecKGDFooAbETjs}.

%---------------------------------------------------------------------------------------------------------------------------
\subsection{Introduction par les racines de l'unité}
%---------------------------------------------------------------------------------------------------------------------------

\begin{definition}      \label{DEFooWYIGooRVBTil}
La fonction \( \varphi\) donnée par
\begin{equation}    \label{EqEulerGqPsvi}
    \varphi(n)=\Card(\Delta_n)
\end{equation}
est l'\defe{indicatrice d'Euler}{indicatrice d'Euler}\index{Euler!indicatrice}.
\end{definition}
Si \( p\) est un nombre premier, alors \( \varphi(p)=p-1\).

\begin{lemma}       \label{LemKcpjee}
    Nous avons
    \begin{equation}        \label{EqpZuIyL}
        \gU_n=\bigcup_{d\divides n}\Delta_d
    \end{equation}
    et l'union est disjointe. Nous avons aussi la formule
    \begin{equation}        \label{EqTPHqgJ}
        n=\sum_{d\divides n}\varphi(d).
    \end{equation}
\end{lemma}

\begin{proof}
    À l'application \( x\mapsto  e^{2i\pi x}\) près, nous pouvons considérer
    \begin{equation}
        \Delta_d=\{ \frac{ k }{ d }\tq k=0,\ldots, d-1, \pgcd(k,d)=1 \},
    \end{equation}
    c'est à dire l'ensemble des fractions irréductibles dont le dénominateur est \( d\). L'union des \( \Delta_d\) sera donc disjointe.
    
    Toujours à l'application \( x\mapsto  e^{2i\pi x}\) près, le groupe \( \gU_n\) est donné par
    \begin{equation}
        \gU_n=\{ \frac{ k }{ n }\tq k=0,\ldots, n-1 \}.
    \end{equation}
    L'égalité \eqref{EqpZuIyL} revient maintenant à dire que toute fraction de la forme \( \frac{ k }{ n }\) s'écrit de façon irréductible avec un dénominateur qui divise \( n\).

    La relation \eqref{EqTPHqgJ} consiste à prendre le cardinal des deux côtés de \eqref{EqpZuIyL}. Nous avons \( \Card(\gU_n)=n\) et l'union étant disjointe, à droite nous avons la somme des cardinaux.


    Pour chaque diviseur \( d\) de \( n\) nous considérons l'ensemble
    \begin{equation}
        \Phi_n(d)=\{ m\in \eN\tq \pgcd(m,n)=\frac{ n }{ d } \}.
    \end{equation}
    Étant donné que tous les entiers entre \( 1\) et \( n\) ont un pgcd avec \( n\) qui est automatiquement un quotient de \( n\) nous avons
    \begin{equation}
        \{ 1,\ldots, n \}=\bigcup_{d\divides n}\Phi_n(d)
    \end{equation}
    où l'union est disjointe. Par ailleurs nous savons que si \( \pgcd(a,b)=1\), alors \( \pgcd(ka,kb)=k\). Donc si \( m\in \Delta_d\), alors \( m\cdot \frac{ n }{ d }\) appartient à \( \Phi_n(d)\). En d'autres termes, \( a\mapsto \frac{ n }{ d }a\) est une bijection entre \( \Delta_d\) et \( \Phi_n(d)\).

    Nous avons donc \( \Card(\Phi_n(d))=\Card(\Delta_d)=\varphi(d)\) et finalement
    \begin{equation}
        \Card\{ 1,\ldots, n \}=\sum_{d\divides n}\Card(\Phi_n(d))=\sum_{d\divides n}\varphi(d).
    \end{equation}
\end{proof}

\begin{lemma}
    Si \( p\) est un nombre premier, alors \( \varphi(p^n)=p^n-p^{n-1}\).
\end{lemma}

\begin{proof}
    Les éléments de \( \{ 1,\ldots,p^n \}\) qui ont un \( \pgcd\) différent de \( 1\) avec \( p^n\) sont des nombres qui s'écrivent sous la forme \( qp\) avec \( q\leq p^{n-1}\)\footnote{Corollaire \ref{CORooQIMHooUzLUJY}.}. Il y a évidemment \( p^{n-1}\) tels nombres.

    Par conséquent le cardinal de \( P_{p^n}\) est \( \varphi(p^{n})=p^n-p^{n-1}\).
\end{proof}

\begin{probleme}
    $P_n$ n'a pas été défini.

    Définition proposée (et vue par après): \( P_n = \{ m \in \eN \tq \pgcd(m,n) = 1 \}. \) À mettre donc en lien avec $\Delta_d$.
\end{probleme}

%--------------------------------------------------------------------------------------------------------------------------- 
\subsection{Générateurs}
%---------------------------------------------------------------------------------------------------------------------------

\begin{proposition}     \label{PropZnmuphiGensn}
    Soit \( n\in\eN^*\) et le groupe (additif) \( \eZ/n\eZ\). L'élément \( [x]_n\) est un générateur de \( \eZ/n\eZ\) si et seulement si \( x\in P_n\). En particulier \( \eZ/n\eZ\) est un groupe contenant \( \varphi(n)\) générateurs.
\end{proposition}

\begin{proof}
    Nous avons \( \gr\big( [1]_n \big)=\eZ/n\eZ\). L'élément \( [x]_n\) sera générateur si et seulement s'il génère \( [1]_n \), c'est à dire s'il existe \( u\) tel que \( u[x]_n=[1]_n\). Cette dernière égalité étant une égalité de classes dans \( \eZ/n\eZ\), elle sera vraie si et seulement s'il existe \( v\) tel que
    \begin{equation}
        ux+vn=1.
    \end{equation}
    Cela signifie entre autres que\footnote{Corollaire \ref{CorgEMtLj}} \( x\eZ+n\eZ=\eZ\), et aussi que \( \pgcd(x,n)=1\) par le théorème de Bézout \ref{ThoBuNjam}, et donc que \( x\in P_n\).
\end{proof}
