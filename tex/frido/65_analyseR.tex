% Copyright (c) 2008-2018
%   Laurent Claessens
% See the file fdl-1.3.txt for copying conditions.

%+++++++++++++++++++++++++++++++++++++++++++++++++++++++++++++++++++++++++++++++++++++++++++++++++++++++++++++++++++++++++++
\section{Espace des fonctions continues}
%+++++++++++++++++++++++++++++++++++++++++++++++++++++++++++++++++++++++++++++++++++++++++++++++++++++++++++++++++++++++++++

\begin{definition}
    Soit \( I\), un intervalle de \( \eR\). L'\defe{oscillation}{oscillation!d'une fonction} sur \( I\) est le nombre
    \begin{equation}
        \omega_f(I)=\sup_{x\in I}f(x)-\inf_{x\in I}f(x).
    \end{equation}
\end{definition}
    Pour chaque \( x\) fixé, la fonction
    \begin{equation}
        x\mapsto \omega_f\big( B(x,\delta) \big)
    \end{equation}
    est une fonction positive, croissante et a donc une limite (pour \( \delta\to 0\)). Nous notons \( \omega_f(x)\) cette limite qui est l'\defe{oscillation}{oscillation!d'une fonction en un point} de \( f\) en ce point. Une propriété immédiate est que \( f\) est continue en \( x_0\) si et seulement si \( \omega_f(x_0)=0\).

    \begin{lemma}       \label{LemuaPbtQ}
    L'ensemble des points de discontinuité d'une fonction \( f\colon \eR\to \eR\) est une réunion dénombrable de fermés.
\end{lemma}

\begin{proof}
    Soit \( D\) l'ensemble des points de discontinuité de \( f\). Nous avons
    \begin{equation}
        D=\bigcup_{n=1}^{\infty}\{ x\tq \omega_f(x)\geq \frac{1}{ n } \}.
    \end{equation}
    Il nous suffit donc de montrer que pour tout \( \epsilon\), l'ensemble
    \begin{equation}
        \{ x\tq \omega_f(x)<\epsilon \}
    \end{equation}
    est ouvert. Soit en effet \( x_0\) dans cet ensemble. Il existe \( \delta\) tel que \( \omega_f\big( B(x_0,\delta) \big)<\epsilon\). Si \( x\in B(x_0,\delta)\), alors si on choisit \( \delta'\) tel que \( B(x,\delta')\subset B(x_0,\delta)\), nous avons \( \omega_f\big( B(x,\delta') \big)<\epsilon\), ce qui justifie que \( \omega_f(x)<\epsilon\) et donc que \( x\) est également dans l'ensemble considéré.
\end{proof}

\begin{theorem}
    L'ensemble des points de discontinuité d'une limite simple de fonctions continues est de première catégorie.
\end{theorem}

\begin{proof}
    Soit \( (f_n)\) une suite de fonctions qui converge simplement vers \( f\). Nous devons écrire l'ensemble des points de discontinuité de \( f\) comme une union dénombrable d'ensembles tels que sur tout intervalle \( I\), aucun de ces ensembles n'est dense. Nous savons déjà par le lemme \ref{LemuaPbtQ} que l'ensemble des points de discontinuité  de \( f\) est donné par
    \begin{equation}
        D=\bigcup_{n=1}^{\infty}\{ x\tq \omega_f(x)\geq \frac{1}{  n } \}.
    \end{equation}
    Nous essayons donc de prouver que pour tout \( \epsilon\), l'ensemble 
    \begin{equation}
        F=\{ x\tq \omega_f(x)\geq \epsilon \}
    \end{equation}
    est nulle part dense. Soit
    \begin{equation}
        E_n=\bigcap_{i,j>n}\{ x\tq | f_i(x)-f_j(x) |<\epsilon \}.
    \end{equation}
    Nous montrons que cet ensemble est fermé en étudiant le complémentaire. Soit \( x\notin E_n\); alors il existe un couple \( (i,j)\) tel que
    \begin{equation}
        | f_i(x)-f_j(x) |>\epsilon.
    \end{equation}
    Par continuité, cette inégalité reste valide dans un voisinage de \( x\). Donc il existe un voisinage de \( x\) contenu dans \( \complement E_n\) et \( E_n\) est donc fermé.

    De plus nous avons \( E_n\subset E_{n+1}\) et \( \bigcup_nE_n=\eR\). Ce dernier point est dû au fait que pour tout \( x\), il existe \( N\) tel que \( i,j>N\) implique \( | f_i(x)-f_j(x) |\leq \epsilon\). Cela est l'expression du fait que la suite \( \big( f_n(x) \big)_{n\in \eN}\) est de Cauchy.

    Soit \( I\), un intervalle fermé de \( \eR\). Nous voulons trouver un intervalle \( J\subset I\) sur lequel \( f\) est continue. Nous écrivons \( I\) sous la forme 
    \begin{equation}
        I=\bigcup_{n=1}^{\infty}(E_n\cap I).
    \end{equation}
    Tous les ensembles \( J_n=E_n\cap I\) ne peuvent être nulle part dense en même temps (à cause du théorème de Baire \ref{ThoQGalIO}). Il existe donc un \( n\) tel que \( J_n\) contienne un ouvert \( J\). Le but est de montrer que \( f\) est continue sur \( J\). Pour ce faire, nous n'allons pas simplement majorer \( | f(x)-f(x_0) |\) par \( \epsilon\) lorsque \( | x-x_0 |\) est petit. Nous allons majorer l'oscillation de \( f\) sur \( B(x_0,\delta)\) lorsque \( \delta\) est petit. Pour cela nous prenons \( x_0\) et \( x\) dans \( J\) et nous écrivons
    \begin{equation}
        | f(x)-f(x_0) |\leq | f(x)-f_n(x) |+| f_n(x)-f_n(x_0) |.
    \end{equation}
    À ce niveau nous rappelons que \( n\) est fixé par le choix de \( J\), dans lequel \( \epsilon\) est déjà inclus. Nous choisissons évidemment \( | x-x_0 |\leq \delta\) de telle sorte que le second terme soit plus petit que \( \epsilon\) en vertu de la continuité de \( f_n\). Pour le premier terme, pour tout \( i,j\geq n\) nous avons
    \begin{equation}
        | f_i(x)-f_j(x) |<\epsilon.
    \end{equation}
    Si nous posons \( j=n\) et \( i\to\infty\), en tenant compte du fait que \( f_i\to f\) simplement,
    \begin{equation}
        | f(x)-f_n(x) |\leq \epsilon.
    \end{equation}
    Nous avons donc obtenu \( | f(x)-f_n(x_0) |\leq 2\epsilon\). Cela signifie que dans un voisinage de rayon \( \delta\) autour de \( x_0\), les valeurs extrêmes prises par \( f(x) \) sont \( f_n(x_0)\pm 4\epsilon\). Nous avons donc prouvé que pour tout \( \epsilon\), il existe \( \delta\) tel que
    \begin{equation}
        \omega_f\big( \mathopen[ x_0-\delta , x_0+\delta \mathclose] \big)\leq 4\epsilon.
    \end{equation}
    De là nous concluons que
    \begin{equation}
        \lim_{\delta\to 0}\omega_f\big( \mathopen[ x_0-\delta , x_0+\delta \mathclose] \big)=0,
    \end{equation}
    ce qui signifie que \( f\) est continue en \( x_0\).
\end{proof}

\begin{example}
    Une fonction discontinue sur \( \eQ\) et continue ailleurs. La fonction 
    \begin{equation}
        f(x)=\begin{cases}
            0    &   \text{si } x\notin \eQ\\
            \frac{1}{ q }    &    \text{si } x=p/q
        \end{cases}
    \end{equation}
    où par «\( x=p/q\)» nous entendons que \( p/q\) est la fraction irréductible.

    Cette fonction est discontinue sur \( \eQ\) parce que si \( q\in \eQ\) alors \( f(q)\neq 0\) alors que dans tous voisinage de \( q\) il existe un irrationnel sur qui la fonction vaudra zéro.

    Montrons que \( f\) est continue sur les irrationnels. Si \( x_0\notin \eQ\) alors \( f(x_0)=0\). Mais si on prend un voisinage suffisamment petit de \( x_0\), nous pouvons nous arranger pour que tous les rationnels aient un dénominateur arbitrairement grand. En effet si nous nous fixons un premier rayon \( r_0>0\) alors il existe un nombre fini de fractions de la forme \( 1\), \( \frac{ k }{2}\), \( \frac{ k }{ 3 }\),\ldots, \( \frac{ k }{ N }\) dans \( B(x_0,r_0)\). Il suffit maintenant de choisir \( 0<r\leq r_0\) tel que ces fractions soient toutes hors de \( B(x_0,r)\). Dans cette boule nous avons \( f<\frac{1}{ N }\). Du coup \( f\) est continue en \( x_0\).
\end{example}

\begin{definition}[Point périodique]
    Soit \( f\colon I\to I\) une application d'un ensemble \( I\) dans lui-même. Si \( x\in I\) vérifie \( f^n(x)=x\) et \( f^k(x)\neq x\) pour \( k=1,\ldots, n-1\) alors on dit que \( x\) est un point \( n\)-périodique.
\end{definition}

\begin{lemma}       \label{LemAONBooGZBuYt}
    Soit \( I\) un segment\footnote{définition \ref{DefLISOooDHLQrl}. Un segment est un intervalle fermé borné.} de \( \eR\) et une fonction continue \( f\colon I\to I\). Si \( K\) est un segment fermé avec \( K\subset f(I)\) alors il existe un segment fermé \( L\subset I\) tel que \( K=f(L)\).
\end{lemma}

\begin{proof}
    Mentionnons immédiatement que \( f\) est continue sur \( I\) qui est compact\footnote{Par le lemme \ref{LemOACGWxV}.}. Par conséquent tous les nombres dont nous allons parler sont finis parce que \( f\) est bornée par le théorème \ref{ThoMKKooAbHaro}.

    Soit \( K=\mathopen[ \alpha , \beta \mathclose]\). Si \( \alpha=\beta\) alors le segment \( L=\{ a \}\) convient. Nous supposons donc que \( \alpha\neq \beta\) et nous considérons \( a,b\in I\) tels que \( \alpha=f(a)\) et \( \beta=f(b)\). Vu que \( a\neq b\) nous supposons \( a<b\) (le cas \( a>b\) se traite de façon similaire).

    Nous posons
    \begin{equation}
        A=\{ x\in\mathopen[ a , b \mathclose]\tq f(x)=\alpha \}.
    \end{equation}
    C'est un ensemble borné par \( a\) et \( b\). De plus il est fermé; ce dernier point n'est pas tout à fait évident parce que \( f\) n'est pas définit sur \( \eR\) mais sur \( I\) qui est fermé, le corollaire \ref{CorNNPYooMbaYZg} n'est donc pas immédiatement utilisable. Prouvons donc que \( Z=\{ x\in \eR\tq f(x)=\alpha \}\) est fermé. Si \( x_0\) est hors de \( Z\) alors soit \( x_0\) est dans \( I\) soit il est hors de \( I\). Dans ce second cas, le complémentaire de \( I\) étant ouvert, on a un voisinage de \( x_0\) hors de \( I\) et par conséquent hors de \( Z\). Si au contraire \( x_0\in I\) alors il y a (encore) deux cas : soit \( x_0\in\Int(I)\) soit \( x_0\) est sur le bord de \( I\). Dans le premier cas, le théorème des valeurs intermédiaires\footnote{Théorème \ref{ThoValInter}.} fonctionne. Pour le second cas, nous supposons \( x_0=\max(I)\) (le cas \( x_0=\min(I)\) est similaire). Le théorème des valeurs intermédiaires dit que sur \( \mathopen[ x_0-\epsilon , x_0 \mathclose]\), \( f\neq \alpha\) et en même temps, sur \( \mathopen] x_0 , x_0+\epsilon \mathclose]\), nous sommes en dehors du domaine. Au final \( \{ f(x)=\alpha \}\) est fermé et \( A\) est alors fermé en tant que intersection de deux fermés.

    L'ensemble \( A\) étant non vide (\( a\in A\)), il possède donc un maximum que nous nommons \( u\) :
    \begin{equation}
        u=\max(A).
    \end{equation}
    Nous posons aussi
    \begin{equation}
        B=\{ x\in \mathopen[ u , b \mathclose]\tq f(x)=\beta \}
    \end{equation}
    qui est encore fermé, borné et non vide. Nous pouvons donc définir
    \begin{equation}
        v=\min(B).
    \end{equation}
    Nous prouvons maintenant que \( f\big( \mathopen[ u , v \mathclose] \big)=\mathopen[ \alpha , \beta \mathclose]\). D'abord \( f\big( \mathopen[ u , v \mathclose] \big)\) est un intervalle compact\footnote{Corollaire \ref{CorImInterInter} et théorème \ref{ThoImCompCotComp}.} contenant \( f(u)=\alpha\) et \( f(v)=\beta\). Par conséquent \( \mathopen[ \alpha , \beta \mathclose]\subset f\big( \mathopen[ u , v \mathclose] \big)\). Pour l'inclusion inverse supposons \( t\in \mathopen[ u , v \mathclose]\) tel que \( f(t)>\beta\). Vu que \( f(a)=\alpha\) et \( \alpha<\beta\) le théorème des valeurs intermédiaires il existe \( t_0\in \mathopen[ a , t \mathclose]\) tel que \( f(t_0)=\beta\). Cela donne \( t_0<v\) et donc contredit la minimalité de \( v\) dans \( B\). Nous en déduisons que \( f\big( \mathopen[ u , v \mathclose] \big)\) ne contient aucun élément plus grand que \( \beta\). Même jeu pour montrer que ça ne contient aucun élément plus petit que \( \alpha\).

    En définitive, le segment \( L=\mathopen[ u , v \mathclose]\) fonctionne.
\end{proof}

Lorsque \( I_2\subset f(I_1)\) nous notons \( I_1\to I_2\) ou, si une ambiguïté est à craindre, \( I_1\stackrel{f}{\longrightarrow}I_2\). Cette flèche se lit «recouvre».
\begin{lemma}[\cite{PAXrsMn}]      \label{LemSSPXooMkwzjb}
    Soient les segments \( I_0,\ldots, I_{n-1}\) tels que nous ayons le cycle 
    \begin{equation}
        I_0\to I_1\to\ldots\to I_{n-1}\to I_0.
    \end{equation}
    Alors \( f^n\) admet un point fixe \( x_0\in I_0\) tel que \( f^k(x_0)\in I_k\) pour tout \( k=0,\ldots, n-1\).
\end{lemma}

\begin{proof}
    Nous prouvons les cas \( n=1\) et \( n=2\) séparément.
    \begin{subproof}
    \item[\( n=1\)]
        Nous avons \( I_0\to I_0\), c'est à dire que $I_0\subset f(I_0)$. Si \( I_0=\mathopen[ a , b \mathclose]\) alors nous posons \( a=f(\alpha)\) et \( b=f(\beta)\) pour certains \( \alpha,\beta\in I_0\). Nous posons ensuite \( g(x)=f(x)-x\).

        Dans un premier temps, \( g(\alpha)=a-\alpha\leq 0\) parce que \( a=\in(I_0)\) et \( \alpha\in I_0\). Pour la même raison, \( g(\beta)=b-\beta\geq 0\). Le théorème des valeurs intermédiaires donne alors \( t_0\in \mathopen[ \alpha , \beta \mathclose]\subset I_0\) tel que \( g(t_0)=0\). Nous avons donc \( f(t_0)=t_0\).
    \item[\( n=2\)]
        Nous avons \( I_0\to I_1\to I_0\). Vu que \( I_1\subset f(I_0)\), le lemme \ref{LemAONBooGZBuYt} donne un segment \( J_1\subset I_0\) tel que \( f(J_1)=I_1\). Mézalors
        \begin{equation}
            J_1\subset I_0\subset f(I_1)=f^2(J_1).
        \end{equation}
        Nous avons donc \( J_1\stackrel{f^2}{\longrightarrow}J_1\) et par le cas \( n=1\) traité plus haut, la fonction \( f^2 \) a un point fixe \( x_0\) dans \( J_1\). De plus
        \begin{equation}
            f(x_0)\in f(J_1)=I_1,
        \end{equation}
        le point \( x_0\) est donc bien celui que nous cherchions.
    \item
        Cas général. Nous avons
        \begin{equation}
            I_0\to I_1\to\ldots\to I_{n-1}\to I_0.
        \end{equation}
        Vu que \( I_1\subset f(I_0)\), il existe \( J_1\subset I_0\) tel que \( f(J_1)=I_1\). Mais
        \begin{equation}
            I_2\subset f(I_1)=f^2(J_1),
        \end{equation}
        donc il existe \( J_2\subset J_1\) tel que \( I_2=f^2(J_2)\). En procédant encore longtemps ainsi nous construisons les ensembles \( J_1,\ldots, J_{n-1}\) tels que
        \begin{equation}
            J_{n-1}\subset J_{n-2}\subset\ldots\subset J_1\subset J_0
        \end{equation}
        tels que \( I_k=f^k(J_k)\) pour tout \( k=1,\ldots, n-1\). La dernière de ces inclusions est \( I_{n-1}=f^{n-1}(J_{n-1})\), mais \( I_{n-1}\to I_0\), c'est à dire que
        \begin{equation}
            I_0\subset f(I_{n-1})=f^n(J_{n-1}),
        \end{equation}
        et il existe \( J_n\subset J_{n-1}\) tel que \( I_0\subset f^n(J_n)\). Mais comme \( J_n\subset J_0\) nous avons en particulier \( J_n\subset f^n(J_n)\).

        Cela donne un point fixe \( x_0\in J_n\) pour \( f^n\). Par construction nous avons \( J_n\subset J_{n-1}\subset\ldots\subset J_1\subset J_0\) et donc \( x_0\in J_k\) pour tout \( k\). En  particulier 
        \begin{equation}
            f^k(x_0)\in f^k(J_k)=I_k
        \end{equation}
        pour tout \( k\).
    \end{subproof}
\end{proof}

\begin{theorem}[Théorème de Sarkowski\cite{PAXrsMn}]
    Soit \( I\), un segment de \( \eR\) et une application continue \( f\colon I\to I\). Si \( f\) admet un point \( 3\)-périodique, alors \( f\) admet des points \( n\)-périodiques pour tout \( n\geq 1\).
\end{theorem}

\begin{proof}
    Soit \( a\in I\) un point \( 3\)-périodique pour \( f\) et notons \( b=f(a)\), \( c=f(b)\). Les points \( b\) et \( c\) sont également des points \( 3\)-périodiques. Quitte à renommer, nous pouvons supposer que \( a\) est le plus petit des trois. Il reste deux possibilités : \( a<b<c\) et \( a<c<b\). Nous traitons d'abord le premier cas.

    Supposons \( a<b<c\). Nous posons \( I_0=\mathopen[ a , b \mathclose]\) et \( I_1=\mathopen[ b , c \mathclose]\). Nous avons immédiatement \( I_1\subset f(I_0)\) et comme \( f(b)=c\) et \( f(c)=a\), \( f(I_1)\) recouvre \( \mathopen[ a , c \mathclose]\) et donc recouvre en même temps \( I_1\) et \( I_2\). Nous avons donc \( I_0\to I_1\), \( I_1\to I_0\) et \( I_1\to I_1\).
    \begin{subproof}
    \item[Un point \( 1\)-périodique]
        Nous avons \( I_1\to I_1\) qui prouve que \( f\) a un point fixe dans \( I_1\). C'est le cas \( n=1\) du lemme \ref{LemSSPXooMkwzjb}. Voilà un point \( 1\)-périodique.
    \item[Un point \( 2\)-périodique]
        Nous avons \( I_0\to I_1\to I_0\). Par conséquent, le lemme \ref{LemSSPXooMkwzjb} dit que \( f^2\) a un point fixe \( x_0\in I_0\) tel que \( f(x_0)\in I_1\). Montrons que \( f(x_0)\neq x_0\). Pour avoir \( x_0=f(x_0)\), il faudrait \( x_0\in I_0\cap I_1=\{ b \}\). Mais \( b\) est un point \( 3\)-périodique, donc ne vérifiant certainement pas \( f^2(b)=b\). Nous en déduisons que \( f(x_0)\neq x_0\) et donc que \( x_0\) est \( 2\)-périodique.
    \item[Un point \( 3\)-périodique]
        On en a par hypothèse.
    \item[Un point \( n\)-périodique pour \( n\geq 4\)]
        Nous avons le cyle
        \begin{equation}
            I_0\to \underbrace{I_1\to I_1\to\ldots\to I_1}_{\text{n-1} fois}\to I_0.
        \end{equation}
        Le lemme donne alors un point fixe \( x\in I_0\) pour \( f^n\) tel que \( f^k(x)\in I_1\) pour \( k=1,\ldots, n-1\). Est-ce possible que \( x=b\) ? Non parce que \( f^2(b)=a\in I_0\) alors que \( f^2(x)\in I_1\). Mais \( I_0\cap I_1=\{ b \}\).

        Par conséquent la relation \( f^k(x)\in I_1\) exclu d'avoir \( f^k(x)=x\), et le point \( x\) est bien \( n\)-périodique.
    \end{subproof} 
    
    Passons au cas \( a<c<b\). Alors nous posons \( I_0=\mathopen[ a , c \mathclose]\) et \( I_1=\mathopen[ c , b \mathclose]\). Encore une fois \( f(I_0)\) contient \( a\) et \( b\), donc \( I_0\to I_0\) et \( I_0\to I_1\). Mais en même temps \( f(I_1)\) contient \( a\) et \( c\), donc \( I_1\to I_0\).

    Nous pouvons donc refaire comme dans le premier cas, en inversant les rôles de \( I_0\) et \( I_1\). En particulier nous pouvons considérer le cycle
    \begin{equation}
        I_1\to I_0\to I_0\to\ldots\to I_0\to I_1.
    \end{equation}
\end{proof}

%+++++++++++++++++++++++++++++++++++++++++++++++++++++++++++++++++++++++++++++++++++++++++++++++++++++++++++++++++++++++++++
\section{Uniforme continuité}		\label{SecUnifContinue}
%+++++++++++++++++++++++++++++++++++++++++++++++++++++++++++++++++++++++++++++++++++++++++++++++++++++++++++++++++++++++++++

\begin{definition}
	Une partie $A\subset\eR^m$ est dite \defe{bornée}{bornée!partie de $\eR^m$} s'il existe un $M>0$ tel que $A\subset B(0,M)$. Le \defe{diamètre}{diamètre} de la partie $A$ est\nomenclature[T]{$\Diam(A)$}{Diamètre de la partie $A$} le nombre
	\begin{equation}
		\Diam(A)=\sup_{x,y\in A}\| x-y \|\in\mathopen[ 0 , \infty \mathclose].
	\end{equation}
\end{definition}
Lorsque $A$ est borné, il existe un $M$ tel que $\| x \|\leq M$ pour tout $x\in A$.

\begin{lemma}
	Si $A$ est une partie non vide de $\eR^m$, alors $\Diam(A)=\Diam(\bar A)$.
\end{lemma}
Nous n'allons pas donner de démonstrations de ce lemme.


Si $(x_n)$ est une suite et $I$ est un sous-ensemble infini de $\eN$, nous désignons par $x_I$ la suite des éléments $x_n$ tels que $n\in I$. Par exemple la suite $x_{\eN}$ est la suite elle-même, la suite $x_{2\eN}$ est la suite obtenue en ne prenant que les éléments d'indice pair.

Les suites $x_I$ ainsi construites sont dites des \defe{sous-suites}{sous-suite} de la suite $(x_n)$.


Pour une fonction $f\colon D\subset\eR^m\to \eR$, la continuité au point $a$ signifie que pour tout $\varepsilon>0$,
\begin{equation}
	\exists\delta>0\tq 0<\| x-a \|<\delta\Rightarrow | f(x)-f(a) |<\varepsilon.
\end{equation}
Le $\delta$ qu'il faut choisir dépend évidemment de $\varepsilon$, mais il dépend en général aussi du point $a$ où l'on veut tester la continuité. C'est à dire que, étant donné un $\varepsilon>0$, nous pouvons trouver un $\delta$ qui fonctionne pour certains points, mais qui ne fonctionne pas pour d'autres points.

Il peut cependant également arriver qu'un même $\delta$ fonctionne pour tous les points du domaine. Dans ce cas, nous disons que la fonction est uniformément continue sur le domaine.

\begin{definition}
	Une fonction $f\colon D\subset\eR^m\to \eR$ est dite \defe{uniformément continue}{continue!uniformément} sur $D$ si
	\begin{equation}	\label{EqConditionUnifCont}
		\forall\varepsilon>0,\,\exists\delta>0\tq\,\forall x,y\in D,\,\| x-y \|\leq\delta \Rightarrow| f(x)-f(a) |<\varepsilon.
	\end{equation}
\end{definition}

Il est intéressant de voir ce que signifie le fait de \emph{ne pas} être uniformément continue sur un domaine $D$. Il s'agit essentiellement de retourner tous les quantificateurs de la condition \eqref{EqConditionUnifCont} :
\begin{equation}	\label{EqConditionPasUnifCont}
	\exists\varepsilon>0\tq\forall\delta>0,\,\exists x,y\in D\tq \| x-y \|<\delta\text{ et }\big| f(x)-f(y) \big|>\varepsilon.
\end{equation}
Dans cette condition, les points $x$ et $y$ peuvent être fonction du $\delta$. L'important est que pour tout $\delta$, on puisse trouver deux points $\delta$-proches dont les images par $f$ ne soient pas $\varepsilon$-proches.

\begin{example}
	Prenons la fonction $f(x)=\frac{1}{ x }$, et demandons nous pour quel $\delta$ nous sommes sûr d'avoir
	\begin{equation}
		| f(a+\delta)-f(a) |=\left| \frac{1}{ a+\delta }-\frac{1}{ a } \right| <\varepsilon.
	\end{equation}
	Pour simplifier, nous supposons que $a>0$. Nous calculons
	\begin{equation}
		\begin{aligned}[]
			\frac{ 1 }{ a }-\frac{1}{ a+\delta }&<	\varepsilon\\
			\frac{ \delta }{ a(a+\delta) }&<\varepsilon\\
			\delta&<\varepsilon a^2+\varepsilon a\delta\\
			\delta(1-\varepsilon a)&<\varepsilon a^2\\
			\delta&<\frac{ \varepsilon a^2 }{ 1-\varepsilon a }.
		\end{aligned}
	\end{equation}
	Notons que, à $\varepsilon$ fixé, plus $a$ est petit, plus il faut choisir $\delta$ petit. La fonction $x\mapsto\frac{1}{ x }$ n'est donc pas uniformément continue. Cela correspond au fait que, proche de zéro, la fonction monte très vite. Une fonction uniformément continue sera une fonction qui ne montera jamais très vite.
\end{example}

\begin{proposition}
	Quelques propriétés des fonctions uniformément continues.
	\begin{enumerate}
		\item
			Toute application uniformément continue est continue;
		\item
			la composée de deux fonctions uniformément continues est uniformément continue;
	\end{enumerate}
\end{proposition}
Nous verrons qu'une application lipschitzienne est uniformément continue (proposition \ref{PROPooVZSAooUneOQK}).

Une fonction peut être uniformément continue sur un domaine et pas sur un autre. Le théorème suivant donne une importante indication à ce sujet.
\begin{theorem}[Heine]\index{théorème!Heine}\index{Heine (théorème)}		\label{ThoHeineContinueCompact}
	Une fonction continue sur un compact (fermé et borné) est uniformément continue.
\end{theorem}

La démonstration qui suit est valable pour une fonction \( f\colon \eR^n\to \eR^m\) et utilise le fait que le produit cartésien de compacts est compact. Dans le cas de fonctions sur \( \eR\), nous pouvons modifier la démonstration pour ne pas utiliser ce résultat; voir plus bas.
%TODO : trouver où se trouve la preuve du produit de compacts et la référentier ici.
\begin{proof}
	Nous allons prouver ce théorème par l'absurde. Nous commençons par écrire la condition \eqref{EqConditionPasUnifCont} qui exprime que $f$ n'est pas uniformément continue sur le compact \( K\) :
	\begin{equation}
		\exists\varepsilon>0\tq\forall\delta>0,\,\exists x,y\in K\tqs \| x-y \|<\delta\text{ et }\big| f(x)-f(y) \big|>\varepsilon.
	\end{equation}
	En particulier (en prenant $\delta=\frac{1}{ n }$ pour tout $n$), pour chaque $n$ nous pouvons trouver $x_n$ et $y_n$ dans $K$ qui vérifient simultanément les deux conditions suivantes :
	\begin{subequations}
		\begin{numcases}{}
			\| x_n-y_n \|<\frac{1}{ n }\\
			\big| f(x_n)-f(y_n) \big|>\varepsilon.	\label{EqCond3107fxfyepsppt}
		\end{numcases}
	\end{subequations}
    Nous insistons que c'est le même $\varepsilon$ pour chaque $n$. L'ensemble $K$ étant compact, l'ensemble \( K\times K \) est compact (théorème \ref{THOIYmxXuu}) et nous pouvons trouver une sous-suite convergente \emph{du couple} \( (x_n,y_n)\) dans \( K\times K\). Quitte à passer à ces sous-suites, nous  nous supposons que \( (x_n,y_n)\) converge dans \( K\times K\) et en particulier que les suites $(x_n)$ et $(y_n)$ sont convergentes. Étant donné que pour chaque $n$ elles vérifient $\| x_n-y_n \|<\frac{1}{ n }$, les limites sont égales :
	\begin{equation}
		\lim x_n=\lim y_n=x.
	\end{equation}
	L'ensemble $K$ étant fermé, la limite $x$ est dans $K$. Par continuité de $f$, nous avons finalement
	\begin{equation}
		\lim f(x_n)=\lim f(y_n)=f(x),
	\end{equation}
	mais alors 
	\begin{equation}
		\lim_{n\to\infty}\big| f(x_n)-f(y_n) \big|=0,
	\end{equation}
	ce qui est en contradiction avec le choix \eqref{EqCond3107fxfyepsppt}.

	Tout ceci prouve que $f(K)$ est bornée supérieurement et que $f$ atteint son supremum (qui est donc un maximum). Le fait que $f(K)$ soit borné inférieurement se prouve en considérant la fonction $-f$ au lieu de $f$.

\end{proof}

\begin{remark}
    Nous pouvons ne pas utiliser le fait que le produit de compacts est compact. Cela est particulièrement commode lorsqu'on considère des fonctions de \( \eR\) dans \( \eR\) parce que dans ce cadre nous ne pouvons pas supposer connue la notion de produit d'espace topologiques.

    Pour choisir les sous-suites \( (x_n)\) et \( (y_n)\), il suffit de prendre une sous-suite convergente de \( (x_n)\) et d'invoquer le fait que \( \| x_n-y_n \|\leq \frac{1}{ n }\). Les suites \( (x_n)\) et \( (y_n)\) étant adjacentes, la convergence de \( (x_n)\) implique la convergence de \( (y_n)\) vers la même limite.

    Il est donc un peu superflus de parler de la convergence du couple \( (x_n,y_n)\).
\end{remark}

%+++++++++++++++++++++++++++++++++++++++++++++++++++++++++++++++++++++++++++++++++++++++++++++++++++++++++++++++++++++++++++
\section{Compacité}
%+++++++++++++++++++++++++++++++++++++++++++++++++++++++++++++++++++++++++++++++++++++++++++++++++++++++++++++++++++++++++++
%http://fr.wikipedia.org/wiki/Espace_compact
%http://fr.wikipedia.org/wiki/Théorème_de_Heine-Borel
%http://fr.wikipedia.org/wiki/Émile_Borel
%http://fr.wikipedia.org/wiki/Henri_Léon_Lebesgue

Soit $E$, un sous-ensemble de $\eR$. Nous pouvons considérer les ouverts suivants : 
\begin{equation}
    \mO_x=B(x,1)
\end{equation}
pour chaque $x\in E$. évidemment,
\begin{equation}
    E\subseteq \bigcup_{x\in E}\mO_x.
\end{equation}
Cette union est très souvent énorme, et même infinie. Elle contient de nombreuses redondances. Si par exemple $E=[-10,10]$, l'élément $3\in E$ est contenu dans $\mO_{3.5}$, $\mO_{2.7}$ et bien d'autres. Pire : même si on enlève par exemple $\mO_2$ de la liste des ouverts, l'union de ce qui reste continue à être tout $E$. La question est : \emph{est-ce qu'on peut en enlever suffisamment pour qu'il n'en reste qu'un nombre fini ?}
\begin{definition}
Soit $E$, un sous-ensemble de $\eR$. Une collection d'ouverts $\mO_i$ est un \defe{recouvrement}{recouvrement} de $E$ si $E\subseteq \bigcup_{i}\mO_i$. Un sous-ensemble $E$ de $\eR$ tel que de tout recouvrement par des ouverts, on peut extraire un sous-recouvrement fini est dit \defe{\href{http://fr.wikipedia.org/wiki/Espace_compact}{compact}}{compact}.
\end{definition}

\begin{proposition}
Les ensembles compacts sont fermés et bornés.
\end{proposition}

\begin{proof}
Prouvons d'abord qu'un ensemble compact est borné. Pour cela, supposons que $K$ est un compact non borné vers le haut\footnote{Nous laissons à titre d'exercice le cas où $K$ est borné par le haut et pas par le bas.}. Donc il existe une suite infinie de nombres strictement croissante $x_1<x_2<\ldots$ tels que $x_i\in K$. Prenons n'importe quel recouvrement ouvert de la partie de $K$ plus petite ou égale à $x_1$, et complétons ce recouvrement par les ouverts $\mO_i=]x_{i-1},x_i[$. Le tout forme bien un recouvrement de $K$ par des ouverts. 

Il n'y a cependant pas moyen d'en tirer un sous recouvrement fini parce que si on ne prend qu'un nombre fini parmi les $\mO_i$, on en aura fatalement un maximum, disons $\mO_k$. Dans ce cas, les points $x_{k+1}$, $x_{k+1}$,\ldots ne seront pas dans le choix fini d'ouverts.

Cela prouve que $K$ doit être borné.

Pour prouver que $K$ est fermé, nous allons prouver que le complémentaire est ouvert. Et pour cela, nous allons prouver que si le complémentaire n'est pas ouvert, alors nous pouvons construire un recouvrement de $K$ dont on ne peut pas extraire de sous recouvrement fini.

Si $\eR\setminus K$ n'est pas ouvert, il possède un point, disons $x$, tel que tout voisinage de $x$ intersecte $K$. Soit $B(x,\epsilon_1)$, un de ces voisinages, et prenons $k_1\in K\cap B(x,\epsilon_1)$. Ensuite, nous prenons $\epsilon_2$ tel que $k_1$ n'est pas dans $B(x,\epsilon_1)$, et nous choisissons $k_2\in K\cap B(x,\epsilon_2)$. De cette manière, nous construisons une suite de $k_i\in K$ tous différents et de plus en plus proches de $x$. Prenons un recouvrement quelconque par des ouverts de la partie de $K$ qui n'est pas dans $B(x,\epsilon_1)$. Les nombres $k_i$ ne sont pas dans ce recouvrement.

Nous ajoutons à ce recouvrement les ensembles $\mO=]k_i,k_{i+1}[$. Le tout forme un recouvrement (infini) par des ouverts dont il n'y a pas moyen de tirer un sous recouvrement fini, pour exactement la même raison que la première fois.
\end{proof}

Le résultat suivant le théorème de \href{http://fr.wikipedia.org/wiki/Théorème_de_Heine-Borel}{Borel-Lebesgue}, et la démonstration vient de wikipédia.
\begin{theorem}[\href{http://fr.wikipedia.org/wiki/Émile_Borel}{borel}-\href{http://fr.wikipedia.org/wiki/Henri_Léon_Lebesgue}{Lebesgue}]   \label{ThoBOrelLebesgue}
    Les intervalles de la forme $[a,b]$ sont compacts.
\end{theorem}

\begin{proof}
    Soit $\Omega$, un recouvrement du segment $[a,b]$ par des ouverts, c'est à dire que
    \begin{equation}
        [a,b]\subseteq\bigcup_{\mO\in\Omega}\mO.
    \end{equation}
    Nous notons par $M$ le sous-ensemble de $[a,b]$ des points $m$ tels que l'intervalle $[a,m]$ peut être recouvert par un sous-ensemble fini de $\Omega$. C'est à dire que $M$ est le sous-ensemble de $[a,b]$ sur lequel le théorème est vrai. Le but est maintenant de prouver que $M=[a,b]$.
    \begin{description}
        \item[$M$ est non vide] En effet, $a\in M$ parce que il existe un ouvert $\mO\in\Omega$ tel que $a\in\mO$. Donc $\mO$ tout seul recouvre l'intervalle $[a,a]$. 
        \item[$M$ est un intervalle] Soient $m_1$, $m_2\in M$. Le but est de montrer que si $m'\in[m_1,m_2]$, alors $m'\in M$. Il y a un sous recouvrement fini de l'intervalle $[a,m_2]$ (par définition de $m_2\in M$). Ce sous recouvrement fini recouvre évidemment aussi $[a,m']$ parce que $[a,m']\subseteq [a,m_2]$, donc $m'\in M$.
        \item[$M$ est une ensemble ouvert] Soit $m\in M$. Le but est de prouver qu'il y a un ouvert autour de $m$ qui est contenu dans $M$. Mettons que $\Omega'$ soit un sous recouvrement fini qui contienne l'intervalle $[a,m]$. Dans ce cas, on a un ouvert $\mO\in\Omega'$ tel que $m\in\mO$. Tous les points de $\mO$ sont dans $M$, vu qu'ils sont tous recouverts par $\Omega'$. Donc $\mO$ est un voisinage de $m$ contenu dans $M$.
        \item[$M$ est un ensemble fermé] $M$ est un intervalle qui commence en $a$, en contenant $a$, et qui finit on ne sait pas encore où. Il est donc soit de la forme $[a,m]$, soit de la forme $[a,m[$. Nous allons montrer que $M$ est de la première forme en démontrant que $M$ contient son supremum $s$. Ce supremum est un élément de $[a,b]$, et donc il est contenu dans un des ouverts de $\Omega$. Disons $s\in\mO_s$. Soit $c$, un élément de $\mO_s$ strictement plus petit que $c$; étant donné que $s$ est supremum de $M$, cet élément $c$ est dans $M$, et donc on a un sous recouvrement fini $\Omega'$ qui recouvre $[a,c]$. Maintenant, le sous recouvrement constitué de $\Omega'$ et de $\mO_s$ est fini et recouvre $[a,s]$.
    \end{description}
    Nous pouvons maintenant conclure : le seul intervalle non vide de $[a,b]$ qui soit à la fois ouvert et fermé est $[a,b]$ lui-même, ce qui prouve que $M=[a,b]$, et donc que $[a,b]$ est compact.
\end{proof}

Par le théorème des valeurs intermédiaires, l'image d'un intervalle par une fonction continue est un intervalle, et nous avons l'importante propriété suivante des fonctions continues sur un compact.

Le théorème suivant est un cas particulier du théorème \ref{ThoMKKooAbHaro}.
\begin{theorem}
    Si $f$ est une fonction continue sur l'intervalle compact $[a,b]$. Alors $f$ est bornée sur $[a,b]$ et elle atteint ses bornes.
\end{theorem}

\begin{proof}
    Étant donné que $[a,b]$ est un intervalle compact, son image est également un intervalle compact, et donc est de la forme $[m,M]$. Ceci découle du théorème \ref{ThoImCompCotComp} et le corollaire \ref{CorImInterInter}. Le maximum de $f$ sur $[a,b]$ est la borne $M$ qui est bien dans l'image (parce que $[m,M]$ est fermé). Idem pour le minimum $m$.
\end{proof}

%+++++++++++++++++++++++++++++++++++++++++++++++++++++++++++++++++++++++++++++++++++++++++++++++++++++++++++++++++++++++++++
\section{Limites à plusieurs variables}
%+++++++++++++++++++++++++++++++++++++++++++++++++++++++++++++++++++++++++++++++++++++++++++++++++++++++++++++++++++++++++++
\label{SecLimVarsPlus}

Prenons une fonction $f\colon \eR^n\to \eR$. Nous disons que
\begin{equation}
    \lim_{x\to x_0}f(x)=l\in\eR
\end{equation}
lorsque $\forall \epsilon>0$, $\exists\delta$ tel que $\| x-x_0 \|\leq\delta$ implique $| f(x)-l |\leq \epsilon$. 

Remarquez qu'ici, $x\in\eR^n$, et sachez distinguer $\| . \|$, la norme dans $\eR^n$ de $| . |$ qui est la valeur absolue dans $\eR$. Une autre façon d'exprimer cette définition est que l'ensemble des valeurs atteintes par $f$ dans une boule de rayon $\delta$ autour de $x_0$ n'est pas très loin de $l$. Nous définissons donc
\begin{equation}
    E_{\delta}=\{ f(x)\tq x\in B(x_0,\delta) \}.
\end{equation}
Notez que si $f$ n'est pas définie en $x_0$, il n'y a pas de valeurs correspondantes au centre de la boule dans $E_{\delta}$. Ceci est évidemment la situation générique lorsqu'il y a une indétermination à lever dans le calcul de la limite. Nous avons alors que
\begin{equation}
    \lim_{x\to x_0}f(x)=l
\end{equation}
lorsque $\forall\epsilon>0$, $\exists\delta$ tel que 
\begin{equation}        \label{Eqvmoinsrapplimdeux}
    \sup\{ | v-l |\tq v\in E_{\delta} \}\leq\epsilon.
\end{equation}
Une façon classique de montrer qu'une limite n'existe pas, est de prouver que, pour tout $\delta$, l'ensemble $E_{\delta}$ contient deux valeurs constantes. Si par exemple $0\in E_{\delta}$ et $1\in E_{\delta}$ pour tout $\delta$, alors aucune valeur de $l$ (même pas $l=\pm\infty$) ne peut satisfaire à la condition \eqref{Eqvmoinsrapplimdeux} pour toute valeur de $\epsilon$.

Nous laissons à la sagacité de l'étudiant le soin d'adapter tout ceci pour le cas $\lim_{x\to x_0}f(x)=\pm\infty$.

La proposition suivante semble évidente, mais nous sera tellement
utile qu'il est préférable de l'expliciter~:
\begin{proposition}
Soit $f : D \to \eR$ une fonction dont le domaine
  s'écrit comme une réunion \emph{finie}
  \begin{equation*}
    D = \bigcup_{i=1}^k A_i
  \end{equation*}  
  où $k$ est un entier. Soit $a \in \Adh D$ tel que $a \in \Adh A_i$
  pour tout $i \leq k$, et soit $b \in \eR$. Alors, la limite
  \begin{equation*}
    \limite x a f(x)
  \end{equation*}
  existe et vaut $b$ si et seulement si chacune des limites
  \begin{equation*}
    \limite[x \in A_i] x a f(x)
  \end{equation*}
  existe et vaut $b$.
\end{proposition}

\begin{proof}On sait déjà que si la limite de $f : D \to \eR$
  existe, alors toute restriction à $A_i$ admet la même limite. Il
  suffit donc de prouver la réciproque.

  Par hypothèse, pour tout $i = 1 \ldots k$, nous savons que
  \begin{equation*}
    \forall \epsilon > 0\, \exists \delta_i > 0 \tq (x \in A_i)
    \text{ et }
    (\norme{x-a} < \delta_i) \Rightarrow \norme{f(x) - b} < \epsilon
  \end{equation*}

  Si $\epsilon$ est fixé, posons $\delta = \min_i\{\delta_i\}$. Nous
  savons alors que
  \begin{enumerate}
  \item pour tout $x \in D$, il existe $i$ tel que $x \in A_i$, et
  \item si $x$ vérifie $\norme{x-a} < \delta$, alors pour tout $i$,
    $\norme{x-a} < \delta_i$ par définition de $\delta$.
  \end{enumerate}
  
  On en déduit que si $x \in D$ et $\norme{x-a} < \delta$, alors il
  existe $i$ tel que $x \in A_i$ et $\norme{x-a} < \delta_i$, ce qui
  implique $\abs{f(x) - b} < \epsilon$ et prouve la continuité.
\end{proof}

\begin{example}
  \begin{enumerate}
  \item Pour qu'une fonction $f : \eR \to \eR$ admette une limite en
    $a \in \eR$, il faut et il suffit qu'elle y admette une limite à
    droite et une limite à gauche qui soient égales.

  \item Une suite $(x_k)$ admet une limite si et seulement si les
    sous-suites $(x_{2k})$ et $(x_{2k+1})$ convergent vers la même
    limite. Ceci n'est pas une application directe de la proposition,
    mais la teneur est la même.
  \end{enumerate}
\end{example}

Il existe de nombreuses façons de calculer des limites à plusieurs variables. Plus nous connaîtrons de mathématiques, plus nous aurons de techniques à notre disposition. Nous allons tout de suite voir quelque méthodes. Voir le thème \ref{THEMEooLTCIooGDIPnF} pour plus de techniques et d'exemples.

%---------------------------------------------------------------------------------------------------------------------------
\subsection{Règle de l'étau}
%---------------------------------------------------------------------------------------------------------------------------

Une première façon de calculer la limite d'une fonction est de la «\wikipedia{en}{Squeeze_theorem}{coincer}» entre deux fonctions dont nous connaissons la limite. Le théorème, que nous acceptons sans démonstration, est le suivant :
\begin{theorem}[Règle de l'étau]		\label{ThoRegleEtau}
	Soit $\mO$, un ouvert de $\eR^m$ contenant le point $a$. Soient $f$, $g$ et $h$, trois fonctions définies sur $\mO$ (éventuellement pas en $a$ lui-même). Supposons que pour tout $x\in\mO$ (à part éventuellement $a$), nous ayons les inégalités
	\begin{equation}
		g(x)\leq f(x)\leq h(x).
	\end{equation}
	Supposons de plus que
	\begin{equation}
		\lim_{x\to a} g(x)=\lim_{x\to a} h(x)=\ell.
	\end{equation}
	Alors la limite $\lim_{x\to a} f(x)$ existe et vaut $\ell$.
\end{theorem}

Nous insistons sur le fait que les deux fonctions entre lesquelles nous coinçons $f$ doivent tendre vers \emph{la même} valeur.

Cette méthode est très pratique lorsqu'on a des fonctions trigonométriques qui se factorisent parce qu'elles sont toujours majorables par $1$; voir l'exemple \ref{EXooSPFDooSluUGV}.

\begin{example}
	Prouver la continuité en $(0,0)$ de la fonction
	\begin{equation}
		f(x,y)=\begin{cases}
			\frac{ x | y | }{ \sqrt{x^2+y^2} }	&	\text{si }(x,y)\neq (0,0)\\
			0	&	 \text{sinon.}
		\end{cases}
	\end{equation}
	Considérons une suite $(x_n,y_n)\in\eR^2$ qui tend vers $(0,0)$. Étant donné que $\frac{ | y | }{ \sqrt{x^2+y^2} }<1$ pour tout $x$ et $y$, nous avons
	\begin{equation}
		0\leq | f(x_n,y_n) |=\left| \frac{ x_n | y_n | }{ \sqrt{x_n^2+y_n^2} } \right| \leq | x_n |\to 0.
	\end{equation}
	Donc nous avons
	\begin{equation}
		\lim_{(x,y)\to(0,0)}f(x,y)=0=f(0,0),
	\end{equation}
	ce qui prouve que la fonction est continue en $(0,0)$ par la proposition \ref{PropFnContParSuite}. Nous avons utilisé la règle de l'étau (théorème \ref{ThoRegleEtau}).
\end{example}

\begin{normaltext}
    Nous notons \( f\sim g\)\nomenclature[Y]{\( f\sim g\)}{fonctions ayant des limites équivalentes} pour \( x\to a\) lorsque \( \lim_{x\to a} \frac{ f(x) }{ g(x) }=1\). 

    Cela signifie que \( f\) et \( g\) tendent vers la même limite, à la même vitesse. 
\end{normaltext}

%---------------------------------------------------------------------------------------------------------------------------
\subsection{Méthode des chemins}
%---------------------------------------------------------------------------------------------------------------------------

Lorsque la limite n'existe pas, il y a une façon en général assez simple de le savoir, c'est la \defe{méthode des chemins}{méthode!des chemins}.

\newcommand{\CaptionFigMethodeChemin}{Sur toute la droite $y=-x$, la fonction vaut $-1/2$, tandis que sur toute la droite $y=x/2$, elle vaut $\frac{2}{ 5 }$. Il est donc impossible que la fonction ait une limite en $(0,0)$, parce que dans toute boule autour de zéro, il y aura toujours un point de chacune de ces deux droites.}
	\input{auto/pictures_tex/Fig_MethodeChemin.pstricks}

\begin{example}		\label{ExFNExempleMethodeTrigigi}
	Considérons la fonction
	\begin{equation}
		f(x,y)=\frac{ xy }{ x^2+y^2 },
	\end{equation}
	et remarquons que, quelle que soit la valeur de $y$, cette fonction est nulle lorsque $x=0$. De la même manière, nous voyons que si $x=y$, alors la fonction vaut\footnote{En fait ce que nous sommes en train de faire est de poser $\theta=\pi/2$ et $\theta=\pi/4$ dans \eqref{Eq2807fpolairerhodeuxcossin}.} $\frac{ 1 }{2}$. 

	Il est impossible que la fonction ait une limite en $(0,0)$ parce qu'on ne peut pas trouver un $\ell$ dont on s'approche à la fois en suivant la ligne $x=0$ et la ligne $x=y$.

	Deux autres chemins avec encore deux autres valeurs sont dessinés sur la figure \ref{LabelFigMethodeChemin}.

    Cet exemple pourra être formalisé en utilisant le théorème \ref{ThoLimSuite}. Voir l'exemple \ref{EXooNBTYooFyKRTB}.

\end{example}

Tout ceci est formalisé et généralisé dans la proposition suivante.
\begin{proposition}     \label{PROPooSAFIooWvmSiT}
	Soit $f\colon D\subset\eR^m\to \eR^n$ et $a$ un point d'adhérence de $D$. Alors nous avons
	\begin{equation}
		\lim_{x\to a} f(x)=\ell
	\end{equation}
	si et seulement si pour toute fonction $\gamma\colon \eR\to \eR^m$ telle que $\lim_{t\to 0} \gamma(t)=a$, nous avons
	\begin{equation}
		\lim_{t\to 0} (f\circ\gamma)(t)=\ell.
	\end{equation}	
\end{proposition}

\begin{corollary}	\label{CorMethodeChemin}
	Soient $f\colon D\subset\eR^m\to \eR^n$ et $a$ un point d'accumulation de $D$. Si nous avons deux fonctions $\gamma_1,\gamma_2\colon \eR\to \eR^m$ telles que
	\begin{equation}
		\lim_{t\to 0} \gamma_1(t)=\lim_{t\to 0} \gamma_2(t)=a
	\end{equation}
	tandis que
	\begin{equation}
		\lim_{t\to 0} (f\circ \gamma_1)(t)\neq\lim_{t\to 0} (f\circ \gamma_2)(t),
	\end{equation}
	ou bien que l'une des deux limites n'existe pas, alors la limite de $f(x)$ lorsque $x\to a$ n'existe pas.
\end{corollary}

\begin{corollary}	\label{CorMethodeChemoinNegatif}
	Soient $f\colon D\subset\eR^m\to \eR^n$ et $a$ un point d'accumulation de $D$. Si il existe une fonction $\gamma\colon \eR\to \eR^m$ avec $\gamma(0)=a$ telle que la limite $\lim_{t\to 0} (f\circ\gamma)(t)$ n'existe pas, alors la limite $\lim_{x\to a} f(x)$ n'existe pas.
\end{corollary}

En ce qui concerne le calcul de limites, la méthode des chemins peut être utilisé de trois façons :
\begin{enumerate}
	\item
		Dès que l'on trouve une fonction $\gamma\colon \eR\to \eR^m$ telle que $\lim_{t\to 0} (f\circ \gamma)(t)=\ell$, alors nous savons que \emph{si la limite $\lim_{x\to a} f(x)$ existe}, alors cette limite vaut $\ell$.
	\item
		Dès que l'on a trouvé deux fonctions $\gamma_i$ qui tendent vers $a$, mais dont les limites de $\lim_{t\to 0} (f\circ\gamma_i)(t)$ sont différentes, alors la limite $\lim_{x\to a} f(x)$ n'existe pas.
	\item
		Dès qu'on trouve une chemin le long duquel il n'y a pas de limite, alors la limite n'existe pas (corollaire \ref{CorMethodeChemoinNegatif}).
\end{enumerate}
La méthode des chemins ne permet donc pas de de calculer une limite quand elle existe. Elle permet uniquement de la «deviner», ou bien de prouver que la limite n'existe pas.

\begin{example}
	Soit à calculer
	\begin{equation}	\label{Eq3007ExempleLimiche}
		\lim_{(x,y)\to(0,0)}\frac{ x-y }{ x+y }.
	\end{equation}
	Si nous prenons le chemin $\gamma_1(t)=(t,t)$, nous avons bien $\lim_{t\to 0} \gamma_1(t)=(0,0)$, et nous avons
	\begin{equation}
		\lim_{t\to 0} (f\circ\gamma_1)(t)=\lim_{t\to 0} \frac{ t-t }{ t+t }=0.
	\end{equation}
	Donc si la limite \eqref{Eq3007ExempleLimiche} existait, elle vaudrait obligatoirement $0$. Mais si nous considérons $\gamma_2(t)=(0,t)$, nous avons
	\begin{equation}
		(f\circ\gamma_2)(t)=\frac{ -t }{ t }=-1,
	\end{equation}
	donc si la limite existe, elle doit obligatoirement valoir $-1$. Ne pouvant être égale à $0$ et à $-1$ en même temps, la limite \eqref{Eq3007ExempleLimiche} n'existe pas.
\end{example}

%+++++++++++++++++++++++++++++++++++++++++++++++++++++++++++++++++++++++++++++++++++++++++++++++++++++++++++++++++++++++++++
\section{Dérivée : exemples introductifs}
%+++++++++++++++++++++++++++++++++++++++++++++++++++++++++++++++++++++++++++++++++++++++++++++++++++++++++++++++++++++++++++

%---------------------------------------------------------------------------------------------------------------------------
\subsection{La vitesse}
%---------------------------------------------------------------------------------------------------------------------------

Lorsqu'un mobile se déplace à une vitesse variable, nous obtenons la \emph{vitesse instantanée} en calculant une vitesse moyenne sur des intervalles de plus en plus petits. Si le mobile a un mouvement donné par $x(t)$, la vitesse moyenne entre $t=2$ et $t=5$ sera
\[ 
  v_{\text{moy}}(2\to 5)=\frac{ x(5)-x(2) }{ 5-2 }.
\]
Plus généralement, la vitesse moyenne entre $2$ et $2+\Delta t$ est donnée par
\[ 
  v_{\text{moy}}(2\to 2+\Delta t)=\frac{ x(2+\Delta t)-x(2) }{ \Delta t }.
\]
Cela est une fonction de $\Delta t$. Oui, mais je te rappelle qu'on a dans l'idée de calculer une vitesse instantanée, c'est à dire de voir ce que vaut la vitesse moyenne sur un intervalle très {\small très} {\footnotesize très} {\scriptsize très} {\tiny petit}. La notion de limite semble toute indiquée pour décrire mathématiquement l'idée physique de vitesse instantanée.

Nous allons dire que la vitesse instantanée d'un mobile est la limite quand $\Delta t$ tend vers zéro de sa vitesse moyenne sur l'intervalle de temps $\Delta t$, ou en formule :
\begin{equation}		\label{Eqvinstlimite}
	v(t_0)=\lim_{\Delta t\to 0}\frac{ x(t_0)-x(t_0+\Delta t) }{ \Delta t }.
\end{equation}

%---------------------------------------------------------------------------------------------------------------------------
\subsection{La tangente à une courbe}
%---------------------------------------------------------------------------------------------------------------------------

Passons maintenant à tout autre chose, mais toujours dans l'utilisation de la notion de limite pour résoudre des problèmes intéressants. Comment trouver l'équation de la tangente à la courbe $y=f(x)$ au point $(x_0,f(x_0))$ ?

Essayons de trouver la tangente au point $P$ donné de la courbe donnée à la figure \ref{LabelFigTangenteQuestion}.

\newcommand{\CaptionFigTangenteQuestion}{Comment trouver la tangente à la courbe au point $P$ ?}
\input{auto/pictures_tex/Fig_TangenteQuestion.pstricks}

La tangente est la droite qui touche la courbe en un seul point sans la traverser. Afin de la construire, nous allons dessiner des droites qui touchent la courbe en $P$ et un autre point $Q$, et nous allons voir ce qu'il se passe quand $Q$ est très proche de $P$. Cela donnera une droite qui, certes, touchera la courbe en deux points, mais en deux points \emph{tellement proches que c'est comme si c'étaient les mêmes}. Tu sens que la notion de limite va encore venir.

%Pour rappel cette figure TangenteDetail est générée par phystricksRechercheTangente.py
\newcommand{\CaptionFigTangenteDetail}{Traçons d'abord une corde entre le point $P$ et un point $Q$ un peu plus loin.}
\input{auto/pictures_tex/Fig_TangenteDetail.pstricks}

Nous avons placé le point, sur la figure \ref{LabelFigTangenteDetail}, le point $P$ en $a$ et le point $Q$ un peu plus loin $x$. En d'autres termes leurs coordonnées sont
\begin{align}
	P=\big(a,f(a)\big)&& Q=\big(x,f(x)\big).
\end{align}
Comme tu devrais le savoir sans même regarder la figure \ref{LabelFigTangenteDetail}, le coefficient directeur de la droite qui passe par ces deux points est donné par
\begin{equation}
	\frac{ f(x)-f(a) }{ x-a },
\end{equation}
et bang ! Encore le même rapport que celui qu'on avait trouvé à l'équation \eqref{Eqvinstlimite} en parlant de vitesses. Si tu regardes la figure \ref{LabelFigLesSubFigures}, tu verras que réellement en faisant tendre $x$ vers $a$ on obtient la tangente.

\newcommand{\CaptionFigLesSubFigures}{Recherche de la tangente par approximations successives.}
\input{auto/pictures_tex/Fig_LesSubFigures.pstricks}
%See also the subfigure \ref{LabelFigLesSubFiguressssubZ}
%See also the subfigure \ref{LabelFigLesSubFiguressssubO}
%See also the subfigure \ref{LabelFigLesSubFiguressssubT}
%See also the subfigure \ref{LabelFigLesSubFiguressssubTh}
%See also the subfigure \ref{LabelFigLesSubFiguressssubF}
%See also the subfigure \ref{LabelFigLesSubFiguressssubFi}

%---------------------------------------------------------------------------------------------------------------------------
\subsection{L'aire en dessous d'une courbe}		\label{SubSecAirePrimInto}
%---------------------------------------------------------------------------------------------------------------------------

Encore un exemple. Nous voudrions bien pouvoir calculer l'aire en dessous d'une courbe. Nous notons $S_f(x)$ l'aire en dessous de la fonction $f$ entre l'abscisse $0$ et $x$, c'est à dire l'aire bleue de la figure \ref{LabelFigNOCGooYRHLCn}. % From file NOCGooYRHLCn
\newcommand{\CaptionFigNOCGooYRHLCn}{L'aire en dessous d'une courbe. Le rectangle rouge d'aire $f(x)\Delta x$ approxime l'augmentation de l'aire lorsqu'on passe de $x$ à $x+\Delta x$.}
\input{auto/pictures_tex/Fig_NOCGooYRHLCn.pstricks}

Si la fonction $f$ est continue et que $\Delta x$ est assez petit, la fonction ne varie pas beaucoup entre $x$ et $x+\Delta x$. L'augmentation de surface entre $x$ et $x+\Delta x$ peut donc être approximée par le rectangle de surface $f(x)\Delta x$. Ce que nous avons donc, c'est que quand $\Delta x$ est très petit,
\begin{equation}
	S_f(x+\Delta x)-S_f(x)=f(x)\Delta x,
\end{equation}
c'est à dire
\begin{equation}
	f(x)=\lim_{\Delta x\to 0}\frac{  S_f(x+\Delta x)-S_f(x)}{ \Delta x }.
\end{equation}
Donc, la fonction $f$ est la dérivée de la fonction qui représente l'aire en dessous de $f$. Calculer des surfaces revient donc au travail inverse de calculer des dérivées.

Nous avons déjà vu que calculer la dérivée d'une fonction n'est pas très compliqué. Aussi étonnant que cela puisse paraître, il se fait que le processus inverse est très compliqué : il est en général extrêmement difficile (et même souvent impossible) de trouver une fonction dont la dérivée est une fonction donnée.

Une fonction dont la dérivée est la fonction $f$ s'appelle une \defe{primitive}{primitive} de $f$, et la fonction qui donne l'aire en dessous de la fonction $f$ entre l'abscisse $0$ et $x$ est notée
\begin{equation}
	S_f(x)=\int_0^xf(t)dt.
\end{equation}
Nous pouvons nous demander si, pour une fonction $f$ donnée, il existe une ou plusieurs primitives, c'est à dire s'il existe une ou plusieurs fonctions $F$ telles que $F'=f$. La réponse viendra\ldots
%TODO : faire la référence

%+++++++++++++++++++++++++++++++++++++++++++++++++++++++++++++++++++++++++++++++++++++++++++++++++++++++++++++++++++++++++++ 
\section{Dérivation de fonctions réelles}
%+++++++++++++++++++++++++++++++++++++++++++++++++++++++++++++++++++++++++++++++++++++++++++++++++++++++++++++++++++++++++++
\label{seccontetderiv}

On considère dans la suite une fonction $f : A \to \eR$, où $a \in A \subset \eR$ ; cependant, les notions de continuité et de dérivabilité se généralisent immédiatement au cas de fonctions à valeurs vectorielles ; la notion de continuité se généralise au cas des fonctions à plusieurs variables (la notion de dérivabilité est remplacée par celle de différentiabilité dans ce cadre).

\begin{definition}      \label{DEFooOYFZooFWmcAB}
    La fonction $f$ est \defe{dérivable}{dérivable} en \( a\) si $a \in
  \operatorname{int} A$ et si
  \begin{equation*}
    \lim_{x\to a} \frac{f(x)-f(a)}{x-a}
  \end{equation*}
  existe. On note alors cette quantité $f'(a)$, c'est le nombre
  dérivé de $f$ en $a$. La \defe{fonction dérivée}{fonction dérivée} de $f$ est
  \begin{equation}
      \begin{aligned}
          f'\colon A'&\to \eR \\
          a&\mapsto f(a) 
      \end{aligned}
  \end{equation}
  définie sur l'ensemble noté $A'$ des points $a$ où $f$ est dérivable.
\end{definition}

\begin{example}
      Montrons que la fonction $f : \eR \to \eR : x\mapsto x$ est continue et dérivable. Exceptionnellement (bien qu'on sache que la dérivabilité implique la continuité), montrons ces deux assertions séparément.
      \begin{description}
      \item[Continuité] Pour prouver la continuité au point $a \in \eR$ nous devons montrer que
     \begin{equation}
       \limite x a x = a
     \end{equation}
     c'est-à-dire
     \begin{equation}
       \forall \epsilon > 0, \exists \delta > 0 :  \forall x \in \eR \abs{x-a} <
       \delta \Rightarrow \abs{x-a} < \epsilon
     \end{equation}
     ce qui est clair en prenant $\delta = \epsilon$.

      \item[Dérivabilité] Soit $a \in \eR$. Calculons la limite du quotient différentiel
        \begin{equation}
          \limite[x\neq a]{x}{a} \frac{x-a}{x-a} = \limite[x\neq a]x a 1 = 1
        \end{equation}
        ce qui prouve que $f$ est dérivable et que sa dérivée vaut $1$ en
        tout point $a$ de $\eR$.
      \end{description}

     On a donc montré que la fonction $x \mapsto x$ est continue, dérivable, et que sa dérivée vaut $1$ en tout point $a$ de son domaine.

\end{example}

\begin{proposition} \label{PropSFyxOWF}
    Une fonction dérivable sur un intervalle est continue sur cet intervalle.
\end{proposition}

\begin{proof}
    Soit \( I\) un intervalle sur lequel la fonction \( f\) est dérivable, et soit \( x_0\in I\). Nous allons prouver la continuité de \( f\) en \( x_0\). Le fait que la limite
    \begin{equation}
        f'(x_0)=\lim_{h\to 0} \frac{ f(x_0+h)-f(x_0) }{ h }
    \end{equation}
    existe implique a fortiori que 
    \begin{equation}
        \lim_{h\to 0} f(x_0+h)-f(x_0)=0.
    \end{equation}
    Cela signifie la continuité de \( f\) en vertu du critère \ref{ThoLimCont}.
\end{proof}

%---------------------------------------------------------------------------------------------------------------------------
\subsection{Exemples}
%---------------------------------------------------------------------------------------------------------------------------

\begin{example}
    Commençons par la fonction $f(x)=x$. Dans ce cas nous avons
    \begin{equation}
        \frac{ f(x)-f(a) }{ x-a }=\frac{ x-a }{ x-a }=1.
    \end{equation}
    La dérivée est donc $1$.
\end{example}

\begin{proposition}
    La dérivé de la fonction $x\mapsto x$ vaut $1$, en notations compactes : $(x)'=1$.
\end{proposition}

\begin{proof}
    D'après la définition de la dérivée, si $f(x)=x$, nous avons
    \begin{equation}
        f(x)=\lim_{\epsilon\to 0}\frac{ (x+\epsilon) -x }{\epsilon} =\lim_{\epsilon\to 0}\frac{ \epsilon }{\epsilon} =1,
    \end{equation}
    et c'est déjà fini.
\end{proof}

%///////////////////////////////////////////////////////////////////////////////////////////////////////////////////////////
\subsubsection{La fonction $f(x)=x^2$}
%///////////////////////////////////////////////////////////////////////////////////////////////////////////////////////////

Prenons ensuite $f(x)=x^2$. En utilisant le produit remarquable $(x^2-a^2)=(x-a)(x+a)$ nous trouvons
\begin{equation}
	\frac{ f(x)-f(a) }{ x-a }=x+a.
\end{equation}
Lorsque $x\to a$, cela devient $2a$. Nous avons par conséquent
\begin{equation}
	f'(x)=2x.
\end{equation}

\begin{lemma}           \label{LemDeccCarr}
    Si $f(x)=x^2$, alors $f'(x)=2x$.
\end{lemma}

\begin{proof}
    Utilisons la définition, et remplaçons $f$ par sa valeur :
    \begin{subequations}
        \begin{align}
            f'(x)   &=\lim_{\epsilon\to 0}\frac{ f(x+\epsilon)-f(x) }{ \epsilon }\\
                &=\lim_{\epsilon\to 0}\frac{ (x+\epsilon)^2-x^2 }{ \epsilon }\\
                &=\lim_{\epsilon\to 0}\frac{ x^2+2x\epsilon+\epsilon^2-x^2 }{ \epsilon }\\
                &=\lim_{\epsilon\to 0}\frac{\epsilon(2x+\epsilon)}{ \epsilon }\\
                &=\lim_{\epsilon\to 0}(2x+\epsilon)\\
                &=2x,
        \end{align}
    \end{subequations}
    ce qu'il fallait prouver.
\end{proof}


%///////////////////////////////////////////////////////////////////////////////////////////////////////////////////////////
\subsubsection{La fonction $f(x)=\sqrt{x}$}
%///////////////////////////////////////////////////////////////////////////////////////////////////////////////////////////

Considérons maintenant la fonction $f(x)=\sqrt{x}$. Nous avons
\begin{equation}
	\begin{aligned}[]
		\frac{ f(x)-f(a) }{ x-a }&=\frac{ \sqrt{x}-\sqrt{a} }{ x-a }\\
		&=\frac{ (\sqrt{x}-\sqrt{a})(\sqrt{x}+\sqrt{x}) }{ (x-a)(\sqrt{x}+\sqrt{x}) }\\
		&=\frac{1}{ \sqrt{x}+\sqrt{x} }.
	\end{aligned}
\end{equation}
Lorsque $x\to 0$, nous obtenons
\begin{equation}
	f'(a)=\frac{1}{ 2\sqrt{a} }.
\end{equation}
Notons que la dérivée de $f(x)=\sqrt{x}$ n'existe pas en $x=0$. En effet elle serait donnée par le quotient
\begin{equation}
	f'(0)=\lim_{x\to 0} \frac{ \sqrt{x}-\sqrt{0} }{ x }=\lim_{x\to 0} \frac{ \sqrt{x} }{ x }=\lim_{x\to 0} \frac{1}{ \sqrt{x} }.
\end{equation}
Mais si $x$ devient très petit, la dernière fraction tend vers l'infini.

%--------------------------------------------------------------------------------------------------------------------------- 
\subsection{Dérivée de la réciproque}
%---------------------------------------------------------------------------------------------------------------------------

\begin{proposition}[\cite{XGIooNMtKqx}] \label{PropMRBooXnnDLq}
    Soit \( f\colon I\to J=f(I)\) une fonction bijective, continue et dérivable\footnote{Définition \ref{}.}. Soient \( x_0\in I\) et \( y_0=f(x_0)\). Si \( f'(x_0)\neq 0\) alors la fonction réciproque \( f^{-1}\) est dérivable en \( y_0\) et sa dérivée est donnée par
    \begin{equation}    
        (f^{-1})'(y_0)=\frac{1}{ f'(x_0) }.
    \end{equation}
\end{proposition}
 
\begin{proof}
    Pour rappel, une fonction dérivable est toujours continue (proposition \ref{PropSFyxOWF}).

    Prouvons que \( f^{-1}\) est dérivable au point \( b=f(a)\in J\). Étant donné que \( f\) est dérivable en \( a\), nous avons
    \begin{equation}\label{EqJEWooSjQrfk}
        f'(a)=\lim_{x\to a} \frac{ f(x)-f(a) }{ x-a }.
    \end{equation}
    Par ailleurs, étant donnée la continuité de \( f^{-1}\) donnée par la proposition \ref{ThoKBRooQKXThd}\ref{ItemEJZooKuFoeFiv}, nous avons
    \begin{equation}
        \lim_{\epsilon\to 0} f^{-1}(b+\epsilon)=f^{-1}(b)=a.
    \end{equation}
    Nous pouvons donc remplacer dans \eqref{EqJEWooSjQrfk} tous les \( x\) par \( f^{-1}(b+\epsilon)\) et prendre la limite \( \epsilon\to 0\) au lieu de \( x\to a\) :
    \begin{equation}
        \begin{aligned}[]
            f'(a)&=\lim_{\epsilon\to 0}\frac{ f\big( f^{-1}(b+\epsilon) \big)-f(a) }{ f^{-1}(b+\epsilon)-a }\\
            &=\lim_{\epsilon\to 0}\frac{ b+\epsilon-f(a) }{ f^{-1}(b+\epsilon)-f^{-1}(b) }\\
            &=\lim_{\epsilon\to 0}\frac{ \epsilon }{ f^{-1}(b+\epsilon)-f^{-1}(b) }\\
            &=\frac{1}{ \lim_{\epsilon\to 0}\frac{ f^{-1}(b+\epsilon)-f^{-1}(b) }{ \epsilon } }\\
            &=\frac{1}{ (f^{-1})'(b) }.
        \end{aligned}
    \end{equation}
    Nous avons utilisé le fait que \( f(a)=b\) et \( a=f^{-1}(b)\).
\end{proof}

\begin{normaltext}
 Très souvent on préfère retenir la formule
    \begin{equation}\label{EqWWAooBRFNsv}
      (f^{-1})'(y_0) = \frac{1}{f'\left((f^{-1})(y_0)\right)}
    \end{equation}

    Elle est très simple à retrouver : il suffit d'écrire
    \begin{equation}    
        f^{-1}\big( f(x) \big)=x
    \end{equation}
    puis de dériver les deux côtés par rapport à \( x\) en utilisant la règle de dérivation des fonctions composées :
    \begin{equation}
        (f^{-1})'\big( f(x) \big)f'(x)=1.
    \end{equation}
\end{normaltext}

\begin{example}[difféomorphisme entre \( \eR\) et un ouvert borné]      \label{EXooGKPNooZtmJen}
    Nous cherchons à construire une application dérivable et d'inverse dérivable entre \( \eR\) (en entier) et un ouvert borné de \( \eR\). Il serait tentant de prendre l'application arc tangente
    \begin{equation}
        \begin{aligned}
        \arctan\colon \eR&\to \left] -\frac{ \pi }{2} , \frac{ \pi }{2} \right[ \\
            x&\mapsto \arctan(x) 
        \end{aligned},
    \end{equation}
    mais elle ne sera définie que dans le théorème \ref{THOooUSVGooOAnCvC}.

    Nous posons
    \begin{equation}
        f(x)=\begin{cases}
            2+\frac{1}{ x-2 }    &   \text{si } x\leq 1\\
            \frac{1}{ x }    &    \text{si } x>1.
        \end{cases}
    \end{equation}
    Cela est continue en \( x=1\) : il suffit de calculer les deux valeurs. En ce qui concerne la dérivabilité en \( x=1\), nous devons faire
    \begin{equation}
        \lim_{\epsilon\to 0}\frac{ f(1+\epsilon)-f(1) }{ \epsilon }.
    \end{equation}
    La limite à gauche est égale à la dérivée de \( x\mapsto 2+\frac{ 1 }{ x-2 }\) en \( x=1\) et la limite à droite est égale à la dérivée de \( x\mapsto 1/x\) en \( x=1\). Dans les deux cas nous trouvons \( -1\).

    \begin{center}
        \input{auto/pictures_tex/Fig_LMHMooCscXNNdU.pstricks}
    \end{center}

    Nous voyons vite que cette fonction est strictement décroissante; et un calcul de limite nous dit qu'il s'agit d'une bijection dérivable
    \begin{equation}
        f\colon \eR\to \mathopen] 0 , 2 \mathclose[.
    \end{equation}
    La proposition \ref{PropMRBooXnnDLq} s'applique et la bijection réciproque est également dérivable (donc continue aussi).
\end{example}

\begin{probleme}
Si vous connaissez un autre exemple, plus simple, de difféomorphisme \( f\colon \eR\to \mathopen] a , b \mathclose[\), faites-le moi savoir. Ne pas utiliser d'exponentielle (vous pensiez à bricoler quelque chose à partir de la primitive de \( x\mapsto  e^{-x^2}\) ?) ni de fonctions trigonométriques.
\end{probleme}

\begin{example}
    Nous aimerions donner le logarithme comme exemple, mais l'exponentielle ne sera définie que dans longtemps à partir des séries entières. Allez voir l'exemple \ref{ExZLMooMzYqfK} pour le logarithme comme inverse de l'exponentielle.
\end{example}

%---------------------------------------------------------------------------------------------------------------------------
\subsection{Calcul de la dérivée}
%---------------------------------------------------------------------------------------------------------------------------

\begin{proposition}     \label{PROPooOUZOooEcYKxn}
    Soit $f,g\colon I\subset\eR\to\eR $ deux fonctions dérivables. Alors nous avons les propriétés suivante
    \begin{enumerate}
    	\item
    		la fonction $h=f+g$ est dérivable et $h'(x)=f'(x)+g'(x)$.
    	\item
    		la fonction $h=fg$ est dérivable et 
    		\begin{equation}
    			(fg)'(x)=f'(x)g(x)+f(x)g'(x).
    		\end{equation}
    		Cette formule est appelée \defe{règle de Leibnitz}{Leibnitz}.
        \item       \label{ITEMooMUNQooLiKffz}
    		la fonction $h=\frac{ f }{ g }$ est dérivable en tout point $x$ tel que $g(x)\neq 0$ et 
    		\begin{equation}
    			\left( \frac{ f }{ g } \right)'(x)=\frac{ f'(x)g(x)-f(x)g'(x) }{ g(x)^2 }.
    		\end{equation}
        \item   \label{ITEMooLYZCooVUPTyh}
    		la fonction $h=f\circ g$ est dérivable et 
    		    \begin{equation}
    		    	(f\circ g)'(x)=f'\big( g(x) \big)g'(x).
    	    	\end{equation}
    \end{enumerate}
\end{proposition}

\begin{proposition}
	Si $f(x)=x^n$ avec $n\in\eN$, alors $f'(x)=nx^{n-1}$.
\end{proposition}
\begin{proof}
	Nous avons déjà vu que la proposition était vraie avec $n=1$ et $n=2$. Supposons qu'elles soit vraie avec $n=k$, et prouvons qu'elle est vraie pour $n=k+1$. Nous avons
	\begin{equation}
		x^{k+1}=xx^k.
	\end{equation}
	En utilisant la règle de Leibnitz et l'hypothèse de récurrence,
	\begin{equation}
		\begin{aligned}[]
			\big( x^{k+1} \big)'&=(x)'x^k+x\big( x^k \big)'\\
			&=x^k+x\big( kx^{k-1} \big)\\
			&=x^k+kx^k\\
			&=(k+1)x^k,
		\end{aligned}
	\end{equation}
	ce qu'il fallait démontrer.
\end{proof}

%--------------------------------------------------------------------------------------------------------------------------
\subsection[Interprétation géométrique : tangente]{Interprétation géométrique de la dérivée : tangente}
%--------------------------------------------------------------------------------------------------------------------------

Considérons le \defe{graphe}{graphe} de la fonction $f$ sur $I$, c'est à dire l'ensemble
\begin{equation}
	\big\{ \big( x,f(x) \big)\tq x\in I \big\}.
\end{equation}
Le nombre 
\begin{equation}
	\frac{ f(x)-f(a) }{ x-a }
\end{equation}
est la pente de la droite qui joint les points $\big( x,f(x) \big)$ et $\big( a,f(a) \big)$, voir la figure  \ref{LabelFigGWOYooRxHKSm}. % From file GWOYooRxHKSm
\newcommand{\CaptionFigGWOYooRxHKSm}{Le coefficient directeur de la corde entre $a$ et $x$.}
\input{auto/pictures_tex/Fig_GWOYooRxHKSm.pstricks}

Étant donné que $f'(a)$ est le coefficient directeur de la tangente au point $\big( a,f(a) \big)$, l'équation de la tangente est
\begin{equation}		\label{EqTgfaen}
	y-f(a)=f'(a)(x-a).
\end{equation}

%--------------------------------------------------------------------------------------------------------------------------
\subsection[Interprétation géométrique : approximation affine]{Interprétation géométrique de la dérivée : approximation affine}
%--------------------------------------------------------------------------------------------------------------------------

Le fait que la fonction $f$ soit dérivable au point $a\in I$ signifie que
\begin{equation}
	\lim_{x\to a} \frac{ f(x)-f(a) }{ x-a }=\ell
\end{equation}
pour un certain nombre $\ell$. Cela peut être récrit sous la forme
\begin{equation}
	\lim_{x\to a} \frac{ f(x)-f(a) }{ x-a }-\ell=0,
\end{equation}
ou encore
\begin{equation}
	\lim_{x\to a} \frac{ f(x)-f(a)-\ell(x-a) }{ x-a }=0.
\end{equation}
Introduisons la fonction
\begin{equation}
	\alpha(t)=\frac{ f(a+t)-f(a)-t\ell }{ t }.
\end{equation}
Cette fonction est faite exprès pour que
\begin{equation}		\label{EqIntermsaxaama}
	\alpha(x-a)=\frac{ f(x)-f(a)-\ell(x-a) }{ x-a };
\end{equation}
par conséquent $\lim_{x\to a} \alpha(x-a)=0$. Nous récrivons l'équation \eqref{EqIntermsaxaama} sous la forme
\begin{equation}        \label{EqCodeDerviffxam}
	f(x)-f(a)-\ell(x-a)=(x-a)\alpha(x-a).
\end{equation}
Le second membre tend vers zéro lorsque $x$ tend vers $a$ avec une «vitesse au carré» : c'est le produit de deux facteurs tous deux tendant vers zéro. Si $x$ n'est pas très loin de $a$, il n'est donc pas une mauvaise approximation de dire
\begin{equation}
	f(x)-f(a)-\ell(x-a)\simeq 0,
\end{equation}
c'est à dire
\begin{equation}		\label{Eqfxsimesfa}
	f(x)\simeq f(a)+f'(a)(x-a).
\end{equation}
Nous avons retrouvé l'équation \eqref{EqTgfaen}. La manipulation que nous venons de faire revient donc à dire que la fonction $f$, au voisinage de $a$, est bien approximée par sa tangente.

L'équation \eqref{Eqfxsimesfa} peut être aussi écrite sous la forme
\begin{equation}		\label{EqfxdxSimeqfxfpx}
	f(x+\Delta x)\simeq f(x)+f'(x)\Delta x
\end{equation}
qui est une approximation d'autant meilleure que $\Delta x$ est petit.

\begin{proposition}[\cite{ooAAFGooXRSaWs}]      \label{PROPooSGTBooFxUuXK}
    Soit \(f \) une fonction dérivable et strictement monotone de l'intervalle \( I\) sur l'intervalle \( J\)  (f est alors une bijection de $I$ vers $J$). Si  ne s'annule par sur  alors 
    \begin{enumerate}
        \item
            la fonction \( f\) est une bijection de \( I\) vers \( J\),
        \item
            la fonction \( f^{-1}\) est dérivable sur \( J\),
        \item
            et nous avons la formule
            \begin{equation}        \label{EQooELIHooDxUFxH}
                (f^{-1})'=\frac{1}{ f'\circ f^{-1} }.               
            \end{equation}
    \end{enumerate}
\end{proposition}
\index{réciproque!dérivabilité}

%+++++++++++++++++++++++++++++++++++++++++++++++++++++++++++++++++++++++++++++++++++++++++++++++++++++++++++++++++++++++++++
\section{Opérations sur les dérivées}
%+++++++++++++++++++++++++++++++++++++++++++++++++++++++++++++++++++++++++++++++++++++++++++++++++++++++++++++++++++++++++++

Pour continuer, nous allons en faire une un peu plus abstraite.
\begin{proposition}     \label{PropDerrLin}
    La dérivation est une opération linéaire, c'est à dire que
    \begin{enumerate}
        \item $(\lambda f)'=\lambda f'$ pour tout réel $\lambda$ où, pour rappel, la fonction $(\lambda f)$ est définie par $(\lambda f)(x)=\lambda\cdot f(x)$,
        \item $(f+g)'=f'+g'$.
    \end{enumerate}
\end{proposition}

\begin{proof}
Ces deux propriétés découlent des propriétés correspondantes de la limite. Nous allons faire la première, et laisser la seconde à titre d'exercice. Écrivons la définition de la dérivée avec $(\lambda f)$ au lieu de $f$, et calculons un petit peu :
\begin{equation}
    \begin{aligned}[]
        (\lambda f)'(x) &=\lim_{\epsilon\to 0}\frac{ (\lambda f)(x+\epsilon)-(\lambda f)(x) }{ \epsilon }\\
                &=\lim_{\epsilon\to 0}\frac{ \lambda \big( f(x+\epsilon) \big)-\lambda f(x) }{ \epsilon }\\
                &=\lim_{\epsilon\to 0}\lambda \frac{ f(x+\epsilon) -f(x) }{ \epsilon }\\
                &=\lambda \lim_{\epsilon\to 0}\frac{ f(x+\epsilon) -f(x) }{ \epsilon }\\
                &=\lambda f'(x).
    \end{aligned}
\end{equation}
\end{proof}


\begin{proposition}
    La dérivée d'un produit obéit à la \defe{règle de Leibnitz}{Règle de Leibnitz}\index{Leibnitz}:
    \begin{equation}
        (fg)'(x)=f'(x)g(x)+f(g)g'(x).
    \end{equation}
    Cette règle est souvent écrite sous la forme compacte $(fg)'=f'g+g'f$.
\end{proposition}

\begin{proof}
La définition de la dérivée dit que
\begin{equation}        \label{Eqfgrimeepsfgx}
    (fg)'(x)=\lim_{\epsilon\to 0}\frac{f(x+\epsilon)g(x+\epsilon)-f(x)g(x)}{\epsilon}.
\end{equation}
La subtilité est d'ajouter au numérateur la quantité $-f(x)g(x+\epsilon)+f(x)g(x+\epsilon)$, ce qui est permis parce que cette quantité est nulle\footnote{Le coup d'ajouter et enlever la même chose a déjà été fait durant la démonstration du théorème \ref{Tholimfgabab}. C'est une technique assez courante en analyse.}. Le numérateur de \eqref{Eqfgrimeepsfgx} devient donc
\begin{equation}
    \begin{aligned}[]
f(x+\epsilon)g(x+\epsilon)&-f(x)g(x+\epsilon)+f(x)g(x+\epsilon)-f(x)g(x) \\
            &= g(x+\epsilon)\big( f(x+\epsilon)-f(x) \big)+f(x)\big( g(x+\epsilon)-g(x) \big),
    \end{aligned}
\end{equation}
où nous avons effectué deux mises en évidence. Étant donné que nous avons deux termes, nous pouvons couper la limite en deux :
\begin{equation}
    \begin{aligned}[]
        (fg)'(x)    &=\lim_{\epsilon\to 0}g(x+\epsilon)\frac{ f(x+\epsilon)-f(x) }{\epsilon}            &+\lim_{\epsilon\to 0}f(x)\frac{ g(x+\epsilon)-g(x) }{\epsilon}\\
                &=\lim_{\epsilon\to 0}g(x+\epsilon)\lim_{\epsilon\to 0}\frac{ f(x+\epsilon)-f(x) }{\epsilon}    &+f(x)\lim_{\epsilon\to 0}\frac{ g(x+\epsilon)-g(x) }{\epsilon},
    \end{aligned}
\end{equation}
où nous avons utilisé le théorème \ref{Tholimfgabab} pour scinder la première limite en deux, ainsi que la propriété \eqref{Eqbutmultlim} pour sortir le $f(x)$ de la limite dans le second terme. Maintenant, dans le premier terme, nous avons évidemment\footnote{Pas tout à fait évidemment : selon le théorème \ref{ThoLimCont}, \emph{limite et continuité}, il faut que $g$ soit continue.} $\lim_{\epsilon\to 0}g(x+\epsilon)=g(x)$. Les limites qui restent sont les définitions classiques des dérivées de $f$ et $g$ au point~$x$ :
\begin{equation}
    (fg)'(x)=g(x)f'(x)-f(x)g'(x),
\end{equation}
ce qu'il fallait démontrer.
\end{proof}

%--------------------------------------------------------------------------------------------------------------------------- 
\subsection{Développement limité au premier ordre}
%---------------------------------------------------------------------------------------------------------------------------

Si une fonction est dérivable en \( a\) alors elle peut être approximée «au premier ordre» par une formule simple.
\begin{proposition}[Développement limité au premier ordre]  \label{PropUTenzfQ}
    Si \( f\) est dérivable en \( a\) alors nous avons la formule
    \begin{equation}
        f(a+h)=f(a)+hf'(a)+\alpha(h)
    \end{equation}
    pour une fonction \( \alpha\) telle que
    \begin{equation}
        \lim_{h\to 0} \frac{ \alpha(h) }{ h }=0.
    \end{equation}
\end{proposition}
\index{développement!limité!premier ordre}
Ce résultat sera généralisé pour des dérivées d'ordre supérieurs avec les séries de Taylor, théorème \ref{ThoTaylor}.

\begin{proof}
    La fonction \( f\) étant dérivable en \( a\) nous avons l'existence de la limite suivante :
    \begin{equation}
        f'(a)=\lim_{h\to 0} \frac{ f(a+h)-f(a) }{ h },
    \end{equation}
    ce qui revient à dire qu'en définissant la fonction \( \beta\) par
    \begin{equation}
        f'(a)=\frac{ f(a+h)-f(a) }{ h }+\beta(h)
    \end{equation}
    alors \( \beta(h)\to 0\) lorsque \( h\to 0\). En multipliant par \( h\) et en nommant \( \alpha(h)=h\beta(h)\) nous trouvons le résultat :
    \begin{equation}
        f(a+h)=f(a)+hf'(a)+\alpha(h)
    \end{equation}
    avec 
    \begin{equation}
        \lim_{h\to 0} \frac{ \alpha(h) }{ h }=\lim_{h\to 0} \beta(h)=0.
    \end{equation}
\end{proof}

%+++++++++++++++++++++++++++++++++++++++++++++++++++++++++++++++++++++++++++++++++++++++++++++++++++++++++++++++++++++++++++ 
\section{Dérivation à une variable}
%+++++++++++++++++++++++++++++++++++++++++++++++++++++++++++++++++++++++++++++++++++++++++++++++++++++++++++++++++++++++++++

%--------------------------------------------------------------------------------------------------------------------------- 
\subsection{Définition}
%---------------------------------------------------------------------------------------------------------------------------

\begin{definition}[Fonction dérivable]\label{defderivable}
    Nous disons qu'une fonction \( f\) est \defe{dérivable}{dérivable} au point \( x_0\in I\) si la limite
    \begin{equation}
        \lim_{\epsilon\to 0}\frac{ f(x_0+\epsilon)-f(x_0) }{ \epsilon }
    \end{equation}
    existe.
\end{definition}

Si \( f\) est une fonction dérivable, rien n'empêche la fonction dérivée \( f'\) d'être elle-même dérivable. Dans ce cas nous notons \( f''\) ou \( f^{(2)}\) la dérivée de la fonction \( f'\). Cette fonction $f''$ est la \defe{dérivée seconde}{dérivée!seconde} de \( f\). Elle peut encore être dérivable; dans ce cas nous notons \( f^{(3)}\) sa dérivée, et ainsi de suite. Nous définissons \( f^{(n)}=(f^{(n-1)})'\) la dérivée \( n\)\ieme de \( f\). Nous posons évidemment $f^{(0)}=f$.

\begin{theorem}
  Toute fonction $f$ dérivable au point $x_0$ est continue au point $x_0$. 
\end{theorem}

\begin{remark}
     La réciproque du théorème précédent n'est pas vraie : il existent bien des fonctions qui sont continues à un point $x_0$ mais qui ne sont pas dérivables en $x_0$. La fonction valeur absolue, $x\mapsto |x|$, par exemple est continue sur tout $\eR$ mais elle n'est pas dérivable en $0$. 
\end{remark}

%--------------------------------------------------------------------------------------------------------------------------- 
\subsection{Quelques formules à connaître}
%---------------------------------------------------------------------------------------------------------------------------

\begin{Aretenir}\label{formulesderivation}
  \begin{subequations}
    \begin{equation}
      \left(\alpha f(x) + \beta g(x)\right)' = \alpha f'(x)  + \beta g'(x).
    \end{equation}
    \begin{equation}
       \left(f(x)g(x)\right)' =  f'(x) g(x) + f(x) g'(x). 
    \end{equation}
    \begin{equation}
      \left(f(u(x))\right)' =  f'(u(x))u'(x). 
    \end{equation}
    \begin{equation}
      \left(\frac{f(x)}{g(x)}\right)' = \frac{f'(x) g(x) - f(x) g'(x)}{(g(x))^2}.
    \end{equation}
  \end{subequations}
\end{Aretenir}

\begin{proposition}
    Par rapport à la dérivation, les produits scalaire et vectoriel vérifient une règle de Leibnitz. Soit $I$ un intervalle de $\eR$. Si $u$ et $u$ sont dans $C^1(I,\eR^3)$, alors
    \begin{equation}		\label{EqFormLeibProdscalVect}
        \begin{aligned}[]
            \frac{ d }{ dt }\big( u(t)\cdot v(t) \big)&=\big( u'(t)\cdot v(t) \big)+\big( u(t)\cdot v'(t) \big)\\
            \frac{ d }{ dt }\big( u(t)\times v(t) \big)&=\big( u'(t)\times v(t) \big)+\big( u(t)\times v'(t) \big).
        \end{aligned}
    \end{equation}
\end{proposition}
