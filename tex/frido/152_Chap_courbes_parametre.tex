% This is part of Mes notes de mathématique
% Copyright (c) 2010-2016
%   Laurent Claessens, Carlotta Donadello
% See the file fdl-1.3.txt for copying conditions.

%---------------------------------------------------------------------------------------------------------------------------
\subsection{Tangente à une courbe paramétrée}
%---------------------------------------------------------------------------------------------------------------------------

\begin{definition}
Soit $(I,\gamma)$ un arc paramétré de classe $\mathcal{C}^k$ avec $k\geq 1$. Nous disons que la courbe admet une \defe{tangente}{tangente} en $\gamma(t_0)\in\eR^n$ lorsque les deux conditions suivantes sont remplies
\begin{enumerate}
    \item
        $\gamma(t)\neq \gamma(t_0)$ pour tout $t$ dans un voisinage de $t_0$;
    \item
        la direction de la droite qui passe par $\gamma(t)$ et $\gamma(t_0)$ admet une limite lorsque $t\to t_0$.
\end{enumerate}
Dans ce cas, la tangente sera la droite passant par le point $\gamma(t_0)$ et dont la direction est donnée par la limite.
\end{definition}
Dans cette définition, par \defe{direction}{direction} d'une droite, nous entendons le vecteur de norme $1$ parallèle à celle-ci sans tenir compte du signe. La tangente sera donc la droite passant par $\gamma(t_0)$ et parallèle au vecteur
\begin{equation}
\lim_{t\to t_0}\frac{ \gamma(t)-\gamma(t_0) }{ \| \gamma(t)-\gamma(t_0) \| }. 
\end{equation}
Évidement si nous avions écrit $\gamma(t_0)-\gamma(t)$, ça n'aurait pas changé la droite. Par abus de langage, nous parlerons souvent de «la direction $u$» même lorsque $u$ n'est pas de norme $1$.

Formellement, une direction est une classe d'équivalence de vecteurs pour la relation $u\sim v$ s'il existe $\lambda\neq 0$ tel que $u=\lambda v$, mais nous n'aurons pas besoin de cette précision ici.

Sans surprises, la tangente est à peu près toujours donnée par la dérivée lorsqu'elle existe. Plus précisément nous avons le
\begin{theorem}
Soit $(I,\gamma)$, un arc paramétré de classe $\mathcal{C}^k$ ($k\geq 1$) et $t_0\in I$ tel que
\begin{equation}
    \gamma'(t_0)=\gamma''(t_0)=\ldots=\gamma^{(q-1)}(t_0)=0
\end{equation}
et
\begin{equation}
    \gamma^{(q)}(t_0)\neq 0
\end{equation}
pour un entier $1\leq q\leq k$. Alors $\gamma$ admet une tangente en $\gamma(t_0)$ de direction $\gamma^{(q)}(t_0)$.
\end{theorem}

\begin{proof}

Le développement de $\gamma(t_0)$ en série de Taylor autour de $t$ jusqu'à l'ordre $q$ est
\begin{equation}        \label{EqDevTaylfttzq}
    \begin{aligned}[]
        \gamma(t)&=\gamma(t_0)+\gamma'(t_0)| t-t_0 |+\frac{ \gamma't(t_0) }{2}| t-t_0 |^2+\cdots +\frac{ \gamma^{(q)}(t_0) }{ q! }| t-t_0 |^q\\
            &\quad+\varepsilon(t)| t-t_0 |^q
    \end{aligned}
\end{equation}
où $\varepsilon$ est une application $\varepsilon\colon \eR\to \eR^n$ telle que $\lim_{t\to t_0} \varepsilon(t)=0$. En utilisant les hypothèses, nous éliminons la majorité des termes dans le développement \eqref{EqDevTaylfttzq} :
\begin{equation}
    \gamma(t)-\gamma(t_0)=\frac{1}{ q! }\gamma^{(q)}(t_0)| t-t_0 |^q+\varepsilon(t)| t-t_0 |^q.
\end{equation}
La direction de la droite qui joint $\gamma(t)$ à $\gamma(t_0)$ est donc donnée par
\begin{equation}
    \frac{ \gamma(t)-\gamma(t_0) }{ \| \gamma(t)-\gamma(t_0) \| }=\frac{ \frac{1}{ q! }\gamma^{(q)}(t_0)| t-t_0 |^q+\varepsilon(t)| t-t_0 |^q }{ \| \frac{1}{ q! }\gamma^{(q)}(t_0)| t-t_0 |^q+\varepsilon(t)| t-t_0 |^q\|  }
\end{equation}
et la limite lorsque $t\to t_0$ donne $\gamma^{(q)}(t_0)$ comme direction de la tangente.

\end{proof}

Lorsque le théorème s'applique, le vecteur
\begin{equation}
\tau=\frac{ \gamma^{(q)}(t_0) }{ \| \gamma^{(q)}(t_0) \| }
\end{equation}
est appelé le \defe{vecteur unitaire tangent}{vecteur!unitaire tangent} en $\gamma(t_0)$ à l'arc paramétré $\gamma$.


\begin{corollary}       \label{CorTgSoCun}
Si $(I,\gamma)$ est un arc paramétré de classe $\mathcal{C}^1$ régulier (c'est à dire $\gamma'(t)\neq 0$ pour tout $t$) alors l'arc admet une tangente en tout point et le vecteur unitaire de la tangente est donné par
\begin{equation}
    \tau(t)=\frac{ \gamma'(t) }{ \| \gamma'(t) \| },
\end{equation}
pour tout $t$ dans $I$.
\end{corollary}

\begin{corollary}       \label{CorUnitTgtaugpnorma}
Si $\gamma=(J,\gamma_N)$ est un arc paramétré de classe $\mathcal{C}^1$, normal, alors le vecteur unitaire de la tangente au point $\gamma_N(s)$ est donné par $\tau(s)=\gamma_N'(s)$.
\end{corollary}

\begin{proof}
Nous devons démontrer que dans le cas d'une paramétrisation normale nous avons $\| \gamma_N'(s) \|=1$ pour tout $s$. Par définition,
\begin{equation}
    l\big( \mathopen[ s , s' \mathclose],g \big)=\int_s^{s'}\| \gamma_N'(u) \|du=s'-s.
\end{equation}
Par conséquent,
\begin{equation}
    \lim_{h\to 0} \frac{1}{ h }\int_s^{s+h}\| \gamma_N'(u) \|du=\lim_{y\to 0} \frac{ s+h-s }{ h }=1.
\end{equation}
Cela implique que $\| \gamma_N'(s) \|=1$, et donc en particulier que $(J,\gamma_N)$ est un arc régulier. Le corollaire précédent montre alors que $\tau(s)=\gamma_N'(s)/\| \gamma_N'(s) \|=\gamma_N'(s)$.
\end{proof}

\begin{example}
Considérons la courbe $\gamma(t)=(t^2,t^3)$, et cherchons la tangente en $t_0=0$. En dérivant nous avons successivement 
\begin{equation}
    \begin{aligned}[]
        \gamma(t)&=(t^2,t^3)\\
        \gamma'(t)&=(2t,3t^2)\\
        \gamma''(t)&=(2,6t).
    \end{aligned}
\end{equation}
En posant $t=0$, nous trouvons que $\gamma'(0)=0$ mais $\gamma''(0)=(2,0)\neq 0$. Le théorème nous dit donc que la direction de la tangente est horizontale. Nous pouvons faire le calcul directement :
\begin{equation}
    \frac{ \gamma(t)-\gamma(t_0) }{ \| \gamma(t)-\gamma(t_0) \| }=\frac{ (t^2,t^3) }{ \sqrt{t^4+t^6} }=\frac{ (t^2,t^3) }{ t^2\sqrt{1+t^2} }=\frac{ (1,t) }{ \sqrt{1+t^2} },
\end{equation}
dont la limite \( t\to 0\) est bien le vecteur horizontal $(1,0)$.

La figure \ref{LabelFigParamTangente} montre quelque tangente, c'est à dire quelques vecteurs dans la direction $\gamma'(t)$ (pour les $t\neq 0$, il ne faut pas aller à la dérivée seconde). Nous remarquons que de part et d'autres du sommet, les vecteurs ne sont pas dirigés dans le même sens. \emph{En tant que vecteurs} de norme $1$, ces vecteurs n'ont pas de limites quand $t\to 0$. Ce sont bien les \emph{directions} qui ont une limite, parce que la direction ne tient pas compte du sens.
\newcommand{\CaptionFigParamTangente}{Quelques tangentes de la courbe $\gamma(t)=(t^2,t^3)$.}
\input{auto/pictures_tex/Fig_ParamTangente.pstricks}

\end{example}

%+++++++++++++++++++++++++++++++++++++++++++++++++++++++++++++++++++++++++++++++++++++++++++++++++++++++++++++++++++++++++++
\section{Repère de Frenet}      \label{SecFrenet}
%+++++++++++++++++++++++++++++++++++++++++++++++++++++++++++++++++++++++++++++++++++++++++++++++++++++++++++++++++++++++++++

Dans cette section, nous ne considérons que des courbes dans $\eR^3$.

\begin{proposition}     \label{Proptausclataupzero}
    Soit $\gamma=(J,\gamma_N)$ un arc paramétré normal de classe $\mathcal{C}^2$. Alors pour toute valeur de $s$ dans $J$, nous avons
    \begin{equation}
        \tau(s)\cdot\tau'(s)=0
    \end{equation}
    où $\tau(s)=\gamma_N'(s)$. C'est à dire que la dérivée seconde est perpendiculaire à la dérivée première.
\end{proposition}

\begin{proof}
    La paramétrisation étant normale, nous avons 
    \begin{equation}
        \| \gamma_N'(s) \|^2=\sum_{i=1}^nx'_i(s)^2=1;
    \end{equation}
    ce qui implique, en dérivant les deux membres, que
    \begin{equation}
        0=2\sum_{i=1}^nx_i'(s)x''_i(s),
    \end{equation}
    c'est à dire exactement $\gamma_N'(s)\cdot \gamma_N''(s)=0$; d'où la thèse.
\end{proof}

\begin{remark}
    Si nous n'utilisons pas des coordonnées normales, la proposition \ref{Proptausclataupzero} n'est pas spécialement vraie. Prenons par exemple la courbe qui donne la parabole :
    \begin{subequations}
        \begin{align}
            \gamma(t)&=(t,t^2)\\
            \gamma'(t)&=(1,2t)\\
            \gamma''(t)&=(0,2)
        \end{align}
    \end{subequations}
    Nous avons $\gamma'(t)\cdot \gamma''(t)=4t$. Par conséquent, la dérivée seconde n'est la normale à la courbe que en $t=0$. Cela est une propriété très intéressante des coordonnées normales : la dérivée seconde d'une coordonnées normale donne un vecteur normal à la courbe, c'est à dire perpendiculaire à la tangente.
\end{remark}

\begin{definition}      \label{DefCourbureNormleUnit}
    Soit $\gamma=(J,\gamma_N)$ un arc paramétré normal de classe $\mathcal{C}^2$. 
    \begin{enumerate}
        \item
            Le \defe{vecteur unitaire tangent}{tangent!vecteur unitaire} est donné par le corollaire \ref{CorTgSoCun} : \( \tau(t)=\gamma'(t)/\| \gamma'(t) \|\).
        \item
    La \defe{normale principale}{normale!principale} est le vecteur $\tau'(s)$. Le \defe{vecteur unitaire normal}{unitaire!normale principale}\index{vecteur!unitaire normal} est le vecteur\nomenclature[C]{$\nu(s)$}{Vecteur unitaire de la normale principale}
    \begin{equation}
        \nu(s)=\frac{ \tau'(s) }{ \| \tau'(s) \| }=\frac{ \gamma_N''(s) }{ \| \gamma_N''(s) \| }.
    \end{equation}
    Nous déduirons une formule plus pratique en dehors des coordonnées normales en \eqref{EqCourburetermf}.
\item
    La \defe{courbure}{courbure} au point $\gamma_N(s)$ est le réel\nomenclature[C]{$c(s)$}{rayon de courbure}
    \begin{equation}
        c(s)=\| \tau'(s) \|=\| \gamma_N''(s) \|.
    \end{equation}
    Note : il y a une notion de courbure signée qui sera donnée dans la définition \ref{DEFooJFWEooXcIVUs}.
\item
    Le \defe{rayon de courbure}{rayon!de courbure} est le réel
    \begin{equation}
        R(s)=\frac{1}{ c(s) }=\frac{1}{ \| \gamma_N''(s) \| }.
    \end{equation}
    \end{enumerate}
\end{definition}

Par la proposition \ref{Proptausclataupzero}, nous avons $\nu(s)\cdot\tau(s)=0$. En combinant toutes les formules, nous avons les différentes expressions suivantes pour le vecteur normal unitaire :
\begin{equation}        \label{Eq0908nufractauRc}
    \nu(s)=\frac{ \gamma_N''(s) }{ c(s) }=\frac{ \tau'(s) }{ \| \tau'(s) \| }=\frac{ \tau'(s) }{ c(s) }=R(s)\tau'(s)=R(s)\gamma_N''(s).
\end{equation}

\begin{proposition}
    La fonction courbure s'écrit $c=\| \gamma_N'\times \gamma_N'' \|$.
\end{proposition}

\begin{proof}
    Par le point \ref{ItemPropScalMixtLiniv} de la proposition \ref{PropScalMixtLin}, nous avons
    \begin{equation}
        \langle \gamma_N', \gamma_N''\rangle^2 + \| \gamma_N'\times \gamma_N'' \|^2=\| \gamma_N' \|^2\| \gamma_N'' \|^2=\| \gamma_N'' \|^2
    \end{equation}
    parce que, la paramétrisation étant normale, $\| \gamma_N' \|=1$. Mais $\langle \gamma_N', \gamma_N''\rangle =0$, donc il reste $\| \gamma_N'\times \gamma_N'' \|^2=\| \gamma_N'' \|^2$, d'où
    \begin{equation}        \label{Eqcsnormgpgpps}
        c(s)=\| \gamma_N''(s) \|=\| \gamma_N'(s)\times \gamma_N''(s) \|
    \end{equation}
    pour chaque $s$ dans $J$.
\end{proof}

\begin{definition}
    Soit $s$ un point birégulier (c'est à dire $\gamma_N'(s)\neq 0$ et $\gamma_N''(s)\neq 0$) de l'arc normal $\gamma=(J,\gamma_N)$. Le \defe{vecteur unitaire de la binormale}{binormale} est le vecteur\nomenclature[C]{$\beta(s)$}{Vecteur unitaire de la binormale}
    \begin{equation}
        \beta(s)=\tau(s)\times\nu(s)
    \end{equation}
\end{definition}

Par leurs définitions, $\tau$ et $\nu$ sont unitaires, tandis que la proposition \ref{Proptausclataupzero} montre qu'ils sont également orthogonaux. Les propriétés du produit vectoriel font que $\beta$ est également unitaire, et simultanément orthogonal à $\tau$ et à $\nu$.

\begin{definition}
    Le repère orthonormal $\{ \gamma_N(s),\tau(s),\beta(s) \}$ est le \defe{repère de Frenet}{repère!de Frenet} au point $\gamma_N(s)$.
\end{definition}

\begin{lemma}
    Le  vecteur unitaire normal est donné par $\nu(s)=\beta(s)\times \tau(s)$.
\end{lemma}

\begin{proof}
    Ceci est une application de la formule d'expulsion \eqref{EqFormExpluxxx} et de l'orthonormalité de la base de Frenet :
    \begin{equation}
        \beta\times\tau=(\tau\times\nu)\times\tau=-(\nu\cdot\tau)\tau+(\tau\cdot\tau)\nu=\nu.
    \end{equation}
\end{proof}

%---------------------------------------------------------------------------------------------------------------------------
\subsection{Torsion}
%---------------------------------------------------------------------------------------------------------------------------

Décomposons le vecteur $\beta'(s)$ dans la base de Frenet. Pour cela nous allons utiliser la proposition \ref{PropScalCompDec} et montrer que $\beta'(s)\cdot \tau(s)=\beta'(s)\cdot\beta(s)=0$, ce qui voudra dire que, dans la base de Frenet, les composantes de $\beta'$ le long de $\tau$ et $\beta$ sont nulles. Le vecteur $\beta'$ sera donc colinéaire à $\nu$.

D'abord, étant donné que la norme de $\beta(s)$ est constante par rapport à $s$, nous avons
\begin{equation}
    0=\frac{ d }{ ds }\| \beta(s) \|^2=2\beta'(s)\cdot\beta(s).
\end{equation}
Ensuite, nous dérivons la définition $\beta(s)=\tau(s)\times\nu(s)$ en utilisant la formule de Leibnitz \eqref{EqFormLeibProdscalVect} :
\begin{equation}
    \beta'(s)=\tau'(s)\times\nu(s)+\tau(s)\times\nu'(s).
\end{equation}
Mais $\tau'(s)=\gamma_N''(s)$ tandis que $\nu(s)=\frac{ \gamma_N''(s) }{ \| \gamma_N''(s) \| }$, de telle sorte que $\tau'(s)\times\nu(s)=0$. Nous restons donc avec $\beta'(s)=\tau(s)\times\nu'(s)$, ce qui prouve que $\beta'(s)$ est perpendiculaire à $\tau(s)$ et donc que $\beta'(s)\cdot\tau(s)=0$.

Le vecteur $\beta'(s)$ est donc un multiple de $\nu(s)$. Nous notons $t(s)$\nomenclature[C]{$t(s)$}{Torsion} le facteur de proportionnalité : 
\begin{equation}
    \beta'(s)=t(s)\nu(s).
\end{equation}

\begin{definition}      \label{DefTorsion}
    Soit $\gamma=(J,\gamma_N)$ un arc paramétré normal de classe $\mathcal{C}^3$. La \defe{torsion}{torsion} de $\gamma$ au point $\gamma_N(s)$ est le réel
    \begin{equation}
        t(s)=\| \beta'(s) \|=\| \tau(s)\times\nu'(s) \|.
    \end{equation}
    Lorsque $t(s)\neq 0$, le réel $T(s)=\frac{1}{ t(s) }$ est le \defe{rayon de torsion}{rayon!de torsion} de $\gamma$ en $\gamma_N(s)$.
\end{definition}

Étant donné que pour chaque $s$, l'ensemble $\{ \tau(s),\nu(s),\beta(s) \}$ est une base, il est naturel de vouloir décomposer leurs dérivées dans cette base. D'abord, par définition de $c$ et de $t$, nous avons
\begin{equation}
    \begin{aligned}[]
        \tau'(s)&=c(s)\nu(s)\\
        \beta'(s)&=t(s)\nu(s).
    \end{aligned}
\end{equation}
Il reste à décomposer $\nu'(s)$. Définissons $\alpha_{\tau}$, $\alpha_{\nu}$ et $\alpha_{\beta}$ (qui peuvent dépendre de $s$) par
\begin{equation}
    \nu'(s)=\alpha_{\tau}\tau(s)+\alpha_{\nu}\nu(s)+\alpha_{\beta}\beta(s).
\end{equation}
En vertu de la proposition \ref{PropScalCompDec}, nous avons
\begin{equation}
    \begin{aligned}[]
        \alpha_{\tau}=\langle \nu'(s), \tau(s)\rangle&=-\langle \nu(s), \tau'(s)\rangle =-\langle \nu(s), c(s)\nu(s)\rangle =-c(s) ,\\
        \alpha_{\nu}=\langle \nu'(s),  \nu(s)\rangle &=0,\\
        \alpha_{\beta}=\langle \nu'(s), \beta(s)\rangle &=-\langle \nu(s), \beta'(s)\rangle =-t(s),
    \end{aligned}
\end{equation}
où nous avons utilisé le fait que $\langle \nu(s), \nu(s)\rangle =\| \nu(s) \|^2=1$. Si nous mettons ces résultats sous forme matricielle, nous avons les \defe{formules de Frenet}{Frenet!formules} :
\begin{equation}
    \begin{pmatrix}
        \tau'(s)    \\ 
        \nu'(s) \\ 
        \beta'(s)   
    \end{pmatrix}=
    \begin{pmatrix}
        0   &   c(s)    &   0   \\
        -c(s)   &   0   &   -t(s)   \\
        0   &   t(s)    &   0
    \end{pmatrix}
    \begin{pmatrix}
        \tau(s) \\ 
        \nu(s)  \\ 
        \beta(s)    
    \end{pmatrix}.
\end{equation}


\begin{proposition}
    Si $s$ est un point birégulier, alors la torsion est donnée par
    \begin{equation}
        t(s)=-\frac{ (\gamma_N'\times \gamma_N'')\times \gamma_N''' }{ \| \gamma_N'(s)\times \gamma_N''(s) \|^2 }.
    \end{equation}
\end{proposition}

\begin{proof}
    Par l'équation \eqref{Eq0908nufractauRc}, nous avons $\gamma_N''(s)=c'(s)\nu(s)$, et par conséquent
    \begin{equation}
        \gamma_N'''(s)=c'(s)\nu(s)+c(s)\nu'(s)=c'(s)\nu(s)+c(s)\big[ -c(s)\tau(s)-t(s)\beta(s) \big],
    \end{equation}
    où nous avons utilisé la formule de Frenet pour $\nu'(s)$. Par ailleurs, sachant le corollaire \ref{CorUnitTgtaugpnorma} et la formule de Frenet pour $\tau'$, nous avons
    \begin{equation}
        \gamma_N'\times \gamma_N''=\tau(s)\times \tau'(s)=\tau(s)\times c(s)\nu(s)=c(s)\beta(s).
    \end{equation}
    En combinant les deux dernières équations, et en se souvenant que la base de Frenet et orthonormale,
    \begin{equation}
        (\gamma_N'\times \gamma_N'')\cdot \gamma_N'''(s)=-c(s)^2t(s),
    \end{equation}
    et donc, en remplaçant $c(s)$ par la formule \eqref{Eqcsnormgpgpps},
    \begin{equation}
        t(s)=-\frac{  (\gamma_N'\times \gamma_N'')\cdot \gamma_N'''   }{ \| \gamma_N'\times \gamma_N'' \|^2 }.
    \end{equation}
\end{proof}

%+++++++++++++++++++++++++++++++++++++++++++++++++++++++++++++++++++++++++++++++++++++++++++++++++++++++++++++++++++++++++++
\section{Hors des coordonnées normales}
%+++++++++++++++++++++++++++++++++++++++++++++++++++++++++++++++++++++++++++++++++++++++++++++++++++++++++++++++++++++++++++

\begin{remark}      \label{Remfougnormoupad}
    Notons que la définition de $\tau$ est donnée pour tout arc $\mathcal{C}^1$ régulier $(I,\gamma)$ par $\tau(t)=\gamma'(t)/\| \gamma'(t) \|$. La propriété $\tau=\gamma_N'$ n'est valable que lorsque la paramétrisation est normale. Les autres définitions ont toutes été données dans le cas d'une paramétrisation normale.
\end{remark}

La remarque \ref{Remfougnormoupad} nous incite à exprimer toute la base de Frenet en terme de $\gamma$ lorsque la paramétrisation n'est pas normale. Étant donné que nous pouvons toujours faire le changement de variable $\gamma(t)=\gamma_N\big( \phi(t) \big)$ (proposition \ref{PropExisteChmNorm}), il est possible d'exprimer les vecteurs $\tau$, $\nu$ et $\beta$ ainsi que les réels $c$ et $t$ en fonction de $\gamma$ et de ses dérivées. 

Nous allons maintenant travailler à écrire les formules. 

Pour plus de facilité, nous collectons les définitions. Afin d'alléger la notation, nous n'exprimons pas explicitement les dépendances en $s$ :
\begin{description}
    \item[Vecteur unitaire tangent] 
        Par le corollaire \ref{CorUnitTgtaugpnorma}, $\tau$ est donné par $\tau=\gamma_N'$.
    \item[Vecteur unitaire normal] 
        Par la définition \ref{DefCourbureNormleUnit}, $\nu$ est donné par
        $\nu=\frac{ \tau' }{ \| \tau' \| }$.
    \item[Vecteur unitaire de la binormale] 
        Par la définition \ref{DefCourbureNormleUnit}, $\beta$ est donné par
            $\beta=\tau\times\nu$.
    \item[Courbure] 
        Par la définition \ref{DefCourbureNormleUnit}, $c$ est donné par
            $c=\| \tau' \|$.
    \item[Torsion]
        Par la définition \ref{DefTorsion}, $t$ est donné par
            $t=\| \beta' \|$.
\end{description}


Le schéma du changement de variable est
\begin{equation}        \label{EqDiagIJstgvpR}
    \xymatrix{%
    t\in I \ar[r]^{f}\ar[d]_{\phi}      &   \eR^3\\
    s\in J \ar[ru]_{g}  &   
       }
\end{equation}
La difficulté ne sera pas d'éliminer $\gamma_N$ de toutes les formule, mais bien de se débarrasser des fonctions $\phi$ qui arrivent quand nous exprimons $\gamma_N$ en termes de $\gamma$, et en particulier lorsque nous voulons exprimer les dérivées de $\gamma_N$ en termes de $\gamma$ et de ses dérivées.

Regardons d'abord comment les dérivées de $\gamma_N$ s'expriment en termes de $\gamma$. En utilisant le fait que $\gamma_N(s)=(\gamma\circ\phi^{-1})(s)$ et que $\| \gamma_N'(s) \|=1$, nous avons
\begin{equation}        \label{EqgpNgpnNnr}
    \gamma_N'(s)=\frac{ \gamma_N'(s) }{ \| \gamma_N'(s) \| }
    =\frac{ (\gamma\circ\phi^{-1})'(s) }{ \| (\gamma\circ\phi^{-1})'(s) \| }
    =\frac{ \gamma'\big( \phi^{-1}(s) \big)   (\phi^{-1})'(s)   }{ \| \gamma'\big( \phi^{-1}(s) \big) \|  |(\phi^{-1})'(s) |}
    =\frac{ \gamma'(t) }{ \| \gamma'(t) \| }
\end{equation}
où nous avons utilisé le fait que $\phi^{-1}$ étant croissante (parce que l'inverse d'une fonction croissante est croissante), $(\phi^{-1})'(s)=| (\phi^{-1})'(s) |$. Pourquoi écrivons nous $| \phi^{-1}(s) |$ et non $\| \phi^{-1}(s) \|$ ?

Pour la dérivée seconde, nous dérivons la relation \eqref{EqgpNgpnNnr} :
\begin{equation}
    \gamma_N''(s)=\frac{ \gamma''\big( \phi^{-1}(s) \big)(\phi^{-1})'(s) }{ \| \gamma'\big( \phi^{-1}(s) \big) \| }+\gamma'\big( \phi^{-1}(s) \big)\frac{ d }{ ds }\Big[ \| \gamma'\big( \phi^{-1}(s) \big) \| \Big].
\end{equation}
Le petit calcul suivant va nous permettre de simplifier cette expression :
\begin{equation}        \label{Eavpemuetfpnorm}
    (\phi^{-1})'(s)=(\phi^{-1})'\big( \phi(t) \big)=\frac{1}{ \phi'(t) }=\frac{1}{ \| \gamma'(t) \| }.
\end{equation}
Donc
\begin{equation}
    \gamma_N''(s)=\frac{ \gamma''(t) }{ \| \gamma'(t) \|^2 }+\gamma'(t)\frac{ d }{ ds }\Big[ \| \gamma'(t) \| \Big]
\end{equation}
où il est entendu que $t=\phi^{-1}(s)$. Avec cette expression, nous ne nous sommes pas encore débarrassés de la fonction $\phi$, mais nous allons voir que cela nous sera suffisant.

Pour le vecteur unitaire tangent $\tau(s)$, nous avons donc immédiatement
\begin{equation}        \label{EqTauavect}
    \tau(s)=\gamma_N'(s)=\frac{ \gamma'(t) }{ \| \gamma'(t) \| }.
\end{equation}
Ici encore il est sous-entendu que le $t$ dans le membre de droite est lié au $s$ du membre de gauche par $t=\phi^{-1}(s)$. Il est donc naturel de nous demander si nous avons gagné quelque chose, étant donné que la formule \eqref{EqTauavect} contient encore la fonction $\phi$.

Géométriquement, le vecteur $\tau(s)$ est le vecteur normal unitaire de la courbe au point $\gamma_N(s)$. En utilisant les relations du diagramme \eqref{EqDiagIJstgvpR}, nous avons en réalité $\gamma_N(s)=\gamma_N\big( \phi(t) \big)=\gamma(t)$. Le vecteur $\frac{ \gamma'(t) }{ \| \gamma'(t) \| }$ représente donc le vecteur normal tangent au point $\gamma(t)$.

Pour calculer la courbure, nous devons d'abord calculer le produit vectoriel
\begin{equation}        \label{eqProdvectogpgpp}
    \begin{aligned}[]
        \gamma_N'(s)\times \gamma_N''(s) &=  \frac{ \gamma'(t) }{ \| \gamma'(t) \| }\times \left( \frac{ \gamma''(t) }{ \| \gamma'(t) \|^2 }+\gamma'(t)\frac{ d }{ ds }\Big[ \| \gamma'(t) \| \Big] \right)\\
        &=\frac{ \gamma'(t)\times \gamma''(t) }{ \| \gamma'(t) \|^3 }
    \end{aligned}
\end{equation}
parce que le deuxième terme dans la parenthèse est un multiple de $\gamma'(t)$, de telle sorte à ce que son produit vectoriel avec $\gamma'(t)/\| \gamma'(t) \|$ soit nul. En prenant la norme,
\begin{equation}        \label{EqCourburetermf}
    c(s)=\frac{ \| \gamma'(t)\times \gamma''(t) \| }{ \| \gamma'(t) \|^3 }.
\end{equation}
Encore une fois, cette équation nous enseigne que la courbure au point $\gamma(t)\in\eR^3$ est donnée par le membre de droite, qui ne dépend que de $t$.

Le vecteur unitaire binormal est donné par $\beta(s)=\tau(s)\times \nu(s)$. En utilisant \eqref{EqTauavect} et \eqref{Eq0908nufractauRc},
\begin{equation}
    \beta(s)=\tau(s)\times\nu(s)=\gamma_N'(s)\times \frac{ \gamma_N''(s) }{ c(s) }.
\end{equation}
Les formules \eqref{eqProdvectogpgpp} pour le produit vectoriel et \eqref{EqCourburetermf} pour la courbure donnent ensuite
\begin{equation}
    \beta(s)=\frac{ \gamma'(t)\times \gamma''(t) }{ \| \gamma'(t) \|^3 }\cdot\frac{1}{ c(s) }=\frac{ \gamma'(t)\times \gamma''(t) }{ \|  \gamma'(t)\times \gamma''(t)  \| }.
\end{equation}
Cela donne le vecteur unitaire binormal au point $\gamma(t)$ en terme de $\gamma'(t)$ et $\gamma''(t)$.

La torsion demande d'utiliser la dérivée troisième de $\gamma_N$. Nous avons
\begin{equation}
    \begin{aligned}[]
        \gamma_N'''(s)&=(\gamma\circ\phi^{-1})'''(s)\\
        &=\Big( \gamma'\big( \phi^{-1}(s) \big)(\phi^{-1})'(s) \Big)''\\
        &=\Big( \gamma''\big( \phi^{-1}(s) \big)(\phi^{-1})'(s)^2+\gamma'\big( \phi^{-1}(s) \big)(\phi^{-1})''(s) \Big)'\\
        &=\gamma'''\big( \phi^{-1}(s) \big)(\phi^{-1})'(s)^3+ v\\
        &=\frac{ \gamma'''\big( \phi^{-1}(s) \big) }{ \| \gamma'(t) \|^3 }+v&&\text{par \eqref{Eavpemuetfpnorm}}
    \end{aligned}
\end{equation}
où $v$ est un élément de $\langle \gamma''\big( \phi^{-1}(s) \big),\gamma'\big( \phi^{-1}(s) \big)\rangle$. Le vecteur $v$ est donc perpendiculaire à $\gamma'\times \gamma''$ et donc à $\gamma_N'\times \gamma_N''$ à cause de la relation \eqref{eqProdvectogpgpp} qui montre que $\gamma'\times \gamma''$ est parallèle à $\gamma_N'\times \gamma_N''$. De ce fait, lorsque nous calculons $(\gamma_N'\times \gamma_N'')\cdot \gamma_N'''$, la partie $v$ de $\gamma_N'''$ n'entre pas en ligne de compte.

Nous avons donc le calcul suivant, en remplaçant les diverses occurrences de $\gamma_N'\times \gamma_N''$ par sa valeur \eqref{eqProdvectogpgpp} en termes de $\gamma$,
\begin{equation}
    \begin{aligned}[]
        t(s)&=-\frac{ (\gamma_N'\times \gamma_N'')\cdot \gamma_N''' }{ \| \gamma_N'\times \gamma_N'' \|^2 }\\
        &=-\frac{ (\gamma_N'\times \gamma_N'')\cdot \gamma'''(t) }{ \| \gamma_N'\times \gamma_N'' \|^2\,\| \gamma'(t) \|^2 }\\
        &=-\frac{ (\gamma'\times \gamma'')\cdot \gamma''' }{ \| \gamma'\times \gamma'' \|^2 }.
    \end{aligned}
\end{equation}
Dans cette expression, il est sous-entendu que tous les $\gamma_N$ sont fonctions de $s$ et tous les $\gamma$ sont fonction de $t$ où $s$ et $t$ sont liés par $s=\phi(t)$.

Ce que nous avons prouvé est le 
\begin{theorem}
    Pour tout représentant $(I,\gamma)$, les éléments métriques $(\tau,\nu,\beta,c,t)$ au point $\gamma(t)$ s'expriment en fonction de $\gamma(t)$, $\gamma'(t)$, $\gamma''(t)$ et $\gamma'''(t)$.
\end{theorem}

\begin{lemma}
    Si \( \gamma\) est le graphe de la fonction \( y\) alors la courbure de \( \gamma\) est donnée par la formule
    \begin{equation}
        c\big( \gamma(t) \big)=\frac{ | y''(t) | }{ \big( 1+y'(t)^2 \big)^{3/2} }
    \end{equation}
\end{lemma}

\begin{proof}
    Nous avons :
    \begin{subequations}
        \begin{align}
            \gamma(t)=\big( t,y(t) \big)\\
            \gamma'(t)=\big( 1,y'(t) \big)\\
            \gamma''(t)=\big( 0,y''(t) \big).
        \end{align}
    \end{subequations}
    Il s'agit maintenant seulement d'utiliser la formule \eqref{EqCourburetermf} en se souvenant comment on calcule un produit vectoriel\footnote{Définition \ref{DEFooTNTNooRjhuJZ}.} :
    \begin{equation}
        \gamma'\times \gamma''=
        \begin{vmatrix}
            e_1&e_2&e_3\\
            1&y'&0\\
            0&y''&0
        \end{vmatrix}=y''.
    \end{equation}
\end{proof}

%+++++++++++++++++++++++++++++++++++++++++++++++++++++++++++++++++++++++++++++++++++++++++++++++++++++++++++++++++++++++++++
\section{Tracer des courbes paramétriques dans $\eR^2$}     \label{SecTracerParmCourbe}
%+++++++++++++++++++++++++++++++++++++++++++++++++++++++++++++++++++++++++++++++++++++++++++++++++++++++++++++++++++++++++++

Nous allons maintenant voir comment les concepts introduits nous aident à effectivement tracer des courbes dans le plan. Les courbes que nous regardons sont de la forme $\gamma(t)=\big( x(t),y(t) \big)$, et nous supposons que ces fonctions soient suffisamment régulières (disons trois fois continument dérivables). Nous ne supposons pas que la courbe soit donnée en coordonnées normales, en particulier, $\gamma''(t)$ n'est pas le vecteur normal en $\gamma(t)$.

La notion clef qui va jouer est le \defe{cercle osculateur}{osculateur (cercle)} de la courbe $\gamma$ au point $\gamma(t)$. Sans rentrer dans les détails, disons que c'est le cercle qui «colle» le mieux possible la courbe. Le rayon de ce cercle est le rayon de courbure :
\begin{equation}
    R(t)=\frac{ \| \gamma(t) \|^3 }{ \| \gamma'(t)\times\gamma''(t) \| }.
\end{equation}
En pratique, le produit vectoriel se calcule comme ceci :
\begin{equation}
    \gamma'(t)\times\gamma''(t)=\begin{vmatrix}
        e_x &   e_y &   e_z \\
        x'(t)   &   y'(t)   &   0   \\
        x''(t)  &   y''(t)  &   0
    \end{vmatrix}=(x'y''-x''y')e_z.
\end{equation}
Le centre du cercle osculateur va se trouver quelque part sur la normale. Le vecteur normal est donné par
\begin{equation}
    n(t)=J\frac{\gamma'(t) }{ \| \gamma'(t) \| }
\end{equation}
où $J$ est la rotation d'angle $\frac{ \pi }{2}$ :
\begin{equation}
    J\begin{pmatrix}
        x'(t)   \\ 
        y'(t)   
    \end{pmatrix}=
    \begin{pmatrix}
        0   &   1   \\ 
        -1  &   0   
    \end{pmatrix}\begin{pmatrix}
        x'(t)   \\ 
        y'(t)   
    \end{pmatrix}=\begin{pmatrix}
        y'(t)   \\ 
        -x'(t)  
    \end{pmatrix}.
\end{equation}
Cela nous laisse deux possibilités pour le centre du cercle osculateur : $\gamma(t)+R(t)n(t)$ ou bien $\gamma(t)-R(t)n(t)$. Il faut savoir de quel côté de la courbe est situé le centre du cercle osculateur. Il faut choisir le côté de la concavité, c'est à dire le côté de la dérivée seconde.

\newcommand{\CaptionFigQuelCote}{De quel coté de $\gamma'(t)$ se trouvent $n(t)$ et $-n(t)$ ?}
\input{auto/pictures_tex/Fig_QuelCote.pstricks}

La difficulté maintenant est de savoir qui de $n(t)$ ou $-n(t)$ est du côté de $\gamma''(t)$. Il faut savoir si $n(t)$ est du même côté de la droite tangente que $\gamma''(t)$ ou non. Par construction, si nous regardons la figure  \ref{LabelFigQuelCote}, le vecteur $n(t)$ sera toujours à gauche de $\gamma'(t)$. Le fait que $\gamma''(t)$ soit à gauche ou à droite de $\gamma'(t)$ est donné par le signe du produit vectoriel $\gamma'(t)\times \gamma''(t)$. Si ce produit vectoriel est positif, il faut choisir $-n(t)$ et s'il est négatif, il faut choisir $n'(t)$.

Le truc pour obtenir le signe de $x'y''-x''y'$ est de faire
\begin{equation}
    \frac{ (\gamma'\times\gamma'')\cdot e_z}{\| \gamma'\times\gamma'' \|}.
\end{equation}

Le centre de courbure sera donc situé à la position
\begin{equation}
    \Omega(t)=\gamma(t)-n(t)\frac{ \| \gamma(t) \|^3 }{ \| \gamma'(t)\times\gamma''(t) \|^2 } (\gamma'\times\gamma'')\cdot e_z
\end{equation}
Nous pouvons écrire cela plus explicitement en nous souvenant que $\gamma'\times\gamma''=(x'y''-x''y')e_z$, par conséquent $\frac{ (\gamma'\times\gamma'')\cdot e_z}{\| \gamma'\times\gamma'' \|^2}=\frac{1}{ x'y''-x''y' }$. Nous avons
\begin{subequations}
    \begin{align}
        \Omega_x(t)&=x(t)-y'(t)\frac{ x'^2+y'^2 }{ x'y''-x''y' }\\
        \Omega_y(t)&=y(t)+x'(t)\frac{ x'^2+y'^2 }{ x'y''-x''y' }.
    \end{align}
\end{subequations}

Quelques exemples de cercles osculateurs sont sur la figure \ref{LabelFigOsculateur}.
\newcommand{\CaptionFigOsculateur}{Exemple de cercles osculateurs.}
\input{auto/pictures_tex/Fig_Osculateur.pstricks}

% TODO : Écrire quelque chose sur les points de rebroussement et d'inflexion, ainsi que sur les asymptotes.
%   Quand ce sera fait, il y a des choses à décommenter dans l'exerice exoCourbesSurfaces0002.tex

%+++++++++++++++++++++++++++++++++++++++++++++++++++++++++++++++++++++++++++++++++++++++++++++++++++++++++++++++++++++++++++ 
\section{Courbes planes}
%+++++++++++++++++++++++++++++++++++++++++++++++++++++++++++++++++++++++++++++++++++++++++++++++++++++++++++++++++++++++++++

\begin{definition}
    Une courbe \( \gamma\colon \mathopen[ a , b \mathclose]\to \eR^n\) est \defe{fermée}{courbe!fermée} si \( \gamma(a)=\gamma(b)\). Elle est \defe{simple}{courbe!simple} si \( \gamma(t)\neq \gamma(t')\) dès que \( t,t'\in\mathopen] a , b \mathclose[\) et \( t\neq  t'\).
\end{definition}

\begin{definition}      \label{DEFooSAZTooZGQrQG}
    Nous disons qu'une courbe fermée est continue, de classe \( C^1\), de classe \( C^2\) ou autre condition de régularité si son extension périodique comme application \( \gamma\colon \eR\to \eR^2\) a cette régularité.
\end{definition}

%--------------------------------------------------------------------------------------------------------------------------- 
\subsection{Angle}
%---------------------------------------------------------------------------------------------------------------------------

\begin{lemma}[\cite{ooIEJXooIYpBbd}]        \label{LEMooUECMooNBDGiR}
    Soient des courbes régulières \( \gamma\) et \( \sigma\) de classe \( C^2\) de l'intervalle ouvert \( I\) vers \( \eR^2\). Soit \( \theta_0\in \eR\) tel que
    \begin{subequations}
        \begin{align}
            \frac{ \gamma'(t_0)\cdot\sigma'(t_0) }{ \| \gamma'(t_0) \|\| \sigma'(t_0) \| }=\cos(\theta_0)\\
            \frac{ \gamma'(t_0)\cdot J\sigma'(t_0) }{ \| \gamma'(t_0) \|\| \sigma'(t_0) \| }=\cos(\theta_0).
        \end{align}
    \end{subequations}
    Alors il existe une unique fonction différentiable \( \theta\colon I\to \eR\)
    \begin{subequations}
        \begin{align}
            \frac{ \gamma'(t)\cdot\sigma'(t) }{ \| \gamma'(t) \|\| \sigma'(t) \| }=\cos\big( \theta(t) \big)\\
            \frac{ \gamma'(t)\cdot J\sigma'(t) }{ \| \gamma'(t) \|\| \sigma'(t) \| }=\sin\big( \theta(t) \big).
        \end{align}
    \end{subequations}
\end{lemma}

\begin{proof}
    Il suffit de prendre
    \begin{equation}
        f(t)=\frac{ \gamma'(t)\cdot\sigma'(t) }{ \| \gamma'(t) \|\| \sigma'(t) \| }
    \end{equation}
    et
    \begin{equation}
        g(t)=\frac{ \gamma'(t)\cdot J\sigma'(t) }{ \| \gamma'(t) \|\| \sigma'(t) \| }
    \end{equation}
    dans la proposition \ref{PROPooWZFGooMVLtFz}. Ces courbes sont de classe \( C^1\) parce que \( \gamma\) et \( \sigma\) sont de classe \( C^2\).
\end{proof}

%--------------------------------------------------------------------------------------------------------------------------- 
\subsection{Courbure signée}
%---------------------------------------------------------------------------------------------------------------------------

Nous avons déjà défini la courbure d'une courbe en la définition \ref{DefCourbureNormleUnit}. Nous introduisons maintenant la courbure signée qui est propre à la dimension deux.

\begin{definition}      \label{DEFooTSJXooTIyRXf}
    La \defe{structure complexe}{structure!complexe} sur \( \eR^2\) est l'application
    \begin{equation}
        \begin{aligned}
            J\colon \eR^2&\to \eR^2 \\
            (x,y)&\mapsto (-y,x). 
        \end{aligned}
    \end{equation}
\end{definition}
\ifbool{isEverything}{Cette définition est un cas très particulier des structures complexes sur les variétés, voir la définition \ref{DefSymHermMGKalg}.}{}

\begin{definition}      \label{DEFooJFWEooXcIVUs}
    La \defe{courbure signée}{courbure!signée} de la courbe \( \gamma\colon I\to \eR^2\) (\( I\) est un intervalle dans \( \eR\)) est la fonction
    \begin{equation}        \label{EQooWOUQooXrVzGx}
        \kappa(t)=\frac{ \gamma''(t)\cdot J\gamma'(t) }{ \| \gamma'(t) \|^3 }
    \end{equation}
    où \( J\) est la structure complexe de la définition \ref{DEFooTSJXooTIyRXf}.
\end{definition}
Cette définition est motivée par le fait qu'en identifiant \( \eR^2\) à \( \eC\), l'application \( J\) revient à l'application \( z\mapsto iz\).

Si \( v,w\in \eR^2\) nous avons formellement
\begin{equation}
    v\times w=-(v\cdot Jw)e_3.
\end{equation}
En particulier pour tout \( v\in \eR^2\) nous avons
\begin{equation}
    v\cdot Jv=0.
\end{equation}

\begin{lemma}
    Soir une courbe régulière \( \gamma\colon \mathopen[ a , b \mathclose]\to \eR^2\) et un difféomorphisme \( h\colon \mathopen[ c , d \mathclose]\to \mathopen[ a , b \mathclose]\). Si nous posons \( \sigma=\gamma\circ h\) alors
    \begin{equation}        \label{EQooSQNMooUKGhPd}
        \kappa_{\sigma}(u)=\signe\big( h'(u) \big)\kappa_{\gamma}\big( h(u) \big).
    \end{equation}
\end{lemma}

\begin{proof}
    Nous utilisons la définition \eqref{EQooWOUQooXrVzGx} de la courbure signée. La règle de dérivation en chaîne donne :
    \begin{subequations}
        \begin{align}
            \sigma'(u)&=\gamma'\big( h(u) \big)h'(u)\\
            \sigma''(u)&=\gamma''\big( h(u) \big)h'(u)^2+\gamma'\big( h(u) \big)h''(u).
        \end{align}
    \end{subequations}
    La numérateur de \( \kappa_{\sigma}(u)\) est :
    \begin{equation}
        (\gamma''\circ h)h'^2\cdot J(\gamma'\circ h)h'+h''(\gamma'\circ h)\cdot J(\gamma'\cdot h)h'
    \end{equation}
    dont le second terme est nul parce que \( v\cdot Jv=0\). Il nous reste donc
    \begin{equation}
        \kappa_{\sigma}(u)=\frac{ (h')^3 }{ | h' |^3} \frac{ (\gamma''\circ h)\cdot J(\gamma'\circ h) }{ \| \gamma'\circ h \|^3 }=\signe(h')\kappa_{\gamma}\big( h(u) \big).
    \end{equation}
\end{proof}

\begin{lemma}[\cite{ooIEJXooIYpBbd}]        \label{LEMooKPORooEGJCRm}
    Si \( \gamma_N\) est un arc paramétré normal, alors 
    \begin{equation}
        \gamma_N''(s)=\kappa(s)J\gamma_N'(s).
    \end{equation}
\end{lemma}

\begin{proof}
    Vu que la paramétrisation est normale, \( \gamma'_N\cdot \gamma'_N=1\), et en dérivant, \( \gamma_N''\cdot\gamma_N'=0\). Donc \( \gamma''_N\) est un multiple de \( J\gamma'\). En tenant compte du fait que la paramétrisation est normale, la courbure est
    \begin{equation}
        \kappa(s)=\gamma_N''(s)\cdot J\gamma'(s).
    \end{equation}
    En y injectant \( \gamma''(s)=\lambda(s)J\gamma'(s)\) nous trouvons
    \begin{equation}
        \kappa(s)=\lambda(s)J\gamma'(s)\cdot J\gamma'(s)=\lambda(s).
    \end{equation}
    Donc le facteur de proportionnalité est \( \kappa(s)\).
\end{proof}

\begin{theorem}     \label{THOooDLDVooFQnLWn}
    Soit une courbe \( \gamma\colon I\to \eR^2\) de classe \( C^2\).
    \begin{enumerate}
        \item
            \( \gamma\) est une partie de droite si et seulement si \( \kappa(t)=0\) pour tout \( t\).
        \item
            \( \gamma\) est une partie d'un cercle de rayon \( r>0\) si et seulement si \( | \kappa(s) |=\frac{1}{ r }\).
    \end{enumerate}
\end{theorem}

\begin{proof}
    Si \( \gamma\) est une droite, la dérivée seconde est nulle et la courbure est nulle. Supposons pour la réciproque que \( \kappa(t)=0\) pour tout \( t\). Nous utilisant une paramétrisation normale de \( \gamma\), ce qui ne change pas que la courbure reste nulle. Nous avons par le lemme \ref{LEMooKPORooEGJCRm} que \( \gamma''(s)=0\) et donc l'existence de \( a,b\in \eR^2\) tels que \( \gamma(t)=at+b\).

    Passons au cas du cercle. Si \( \gamma\) est un cercle, la paramétrisation normale est
    \begin{subequations}
        \begin{align}
            \gamma(t)&=R\begin{pmatrix}
                \cos(\frac{ t }{ R })    \\ 
                \sin(\frac{ t }{ R })    
            \end{pmatrix}\\
            \gamma'(t)&=\begin{pmatrix}
                -\sin(\frac{ t }{ R })    \\ 
                \cos(\frac{ t }{ R })    
            \end{pmatrix}\\
            \gamma''(t)=-\frac{1}{ R }\begin{pmatrix}
                \cos(\frac{ t }{ R })    \\ 
                \sin(\frac{ t }{ R })    
            \end{pmatrix}.
        \end{align}
    \end{subequations}
    Avec tout cela nous avons \( \kappa(s)=\gamma''(s)\cdot J\gamma'(s)=\frac{1}{ R }\).

    Nous supposons enfin que \( \kappa(t)=1/R\) et que le paramétrage soit normal (encore une fois, un reparamétrage ne change pas la courbure lorsqu'elle est constante). Nous définissons la courbe
    \begin{equation}
        \begin{aligned}
            \beta\colon \mathopen[ b , c \mathclose]&\to \eR^2 \\
            t&\mapsto \gamma(t)+rJ\gamma'(t). 
        \end{aligned}
    \end{equation}
    Nous avons \( \beta'(t)=\gamma'(t)+rJ\gamma''(t)\). Mais par le lemme \ref{LEMooKPORooEGJCRm} nous avons \( \gamma''=kJ\gamma'=\frac{1}{ r }J\gamma'\). Donc
    \begin{equation}
        \beta'(t)=\gamma'(t)-\gamma'(t)=0.
    \end{equation}
    Du coup \( \beta\) est constante : \( \beta(t)=a\). Alors \( a=\gamma(t)+rJ\gamma'(t)\) et en particulier
    \begin{equation}
        \| \gamma(t)-a \|=\| rJ\gamma'(t) \|=r.
    \end{equation}
    Donc effectivement \( \gamma\) reste sur un cercle de rayon \( r\) et de centre \( a\).
\end{proof}

\begin{definition}[\cite{ooIEJXooIYpBbd}]
    La \defe{courbure totale}{courbure!totale} de la courbe \( \gamma\colon \mathopen[ a , b \mathclose] \to \eR^2 \) est le nombre
    \begin{equation}        \label{EQooTIFWooQflOfd}
        K=\int_a^b\kappa(t)\| \gamma'(t) \|dt.
    \end{equation}
\end{definition}

\begin{lemma}
    La courbure signée ne change pas sous reparamétrisation positive, et change de signe sous reparamétrisation négative.
\end{lemma}

\begin{proof}
    Soit la courbe \( \gamma\colon \mathopen[ a , b \mathclose]\to \eR^2\) et un difféomorphisme \( h\colon \mathopen[ a , b \mathclose]\to \mathopen[ c , d \mathclose]\). Il s'agit d'intégrer la relation \eqref{EQooSQNMooUKGhPd} en effectuant le changement de variables \( t=h(u)\) :
    \begin{subequations}
        \begin{align}
            K_{\gamma}&=\int_a^b\kappa_{\gamma}(t)\| \gamma'(t) \|dt\\
            &=\int_c^d\underbrace{\kappa_{\gamma}\big( h(u) \big)}_{\kappa_{\sigma}(u)}\| \gamma'\big( h(u) \big) \|h'(u)du.
        \end{align}
    \end{subequations}
    En utilisant le fait que \( \sigma'(u)=\gamma'\big( h(u) \big)h'(u)  \) nous avons alors
    \begin{subequations}
        \begin{align}
            K_{\gamma}&=\int_{c}^d\kappa_{\sigma}(u)\left\| \frac{ \sigma'(u) }{ h'(u) }   \right\|h'(u)du\\
            &=\int_c^d\kappa_{\sigma}(u)\| \sigma'(u) \|\frac{ h'(u) }{ | h'(u) | }du\\
            &=\signe(h')K_{\sigma}.
        \end{align}
    \end{subequations}
\end{proof}

\begin{lemmaDef}[\cite{ooIEJXooIYpBbd}]     \label{LEMDEFooLPWJooAnWZjb}
    Soit une courbe régulière \( \gamma\colon I\to \eR^2\) de classe \( C^2\) et \( t_0\) dans l'intérieur de \( I\). Soit \( \theta_0 \in \eR\) tel que
    \begin{equation}
        \frac{ \gamma'(t_0)  }{ \| \gamma'(t_0) \| }=\big( \cos(\theta_0),\sin(\theta_0) \big).
    \end{equation}
    Alors il existe une unique fonction différentiable \( \theta\colon I\to \eR\) telle que \( \theta(t_0)=\theta_0\) et
    \begin{equation}
        \frac{ \gamma'(t) }{ \| \gamma'(t) \| }=\begin{pmatrix}
            \cos\big( \theta(t) \big)    \\ 
            \sin\big( \theta(t) \big)    
        \end{pmatrix}
    \end{equation}
    pour tout \( t\in I\).

    Cette fonction est l'\defe{angle}{angle!d'une courbe} de \( \gamma\) déterminé par \( \theta_0\).
\end{lemmaDef}

\begin{proof}
    Soit \(  \beta(t)=(t,0)  \); alors \( \beta'(t)=(1,0)\) et nous avons
    \begin{subequations}
        \begin{align}
            \gamma'\cdot \beta'=\gamma_x'\\
            \gamma'\cdot J\beta'=\gamma_y'.
        \end{align}
    \end{subequations}
    Par la proposition \ref{LEMooUECMooNBDGiR} il existe une unique fonction \( \theta\colon I\to \eR\) telle que \( \theta(t_0)=\theta_0\) et
    \begin{subequations}
        \begin{numcases}{}
            \cos\big( \theta(t) \big)=\frac{ \gamma'\cdot \beta' }{ \| \gamma' \|\| \beta' \| }=\frac{ \gamma_x' }{ \| \gamma' \| }\\
            \sin\big( \theta(t) \big)=\frac{ \gamma'\cdot J\beta' }{ \| \gamma' \|\| \beta' \| }=\frac{ \gamma_y' }{ \| \gamma' \| }.
        \end{numcases}
    \end{subequations}
    Une telle fonction est bien celle que l'on demande ici.
\end{proof}

\begin{lemma}       \label{LEMooWLAUooKetUiW}
    Si \( \gamma\) est une courbe régulière de classe \( C^2\), alors sa courbure et son angle vérifient la relation
    \begin{equation}
        \theta'(t)=\| \gamma'(t) \|\kappa(t).
    \end{equation}
\end{lemma}

\begin{proof}
    Par définition de l'angle (lemme \ref{LEMDEFooLPWJooAnWZjb}) nous avons
    \begin{equation}
        \frac{ \gamma'(t) }{ \| \gamma'(t) \| }=\big( \cos\theta(t),\sin\theta(t) \big).
    \end{equation}
    Dérivant cela,
    \begin{subequations}
        \begin{align}
            \frac{ \gamma''(t) }{ \| \gamma'(t) \| }+\gamma'(t)\frac{ d  }{ dt }\left( \frac{1}{ \| \gamma'(t) \| } \right)&=\theta'(t)
            \begin{pmatrix}
                -\sin\big( \theta(t) \big)\\
                \cos\big( \theta(t) \big)
            \end{pmatrix}\\
            &=\theta'(t)J\begin{pmatrix}
                \cos\big( \theta(t) \big)   \\ 
                \sin\big( \theta(t) \big)    
            \end{pmatrix}\\
            &=\theta'(t)\frac{ J\gamma'(t) }{ \| \gamma'(t) \| }.
        \end{align}
    \end{subequations}
    Nous prenons le produit scalaire de cette égalité avec \( J\gamma'\) en tenant compte du fait que \( \gamma'\cdot J\gamma'=0\) :
    \begin{equation}
        \frac{ \gamma''\cdot J\gamma' }{ \| \gamma' \| }=\frac{ \theta' }{ \| \gamma' \| }J\gamma'\cdot J\gamma'.
    \end{equation}
    En remarquant que \( Jv\cdot Jv=\| v \|^2\) nous trouvons \( \theta'\| \gamma' \|=\| \gamma' \|^2\kappa_{\gamma}\) et donc
    \begin{equation}
        \theta'(t)=\| \gamma'(t) \|\kappa_{\gamma}(t),
    \end{equation}
    ce qu'il fallait prouver.
\end{proof}
Ce lemme nous fournit la formule attendue pour la courbure totale.

\begin{lemma}[\cite{ooIEJXooIYpBbd}]
    Soit une courbe régulière de classe \( C^2\) \( \gamma\colon \mathopen[ a , b \mathclose]\to \eR^2\). Sa courbure totale est donnée par
    \begin{equation}
        K=\theta(b)-\theta(a).
    \end{equation}
\end{lemma}

\begin{proof}
    Il suffit de remplacer dans la définition \eqref{EQooTIFWooQflOfd} de la courbure totale l'intégrante par son expression 
    du lemme \ref{LEMooWLAUooKetUiW} : 
    \begin{equation}
        K=\int_a^b\kappa(t)\| \gamma'(t) \|dt=\int_a^b\theta'(t)dt=\theta(b)-\theta(a)
    \end{equation}
    par le théorème \ref{ThoRWXooTqHGbC}.
\end{proof}

%--------------------------------------------------------------------------------------------------------------------------- 
\subsection{Degré, indice et homotopie}
%---------------------------------------------------------------------------------------------------------------------------

\begin{definition}[\cite{ooIEJXooIYpBbd}]
    Le \defe{nombre de tours}{nombre!tours d'une courbe plane} d'une courbe fermée de classe \( C^1\) \( \gamma\colon \mathopen[ a , b \mathclose]\to \eR^2\) est le nombre
    \begin{equation}
        \Turn(\gamma)=\frac{1}{ 2\pi }\int_a^b\kappa(t)\| \gamma'(t) \|dt
    \end{equation}
    où \( \kappa\) est la courbure signée définie en \ref{DEFooJFWEooXcIVUs}.
\end{definition}

\begin{lemmaDef}        \label{DEFooTKBUooNVcheO}
    Soit une application continue \( \phi\colon S^1\to S^1\). Un \defe{relèvement}{relèvement} de \( \sigma\) est une application \( \varphi\colon \eR\to \eR\) une application continue telle que
    \begin{equation}
        \phi\big( \cos(t),\sin(t) \big)=\begin{pmatrix}
            \cos\big( \varphi(t) \big)    \\ 
            \sin\big( \varphi(t) \big)    
        \end{pmatrix}.
    \end{equation}
    Le \defe{degré}{degré!application \( S^1\to S^1\)} est l'entier \( \deg(\phi)\) tel que
    \begin{equation}
        \varphi(2\pi)-\varphi(0)=2\deg(\phi)\pi.
    \end{equation}
    Ce nombre ne dépend pas du choix du relèvement \( \varphi\).
\end{lemmaDef}

\begin{proof}
    Soient \( \varphi_1\) et \( \varphi_2\), deux application qui satisfont les contraintes. Nous avons une fonction continue \( n\colon S^1\to \eR\) telle que
    \begin{equation}
        \varphi_1(t)-\varphi_2(t)=2\pi n(t).
    \end{equation}
    La fonction \( n\) ne pouvant prendre que des valeurs entières et étant continue, elle est constante. Par conséquent
    \begin{equation}
        \varphi_1(2\pi)-\varphi_1(0)=\varphi_2(2\pi)-\varphi_2(0),
    \end{equation}
    ce que nous voulions.
\end{proof}

\begin{definition}      \label{DEFooOCUQooUAlbLo}
    Soit une courbe fermée \( \gamma\colon \mathopen[ 0 , L \mathclose]\to \eR^2\) de classe \( C^1\). Nous posons
    \begin{equation}
        \Phi_{\gamma}(t)=\frac{ \tilde \gamma'(t) }{ \| \tilde \gamma'(t) \| }
    \end{equation}
    où \( \tilde \gamma(u)=\gamma\big( \frac{ L }{ 2\pi }u \big)\). L'\defe{indice de rotation}{indice!de rotation} de \( \gamma\) est le degré de \( \Phi_{\gamma}\), c'est à dire
    \begin{equation}
        \IR(\gamma)=\deg(\Phi_{\gamma}).
    \end{equation}
\end{definition}
Notons que dans cette définition, \( \Phi_{\gamma}\) n'est rien d'autre que le vecteur unitaire tangent à \( \gamma\), ramené à \( \mathopen[ 0 , 2\pi \mathclose]\).

\begin{proposition}     \label{PROPooXHSDooDDnlJQ}
    Pour une courbe fermée de classe\footnote{La dérivée seconde arrive dans la définition de la courbure; il faudrait donc supposer au moins \( C^2\) pour avoir la continuité de la courbure.} \( C^{2}\), l'indice de rotation est égal au nombre de tours.
\end{proposition}

\begin{proof}
    Soit \( \gamma\colon \mathopen[ 0 , L \mathclose]\to \eR^2\) une courbe vérifiant les hypothèses. Par définition, \( \IR(\gamma)=\deg(\Phi_{\gamma})\) où
    \begin{equation}        \label{EQooGTAPooDRCFPG}
        \Phi_{\gamma}(t)=\frac{ \tilde \gamma'(t) }{ \| \tilde \gamma'(t) \| }=\begin{pmatrix}
            \cos \tilde \theta(t)    \\ 
            \sin \tilde \theta(t)    
        \end{pmatrix}
    \end{equation}
    où nous avons noté \( \tilde \theta\) l'angle tournant\footnote{Définition \ref{LEMDEFooLPWJooAnWZjb}.} de \( \tilde \gamma\) et \( \theta\) celui de \( \gamma\). Vu que \( \tilde \theta\) vérifie les hypothèses de la définition \ref{DEFooTKBUooNVcheO} nous pouvons calculer le degré de \( \Phi_{\gamma}\) par
    \begin{equation}        \label{EQooKNCJooTwTBMO}
        \deg(\Phi_{\gamma})=\frac{1}{ 2\pi }\big( \tilde \theta(2\pi)-\tilde \theta(0) \big)=\frac{1}{ 2\pi }\int_0^{2\pi}\tilde \theta'(s)ds.
    \end{equation}
    Il faut trouver le lien entre \( \theta\) et \( \tilde \theta\). Pour cela nous notons que
    \begin{equation}
        \frac{ \tilde \gamma'(t) }{ \| \tilde \gamma'(t) \| }=\frac{ \gamma'\left( \frac{ L }{ 2\pi }t \right) }{ \| \gamma'\left( \frac{ L }{ 2\pi } \right) \| }.
    \end{equation}
    En comparant avec \eqref{EQooGTAPooDRCFPG} il vient
    \begin{equation}
        \tilde \theta(t)=\theta\big( \frac{ L }{ 2\pi }t \big)
    \end{equation}
    et
    \begin{equation}
        \tilde \theta'(s)=\frac{ L }{ 2\pi }\theta'\left( \frac{ L }{ 2\pi }s \right).
    \end{equation}
    Le changement de variables \( t=\frac{ L }{ 2\pi }s\) est donc tout vu dans l'intégrale \eqref{EQooKNCJooTwTBMO} :
    \begin{equation}
        \deg(\Phi_{\gamma})=\frac{1}{ 2\pi }\int_0^{2\pi}\frac{ L }{ 2\pi }\theta'\big( \frac{ L }{ 2\pi }s \big)ds=\frac{1}{ 2\pi }\int_0^L\theta'(t)dt=\frac{1}{ 2\pi }\int_0^L\kappa_{\gamma}(t)\| \gamma'(t) \|dt=\Turn(\gamma)
    \end{equation}
    où nous avons aussi utilisé le lemme \ref{LEMooWLAUooKetUiW} qui donne le lien entre \( \theta'\) et \( \kappa\).
\end{proof}

\begin{definition}[homotopie de chemins fermés]     \label{DEFooHJQTooYUFcee}
    Les courbes fermées \( \gamma_0,\gamma_1\colon \mathopen[ 0 , L \mathclose]\to Y\) (\( Y\) est un espace topologique) sont \defe{homotopes}{homotopie} s'il existe une application continue
    \begin{equation}
        F\colon \mathopen[ 0 , 1 \mathclose]\times \mathopen[ 0 , L \mathclose]\to Y
    \end{equation}
    telle que
    \begin{enumerate}
        \item
           \( F(0,t)=\gamma_0(t)\) pour tout \( t\),
       \item
           \( F(1,t)=\gamma_1(t)\) pour tout \( t\),
       \item
           \( F(u,L)=F(u,0)\) pour tout \( u\).
    \end{enumerate}
    L'application \( F\) est l'homotopie entre \( \gamma_0\) et \( \gamma_1\).
\end{definition}

\begin{proposition}[Homotopie, degré et indice\cite{ooIEJXooIYpBbd}]      \label{PROPooZIAKooHqtnZj}
    Il y a deux résultats à ne pas confondre.
    \begin{enumerate}
        \item   \label{ITEMooLEHFooXEyTHY}
            Si \( \phi_i\colon S^1\to S^1\) sont homotopes, alors \( \deg(\phi_1)=\deg(\phi_2)\).
        \item
            Si \(\gamma_i\colon \mathopen[ 0 , L \mathclose]\to \eR^2 \) sont homotopes, alors \( \IR(\phi_1)=\IR(\phi_2)\).
    \end{enumerate}
\end{proposition}

\begin{proof}
    Nous allons décomposer la preuve en deux parties.
    \begin{subproof}
    \item[Le degré pour les applications \( S^1\to S^1\)]
        Nous supposons avoir deux applications homotopes \( \gamma_i\colon S^1\to S^1\). Si \( F\) est l'homotopie\footnote{Définition \ref{DEFooHJQTooYUFcee}.} entre \( \gamma_0\) et \( \gamma_1\), nous posons \( \gamma_u(t)=F(u,t)\) qui est encore une courbe fermée \( \gamma_u\colon S^1\to S^2\). Nous pouvons donc considérer le degré de \( \gamma_u\). C'est un entier \( n_u\) qui vérifie
        \begin{equation}
            \tilde \gamma_u(2\pi)-\tilde \gamma_u(0)=2\pi n_u.
        \end{equation}
        Le membre de gauche est une fonction continue de \( u\); dans le membre de droite \( n_u\) ne pouvant prendre que des valeurs entières, elle est alors constante.
    \item[Indice de rotation pour des courbes dans $\eR^2$]
        Soient maintenant \( \gamma_0\) et \( \gamma_1\) deux courbes homotopes de \( \mathopen[ 0 , L \mathclose]\) dans \( \eR^2\). Par définition, \( \IR(\gamma_i)=\deg(\Phi_{\gamma_i})\). Prouvons alors que \( \Phi_{\gamma_0}\) et homotope à \( \Phi_{\gamma_1}\). De cette façon, la première partie de la preuve conclura à
        \begin{equation}
            \IR(\gamma_0)=\deg(\Phi_{\gamma_0})=\deg(\Phi_{\gamma_1})=\IR(\gamma_1).
        \end{equation}
        Nous savons que 
        \begin{equation}
            \Phi_{\gamma}=\frac{ \gamma'\left( \frac{ L }{ 2\pi }t \right) }{ \| \gamma'\left( \frac{ L }{ 2\pi }t \right) \| },
        \end{equation}
        donc en posant
        \begin{equation}
            \Phi_F(u,t)=\frac{ \gamma_u'\big( \frac{ L }{ 2\pi }t \big) }{ \| \gamma'_u\big( \frac{ L }{ 2\pi }t \big) \| }
        \end{equation}
        nous avons une homotopie entre \( \Phi_{\gamma_0}\) et \( \Phi_{\gamma_1}\).
    \end{subproof}
\end{proof}

\begin{theorem}[\cite{ooIEJXooIYpBbd}]      \label{THOooEQWOooBCRMMZ}
    Le nombre de tours d'une courbe simple fermée de classe \(  C^{2}\) dans \( \eR^2\) est \( \pm 1\).
\end{theorem}

\begin{proof}
    Soit une telle courbe \( \gamma\) et son image \( \Gamma\). 
    \begin{subproof}
        \item[Choix d'un point et d'une tangente]
            
            Si \( \ell\) est une droite dans le plan, soit \( p\) le point de \( \Gamma\) le plus proche de \( \ell\). Il n'est peut-être pas unique, mais la parallèle à \( \ell\) passant par \( p\) est une tangente à \( \Gamma\) telle que tout \( \Gamma\) se trouve d'un seul côté de \( \ell_p\). Pour voir cela, il suffit de choisir un système d'axes pour lequel \( \ell\) est l'axe \( y=0\). La distance entre \( \ell\) et les points de \( \Gamma\) est donnée par \( \gamma_y\), et donc les extrema sont atteints là où \( \gamma_y'=0\), c'est à dire pour les points sur lesquels la tangente est parallèle à \( \ell\).

            Étant donné qu'il y a (au moins) un maximum et un minimum distincts, pour pouvons choisir le point \( p\) de telle sorte que pour tout point \( q\in \Gamma\), l'angle du vecteur \( q-p\) soit entre \( \alpha_0\) et \( \alpha_0+\pi\) et non entre \( \alpha_0-pi\) et \( \alpha_0\). Ce choix revient à choisir \( p\) de telle sorte que \( \Gamma\) soit d'un côté ou de l'autre de \( \ell_p\).

            Pour simplifier les notations plus tard nous choisissons \( \ell_p\) horizontale, de telle sorte que \( \alpha_0=0\), et que \( \Gamma\) est au dessus de \( \ell_p\). Les angles des vecteurs \( q-p\) pour \( q\in \Gamma\) sont donc tous entre \( 0\) et \( \pi\).

        \item[Définition de \( \Sigma\)]

            Soit \( L\) la longueur\footnote{Définition \ref{DEFooDNZWooXmxhsU}.} de \( \gamma\) et \( \beta\), une paramétrisation de \( \gamma\) telle que \( \beta(0)=p\). Nous considérons le triangle
            \begin{equation}
                \mT=\{ (t_1,t_2)\in \eR^2\tq 0\leq t_1\leq t_2\leq L \}
            \end{equation}
            et l'application sécante \( \Sigma\colon \mT\to S^1\) définie par
            \begin{equation}
                \Sigma(t_1,t_2)=\begin{cases}
                    \dfrac{ \beta'(t) }{ \| \beta'(t) \| }    &   \text{si } t_1=t_2=t\\
                    -\dfrac{ \beta'(0) }{ \| \beta'(0) \| }    &   \text{si } t_1=0\text{ et } t_2=L\\
                    \dfrac{ \beta(t_2)-\beta(t_1) }{ \| \beta(t_2)-\beta(t_1) \| }    &   \text{sinon}.
                \end{cases}
            \end{equation}
    \item[Continuité de \( \Sigma\)]
        Nous devons prouver les limites\footnote{Limite au sens de la définition \ref{DefYNVoWBx}, en sachant qu'elle est unique par la proposition \ref{PropFObayrf}.} suivantes :
        \begin{subequations}
            \begin{align}
                \lim_{\substack{(t_1,t_2)\to (t,t)\\0\leq t_1\leq t\leq t_2\leq L}}\frac{ \beta(t_1)-\beta(t_2) }{ \| \beta(t_1)-\beta(t_2) \| }=\frac{ \beta'(t) }{ \| \beta'(t) \| }      \label{SUBEQooDXJDooOIxcBD}  \\
                \lim_{\substack{(t_1,t_2)\to (0,L)\\0\leq t_1\leq t\leq t_2\leq L}}\frac{ \beta(t_1)-\beta(t_2) }{ \| \beta(t_1)-\beta(t_2) \| }=-\frac{ \beta'(0) }{ \| \beta'(0) \| }        \label{SUBEQooOXGSooXHEHHh}
            \end{align}
        \end{subequations}
        Nous commençons par \eqref{SUBEQooDXJDooOIxcBD} en multipliant et divisant par \( t_2-t_1\) :
        \begin{equation}
            \frac{ \beta(t_2)-\beta(t_1) }{ \|  \beta(t_2)-\beta(t_1)  \| }=\frac{  \beta(t_2)-\beta(t_1)  }{ t_2-t_1 }\frac{ t_2-t_1 }{ \|  \beta(t_2)-\beta(t_1)  \| }.
        \end{equation}
        Si chacune des limites des deux facteurs existent dans \( \eR\), la limite du produit sera le produit des limites. Pour le premier facteur nous développons \( \beta(t_2)\) autour de \( t=t_1\) via la formulation \eqref{SUBEQooPYABooKpDgdu} :
        \begin{equation}
            \beta(t_2)=\beta(t_1)+(t_2-t_1)\beta'(t_1)+\alpha(t_2-t_1)
        \end{equation}
        où \( \alpha\) est une fonction ayant la propriété \( \lim_{t\to 0} \frac{ \alpha(t) }{ t }=0\). Nous avons à calculer la limite de
        \begin{equation}
            \frac{ (t_1-t_1)\beta'(t_1)+\alpha(t_2-t_1) }{ t_2-t_1 }=\beta'(t_1)+\frac{ \alpha(t_2-t_1) }{ t_2-t_1 }.
        \end{equation}
        Prendre la limite \( (t_1,t_2)\to (t,t)\) donne bien \( \beta'(t)\) parce que \( \beta\) est de classe \( C^1\). En ce qui concerne la limite de la deuxième partie, nous allons la faire plus en détail pour la limite \eqref{SUBEQooOXGSooXHEHHh}.

        La limite \eqref{SUBEQooOXGSooXHEHHh}. Nous prenons la prolongation périodique de \( \beta\). Alors si \( t_2=L-\epsilon\) nous pouvons écrire \( \beta(-\epsilon)\) au lieu de \( \beta(t_2)\). Nous développons \( \beta(t_1)\) autour de \( t=-\epsilon\) (parce que \( t_1\) est petit) :
        \begin{equation}
            \beta(t_1)=\beta(-\epsilon)+(t_1+\epsilon)\beta'(-\epsilon)+\alpha(t_1+\epsilon).
        \end{equation}
        Après multiplication et division par \( t_1+\epsilon\), la première limite à calculer est celle de
        \begin{equation}
            \frac{ \beta(-\epsilon)-\beta(t_1) }{ t_1+\epsilon }=-\beta'(-\epsilon)+\frac{ \alpha(t_1+\epsilon) }{ t_1+\epsilon },
        \end{equation}
        pour \( (\epsilon,t_1)\to (0,0)\). Cela donne bien \( -\beta'(0)\). La seconde limite à calculer est celle de
        \begin{equation}
            \frac{ t_1+\epsilon }{ \| \beta(-\epsilon)-\beta(t_1) \| }=\frac{ t_1+\epsilon }{ \| -(t_1+\epsilon)\beta'(-\epsilon)-\alpha(t_1+\epsilon) \| }=\frac{ t_1+\epsilon }{ \| (t_1+\epsilon)\beta'(-\epsilon)+\alpha(t_1+\epsilon) \| }.
        \end{equation}
        Nous calculons la limite de l'inverse (qui, si elle est non nulle donnera la réponse en inversant à nouveau) en nous souvenant de la formule
        \begin{equation}
           \big|  a-| b | \big|  \leq | a+b |\leq \big|    a+| b |\big|.
        \end{equation}
        Nous avons l'encadrement
        \begin{equation}
            \frac{  \big| (t_1+\epsilon)\beta'(-\epsilon)-| \alpha(t_1+\epsilon) | \big| }{ t_1+\epsilon }\leq \frac{ \|  (t_1+\epsilon)\beta'(-\epsilon)+ \alpha(t_1+\epsilon) \|  }{ t_1+\epsilon }\leq  \frac{ \big|   (t_1+\epsilon)\beta'(-\epsilon)+| \alpha(t_1+\epsilon) | \big| }{ t_1+\epsilon }
        \end{equation}
        Les limites des deux extrêmes existent et valent \( \beta'(0)\); la règle de l'étau \ref{ThoRegleEtau} conclu.

    \item[Deux chemins homotopes]
            Nous considérons dans \( \mT\) les points \( A=(0,0)\), \( B=(0,L)\) et \( C=(L,L)\). Nous allons considérer les chemins direct de \( A\) à \( C\) et celui passant via \( B\). Et comme ces chemins doivent être paramétrés de \( 0\) à \( 2 \pi\), il faut faire un peu attention. Nous définissons les chemins 
            \begin{equation}
                \sigma_i\colon S^1\to S^1
            \end{equation}
            par
            \begin{subequations}
                \begin{align}
                    \sigma_1(t)&=\Sigma\left( \frac{ L }{ 2\pi }t,\frac{ L }{ 2\pi }t \right)=\frac{ \beta'(tL/2\pi) }{ \| \beta'(tL/2\pi) \| }=\frac{ \tilde \beta'(t) }{ \|\tilde  \beta'(t) \| }=\Phi_{\beta}\\
                    \sigma_2(t)&=\begin{cases}
                        \Sigma(0,\frac{ L }{ \pi }t)    &   \text{si } 0\leq t\leq \pi\\
                        \Sigma\big( \frac{ L }{ \pi }(t-\pi),L  \big)    &    \text{si }\pi\leq t\leq 2\pi
                    \end{cases}
                \end{align}
            \end{subequations}
            où nous avons repris les notations de la définition \ref{DEFooOCUQooUAlbLo}. Notons que pour \( \sigma_2\), en \( t=\pi\) les deux expressions donnent
            \begin{equation}
                \sigma_2(\pi)=\Sigma(0,L)=-\frac{ \beta'(0) }{ \| \beta'(0) \| }.
            \end{equation}
            Ce sont des chemins fermés parce que
            \begin{equation}
                \sigma_1(2\pi)=\Sigma(L,L)=\frac{ \beta'(L) }{ \| \beta'(L) \| }=\frac{ \beta'(0) }{ \| \beta'(0) \| }=\Sigma(0,0)=\sigma_1(0).
            \end{equation}
            Notons que dans toutes ces définitions et calculs, nous avons utilisé de façon assez cruciale la définition \ref{DEFooSAZTooZGQrQG} pour définir la dérivée de \( \beta\) en \( t=0\).

            Les chemins \( \sigma_1\) et \( \sigma_2\) sont homotopes par construction.
        \item[Indices et degrés]
            La proposition \ref{PROPooZIAKooHqtnZj}\ref{ITEMooLEHFooXEyTHY} nous donne \( \deg(\sigma_1)=\deg(\sigma_2)\). Nous avons alors la chaîne d'égalités
            \begin{equation}        \label{EQooSXOAooZVOVxc}
                \deg(\sigma_2)=\deg(\sigma_1)=\deg(\Phi_{\beta})=\IR(\beta)=\IR(\gamma)=\Turn(\gamma)
            \end{equation}
            où les justifications sont :
            \begin{enumerate}
                \item
                    \( \deg(\sigma_2)=\deg(\sigma_1)\) par homotopie : proposition \ref{PROPooZIAKooHqtnZj}.
                \item
                    \( \deg(\sigma_1)=\deg(\Phi_{\beta})\) parce que \( \sigma_1=\Phi_{\beta}\).
                \item
                    \( \deg(\Phi_{\beta}=\IR(\beta)\) par définition \ref{DEFooOCUQooUAlbLo} de l'indice.
                \item
                    \(  \IR(\beta)=\IR(\gamma) \) par invariance de l'indice sous reparamétrisation.
                \item
                    \( \IR(\gamma)=\Turn(\gamma)\) par la proposition \ref{PROPooXHSDooDDnlJQ}.
            \end{enumerate}
            Il nous reste à montrer que \( \deg(\sigma_2)=\pm 1\). 
        \item[La géométrie de \( \sigma_2\)]
            Notons que par définition les valeurs de \( \sigma_2(t)\) sont les vecteurs (unitaires) joignant \( p\) aux points de \( \Gamma\) lorsque \( 0\leq t\leq \pi\), et les vecteurs inverses pour \( \pi\leq t\leq 2\pi\). Plus précisément nous avons, si \( 0<a<\pi\) :
            \begin{equation}
                \sigma_2(a)=\Sigma(0,\frac{ L }{ \pi }a)=\frac{ \beta(\frac{ L }{ \pi }a)-\beta(L) }{ \|    \beta(\frac{ L }{ \pi }a)-\beta(L)   \| }
            \end{equation}
            et
            \begin{equation}
                \sigma_2(\pi+a)=\Sigma(\frac{ L }{\pi  }a,L)=\frac{ \beta(L)-\beta(\frac{ L }{ \pi }a) }{ \|   \beta(L)-\beta(\frac{ L }{ \pi }a) \| }.
            \end{equation}
            En sachant que \( \beta(0)=\beta(L)\) nous avons alors
            \begin{equation}        \label{EQooKMSRooEzWkyL}
                \sigma_2(\pi+a)=-\sigma_2(a).
            \end{equation}

            Notons aussi que le fait que \( \gamma\) soit une courbe simple assure que le numérateur et le dénominateur de \( \sigma_2\) ne s'annulent pas autrement que pour \( t=L\) ou \( t=0\).
            
        \item[Un relèvement pour \( \sigma_2\)]
            Ces propriétés motivent cette idée pour le relèvement de \( \sigma_2\) :
            \begin{equation}
                \begin{aligned}
                    \theta\colon \mathopen[ 0 , 2\pi \mathclose[&\to\mathopen[ 0 , 2\pi \mathclose[  \\
                        t&\mapsto \text{l'angle du vecteur } \sigma_2(t),
                \end{aligned}
            \end{equation}
            avec \( \theta(2\pi)\) définit par continuité. Nous allons cependant voir, en étant plus prudent, que cette définition n'assure pas la continuité (surtout en \( t=\pi\)).

            Soyons donc plus prudent et construisons \( \theta\) petit à petit.

            Étant donné que \( \ell_p\) est tangente à \( \Gamma\) au point \(p \), la droite \( \ell_p\) est parallèle à \( \beta'(0)\), et l'angle entre \( \ell_p\) et \( \beta'(0)\) est soit \( 0\) soit \( \pi\). Donc \( \theta(0)\) devrait valoir soit \(0\) soit \(\pi\).

            Nous commençons par définir ceci :
            \begin{equation}
                \begin{aligned}
                    \theta\colon \mathopen[ 0 , \pi \mathclose[&\to \mathopen[ 0 , \pi \mathclose] \\
                    t&\mapsto \text{angle de } \sigma_2(t)=\arccos\big(  \sigma_2(t)_x   \big). 
                \end{aligned}
            \end{equation}
            C'est le choix d'avoir \( \ell_p\) horizontale et \( \Gamma\) au dessus de \( \ell_p\) qui nous assure que pour tout \( t\in\mathopen[ 0 , \pi \mathclose]\), l'angle de \( \sigma_2(t)\) peut être choisi entre \( 0\) et \( \pi\). De plus cette fonction est continue en tant que partie de la fonction \( \arccos\colon \mathopen[ -1 , 1 \mathclose]\to \mathopen[ 0 , \pi \mathclose]\) qui est elle-même continue par la proposition \ref{PropoInvCompCont}.

            Le fait que \( \theta\) soit continue est assuré par le fait que \( \sigma_2\) est \(  C^{\infty}\).

            \begin{subproof}
                \item[Si \( \theta(0)=0\)]

                    Cela correspond à la situation des vecteurs rouges sur la figure \ref{LabelFigERPMooZibfNOiU}.

                    Vu que \( \sigma_2(\pi)=-\sigma_2(0)\), l'angle de \( \sigma_2\) en \( t=\pi\) est le supplémentaire de celui en \( t=0\). Mais pour \( 0\leq t<\pi\), \( \theta(t)\) prend ses valeurs entre \( 0\) et \( \pi\), le seul supplémentaire de \( 0\) à être disponible est \( \theta(\pi)=\pi\) (et non \( \theta(\pi)=-\pi\) par exemple). Nous définissons donc \( \theta(\pi)=\pi\) pour la continuité.

                En ce qui concerne les \( \pi<t< 2\pi\) nous savons que \( \sigma_2(\pi+a)=-\sigma_2(a)\). Les angles sont donc les supplémentaires de ceux pour \( 0\leq t\leq \pi\). Pour assurer la continuité en \( t=pi\) nous sélectionnons la place \( \mathopen] \pi , 2\pi \mathclose]\); et nous définissons
                \begin{equation}
                    \begin{aligned}
                    \theta\colon \mathopen] \pi , 2\pi \mathclose[&\to \mathopen] \pi , 2\pi \mathclose] \\
                    t&\mapsto \text{angle de } \sigma_2(t). 
                    \end{aligned}
                \end{equation}
                Le nombre \( \theta(2\pi)\) est défini par continuité. Il doit valoir \( \theta(2\pi)=2\pi\).

                La fonction \( \theta\) ainsi définie est un relèvement pour \( \sigma_2\), et le degré peut être calculé :
                \begin{equation}
                    \deg(\sigma_2)=\frac{1}{ 2\pi }\big( \theta(2\pi)-\theta(0) \big)=1.
                \end{equation}

\newcommand{\CaptionFigERPMooZibfNOiU}{Les vecteurs représenant \( \sigma_2\) dans le cas où \( \beta'(0)\) est dans le sens de \( \ell_p\) ou dans le sens inverse. Pour le sport nous avons dessiné la situation avec une droite \( \ell\) quelconque plutôt que horizontale.}
\input{auto/pictures_tex/Fig_ERPMooZibfNOiU.pstricks}
                
                \item[Si \( \theta(0)=\pi\)]
                    Cela correspond à la situation des vecteurs bleus sur la figure \ref{LabelFigERPMooZibfNOiU}.

                Alors \( \theta(\pi)=0\) est obligatoire parce qu'il doit être supplémentaire à \( \theta(0)\). Les angles atteints par \( \sigma_2(t)\) pour \( t\in\mathopen] \pi , 2\pi \mathclose[ \) sont encore les complémentaires, mais cette fois la continuité en \( t=\pi\) nous impose de les chercher dans \( \mathopen] -\pi , 0 \mathclose]\) et nous définissons
                \begin{equation}
                    \begin{aligned}
                    \theta\colon \mathopen] \pi , 2\pi \mathclose[&\to \mathopen] -\pi , 0 \mathclose] \\
                    t&\mapsto \text{angle de } \sigma_2(t). 
                    \end{aligned}
                \end{equation}
                Par continuité nous devons avoir \( \theta(2\pi)=\theta(\pi)-\pi\) et donc \( \theta(2\pi)=-\pi\).

                Le degré de \( \sigma_2\) est alors
                \begin{equation}
                    \deg(\sigma_2)=\frac{1}{ 2\pi }\big( \theta(2\pi)-\theta(0) \big)=-1.
                \end{equation}
                
            \end{subproof}

        \item[Conclusion]

            Nous avons montré que le degré de \( \sigma_2\) est \( 1\) ou \( -1\), et en remontant les égalités \eqref{EQooSXOAooZVOVxc} nous déduisons que \( \Turn(\gamma)=\pm 1\).

    \end{subproof}
\end{proof}

%+++++++++++++++++++++++++++++++++++++++++++++++++++++++++++++++++++++++++++++++++++++++++++++++++++++++++++++++++++++++++++ 
\section{Courbes fermées planes}
%+++++++++++++++++++++++++++++++++++++++++++++++++++++++++++++++++++++++++++++++++++++++++++++++++++++++++++++++++++++++++++

%--------------------------------------------------------------------------------------------------------------------------- 
\subsection{Cercle circonscrit}
%---------------------------------------------------------------------------------------------------------------------------

La proposition suivante est dans le même esprit que l'ellipse de John-Loewer\footnote{Proposition \ref{PropJYVooRMaPok}.}.
\begin{propositionDef}[Cercle circonscrit\cite{ooCQFXooOLEkIr,MonCerveau}]      \label{PROPDEFooCWESooVbDven} 
    Soit une courbe fermée simple et continue \( \gamma\colon \mathopen[ 0 , 1 \mathclose]\to \eR^2\). Soit \( \Gamma\) son image. Il existe un unique cercle de rayon minimum contenant \( \Gamma\). Ce cercle est le \defe{cercle circonscrit}{cercle!circonscrit à une courbe} à \( \gamma\).

    Il a les propriétés suivantes :
    \begin{enumerate}
        \item
            Le cercle circonscrit à \( \gamma\) coupe \( \Gamma\) en au moins deux points distincts.
        \item
            Tout arc du cercle circonscrit plus grand que le demi-cercle intersection \( \Gamma\).
    \end{enumerate}
\end{propositionDef}

\begin{proof}
    Division de la preuve.
    \begin{subproof}
        \item[Existence]
        L'application \( \gamma\) étant continue, l'ensemble \( \Gamma\) est compact (théorème \ref{ThoImCompCotComp}). Nous considérons l'ensemble \( Q\) des formes quadratiques de la forme 
        \begin{equation}
            q_{a,r}(x)=\| a-x \|^2-r^2
        \end{equation}
        avec \( a\in \eR^2\) et \( r\in \eR\). Nous mettons sur cet ensemble la topologie de \( \eR^2\times \eR=\eR^3\). Le nombre \( q_{a,r}(x)\) est continu en \( a\), \( r\) et \( x\). Soit \( A\) l'ensemble des formes quadratiques de cette forme et vérifiant
        \begin{equation}
            q\big( \gamma(t) \big)\leq 0
        \end{equation}
        pour tout \( t\in\mathopen[ 0 , 1 \mathclose]\). 

        Cet ensemble \( A\) est non vide parce que \( \Gamma \) est compact et donc borné; il existe donc une boule qui contient \( \Gamma\) en son intérieur.

        Montrons que \( A\) est fermé dans \( Q\). Si \( q_{a,r}\notin A\) alors il existe \( t_0\) tel que \( q_{a,r}\big( \gamma(t_0) \big)>0\). Par continuité, il existe un voisinage de \( (a,r)\) dans \( \eR^3\)  et donc de \( q_{a,r}\) dans \( Q\) tel que \( q_{a',r'}\big( \gamma(t_0) \big)\) reste strictement positif pour tout \( (a',r')\) dans ce voisinage.

        En particulier l'ensemble 
        \begin{equation}
            \{ r\in \eR^+\tq \exists\,(a,r)\tq q_{a,r}\in A \}
        \end{equation}
        est fermé et borné vers le base. De plus \( r=0\) n'est pas dans cet ensemble. Il possède donc un minimum strictement positif. Le cercle correspondant donne l'existence.
    \item[Unicité]

        En ce qui concerne l'unicité, si \( \Gamma\) est contenu dans les boules \( B(a,R)\) et \( B(b,R)\) alors 
        \begin{equation}
            \Gamma\subset B(a,R)\cap B(b,R).
        \end{equation}

        \begin{center}
            \input{auto/pictures_tex/Fig_QMWKooRRulrgcH.pstricks}
        \end{center}

        Nous choisissons les axes comme indiqué sur le dessin et nous montrons que l'intersection est dans le cercle de centre \( O\) et de rayon \( \| OI \|<R\). Soit \( Q\) un point de l'intersection; par symétrie il est suffisant de supposer \( Q_x<0\) et \( Q_y>0\). Vu que \( Q\) est dans le cercle de centre \( B\) et de rayon \( R\), il doit satisfaire
        \begin{equation}
            (B_x-Q_x)^2+Q_{y}^2<R.
        \end{equation}
        D'autre part le point \( I\) est d'abscisse \( I_x=0\) et d'ordonnée donnée par \( I_y^2=R^2-B_x^2\). 

        Nous devons prouver que \( Q_x^2+Q_y^2\leq I_y^2\). Il s'agit simplement de calculer
        \begin{equation}
            Q_x^2+Q_y^2\leq Q_x^2+R^2-B_x^2+2B_xQ_x-Q_x^2=R^2-B_x^2+2B_xQ_x\leq R^2-B_x^2=I_y^2
        \end{equation}
        parce que \( B_x>0\) et \( Q_x<0\).

        Nous concluons que \( \Gamma\) est inclus à un cercle de rayon plus petit que \( R\) et donc que \( R\) n'est pas minimum. D'où l'unicité.

    \item[Au moins deux intersections]

        Nous nommons \( C\) le cercle circonscrit à \( \gamma\), et nous écrivons, pour \( p\in \Gamma\) 
        \begin{equation}
            r(p)=d(p,C)
        \end{equation}
        la distance entre \( p\) et \( C\). Cela est une fonction continue sur le compact \( \Gamma\). Elle atteint donc ses bornes sur \( \Gamma\). 

        Si \( C\cap \Gamma=\emptyset\) alors \( r(p)>0\) pour tout \( p\) et le minimum est également strictement positif. Soit \( r_0\) ce minimum. Alors le cercle \( C_2\) même centre que \( C\) mais de rayon \( r_0/2\) n'intersecte pas non plus \( \Gamma\) parce que
        \begin{equation}
            d(p,C)\leq d(p,C_2)+d(C_2,C)
        \end{equation}
        où \( d(p,C)=r(p)\) et \( d(C_2,C)=r_0/2\).
        \begin{equation}
            d(p,C_2)\geq r(p)-\frac{ r_0 }{2}\geq \frac{ r_0 }{2}>0.
        \end{equation}

        Si \( C\cap \Gamma=\{ q \}\) alors un choix d'axe place le centre de \( C\) en \( (0,0)\), le point \( q\) en \( (1,0)\) et fixe le rayon de \( C\) à \( 1\). Pour \( p\in \Gamma\) nous notons \( r(p)\) la distance entre \( p\) et le point de \( C\) situé sur sa gauche, c'est à dire, si \( p=(p_x,p_y)\),
        \begin{equation}
            r(p)=\sqrt{ 1-p_y^2 }+p_x.
        \end{equation}
        Cela est encore une fonction continue sur \( \Gamma\) qui atteint son minimum valant \( r_0\). Alors le cercle de centre \( (\frac{ r_0 }{2},0)\) et de même rayon contient encore \( \Gamma\) mais n'a plus de points d'intersection avec \( \Gamma\).

        Enfin, tout nombre de points d'intersection entre \( C\) et \( \Gamma\) est possible à partir de \( 2\). Pour en avoir deux, prendre une ellipse, et pour en avoir plus, prendre des polynôme dont les angles sont un peu modifiés de façon à rester \( C^1\).

    \item[Intersection avec les demi-arcs]

        Supposons, en fixant encore les axes, que le cercle circonscrit soir encore centré en \( (0,0)\) et que \( \Gamma\) n'intersecte pas le demi-cercle \( x<0\). Alors pour tout \( p\in \Gamma\) la distance entre \( p\) et ce demi-cercle est strictement positive. Il y a un minimum \( r_0\). En décalant le centre du cercle de \( r_0/2\) vers la droite, nous obtenons un nouveau cercle contenant \( \Gamma\) mais ne l'intersectant pas.
    \end{subproof}
\end{proof}

%--------------------------------------------------------------------------------------------------------------------------- 
\subsection{Description locale}
%---------------------------------------------------------------------------------------------------------------------------

\begin{definition}      \label{DEFooVQODooJSNYLw}
    Une courbe plane différentiable est \defe{convexe}{convexe!courbe plane} si son graphe est en tout point d'un seul côté de sa tangente.
\end{definition}

\begin{lemma}[\cite{MonCerveau}]        \label{LEMooGEVEooHxPTMO}
    Soit une courbe simple, convexe \( \gamma\colon \mathopen[ 0 , 1 \mathclose]\to \eR^2\) que nous supposons être de classe \( C^1\). Nous notons \( \Gamma\) l'image de \( \gamma\). Alors \( \Gamma\) est localement le graphe d'une fonction convexe (définition \ref{DefVQXRJQz}).
\end{lemma}

\begin{proof}
    Soit \( p\in \Gamma\) et \( l_p\) la tangente à \( \Gamma\) en \( p\). Nous considérons un système d'axe centré en \( p\) de telle sorte que \( l_p\) soit l'axe des abscisses et l'axe des ordonnées soit dirigé de tel manière que \( \Gamma\) se trouve dans la partie \( y>0\). De plus nous paramétrons \( \gamma\) de telle sorte à avoir \( \gamma(0)=p=(0,0)\).

    Vu que \( l_p\equiv y=0\) est la tangente à \( \Gamma\) nous avons \( \gamma'_y=0\) et \( \gamma_x'(0)>0\). Nous en déduisons que \( \gamma_x'(t)>0\) pour tout \( t\in B(0,\delta)\) pour \( \delta\) suffisamment petit. Nous posons alors
    \begin{equation}
        g(x)=\gamma_y\big( \gamma_x^{-1}(x) \big)
    \end{equation}
    qui est bien définie parce que \( \gamma_x\) est une bijection entre \( B(0,\delta)\) et son image. La fonction \( g\) est continue par la proposition \ref{PropIntContMOnIvCont} et même dérivable par la proposition \ref{PROPooSGTBooFxUuXK}. De plus, vu la formule \eqref{EQooELIHooDxUFxH}, la fonction \( g^{-1}\) est de classe \( C^1\) parce que \( (g^{-1})'\) est une composée d'applications continues.

    \begin{subproof}
        \item[Si \( g\) est \( C^2\)]
            Dans ce cas, \( g''\) ne peut pas changer de signe, sinon la tangente coupe le graphe. Par positivité de \( g\) (et le fait que \( g(0)=g'(0)=0\)), il n'est pas possible d'avoir \( g''<0\) partout. Donc \( g''\geq 0\) partout. Cela prouve que \( g\) est convexe par la caractérisation \ref{ThoGXjKeYb}.
        \item[Si \( g\) est seulement de classe \( C^1\)]
            Le graphe de \( g\) correspond au graphe de \( \Gamma\). Nous montrons que \( g\) est convexe en utilisant la caractérisation de la proposition \ref{PROPooQPOSooDZlUAJ}.
    
            La tangente au graphe de \( g\) en \( x=x_0\), que nous notons \( l_0\), est la tangente à \( \Gamma\) en \( t=\gamma_x^{-1}(x_0)\). Le graphe de \( g\), qui est une partie de \( \Gamma\) se trouve donc d'un seule côté de \(l_0\).

            Nous nous restreignons \( g\) à un compact  \( I\) et nous considérons la fonction
            \begin{equation}
                d_a(x)=g(x)-l_a(x)
            \end{equation}
            qui donne la distance entre le graphe de \( g\) et la tangente à \( g\) en \( x=a\). Cela est une fonction continue en \( x\) et en \( a\). Le graphe de \( g\) est au dessus de la tangente en \( x=0\) (par construction des axes). Supposons que le graphe de \( g\) soit en dessous de la tangente en \( x=x_2\). Alors nous avons, pour tout \( x\) :
            \begin{subequations}
                \begin{numcases}{}
                    d_0(x)\geq 0\\
                    d_{x_1}(x)\leq 0.
                \end{numcases}
            \end{subequations}
            Nous posons 
            \begin{equation}
                s(a)=\sup_{x\in I}d_a(x),
            \end{equation}
            qui est une fonction continue par la proposition \ref{PROPooWXBAooAEweSF}. Vu le définitions, \( s(0)\geq 0\) et \( s(x_2)\leq 0\). Il existe donc \( m\in \mathopen[ 0 , m_2 \mathclose]\) tel que \( s(m)=0\). À ce moment nous avons \( g(x)=l_m(x)\) pour tout \( x\in I\) et donc \( g\) est une droite, et en réalité toutes les inégalités sont des égalités. La fonction \( g\) est alors bien convexe (mais pas strictement).
    \end{subproof}
\end{proof}

%--------------------------------------------------------------------------------------------------------------------------- 
\subsection{Enveloppe convexe}
%---------------------------------------------------------------------------------------------------------------------------

\begin{proposition}[\cite{ooHJQTooKVMAdi,MonCerveau}]       \label{PROPooWZITooTFiWsi}
    Soit une courbe simple, fermée et convexe \( \gamma\colon \mathopen[ 0 , 1 \mathclose]\to \eR^2\) que nous supposons être de classe \( C^2\). Nous notons \( \Gamma\) l'image de \( \gamma\). Alors il existe un convexe \( D\) tel que \( \partial D=\Gamma\).
\end{proposition}

\begin{proof}
    Pour \( p\in \Gamma\) nous notons \( l_p\) la tangente à \( \Gamma\) en \( p\) et \( H_p\) le demi-plan (fermé) contenant \( \Gamma\). Nous posons
    \begin{equation}        \label{EQooDYFTooCHRbsD}
        D=\bigcap_{p\in \Gamma}H_p.
    \end{equation}
    Cet ensemble est convexe comme intersection de convexes et fermé comme intersection de fermés. Nous prouvons que \( \Gamma=\partial D\). 


    L'inclusion \( \Gamma\subset\partial D\) est la plus facile. Si \( p\in \Gamma\) alors \( p\) est dans chacun des \( H_q\) et donc dans \( D\). De plus tout voisinage de \( p\) contient des points en dehors de \( H_p\), donc \( p\) n'est pas dans l'intérieur de \( D\). Ce dernier étant fermé, un point hors de l'intérieur est sur le bord. Ergo \( p\in\partial D\).

    Pour l'inclusion inverse, soit \( p\in\partial D\).
    \begin{subproof}
        \item[Il existe \( q\) tel que \( p\in l_q\)] Vu que \( D\) est fermé, le point \( p\) est dans \( D\), et donc dans tous les \( H_q\). Supposons qu'il soit dans l'intérieur de tous les \( H_q\). Alors nous considérons la fonction
            \begin{equation}
                r(q)=\frac{ d(p,H_q) }{ 2 }
            \end{equation}
            définie sur \( \Gamma\). C'est une fonction continue\footnote{L'équation de la droite \( l_q\) a des coefficients continus parce que \( \gamma\) est de classe \( C^1\).} strictement positive définie sur le compact \( \Gamma\) qui possède donc un minimum strictement positif. Si \( r_0\) est ce minimum, alors \( B(p,r_0)\) est inclue à tous les \( H_q\), ce qui ferait que \( p\) est à l'intérieur de \( D\). Nous concluons à l'existence de \( q\in \Gamma\) tel que \( p\in l_q\).
        \item[Le point où ça décolle]
            Nous supposons que \( p\notin \Gamma\), sinon ce serait trop facile. Nous paramétrons \( \gamma\) de telle sorte à avoir \( q=\gamma(0)\) et nous posons
            \begin{equation}
                T=\{ t\in \eR^+\tq l_{\gamma(t)}=l_q \}.
            \end{equation}
            Cela est un fermé dans \( \eR\) parce que \( \gamma\) est de classe \( C^1\). Nous posons \( t_0=\inf(T^c)\) et pour tout \( \epsilon\) suffisamment petit,
            \begin{subequations}
                \begin{numcases}{}
                    l_{\gamma(t_0+\epsilon)}\neq l_q,\\
                    l_{\gamma(t_0-\epsilon)}=l_q.
                \end{numcases}
            \end{subequations}
            La seconde est parce que si \( l_{\gamma(t_0-\epsilon)}\neq l_q\) nous aurions \( t_0-\epsilon\in T^c\). Soit \( r=\gamma(t_0)\); nous avons \( l_r=l_q\) parce que si \( l_r\neq l_q\) alors par continuité de \( \gamma'\) nous aurions \( l_{r'}\neq l_q\) pour tout \( r'\in \gamma\big( B(t_0,\delta) \big)\).

        \item[Graphe d'une fonction strictement convexe] En suivant le lemme \ref{LEMooGEVEooHxPTMO}, l'ensemble \( \Gamma\) est localement (autour de \( r\)) le graphe d'une fonction convexe au-dessus de \( l_r\). Soit \( g\colon B(0,\delta)\to \eR^2\) cette fonction convexe.

        Soit \( \epsilon>0\). Si \( g''(x)=0\) sur \( \mathopen] 0 , \epsilon \mathclose]\) alors \( g'\) y est constante. Mais \( g'(0)=\), ce qui signifierait que sur \( \mathopen[ 0 , \epsilon \mathclose]\) nous ayons \( g'(x)=0\) et donc \( g(x)=0\). Cela ferait que \( l_{r'}\equiv y=0\) pour tout \( r'\) de la forme \( (x,0)\) avec \( x\in \mathopen[ 0 , \epsilon \mathclose]\) (qui sont des points de \( \Gamma\)). Cela est en contradiction avec la définition de \( r\). Donc il existe un point \(x\in \mathopen[ 0 , \epsilon \mathclose[\) tel que \( g''(x)>0\).

        Rappelons que \( p\in l_q\), ce qui fait que \( p\) a pour coordonnées \( (p_x,0)\). Nous restreignons \( \delta\) et \( \epsilon\) de telle sorte que \( p_x\) soit plus grand à la fois que \( \epsilon\) et \(\delta\).

        Il existe donc un intervalle \( \mathopen[ a , b \mathclose]\) avec \( a,b\geq 0\) et \( a,b<p_x\) sur lequel \( g\) est strictement convexe. 
        
    \item[La tangente qui tue]

        En particulier \( g(b)>0\) et le théorème de des accroissements finis \ref{ThoAccFinis}\ref{ITEMooFZONooXJqLyX} nous donne l'existence de \( m\in\mathopen[ 0 , b \mathclose]\) tel que la tangente à \( g\) en \( x=m\) est parallèle au segment joignant \( (0,0)\) à \( (b,f(b))\). Cette tangente, que nous nommons \( l_m\), est en dessous de la corde, par strict convexité. En particulier, son point d'intersection avec \( y=0\) est strictement entre \( 0\) et \( m\).

        L'ensemble \( D\) est d'un seul côté de \(l_m\). Ce côté est forcément celui de \( q=(0,0)\) (parce que \( q\in \Gamma\subset D\)), et donc le points de coordonnées \( (x,0)\) avec \( x>m\) ne sont pas dans \(D\). Pas de chance, \( p\) est un point de ce type.

    \item[La contradiction]
        Nous avons prouvé que si \( p\in \partial D\setminus \Gamma\) alors \( p\) n'est pas dans \( D\), ce qui est impossible parce que, l'ensemble \( D\) étant fermé, nous avons \( \partial D\subset D\).
    \end{subproof}
\end{proof}

\begin{remark}
   Bien que cela puisse paraître évident dès le début, nous ne démontrerons que dans la proposition \ref{PROPooOORPooCXrIQi} que \( D\) est l'enveloppe convexe de \( \Gamma\).
\end{remark}

\begin{corollary}       \label{CORooSXDGooJEmVcf}
    Si \( p\in\Int(D)\) alors toute droite passant par \( p\) intersecte \( \Gamma\) en exactement \( 2\) points.
\end{corollary}

\begin{proof}
    Vu que la partie \( D\) est bornée, toute droite passant par son intérieur coupe \( \partial D\) en au moins deux points (un dans chaque sens, et en utilisant le lemme de passage de douane \ref{LEMooLKWEooItGnkP}).

    Soit \( \ell\) une droite passant par \( p\) et supposons qu'elle coupe \( \Gamma\) en trois points distincts. Alors au moins deux d'entre eux sont du même côté de \( p\). Soient \( q_1\) et \( q_2\) ces points. Nous avons donc dans l'ordre \( p\in \Int(D)\), \( q_1\in\partial D\) et \( q_2\in \partial D\). 

    Vu que tout \( \Gamma\) est d'un seul côté de ses tangentes, lesdites tangentes ne passent pas par l'intérieur de \( D\). Ni \( p\) ni \( q_2\) ne sont sur \( \ell_{q_1}\), parce que si \( q_2\in\ell_{q_1}\) alors \( \ell_{q_1}=\ell\), ce qui est impossible parce que \( \ell\) passe par \( p\in\Int(D)\).

    Or \( p\) et \( q_2\) sont de part et d'autres de \( \ell_{q_1}\), ce qui est impossible parce que \( \Gamma\) est d'un seul côté de cette droite.
\end{proof}

\begin{proposition}     \label{PROPooOORPooCXrIQi}
    Soit \( D\) l'ensemble définit en \eqref{EQooDYFTooCHRbsD}.
    \begin{enumerate}
        \item
            \( D=\Conv(\Gamma)\) (\( \Conv(\Gamma)\) désigne l'enveloppe convexe de \( \Gamma\))
        \item
            \( \partial\Conv(\Gamma)=\Gamma\).
    \end{enumerate}
    Pour la définition d'enveloppe convexe, voir la définition \ref{DefNLYYooXUHFUY}.
\end{proposition}

\begin{proof}
    L'ensemble \( D\) est un convexe contenant \( \Gamma\). Donc \( \Conv(\Gamma)\subset D\). L'inclusion inverse est à prouver.
    
    Soit \( x\in D\). Si \( x\in \partial D\) alors \( x\in \Gamma\) (proposition \ref{PROPooWZITooTFiWsi}) et donc \( x\in\Conv(\Gamma)\). Nous ne devons donc traiter que le cas \( x\in\Int(D)\).
    
    Le corollaire \ref{CORooSXDGooJEmVcf} nous dit que toute droite passant par \( x\) coupe \( \Gamma\) en exactement deux points. Soient \( p\) et \( q\) ces points. Alors \( p,q\in \Gamma\subset \Conv(\Gamma)\). Vu que \( \Conv(\Gamma) \) est convexe, tout le segment \( [p,q]\) est dans \( \Conv(\Gamma)\), et en particulier \( p\in\Conv(\Gamma)\).

    Nous passons à la seconde affirmation. Nous savons que \( D=\Conv(\Gamma)\). En prenant le bord des deux côtés, \( \partial D=\partial\Conv(\Gamma)\), donc \( \Gamma=\partial\Conv(\Gamma)\).
\end{proof}

\begin{lemma}[Des tangentes parallèles\cite{ooYGXBooTzOAtL}]        \label{LEMooUEKQooWhGyKn}
    Une courbe fermée \( \gamma\) de classe \( C^1\) est convexe si et seulement si elle ne possède pas \( 3\) tangentes parallèles distinctes.
\end{lemma}

\begin{proof}
    Si le graphe \( \Gamma\) possédait trois tangentes parallèles distinctes, une serait entre les deux autres et le graphe \( \Gamma\) serait de part et d'autres des cette tangente. Dans ce cas, \( \gamma\) n'est pas convexe.

    Nous montrons maintenant que si \( \gamma\) n'est pas convexe, alors elle possède trois tangentes parallèles distinctes. Pour \( p\in \Gamma\) tel que \( \Gamma\) soit des deux côtés de \( l_p\).

    Soient \( \Gamma_1\) et \( \Gamma_2\) les parties de \( \Gamma\) délimitées par \( l_p\). Nous notons \( q_i\) le point de \( \Gamma_i\) le plus éloigné de la droite \( l_p\), il existe parce que \( G\) est compact et que la fonction distance à \( l_p\) est continue sur \( \Gamma\). Les points \( p\), \( q_1\) et \( q_2\) sont distincts (sinon \( l_p\) ne couperait pas \( \Gamma\) en deux parties).

    Montrons que \( l_{q_i}\parallel l_p\). Pour cela nous choisissons un système d'axe dans lequel \( l_p\equiv y=0\). Dans ce système, la distance entre \( \gamma(t)\) et \( l_p\) est \( \gamma_y(t)\) et les extrema de cette fonction ont lieu aux points \( t\) tels que \( \gamma_y'(t)=0\), c'est à dire aux points sur lesquels la tangente est parallèle à \( l_p\).

    À quel moment avons nous utilisé le fait que la courbe soit fermée ? Au moment de dire que le point le plus éloigné devait vérifier \( \gamma'_y(t)=0\). En effet un extrema peut ne pas vérifier cette condition s'il n'est pas à l'intérieur du domaine. Dans notre cas, nous avons \( \gamma\colon \mathopen[ 0 , 1 \mathclose]\to \eR^2\) qui est fermée :  \( \gamma(0)=\gamma(1)\). Donc en réalité nous pouvons considérer \( \gamma\colon \eR\to \eR^2\) et tous les points du domaine sont intérieurs au domaine. L'extremum doit donc vérifier la condition d'annulation de la dérivée.
\end{proof}

\begin{example}
    Si la courbe n'est pas fermée, alors le lemme \ref{LEMooUEKQooWhGyKn} ne tient pas comme le montre le contre-exemple du graphe de \( f(x)=x^3-x\). Il est non convexe et pourtant ne présente que \( 2\) tangentes parallèles (dans chaque directions).
\end{example}

\begin{lemma}       \label{LEMooCSXCooIDPiKW}
    Soit \( \gamma\) une courbe convexe et \( \ell\) une droite qui intersecte \( \Gamma\) mais qui n'est pas tangente. Alors l'intersection entre \( \ell\) et \( \Gamma\) comprend au maximum \( 2\) points.
\end{lemma}

\begin{proof}
    Supposons que \( \ell\) coupe \( \Gamma\) en trois points \( p,q,r\) (dans cet ordre). Vu que \( \ell\) n'est pas tangente à \( \Gamma\), la tangente \( \ell_q\) est distincte de \( \ell\) (et intersecte \( \ell\) en l'unique point \( q\)). Les points \( p\) et \( r\) sont dans \( \Gamma\) et sont pourtant de deux côtés différents de \( \ell_q\). Contradiction avec la convexité de \( \gamma\).
\end{proof}

La proposition suivante nous dit que si deux points de \( \Gamma\) ont la même tangente, alors entre ces deux points, \( \Gamma\) est le segment de droite les joignant.
\begin{proposition}[\cite{MonCerveau}]     \label{PROPooCKTZooIPcUca}
    Soit \( \gamma\colon \mathopen[ 0 , L \mathclose]\to \eR^2\) une courbe fermée simple et convexe de classe \( C^1\). Si la tangente en \( p=\gamma(s_p)\) et la tangente en \( q=\gamma(s_q)\) sont identiques (pas seulement parallèles), alors soit 
    \begin{equation}
        \gamma\big( \mathopen[ s_p , s_q \mathclose] \big)=[p,q]
    \end{equation}
    soit
    \begin{equation}
        \gamma\big( \mathopen[ s_q , L \mathclose] \big)=[p,q]
    \end{equation}
\end{proposition}

\begin{proof}
    Nous considérons un système d'axe dans lequel \( p=(0,0)\), \( \ell_p=\ell_q\equiv y=0\) et tel que \( \gamma_y(t)\geq 0\) pour tout \( t\). Nous choisissons enfin une paramétrisation de \( \gamma\) telle que \( p=\gamma(0)\).

    Si \( \gamma_y\big( \mathopen[ 0 , s_q \mathclose] \big)=\{ 0 \}\) alors par le théorème des valeurs intermédiaires \ref{ThoValInter}, tous les points du type \( (t,0)\) avec \( 0\leq t\leq q_x\) sont atteints par \(  \gamma(t)   \) avec \( 0\leq t\leq s_q\). De plus aucun autre point ne peut être atteint parce que \( \gamma\) étant simple, elle ne peut pas faire de retour en arrière.

    Nous supposons donc que \( \gamma_y\) n'est pas identiquement nulle sur \( \mathopen[ 0 , s_q \mathclose]\); il existe donc un \( s_M\) avec  \( 0<s_M<q_q\) qui maximise \( \gamma_y\) sur \( \mathopen[ 0 , s_q \mathclose]\). La tangente à \( \Gamma\) en \( \gamma(s_M)\) est horizontale. Les droites \( y=0\) et \( y=\gamma_y(s_M)\) sont donc deux tangentes parallèles à \( \Gamma\). Par le lemme des tangentes parallèles \ref{LEMooUEKQooWhGyKn}, il n'y a pas d'autres tangentes horizontales. Donc pour tout \( n\in \eN\) la droite \( y=\frac{1}{ n }\) n'est tangente nulle part\footnote{À part \( n=0\) et si par manque de chance, \( \gamma_y(s_M) \) est un nombre de la forme \( 1/n\).} à \( \Gamma\).

    Par le lemme \ref{LEMooCSXCooIDPiKW}, la droite \( y=\frac{1}{ n }\) ne peut intersecter \( \Gamma\) qu'en seulement deux points. Or le théorème des valeurs intermédiaires appliqué à \( \gamma_y\) sachant que \( \gamma_y(0)=\gamma_y(s_q)=0\) et \( \gamma_y(s_M)>0\) nous donne \( a_n,b_n\) tels que
    \begin{subequations}
        \begin{align}
            0<a_n<s_M\\
            s_M<b_n<s_q
        \end{align}
    \end{subequations}
    et \( \gamma_y(a_n)=\gamma_y(b_n)=\frac{1}{ n }\). Donc la droite \( y=\frac{1}{ n }\) intersecte \( \gamma\) deux fois dans \( \mathopen[ 0 , s_q \mathclose]\). En conséquence de quoi \( \gamma_y(t)<\frac{1}{ n }\) pour tout \( t\in\mathopen[ q_q , L \mathclose]\). Cela étant valable pour tout \( n\) nous avons \( \gamma_y\big( \mathopen[ s_q , L \mathclose] \big)=\{ 0 \}\) et nous sommes ramenés essentiellement au premier cas.
\end{proof}

\begin{proposition}     \label{PROPooKHUQooIOUxFw}
    Soit une courbe fermée simple \( \gamma\) de classe \( C^1\) et une droite \( \ell\). Alors il existe au moins deux points distincts \( q_1,q_2\) tels que \( \ell_{q_1}\parallel \ell_{q_2}\parallel\ell\) avec \( \ell_p\neq \ell_q\).
\end{proposition}

\begin{proof}
    Pour \( p\in \Gamma\) nous considérons le nombre \( r(p)=d(p,\ell)\). En tant que fonction continue sur un compact, elle possède un minimum et un maximum. Dans un système d'axe pour lequel \( \ell\equiv y=0\), la fonction \( r\) s'écrit \( r(p)=p_y\) et les extrema arrivent en \( \gamma(s)\) avec \( \gamma_y'(s)=0\), ce qui signifie que les tangentes aux extrema sont parallèles à \( \ell\). Vu que le maximum et le minimum ne peuvent pas être égaux (sinon la courbe serait horizontale et pas simple), les tangentes en ces points sont distinctes.
\end{proof}

\begin{corollary}
    Soit une courbe convexe fermée simple \( \gamma\) de classe \( C^2\). L'ensemble $\Conv(\Gamma)$ est compact.
\end{corollary}

\begin{proof}
    Nous savons que \( \Conv(\Gamma)\) n'est autre que \( D\) par la proposition \ref{PROPooOORPooCXrIQi}. Nous savons déjà que \( D\) est fermé. Il nous suffit donc de prouver qu'il est borné (théorème de Borel-Lebesgue \ref{ThoXTEooxFmdI}). Nous considérons deux droites perpendiculaires et les \( 4\) tangentes correspondantes par la proposition \ref{PROPooKHUQooIOUxFw}. Vu que \( \Gamma\) est d'un seul côté de chacune de ces tangentes, elle est contenue dans le rectangle délimité par ces \( 4\) droites.
\end{proof}

%--------------------------------------------------------------------------------------------------------------------------- 
\subsection{Courbure et convexité}
%---------------------------------------------------------------------------------------------------------------------------
\label{SUBSECooNJOLooYuGRjA}

\begin{lemma}       \label{LEMooHMFSooFlhanD}
    Soit \(   \gamma\colon \mathopen[ 0 , L \mathclose]\to \eR^2    \) une courbe simple, fermée en paramétrisation normale. Alors l'application
    \begin{equation}
        \begin{aligned}
            \sigma&\colon \mathopen[ 0 , L \mathclose]\to S^1\\
            s&\mapsto \gamma(s) 
        \end{aligned}
    \end{equation}
    est surjective.
\end{lemma}

\begin{proof}
    Le théorème \ref{THOooEQWOooBCRMMZ} nous dit que si \( \theta(0)=a\) alors \( \theta(2\pi)=a+2\pi\) ou \( a-2\pi\). Le théorème des valeurs intermédiaires nous dit alors que \( \theta\) prend toutes les valeurs entre \( a\) et \( a+2\pi\) ou \( a-2\pi\).
\end{proof}

\begin{proposition}[\cite{ooYGXBooTzOAtL}]      \label{PROPooWXUKooPOtPdj}
    Une courbe fermée simple de classe \(  C^{2}\) est convexe si et seulement si sa courbure est soit toujours positive, soit toujours négative.
\end{proposition}

\begin{proof}
    Nous considérons la courbe \( \gamma\) munie d'un paramétrage de vitesse \( 1\), c'est à dire avec \( \| \gamma'(t) \|=1\) pour tout \( t\). Si \( \theta\) est sa fonction d'angle, alors nous avons \( \theta'=\kappa\) par le lemme \ref{LEMooWLAUooKetUiW}. Donc la fonction \( \theta\) est monotone si et seulement si la courbure ne change pas de signe. Nous allons montrer que \( \theta\) est monotone si et seulement si \( \gamma\) est convexe.

    \begin{subproof}
        \item[\( \Rightarrow\)]

        Nous supposons que \( \theta\) est monotone et \( \gamma\) non convexe. Soient \( p\in \Gamma\) tel que \( \Gamma\) soit des deux côtés de \( \ell_p\), et soient les points \( q_1\), \( q_2\) donnés par le lemme des tangentes parallèles \ref{LEMooUEKQooWhGyKn} tels que \( \ell_p\parallel\ell_{q_1}\parallel\ell_{q_2}\). Parmi les vecteurs tangents en \( p\), \( q_1\) et \( q_2\), deux au moins ont la même direction; supposons que ce sont \( q_1\) et \( q_2\). C'est à dire que si \( p=\gamma(s_0)\), \( q_1=\gamma(s_1)\) et \( q_2=\gamma(s_2)\) alors nous avons \( \gamma'(s_1)=\gamma'(s_2)\) et donc aussi
        \begin{equation}
            \theta(s_1)=\theta(s_2)+2n\pi
        \end{equation}
        pour un certain \( n\). Mais \( \theta\) est monotone et la différence entre sa première et sa dernière valeur doit valoir \( 2\pi\) ou \( -2\pi\) par le théorème \ref{THOooEQWOooBCRMMZ}. Donc \( n\) ne peut valoir que \( -1\), \( 0\) et \( 1\).

        Si \( n=0\) alors \( \theta\) est constante sur \( \mathopen[ s_1 , s_2 \mathclose]\). Si \( n=1\) alors \( \theta(s_1)=\sigma(s_2)+2\pi\) alors que sur toute la courbe, \( \theta\) ne peut faire que \( 2\pi\). Donc \( \theta\) est constant sur \( \mathopen[ 0 , s_1 \mathclose]\) et sur \( \mathopen[ s_2 , L \mathclose]\) (où \( L\) est le bord de la paramétrisation). Si \( n=-1\), même conclusion.

        Dans tous les cas, \( \Gamma\) contient une ligne droite, soit de \( q_1\) à \( q_2\), soit de \( q_2\) à \( q_1\). Et dans ces cas nous avons \( \ell_{q_1}=\ell_{q_2}\), ce qui est contraire à la construction de \( q_i\). 

        Nous concluons que \( \gamma\) est convexe.

        \item[\( \Leftarrow\)]

            Nous supposons que \( \gamma\) est convexe, mais que \( \theta\) n'est pas monotone. Il existe donc \( s_1<s_0<s_2\) tels que
            \begin{equation}
                \theta(s_1)=\theta(s_2)\neq \theta(s_0).
            \end{equation}
            Et vu le lemme \ref{LEMooHMFSooFlhanD}, il existe \( s_3\) tel que \( \gamma'(s_3)=-\gamma'(s_1)\).

            Donc en \( s_1\), \( s_2\) et \( s_3\) nous avons trois tangentes parallèles. La proposition \ref{LEMooUEKQooWhGyKn} est alors formelle, \( \gamma\) étant convexe, deux de ces tangentes doivent être identiques.

            La proposition \ref{PROPooCKTZooIPcUca} dit qu'entre deux points dont les tangentes sont identiques, la courbe doit être un segment de droite. Or sur un segment de droite, \( \kappa=0\) et \( \theta\) est constante.

            \begin{itemize}
                \item 
            La partie \( \gamma\big( \mathopen[ s_1 , s_2 \mathclose] \big)\) ne peut pas être droite parce que nous avions supposé l'existence d'un \( s_0\in \mathopen] s_1 , s_2 \mathclose[\) tel que \( \theta(s_1)=\theta(s_2)\neq \theta(s_0)\).

            \item
            La partie \( \gamma\big( \mathopen[ s_1 , s_3 \mathclose] \big)\) ne peut pas être droite parce que \( \theta(s_3)\neq \theta(s_1)\).
            \item
            La partie \( \gamma\big( \mathopen[ s_2 , s_3 \mathclose] \big)\) ne peut pas être droite parce que \( \theta(s_2)\neq \theta(s_1)\).
            \end{itemize}
            Nous sommes donc devant une contradiction.

            Nous en concluons que \( \theta\) doit être monotone.
    \end{subproof}
\end{proof}

%--------------------------------------------------------------------------------------------------------------------------- 
\subsection{Théorème des quatre sommets}
%---------------------------------------------------------------------------------------------------------------------------

\begin{lemma}       \label{LEMooELIRooNDVXPh}
    Soit une droite \( \ell\) du plan. Il existe \( a,c\in \eR^2\) avec \( c\neq 0\) tels que \( z\in \ell\) si et seulement si \( (z-a)\cdot c=0\).
\end{lemma}

\begin{proof}
    Une droite est paramétrée par \( \gamma(t)=p+tq\). En posant \( a=p\) et \( c=Jq\) nous avons la réponse. En effet nous allons montrer qu'avec ces valeurs de \( a\) et \( c\), nous avons \( z\in \Gamma\) si et seulement si \( (z-a)\cdot c=0\).

    D'abord un point de \( \gamma\) est de la forme \( z=\gamma(t)=p+tq\). Nous avons :
    \begin{equation}
        \big( \gamma(t)-a \big)\cdot c=\big( \gamma(t)-p \big)\cdot Jq=tq\cdot Jq=0.
    \end{equation}
    
    Et dans l'autre sens, si \( (z-a)\cdot c=0\) nous devons prouver que \( z\in \Gamma\). Nous avons : \( (z-p)\cdot Jq=0\), ce qui fait que \( z-p\) est un multiple de \( q\). Autrement dit : \( z-p=\lambda q\) ou encore \( z=\alpha q+p\), qui est sur la droite \( \Gamma\).
\end{proof}

\begin{definition}
    Un \defe{sommet}{sommet} d'une courbe est un point d'extremum local de la courbure.
\end{definition}

\begin{theorem}[Théorème des quatre sommets\cite{KXjFWKA,ooIEJXooIYpBbd}]       \label{THOooFRBBooWKZcfY}
    Soit un arc paramétrique \( \gamma\colon \eR\to \eR^2\) fermé, simple et convexe\footnote{Par la proposition \ref{PROPooWXUKooPOtPdj} nous pouvons aussi bien demander à la courbure d'être toujours strictement positive, comme le fait \cite{KXjFWKA}.} de classe \( C^3\) et \( T\)-périodique.

    Alors \( \gamma\) possède au moins \( 4\) points critiques sur chaque période.
\end{theorem}

\begin{proof}
    Nous supposons que la paramétrisation de \( \gamma\) soit normale.

    Si la courbure \( \kappa\) est constante sur une partie ouverte de la (paramétrisation de la) courbe, alors tous les points de cette partie sont des sommets et le théorème est fait. Nous supposons que \( \kappa\) n'est pas constante et en particulier que \( \Gamma\) ne contient ni bouts de droites ni bouts de cercles (théorème \ref{THOooDLDVooFQnLWn}).

    La fonction \( \kappa\) étant de classe \( C^1\) sur le compact \( \Gamma\), elle admet au moins un maximum et un minimum distincts. Vu que ces points sont intérieurs, ils correspondent au changement de signe de \( \kappa'\). Soient \( p\) et \( q\) ces points. Pour la simplicité nous supposons que \( \gamma\) est paramétré de telle sorte que \( \gamma(0)=p\), et \( q=\gamma(s_q)\) avec \( 0<s_q<T\).

    Nous supposons que \( p\) et \( q\) sont les seuls points de changement de signe de \( \kappa'\).

    Soit \( \ell\) la droite passant par \( p\) et \( q\). Tous les points du segment \( [p,q]\) (qui sont dans \( \Conv(\Gamma)\)) ne peuvent pas être sur \( \Gamma\) (sinon nous aurions un morceau de droite). Donc certains points sont dans l'intérieur de \( \Conv(\Gamma)\). Donc la droite \( \ell\) passe par l'intérieur de \( \Conv(\Gamma)\) et le corollaire \ref{CORooSXDGooJEmVcf} nous dit que la droite \( \ell\) ne coupe \( \Gamma\) en seulement deux points.

    Par conséquent, les ensembles \( \gamma\big( \mathopen[ 0 , s_q \mathclose] \big)\) et \( \gamma\big( \mathopen[ s_q , T \mathclose] \big)\) sont de part et d'autre de \( \Gamma\). Vu qu'en ces points, \( \kappa'\) change de signe et qu'il ne change de signe en aucun autre points, la fonction \( \kappa'\) est positive d'un côté de \( \ell\) et négative de l'autre. 

    D'autre part par le lemme \ref{LEMooELIRooNDVXPh}, il existe \( a\in \eR^2\) et \( c\neq 0\) tels que \( z\in\ell\) si et seulement si \( (z-a)\cdot c=0\). La fonction \( z\mapsto (z-a)\cdot c\) est donc positive d'un côté de \( \ell\) et négative de l'autre.

    En résumé les fonctions 
    \begin{subequations}
        \begin{align}
            s&\mapsto \kappa'(s)\\
            s&\mapsto \big( \gamma(s)-a \big)\cdot c
        \end{align}
    \end{subequations}
    changent de signe en même temps et le produit a donc un signe constant. Ce produit n'est de plus pas nul parce que \( \kappa'\) n'est nul sur aucun intervalle (sinon \( \kappa\) y serait constant et \( \Gamma\) un segment de droite) et \( \big( \gamma(s)-a \big)\cdot c\) ne s'annule pour aucun \( s\) sauf ceux qui correspondent à \( p\) et \( q\).

    Nous avons donc
    \begin{subequations}
        \begin{align}
            0&\neq \int_0^T\kappa'(s)\big( \gamma(s)-a \big)\cdot c\,ds\\
            &=\underbrace{\Big[ \big( \gamma(s)-a \big)\cdot c\kappa(s) \Big]_0^{T}}_{A=0}-\int_0^T\kappa(s)\gamma'(s)\cdot c\,ds\\
            &=-\int_0^T\kappa(s)\big( \gamma'(s)\cdot c \big)ds\\
            &=\int_0^TJ\gamma''(s)\cdot c\,ds\\
            &=J\int_0^T\gamma''(s)\cdot c\,ds\\
            &=J\big[ \gamma'(s)\cdot c \big]_0^T\\
            &=0.
        \end{align}
    \end{subequations}
    Justifications :
    \begin{itemize}
        \item 
    L'expression \( A\) est nulle parce que les valeurs en \( 0\) et en \( T\) sont identiques. 
        \item
            Nous utilisons le lemme \ref{LEMooKPORooEGJCRm} pour faire \( -\kappa(s)\gamma'(s)=J\gamma''(s)\).
    \end{itemize}
    Le tout est une contradiction de la forme \( 0\neq a=0\).

    Nous avons donc au moins un troisième point de changement de signe de \( \kappa'\). Vu que la courbe est périodique, il en faut un nombre pair et donc un quatrième.
\end{proof}

L'exemple de l'ellipse montre qu'il n'y a pas lieu de chercher d'autres extrema de \( \kappa\) à part les \( 4\) déjà trouvés.

\begin{example}
    Nous trouvons les sommets de l'ellipse.
    \begin{subequations}
        \begin{align}
            \gamma(t)&=\big( a\cos(t),b\sin(t) \big)\\
            \gamma'(t)&=\big( -a\sin(t),b\cos(t) \big)\\
            \gamma''(t)&=-\big( a\cos(t),b\sin(t) \big)\\
        \end{align}
    \end{subequations}
    La courbure est
    \begin{subequations}
        \begin{align}
        \kappa(t)&=\frac{ \gamma''\cdot J\gamma' }{ \| \gamma' \|^3 }\\
        &=\frac{-1}{ [a^2\sin^2(t)+b^2\cos^2(t)]^{3/2} }\begin{pmatrix}
            a\cos(t)    \\ 
            b\sin(t)    
        \end{pmatrix}\cdot\begin{pmatrix}
            -b\cos(t)    \\ 
            -a\sin(t)    
        \end{pmatrix}\\
        &=\frac{ ab }{  [a^2\sin^2(t)+b^2\cos^2(t)]^{3/2}  }.
        \end{align}
    \end{subequations}
    Vu que \( ab>0\), les extrema de cela sont ceux du dénominateur et il suffit donc d'étudier les extrema de
    \begin{equation}
        f(t)=a^2\sin^2(t)+b^2\cos^2(t).
    \end{equation}
    Nous avons 
    \begin{equation}
        f'(t)=2(a^2-b^2)\cos(t)\sin(t),
    \end{equation}
    fonction qui s'annule effectivement $4$ fois sur une période. Deux maxima et deux minima.
\end{example}

%--------------------------------------------------------------------------------------------------------------------------- 
\subsection{Le théorème de Jordan}
%---------------------------------------------------------------------------------------------------------------------------

\begin{definition}[\cite{ooTXKNooIgJrPw}]
    Une \defe{courbe de Jordan}{courbe!de Jordan} est une courbe simple fermée dans le plan.
\end{definition}

Le théorème suivant a un énoncé relativement simple, mais la démonstration est en réalité très longue.
\begin{theorem}[Théorème de Jordan\cite{ooTXKNooIgJrPw}]
     Le complémentaire d'une courbe de Jordan \( \Gamma\) dans un plan affine réel est formé de exactement deux composantes connexes distinctes, dont l'une est bornée et l'autre non. Toutes deux ont pour frontière la courbe \( \Gamma\).
\end{theorem}
\index{théorème!de Jordan}
% Si un jour on travaille sur ce théorème, il y a moyen de revoir la réponse de Alphago dans
% http://math.stackexchange.com/questions/1727310/convex-curve-as-boundary-of-a-convex-set

