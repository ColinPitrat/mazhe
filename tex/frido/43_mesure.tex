% This is part of Mes notes de mathématique
% Copyright (c) 2011-2017
%   Laurent Claessens, Carlotta Donadello
% See the file fdl-1.3.txt for copying conditions.

%+++++++++++++++++++++++++++++++++++++++++++++++++++++++++++++++++++++++++++++++++++++++++++++++++++++++++++++++++++++++++++
\section{Mesure de Lebesgue sur \texorpdfstring{$ \eR$}{R}}
%+++++++++++++++++++++++++++++++++++++++++++++++++++++++++++++++++++++++++++++++++++++++++++++++++++++++++++++++++++++++++++
\label{SecZTFooXlkwk}

Nous notons \( \mS\) l'ensemble des intervalles\footnote{Définition \ref{DefEYAooMYYTz}.} de \( \eR\).

\begin{proposition}
    L'ensemble réunions finies d'éléments de \( \mS\) est une algèbre de parties de \( \eR\) que nous allons noter \( \tribA_{\mS}\).
\end{proposition}

\begin{proof}

    Nous devons vérifier la définition \ref{DefTCUoogGDud}. Les ensembles \( \eR\) et \( \emptyset\) sont des intervalles et font donc partie de \( \tribA_{\mS}\).
    
    Si \( A\in\tribA_{\mS}\) se décompose en union d'intervalles de la forme \( (a_k,b_k)\) avec \( k=1,\ldots, n\) (ici nous mettons des parenthèses au lieu de crochets parce qu'a priori nous ne savons pas). Alors
    \begin{equation}
        A^c=\bigcup_{k=0}^{k}(b_k,a_{k+1})
    \end{equation}
    où nous avons posé \( b_0=-\infty\) et \( a_{n+1}=+\infty\). Ici encore les parenthèses sont soit fermées soit ouvertes en fonction de ce qu'étaient celles dans la décomposition de \( A\). Quoi qu'il en soit, cette décomposition de \( A^c\) montre que \( A^c\in\tribA_{\mS}\).

    Enfin si \( A,B\in\tribA_{\mS}\) alors \( A\cup B\in\tribA_{\mS}\).
\end{proof}

\begin{lemma}
    Tout élément de \( \tribA_{\mS}\) admet une décomposition minimale unique en réunion finie d'intervalles. Cette décomposition est formée d'intervalles deux à deux disjoints.
\end{lemma}

\begin{proof}
    Nous allons montrer que si \( A\in\tribA_{\mS}\), alors la décomposition minimale consiste en les composantes connexes de \( A\). Pour cela nous rappelons que la proposition \ref{PropInterssiConn} dit qu'une partie de \( \eR\) est connexe si et seulement si elle est un intervalle. D'abord cela nous dit immédiatement que les composantes connexes de \( A\) forment une décomposition de \( A\) en intervalles. Nous devons prouver qu'elle est minimale.

    Soit \( \{ C_k \}_{k=1,\ldots, n}\) les composantes connexes de \( A\). Aucun connexe de \( \eR\) contenu dans \( A\) ne peut intersecter plus d'un des \( C_k\), et par conséquent nous ne pouvons pas décomposer \( A\) en moins de \( n\) intervalles. 
    
    Pour l'unicité, soit \( \{ I_k \}_{k=1,\ldots, n}\) un ensemble de \( n\) intervalles tels que \( \bigcup_{k=1}^nI_k=A\). Chacun des \( I_k\) intersecte un et un seul des \( C_k\). En effet si \( x\in I_k\cap C_i\) et \( y\in I_k\cap C_j\), alors \( \mathopen[ x , y \mathclose]\subset I_k\) parce que \( I_k\) est un intervalle. Mais \( C_i\) étant le plus grand connexe contenant \( x\), \( \mathopen[ x , y \mathclose]\subset C_i\) et de la même façon, \( \mathopen[ x , y \mathclose]\subset C_j\). Par conséquent \( C_i\) et \( C_j\) sont tous deux la composante connexe de \( x\) et \( y\). Nous en déduisons que \( C_i=C_j\), c'est à dire \( i=j\).

    Par ailleurs nous avons \( I_k\cap I_l=\emptyset\) dès que \( k\neq l\) parce que sinon l'ensemble \( I_k\cap I_l\) serait connexe et la décomposition des \( \{ I_k \}_{k=1,\ldots, n} \) ne serait pas minimale : en remplaçant \( I_k\) et \( I_l\) par \( I_k\cup I_l\) on aurait eu une décomposition contenant moins d'éléments. Donc à renumérotation près nous pouvons supposer que \( I_k\) intersecte \( C_l\) si et seulement si \( k=l\).

    Dans ce cas nous devons avoir \( I_k=C_k\), sinon les éléments de \( C_k\setminus I_k\) ne seraient pas dans \( \bigcup_{i=1}^nI_i\).
\end{proof}

\begin{definition}[longueur d'intervalle\cite{MesureLebesgueLi}]
    Si \( I\) est un intervalle d'extrémités \( a\) et \( b\) avec \( -\infty\leq a\leq b\leq +\infty\) alors nous définissons la \defe{longueur}{longueur!d'un intervalle}\index{intervalle!longueur} de \( I\) par
    \begin{equation}
        \ell(I)=\begin{cases}
            b-a    &   \text{si } -\infty<a\leq b< +\infty\\
            \infty    &    \text{si } a\text{ ou } b\text{ est infini}
        \end{cases}
    \end{equation}
    Si \( A\in\tribA_{\mS}\) et si sa décomposition minimale est \( A=\bigcup_{k=1}^nI_k\), alors on définit
    \begin{equation}
        \ell(A)=\sum_{k=1}^n\ell(I_k).
    \end{equation}
\end{definition}

Le lemme suivant nous indique que nous pouvons calculer la longueur d'un élément de \( \tribA_{\mS}\) sans savoir la décomposition minimale, pourvu que l'on connaisse une décomposition disjointe.
\begin{lemma}[\cite{MesureLebesgueLi}]\label{LemIUQooEzHun}
    Si
    \begin{equation}
        B=\bigcup_{r=1}^pJ_r
    \end{equation}
    est une décomposition de \( B\in\tribA_{\mS}\) en intervalles deux à deux disjoints alors
    \begin{equation}
        \ell(B)=\sum_{r=1}^p\ell(J_r).
    \end{equation}
\end{lemma}

\begin{proof}
    Nous prouvons dans un premier temps le résultat dans le cas où \( B=I\) est un intervalle. Soit \( I\) un intervalle et une décomposition en intervalles disjoints \( I=\bigcup_{r=1}^pJ_r\). Nous montrons qu'alors \( \ell(I)=\sum_{r=1}^p\ell(J_r)\). Nous verrons ensuite comment passer au cas où \( B\) est un élément générique de \( \tribA_{\mS}\).
    \begin{subproof}
    \item[Si \( B=I\) est un intervalle infini]

        Si \( I\) est infini alors un des \( J_r\) soit l'être et donc \( \sum_{r=1}^p\ell(J_r)=\infty=\ell(I)\).
    \item[Si \( B=I\) est un intervalle ininfini]

    Pour chaque \( r=1,\ldots, p\) nous notons \( a_r\) et \( b_r\) les extrémités de \( J_r\). Vu que les \( J_r\) sont connexes et disjoints, si \( a_k\leq a_l\) alors \( b_k\leq a_l\), sinon l'ensemble (non vide) \( \mathopen] a_l , b_k \mathclose[ \) serait dans l'intersection \( I_k\cap I_l\) qui, elle, est vide. Plus généralement, si \( x\in J_k\) et \( y\in J_l\) avec \( x<y\) alors pour tout \( x'\in J_k\) et tout \( y'\in J_l\) nous avons \( x'<y'\). Vu qu'il y a un nombre fini d'ensembles \( J_r\), nous pouvons les classer dans l'ordre croissant :
        \begin{equation}
            a_1\leq b_1\leq a_2\leq b_2\leq \ldots\leq b_{p-1}\leq a_p\leq b_p.
        \end{equation}
        Vu que les \( J_r\) sont disjoints et que leur union est connexe nous avons en réalité
        \begin{equation}
            a=a_1\leq b_1=a_2\leq b_2=a_3\leq\ldots\leq b_{p-1}= a_p\leq b_p,
        \end{equation}
        donc une somme télescopique donne
        \begin{equation}
            \ell(I)=b-a=\sum_{r=1}^p(b_r-a_r)=\sum_{r=1}^p\ell(J_r).
        \end{equation}

    \item[Si \( B\) n'est pas un intervalle]
        Soit \( \{ I_k \}_{k=1,\ldots, n}\) la décomposition minimale de \( B\). Alors
        \begin{equation}
            \spadesuit=\ell(B)=\sum_{k=1}^n\ell(I_k)=\sum_{k=1}^n\ell\big( \bigcup_{r=1}^p(I_k\cap J_r) \big).
        \end{equation}
        Mais \( I_k\) est un intervalle et s'écrit comme union disjointe \( I_k=\bigcup_{r=1}^p(I_k\cap J_r)\), donc par la première partie
        \begin{equation}
            \spadesuit=\sum_{k=1}^n\sum_{r=1}^p\ell(I_k\cap J_r)=\sum_{r=1}^p\sum_{k=1}^n\ell(I_k\cap J_r).
        \end{equation}
        Ici \( J_r\) est un intervalle qui se décompose en \( J_r=\bigcup_{k=1}^n(I_k\cap J_r)\), donc nous pouvons encore utiliser la première partie :
        \begin{equation}
            \spadesuit=\sum_{r=1}^p\ell(J_r),
        \end{equation}
        ce qu'il fallait.
    \end{subproof}
\end{proof}

\begin{lemma}   \label{LemPIOooRLkbo}
    Si \( A,B\in\tribA_{\mS}\) avec \( A\subset B\) alors \( \ell(A)\leq \ell(B)\).
\end{lemma}

\begin{proof}
    Nous avons évidemment \( B=A\cup B\setminus A\). Notons que \( B\setminus A\in\tribA_{\mS}\) par le lemme \ref{LemBFKootqXKl}. Si \( \{ I_k \}\) est une décomposition disjointe de \( A\) et \( \{ J_i \}\) une de \( B\setminus A\) alors \( \{ I_k \}\cup\{ J_i \}\) est une décomposition disjointe de \( A\cup B\setminus A\) et le lemme \ref{LemIUQooEzHun} nous dit que
    \begin{equation}
        \ell(B)=\ell(A\cup B\setminus A)=\ell(A)+\ell(B\setminus A).
    \end{equation}
    Par conséquent \( \ell(B)\geq \ell(A)\).
\end{proof}

\begin{lemma}   \label{LemUMVooZJgMu}
    Si \( I\) est un intervalle et s'il se décompose en
    \begin{equation}
        I=\bigcup_{n\in \eN}I_n
    \end{equation}
    où les \( I_n\) sont des intervalles disjoints, alors
    \begin{equation}
        \ell(I)=\sum_{n=1}^{\infty}\ell(I_n).
    \end{equation}
\end{lemma}

\begin{proof}
    Nous allons encore diviser la preuve en deux parties suivant que \( I\) soit de longueur finie ou pas.   
    \begin{subproof}

        \item[Si \( I\) est de longueur finie]
        
            Soient \( a\) et \( b\) les extrémités de \( I\) : \( -\infty<a\leq b< +\infty\). Pour tout \( N\geq 1\) nous avons
            \begin{equation}
                \sum_{n=1}^N\ell(I_n)=\ell\big( \bigcup_{n=1}^nI_n \big)\leq \ell(I).
            \end{equation}
            La première égalité est le lemme dans le cas d'une union finie \ref{LemIUQooEzHun}. L'inégalité est le lemme \ref{LemPIOooRLkbo}. Cela étant vrai pour tout $N$, à la limite \( N\to\infty\) nous conservons l'inégalité :
            \begin{equation}
                \sum_{n=1}^{\infty}\ell(I_n)\leq \ell(I).
            \end{equation}
            Nous devons encore voir l'inégalité inverse. Pour cela nous supposons que \( a<b\). Sinon \( \ell(I)=0\) et tous les \( I_n\) doivent être vide sauf un qui contiendra seulement \( \{ a \}\) (si \( I\) le contient).

            Soit \( \epsilon>0\) avec \( \epsilon<b-a\) et l'intervalle
            \begin{equation}
                \mathopen[ a+\frac{ \epsilon }{ 4 } , b-\frac{ \epsilon }{ 4 } \mathclose]=\mathopen[ a' , b' \mathclose]\subset I.
            \end{equation}
            Si les \( a_n\) et le \( b_n\) sont le extrémités des \( I_n\) alors
            \begin{equation}
                \mathopen[ a' , b' \mathclose]\subset I=\bigcup_{n\geq 1}I_n\subset\bigcup_{n\geq 1}\mathopen] a_n-\frac{ \epsilon }{ 2^{n+2} } , b_n+\frac{ \epsilon }{ 2^{n+2} } \mathclose[=\bigcup_{n\geq 1}\mathopen] a'_n , b'_n \mathclose[
            \end{equation}
            où nous avons posé \( a'_n=a_n-\epsilon/2^{n+2}\) et \( b'_n=b_n+\epsilon/2^{n+2}\). Nous avons donc recouvert le compact\footnote{Lemme \ref{LemOACGWxV}.} \( \mathopen[ a' , b' \mathclose]\) par des ouverts. Nous pouvons donc en extraire un sous-recouvrement fini (c'est la définition de la compacité), c'est à dire une partie finie \( F\) de \( \eN\) telle que 
            \begin{equation}
                \mathopen[ a' , b' \mathclose]\subset \bigcup_{n\in F}\mathopen] a'_n , b'_n \mathclose[.
            \end{equation}
            Le lemme \ref{LemPIOooRLkbo} nous dit alors que
            \begin{equation}
                \heartsuit=b'-a'\leq \ell\big( \bigcup_{n\in F}\mathopen] a'_n , b'_n \mathclose[ \big)\leq \sum_{n\in F}(b'_n-a'_n).
            \end{equation}
            La seconde inégalité se prouve en recopiant\footnote{Nous ne pouvons pas invoquer directement le lemme \ref{LemZQUooMdCpq} parce que nous n'avons pas encore prouvé que \( \ell\) était une mesure sur \( (\eR,\tribA_{\mS})\).} la preuve de \ref{LemZQUooMdCpq}. Nous continuons le calcul :
            \begin{equation}
                \heartsuit\leq\sum_{n\in F}(b_n-a_n)+\sum_{n\in F}\frac{ \epsilon }{ 2^{n+1} }\leq \sum_{n\in F}(b_n-a_n)+\frac{ \epsilon }{2}.
            \end{equation}
            Mais \( b'-a'=(b-a)-\frac{ \epsilon }{2}\), donc
            \begin{equation}
                b-a-\frac{ \epsilon }{2}\leq \sum_{n\in F}(b_n-a_n)+\frac{ \epsilon }{2}.
            \end{equation}
            D'où nous déduisons que
            \begin{equation}
                \ell(I)=b-a\leq \sum_{n\in F}(b_n-a_n)+\epsilon\leq \sum_{n\in \eN}(b_n-a_n)+\epsilon=\sum_{n\in \eN}\ell(I_n)+\epsilon.
            \end{equation}
            Cela étant valable pour tout \( \epsilon\) nous déduisons que
            \begin{equation}
                \ell(I)\leq\sum_{n\in \eN}\ell(I_n).
            \end{equation}

        \item[Si \( I\) est de longueur infinie]

        Étant donné que \( I\) est un intervalle de longueur infinie, il contient au moins un ensemble du type \( \mathopen] -\infty , a \mathclose]\) ou \( \mathopen[ a , +\infty [\); donc  pour tout \( M>0\), il existe \( N\geq 1\) tel que
            \begin{equation}
                \ell\big( I\cap\mathopen[ -N , N \mathclose] \big)\geq M.
            \end{equation}
            Mais \( I\cap\mathopen[ -N , N \mathclose]\) est un intervalle et 
            \begin{equation}
                I\cap\mathopen[ -N , N \mathclose]=\bigcup_{n\in \eN}I_n\cap\mathopen[ -N , N \mathclose]
            \end{equation}
            qui est une union disjointe. Par conséquent,
            \begin{equation}
                M\leq \ell\big( I\cap\mathopen[ -N , N \mathclose] \big)=\sum_n\ell\big( I_n\cap\mathopen[ -N , N \mathclose] \big)\leq\sum_n\ell(I_n).
            \end{equation}
            Cela étant vrai pour tout \( M>0\), nous concluons que
            \begin{equation}
                \sum_{n\in \eN}\ell(I_n)=\infty.
            \end{equation}
    \end{subproof}
\end{proof}

\begin{remark}
    Pour la preuve de \ref{LemUMVooZJgMu} nous ne pouvons pas classer les \( I_n\) en ordre croissant comme nous l'avons fait dans la preuve de \ref{LemIUQooEzHun}. En effet si \( I=\mathopen[ 0 , 1 \mathclose]\) et que nous recouvrons \( \mathopen[ 0 , \frac{ 1 }{2} [\) et \( \mathopen] \frac{ 1 }{2} , 1 \mathclose]\) par une infinité d'intervalles chacun, nous ne pouvons plus les classer par ordre croissant.
\end{remark}

\begin{proposition}[\cite{MesureLebesgueLi}]     \label{PropULFoodgXrR}
    La fonction \( \ell\) ainsi définie est une mesure \( \sigma\)-finie sur l'algèbre de parties \( \tribA_{\mS}\).
\end{proposition}

\begin{proof}
    Le fait que \( \ell\) soit \( \sigma\)-finie provient par exemple du fait que \( \ell\big( \mathopen] -n , n \mathclose[ \big)=2n\) tandis que \( \bigcup_n\mathopen] -n , n \mathclose[=\eR\).

        Nous devons à présent prouver que \( \ell\) est additive. Soient \( (A_i)_{i\in \eN}\) des éléments disjoints de \( \tribA_{\mS}\), avec leurs décomposition minimales
            \begin{equation}
                A_i=\bigcup_{k=1}^nI^{(i)}_k.
            \end{equation}
            Pour chaque \( i\in \eN\), le lemme \ref{LemUMVooZJgMu} nous indique que
            \begin{equation}
                \ell(A_i)=\sum_{k\in \eN}\ell(I^{(i)}_k).
            \end{equation}
            L'ensemble \( \eN\times \eN\) est dénombrable et nous pouvons considérer la décomposition 
            \begin{equation}
                \bigcup_{i\in \eN}A_i=\bigcup_{(i,k)\in \eN\times \eN}I^{(i)}_k.
            \end{equation}
            Cette décomposition n'est pas spécialement minimale\footnote{\( A_1\) pourrait contenir \( \mathopen[ 0 , 1 \mathclose]\) et \( A_2\) contenir \( \mathopen] 1 , 2 \mathclose]\).} mais elle est disjointe.
            Le lemme \ref{LemUMVooZJgMu} donne
            \begin{equation}
                \ell(\bigcup_i A_i)=\sum_{(i,k)\in \eN\times \eN}\ell(I_k^{(i)})=\sum_{i\in \eN}\left( \sum_{k\in \eN}\ell(I^{(i)}_k)\right)=\sum_{i\in \eN}\ell(A_i).
            \end{equation}
            La décomposition de la somme sur \( \eN^2\) en deux sommes sur \( \eN\) est faite en vertu de la proposition \ref{PropVQCooYiWTs}.
            
\end{proof}

%--------------------------------------------------------------------------------------------------------------------------- 
\subsection{Mesure et tribu de Lebesgue}
%---------------------------------------------------------------------------------------------------------------------------

\begin{theorem} \label{ThoDESooEyDOe}
    Il existe une unique mesure \( \lambda\) sur \( \big( \eR,\Borelien(\eR) \big)\) telle que
    \begin{equation}
        \lambda\big( \mathopen] a , b \mathclose[ \big)=b-a
    \end{equation}
    pour tout \( a\leq b\) dans \( \eR\).
\end{theorem}

\begin{proof}
    
    L'existence provient du théorème de prolongement de Hahn \ref{ThoLCQoojiFfZ} : la mesure \( \ell\) sur \( (\tribA_{\mS})\) se prolonge à \( \sigma(\tribA_{\mS})=\Borelien(\eR)\).

    Nous ne pouvons pas prouver l'unicité en invoquant la partie unicité de Hahn (c'est tentant parce que \( \ell\) est \( \sigma\)-finie) parce que dans ce théorème nous ne fixons la valeur de \( \lambda\) que sur une toute petite partie de \( \tribA_{\mS}\). Nous allons cependant voir que cette petite partie suffit à garantir l'unicité.

    La classe 
    \begin{equation}
        \tribD=\{ \mathopen] a , b \mathclose[\tq -\infty<a\leq b< +\infty \}
    \end{equation}
    est stable par intersection finie et engendre la tribu borélienne. En effet \( \tribD\) contient toutes les boules et donc une base dénombrable de la topologie de \( \eR\) (proposition \ref{PropNBSooraAFr}). Donc tous les ouverts de \( \eR\) sont dans \( \sigma(\tribD)\) et \( \sigma(\tribD)=\Borelien(\eR)\). Nous pouvons donc dire grâce au théorème \ref{ThoJDYlsXu} qu'il y a unicité de la mesure sur \( \Borelien(\eR)\) lorsque les valeurs sur \( \tribD\) sont fixées.
\end{proof}

\begin{definition}      \label{DefooYZSQooSOcyYN}
    La mesure de l'espace mesuré \( \big( \eR,\Borelien(\eR),\lambda \big)\) donné par le théorème \ref{ThoDESooEyDOe} est la \defe{mesure de Lebesgue}{mesure!de Lebesgue} sur \( \big( \eR,\Borelien(\eR) \big)\).

    Nous définissons aussi la \defe{tribu de Lebesgue}{tribu!de Lebesgue} par la proposition \ref{PropIIHooAIbfj} : \( \big( \eR,\Lebesgue(\eR),\lambda \big)\) est l'espace mesuré complété de \( \big( \eR,\Borelien(\eR), \lambda \big)\).
\end{definition}


\begin{remark}
    Il n'est pas évident que la tribu de Lebesgue soit plus grande que celle des boréliens, ni que la tribu des parties soit plus grande que celle de Lebesgue. Nous mentionnons cependant les faits suivants.
    %TODO : donner des exemples
    \begin{enumerate}
        \item
            Il existe des ensembles mesurables non-boréliens, et cela ne nécessite pas l'axiome du choix. Un argument classique de cardinalité est donné dans \cite{SFYoobgQUp}. La construction la plus explicite que j'aie trouvée est dans \cite{XSHoosgoQa}, mais ça a l'air de demander des connaissances précises sur les ordinaux.
        \item
            Vu que l'ensemble de Cantor \( C\) est mesurable de mesure nulle, tout sous-ensemble de Cantor est mesurable de mesure nulle parce que la tribu de Lebesgue est complète par définition. Le cardinal de \( \partP(C)\) est strictement supérieur à la puissance du continu, alors que le cardinal de l'ensemble des boréliens est au plus égal à la puissance du continu. Donc il existe des non boréliens contenus dans Cantor; de tels non boréliens sont alors mesurables au sens de Lebesgue.

        \item
            Si nous admettons l'axiome du choix alors il existe des ensembles non mesurables au sens de Lebesgue. Nous en verrons un dans l'exemple \ref{EXooCZCFooRPgKjj}.
    \end{enumerate}
\end{remark}

\begin{example}[Un ouvert contenant tous les rationnel et de mesure arbitrairement petite]
    Il est possible de construire un ouvert de $\eR$ contenant \( \eQ\) et de mesure de Lebesgue plus petite que \( \epsilon\). Pour cela si \( (q_i)\) est une énumération des rationnels, il suffit de prendre
    \begin{equation}
        \mO=\bigcup_{n=1}^{\infty}B(q_n,\frac{ \epsilon }{ 2^{n+1} }).
    \end{equation}
    Cela est un ouvert comme union d'ouverts, ça contient tous les rationnels, et sa mesure se majore. En effet le théorème \ref{ThoDESooEyDOe} donne \( \lambda\big( B(q_n,\frac{\epsilon }{ 2^n }) \big)=\frac{ \epsilon }{ 2^n }\). Vu que ces boules ne sont a priori pas disjointes, le lemme \ref{LemPMprYuC} donne 
    \begin{equation}
        \lambda(\mO) \leq \sum_{n=1}^{\infty}\frac{ \epsilon }{ 2^n }=\epsilon
    \end{equation}
    par \eqref{EqPZOWooMdSRvY} avec \( q=\frac{ 1 }{2}\).

    Par complémentarité, nous pouvons construire un ensemble fermé de mesure non nulle et ne contenant aucun rationnel. Et même un fermé dans \( \mathopen[ 0 , 1 \mathclose]\), de mesure \( 1-\epsilon\) ne contenant aucun rationnel. 
    
    Cela peut surprendre parce qu'il existe des tonnes de suites d'irrationnels qui convergent vers des rationnels\footnote{Si \( q\in \eQ\) et \( r\in \eR\setminus \eQ\) alors la suite \( (q+r/10^k)_k\) est une suite d'irrationnels convergente vers le rationnel \( q\)}, et il semble difficile de créer un ensemble contenant beaucoup d'irrationnels tout en préservant la propriété de fermeture vis à vis des suites convergentes.
\end{example}

%--------------------------------------------------------------------------------------------------------------------------- 
\subsection{Propriétés de la mesure de Lebesgue}
%---------------------------------------------------------------------------------------------------------------------------

\begin{proposition}
    Tout ensemble dénombrable de \( \eR\) est mesurable de mesure nulle.
\end{proposition}

\begin{proof}
    Un point de \( \eR\) est un intervalle de mesure nulle. Si \( D\) est dénombrable, il est union disjointes et dénombrable de points. Le lemme \ref{LemUMVooZJgMu} nous dit alors que sa mesure est \( \lambda(D)=\sum_{i=1}^{\infty}\lambda(\{ a_i \})=0\).
\end{proof}

\begin{remark}
    Il existe cependant des ensembles non dénombrables et tout de même de mesure nulle. Par exemple l'ensemble de Cantor (voir la proposition \ref{PropBEWooXZdKN}).
\end{remark}


\begin{proposition}     \label{PropooOACLooLMIUuY}
    La mesure de Lebesgue est invariante par translation, c'est à dire que si \( A\) est mesurable alors \( \lambda(A)=\lambda(A+\alpha)\) pour tout réel \( \alpha\).
\end{proposition}

\begin{proof}
    Nous commençons par les intervalles ouverts :
    \begin{equation}
    \lambda\big( \mathopen] a , b \mathclose[+\alpha \big)=\lambda\big( \mathopen] a+\alpha , b+\alpha \mathclose[ \big)=(b+\alpha)-(a+\alpha)=b-a=\lambda\big( \mathopen] a , b \mathclose[ \big).
    \end{equation}
    D'après ce qui est dit dans l'exemple \ref{ExDMPoohtNAj}, la mesure de Lebesgue sur les boréliens est invariante par translation.

    Si \( A\) est mesurable alors il existe un borélien \( B\) et un ensemble négligeable \( N\) tels que \( A=B\cup N\) par la caractérisation \ref{EqFJIoorxZNU} de la complétion. Alors \( A+\alpha=B+\alpha\cup N+\alpha\) et \( N+\alpha\) est encore un ensemble négligeable. Donc \( \lambda(A+\alpha)=\alpha(B+\alpha)=\lambda(B)\).
\end{proof}

Le mesure \( \ell\) définie sur l'algèbre de parties \( \tribA_{\mS}\) (voir proposition \ref{PropULFoodgXrR}). La proposition \ref{PropIUOoobjfIB} nous donne donc une mesure extérieure par
\begin{equation}    \label{EqJGXoogdKqb}
    \lambda^*(X)=\inf\{ \sum_n\ell(A_n);A_n\in\tribA_{\mS},X\subset\bigcup_nA_n \}.
\end{equation}

La proposition suivante montre que cette mesure extérieure peut être exprimée seulement avec des intervalles ouverts.
\begin{proposition} \label{PropTNOooDcfwn}
    Nous avons
    \begin{equation}
        \lambda^*(X)=\inf\{ \sum_{n\geq 1}\ell(I_n); I_n\text{ sont des intervalles ouverts et }X\subset\bigcup_nI_n \}.
    \end{equation}
\end{proposition}

\begin{proof}
    Nous savons que dans la définition \eqref{EqJGXoogdKqb}, chacun des \( A_n\) est une réunion disjointe d'intervalles (pas spécialement ouverts) deux à deux disjoints; donc
    \begin{equation}
        \lambda^*(X)=\inf\{ \sum_n\ell(I_n);I_n\in\mS,X\subset\bigcup_nI_n \}.
    \end{equation}
    Soit \( \epsilon>0\). Si \( A\subset\bigcup_nI_n\), pour chaque \( n\geq 1\) nous considérons un intervalle ouvert \( J_n\) tel que \( I_n\subset J_n\) et \( \ell(I_n)+\frac{ \epsilon }{ 2^n }\leq \ell(J_n)\). Faisant cela pour chacun des découpages de \( X\) en intervalles nous trouvons
    \begin{equation}
        \lambda^*(X)\leq \inf\{ \sum_n\ell(J_n)\text{ } J_n\text{ est ouvert et }X\subset\bigcup_nJ_n \}+\epsilon.
    \end{equation}
    Étant donné que \( \epsilon\) est arbitraire nous avons l'égalité.
\end{proof}

\begin{proposition}[\cite{MesureLebesgueLi}]    \label{PropMXIoojpKvd}
    Si \( X\subset \eR\) est tel que \( \lambda^*(X)<\infty\) alors
    \begin{enumerate}
        \item   \label{ItemGJUoozrDILi}
            Pour tout \( \epsilon>0\) il existe un ouvert \( \Omega_{\epsilon}\) tel que
            \begin{subequations}
                \begin{numcases}{}
                    X\subset\Omega_{\epsilon}\\
                    \lambda(\Omega_{\epsilon})\leq \lambda^*(X)+\epsilon.
                \end{numcases}
            \end{subequations}
        \item   \label{ItemGJUoozrDILii}
            Il existe une intersection dénombrable d'ouverts \( G\) telle que
            \begin{subequations}
                \begin{numcases}{}
                    X\subset G\\
                    \lambda(G)=\lambda^*(X).
                \end{numcases}
            \end{subequations}
    \end{enumerate}
\end{proposition}

\begin{proof}
    Pour \ref{ItemGJUoozrDILi}, la proposition \ref{PropTNOooDcfwn} nous a déjà dit que
    \begin{equation}
        \lambda^*(X)=\inf\{ \sum_n\ell(I_n)\text{ } I_n\text{ est un intervalle ouvert}, X\subset\bigcup_nI_n \},
    \end{equation}
    donc si \( \epsilon>0\), il existe des intervalles ouverts \( I_n\) tels que 
    \begin{subequations}
        \begin{numcases}{}
            X\subset\bigcup_nI_n\\
            \sum_n\ell(I_n)\leq \lambda^*(X)+\epsilon.
        \end{numcases}
    \end{subequations}
    Si nous posons \( \Omega_{\epsilon}=\bigcup_nI_n\), alors nous avons bien
    \begin{subequations}
        \begin{numcases}{}
            X\subset\Omega_{\epsilon}\\
            \lambda(\Omega_{\epsilon})\leq\sum_n\ell(I_n)\leq \lambda^*(X)+\epsilon.
        \end{numcases}
    \end{subequations}

    En ce qui concerne \ref{ItemGJUoozrDILii}, pour chaque \( k\geq 1\) nous considérons l'ensemble \( \Omega_{1/k}\) obtenu comme précédemment avec \( \epsilon=1/k\) et nous posons \( G=\bigcap_{k\geq 1}\Omega_{1/k}\). Cela est une intersection dénombrable d'ouverts vérifiant \( X\subset G\) (parce que \( X\subset \Omega_{1/k}\) pour tout \( k\)) et donc \( \lambda^*(X)\leq\lambda^*(G)=\lambda(G)\). De plus pour tout \( k\) nous avons 
    \begin{equation}
        \lambda(G)\leq(\Omega_{1/k})\leq \lambda^*(X)+\frac{1}{ k }
    \end{equation}
    pour tout \( k\). En faisant \( k\to \infty\) nous avons
    \begin{equation}
        \lambda(G)\leq \lambda^*(X).
    \end{equation}
    Au final
    \begin{equation}
        \lambda(G)\leq \lambda^*(X)\leq \lambda(G),
    \end{equation}
    d'où l'égalité.
\end{proof}

\begin{corollary}
    Une partir \( N\subset \eR\) est négligeable\footnote{Définition \ref{DefAVDoomkuXi}.} si et seulement si \( \lambda^*(N)=0\).
\end{corollary}

\begin{proof}
    Nous savons que si \( N\) est négligeable il existe un borélien \( Y\) tel que \( N\subset Y\) avec \( \lambda(Y)=0\). Par conséquent\footnote{Au péril d'être lourd nous rappelons que \( \lambda^*\) est défini sur toutes les parties de \( \eR\).}
    \begin{equation}
        \lambda^*(N)\leq \lambda^*(Y)=\lambda(Y)=0.
    \end{equation}
    
    Pour l'implication inverse nous supposons que \( \lambda^*(N)=0\) et nous prenons l'ensemble \( G\) définit par la proposition \ref{PropMXIoojpKvd}\ref{ItemGJUoozrDILii} : c'est un borélien contenant \( N\) et tel que \( \lambda(G)=\lambda^*(N)=0\). L'ensemble \( N\) est donc négligeable.
\end{proof}

\begin{theorem}[Régularité extérieure de la mesure de Lebesgue] \label{ThoHFXooONFRN}
    Pour tout mesurable \( A\subset \eR\) nous avons
    \begin{equation}
        \lambda(A)=\inf\{ \lambda(\Omega); \Omega\text{ ouvert contenant } A \}.
    \end{equation}
\end{theorem}
\index{régularité!extérieure de la mesure de Lebesgue}

\begin{proof}
    Nous commençons par le cas où \( B\) est un borélien.
    \begin{subproof}

        \item[Si \( B\) borélien, \( \lambda(B)<\infty\)]
        
        Soit \( \epsilon>0\); par la proposition \ref{PropMXIoojpKvd}\ref{ItemGJUoozrDILi} il existe un ouvert \( \Omega_{\epsilon}\) contenant \( B\) tel que \( \lambda(\Omega_{\epsilon})\leq \lambda^*(B)+\epsilon\). Vu qu'ici \( B\) est borélien, \( \lambda^*(B)=\lambda(B)\) et nous concluons que pour tout \( \epsilon\) il existe un ouvert \( \Omega_{\epsilon}\) tel que
        \begin{subequations}
            \begin{numcases}{}
                B\subset\Omega_{\epsilon}\\
                \lambda(\Omega_{\epsilon})\leq \lambda(B)+\epsilon,
            \end{numcases}
        \end{subequations}
        et donc
        \begin{equation}
            \lambda(B)=\inf\{ \lambda(\Omega);\text{ } \Omega\text{ ouvert contenant } B\text{ } \}.
        \end{equation}
        
        \item[Si \( B\) borélien, \( \lambda(B)=+\infty\)]

            Dans ce cas l'infimum est pris uniquement sur des ouverts \( \Omega\) tels que \( \lambda(\Omega)=\infty\).

        \item[Si \( A\) est mesurable non borélien]
    
    Nous passons maintenant au cas où \( A \) est mesurable sans être borélien. Il s'écrit donc \( A=B\cup N\) avec \( B\) borélien et \( N\) négligeable par la proposition \ref{thoCRMootPojn}, et par définition \( \lambda(A)=\lambda(B)\). Si \( Y\) est un borélien tel que \( N\subset Y\) et \( \lambda(Y)=0\) alors
    \begin{subequations}
        \begin{align}
            \lambda(A)=\lambda(B)&=\inf\{ \lambda(\Omega)\tq \text{ } \Omega\text{ ouvert}, B\subset\Omega \}\label{subeqMTHoopkSKOi}\\
            &\leq\inf\{ \lambda(\Omega)\tq \text{ } \Omega\text{ ouvert}, B\cup N\subset\Omega \}  \label{subeqMTHoopkSKOii}\\
            &\leq\inf_{\Omega',Y'}\{ \lambda(\Omega'\cup Y')\tq \text{ } \Omega'\text{, } Y'\text{ ouverts}, B\subset\Omega', Y\subset Y' \}\label{subeqMTHoopkSKOiii}\\
            &\leq\inf_{\Omega',Y'}\{ \lambda(\Omega')+\lambda(Y')\tq \text{ } \Omega'\text{, } Y'\text{ ouverts},  B\subset\Omega',Y\subset Y' \}\label{subeqMTHoopkSKOiv}\\
            &\leq\inf_{\Omega'}\{ \lambda(\Omega')\tq \text{ } \Omega'\text{ ouvert},  B\subset\Omega \}\label{subeqMTHoopkSKOv}\\
            &=\lambda(B).
        \end{align}
    \end{subequations}
    Justifications :
    \begin{itemize}
        \item \eqref{subeqMTHoopkSKOi} Le cas borélien déjà fait.
        \item \eqref{subeqMTHoopkSKOii} Les ouverts \( \Omega\) tels que \( B\cup N\subset \Omega\) vérifient a fortiori \( B\subset \Omega\); nous avons donc agrandit l'ensemble sur lequel l'infimum est pris.
        \item \eqref{subeqMTHoopkSKOiii} Parmi les ouverts \( \Omega\) qui recouvrent \( B\cup N\), il y a ceux de la forme \( \Omega'\cup Y'\) où \( \Omega'\) recouvre \( B\) et \( Y'\) est un ouvert contenant \( Y\). Donc nous avons rétréci l'ensemble sur lequel l'infimum est pris et par conséquent agrandit l'infimum.
        \item \eqref{subeqMTHoopkSKOiv} Mesure d'une union majorée par la somme des mesures.
        \item \eqref{subeqMTHoopkSKOv} Vu que \( Y\) est borélien, \( \lambda(Y)=\inf_{ Y'\text{ ouvert}}\{ \lambda(Y')\tq Y\subset Y' \}=0\). Donc pour tout \( \Omega'\) et tout \( \epsilon>0\), nous pouvons trouver un \( Y'\) vérifiant les conditions tel que \( \lambda(\Omega')+\lambda(Y')\leq \lambda(\Omega')+\epsilon\).
    \end{itemize}
    Toutes les inégalités sont des égalités en en particulier \eqref{subeqMTHoopkSKOii} donne
    \begin{equation}
        \lambda(A)=\inf\{ \lambda(\Omega)\tq \text{ } \Omega\text{ ouvert}, B\cup N\subset\Omega \},
    \end{equation}
    ce qu'il fallait.
    \end{subproof}
    
\end{proof}

\begin{proposition}[\cite{MesureLebesgueLi}]    \label{PropEZNoofLkVb}
    Si \( A\) est mesurable dans \( \eR\) et si \( \epsilon>0\) alors il existe un ouvert \( \Omega_{\epsilon}\) et un fermé \( F_{\epsilon}\) tels que
    \begin{subequations}    \label{subeqHNEooaNqDu}
        \begin{numcases}{}
            F_{\epsilon}\subset A\subset \Omega_{\epsilon}\\
            \lambda(\Omega_{\epsilon}\setminus F_{\epsilon})\leq \epsilon.
        \end{numcases}
    \end{subequations}
\end{proposition}

\begin{proof}
    Nous commençons par le cas d'un borélien \( B\).
    \begin{subproof}
        \item[Première étape]
        
            Montrons qu'il existe un ouvert \( U_{\epsilon}\) tel que
            \begin{subequations}
                \begin{numcases}{}
                    B\subset U_{\epsilon}\\
                    \lambda(U_{\epsilon}\setminus B)\leq \frac{ \epsilon }{2}.
                \end{numcases}
            \end{subequations}
            Si \( \lambda(B)<\infty\) alors le théorème \ref{ThoHFXooONFRN} nous donne un ouvert \( U_{\epsilon}\) tel que \( B\subset U_{\epsilon}\) et \( \lambda(U_{\epsilon})\leq \lambda(B)+\frac{ \epsilon }{2}\). Nous avons alors
            \begin{equation}
                \lambda(\Omega_{\epsilon}\setminus B)=\lambda(\Omega_{\epsilon})-\lambda(B)\leq \frac{ \epsilon }{2}.
            \end{equation}
            Si par contre \( \lambda(B)=\infty\), nous posons \( B_n=B\cap\mathopen[ -n , n \mathclose]\) et \( \epsilon_n=\epsilon/2^{n+1}\). Pour chaque \( n\) nous avons un ouvert \( \Omega_n\) tel que
            \begin{subequations}
                \begin{numcases}{}
                    B_n\subset \Omega_n\\
                    \lambda(\Omega_n\setminus B_n)\leq \frac{ \epsilon }{ 2^{n+1} }
                \end{numcases}
            \end{subequations}
            Par conséquent en posant \( \Omega=\bigcup_{n\geq 1}\Omega_n\) nous avons\footnote{Nous utilisons la petite relation ensembliste \( \big( \bigcup_nA_n \big)\setminus\big( \bigcup_nB_n \big)\subset \bigcup_n(A_n\setminus B_n)\).}
            \begin{subequations}
                \begin{numcases}{}
                    B\subset \Omega\\
                    \lambda(\Omega\setminus B)\leq \lambda\big( \bigcup_n(\Omega_n\setminus B_n) \big)\leq \sum_{n\geq 1}\lambda(\Omega_n\setminus B_n)=\frac{ \epsilon }{2}.
                \end{numcases}
            \end{subequations}
            La première étape est terminée.

        \item[Deuxième étape]

            Nous prouvons à présent qu'il existe un ouvert \( \Omega_{\epsilon}\) et un fermé \( F_{\epsilon}\) tels que
            \begin{subequations}
                \begin{numcases}{}
                    F_{\epsilon}\subset B\subset \Omega_{\epsilon}\\
                    \lambda(\Omega_{\epsilon}\setminus B)\leq \frac{ \epsilon }{2}\\
                    \lambda(B\setminus F_{\epsilon})\leq \frac{ \epsilon }{2}.
                \end{numcases}
            \end{subequations}
            L'ouvert \( \Omega_{\epsilon}\), nous l'avons déjà de l'étape précédente. Pour le fermé, nous appliquons la première étape au borélien \( B^c\); ce qui nous trouvons est un ouvert \( G_{\epsilon}\) tel que
            \begin{subequations}
                \begin{numcases}{}
                    B^c\subset G_{\epsilon}\\
                    \lambda(G_{\epsilon}\setminus B^c)\leq \frac{ \epsilon }{2}.
                \end{numcases}
            \end{subequations}
            En posant \( F_{\epsilon}=G_{\epsilon}^c\) nous avons un fermé tel que \( F_{\epsilon}\subset B\) et
            \begin{equation}
                \lambda(B\setminus F_{\epsilon})=\lambda(F_{\epsilon}^c\setminus B^c)=\lambda(G_{\epsilon}\setminus B^c)\leq \frac{ \epsilon }{2}.
            \end{equation}
            
        \item[Dernière étape]

            Les ensembles \( F_{\epsilon}\) et \( \Omega_{\epsilon}\) trouvés à la deuxième étape donnent bien les relations \eqref{subeqHNEooaNqDu}. En effet \( \Omega_{\epsilon}\setminus F_{\epsilon}=(\Omega_{\epsilon}\setminus B)\cup(B\setminus F_{\epsilon})\), donc
            \begin{equation}
                \lambda(\Omega_{\epsilon}\setminus F_{\epsilon})\leq \lambda(\Omega_{\epsilon}\setminus B)+\lambda(B\setminus F_{\epsilon})=\epsilon.
            \end{equation}
    \end{subproof}
    Nous passons au cas où \( A=B\cup N\) est mesurable. Nous commençons par prendre les \( \Omega_{\epsilon}\) et \( F_{\epsilon}\) qui correspondent à \( B\) :
    \begin{subequations}
        \begin{numcases}{}
            F_{\epsilon}\subset B\subset \Omega_{\epsilon}\\
            \lambda(\Omega_{\epsilon}\setminus F_{\epsilon})\leq \epsilon.
        \end{numcases}
    \end{subequations}
    Soit \( Y\) un borélien tel que \( N\subset Y\) et \( \lambda(Y)\) puis un ouvert \( Y'\) tel que \( \lambda(Y')\leq \epsilon\) et \( Y\subset Y'\). L'existence d'un tel \( Y'\) est assurée par la proposition \ref{ThoHFXooONFRN} appliquée à \( Y\). Nous vérifions que les ensembles \( F_{\epsilon}\) et \( \Omega_{\epsilon}\cup Y'\) fonctionnent. En effet \( \Omega_{\epsilon}\cup Y'\setminus F_{\epsilon}\subset (\Omega_{\epsilon}\setminus F_{\epsilon})\cup Y'\), donc
    \begin{subequations}
        \begin{numcases}{}
            F_{\epsilon}\subset B\cup N\subset \Omega_{\epsilon}\cup Y'\\
            \lambda\big( (\Omega_{\epsilon}\setminus F_{\epsilon}) \big)\leq \lambda(\Omega_{\epsilon}\setminus F_{\epsilon})+\lambda(Y')\leq 2\epsilon.
        \end{numcases}
    \end{subequations}
    Donc en réalité il faut choisir \( \Omega_{\epsilon/2}\), \( F_{\epsilon/2}\) et \( \lambda(Y')\leq \epsilon/2\).
\end{proof}

\begin{theorem}[Régularité intérieure de la mesure de Lebesgue]
    Si \( A\) est mesurable dans \( \eR\) alors
    \begin{equation}
        \lambda(A)=\sup\{ \lambda(K);  K\text{ compact contenu dans } A \}.
    \end{equation}
\end{theorem}
\index{régularité!intérieure de la mesure de Lebesgue}

\begin{proof}
    Par la proposition \ref{PropEZNoofLkVb} nous avons
    \begin{equation}    \label{EqTPEooUHTbH}
        \lambda(A)=\sup_{ F\text{ fermé dans } A}\lambda(F).
    \end{equation}
    Pour un tel \( F\) nous posons \( K_n=F\cap\mathopen[ -n , n \mathclose]\) qui est compact\footnote{parce que fermé et borné, théorème de Borel-Lebesgue \ref{ThoXTEooxFmdI}.} et contenu dans \( B\). De plus le lemme \ref{LemAZGByEs}\ref{ItemJWUooRXNPcii} nous dit que
    \begin{equation}
        \lambda(F)=\lim_{n\to \infty} \lambda(K_n)
    \end{equation}
    Donc tous les \( \lambda(F)\) peuvent être arbitrairement approchés par un \( \lambda(K)\) avec \( K\) compact dans \( A\), et le supremum \eqref{EqTPEooUHTbH} n'est pas affecté en nous restreignant à prendre des compacts contenus dans \( B\) :
    \begin{equation}    
        \lambda(A)=\sup_{ F\text{ fermé dans } A}\lambda(F)=\sup_{ K\text{ compact dans } A}\lambda(K).
    \end{equation}
\end{proof}

%--------------------------------------------------------------------------------------------------------------------------- 
\subsection{Ensemble de Cantor}
%---------------------------------------------------------------------------------------------------------------------------

Nous considérons la fonction donnant l'écriture décimale des nombres définie en \eqref{EqXXXooOTsCK}.

\begin{definition}[Ensemble de Cantor]  \label{DefIYDooVIDJs}
    Soit \( K_0=\mathopen[ 0 , 1 [\) et les ensembles \( K_n\) définis par la récurrence
        \begin{equation}
            K_{n+1}=\big( \frac{1}{ 3 }K_n \big)\cup\big( \frac{1}{ 3 }(K_n+2) \big).
        \end{equation}
        L'ensemble
        \begin{equation}
            K=\bigcup_{n\geq 0}K_n
        \end{equation}
        est l'\defe{ensemble triadique de Cantor}{Cantor!ensemble}\index{ensemble!de Cantor}.
\end{definition}
Les principales propriétés de l'ensemble de Cantor sont qu'il est non dénombrable (proposition \ref{PropTPPooDySbm}) et borélien de mesure nulle (proposition \ref{PropBEWooXZdKN}).

\begin{normaltext}
    L'idée de base pour prouver que l'ensemble \( K\) est non dénombrable est que ses éléments sont les nombres qui s'écrivent en base \( 3\) sans utiliser le chiffre \( 1\). En prenant un nombre sans \( 1\) écrit en base \( 3\), en changeant tous les \( 2\) en \( 1\) et en lisant le résultat en base \( 2\), nous obtenons tous les nombres possibles en base \( 2\) et donc une quantité non dénombrable. L'idée est donc simple et astucieuse. La mise en musique est un peu plus délicate parce qu'il faut faire attention aux queues de suites; c'est pour cela que nous avons construit l'ensemble de Cantor en partant de \( \mathopen[ 0 , 1 [\) et non de \( \mathopen[ 0 , 1 \mathclose]\).
\end{normaltext}

Le lemme suivant dit précisément ce que nous entendons en disant que les éléments de l'ensemble de Cantor sont les nombres qui s'écrivent en base \( 3\) sans utiliser le chiffre \( 1\).
\begin{lemma}[\cite{MonCerveau}]   \label{LemAZGoosKzEm}
    Soit \( n\in \eN\) et \( x\in \eD_3\); nous avons \( \varphi_3(x)\in K_n\in\) si et seulement si \( x_1,\ldots, x_n\in\{ 0,2 \}\).
\end{lemma}

\begin{proof}
    Nous procédons par récurrence en commençant avec \( n=1\). Si \( x_1=1\) alors
    \begin{equation}
        \varphi_3(x)=\frac{1}{ 3 }+\sum_{k=2}^{\infty}\frac{ x_k }{ 3^k }\in\mathopen[ \frac{1}{ 3 } , \frac{ 2 }{ 3 } [.
    \end{equation}
    Notons que \( \varphi_3(x)=\frac{ 2 }{ 3 }\) est impossible parce que ça demanderait une queue de suite de \( 2\). Par conséquent \( \varphi_3(x)=\mathopen[ 0 , 1 [\setminus\mathopen[ \frac{1}{ 3 } , \frac{ 2 }{ 3 } [=K_1\).

        Nous passons à la récurrence. 
        
        \begin{subproof}
        \item[Sens direct]
        
        Nous supposons que \( x_1,\ldots, x_{n+1}\in\{ 0,2 \}\) et nous montrons que \( \varphi_3(x)\in K_{n+1}\). La chose surprenante est que nous n'allons pas considérer deux cas suivant que \( x_{n+1}\) vaut \( 0\) ou \( 1\); nous allons considérer deux cas suivant\footnote{Pour comprendre pourquoi, faire un dessin de comment \( K_n\) se transforme en \( K_{n+1}\) et remarquer dans \( K_2\), les deux premiers segments ne sont pas une division du premier segment de \( K_1\), mais bien une copie des \emph{deux} segments de \( K_1\).} que \( x_1\) vaut \( 0\) ou \( 1\). Écrivons encore \( \varphi_3(x)\) :
    \begin{equation}
        \varphi_3(x)=\sum_{k=1}^{n+1}\frac{ x_k }{ 3^k }+\sum_{k=n+2}^{\infty}\frac{ x_k }{ 3^k }.
    \end{equation}
    \begin{subproof}
        \item[Si \( x_1=0\)]
            Alors nous avons
            \begin{equation}
                3\varphi_3(x)=\sum_{k=2}^{\infty}\frac{ x_k }{ 3^{k-1} }=\sum_{k=1}^{\infty}\frac{ x_{k+1} }{ 3^k }=\varphi_3(x_2,\ldots, x_n,x_{n+1},\ldots)
            \end{equation}
            Vu que par hypothèse \( x_2,\ldots, x_{n+1}\) sont dans \( \{ 0,2 \}\) nous avons \( 3\varphi_3(x)\in K_n\) par hypothèse de récurrence. Cela implique que \( \varphi_3(x)\in K_{n+1}\).
        \item[Si \( x_1=2\)]
            Alors
            \begin{equation}
                \varphi_3(x)=\frac{ 2 }{ 3 }+\sum_{k=2}^{\infty}\frac{ x_k }{ 3^k },
            \end{equation}
            et
            \begin{equation}
                3\varphi_3(x)-2=\sum_{k=1}^{\infty}\frac{ x_{k+1} }{ 3^k }=\varphi(x_2,\ldots, x_{n+1},\ldots),
            \end{equation}
            et donc là nous avons \( 3\varphi_3(x)-2\in K_n\), ce qui implique encore \( \varphi_3(x)\in K_{n+1}\).
    \end{subproof}
    
        \item[Sens réciproque]
        
            Nous devons maintenant prouver que \( \varphi_3(x)\in K_{n+1}\) implique \( x_1,\ldots, x_{n+1}\in\{ 0,2 \}\). Par le même calcul que précédemment nous avons soit
            \begin{equation}
                3\varphi_3(x)=\varphi_3(x_2,\ldots, x_{n+1},\ldots),
            \end{equation}
            si \( x_1=0\), soit
            \begin{equation}
                3\varphi_3(x)-2=\varphi_3(x_2,\ldots, x_{n+1},\ldots),
            \end{equation}
            si \( x_1=2\). Dans les deux cas, si \( x_l=1\) pour un certain \( 2\leq l\leq n+1\), alors l'hypothèse de récurrence donne que ces éléments ne sont pas dans \( K_n\) et donc \( \varphi_3(x)\) pas dans \( K_{n+1}\).

        \end{subproof}
\end{proof}

\begin{corollary}[\cite{MonCerveau}]   \label{CorSEDooJmeXt}
    En posant \( \eE=\{ x\in\eD_3\tq x_i\neq 1\forall i \}\) nous avons \( K=\varphi_3(\eE)\). Et plus précisément, \( \varphi_3\colon \eE\to K\) est une bijection.
\end{corollary}

\begin{proof}
    Nous divisons la preuve en trois étapes.
    \begin{subproof}
    \item[Image contenue dans \( K\)]
        Si \( x\in \eE\) et \( n\in \eN\) nous avons \( x_1,\ldots, x_n\in\{ 0,2 \}\) et donc \( \varphi_3(x)\in K_n\) par la proposition \ref{LemAZGoosKzEm}. Donc
        \begin{equation}
            \varphi_3(x)\in\bigcup_{n\geq 1}K_n=K.
        \end{equation}
    \item[Injective]
        L'application \( \varphi_3\colon \eE\to K\) est injective parce qu'elle est déjà injective depuis \( \eD_3\).
    \item[Surjective]
        Soit \( p\in K\subset\mathopen[ 0 , 1 [\). Vu que \( \varphi_3\colon \eD_3\to \mathopen[ 0 , 1 [\) est surjective (théorème \ref{ThoRXBootpUpd}), il existe \( x\in \eD_3\) tel que \( \varphi_3(x)=p\). Pour tout \( n\) nous avons \( \varphi_3(x)\in K_n\) et donc \( x_1,\ldots, x_n\in\{ 0,2 \}\) et donc au final \( x\in \eE\).
    \end{subproof}
\end{proof}

\begin{proposition}[\cite{MonCerveau}]    \label{PropTPPooDySbm}
    L'ensemble de Cantor est non dénombrable.
\end{proposition}

\begin{proof}

    Nous avons prouvé à la proposition \ref{PropNNHooYTVFw} que l'ensemble \( \eD_2\) n'était pas dénombrable. Nous allons à présent prouver que l'application
    \begin{equation}
        \begin{aligned}
            \psi\colon \eD_2&\to K \\
            c&\mapsto \varphi_3(   c\text{ en remplaçant les } 1\text{ par des } 2  ) 
        \end{aligned}
    \end{equation}
    est une bijection. Le fait que \( \psi\) soit injective est une conséquence du fait que ce soit la composition de deux applications injectives (le remplacement et \( \varphi_3\)). Il faut par contre montrer que l'image est égale à \( K\), en notant qu'il n'est pas évident a priori que l'image soit contenue dans \( K\).

    L'opération qui consiste à remplacer les \( 1\) par des \( 2\) est une bijection \( \eD_2\to \eE\). Le corollaire \ref{CorSEDooJmeXt} nous dit aussi que \( \varphi_3\colon \eE\to K\) est une bijection. En tant que composée de bijections, \( \psi\) est une bijection.

    Étant en bijection avec \( \eD_2\) qui n'est pas dénombrable par la proposition \ref{PropNNHooYTVFw}, l'ensemble de Cantor n'est pas dénombrable.
\end{proof}

\begin{proposition}[Ensemble de Cantor]    \label{PropBEWooXZdKN}
    L'ensemble de Cantor\footnote{Définition \ref{DefIYDooVIDJs}} est borélien, non dénombrable et de mesure nulle.
\end{proposition}

\begin{proof}
    Nous reprenons les notations de la définition \ref{DefIYDooVIDJs}. Le fait que l'ensemble de Cantor soit non dénombrable a été prouvé dans la proposition \ref{PropTPPooDySbm}.

    L'ensemble de Cantor étant une intersection dénombrable de boréliens, il est borélien par le lemme \ref{LemBWNlKfA}. Vu que \( K_n\subset\mathopen[ 0 , 1 [\) nous avons \( \frac{1}{ 3 }K_n\leq \frac{1}{ 3 }\) et \( \frac{1}{ 3 }(K_n+2)\geq \frac{ 2 }{ 3 }\), donc \( K_n\) est une union disjointe de \( 2^n\) intervalles de mesure \( 2/3^n\). Nous avons donc
        \begin{equation}
            \lambda(K_n)=\left( \frac{ 2 }{ 3 } \right)^n.
        \end{equation}
        L'ensemble de Cantor étant contenu dans chacun des \( K_n\), sa mesure est plus petite que la mesure de chacun des \( K_n\) (lemme \ref{LemPMprYuC}) et donc \( \lambda(K)\leq \left( \frac{ 2 }{ 3 } \right)^n\) pour tout \( n\); ergo \( \lambda(K)=0\).
\end{proof}

%--------------------------------------------------------------------------------------------------------------------------- 
\subsection{Ensemble de Vitali (non mesurable)}
%---------------------------------------------------------------------------------------------------------------------------

\begin{example}[Un ensemble non mesurable au sens de Lebesgue]      \label{EXooCZCFooRPgKjj}
    Nous considérons\cite{ooIARBooPdOgAQ} 
    l'ensemble quotient \( \eR/\eQ\); chaque classe intersecte l'intervalle \( \mathopen[ 0 , 1 \mathclose]\). Grâce à l'axiome du choix (voir \ref{NORooLMBYooYjUoju}) nous pouvons construire un ensemble \( V\) contenant un représentant dans \( \mathopen[ 0 , 1 \mathclose]\) de chaque classe. Un tel ensemble est un \defe{ensemble de Vitali}{Vitali (ensemble)}. Nous allons prouver que \( V\) n'est pas mesurable.

    Supposons que \( V\) soit mesurable.Alors tous les ensemble de la forme \( V+q\) (\( q\in \eQ\)) sont mesurables et ont même mesure par la proposition \ref{PropooOACLooLMIUuY}. Nous posons
    \begin{equation}
        A=\bigcup_{\substack{q\in\eQ\\-1\leq q\leq 1}}(V+q)\subset\mathopen[ -1 , 2 \mathclose].
    \end{equation}
    Cela est une union disjointe d'ensembles mesurables. Donc
    \begin{equation}
        \lambda(A)=\sum_{\substack{q\in\eQ\\-1\leq q\leq 1}}\lambda(V+q).
    \end{equation}
    Vu que \( A\subset\mathopen[ -1 , 2 \mathclose]\) nous avons \( \lambda(A)\leq 3\) et donc tous les termes de la somme doivent être nuls. Nous avons donc \( \lambda(A)=0\).

    Prouvons toutefois que \( \mathopen[ 0 , 1 \mathclose]\subset A\), ce qui serait une contradiction. Soit \( x\in\mathopen[ 0 , 1 \mathclose]\); il est dans une des classes de \( \eR/\eQ\) et donc il existe \( v\in V\) tel que \( x-v\in \eQ\). De plus \( x,v\in \mathopen[ 0 , 1 \mathclose]\), donc
    \begin{equation}
        -1\leq x-v\leq 1.
    \end{equation}
    Cela fait que \( x\in V+(x-v)\subset A\). Nous avons donc \( x\in A\) et donc \( \mathopen[ 0 , 1 \mathclose]\subset A\). En conséquence de quoi nous aurions \( \lambda(A)\geq 1\).
\end{example}

%+++++++++++++++++++++++++++++++++++++++++++++++++++++++++++++++++++++++++++++++++++++++++++++++++++++++++++++++++++++++++++ 
\section{Tribu et mesure de Lebesgue sur \texorpdfstring{$ \eR^d$}{Rd}}
%+++++++++++++++++++++++++++++++++++++++++++++++++++++++++++++++++++++++++++++++++++++++++++++++++++++++++++++++++++++++++++

Quelques liens internes :
\begin{itemize}
    \item Le produit de tribus est donné par la définition \ref{DefTribProfGfYTuR},     % Cette référence doit être vers le haut.
    \item le produit d'espaces mesurés est donné par la définition \ref{DefUMlBCAO}.     % Cette référence doit être vers le haut.
\end{itemize}

La mesure de Lebesgue sur \( \eR^d\), notée \( \lambda_N\) est d'abord la mesure définie sur
\begin{equation}
 \big( \eR^d,\Borelien(\eR)\otimes\ldots\otimes\Borelien(\eR) \big)
\end{equation}
comme le produit \( \lambda\otimes\ldots\otimes \lambda\). Ensuite nous nous souvenons du corollaire \ref{CorWOOOooHcoEEF} qui donne \( \lambda_N\) comme une mesure sur
\begin{equation}
 \big( \eR^N,\Borelien(\eR^N) \big).
\end{equation}
Et enfin \( \lambda_N\) est la complétion de cette mesure.

\begin{proposition}[\cite{OYRmzAa}]     \label{PropSKXGooRFHQst}
    Tout ouvert de \( \eR^n\) est une union dénombrable et disjointe de cubes semi-ouverts.
\end{proposition}

\begin{proof}
    Nous allons même montrer que ces cubes peuvent être choisis sur un quadrillage.

    Soit \( G\) un ouvert de \( \eR^n\). Soit \( \{ Q_i^{1} \}_{i\in \eN}\) un découpage de \( \eR^n\) en cubes semi-ouverts de côté \( 1\) et dont les sommets sont en les coordonnées entières. Ils sont de la forme
    \begin{equation}
        \prod_{i=1}^n\mathopen[ n_i , n_i+1 \mathclose[
    \end{equation}
    où les \( n_i\) sont des entiers. Ce sont des cubes disjoints. Nous considérons ensuite pour chaque \( k>1\) le découpage \( \{ Q_i^{(k)} \}_{i\in\eN}\) de \( \eR^n\) en cubes de côtés \( 2^{-k}\) qui consiste à découper en \( 2\) les côtés des cubes du découpage \( Q^{(k-1)}\). Ces cubes forment encore un découpage dénombrable de \( \eR^n\) en des cubes disjoints. Ils sont de la forme
    \begin{equation}
        \prod_{i=1}^n\mathopen[ \frac{ n_i }{ 2^k } , \frac{ n_i+1 }{ 2^k } \mathclose[
    \end{equation}
    où les \( n_i\) sont encore entiers. Ensuite nous considérons \( \mE\) l'union de tous les \( Q_i^{(k)}\) contenus dans \( G\).

    Montrons que \( \mE=G\). D'abord \( \mE\subset G\) parce que \( \mE\) est une union d'ensembles contenus dans \( G\). Ensuite si \( x\in G\), il existe une boule de rayon \( r\) autour de \( x\) contenue dans \( G\); alors un des ensembles \( Q_i^{(k)}\) avec \( 2^{-j}<\frac{ r }{2}\) est contenue dans \( B(x,r)\) et donc dans \( \mE\).

    Bien entendu l'union qui donne \( \mE\) n'est pas satisfaisante par ce que les \( Q_i^{(k+1)}\) sont contenus dans les \( Q_i^{(k)}\); les intersections sont donc loin d'être vides.

    Nous faisons ceci : 
    \begin{subequations}
        \begin{align}
            R^{(0)}&=\{ Q_i^{(1)} \text{contenu dans } G \}\\
            R^{(k+1)}&=\{ Q_i^{(k+1)}\text{contenus dans } G\text{ et pas dans } R^{(k)} \}.
        \end{align}
    \end{subequations}
    En fin de compte l'union de tous les ensembles contenus dans les \( R^{(k)}\) forment encore \( \eR^n\), mais sont d'intersection vide.
\end{proof}

Les cubes dont il est question dans cette preuve, de côtés \( 2^{-k}\) sont souvent appelés des cubes \defe{dyadiques}{dyadique}.

\begin{corollary}[\cite{OYRmzAa}]     \label{CorTHDQooWMSbJe}
    Tout ouvert de \( \eR^n\) est une union dénombrable de cubes presque disjoints\footnote{«presque» au sens où les intersections éventuelles sont de mesure de Lebesgue nulle.}.
\end{corollary}

\begin{proof}
    Il suffit de prendre les cubes de la proposition \ref{PropSKXGooRFHQst} et de les fermer. Ce que l'on ajoute est de mesure nulle.
\end{proof}

\begin{remark}
    La proposition \ref{PropSKXGooRFHQst} est une propriété seulement de la topologie de \( \eR^n\) alors que le corollaire fait intervenir la mesure de Lebesgue parce qu'il faut bien dire que les intersections sont de mesure (de Lebesgue) nulle.
\end{remark}

%--------------------------------------------------------------------------------------------------------------------------- 
\subsection{Ensembles négligeables}
%---------------------------------------------------------------------------------------------------------------------------

\begin{lemma}[\cite{VSMEooLwNLHd}]      \label{LemWHKJooGPuxEN}
    L'image d'une partie négligeables de \( \eR^N\) par une application Lipschitz est négligeable.
\end{lemma}

\begin{proof}
    Soit \( N\) une partie négligeable de \( \eR^N\) et une application Lipschitz \( f\colon N\to \eR^N\). Soit \( Q\subset \eR^N\) un cube borné de côté \( r\). Pour tout \( x,x'\in N\cap Q\) nous avons
    \begin{equation}
        \| f(x)-f(x') \|\leq C\| x-x' \|\leq Cr.
    \end{equation}
    Donc \( f(N\cap Q)\) est dans une boule de rayon \( Cr\). Mais comme toutes les normes sont équivalentes\footnote{Proposition \ref{PropLJEJooMOWPNi}} sur \( \eR^N\) nous pouvons tout aussi bien prendre la norme \( \| . \|_1\) au lieu de la norme \( \| . \|_2\) (qui est touours la norme prise implicitement lorsqu'on parle de \( \eR^n\)), de telle sorte que les boules soient des cubes. Quoi qu'il en soit, \( f(N\cap Q)\) est contenu dans un cube de coté \( 2Cr\) et au niveau de la mesure extérieure,
    \begin{equation}
        m^*\big( f(N\cap Q) \big)\leq (2Cr)^N=(2C)^Nr^N,
    \end{equation}
    ou encore
    \begin{equation}
        m\big(f(N\cap Q)\big)\leq (2C)^Nm(Q)
    \end{equation}
    parce que \( r^N\) est la mesure du cube \( Q\).

    Soit maintenant \( \epsilon>0\); vu que \( N\) est négligeable, il existe un ouvert \( U\) contenant \( N\) et tel que \( m(U)<\epsilon\). Ce \( U\) est une union presque disjointe de cubes dyadiques \( (Q_n)\) par le corollaire \ref{CorTHDQooWMSbJe}. Nous avons alors
    \begin{subequations}
        \begin{align}
            m^*\big( f(N) \big)&=m^*\big( f(\bigcup_nN\cap Q_n) \big)\\
            &=m^*\big( \bigcup_nf(N\cap Q_n) \big)\\
            &\leq \sum_nm^*(f(N\cap Q_n))\\
            &\leq \sum_n(2C)^Nm(Q_n)\\
            &=(2C)^Nm(U)\\
            &<(2C)^d\epsilon.
        \end{align}
    \end{subequations}
    Au final, \( m^*\big( f(N) \big)\leq (2C)^N\epsilon\).  L'ensemble \( N\) est donc négligeable parce que le lemme \ref{LemXOUNooUbtpxm} le dit : \( m^*(N)=0\).
\end{proof} 

\begin{corollary}
    Un sous-espace vectoriel strict de \( \eR^N\) est négligeable.
\end{corollary}

\begin{proof}
    Un sous-espace vectoriel strict de \( \eR^N\) de dimension \( k<N\) est l'image de
    \begin{equation}
        A=\{ t_1e_1+\cdots +t_ke_k\tq t_i\in \eR \}
    \end{equation}
    par une application linéaire. Ce \( A\) est un pavé de mesure de Lebesgue nulle. Donc l'image est négligeable par le lemme \ref{LemWHKJooGPuxEN}.
\end{proof}

%--------------------------------------------------------------------------------------------------------------------------- 
\subsection{Parties et fonctions mesurables}
%---------------------------------------------------------------------------------------------------------------------------

Pour rappel, la notion d'application de classe \( C^1\) est donnée par la définition \ref{DefJYBZooPTsfZx}.

\begin{proposition}     \label{PropRDRNooFnZSKt}
    Soient \( U\) et \( V\) des ouverts de \( \eR^N\) et \( \phi\colon U\to V\) un \( C^1\)-difféomorphisme. Si \( E\subset U\) est mesurable, alors \( \phi(E)\) est mesurable\footnote{Ici «mesurable» parle de mesurabilité au sens de la tribu de Lebesgue, c'est à dire pas seulement les boréliens.}.
\end{proposition}

\begin{proof}
    Si \( E\) est mesurable, il existe un borélien \( B\) et un ensemble négligeable \( N\) tels que \( E=B\cup N\). Vu que \( \phi\) est un homéomorphisme, l'application \( \phi^{-1}\) est borélienne parce que continue (théorème \ref{ThoJDOKooKaaiJh}). Nous avons 
    \begin{equation}
        \phi(B)=(\phi^{-1})^{-1}(B),
    \end{equation}
    c'est à dire que \( \phi(B)\) est l'image inverse de \( B\) par \( \phi^{-1}\). L'ensemble \( \phi(B)\) est donc borélien.

    Il reste à voir que \( \phi(N)\) est négligeable. Soit \( Q\subset U\) une cube compact. L'application \( d\phi\colon Q\to \aL(\eR^N,\eR^N)\) est continue et donc bornée (par la remarque \ref{RemATQVooDnZBbs}) sur le compact \( Q\). Par les accroissements finis (théorème \ref{ThoNAKKght}), l'application \( \phi\) est donc Lipschitz sur \( Q\). La partie \( \phi(N\cap Q)\) est alors négligeable par le lemme \ref{LemWHKJooGPuxEN}. Pour conclure,
    \begin{equation}
        \phi(N)=\bigcup_i\phi(N\cap Q_i)
    \end{equation}
    où les \( Q_i\) sont tous des cubes compacts. Donc \( \phi(N)\) est une union dénombrable d'ensembles négligeables; ergo négligeable lui-même par le lemme \ref{LemVKNooOCOQw}.
\end{proof}

\begin{proposition}
    Soient \( U\) et \( V\) des ouverts de \( \eR^N\) et \( \phi\colon U\to V\) un \( C^1\)-difféomorphisme. Si \( f\colon V\to \eC\) est mesurable, alors \(f\circ \phi\colon U\to \eC\) l'est.
\end{proposition}

\begin{proof}
    Soit \( A\) une partie mesurable de \( \eC\). Il nous faut prouver que
    \begin{equation}
        (f\circ\phi)^{-1}(A)=\phi^{-1}\big( f^{-1}(A) \big)
    \end{equation}
    soit mesurable. Par hypothèse , \( f^{-1}(A)\) est mesurable. Vu que \( \phi\) est un \( C^1\)-difféomorphisme, elle et son inverse sont mesurables par la proposition \ref{PropRDRNooFnZSKt}. Donc l'image du mesurable \( f^{-1}(A)\) par \( \phi^{-1}\) est encore mesurable. 
\end{proof}

%--------------------------------------------------------------------------------------------------------------------------- 
\subsection{Propriétés d'unicité}
%---------------------------------------------------------------------------------------------------------------------------

\begin{corollary}       \label{CorMPDAooDJRrom}
    La mesure \( \lambda_N\) est l'unique mesure sur \(   (\eR^N,  \Borelien(\eR^N) )   \) à satisfaire 
    \begin{equation}
        \mu\big( \prod_{i=1}^N\mathopen[ a_i , b_i \mathclose] \big)=\prod_{i=1}^n| a_i-b_i |
    \end{equation}
\end{corollary}

\begin{proof}
    Par définition de la mesure produit, \( \lambda_N\) est l'unique mesure sur \(   (\eR^N,  \Borelien(\eR)\otimes\ldots\otimes\Borelien(\eR) )   \) à satisfaire la condition. La proposition \ref{CorWOOOooHcoEEF} conclut.
\end{proof}

Vu que les compacts de \( \eR^n\) sont les fermés bornés (théorème \ref{ThoXTEooxFmdI}), et que tout borné est dans un tel produit d'intervalle, la mesure de Lebesgue est une mesure de Borel (définition \ref{DefFMTEooMjbWKK}\ref{ItemTTPTooStDcpw}).

\begin{theorem}[\cite{PMTIooJjAmWR}]
    La mesure de Lebesgue est invariante par translation. Autrement dit si \( A\) est mesurable dans \( \eR^n\) et si \( a\in \eR^n\) alors \( A+a\) est mesurable et
    \begin{equation}
        \lambda_N(A+a)=\lambda_N(A).
    \end{equation}
\end{theorem}


\begin{proof}
    Nous supposons que \( A\) est borélien; sinon il l'est à ensemble négligeable près. Nous notons \( t_a\) la translation et nous nommons \( \mu\) la mesure donnée par
    \begin{equation}
        \mu(A)=\lambda_N(A+a).
    \end{equation}
    Vu que
    \begin{equation}
        \mu\big( \prod_{n=1}^N\mathopen[ r_n , s_n \mathclose[ \big)=\lambda_N\big( \prod_i\mathopen[ r_n+a_n , s_n+a_n [ \big)=\prod_i| s_n-r_n |.
    \end{equation}
    Vu qu'il y a unicité de la mesure vérifiant cette propriété (corollaire \ref{CorMPDAooDJRrom}), nous avons \( \mu=\lambda_N\).
\end{proof}

Pour la suite nous notons \( Q_0\) le cube unité de \( \eR^N\) : \( Q_0=\big( \mathopen[ 0 , 1 \mathclose[ \big)^N\).

\begin{theorem}[\cite{PMTIooJjAmWR}]        \label{ThoCABFooHbUzWc}
    Soit \( \mu\) une mesure positive sur \( \eR^N\) telle que
    \begin{enumerate}
        \item
            \( \mu\) soit invariante par translation (des boréliens),
        \item
            \( \mu(Q_0)=1\).
    \end{enumerate}
    Alors \( \mu=\lambda_N\).
\end{theorem}
    
\begin{proof}
    Pour simplifier l'écriture nous faisons \( N=2\). Notre but est de prouver que \( \mu(  \mathopen[ 0 , r \mathclose[\times \mathopen[ 0 , r' \mathclose[ )=rr'\) pour tout \( r,r'\in \eR\).

    \begin{subproof}
    \item[Longueur =\( 1/J\)]
        Soient \( J,K\) des entiers. Nous pouvons diviser le cube \( Q_0\) en rectangles de cotés \( 1/J\) et \( A/K\) :
        \begin{equation}
            Q_0=\bigcup_{\substack{1\leq j\leq J\\1\leq k\leq K}}\mathopen[ \frac{ j-1 }{ J } , \frac{ j }{ J } \mathclose[\times \mathopen[ \frac{ k-1 }{ K } , \frac{ k }{ K } \mathclose[
        \end{equation}
        où l'union est disjointe. En ce qui concerne la mesure nous commençons par utiliser la sous-additivité :
        \begin{equation}
            \mu(Q_0)=\sum_{\substack{1\leq j\leq J\\1\leq k\leq K}}\mu\left(  \mathopen[ \frac{ j-1 }{ J } , \frac{ j }{ J } \mathclose[\times \mathopen[ \frac{ k-1 }{ K } , \frac{ k }{ K } \mathclose[      \right).
        \end{equation}
        Nous utilisons ensuite, sur chacun des termes séparément l'invariance par translation selon les vecteurs \( (\frac{ j-1 }{ J },0)\) et \( ( 0,\frac{ k-1 }{ K } )\) :
        \begin{equation}
            1=\mu(Q_0)=\sum_{\substack{1\leq j\leq J\\1\leq k\leq K}}\mu\left(  \mathopen[ 0,\frac{1}{ J } \mathclose[\times \mathopen[0,\frac{1}{ K }\mathclose[      \right)=JK\mu\mu\left(  \mathopen[ 0,\frac{1}{ J } \mathclose[\times \mathopen[0,\frac{1}{ K }\mathclose[      \right),
        \end{equation}
        et donc
        \begin{equation}
            \mu\left(  \mathopen[ 0,\frac{1}{ J } \mathclose[\times \mathopen[0,\frac{1}{ K }\mathclose[      \right)=\frac{1}{ J }\times \frac{1}{ K }.
        \end{equation}
    \item[Longueur \( L/K\)]

        Soient \( L,M\) des entiers et calculons :
        \begin{subequations}
            \begin{align}
                \mu\left( \mathopen[ \frac{ 0 }{ J } , \frac{ L }{ J } \mathclose[\times \mathopen[ \frac{ 0 }{ K } , \frac{ M }{ K } \mathclose[ \right)&=\sum_{\substack{0\leq l\leq L-1\\0\leq m\leq M-1}}\mu\left(   \mathopen[    \frac{ l }{ J },\frac{ l+1 }{ J }  \mathclose[\times \mathopen[ \frac{ m }{ K } , \frac{ m+1 }{ K } \mathclose[      \right)\\
                    &=LM\mu\left(  \mathopen[ \frac{ 0 }{ J } , \frac{ 1 }{ J } \mathclose[\times \mathopen[ \frac{ 0 }{ K } , \frac{ 1 }{ K } \mathclose[  \right)\\
                        &=LM\times \frac{1}{ J }\times \frac{1}{ K }.
            \end{align}
        \end{subequations}
        Nous avons donc, pour tout \( J,K,L,M\) : 
        \begin{equation}
            \mu\left( \mathopen[ 0 , \frac{ L }{ J } \mathclose[\times \mathopen[ 0, \frac{ M }{ K } \mathclose[ \right)=\frac{ L }{ J }\times \frac{ M }{ K },
        \end{equation}
        c'est à dire que pour tout \( r,s\in \eQ^+\) nous avons
        \begin{equation}
            \mu\big(   \mathopen[ 0 , r \mathclose[\times \mathopen[ 0 , s \mathclose[ \big)=rs.
        \end{equation}
    \item[Longueur réelle]
        Nous passons au cas de longueur réelle. Soit \( a>0\) et une suite croissante de rationnels \( r_n\to a\). Une telle suite existe par la proposition \ref{PropSLCUooUFgiSR}. Nous avons \( \mathopen[ 0 , a \mathclose[=\bigcup_{n\geq 1}\mathopen[ 0 , r_n \mathclose[\) où l'union n'est pas disjointe mais croissante, ce qui permet d'utiliser le lemme \ref{LemAZGByEs}\ref{ItemJWUooRXNPci} pour écrire
        \begin{equation}
            \mu\big( \mathopen[ 0 , a \mathclose[ \big)=\mu\left( \bigcup_{n\geq 1}\mathopen[ 0 , r_n \mathclose[ \right)=\lim_{n\to \infty} \mu\big( \mathopen[ 0 , r_n \mathclose[ \big)=\lim_{n\to \infty} r_n=a.
        \end{equation}
    \end{subproof}

    Enfin, si \( a,a'\in \eR\), l'invariance par translation donne
    \begin{equation}
        \mu\big( \mathopen[ a , a' \mathclose[ \big)=\mu\big( \mathopen[ 0 , a'-a \mathclose[ \big)=a'-a.
    \end{equation}
    Par unicité de la mesure ayant cette propriété, nous avons \( \mu=\lambda_N\).
\end{proof}

\begin{corollary}       \label{CorKGMRooHWOQGP}
    Si \( \mu\) est une mesure positive sur \( \eR^N\) invariante par translation et telle que \( \mu(Q_0)=C<\infty\) alors \( \mu=C\lambda_N\).
\end{corollary}

\begin{proof}
    Si \( C>0\) nous considérons la mesure \( \frac{1}{ C }\mu\) qui vérifie \( (\frac{1}{ C }\mu)(Q_0)=1\). En conséquence du théorème \ref{ThoCABFooHbUzWc}, \( \frac{1}{ C }\mu=\lambda_N\) et \( \mu=C\lambda_N\).

    Si au contraire \( C=0\) alors nous pouvons paver \( \eR^N\) avec des cubes \( Q_i\) de côté \( 1\) qui ont tous mesure \( 0\). Par conséquent, \( \eR^N=\bigcup_{i=1}^{\infty}Q_i\), donc \( \mu(\eR^N)=\sum_i\mu(Q_i)=0\). Par conséquent \( \mu=0\) parce que toute partie de \( \eR^N\) a une mesure au maximum égale à celle de \( \eR^N\).
\end{proof}

%--------------------------------------------------------------------------------------------------------------------------- 
\subsection{Régularité}
%---------------------------------------------------------------------------------------------------------------------------

Les différentes notions de régularité pour une mesure sont données dans la définition \ref{DefFMTEooMjbWKK}. Ce sont essentiellement des questions de compatibilité entre la mesure et la topologie.
\begin{proposition}
    La mesure de Lebesgue est une mesure de Radon sur tout ouvert de \( \eR^N\).
\end{proposition}

\begin{proof}
    Soit \( V\) un ouvert de \( \eR^N\). C'est localement compact et dénombrable à l'infini. Il suffit de prouver que \( \lambda_N\) est de Borel sur \( V\) pour que le théorème \ref{PropNCASooBnbFrc} conclue à la régularité de la mesure de Lebesgue.

    Soit \( K\) un compact de \( V\). Par la proposition \ref{PropGBZUooRKaOxy} c'est également un compact de \( \eR^N\). Par conséquent \( K\) est dans un pavé fermé de \( \eR^N\) du type
    \begin{equation}
        K\subset \prod_{n=1}^N\mathopen[ a_n , b_n \mathclose]
    \end{equation}
    et donc en passant par le corollaire \ref{CorMPDAooDJRrom},
    \begin{equation}
        \lambda_N(K)\leq \prod_{i=1}^N(b_n-a_n)<\infty.
    \end{equation}
    Nous avons démontré que \( \lambda_N\) reste fini sur tout compact de \( V\).
\end{proof}

%+++++++++++++++++++++++++++++++++++++++++++++++++++++++++++++++++++++++++++++++++++++++++++++++++++++++++++++++++++++++++++ 
\section{Propriétés de l'intégrale de Lebesgue}
%+++++++++++++++++++++++++++++++++++++++++++++++++++++++++++++++++++++++++++++++++++++++++++++++++++++++++++++++++++++++++++

%--------------------------------------------------------------------------------------------------------------------------- 
\subsection{Théorème de la moyenne}
%---------------------------------------------------------------------------------------------------------------------------

\begin{theorem}[\cite{MonCerveau}]      \label{ThoooEZLGooMChwLT}
    Soit \( Q\) un compact connexe par arcs et une fonction continue \( f\colon Q\to \eR\). Si \( \lambda\) est la mesure de Lebesgue, alors il existe \( a\in Q\) tel que
    \begin{equation}
        f(a)=\frac{1}{ \lambda(Q) }\int_Qfd\lambda
    \end{equation}
\end{theorem}

\begin{proof}
    En posant \( I=\int_Qfd\lambda\) nous avons immédiatement
    \begin{equation}        \label{EqooTYQCooVxdazW}
        \min(f)\lambda(Q)\leq I\leq \max(f)\lambda(Q)
    \end{equation}
    où le minimum et le maximum existent parce que \( f\) est continue sur un compact. Si une des deux inégalités est une égalité alors la fonction est constante. En effet supposons que la première inégalité soit une égalité; si la fonction n'était pas constante, il existerait une boule sur laquelle \( f\) serait strictement supérieure à \( \min(f)\). En intégrant d'abord sur cette boule et ensuite sur le complémentaire nous obtenons une intégrale plus grande que \( \min(f)\lambda(Q)\).

    Soit \( \epsilon>0\). Il existe \( \alpha,\beta\in Q\) tels que \( f(\alpha)\leq\min(f)+\epsilon\) et \( f(\beta)\geq\max(f)-\epsilon\). Soit \( \gamma\colon \mathopen[ 0 , 1 \mathclose]\to Q\) un chemin continu tel que \( \gamma(0)=\alpha\) et \( \gamma(1)=\beta\). La fonction \( f\circ \gamma\colon \mathopen[ 0 , 1 \mathclose]\to \eR\) est alors continue et vérifie \( (f\circ\gamma)(0)\leq \min(f)+\epsilon\) et \( (f\circ\gamma)(1)\geq \max(f)-\epsilon\).

    Si \( \epsilon\) est assez petit et vu que les inégalités \eqref{EqooTYQCooVxdazW} sont strictes,
    \begin{equation}
        \lambda(Q)(f\circ\gamma)(0)\leq \min(f)\lambda(Q)+\epsilon\lambda(Q)<I<\max(f)\lambda(Q)-\epsilon\lambda(Q)\leq\lambda(Q)(f\circ \gamma)(1).
    \end{equation}
    Par le théorème des valeurs intermédiaires \ref{ThoValInter}, il existe \( t_0\in\mathopen[ 0 , 1 \mathclose]\) tel que \( \lambda(Q)(f\circ\gamma)(t_0)=I\). Le point \( a=\gamma(t_0)\) vérifie
    \begin{equation}
        f(a)=\frac{1}{ \lambda(Q) }\int_Qfd\lambda.
    \end{equation}
\end{proof}

%--------------------------------------------------------------------------------------------------------------------------- 
\subsection{Primitives et intégrales}
%---------------------------------------------------------------------------------------------------------------------------

En termes de notations, si \( a<b\) nous posons
\begin{equation}
    \int_a^bf(t)dt=\int_{\mathopen[ a , b \mathclose]}f.
\end{equation}
Si par contre \( a>b\) nous posons \( \int_a^bf=-\int_b^af\).

\begin{proposition}[Primitive et intégrale] \label{PropEZFRsMj}
    Soit \( f\) une fonction intégrable sur \( \mathopen[ a , b \mathclose]\) et continue sur \( \mathopen] a , b \mathclose[\). Alors la fonction
    \begin{equation}
        \begin{aligned}
            F\colon \mathopen[ a , b \mathclose]&\to \eR \\
            x&\mapsto \int_{\mathopen[ a , x \mathclose]}f(t)dt.
        \end{aligned}
    \end{equation}
est l'unique primitive de \( f\) sur \( \mathopen] a , b \mathclose[\) s'annulant en \( x=a\).
\end{proposition}
\index{primitive!et intégrale}
\index{théorème!fondamental du calcul intégral}

\begin{proof}
Nous devons prouver que \( F\) est dérivable et que pour tout \( x_0\in\mathopen] a , b \mathclose[\) nous ayons \( F'(x_0)=f(x_0)\). Soit \( \epsilon>0\). Par continuité de \( f\) en \( x_0\), il existe une fonction \( \alpha\colon \eR\to \eR\) telle que
    \begin{equation}
        f(x_0+h)=f(x_0)+\alpha(h)
    \end{equation}
    avec \( \lim_{h\to 0} \alpha(h)=0\). De plus il existe un \( \delta>0\) tel que \( |\alpha(h)|<\epsilon\) pour tout \( h<\delta\). À partir de maintenant nous ne considérons plus que de tels \( h\).

    Nous calculons la dérivée de \( F\) en \( x_0\). Pour cela,
    \begin{subequations}
        \begin{align}
            F(x_0+h)-F(x_0)&=\int_{x_0}^{x_0+h}f(t)dt\\
        &=\int_0^hf(x_0+t)dt\\
        &=\int_0^h\big[ f(x_0)+\alpha(t) \big]dt\\
        &=hf(x_0)+\int_0^{h}\alpha(t)dt.
        \end{align}
    \end{subequations}
    Nous avons donc, pour tout \( h<\delta\),
    \begin{equation}
        hf(x_0)-h\epsilon\leq F(x_0+h)-F(x_0)\leq hf(x_0)+h\epsilon.
    \end{equation}
    En divisant par \( h\) et en prenant la limite \( h\to 0\),
    \begin{equation}
        F'(x_0)\in B\big( f(x_0),\epsilon \big).
    \end{equation}
    Cela étant valable pour tout \( \epsilon>0\) nous en déduisons que
    \begin{equation}
        F'(x_0)=f(x_0).
    \end{equation}

    Le fait que \( F\) s'annule en \( x=a\) est par sa définition. L'unicité provient du corollaire \ref{CorZeroCst}.
\end{proof}

Le théorème suivant est à utiliser pour calculer des intégrales des fonctions réelle lorsqu'on a des primitives sur un domaine strictement plus large que le domaine sur lequel nous voulons intégrer. Une version pour les intégrales impropres sera donnée au corollaire \ref{CorMUIooXREleR}.
\begin{theorem}[Théorème fondamental du calcul intégral]    \label{ThoRWXooTqHGbC}
    Soit \( f\) une fonction continue sur un intervalle ouvert \( I\) contenant strictement l'intervalle \( \mathopen[ a , b \mathclose]\subset \eR\) et \( F\) une primitive de \( f\) sur \( I\). Alors
    \begin{equation}
        \int_a^bf(t)dt=F(b)-F(a).
    \end{equation}
\end{theorem}

\begin{proof}
    Nous avons vu par la proposition \ref{PropEZFRsMj} que la fonction
    \begin{equation}
        \begin{aligned}
            \tilde F\colon \mathopen[ a , b \mathclose]&\to \eR \\
            x&\mapsto  \int_a^xf(t)dt
        \end{aligned}
    \end{equation}
    était une primitive de \( f\); c'est même l'unique\footnote{} primitive de \( f\) sur \( \mathopen[ a , b \mathclose]\) à s'annuler pour \( x=a\). Nous avons évidemment
    \begin{equation}
        \int_a^bf(t)dt=\tilde F(b).
    \end{equation}
    Si \( F\) est une primitive quelconque, il suffit de soustraire sa valeur en \( x=a\) : \( \tilde F(x)=F(x)-F(a)\) et donc
    \begin{equation}
        \int_a^bf(t)dt=\tilde F(b)=F(b)-F(a),
    \end{equation}
    comme il fallait le prouver.
\end{proof}

\begin{remark}
    Le lien entre primitive et intégrale est fondamentalement lié à l'invariance par translation de la mesure de Lebesgue, et non à la construction précise de cette mesure. Mais en même temps, la mesure de Lebesgue est l'unique à être invariante par translation.
\end{remark}

\begin{remark}
    Une primitive est forcément une fonction continue parce qu'une primitive est dérivable.
\end{remark}

Ce petit résultat nous donne une façon «pratique» de calculer des intégrales en cherchant des primitives. Nous rappelons qu'en vertu du corollaire \ref{CorZeroCst}, une fonction ne possède qu'une seule primitive à constante près.

\begin{corollary}[Théorème fondamental du calcul intégral]       \label{CORooKRBZooEMDobC}
    Soit une fonction \( f\colon I\to \eR\) où \( I\) est un intervalle. Nous supposons que \( f\) est continue sur \( I\). Soient \( a<b\) dans \( I\) et une primitive \( F\) de \( f\) sur \( I\).

    Alors
    \begin{equation}        \label{EQooXEKDooEnWcbn}
        \int_a^bf=F(b)-F(a).
    \end{equation}
\end{corollary}

\begin{proof}
    Nous savons par la proposition \ref{PropEZFRsMj} que la fonction donnée par
    \begin{equation}
        G(x)=\int_a^xf
    \end{equation}
    est une primitive de \( f\). Vu que \( F\) est également une primitive nous avons \( F(x)=G(x)+C\) pour une certaine constante \( C\in \eR\) (c'est le corollaire \ref{CorZeroCst}). Vu que \( G(a)=0\), l'égalité \( F(a)=G(a)+C\) donne \( C=F(a)\).

    Ce que l'on nous demande de calculer dans \eqref{EQooXEKDooEnWcbn} est \( G(b)\), c'est à dire l'égalité
    \begin{equation}
        F(b)=G(b)+C=G(b)+F(a),
    \end{equation}
    et donc \( G(b)=F(b)-F(a)\), ce qu'il fallait.
\end{proof}

%--------------------------------------------------------------------------------------------------------------------------- 
\subsection{Exemples et applications}
%---------------------------------------------------------------------------------------------------------------------------

Si \( f\) est une fonction définie sur un intervalle \( I\) et y admettant des primitives, nous notons
\begin{equation}
    \int f(x)dx
\end{equation}
l'ensemble des primitives de \( f\) sur \( I\) :
\begin{equation}
    \int f(x)dx=\left\{    F(x)+C\tq C\in \eR   \right\}
\end{equation}
où \( F\) est une quelconque primitive de \( f\).

\begin{example}
    Une primitive bien connue de \(  f\colon x\mapsto x^2 \) est la fonction \( F\colon x\to \frac{ x^3 }{ 3 }\). Nous écrivons donc
    \begin{equation}
        \int x^2dx=\frac{ x^3 }{ 3 }+C.
    \end{equation}
    Cela est un abus de notations terrible pour dire en réalité
    \begin{equation}
        \{ x\mapsto \frac{ x^3 }{ 3 }+C\tq C\in \eR \}.
    \end{equation}
\end{example}

En termes de notations, nous posons
\begin{equation}\label{Thfondcalc}
    \int_a^bf(t)dt=\Big[ F(t) \Big]_{t=a}^{t=b}=F(b)-F(a).
\end{equation}

\begin{remark}
  La valeur de l'intégrale ne dépend pas de la primitive qu'on choisi pour le calculer, car si $F_1$ et $F_2$ sont deux primitives de $f$ alors $F_1 = F_2 + C$ et $F_1(b)-F_1(a) = (F_2(b) + C)-(F_2(a)+C) = F_2(b)-F_2(a)$.
\end{remark}

\begin{remark}
  Si l'intervalle d'intégration est réduit à un seul point alors la valeur de l'intégrale est zéro, car $ \int_a^af(t)dt=F(a)-F(a) =0$.
\end{remark}

\begin{remark}
    Conformément à ce que nous montre la figure \ref{LabelFigKKRooHseDzC}, si une fonction continue est positive sur l'intervalle \( \mathopen[ a , b \mathclose]\), alors le nombre \( \int_a^bf(t)dt\) est l'aire de la portion de plan comprise entre les droites verticales \( x=a\), \( x=b\), la courbe représentant la fonction \( f\) et l'axe des abscisses.

    Si la fonction est négative : l'aire est comptée négativement.
\end{remark}

\begin{example} 
    Comme nous le voyons sur le dessin suivant,
    \begin{equation}
        \int_{-3\pi/2}^{3\pi/2}\sin(x)\,dx=0
    \end{equation}
    parce que les deux parties bleues s'annulent avec les deux parties rouges (qui sont comptées comme des aires négatives).
    \begin{center}
       \input{auto/pictures_tex/Fig_JSLooFJWXtB.pstricks}
    \end{center}
\end{example}

\begin{remark}
  Toute intégrale d'une fonction impaire sur un intervalle symétrique par rapport à l'origine est nulle. 
\end{remark}

%--------------------------------------------------------------------------------------------------------------------------- 
\subsection{Intégrales impropres}
%---------------------------------------------------------------------------------------------------------------------------
\label{SecGAVooBOQddU}

% TODO : l'exemple avec arcsin(1/x)-1/x de la page 
%  http://fr.wikipedia.org/wiki/Intégrale_impropre

\begin{definition}[\cite{TrenchRealAnalisys}]
    Une fonction \( f\colon D\subset\eR\to \eR\) est \defe{localement intégrable}{localement!intégrable} sur un intervalle \( I\) si \( f\) est intégrable sur tout intervalle compact contenu dans \( I\).
\end{definition}
\index{intégrale!impropre}

%Dans \cite{TrenchRealAnalisys}, la proposition \ref{PropCJAooQhNYkp} est prise comme une définition de \( \int_a^bf\) lorsque \( f\) est localement intégrable sur \( \mathopen[ a , b [\). Le point est que lui, il ne passe pas par Lebesgue et la construction abstraite d'intégrale par rapport à une mesure. Nous par contre nous avons déjà une définition de
%\begin{equation}
%    \int_a^bf=\int_{\mathopen[ a , b \mathclose]}f
%\end{equation}
%pour tout choix de \( a\), \( b\) et \( f\), que ce soit borné ou non.

\begin{proposition}     \label{PropCJAooQhNYkp}
    Soit \( f\colon \mathopen[ a , b \mathclose]\to \eR\) une fonction intégrable. Alors
    \begin{equation}    \label{EqPPMooBQDTYl}
        \int_{\mathopen[ a , b \mathclose]}f=\lim_{x\to b^-} \int_a^xf.
    \end{equation}
\end{proposition}

\begin{proof}
    Notons que la valeur de \( f\) en \( b\) n'a strictement aucune importance parce que l'intégrale de Lebesgue ne dépend pas du choix de la valeur de la fonction en un ensemble de mesure nulle; et en même temps la limite à gauche de \eqref{EqPPMooBQDTYl} ne dépend pas non plus de la valeur de \( f\) en \( b\). Bref si \( f\) n'est pas définie en \( b\), nous pouvons poser \( f(b)=42\).

    Notons de plus que du point de vue de l'intégrale de Lebesgue, \( \int_{\mathopen[ a , b \mathclose]}\) et \( \int_{\mathopen[ a , b [}\) sont identiques et valent toutes les deux \( \int_a^b\) (lorsque ça existe).

    Supposons d'abord que \( f\) est positive. Alors nous posons \( f_n=f\mtu_{\mathopen[ a , b-\frac{1}{ n } \mathclose]}\). Ponctuellement nous avons la limite croissante \( f_n\to f\) et de plus
    \begin{equation}
        \lim_{x\to b^-} \int_{\mathopen[ a , x \mathclose]}f=\lim_{n\to \infty} \int_{\mathopen[ a , b \mathclose]}f_n.
    \end{equation}
    Chacun des \( f_n\) est intégrable sur \( \mathopen[ a , b \mathclose]\). Le théorème de Beppo-Levi \ref{ThoRRDooFUvEAN} implique que \( f\) est intégrable sur \( \mathopen[ a , b \mathclose]\) et que
    \begin{equation}
        \lim_{n\to \infty} \int_a^bf_n=\int_a^bf.
    \end{equation}
    Cela montre que dans le cas d'une fonction \( f\) positive nous avons bien \eqref{EqPPMooBQDTYl}.

    Si \( f\) n'est pas positif, alors nous la décomposons en partie positive et négative \( f=f^+-f^{-}\) et par définition de l'intégrale d'une fonction non positive,
    \begin{equation}
        \lim_{x\to b^-} \int_{\mathopen[ a , x [}f=\lim\int f^{+}-\lim\int f^-.
    \end{equation}
\end{proof}

Il peut cependant arriver que la limite \( \lim_{x\to b} \int_a^bf\) existe alors que \( f\) n'est pas intégrable sur \( \mathopen[ a , b \mathclose]\). C'est l'ennui des fonctions non positives. Un exemple classique est
\begin{equation}\label{EqMMVooDSpgfz}
    \int_0^{\infty}\frac{ \sin(t) }{ t }dt
\end{equation}

\begin{definition}[\cite{DWNooWUZxRP}]      \label{DEFooINPOooWWObEz}
    Si
    \begin{equation}
        \lim_{x\to b} \int_a^bf
    \end{equation}
    existe alors nous disons que l'intégrale est \defe{convergente}{intégrale!convergente} en \( b\). Ce procédé de limite est l'intégrale \defe{impropre}{intégrale!impropre} de \( f\) sur \( \mathopen[ a , b \mathclose]\).
\end{definition}

\begin{example}[Intégale impropre]
    Nous considérons la fonction \( f\colon \mathopen[ 0 , \infty [\to \eR\) définie par
    \begin{equation}
        f(x)=\begin{cases}
            \frac{1}{ n }    &   \text{si } x\in\mathopen[ 2n-2 , 2n-1 [\\
                -\frac{1}{ n }    &    \text{si } x\in\mathopen[ 2n-1 , 2n [\text{.}
        \end{cases}
    \end{equation}
    Par la divergence de la série harmonique, \( \int_{0}^{\infty}| f |\) n'existe pas. La fonction \( f\) n'est donc pas intégrable au sens de Lebesgue (définition \ref{DefTCXooAstMYl}).

    Cependant pour tout \( n\) pair nous avons
    \begin{equation}
        \int_0^nf=0.
    \end{equation}
    Du coup pour tout \( x\geq 0\) nous avons
    \begin{equation}
        \int_0^xf=\int_{2n}^xf
    \end{equation}
    où \( 2n\) est le plus grand nombre pair inférieur à \( x\). Nous avons \( | x-2n |\leq 2\) et \( | f(x) |\leq \frac{1}{ n }\) pour \( x\in\mathopen[ 2n , x \mathclose]\). Donc
    \begin{equation}
        \int_{2n}^xf\leq \frac{ 2 }{ n }.
    \end{equation}
    Nous avons par conséquent
    \begin{equation}
        \lim_{x\to \infty} \int_0^xf=0,
    \end{equation}
    ce qui signifie que l'intégrale de \( f\) sur \( \mathopen[ 0 , \infty [\) converge au sens des intégrales impropres.
\end{example}


L'intégrale \eqref{EqMMVooDSpgfz} est une intégrale convergente mais la fonction n'est pas intégrable (parce que pour être intégrale il faut que \( | f |\) soit intégrable). Nous pouvons ainsi dire que cette intégrale converge mais n'existe pas.

Le corollaire suivant nous autorise à utiliser le théorème fondamental du calcul intégral \ref{ThoRWXooTqHGbC} même dans les cas limites.
\begin{corollary}   \label{CorMUIooXREleR}
    Si \( f\) est localement intégrable sur \( \mathopen[ a , b \mathclose]\) et si \( F\) est une primitive de \( f\) sur tout ouvert de \( \mathopen[ a , b \mathclose]\) alors
    \begin{equation}
        \int_a^bf=\lim_{x\to b^-} F(x)-F(a).
    \end{equation}
\end{corollary}
\index{primitive!et intégrale}

\begin{proof}
    Pour chaque \( x\) dans \( \mathopen[ a , b [\) nous avons
    \begin{equation}
        \int_a^xf=F(x)-F(b).
    \end{equation}
    La proposition \ref{PropCJAooQhNYkp} nous explique que la limite \( x\to b^-\) du membre de gauche existe et vaut \( \int_a^bf\). Donc également le membre de droite :
    \begin{equation}
        \int_a^bf=\lim_{x\to b^-} \int_a^xf=\lim_{x\to b^-} F(x)-F(b).
    \end{equation}
\end{proof}

La convergence des intégrales de fonctions \( \frac{1}{ x^{\alpha} }\) en \( 0\) et \( \infty\) est une question classique de l'intégration. De plus ces fonctions servent souvent à utiliser une théorème de comparaison (type intégrale dominée de Lebesgue).
\begin{proposition} \label{PropBKNooPDIPUc}
    Deux intégrales remarquables.
    \begin{enumerate}
        \item
            
            Nous avons 
    \begin{equation}
        \int_0^1\frac{1}{ x^\alpha }=\infty
    \end{equation}
    si et seulement si \( \alpha\geq 1\).

\item

    Nous avons
    \begin{equation}
        \int_1^{\infty}\frac{1}{ x^{\alpha} }=\infty
    \end{equation}
    si et seulement si \( \alpha\leq1\).

    \end{enumerate}
    
\end{proposition}

\begin{proof}
La fonction \( \frac{1}{ x^{\alpha} }\) admet la primitive \( F(x)=\frac{1}{ 1-\alpha }\frac{1}{ x^{\alpha-1} }\) sur tout compact de \( \mathopen] 0 , \infty \mathclose[\). Le corollaire \ref{CorMUIooXREleR} nous permet\footnote{Tout ce que nous avons fait avec la borne \( b\) de l'intégrale \( \int_a^b\) reste valable avec la borne \( a\).} de dire que \( \int_0^1\frac{1}{ x^{\alpha} }\) vaudra
    \begin{equation}
        \lim_{x\to 0-^+} \frac{1}{ 1-\alpha }\frac{1}{ x^{\alpha-1} }.
    \end{equation}
    Cela est strictement plus petit que \( \infty\) si et seulement si \( \alpha<1\).
\end{proof}

%++++++++++++++++++++++++++++++++++++++++++++++++++++++++++++++++++++++++++++++++++++++++++++++++++
\section{Changement de variables dans une intégrale multiple}
%++++++++++++++++++++++++++++++++++++++++++++++++++++++++++++++++++++++++++++++++++++++++++++++++++

Dans ce qui suit, \( U\) et \( V\) sont des ouverts de \( \eR^N\) et \( \phi\colon U\to V\) est un \( C^1\)-difféomorphisme. Nous notons \( \mQ\) l'ensemble des cubes fermés dans \( U\) dont les cotés sont parallèles aux axes.

%--------------------------------------------------------------------------------------------------------------------------- 
\subsection{Des lemmes}
%---------------------------------------------------------------------------------------------------------------------------

\begin{lemma}[\cite{PMTIooJjAmWR}]      \label{LemooJYCGooIkkDVn}
    Soient \( \mu\) et \( \nu\) deux mesures de Borel sur l'ouvert \( U\) de \( \eR^N\). Si \( \mu(Q)\leq \nu(Q)\) pour tout \( Q\in \mQ\) alors \( \mu(B)\leq \nu(B)\) pour tout borélien \( B\).
\end{lemma}

\begin{proof}
    Si \( Q\) est un cube semi-ouvert, c'est à dire de la forme
    \begin{equation}
        Q=\prod_{i=1}N\mathopen[ a_n , a_n+h \mathclose[\subset U
    \end{equation}
    alors \( Q\) est une réunion croissante de cubes fermés du type \( \mathopen[ a_n+\epsilon , a_n+h-\epsilon \mathclose]\), et donc \( \mu(Q)\leq \nu(Q)\) par le lemme \ref{LemAZGByEs}\ref{ItemJWUooRXNPci}. La propriété est donc vraie pour les cubes semi-ouverts.

    Si \( \Omega\) est un ouvert, alors il est réunion disjointe dénombrable de cubes semi-ouverts par la proposition \ref{PropSKXGooRFHQst}. Donc pour tout ouvert \( \Omega\subset U\) nous avons \( \mu(\Omega)\leq\nu(\Omega)\). En vertu de la proposition \ref{PropNCASooBnbFrc} et de la remarque \ref{RemooOAGCooRHpjxd}, les mesures \( \mu\) et \( \nu\) sont régulières, et l'inégalité au niveau des ouverts se répercute en inégalité pour tout boréliens de \( U\) :
    \begin{equation}
        \mu(B)\leq \nu(B)
    \end{equation}
    pour tout \( B\in\Borelien(U)\). Notons que \( U\) étant ouvert dans \( \eR^N\), les boréliens de \( U\) sont exactement les boréliens de \( \eR^N\) inclus à \( U\) par le corollaire \ref{CorooMJQYooFfwoTd}.
\end{proof}

\begin{lemma}[\cite{PMTIooJjAmWR}]      \label{LemooJCEDooBRyjRg}
    Soit une application \( \theta\colon U\to \eR^N\) de classe \( C^1\) où \( U\) est ouvert dans \( \eR^N\). Pour tout \( Q\in\mQ\) nous avons
    \begin{equation}
        \lambda_N\big( \theta(Q) \big)\leq\sup_{s\in Q}\| d\theta_s \|^N\lambda_N(Q).
    \end{equation}
\end{lemma}

\begin{proof}
    Nous notons \( h\) la longueur du côté du cube. Le théorème des accroissements finis \ref{val_medio_2}, pour la composante \( \theta_i\) donne, pour \( u,v\in Q\) :
    \begin{equation}        \label{EqooFZMAooKWdzxJ}
        \big|  \theta_i(u)-\theta_i(v) \big|\leq\sup_{s\in Q}\| (d\theta_i)_s \|\| u-v \|\leq \sum_{s\in Q}\| (d\theta_i)_s \|h.
    \end{equation}
    D'autre part nous avons (nous écrivons pour \( N=2\) pour être plus court) :
    \begin{equation}
        d\theta_s(u)=\Dsdd{ \theta_1(s+tu)e_1+\theta_2(s+tu)e_2 }{t}{0}=(d\theta_1)_s(u)e_1+(d\theta_2)_s(u)e_2.
    \end{equation}
    Donc pour chaque \( i\) : \( \| d\theta_s \|\geq \| (d\theta_i)_s \|\), et nous continuons la majoration \eqref{EqooFZMAooKWdzxJ} :
    \begin{equation}
        \big|  \theta_i(u)-\theta_i(v) \big|\leq\leq \sum_{s\in Q}\| (d\theta_i)_s \|h\leq \sup_{s\in Q}\| d\theta_s \|h.
    \end{equation}
    
    Les points \( \theta(u)\) et \( \theta(v)\) sont donc dans un cube de côté \( \sup_{s\in Q}\| d\theta_s \|h\), ce qui permet de majorer \( \lambda_N\big( \theta(Q) \big)\) par
    \begin{equation}
        \lambda_N\big( \theta(Q) \big)\leq \left( \sup_{s\in Q}\| d\theta_s \|h \right)^N=\left( \sup_{s\in Q}\| d\theta_s \| \right)^N\lambda_N(Q)
    \end{equation}
    où le dernier facteur provient de l'égalité \( h^N=\lambda_N(Q)\).
\end{proof}

%--------------------------------------------------------------------------------------------------------------------------- 
\subsection{Déterminant et mesure de Lebesgue}
%---------------------------------------------------------------------------------------------------------------------------

Dans la suite, \( Q_0\) désigne le cube unité : \( Q_0=\big( \mathopen[ 0 , 1 \mathclose[ \big)^N\).

\begin{theorem}[Interprétation géométrique du déterminant\cite{PMTIooJjAmWR}]    \label{ThoBVIJooMkifod}
    Soit une application linéaire \( T\colon \eR^N\to \eR^N\). Alors pour tout borélien \( B\) de \( \eR^N\),
    \begin{equation}
        \lambda_N\big( T(B) \big)=| \det(T) |\lambda_N(B).
    \end{equation}
\end{theorem}
\index{déterminant!interprétation géométrique}

\begin{proof}
    Nous considérons la mesure positive \( \mu\) donnée par \( \mu(B)=\lambda_N\big( T(B) \big)\), qui est bien une mesure par la proposition \ref{PropJCJQooAdqrGA}. Cette mesure est invariante par translation parce que \( \lambda_N\) l'est :
    \begin{equation}
        \mu(B+a)=\lambda_N\big( T(B)+a \big)=\lambda_N\big( T(B) \big)=\mu(B).
    \end{equation}
    De plus, \( T(Q_0)\) est borné et nous notons \( \mu(Q_0)=C\). Nous avons \( \mu=C\lambda_N\) par le corollaire \ref{CorKGMRooHWOQGP}.

    \begin{subproof}
        \item[\( C(T_1T_2)=C(T_1)C(T_2)\)]
            Par définition, 
            \begin{equation}
                C(T_1T_2)\lambda_N(B)=\lambda_N\big( (T_1T_2)(B) \big)=\lambda_N\big( T_1(T_2B) \big)=C(T_1)\lambda_N\big( T_2(B) \big)=C(T_1)C(T_2)\lambda_N(B).
            \end{equation}
            Par conséquent la fonction \( C\) est multiplicative : 
            \begin{equation}
                C(T_1T_2)=C(T_1)C(T_2).
            \end{equation}
            Et en plus, \( C(\id)=1\).
        \item[Matrice diagonale]
            Nous considérons pour \( T=D\) l'application linéaire diagonale \( D=\diag(d_1,\ldots, d_N)\) qui fait
            \begin{equation}
                T(Q_0)=\mathopen[ 0 , d_1 \mathclose[\times \ldots\times \mathopen[  0, d_N \mathclose[
            \end{equation}
            La mesure de cela est \( |d_1\cdots d_N|\), ce qui nous donne
            \begin{equation}
                C(D)=| d_1\ldots d_N |=| \det(D) |.
            \end{equation}
        \item[Matrice orthogonale]
            Nous considérons maintenant \( T=U\) où \( U\) est une matrice orthogonale (\( UU^t=1\)). Une matrice orthogonale est une isométrie\footnote{Proposition \ref{PropKBCXooOuEZcS}.} qui conserve donc la boule unité : \( UB(0,1)=B(0,1)\). Nous avons
            \begin{equation}
                \lambda_N\big( B(0,1) \big)=\lambda_N\big( UB(0,1) \big)=C(U)\lambda_N\big( B(0,1) \big)
            \end{equation}
            par conséquent \( C(U)=1\), et \( 1\) est justement le déterminant de \( U\).
        \item[Matrice quelconque]
            Nous savons par le corollaire \ref{CorAWYBooNCCQSf} de la décomposition polaire que toute matrice peut être écrite sous la forme \( T=U_1DU_2\) où \( U_i\) sont orthogonales et \( D\) est diagonale. Donc \( C(T)=C(U_1)C(D)C(U_2)=\det(U_1)\det(D)\det(U_2)=\det(U_2DU_2)=\det(T)\) parce que le déterminant est multiplicatif (proposition \ref{PropYQNMooZjlYlA}\ref{ItemUPLNooYZMRJy}).
    \end{subproof}
\end{proof}

Ce théorème donne une interprétation géométrique du déterminant en tant que facteur de dilatation des volumes lors de l'utilisation d'une application linéaire. Si \( T\) est une application linéaire quelconque,
\begin{equation}
    \lambda_N\big( T(Q_0) \big)=| \det(T) |\lambda_N(Q_0)=| \det(T) |.
\end{equation}
Le déterminant de \( T\) est le volume de l'image du cube unité par l'application \( T\).

De la même façon, en utilisant l'application linéaire \( T(x)=ax\) nous avons pour tout borélien \( B\) :
\begin{equation}
    \lambda_N(aB)=a^N\lambda_N(B).
\end{equation}
Une dilatation d'un facteur \( a\) des longueurs provoque une multiplication par \( a^N\) des volumes.

%--------------------------------------------------------------------------------------------------------------------------- 
\subsection{Le théorème et sa démonstration}
%---------------------------------------------------------------------------------------------------------------------------

\begin{theorem}[Changement de variable\cite{VSMEooLwNLHd,PMTIooJjAmWR}]         \label{THOooUMIWooZUtUSg}
    Soient \( U\) et \( V\) des ouverts de \( \eR^N\) ainsi qu'un \( C^1\)-difféomorphisme \(\phi\colon U\to V\).
    \begin{enumerate}
        \item   \label{ItemVWYDooOzwnyfi}
            Si \( E\subset U\) est borélien, alors \( \phi(E)\) est borélien et
            \begin{equation}
                \lambda_N\big( \phi(E) \big)=\int_E| J_{\phi} |d\lambda_N,
            \end{equation}
            c'est à dire \( \phi^{-1}(\lambda_N)=| J_{\phi} |\cdot \lambda_N\).
        \item       \label{ITEMooEZUBooGBuDOS}
            Si \( f\colon V\to \mathopen[ 0 , +\infty \mathclose]\) est mesurable alors la fonction
            \begin{equation}
                (f\circ\phi)\times | J_{\phi} |\colon U\to \mathopen[ 0 , \infty \mathclose]
            \end{equation}
            l'est également et\footnote{L'intégrabilité d'une fonction est la définition \ref{DefTCXooAstMYl} qui stipule que l'intégrale de \( | f(x) |\) est finie. L'égalité proposée a un sens si les deux membres sont infinis. Il n'y a donc pas d'hypothèses d'intégrabilité obligatoire pour écrire une intégrale lorsque la fonction a des valeurs positives.}
            \begin{equation}        \label{EqRANEooQsFhbC}
                \int_Vfd\lambda_N=\int_U(f\circ\phi)(x)| J_{\phi}(x) |d\lambda_N(x).
            \end{equation}
        \item       \label{ITEMooAJGDooGHKnvj}
            Si \( f\colon V\to \eC\) est mesurable alors elle est intégrable si et seulement si \( (f\circ \phi)\times | J_{\phi} |\colon U\to \eC\) est intégrable. Si c'est le cas, alors nous avons encore la formule de changement de variables :
            \begin{equation}        \label{EQooLYAWooTArAZR}
                \int_Vfd\lambda_N=\int_U (f\circ \phi)| J_{\phi} |d\lambda_N.
            \end{equation}
    \end{enumerate}
\end{theorem}


\begin{proof}
    Attention : la preuve va être longue.
    \begin{enumerate}
        \item
            Le fait que \( \phi(E)\) soit borélien lorsque \( E\) l'est est la proposition \ref{PropRDRNooFnZSKt}. En ce qui concerne la formule annoncée, il faut travailler.
            \begin{subproof}
            \item[Inégalité dans un sens (cubes)]
                Nous commençons par prouver l'inégalité
                \begin{equation}        \label{EqooQCXXooSjGzks}
                    \lambda_N\big( \phi(Q) \big)\leq \int_Q| J_{\phi}(x) |dx
                \end{equation}
                pour tout \( Q\in \mQ\). On peut diviser le côté du cube \( Q\) en \( k\) éléments de longueurs égales. Le cube est alors divisé en \( k^N\) petits cubes d'intérieurs disjoints. Nous les nommons \( Q_i\) (\( i=1,\ldots, k^N\)) Nous avons alors
                \begin{equation}
                    \sum_i\lambda_N(Q_i)=\sum_i\lambda_N\big( \Int(Q_i) \big)=\lambda_N\big( \bigcup_i\Int(Q_i) \big)\leq \lambda_N(Q)\leq \sum_i\lambda_N(Q_i).
                \end{equation}
                La dernière inégalité est le fait que les intersections ne sont pas disjointes. Toutes ces inégalités sont en réalité des égalités et en particulier : \( \lambda_N(Q)=\sum_i\lambda_N(Q_i)\).

                Soit \( a\in Q_i\) et posons
                \begin{equation}
                    \begin{aligned}
                        \theta&\colon U&\to U \\
                        \theta&=(d\phi_{a})^{-1}\circ\phi 
                    \end{aligned}
                \end{equation}
                Cela appelle deux commentaires. D'abord l'application \( d\phi_{a}\colon U\to V\) est inversible parce que \( \phi\) est un difféomorphisme (lemme \ref{LemooTJSZooWkuSzv}). Ensuite, l'application \( \theta\) est la composée de \( (d\phi_{a})\) (qui est linéaire) et de \( \phi\) qui est de classe \( C^1\); donc \( \theta\) est de classe \( C^1\). Donc le lemme \ref{LemooJCEDooBRyjRg} s'applique. La différentielle de \( \theta\) n'est pas trop compliquée à écrire parce que nous avons la formule de différentielle d'une composée (théorème \ref{ThoAGXGuEt}) et le fait que \( (d\phi_{a})^{-1}\) qui est linéaire et donc sa propre différentielle (lemme \ref{LemooXXUGooUqCjmp}). Nous avons donc \( d\theta=(d\phi_a)^{-1}\circ d\phi\), et le lemme donne
                \begin{equation}
                    \lambda_N\left( (d\phi_a)^{-1}\phi(a) \right)\leq \sup_{s\in Q_i}\|    (d\phi_a)^{-1}\circ d\phi_s  \|^N\lambda_N(Q_i)
                \end{equation}
                Étant donné que \( (d\phi_a)^{-1}\) est une application linéaire, la proposition \ref{ThoBVIJooMkifod} s'applique, et donc
                \begin{equation}
                    \lambda_N\left( (d\phi_a)^{-1}\phi(a) \right)=| \det(d\phi_a)^{-1} |\lambda_N\big( \phi(a) \big).
                \end{equation}
                Le déterminant d'une application réciproque est donné par la proposition \ref{PropYQNMooZjlYlA}\ref{ItemooPJVYooYSwqaE} :
                \begin{equation}
                    \det\big( (d\phi_a)^{-1} \big)=\frac{1}{ \det\big( d\phi_a \big) }=\frac{1}{ J_{\phi}(a) }.
                \end{equation}
                Recollant les morceaux,
                \begin{equation}
                    \lambda_N\big( \phi(Q_i) \big)\frac{1}{ J_{\phi}(a) }\leq \sup_{s\in Q_i}\| (d\phi_a)^{-1}\circ d\phi_s \|^N\lambda_N(Q_),
                \end{equation}
                ou encore :
                \begin{equation}
                    \lambda_N\big( \phi(Q_i) \big)\leq | J_{\phi}(a) |\sup_{s\in Q_i}\| (d\phi_a)^{-1}\circ d\phi_s \|^N\lambda_N(Q_i).
                \end{equation}
                Vu que \( a\) et \( s\) sont proches l'un de l'autre (on peut choisir encore la taille du cube), nous pouvons espérer que \( (d\phi_a)^{-1}\) ne soit pas loin d'être l'inverse de \( d\phi_s\). Et c'est en effet le cas. Pour s'en assurer, remarquons que l'application
                \begin{equation}
                    d\phi\colon Q_i\to \aL(\eR^N,\eR^N)
                \end{equation}
                est continue et même uniformément continue parce que \( Q_i\) est compact. De plus la composition de différentielles étant un produit de matrices nous pouvons permuter la limite dans le calcul suivant :
                \begin{equation}
                    \lim_{s\to a}(d\phi_a)^{-1}\circ d\phi_s=(d\phi_a)^{-1}\circ\lim_{s\to a}d\phi_s=\mtu.
                \end{equation}
                Donc si \( \epsilon>0\) est donné, il existe \( \delta\) tel que pour tout \( s\in B(a,\delta)\), \( \| (d\phi_a)^{-1}\circ d\phi_s-\mtu \|\leq \epsilon\). En ce qui concerne les  normes, si \( \| A-\mtu \|\leq \epsilon\) alors \( \| A \|\leq \| A-\mtu \|+\| \mtu \|\leq \epsilon+1\).

                Cela étant dit, nous nous souvenons que nous avions découpé \( U\) en un nombre fini de cubes \( Q_i\) d'égales dimensions; il suffit de prendre \( k\) suffisamment grand pour que la diagonale des cubes sot plus petite que le minimum des \( \delta_i\). Avec un tel découpage,
                \begin{equation}
                    \sup_{s\in Q_i}\| (d\phi_a)^{-1}\circ d\phi_s \|\leq 1+\epsilon
                \end{equation}
                et par conséquent
                \begin{equation}        \label{EqooQRMNooZduAkX}
                    \lambda_N\big( \phi(Q_i) \big)\leq (1+\epsilon)^N| J_{\phi}(a_i) |\lambda_N(Q_i)
                \end{equation}
                où nous avons ajouté un indice \( i\) au point \( a\) pour nous rappeler que nous avons choisit \( a\in Q_i\). 

                Le théorème de la moyenne \ref{ThoooEZLGooMChwLT} appliqué à l'intégrale \( \int_{Q_i}| J_{\phi}(t) |d\lambda_N(t)\) donne l'existence d'un \( a_i\in Q_i\) tel que
                \begin{equation}
                    | J_{\phi}(a_i) |=\frac{1}{ \lambda_N(Q_i) }\int_{Q_i}| J_{\phi} |d\lambda_N.
                \end{equation}
                Ce point \( a_i\) vérifie l'inégalité \eqref{EqooQRMNooZduAkX} comme tout point de \( Q_i\). Nous sommons ces inégalités sur tous les \( i\) :
                \begin{subequations}
                    \begin{align}
                        \lambda_N\big( \phi(Q) \big)&\leq\sum_i\lambda_N\big( \phi(Q_i) \big)\\
                        &\leq (1+\epsilon^N\sum_i\left( \frac{1}{ \lambda_N(Q_i)\int_{Q_i}| J_{\phi} |d\lambda_N } \right)\lambda_N(Q_i)\\
                        &=(1+\epsilon)^N\sum_i\int_{Q_i}| J_{\phi} |d\lambda_N\\
                        &=(1+\epsilon)^N\int_Q| J_{\phi} |d\lambda_N
                    \end{align}
                \end{subequations}
                où nous avons utilisé le fait que \( \mtu_Q=\sum_i\mtu_{Q_i}\) presque partout. En prenant le limite \( \epsilon\to 0\) nous trouvons
                \begin{equation}
                    \lambda_N\big( \phi(Q) \big)\leq \int_Q| J_{\phi} |d\lambda_N.
                \end{equation}
                L'inégalité \eqref{EqooQCXXooSjGzks} est prouvée.
            \item[Inégalité pour les boréliens]

                Soit \( B\) un borélien de \( U\). Vu que \( U\) et \( V\) sont des ouverts de \( \eR^N\), les mesures de Lebesgue sur \( U\) et sur \( V\) sont les mêmes que celles sur \( \eR^n\)  par le corollaire \ref{CorooMJQYooFfwoTd}.

                Par les définitions \ref{PropooVXPMooGSkyBo} et \ref{PropJCJQooAdqrGA}, les applications \( \mu\) et \( n\) définies par \( \mu=\phi^{-1}(\lambda_N)\) et \( \nu=| J_{\phi} |\lambda_N\) sont des mesures positives sur \( U\) (de Borel, qui plus est). L'inégalité \eqref{EqooQCXXooSjGzks} à peine prouvée s'écrit \( \mu(Q)\leq \nu(Q)\) pour tout cube \( Q\). Le lemme \ref{LemooJYCGooIkkDVn} nous dit alors que l'inégalité tient pour tout borélien.

            \item[Inégalité dans l'autre sens]

                En utilisant la notation de la mesure image et du produit d'une mesure par une fonction\footnote{Définition \ref{PropJCJQooAdqrGA} et \ref{PropooVXPMooGSkyBo}}, nous pouvons écrire l'inégalité prouvée sous la forme \( \phi^{-1}(\lambda_N)\leq | J_{\phi} |\lambda_N\). En inversant les rôles de \( U\) et \( V\) (et donc de \( \phi\) et \( \phi^{-1}\)) nous avons aussi
                \begin{equation}
                    \phi(\lambda_N)\leq| J_{\phi^{-1}} |\lambda_N.
                \end{equation}
                En y appliquant \( \phi^{-1}\) et le lemme \ref{PropJCJQooAdqrGA},
                \begin{equation}        \label{EqooHJCHooVIaheI}
                    \lambda_N\leq \phi^{-1}\big( | J_{\phi^{-1}} |\lambda_N \big).    
                \end{equation}
                Nous prouvons à présent que \( \phi^{-1}\big( | J_{\phi^{-1}} |\cdot \lambda_N \big)=\Big( | J_{\phi^{-1}} |\circ\phi \Big)\cdot \phi^{-1}(\lambda_N)\) en appliquant à un borélien \( B\) de \(U\).
                D'une part 
                \begin{subequations}
                    \begin{align}
                        \phi^{-1}\big( | J_{\phi^{-1}} |\cdot\lambda_N \big)(B)&=\big( | J_{\phi^{-1}} |\cdot\lambda_N \big)\phi(B)\\
                        &=\int_{\phi(B)}| J_{\phi^{-1}} |d\lambda_N,
                    \end{align}
                \end{subequations}
                et d'autre part,
                \begin{subequations}
                    \begin{align}
                        \big( | J_{\phi^{-1}} |\circ\phi \big)\cdot\phi^{-1}(\lambda_N)B&=\int_{\eR^N}\mtu_B(x)\big( | J_{\phi^{-1}} |\circ\phi \big)(x)d\big( \phi^{-1}(\lambda_N) \big)(x)\\
                        &=   \int_{\eR^N}\mtu_B\big( \phi^{-1}(x) \big)\big( | J_{\phi^{-1}} |\circ\phi \big)\big( \phi^{-1}(x) \big)d\lambda_N(x)       \label{ooDKSWooXwQwgO}\\
                        &=\int_{\eR^N}\mtu_{\phi(B)}| J_{\phi^{-1}} |\\
                        &=\int_B| J_{\phi^{-1}} |d\lambda_N.
                    \end{align}
                \end{subequations}
                Justification :
                \begin{itemize}
                    \item Pour \eqref{ooDKSWooXwQwgO}, le théorème \ref{THOooVADUooLiRfGK}\ref{ItemooLAPYooUreDEl}.
                \end{itemize}

                L'équation \eqref{EqooHJCHooVIaheI} devient alors
                \begin{equation}
                    \lambda_N\leq \big( | J_{\phi^{-1}} |\circ\phi \big)\cdot \phi^{-1}(\lambda_N).
                \end{equation}
                Nous allons faire le produit de cette mesure par \( | J_{\phi} |\) en nous souvenant que \( J_{\phi}(x)=\det\big( d\phi_x \big)\). Par le lemme \ref{LemooTJSZooWkuSzv} nous avons aussi \(   (d\phi_x)^{-1}=d\phi^{-1}_{\phi(x)} \) et donc, par la propriété \ref{PropYQNMooZjlYlA}\ref{ITEMooZMVXooLGjvCy} du déterminant,
                \begin{equation}
                    J_{\phi}(x)=\frac{1}{ \det\big( d\phi^{-1}_{\phi(x)} \big) }=\frac{1}{ J_{\phi^{-1}}\big( \phi(x) \big) }.
                \end{equation}
                Nous avons
                \begin{equation}
                    | J_{\phi} |\cdot\lambda_N\leq | J_{\phi} |\cdot\big( | J_{\phi^{-1}} |\circ\phi \big)\cdot\phi^{-1}(\lambda_N).
                \end{equation}
                En utilisant la proposition \ref{PropooJMWAooDzfpmB}, il s'agit de multiplier la mesure \( \phi^{-1}(\lambda_N)\) par la fonction
                \begin{equation}
                    x\mapsto | J_{\phi}(x)J_{\phi^{-1}}\big( \phi(x) \big) |=1.
                \end{equation}
                Nous avons donc bien
                \begin{equation}
                    | J_{\phi} |\cdot \lambda_N\leq \phi^{-1}(\lambda_N),
                \end{equation}
                et donc l'égalité
                \begin{equation}
                    | J_{\phi} |\cdot\lambda_N=\phi^{-1}(\lambda_N),
                \end{equation}
                c'est à dire le point \ref{ItemVWYDooOzwnyfi}.
            \end{subproof}
        \item
            Le fait que la fonction proposée soit mesurable est le fait que la mesurabilité n'est pas affectée par produit et composition (propositions \ref{PROPooODDVooEEmmTX} et \ref{PROPooEFHKooARJBwW}), et le fait que pour les mêmes raisons, l'application \( J_{\phi}\colon U\to \eR\) est également mesurable. En ce qui concerne la formule nous allons la démontrer dans le cas de fonctions de plus en plus générales.
            \begin{subproof}
            \item[Pour les fonctions indicatrices]
                Soit \( B\) un borélien de \( U\), et considérons la fonction \( f=\mtu_{\phi(B)}\). Alors
                \begin{equation}    \label{EqYXRFooJEqVBH}
                        \int_V fd\lambda_N=\int_{\eR^N}\mtu_{\phi(B)}(y)\mtu_V(y)d\lambda_N(y)
                        =\int_{\eR^N}\mtu_{\phi(B)}d\lambda_N
                        =\lambda_N\big( \phi(B) \big).
                \end{equation}
                parce que \( V=\phi(U)\) et \( B\subset U\), donc \( \mtu_{\phi(B)}\mtu_{\phi(U)}=\mtu_{\phi(B)}\). D'autre part, pour calculer l'autre membre de \eqref{EqRANEooQsFhbC} nous remarquons que \( f=\mtu_{\phi(B)}=\mtu_B\circ\phi^{-1}\), ce qui donne
                \begin{equation}        \label{EqHWRQooKIfPTu}
                    \int_Uf\big( \phi(x) \big)| J_{\phi}(x) |d\lambda_N(x)=\int_U\mtu_B| J_{\phi} |d\lambda_N=\int_B| J_{\phi} |d\lambda_N.
                \end{equation}
                L'ensemble \( B\) étant borélien, il est extrêmement mesurable, ce qui fait que le point \ref{ItemVWYDooOzwnyfi} s'applique : les expressions \eqref{EqYXRFooJEqVBH} et \eqref{EqHWRQooKIfPTu} sont égales.

            \item[Pour les fonctions étagées]

                   Soit \( f\colon V\to \eR^+\) une fonction étagée :
                   \begin{equation}
                       f(x)=\sum_{i=1}^na_i\mtu_{A_i}(x)
                   \end{equation}
                   Nous pouvons faire le calcul suivant :
                   \begin{subequations}
                       \begin{align}
                           \int_Vfd\lambda_N&=\int_V\sum_ia_i\mtu_{A_i}d\lambda_N\\
                           &=\sum_ia_i\int_{V}\mtu_{A_i}d\lambda_N      \label{ooNESRooDuNUYF}\\
                           &=\sum_i\int_U(\mtu_{a_i}\circ\phi)(x)| J_{\phi}(x) |d\lambda_N(x)   \label{ooYXHSooKMPrIT}\\
                           &=\sum_ia_i\int_U\mtu_{\phi^{-1}(A_i)}| J_{\phi}(x) |d\lambda_N(x)\\
                           &=\int_V\underbrace{\sum_ia_i\mtu_{\phi^{-1}(A_i)}(x)}_{=(f\circ\phi)(x)}| J_{\phi}(x) |d\lambda_N(x)\\
                           &=\int_V(f\circ\phi)| J_{\phi} |d\lambda_N.
                       \end{align}
                   \end{subequations}
                   Justifications :
                   \begin{itemize}
                       \item Pour \eqref{ooNESRooDuNUYF} : linéarité de l'intégrale, théorème \ref{ThoooCZCXooVvNcFD}\ref{ITEMooBLEVooDznQTY}\footnote{Il est remarquable que nous n'utilisons cette linéarité que pour les fonction étagées.}
                       \item Pour \eqref{ooYXHSooKMPrIT} : le cas des fonctions indicatrices est utilisé pour chaque \( i\) entre \( 1\) et \( n\).
                   \end{itemize}

               \item[Fonction mesurable positive]
                   Soit \( f\colon V\to \mathopen[ 0 , \infty \mathclose]\). Par le théorème fondamental d'approximation \ref{THOooXHIVooKUddLi}, il existe une suite croissante de fonctions étagées et mesurables \( \varphi_n\colon V\to \mathopen[ 0 , \infty \mathclose[\) dont la limite ponctuelle est \( f\).  Nous avons alors le calcul suivant :
                       \begin{subequations}
                           \begin{align}
                               \int_Vfd\lambda_N&=\lim_{n\to \infty} \int_V\varphi_nd\lambda_N  \label{ooGMMFooXLHijj}\\
                               &=\lim_{n\to \infty} \int_U(\varphi_n\circ\phi)| J_{\phi} |d\lambda_N \label{ooWIFWooXELNUs}\\
                               &=\int_U\lim_{n\to \infty} (\varphi_n\circ\phi)| J_{\phi} |d\lambda_N \label{ooNKXNooUYeWKo}\\
                               &=\int_U(f\circ\phi)| J_{\phi} |d\lambda_N       \label{ooOAIDooAILHIB}.
                           \end{align}
                       \end{subequations}
                       Justifications :
                       \begin{itemize}
                           \item Pour \eqref{ooGMMFooXLHijj}, c'est le théorème de la convergence monotone \ref{ThoRRDooFUvEAN}.
                           \item Pour \eqref{ooWIFWooXELNUs}, c'est le présent théorème pour la fonction étagée \( \varphi_n\).
                           \item Pour \eqref{ooNKXNooUYeWKo}, c'est encore la convergence dominée, justifiée par le fait que \(  \varphi_n\circ\phi    \) est également une suite croissante : si \( x\in U\) alors \( \varphi_{n+1}\big( \phi(x) \big)\geq \varphi_n\big( \phi(x) \big)   \).\
                           \item Pour \eqref{ooOAIDooAILHIB}, c'est la limite ponctuelle \( \varphi_n\big( \phi(x) \big)\to f\big( \phi(x) \big)\).
                       \end{itemize}
            \end{subproof}
        \item
            La partie sur l'intégrabilité repose sur le fait que  \( | f |\circ\phi=| f\circ\phi |\). Ici \( | . |\) est le module et non une valeur absolue. Les faits suivants sont équivalents :
            \begin{itemize}
                \item la fonction \( f\colon V\to \eC\) est intégrable
                \item la fonction \( | f |\colon V\to \eR\) est intégtrable
                \item la fonction \( (| f |\circ\phi)| J_{\phi} |\colon U\to \eR\) est intégrable (par le point \ref{ITEMooEZUBooGBuDOS}).
                \item la fonction \( (f\circ\phi)| J_{\phi} |\colon U\to \eR\) est intégrable.
            \end{itemize}
            En ce qui concerne la formule, il s'agit seulement d'appliquer le point \ref{ITEMooEZUBooGBuDOS} aux parties positives, négatives, imaginaires et réelles de \( f\).
    \end{enumerate}
\end{proof}

Notons que la formule peut être écrite sous la forme
\begin{equation}        \label{EQooQKARooELPCFO}
    \langle f, g\rangle_V=\langle f\circ\phi, (g\circ\phi)| J |\rangle_U,
\end{equation}
qui est plus pratique lorsqu'on parle de produits scalaires. Pour rappel, \( \phi\colon U\to C\) est un \( C^1\)-difféomorphisme.

\begin{normaltext}
La formule de changement de variables peut être comprise de la façon suivante. Si $\phi$ est linéaire  alors le facteur $|J_{\phi}|$ est la mesure de l'image par $\phi$ d'une portion de $\eR^p$ de mesure $1$, sinon  $|J_{\phi}|$ est le rapport entre la mesure de l'image d'un élément infinitésimale de volume de $\eR^p$ et sa mesure originale. 

Soit $\phi(u,v)=g(u,v)e_1+h(u,v)e_2$ un difféomorphisme dans $\eR^2$. Soit $(x_0, y_0)$ l'image par $\phi$ de $(u_0,v_0)$. On considère le petit rectangle $R$ de sommets $(u_0,v_0)$, $(u_0+\Delta u,v_0)$, $(u_0+\Delta u,v_0+\Delta v)$ et $(u_0,v_0+\Delta v)$. L'image de $R$ n'est pas un rectangle en général, mais peut être bien approximée par le rectangle de sommets $(x_0,y_0)$, $(x_0 ,y_0)+ \phi_{u}\Delta u$, $(x_0 ,y_0)+\phi_{u}\Delta u +\phi_{v}\Delta v$ et  $(x_0 ,y_0)+ \phi_{v}\Delta v$ et son aire est $\| \phi_{u}\times \phi_{v}\| \Delta u\Delta v$. La valeur $|\phi_{u}\times \phi_{v}|$ est exactement $|J_{\phi}|$ 
\end{normaltext}

%--------------------------------------------------------------------------------------------------------------------------- 
\subsection{Exemples}
%---------------------------------------------------------------------------------------------------------------------------

\begin{example}
Soit $V$ la région trapézoïdale de sommets $(0,-1)$, $(1,0)$, $(2,0)$, $(0,-2)$, comme à la figure \ref{LabelFigZTTooXtHkcissLabelSubFigZTTooXtHkci0}. Calculons ensemble l'intégrale double  
\[
\int_{V}e^{\frac{x+y}{x-y}}\,dV,
\] 
avec le changement de variable $\psi(x,y)=(x+y,x-y)$. C'est à dire que nous considérons les nouvelles variables
\begin{subequations}
	\begin{numcases}{}
		u=x+y\\
		v=x-y.
	\end{numcases}
\end{subequations}
Il faut remarquer d'abord que le changement de variable proposé est dans le mauvais sens. On écrit alors $\phi(u,v)=\psi^{-1}(u,v)=\big((u+v)/2, (u-v)/2\big)$, c'est à dire
\begin{subequations}
	\begin{numcases}{}
		x=\frac{ u+v }{ 2 }\\
		y=\frac{ u-v }{2}.
	\end{numcases}
\end{subequations}
La région qui correspond à $V$ est $U$, le trapèze de sommets  $(-1,1)$, $(1,1)$, $(2,2)$ et $(-2,2)$, qu'on voit sur la figure \ref{LabelFigZTTooXtHkcissLabelSubFigZTTooXtHkci1} et qu'on décrit par
\[
U=\{ (u,v)\in\eR^2\,\vert\, 1\leq v\leq 2, \, -v\leq u\leq v\}.
\] 

% Celui-ci a été supprimée le 17 juillet 2014
%\ref{LabelFigexamplechangementvariables}
%\newcommand{\CaptionFigexamplechangementvariables}{Avant et après le changement de variables}
%\input{auto/pictures_tex/Fig_examplechangementvariables.pstricks}

%The result is on figure \ref{LabelFigZTTooXtHkci}. % From file ZTTooXtHkci
%See also the subfigure \ref{LabelFigZTTooXtHkcissLabelSubFigZTTooXtHkci0}
%See also the subfigure \ref{LabelFigZTTooXtHkcissLabelSubFigZTTooXtHkci1}
\newcommand{\CaptionFigZTTooXtHkci}{Avant et après le changement de variables}
\input{auto/pictures_tex/Fig_ZTTooXtHkci.pstricks}

On observe que $U$ est une région du premier type tandis que $V$ n'est pas du premier ou du deuxième type. Le déterminant de la  matrice  jacobienne de $\psi^{-1}$ est  $J_{\psi^{-1}}$,
\begin{equation}
 J_{\psi^{-1}}(u,v)= \left\vert\begin{array}{cc}
\frac{1}{2} & \frac{1}{2} \\
\frac{1}{2}  & -\frac{1}{2}
\end{array}\right\vert= -\frac{1}{2}.
\end{equation}
On a alors 
\[
\int_{V}e^{\frac{x+y}{x-y}}\,dV=\int_{U}e^{\frac{u}{v}}\,\frac{1}{2}\,dV=\int_1^2\int_{-v}^{v}e^{\frac{u}{v}}\,\frac{1}{2}\, du\,dv= \frac{3}{4}(e-e^{-1}).
\] 
\end{example}

Énormément d'exemples sont disponibles avec les coordonnées polaires et toutes leurs variations. Cependant les fonctions trigonométriques ne seront vues que plus tard. Ce sera pour \ref{SUBSECooUUAKooMSJHsL}.

%+++++++++++++++++++++++++++++++++++++++++++++++++++++++++++++++++++++++++++++++++++++++++++++++++++++++++++++++++++++++++++ 
\section{Théorème de Fubini-Tonelli et de Fubini}
%+++++++++++++++++++++++++++++++++++++++++++++++++++++++++++++++++++++++++++++++++++++++++++++++++++++++++++++++++++++++++++

Nous rappelons que \( \eR^n\) muni de la mesure de Lebesgue est un espace mesuré \( \sigma\)-fini, conformément à la définition \ref{DefBTsgznn}.

Le théorème de Fubini-Tonelli parle de fonctions réelles et non complexes, et même positives. Le truc est que ce théorème va servir de base pour construire les autres. Si nous avons une fonction à valeurs complexes, elle se décompose en parties réelles et imaginaires qui elles-mêmes se décomposent en parties positives et négatives. Au final, les preuves pour \( f\colon \Omega\to \eC\) se ramènent à appliquer quatre fois le théorème pour \( f\colon \Omega\to \bar \eR^+\).
\begin{theorem}[Fubini-Tonelli\cite{NBoIEXO}]\label{ThoWTMSthY}
    Soient \( (\Omega_i,\tribA_i,\mu_i)\) deux espaces mesurés \( \sigma\)-finis, et \( (\Omega,\tribA,\mu)\) l'espace produit. Soit une fonction \( f\colon \Omega_1\times \Omega_2\to \eR\) une fonction mesurable et positive (valant éventuellement \( \infty\) à certains endroits)
    Alors :
    \begin{enumerate}
        \item       \label{ITEMooUTMNooVIBdpP}
            Les fonction
            \begin{equation}        \label{EQooWLADooQwNhEy}
                F_1\colon x\mapsto \int_{\Omega_2}f(x,y)d\mu_2(y)
            \end{equation}
            et
            \begin{equation}
                F_2\colon y\mapsto \int_{\Omega_1}f(x,y)d\mu_1(x)
            \end{equation}
            sont mesurables.
        \item   \label{ITEMooFKQUooCoCOLV}
            Toutes les intégrales imaginables existent et sont égales :
            \begin{subequations}    \label{EqJRVtOGx}
                \begin{align}
                    \int_{\Omega_1\times \Omega_2}f(x,y)d(\mu_1\otimes \mu_2)(x,y)&=\int_{\Omega_1}\left[ \int_{\Omega_2}f(x,y)d\mu_2(y) \right]d\mu_1(x)\\
                &=\int_{\Omega_2}\left[ \int_{\Omega_1}f(x,y)d\mu_1(x) \right]d\mu_2(y)
                \end{align}
            \end{subequations}
            où tous les membres de l'égalité valent éventuellement \( +\infty\).
    \end{enumerate}
\end{theorem}
\index{théorème!Fubini-Tonelli}

\begin{proof}
    Commençons par prouver le théorème dans le cas d'une fonction caractéristique d'un ensemble mesurable : \( f(x,y)=\mtu_{A}(x,y)\) pour un certain ensemble \( A\subset \Omega_1\times \Omega_2\). Dans ce cas,
    \begin{equation}
        F_1(x)=\int_{\Omega_2}\mtu_A(x,y)d\mu_2(y)=\int_{\Omega_2}\mtu_{A_1(y)}(x)d\mu_2(y)=\mu_2\big( A_1(x) \big),
    \end{equation}
    et nous avons déjà vu au théorème \ref{ThoCCIsLhO} que cette fonction \( F_1\) était alors mesurable. En utilisant maintenant les égalités \eqref{EqDFxuGtH} ainsi que le fait que \( \mtu_A(x,y)=\mtu_{A_2(x)}(y)\) nous avons
    \begin{subequations}
        \begin{align}
            \int_{\Omega_1\times \Omega_2}\mtu_A(x,y)d(\mu_1\otimes \mu_2)(x,y)&=(\mu_1\otimes \mu_2)(A)\\
            &=\int_{\Omega_1}\mu_2\big( A_2(x) \big)d\mu_1(x)\\
            &=\int_{\Omega_1}\left[   \int_{\Omega_2}\mtu_{A_2(x)}(y)d\mu_2(y)  \right]d\mu_1(x)\\
            &=\int_{\Omega_1}\left[ \int_{\Omega_2}\mtu_A(x,y)d\mu_2(y) \right]d\mu_1(x).
        \end{align}
    \end{subequations}
    Le théorème étant valable pour les fonctions caractéristiques, il est valable pour les fonctions simples (définition \ref{DefBPCxdel}) par linéarité de l'intégrale.

    Si \( f\) n'est pas une fonction simple, alors la proposition \ref{THOooXHIVooKUddLi} nous donne une suite croissante de fonctions simples et positives convergeant ponctuellement vers \( f\). La partie du théorème sur les fonctions simples dit que pour chaque \( n\) l'intégrale
    \begin{equation}
        \int_{\Omega_1\times \Omega_2}f_n(x,y)d(\mu_1\otimes\mu_2)(x,y)
    \end{equation}
    peut être décomposée comme il faut en suivant la formule \eqref{EqJRVtOGx}. Il faut pouvoir permuter la limite et l'intégrale dans chacun de cas. D'abord le théorème de la convergence monotone \ref{ThoRRDooFUvEAN} appliqué à l'espace \( \Omega_1\times \Omega_2\) dit que
    \begin{equation}
        \lim_{n\to \infty} \int_{\Omega_1\times \Omega_2}f_n(x,y)d(\mu_1\otimes \mu_2)(x,y)= \int_{\Omega_1\times \Omega_2}f(x,y)d(\mu_1\otimes \mu_2)(x,y).
    \end{equation}
    Ensuite, pour chaque \( x\in\Omega_1\), les fonctions
    \begin{equation}
        \sigma_n(y)=\int_{\Omega_1}f_n(x,y)d\mu_1(x)
    \end{equation}
    forment une suite croissante de fonctions mesurables; nous leur appliquons encore le théorème de la convergence monotone :
    \begin{subequations}
        \begin{align}
            \lim_{n\to \infty} \int_{\Omega_2}\left[ \int_{\Omega_1}f_n(x,y)d\mu_1(x) \right]d\mu_2(y)&=\lim_{n\to \infty} \int_{\Omega_2}\sigma_n(y)d\mu_2(y)\\
            &=\int_{\Omega_2}\left[\lim_{n\to \infty} \int_{\Omega_1}f_n(x,y)d\mu_1(x)\right]d\mu_2(y)\\
            &=\int_{\Omega_2}\left[ \int_{\Omega_1}f(x,y)d\mu_1(x) \right]d\mu_2(y)
        \end{align}
    \end{subequations}
    où nous avons utilisé une seconde fois Beppo-Levi.
\end{proof}

\begin{remark}
    Les formules \eqref{EqJRVtOGx} sont bien, mais ne garantissent en aucun cas que \( f\in L^1(\Omega_1\times \Omega_2)\) : il faut encore que ces intégrales soient finies.
\end{remark}

\begin{corollary}[\cite{MesIntProbb}]           \label{CorTKZKwP}
    Soient \( (\Omega_i,\tribA_i,\mu_i)\) deux espaces mesurés \( \sigma\)-finis, et \( (\Omega,\tribA,\mu)\) l'espace produit\footnote{Définition \ref{DefUMlBCAO}.}. Soit une fonction mesurable \( f\colon \Omega\to \eR\text{ ou }\eC\). Alors les conditions suivantes sont équivalentes
    \begin{enumerate}
        \item   \label{ITEMooZRAXooTRDIlZ}
            \( f\in L^1(\Omega_1\times \Omega_2)\),
        \item       \label{ITEMooJMPLooZKwxQC}
            \begin{equation}
                \int_{\Omega_1}\left[ \int_{\Omega_2}| f |d\mu_2 \right]d\mu_1 <\infty,
            \end{equation}
        \item   \label{ITEMooLLBCooTRycwG}
            \begin{equation}
                \int_{\Omega_2}\left[ \int_{\Omega_1}| f |d\mu_1 \right]d\mu_2 <\infty.
            \end{equation}
    \end{enumerate}
\end{corollary}

\begin{proof}

    Nous commençons par supposer que \( f\) est à valeurs dans \( \eR\). La notation \( | f |\), pour l'instant,  dénote donc bien la valeur absolue et non le module.

    La fonction \( | f |\) est mesurable et positive par hypothèse et par le fait que si \( f\) est mesurable, alors \( | f |\) l'est également par le corollaire \ref{CORooNXYUooEcvDlP}. Le théorème \ref{ThoWTMSthY}\ref{ITEMooFKQUooCoCOLV} nous dit alors que les intégrales suivantes existent et sont égales :
    \begin{equation}        \label{EQooAIQGooNtBOuC}
            \int_{\Omega_1\times \Omega_2}| f |d(\mu_1\otimes \mu_2)=\int_{\Omega_1}\left[ \int_{\Omega_2}|f(x,y)|d\mu_2(y) \right]d\mu_1(x)
            =\int_{\Omega_2}\left[ \int_{\Omega_1}|f(x,y)|d\mu_1(x) \right]d\mu_2(y).
    \end{equation}
    Attention : rien ne dit encore que ces intégrales sont finies.

    \begin{subproof}
        \item[\ref{ITEMooZRAXooTRDIlZ} implique \ref{ITEMooJMPLooZKwxQC} et \ref{ITEMooLLBCooTRycwG}]
            Si \( f\in L^1(\Omega_1\times \Omega_2)\) alors \( | f |\) y est également. Cela implique que le membre de droite de \eqref{EQooAIQGooNtBOuC} est fini. Les deux autres sont alors également finis.
        \item[\ref{ITEMooJMPLooZKwxQC} ou \ref{ITEMooLLBCooTRycwG} implique \ref{ITEMooZRAXooTRDIlZ}]
            Les expressions à droite de \eqref{EQooAIQGooNtBOuC} sont finies. Donc celle de gauche également. Cele signifie que \( | f |\in L^1(\Omega_1\times \Omega_2)\). Par conséquent \( f\) est également dans \(L^1(\Omega_2\times \Omega_2) \).
    \end{subproof}

    Nous passons maintenant au cas où \( f\) est à valeurs dans \( \eC\). Nous décomposons 
    \begin{equation}
        f=f_R+if_I
    \end{equation}
    où \( f_R\) et \( f_I\) sont des fonctions réelles. Nous avons
    \begin{equation}        \label{EQooZEOAooIMwKwk}
        \int_{\Omega}| f |\leq \int_{\Omega}| f_R |+\int_{\Omega}| f_I |.
    \end{equation}
    Donc si \( f_R\) et \( f_I\) sont dans \( L^1(\Omega)\), la fonction \( f\) le sera aussi. De même,
    \begin{equation}
        \int_{\Omega}| f_R |\leq \int_{\Omega}| f |,
    \end{equation}
    qui donne l'inverse : si \( f\in L^1(\Omega)\) alors \( f_R,f_I\in L^1(\Omega)\). Bref, \( f\) est intégrable sur \( \Omega\) si et seulement si \( f_R\) et \( f_I\) le sont.

    Supposons que \( f\in L^1(\Omega_1\times \Omega_2)\). Alors
    \begin{subequations}
        \begin{align}
            \int_{\Omega_1}\left[ \int_{\Omega_2}| f |d\mu_2 \right]d\mu_1&\leq \int_{\Omega_1}\left[ \int_{\Omega_2}| f_R | \right]+\int_{\Omega_1}\left[ \int_{\Omega_2}| f_I | \right]<\infty
        \end{align}
    \end{subequations}
    où nous avons appliqué \ref{ITEMooZRAXooTRDIlZ} implique \ref{ITEMooJMPLooZKwxQC} aux fonctions \( f_R\) et \( f_I\) qui sont dans \( L^1(\Omega_1\times \Omega_2)\) parce que \( f\) y est.

    Dans l'autre sens, si
    \begin{equation}
        \int_{\Omega_1}\left[ \int_{\Omega_2}| f | \right]<\infty,
    \end{equation}
    alors en remplaçant \( | f |\) par \( | f_R |\) ou par \( | f_I |\) nous restons fini. En appliquant alors «\ref{ITEMooJMPLooZKwxQC} implique \ref{ITEMooZRAXooTRDIlZ}» nous trouvons que \( f_R\) et \( f_I\) sont dans \( L^1(\Omega_1\times \Omega_2)\). Et cela implique que \( f\in L^1(\Omega_1\times \Omega_2)\).
\end{proof}

\begin{theorem}[Fubini\cite{MesIntProbb}]\label{ThoFubinioYLtPI}
    Soient \( (\Omega_i,\tribA_i,\mu_i)\) deux espaces mesurés \( \sigma\)-finis, et \( (\Omega,\tribA,\mu)\) l'espace produit. Soit 
    \begin{equation}
        f\in L^1\big( (\Omega,\tribA),\eC \big),
    \end{equation}
    c'est à dire une fonction à valeurs mesurable et intégrable sur \( \Omega\). Alors :
    \begin{enumerate}
        \item       \label{ITEMooVFGWooZTePQS}
            Pour presque tout \( x\in \Omega_1\), la fonction \( y\mapsto f(x,y)\) est \( L^1(\Omega_2)\).
        \item       \label{ITEMooCYMKooUdizni}
            Si nous posons
            \begin{equation}
                \varphi_f(x)=\int_{\Omega_2}f(x,y)d\mu_2(y);
            \end{equation}
            alors \( \varphi_f\in L^1(\Omega_1)\).
        \item   \label{ItemQMWiolgiii}
            Nous avons la formule d'inversion d'intégrale
            \begin{subequations}
                \begin{align}
                \int_{\Omega}fd(\mu_1\otimes \mu_2)&=\int_{\Omega_1}\varphi_fd\mu_1\\
                &=\int_{\Omega_1}\left[ \int_{\Omega_2}f(x,y)d\mu_2(y) \right]d\mu_1(x)\\
                &=\int_{\Omega_2}\left[ \int_{\Omega_1}f(x,y)d\mu_1(x) \right]d\mu_2(y).
                \end{align}
            \end{subequations}
    \end{enumerate}

\end{theorem}
\index{théorème!Fubini!espace mesuré}

\begin{proof}
    Nous commençons par supposer que \( f\) est à valeurs réelles : \( f\in L^1\big( (\Omega_1\times \Omega_2,\tribA_1\otimes\tribA_2 ),\eR\big)\). Nous décomposons la fonction \( f\) en parties positives et négatives : \( f=f^+-f^-\) avec \( f^+\) et \( f^-\) positives ou nulles. Nous avons évidemment
    \begin{equation}
        \int_{\Omega_1\times \Omega_2}| f^+ |\leq \int_{\Omega_1\times \Omega_2}| f |<\infty.
    \end{equation}
    Donc \( f^+\) et \( f^-\) sont des éléments de \( L^1(\Omega_1\times \Omega_2)\). 
    
    \begin{subproof}
    \item[Pour \ref{ITEMooVFGWooZTePQS}]

    Nous posons
    \begin{equation}
        \varphi_{f^+}(x)=\int_{\Omega_2}f^{+}(x,y)d\mu_2(y)
    \end{equation}
    pour tous les \( x\in \Omega_1\) pour lesquels cette intégrale est bien définie. Vu que \( f^+\) est positive et mesurable, le théorème de Fubini-Tonelli \ref{ThoWTMSthY}\ref{ITEMooUTMNooVIBdpP} s'applique donc pour nous dire que \( \varphi_{f^+}\) est mesurable.

    De plus le résultat \eqref{EqJRVtOGx} appliqué à \( f^+\) donne
    \begin{equation}        \label{EQooSETWooRwkCuW}
        \int_{\Omega_1}\varphi_{f^+}d\mu_1=\int_{\Omega_1\times \Omega_2}f^+d(\mu_1\otimes \mu_2)<\infty.
    \end{equation}
    Le fait que le tout soit fini est une conséquence du fait déjà mentionné que \( f^+\in L^1(\Omega_1\times \Omega_2\). Vu que \( \varphi_{f^+}\) est une fonction positive, l'inégalité \eqref{EQooSETWooRwkCuW} signifie que \( \varphi_{f^+}\in L^1(\Omega_1,\mu_1)\).

    En particulier, \( \varphi_{f^+}(x)<\infty\) pour presque tout \( x\in\Omega_1\). C'est à dire pour presque tout \( x\in \Omega_1\) :
    \begin{equation}
        \int_{\Omega_2}f^+(x,y)d\mu_2(y)<\infty,
    \end{equation}
    et sachant que \( f^+\geq 0\) nous avons \( f^+(x,\cdot)\in L^1(\Omega_2)\) pour presque tout \( x\).
    
        \item[Pour \ref{ITEMooCYMKooUdizni}]

            Partout où \( \varphi_{f^+}\) et \( \varphi_{f^-}\) sont finies nous avons
            \begin{equation}
                \varphi_f=\varphi_{f^+}-\varphi_{f^-},
            \end{equation}
            et comme cela a lieu presque partout, nous pouvons considérer une partie mesurable \( A\subset \Omega_1\) telle que \( \mu_1(A)=0\) et \( \varphi_f(x)=\varphi_{f^+}(x)-\varphi_{f^-}(x)\) pour tout \( x\) hors de \( A\). Bref, nous posons
            \begin{equation}
                g(x)=\begin{cases}
                    \varphi_{f^+}-\varphi_{f^-}(x)    &   \text{si } x\in A^c\\
                    0    &    \text{si } x\in A.
                \end{cases}
            \end{equation}
            Cette fonction \( g\) est mesurable et \( g=\varphi_f\) presque partout. De plus
            \begin{equation}
                \int_{\Omega_1}| g |d\mu_1   = \int_{A^c}| g |\leq \int_{A^c}\varphi_{f^+}+\int_{A^c}\varphi_{f^-}<\infty.
            \end{equation}
            La dernière inégalité est le fait que \( \varphi_{f^{\pm}}\) sont dans \( L^1(\Omega_1)\). Et notons au passage que nous aurions pu laisser toutes les intégrales sur \( \Omega_1\) sans faire de précisions sur la distinction entre \( \Omega_1\) et \( A^c\) parce que la partie de \( \Omega_1\) sur laquelle \( \varphi_{f^{\pm}}\) sont infinies est trop petite pour changer la valeur de l'intégrale.

            Nous avons donc \( g\in L^1(\Omega_1)\), et par conséquent également \( \varphi_f\in L^1(\Omega_1)\) parce que ces deux fonctions sont égales presque partout (les classes sont égales).

        \item[Pour \ref{ItemQMWiolgiii}]

            En utilisant l'équation \eqref{EQooSETWooRwkCuW} nous avons
            \begin{subequations}
                \begin{align}
                    \int_{\Omega_1}\varphi_fd\mu_1&=\int gd\mu_1=\int_{\Omega_1}\varphi_{f^+}-\int_{\Omega_1}\varphi_{f^-}\\
                    &=\int_{\Omega_1\times }f^+d\mu-\int_{\Omega_1\times \Omega_2}f^-d\mu\\
                    &=\int_{\Omega_1\times \Omega2}fd\mu.
                \end{align}
            \end{subequations}
            Et toutes ces intégrales sont finies.
    \end{subproof}

    Et c'est maintenant que nous considérons le cas complexe. Nous décomposons \( f=f_R+if_I\) avec des fonctions réelles \( f_R\) et \( f_I\). Comme déjà mentionné autour de \eqref{EQooZEOAooIMwKwk}, les fonctions \( f_R\) et \( f_I\) sont intégrables. Nous leur appliquons le théorème.

    Les valeurs de \( x\) pour lesquelles \( f_R(x,\cdot)\) et \( f_I(x,\cdot)\) ne sont pas dans \( L^1(\Omega_2)\) forment un ensemble de mesure nulle, nommons le \( A\). En posant
    \begin{equation}
        g(x,y)=\begin{cases}
            f_R(x,y)+if_I(x,y)    &   \text{si } x\in A^c\\
            0    &    \text{si } x\in A,
        \end{cases}
    \end{equation}
    nous avons que \( g(x,\cdot)\) est intégrable pour tout \( x\in A^c\). Vu que pour ces valeurs de \( x\) nous avons \( g(x,y)=f(x,y)\) nous en déduisons que pour \( x\in A^c\) nous avons aussi \( f(x,\cdot)\in L^1(\Omega_2)\).

    Les autres points se traitent de la même façon\quext{Attention : je n'ai pas vérifié explicitement. C'est juste une intuition.}.
\end{proof}

\begin{normaltext}      \label{NORMooKIRJooPvyPWQ}
    En pratique, il n'est pas toujours évident qu'une fonction soit intégrable sur \( \Omega_1\times \Omega_2\). Pour permuter des intégrales sur une fonction à deux paramètres nous faisons comme suit.
    \begin{enumerate}
        \item
            Nous testons l'intégrabilité en chaîne de \( | f |\), et si c'est bon, le corollaire \ref{CorTKZKwP} nous donne \( f\in L^1(\Omega_1\times \Omega_2)\).
        \item
            Nous utilisons le théorème de Fubini \ref{ThoFubinioYLtPI} pour séparer et permuter les intégrales comme des ingénieurs.
    \end{enumerate}

    Si la fonction \( (x,y)\mapsto f(x)g(y)\) satisfait aux hypothèse du théorème de Fubini alors
    \begin{equation}    \label{EqTJEEsJW}
        \int_{\Omega_1\times \Omega_2} f(x)g(y)dx\otimes dy=\left( \int_{\Omega_1}f(x)dx \right)\left( \int_{\Omega_2}g(y)dy \right).
    \end{equation}
    Le théorème de Fubini est souvent utilisé sous cette forme.
    
\end{normaltext}

\begin{example}[Nécessité d'avoir des mesures \( \sigma\)-finies]
    Nous montrons que le théorème ne tient pas si une des deux mesures n'est pas \( \sigma\)-finie. Soit \( I=\mathopen[ 0 , 1 \mathclose]\). Nous considérons l'espace mesuré
    \begin{equation}
        (I,\Borelien(I),\lambda)
    \end{equation}
    où \( \Borelien(I)\) est la tribu des boréliens sur \( I\) et \( \lambda\) est la mesure de Lebesgue (qui est $\sigma$-finie). D'autre part nous considérons l'espace mesuré
    \begin{equation}
        (I,\partP(I),m)
    \end{equation}
    où \( \partP(I)\) est l'ensemble des parties de \( I\) et \( m\) est la mesure de comptage. Cette dernière n'est pas $\sigma$-finie parce que les seuls ensembles de mesure finie pour la mesure de comptage sont des ensembles finis, or une union dénombrable d'ensemble finis ne peut pas recouvrir l'intervalle \( I\).

    Nous allons montrer que dans ce cadre, l'intégrale de la fonction indicatrice de la diagonale sur \( I^2\) ne vérifie pas le théorème de Fubini. Étant donné que \( \Borelien(I)\subset\partP(I)\) nous avons
    \begin{equation}
        \Borelien(I^2)\subset\Borelien(I)\otimes\partP(I).
    \end{equation}
    Soit \( \Delta=\{ (x,x)\tq x\in I \}\). La fonction
    \begin{equation}
        \begin{aligned}
            g\colon I^2&\to \eR \\
            (x,y)&\mapsto x-y 
        \end{aligned}
    \end{equation}
    est continue et \( \Delta=g^{-1}(\{ 0 \})\) est donc fermé dans \( I^2\). L'ensemble \( \Delta\) est donc un borélien de \( I^2\) et par conséquent un élément de la tribu \( \Borelien(I)\otimes\partP(I)\). La fonction indicatrice \( \mtu_{\Delta}\) est alors mesurable pour l'espace mesuré
    \begin{equation}
        (I\times I,\Borelien(I)\otimes\partP(I),\lambda\otimes m).
    \end{equation}
    Pour \( x\) fixé nous avons
    \begin{equation}
        \mtu_{\Delta}(x,y)=\begin{cases}
            1    &   \text{si } y= x\\
            1    &    \text{si } y\neq x
        \end{cases}=\mtu_{\{ x \}}(y),
    \end{equation}
    et donc
    \begin{subequations}
        \begin{align}
            A_1&=\int_I\left( \int_I\mtu_{\Delta}(x,y)dm(y) \right)d\lambda(x)\\
            &=\int_I\left( \int_I\mtu_{\{ x \}}(y)dm(y) \right)d\lambda(x)\\
            &=\int_I\Big( m(\{ x \}) \Big)d\lambda(x)\\
            &=\int_I 1d\lambda(x)\\
            &=1.
        \end{align}
    \end{subequations}
    Par contre le support de \( \mtu_{\Delta}\) étant de mesure nulle pour la mesure de Lebesgue, nous avons
    \begin{equation}
        \int_I\mtu_{\Delta}(x,y)d\lambda(x)=0
    \end{equation}
    et par conséquent
    \begin{equation}
        A_2=\int_I\left( \int_I\mtu_{\Delta}(x,y)d\lambda(x) \right)dm(y)=0.
    \end{equation}
    Nous voyons donc que le théorème de Fubini ne s'applique pas.
\end{example}

\begin{example}  \label{EXooLUFAooGcxFUW}
    Nous nous proposons de calculer l'intégrale suivante en utilisant le théorème de Fubini :
    \begin{equation}
        G=\int_{\eR} e^{-x^2}dx=\sqrt{ \pi }
    \end{equation}
    alors que la fonction \( x\mapsto  e^{-x^2}\) n'a pas de primitives parmi les fonctions élémentaires.

    Nous allons le faire de deux façons. Une première directe en utilisant le théorème de Fubini sur un domaine non borné, et une seconde en utilisant Fubini sur un domaine borné, et en passant à la limite ensuite.

    \begin{subproof}
        \item[Fubini, domaine non borné]

    Par symétrie nous pouvons nous contenter de calculer
    \begin{equation}
        G_+=\int_0^{\infty} e^{-x^2}dx.
    \end{equation}
    L'astuce est de passer par l'intermédiaire
    \begin{subequations}
        \begin{align}
            H&=\int_{\eR^+\times\eR^+} e^{-(x^2+y^2)}dxdy       \label{EqIntFausasub}\\
            &=\int_{\eR^+}\left( \int_{\eR^+} e^{-x^2} e^{-y^2}dx \right)dy\\
            &=\left( \int_{\eR^+} e^{-x^2} dx\right)^2\\
            &=G_+^2
        \end{align}
    \end{subequations}
    L'intégrale \eqref{EqIntFausasub} se calcule en passant aux coordonnées polaires et le résultat est \( H=\frac{ \pi }{ 4 }\). Nous avons alors \( G=\frac{ \sqrt{\pi} }{ 2 }\) et
    \begin{equation}
        \int_{\eR} e^{-x^2}=\sqrt{\pi}.
    \end{equation}

        \item[Fubini, domaine borné, puis limite]
    Une variante, qui n'applique pas Fubini sur un domaine non borné. Nous commençons par écrire
\begin{equation}
	I=\int_{-\infty}^{+\infty} e^{-x^2} dx := \lim_{R \to +\infty} \int_{-R}^{+R} e^{-x^2} dx 
\end{equation}
et puis nous faisons le calcul
\begin{equation}		\label{EqCalculInteeemoisxcar}
	\begin{aligned}[]
		I^2 &= \lim_{R \to +\infty} \left( (\int_{-R}^{+R} e^{-x^2} dx)( \int_{-R}^{+R} e^{-y^2} dy) \right) \\
		&= \lim_{R \to +\infty} \left( \iint_{K_R}e^{-(x^2+y^2)} dx dy \right) \\
		&= \lim_{R \to +\infty} \left( \iint_{C_R}e^{-(x^2+y^2)} dx dy \right) 
	\end{aligned}
\end{equation}
où $K$ est le carré de demi côté $R$ centré à l'origine et de côtés parallèles aux axes et $C_R$ est le cercle de rayon $R$ centré à l'origine.

	La première étape à justifier est simplement l'application de Fubini. Pour le passage de l'intégrale du carré vers le cercle, définissons
	\begin{equation}
		\begin{aligned}[]
			I_K(r)&=\int_{K_r}f,&I_C(r)&=\int_{C_r}f
		\end{aligned}
	\end{equation}
	où $K_r$ est la carré de demi côté $r$ et $C_r$ est le cercle de rayon $r$. Le demi côté du carré inscrit à $C_r$ est $\sqrt{2}$, donc pour tout $r$ nous avons
	\begin{equation}
		I_K(\sqrt{2}r)\leq I_C(r)<I_K(r),
	\end{equation}
	et en prenant la limite, nous avons évidement
	\begin{equation}
		\lim_{r\to \infty}I_K(\sqrt{2}r)=\lim_{r\to\infty}I_K(r),
	\end{equation}
	et donc cette limite est également égale à $\lim_{r\to\infty}I_C(t)$.

    Il ne reste qu'à calculer la dernière intégrale sur le cercle en passant aux coordonnées polaires :
	\begin{equation}
        \iint_{C_R} e^{-(x^2+y^2)}dxdy=\int_0^{2\pi}d\theta\int_0^Rr e^{-r^2}dr=\pi(1- e^{-R^2}).
	\end{equation}
	La limite donne $\pi$, nous en déduisons que
    \begin{equation}    \label{EqFDvHTg}
		\int_{-\infty}^{\infty} e^{-x^2}dx=\sqrt{\pi}.
	\end{equation}
    \end{subproof}

\end{example}

Le théorème de Fubini-Tonelli nous permet également d'inverser des sommes et des séries. En effet une somme n'est rien d'autre qu'une intégrale pour la mesure de comptage :
\begin{equation}
    \sum_{n=0}^{\infty}a_n=\int_{\eN}a_ndm(n).
\end{equation}
La proposition suivante montre comment il faut faire.

\begin{proposition}\label{PropInversSumIntFub}  
    Soient les espaces mesurés \( (\eN,\partP(\eN),m)\), \( (\eR^n,\Borelien(\eR^n),\lambda)\) où \( \lambda\) est la mesure de Lebesgue ainsi qu'une suite de fonctions positives \( f_n\colon \eR^d\to \eR\). Nous supposons de plus que la fonction \( f_n\) soit intégrable pour tout \( n\) et que les résultats forment une suite sommable. Alors
    \begin{equation}   
        \sum_{n=0}^{\infty}\int_{\eR^n}f_n(x)dx=\int_{\eR^d}\sum_{n\in \eN}f_n(x)dx.
    \end{equation}
\end{proposition}
\index{mesure!de comptage}
\index{permuter!intégrale!et série}

\begin{proof}
    Nous pouvons la récrire le membre de gauche sous la forme
    \begin{equation}
        \int_{\eN}\left( \int_{\eR^n}f(n,x)dx \right)dm(n)
    \end{equation}
    avec la notation évidente \( f(n,x)=f_n(x)\). Prouvons que la fonction \( f\colon \eN\times\eR^d\to \eR\) ainsi définie est une fonction mesurable pour l'espace mesuré
    \begin{equation}
        \big( \eN\times\eR^d,\partP(\eN)\otimes\Borelien(\eR^d),m\otimes\lambda \big).
    \end{equation}
    Si \( A\subset\eR\), nous avons
    \begin{equation}
        f^{-1}(A)=\bigcup_{n\in\eN}\{ n \}\times f_n^{-1}(A).
    \end{equation}
    Chacun des ensembles dans l'union appartient à la tribu \( \partP(\eN)\times\Borelien(\eR^d)\) tandis que les tribus sont stables sous les unions dénombrables. La fonction \( f\) est donc mesurable. Comme nous avons supposé que \( f\) était positive, le théorème de Fubini-Tonelli s'applique et nous avons
    \begin{equation}
        \int_{\eR^d}\left( \int_{\eN}f(n,x)dm(n) \right)dx=\int_{\eR^d}\sum_{n\in \eN}f_n(x)dx.
    \end{equation}
\end{proof}

\begin{theorem}[Fubini]\label{ThoFubini}
Soit $(x,t)\mapsto f(x,y)\in\bar \eR$ une fonction intégrable sur $B_n\times B_m\subset\eR^{n+m}$ où $B_n$ et $B_m$ sont des ensembles mesurables de $\eR^n$ et $\eR^m$. Alors :
\begin{enumerate}
\item pour tout $x\in B_n$, sauf éventuellement en les points d'un ensemble $G\subset B_n$ de mesure nulle, la fonction $y\in B_m\mapsto f(x,y)\in\bar\eR$ est intégrable sur $B_m$
\item
la fonction
\begin{equation}
	x\in B_n\setminus G\mapsto\int_{B_m}f(x,y)dy\in\eR
\end{equation}
est intégrable sur $B_n\setminus G$

\item 
On a
\begin{equation}
	\int_{B_n\times B_m}f(x,y)dxdy=\int_{B_n}\left( \int_{B_m}f(x,y)dy \right)dx.
\end{equation}

\end{enumerate}
\end{theorem}
\index{théorème!Fubini!dans $ \eR^n$}
\index{Fubini!théorème!dans $ \eR^n$}

Notons en particulier que si $f(x,y)=\varphi(x)\phi(y)$, alors $\int_{B_m}\varphi(y)dy$ est une constante qui peut sortir de l'intégrale sur $B_n$, et donc
\begin{equation}		\label{EqFubiniFactori}
	\int_{B_n\times B_m}\varphi(x)\phi(y)dxdy=\int_{B_n}\varphi(x)dx\int_{B_m}\phi(y)dy.
\end{equation}

