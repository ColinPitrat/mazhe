% This is part of Mes notes de mathématique
% Copyright (c) 2011-2019
%   Laurent Claessens
% See the file fdl-1.3.txt for copying conditions.

%---------------------------------------------------------------------------------------------------------------------------
\subsection{Approximation}
%---------------------------------------------------------------------------------------------------------------------------

\begin{lemma}[Théorème fondamental d'approximation \cite{TribuLi}]      \label{LempTBaUw}
    Soit \( \Omega\) un espace mesurable et \( f\colon \Omega\to \mathopen[ 0 , \infty \mathclose]\) une application mesurable. Alors il existe une suite croissante d'applications étagées \( \varphi_n\colon \Omega\to \eR^+\) dont la limite est \( f\).

    De plus si \( f\) est bornée, la convergence est uniforme.
\end{lemma}

\begin{theorem}[\cite{HilbertLi}]       \label{ThoJsBKir}
    Soit \( I\) un intervalle de \( \eR\). L'espace \( \swD(I)\)\nomenclature[Y]{\( C_c(I)\)}{fonctions continues à support compact dans \( I\)} des fonctions continues à support compact sur \( I\) est dense dans \( L^2(I)\).
\end{theorem}
Ce théorème sera généralisé à tous les \( L^p(\eR^d)\) par le théorème~\ref{ThoILGYXhX}. Cependant \( L^p\) n'étant pas un Hilbert, il faudra travailler sans produit scalaire.

\begin{proof}
    Soit \( g\in L^2(I)\) une fonction telle que \( g\perp f\) pour toute fonction \( f\in C_c(I)\). Nous avons donc
    \begin{equation}
        \langle f, g\rangle =\int_If\bar g=0.
    \end{equation}
    En passant éventuellement aux composantes réelles et imaginaires nous pouvons supposer que les fonctions sont toutes réelles. Nous décomposons \( g\) en parties positives et négatives : \( g=g^+-g^-\). Notre but est de montrer que \( g^+=g^-\), c'est-à-dire que \( g\) est nulle. La proposition~\ref{PropqiWonByiBmc} conclura que \( C_c(I)\) est dense dans \( L^2(I)\).

    Soit un intervalle \( \mathopen[ a , b \mathclose]\subset I\) et une suite croissante de fonctions \( f_n\in C_c(I)\) qui converge vers \( \mtu_{\mathopen[ a , b \mathclose]}\). Par hypothèse pour chaque \( n\) nous avons
    \begin{equation}
        \int_If_ng^+=\int_I f_ng^-.
    \end{equation}
    La suite étant croissante, le théorème de la convergence monotone (théorème~\ref{ThoRRDooFUvEAN}) s'applique et nous avons
    \begin{equation}
        \lim_{n\to \infty} \int_I f_ng^+=\int_a^bg^+,
    \end{equation}
    de telle sorte que nous ayons, pour tout intervalle \( \mathopen[ a , b \mathclose]\subset I\) l'égalité
    \begin{equation}        \label{EqYlErAM}
        \int_a^bg^+=\int_a^bg^-.
    \end{equation}
    De plus ces intégrales sont finies parce que
    \begin{equation}
        \int_a^b g^+\leq\int_a^b| g |=\int_I| g |\mtu_{\mathopen[ a , b \mathclose]}=\langle | g |, \mtu_{\mathopen[ a , b \mathclose]}\rangle \leq \| g \|_{L^2}\sqrt{b-a}<\infty
    \end{equation}
    par l'inégalité de Cauchy-Schwarz.

    Soit maintenant un ensemble mesurable \( A\subset I\). La fonction caractéristique \( \mtu_A\) est mesurable et il existe une suite croissante de fonctions étagées \( (\varphi_n)\) convergente vers \( a\) par le lemme~\ref{LempTBaUw}. À multiples près, les fonctions \( \varphi_n\) sont des sommes de fonctions caractéristiques du type \( \mtu_{\mathopen[ a , b \mathclose]}\), par conséquent, en vertu de \eqref{EqYlErAM} nous avons
    \begin{equation}
        \int_I\varphi_ng^+=\int_I\varphi_ng^-.
    \end{equation}
    Une fois de plus nous pouvons utiliser le théorème de la convergence monotone et obtenir
    \begin{equation}
        \int_Ag^+=\int_A g^-
    \end{equation}
    pour tout ensemble mesurable \( A\subset I\). Si nous notons \( dx\) la mesure de Lebesgue, les mesures \( g^+dx\) et \( g^-dx\) sont par conséquent égales et dominées par \( dx\). Par le corollaire~\ref{CorZDkhwS} du théorème de Radon Nikodym, les fonctions \( g^+\) et \( g^-\) sont égales.
\end{proof}

%+++++++++++++++++++++++++++++++++++++++++++++++++++++++++++++++++++++++++++++++++++++++++++++++++++++++++++++++++++++++++++
\section{Convolution}
%+++++++++++++++++++++++++++++++++++++++++++++++++++++++++++++++++++++++++++++++++++++++++++++++++++++++++++++++++++++++++++


\begin{definition}      \label{DEFooHHCMooHzfStu}
    Pour toutes fonctions \( f,g\colon \eR^n\to \eC\) et pour tout \( x\in \eC\) tels que l'intégrale de droite ait un sens\footnote{Attention divlgâchi : ce sera le cas pour \( f,g\in L^1(\eR^n)\) par le théorème \ref{THOooMLNMooQfksn}.}, nous définissons
    \begin{equation}
        (f*g)(x)=\int_{\eR^n} f(y)g(x-y)dy.
    \end{equation}
    L'éventuelle fonction \( f*g\) ainsi définie est le \defe{produit de convolution}{produit!de convolution} de \( f\) et \( g\).
\end{definition}

Le théorème qui permet de dire que le produit de convolution n'est pas tout à fait ridicule est le suivant.

\begin{theorem}[\cite{MesIntProbb}]     \label{THOooMLNMooQfksn}
    Soient \( f,g\in L^1(\eR^n)\).
    \begin{enumerate}
        \item
            Pour presque tout \( x\in \eR^n\), la fonction
            \begin{equation}
                y\mapsto g(x-y)f(y)
            \end{equation}
            est dans \( L^1(\eR^n)\).
        \item
            \( f*g\in L^1(\eR^n)\).
        \item
            \( \| f*g \|_1\leq \| f \|_1\| g \|_1\).
    \end{enumerate}
\end{theorem}

\begin{proposition}     \label{PROPooNBHNooInwoar}
    L'ensemble \( L^1(\eR^n)\) devient alors une algèbre de Banach.
\end{proposition}

\begin{lemma}
    Le produit de convolution est commutatif : \( f*g=g*f\).
\end{lemma}

\begin{proof}
    Le théorème de Fubini (théorème~\ref{ThoFubinioYLtPI}) permet d'écrire
    \begin{equation}
        (f*g)(x)=\int_{\eR^n}f(y)g(x-y)dy=\int_{-\infty}^{\infty}dy_1\ldots \int_{-\infty}^{\infty}dy_nf(y)g(x-y).
    \end{equation}
    En effectuant le changement de variable \( z_i=x_i-y_i\) dans chacune des intégrales nous obtenons
    \begin{equation}
        (f*g)(x)=\int_{\eR^n}g(z)f(x-z)dz=(g*f)(x).
    \end{equation}
    Attention : on pourrait croire qu'un signe apparaît du fait que \( z=x-y\) donne \( dz=-dy\). Mais en réalité, l'intégrale \( \int_{-\infty}^{+\infty}\) devient par le même changement de variables \( \int_{+\infty}^{-\infty}\) qui redonne un nouveau signe au moment de remettre dans l'ordre.
\end{proof}

La proposition suivante est une conséquence de l'inégalité de Minkowski sous forme intégrale de la proposition \ref{PropInegMinkKUpRHg}\ref{ItemDHukLJiii}.
\begin{proposition}     \label{PROPooDMMCooPTuQuS}
    Si \( 1\leq p\leq \infty\) et si \( f\in L^p(\eR^d)\) et \( g\in L^1(\eR^d)\) alors
    \begin{enumerate}
        \item
            \( f*g\in L^p\)
        \item
            \( \| f*g \|_p\leq \| f \|_p\| g \|_1\).
    \end{enumerate}
\end{proposition}

\begin{proposition}[\cite{CXCQJIt}] \label{PropHNbdMQe}
    Si \( f\in L^1(\eR)\) et si \( g\) est dérivable avec \( g'\in L^{\infty}\), alors \( f*g\) est dérivable et \( (f*g)'=f*g'\).
\end{proposition}

\begin{proof}
    La fonction qu'il faut intégrer pour obtenir \( f*g\) est $f(t)g(x-t)$, dont la dérivée par rapport à \( x\) est \( f(t)g'(x-t)\). La norme de cette dernière est majorée (uniformément en \( x\)) par \( G(t)=| f(t) | \| g' \|_{\infty}\). La fonction \( f\) étant dans \( L^1(\eR)\), la fonction \( G\) est intégrable et le théorème de dérivation sous l'intégrale (théorème~\ref{ThoMWpRKYp}) nous dit que \( f*g\) est dérivable et
    \begin{equation}
        (f*g)'(x)=\frac{ d }{ dx }\int_{\eR}f(t)g(x-t)dt=\int_{\eR}f(t)g'(x-t)dt=(f*g')(x).
    \end{equation}
\end{proof}

\begin{corollary}       \label{CORooBSPNooFwYQrc}
    Si \( f\in L^1(\eR^d)\) et si \( g\) est de classe \(  C^{\infty}\), alors \( f*g\) est de classe \(  C^{\infty}\).
\end{corollary}

\begin{proof}
    Il s'agit d'itérer la proposition~\ref{PropHNbdMQe}.
\end{proof}

\begin{lemma}       \label{LemDQEKNNf}
    Soit \( f\in L^2(I)\) telle que
    \begin{equation}
        \int_If\varphi=0
    \end{equation}
    pour toute fonction \( \varphi\in C^{\infty}_c(I)\). Alors \( f=0\) presque partout sur \( I\).
\end{lemma}

\begin{proof}
    Nous considérons la forme linéaire
    \begin{equation}
        \begin{aligned}
            \phi\colon L^2(I)&\to \eC \\
            g&\mapsto \langle f, g\rangle=\int_If\bar g .
        \end{aligned}
    \end{equation}
    Par densité\footnote{Théorème~\ref{ThoILGYXhX}\ref{ItemYVFVrOIv}.} nous pouvons aussi considérer une suite \( (\varphi_n)\) dans \(  C^{\infty}_c(I)\) convergeant dans \( L^2\) vers \( f\). Alors nous avons pour tout \( n\) :
    \begin{equation}
        \langle f, \varphi_n\rangle =0.
    \end{equation}
    En passant à la limite, \( \langle f, f\rangle =0\), ce qui implique \( f=0\) dans \( L^2\) et donc \( f=0\) presque partout en tant que bonne fonction.
\end{proof}
Ce résultat est encore valable dans les espaces \( L^p\) (proposition~\ref{PropUKLZZZh}), mais il demande le théorème de représentation de Riesz\footnote{Théorème~\ref{ThoLPQPooPWBXuv}.}.

%---------------------------------------------------------------------------------------------------------------------------
\subsection{Approximation de l'unité}
%---------------------------------------------------------------------------------------------------------------------------


\begin{definition}[\cite{MonCerveau,TUEWwUN}]       \label{DEFooEFGNooOREmBb}
Nous considérons \( \Omega=\eR^d\) ou \( (S^1)^d\). Une \defe{approximation de l'unité}{approximation!de l'unité} sur \( \Omega\) autour de \( a\in \Omega\) est une suite \( (\varphi_n)\) de fonctions à valeurs réelles dans \( L^1(\Omega)\) telle que
    \begin{enumerate}
        \item
            $\sup_k \| \varphi_k \|_1 <\infty$,
        \item   \label{ITEMooGVRQooHDbrcf}
            pour chaque \( n\) nous avons $\int_{\Omega}\varphi_n=1$,
        \item
            si \( V\) est un voisinage de \( a\), alors
            \begin{equation}
                \lim_{k\to \infty} \int_{\Omega\setminus V}| \varphi_k |=0.
            \end{equation}
    \end{enumerate}
    En pratique, nous allons, sur \( \eR^d\) toujours considérer des approximations de l'unité autour de \( 0\), même si nous ne le préciserons pas. Vous noterez que dans le cas de \( S^1\), le choix du «point de base» est plus arbitraire.
\end{definition}
%TODO : voir si ça n'approxime pas un delta de Dirac d'une façon ou d'une autre.
Ce sont des fonctions dont la masse vient s'accumuler autour de zéro. En effet quel que soit le voisinage \( B(0,\alpha)\), si \( k\) est assez grand, il n'y a presque plus rien en dehors.

Pour le point \eqref{ITEMooTFFQooOUajFw}, si \( \Omega\) est \( S^1\), la mesure que nous considérons est \( \frac{ dx }{ 2\pi }\).


\begin{example}
    Une façon de construire une approximation de l'unité sur \( \eR\) est de considérer une fonction \( \varphi\in L^1(\Omega)\) telle que \( \int\varphi=1\) puis de poser
    \begin{equation}
        \varphi_k(x)=k^d\varphi(kx).
    \end{equation}
    Ici, \( \Omega\) peut être \( \eR\) ou \( S^1\).
\end{example}

Le lemme suivant permet de construire des approximations de l'unité intéressantes. Nous aurons une version pour \( S^1\) dans le lemme \ref{LEMooUNFBooRCzwIn}.
\begin{lemma}[\cite{TUEWwUN}]   \label{LemCNjIYhv}
    Soit \( \varphi\) est une fonction continue et positive à support compact sur \( \eR^d\) telle que \( \varphi(x)>\varphi(0)\) pour tout \( x\neq 0\). Si nous posons
    \begin{equation}
        \varphi_n(x)=\left( \int\varphi(y)^n \right)^{-1}\varphi(x)^n,
    \end{equation}
    alors la suite \( (\varphi_n)\) est une approximation de l'unité.
\end{lemma}

Voici un théorème qui donne les propriétés à propos du produit de convolution avec une approximation de l'unité dans \( \eR^d\). Une version pour \( S^1\) sera le théorème \ref{THOooIAOPooELSNxq}.
\begin{theorem}[\cite{TUEWwUN}] \label{ThoYQbqEez}
    Soit \( (\varphi_k)\) une approximation de l'unité sur \( \eR^d\).
    \begin{enumerate}
        \item
            Si \( g\) est mesurable et bornée sur \( \eR^d\) et si \( g\) est continue en \( x_0\) alors
            \begin{equation}
                (\varphi_k*g)(x_0)\to g(x_0).
            \end{equation}
        \item
            Si \( g\in L^p(\eR^d)\) (\( 1\leq p<\infty\)) alors
            \begin{equation}
                \varphi_k*g\stackrel{L^p}{\to}g.
            \end{equation}
        \item
            Si \( g\) est uniformément continue et bornée, alors
            \begin{equation}
                \varphi_k*g\stackrel{L^{\infty}}{\to}g
            \end{equation}
    \end{enumerate}
\end{theorem}

\begin{proof}
    En plusieurs points.
    \begin{enumerate}
        \item
            Nous notons \( d_k=(\varphi_k*g)(x_0)-g(x_0)\) et nous devons prouver que \( d_k\to 0\). Vu que \( \varphi_k\) est d'intégrale \( 1\) sur \( \eR^d\) nous pouvons écrire
            \begin{equation}
                d_k=\int_{\eR^d}\varphi_k(y)g(x_0-y)dy-\int_{\eR^d}g(x_0)\varphi_k(y)dy,
            \end{equation}
            et donc
            \begin{equation}
                |d_k|=\big| \int_{\eR^d}\big( g(x_0-y)-g(x_0) \big)\varphi_k(y)dy\big|\leq\int_{\eR^d}\big| g(x_0-y)-g(x_0) \big| |\varphi_k(y) |dy.
            \end{equation}
            Nous notons \( M=\sup_k\| \varphi_k \|_1\), et nous considérons \( \alpha>0\) tel que
            \begin{equation}
                \big| g(x_0-y)-g(x_0) \big|\leq \epsilon
            \end{equation}
            pour tout \( y\in B(0,\alpha)\). Nous nous restreignons maintenant aux \( k\) suffisamment grands pour que \( \int_{\complement B(0,\alpha)}| \varphi_k(y) |dy\leq \epsilon\). Alors en découpant l'intégrale en \( B(0,\alpha)\) et son complémentaire dans \( \eR^d\),
            \begin{equation}
                | d_k |\leq \epsilon M+\int_{\complement B(0,\alpha)} 2\| g \|_{\infty}| \varphi_k(y) |dy  \leq \epsilon M+2\| g \|_{\infty}\epsilon\leq \epsilon C.
            \end{equation}
            Donc oui, nous avons \( | d_k |\to 0\), et donc le premier point du théorème.

        \item

            Cette fois \( g\in L^p(\eR^d)\) et nous cherchons à montrer que \( \| d_k \|_p\to 0\). Encore qu'ici \( d_k\) soit défini à partir d'un représentant dans la classe de \( g\) et que d'ailleurs, nous allons travailler avec ce représentant.

            D'abord nous développons un peu ce \( d_k\) :
            \begin{subequations}
                \begin{align}
                \| d_k \|_p&=\left[ \int_{\eR^d}\left|     \int_{\eR^d}\big( g(x-y)-g(x) \big)\varphi_k(y)dy  \right|^pdx \right]^{1/p}\\
                &\leq\left[    \int_{\eR^d}\Big( \int_{\eR^d}| g(x-y)-g(x) |\cdot |\varphi_k(y) |dy \Big)^pdx \right]^{1/p}.
                \end{align}
            \end{subequations}
            À cette dernière expression nous appliquons l'inégalité de Minkowski (théorème~\ref{PropInegMinkKUpRHg}) sous la forme \eqref{EqZSiTZrH} pour la mesure \( d\nu(y)=| \varphi_k(y) |dy\) et \( f(x,y)=g(x-y)-g(x)\) :
            \begin{equation}
                \| d_k \|_p\leq \int_{\eR^d}\Big( \int_{\eR^d}\big| g(x-y)-g(x) \big|^pdx \Big)^{1/p}| \varphi_k(y) |dy=\int_{\eR^d}\| \tau_yg-g \|_p| \varphi_k(y) |dy.
            \end{equation}
            Par le lemme~\ref{LemCUlJzkA} nous pouvons trouver \( \alpha>0\) tel que \( \| \tau_yg-g \|_p\leq \epsilon\) pour tout \( y\in B(0,\alpha)\). Avec cela nous découpons encore le domaine d'intégration :
            \begin{equation}
                \| d_k \|_p\leq \int_{B(0,\alpha)}\underbrace{\| \tau_yg-g \|_p}_{\leq \epsilon}| \varphi_k(y) |dy+\int_{\complement B(0,\alpha)}  \underbrace{\| \tau_yg-g \|_p}_{\leq 2\| g \|_p}| \varphi_k(y) |dy\leq \epsilon M+2\epsilon\| g \|_p.
            \end{equation}

        \item

            Nous posons \( d_k(x)=(\varphi_k*g)(x)-g(x)\) et nous voulons prouver que \( \| d_k \|_{\infty}\to 0\), c'est-à-dire que \( d_k(x)\) converge vers zéro uniformément en \( x\). Nous posons aussi
            \begin{equation}
                \tau_y(g)\colon x\mapsto g(x-y).
            \end{equation}
            En récrivant le produit de convolution, une petite majoration donne
            \begin{equation}
                | d_k(x) |\leq \int_{\eR^d}\| \tau_y(g)-g \|_{\infty}| \varphi_k(y) |dy.
            \end{equation}
            L'uniforme continuité de \( g\) signifie que pour tout \( \epsilon\), il existe un \( \alpha\) tel que pour tout \( y\in B(0,\alpha)\),
            \begin{equation}
                \| \tau_y(g)-g \|_{\infty}\leq \epsilon.
            \end{equation}
            Encore une fois nous découpons le domaine d'intégration en \( B=B(0,\alpha)\) et son complémentaire :
            \begin{subequations}
                \begin{align}
                    \| d_k \|_{\infty}&\leq\int_B\underbrace{\| \tau_y(g)-g \|_{\infty}}_{\leq \epsilon}| \varphi_k(y) |dy+\int_{\complement B}\underbrace{\| \tau_y(g)-g \|_{\infty}}_{\leq 2\| g \|_{\infty}}| \varphi_k(y) |\\
                    &\leq \epsilon M+2\| g \|_{\infty}\epsilon
                \end{align}
            \end{subequations}
            où la seconde ligne est justifiée par le choix d'un \( k\) assez grand pour que \( \int_{\complement B}| \varphi_k(y) |dy\leq \epsilon\).

            Nous avons donc bien \( \| d_k \|_{\infty}\to 0\).
    \end{enumerate}
\end{proof}

\begin{example}
    Une petite remarque en passant : aussi triste que cela en ait l'air, la convergence uniforme n'implique pas la convergence \( L^p(\Omega)\) si \( \Omega\) n'est pas borné. En effet si \( f\in L^p\), la suite donnée par
    \begin{equation}
        f_n(x)=f(x)+\frac{1}{ n }
    \end{equation}
    converge uniformément vers \( f\), mais
    \begin{equation}
        \| f_n-f \|_p=\int_{\Omega}\frac{1}{ n }
    \end{equation}
    n'existe même pas si le domaine \( \Omega\) n'est pas borné.
\end{example}

%---------------------------------------------------------------------------------------------------------------------------
\subsection{Densité des polynômes trigonométriques}
%---------------------------------------------------------------------------------------------------------------------------

\begin{definition}      \label{DEFooGCZAooFecAHB}
    Le \defe{système trigonométrique}{système!trigonométrique} donné par \( \{ e_n \}_{n\in \eZ}\) est
    \begin{equation}
        e_n(t)= \frac{1}{ \sqrt{ 2\pi } } e^{int}.
    \end{equation}
\end{definition}

Une bonne partie de la douleur qu'évoque mot « densité » consiste à montrer que ce système est total dans \( L^2(S^1)=L^2(\mathopen[ 0 , 2\pi \mathclose])\), et donc en est une base hilbertienne.

\begin{definition}
    Un \defe{polynôme trigonométrique}{polynôme!trigonométrique} est une fonction de la forme
    \begin{equation}
        P(t)=\sum_{n=-N}^Nc_n e_n(t).
    \end{equation}
\end{definition}

\begin{definition}[Coefficients de Fourier]
    Pour toute fonction pour laquelle ça a un sens (que ce soit des fonctions \( L^2\) ou non), nous posons
    \begin{equation}\label{EqhIPoPH}
        c_n(f)=\langle f, e_n\rangle .
    \end{equation}
    Ces nombres sont les \defe{coefficients de Fourier}{coefficients!de Fourier} de \( f\). 
\end{definition}

Ces trois définitions n'ont a priori aucun rapport entre elles, et rien en particulier ne devrait vous faire penser à une égalité du type
\begin{equation}
    f(x)=\sum_{n=-\infty}^{\infty}c_n(f)e_n(x).
\end{equation}
Nous avons toutefois quelque liens.

\begin{lemma}   \label{LemZVfZlms}
    Deux petits résultats simples mais utiles à propos des polynômes trigonométriques.
    \begin{enumerate}
        \item
    Si \( f\in L^1(S^1)\), alors nous avons la formule
    \begin{equation}
        f*e_n=c_n(f)e_n.
    \end{equation}
\item

    Si \( P\) est un polynôme trigonométrique et si \( f\in L^1(S^1)\) alors \( f*P\) est encore un polynôme trigonométrique.
    \end{enumerate}
\end{lemma}

\begin{proof}
    Le premier point est un simple calcul :
    \begin{subequations}
        \begin{align}
            (f*e_n)(x)=\int_0^{2\pi}f(x-t)e_n(t)
        \end{align}
    \end{subequations}

    En ce qui concerne le second point, nous notons \( P=\sum_{k=-N}^NP_ke_k\), et par linéarité de la convolution,
    \begin{equation}
        f*P=\sum_{k=-N}^NP_kf*e_k=\sum_{k=-N}^nP_kc_k(f)e_k,
    \end{equation}
    qui est encore un polynôme trigonométrique.
\end{proof}

\begin{example} \label{ExDMnVSWF}
    Sur \( S^1\) nous construisons alors l'approximation de l'unité basée sur la fonction \( 1+\cos(x)\) et le lemme~\ref{LemCNjIYhv}. Cette fonction est évidemment un polynôme trigonométrique parce que
    \begin{equation}
        \cos(x)=\frac{  e^{ix}+ e^{-ix} }{2}.
    \end{equation}
    Ensuite les puissances le sont aussi à cause de la formule du binôme :
    \begin{equation}
        \big( 1+\cos(x) \big)^n=\sum_{k=0}^n\binom{ n }{ k }\cos^n(x),
    \end{equation}
    dans laquelle nous pouvons remettre \( \cos(x)\) comme un polynôme trigonométrique et développer à nouveau la puissance avec (encore) la formule du binôme. La chose importante est qu'il existe une approximation de l'unité \( (\varphi_n)\) formée de polynômes trigonométrique.

    Ce qui fait la spécificité des polynômes trigonométriques est qu'ils sont à la fois stables par convolution (lemme~\ref{LemZVfZlms}) et qu'ils permettent de créer une approximation de l'unité sur \( \mathopen[ 0 , 2\pi \mathclose]\). Ce sont ces deux choses qui permettent de prouver l'important théorème suivant.
\end{example}

\begin{theorem} \label{ThoQGPSSJq}
    Les polynôme trigonométriques sont dense dans \( L^p(S^1)\) pour \( 1\leq p<\infty\).
\end{theorem}

\begin{proof}
    
    \begin{equation}
        \varphi_k*f\stackrel{L^p}{\to}f
    \end{equation}
    par le théorème~\ref{ThoYQbqEez}. Nous avons donc convergence \( L^p\) d'une suite de polynômes trigonométrique, ce qui prouve que l'espace de polynômes trigonométriques est dense dans \( L^p(S^1)\).
\end{proof}

\begin{remark}
    Deux remarques.
    \begin{itemize}
        \item
            Il n'est pas possible que les polynômes trigonométriques soient dense dans \( L^{\infty}\) parce qu'une limite uniforme de fonctions continues est continue (c'est le théorème~\ref{ThoUnigCvCont}). Donc les polynômes trigonométriques ne peuvent engendrer que des fonctions continues.
        \item
            Nous donnerons au théorème~\ref{ThoDPTwimI} une démonstration indépendante de la densité des polynômes trigonométriques dans \( L^p(S^1)\).
    \end{itemize}
\end{remark}

%+++++++++++++++++++++++++++++++++++++++++++++++++++++++++++++++++++++++++++++++++++++++++++++++++++++++++++++++++++++++++++
\section{Espaces \texorpdfstring{$L^2$}{$L^2$}, généralités}
%+++++++++++++++++++++++++++++++++++++++++++++++++++++++++++++++++++++++++++++++++++++++++++++++++++++++++++++++++++++++++++
\label{SECooEVZSooLtLhUm}

L'espace \( L^2\) est l'espace \( L^p\) définit en \ref{DEFooKMJQooXeaUtp} avec \( p=2\). Cependant il possède une propriété extraordinaire par rapport aux autres \( L^p\), c'est que la norme \( | . |_2\) dérive d'un produit scalaire. Il sera donc un espace de Hilbert.

\begin{normaltext}  \label{NORMooUEIEooYtlFse}
    Nous en rappelons la construction. Soit \( (\Omega,\tribA,\mu)\) un espace mesuré. Nous considérons l'opération
    \begin{equation}    \label{DefProdScalLubrgTj}
        \langle f, g\rangle =\int_{\Omega}f(\omega)\overline{ g(\omega)}d\mu(\omega)
    \end{equation}
    et la norme associée
    \begin{equation}
        \| f \|_2=\sqrt{\langle f, f\rangle }.
    \end{equation}
    Nous considérons l'ensemble
    \begin{equation}
        \mL^2(\Omega,\mu)=\{ f\colon \Omega\to \eC\tq \| f \|_2<\infty \}
    \end{equation}
    et la relation d'équivalence \( f\sim g\) si et seulement si \( f(x)=g(x)\) pour \( \mu\)-presque tout \( x\).

    Et enfin, nous considérons le quotient
    \begin{equation}
        L^2(\Omega,\mu)=\mL^2(\Omega,\mu)/\sim.
    \end{equation}
\end{normaltext}


\begin{lemma}   \label{LemIVWooZyWodb}
    Soit un espace mesuré\quext{Est-ce qu'il ne faudrait pas un peu plus d'hypothèses, comme \( \sigma\)-fini par exemple ? Vérifiez et écrivez-moi quand vous avez la réponse.} \( (\Omega,\tribA,\mu)\).
    \begin{enumerate}
        \item
            Pour tout \( f,g\in L^2(\Omega,\tribA,\mu)\), le produit 
            \begin{equation}
                \langle f, g\rangle =\int_{\Omega}f\bar g\,d\mu 
            \end{equation}
            est bien défini et est un nombre complexe\footnote{Par opposition au fait que ce serait l'infini.}.
        \item
            L'opération \( (f,g)\mapsto \langle f, g\rangle \) est un produit hermitien\footnote{Définition \ref{DefMZQxmQ}. Pour rappel, nous considérons des fonctions à valeurs complexes. Si au contraire nous avions considéré seulement des fonctions à valeurs réelles, nous aurions eu un produit scalaire.}.
        \item
            Le couple \( \big( L^2(\Omega,\tribA,\mu),\langle ., .\rangle  \big)\) est un espace de Hilbert\footnote{Définition \ref{DefORuBdBN}.}.
    \end{enumerate}
\end{lemma}

\begin{proof}
    Que \( L^2(\Omega)\) soit un espace vectoriel est un cas particulier de la proposition \ref{PROPooTYCYooAKJWOX}. Voyons cette histoire de produit scalaire.

    D'abord si \( f\) et \( g\) sont dans \( L^2\), alors l'inégalité de Hölder (proposition~\ref{ProptYqspT}) nous indique que le produit \( fg\) est un élément de \( L^1\). Par conséquent la formule a un sens.

    Ensuite nous montrons que la formule passe au quotient. Pour cela, nous considérons des fonctions \( \alpha\) et \( \beta\) nulles presque partout et nous regardons le produit de \( f_1=f+\alpha\) par \( g_1=g+\beta\) :
    \begin{equation}
        \langle f_1, g_1\rangle =\int fg+\beta f+\alpha g+ \alpha\beta.
    \end{equation}
    Les fonctions \( \beta f\), \( \alpha g\) et \( \alpha\beta\) étant nulles presque partout, leur intégrale est nulle et nous avons bien \( \langle f_1, g_1\rangle =\langle f,g \rangle \). Nous pouvons donc considérer le produit sur l'ensemble des classes.

    Pour vérifier que la formule est un produit scalaire, le seul point non évidement est de prouver que \( \langle f, f\rangle =0\) implique \( f=0\). Cela découle du fait que
    \begin{equation}
        \langle f, f\rangle =\int_{\Omega}| f |^2.
    \end{equation}
    La fonction \( x\mapsto | f(x) |^2\) vérifie les hypothèses du lemme~\ref{Lemfobnwt}. Par conséquent \( | f(x) |^2\) est presque partout nulle.

    En ce qui concerne le fait que \( L^2(\Omega)\) soit un espace de Hilbert, il s'agit simplement de se remémorer que c'est un espace complet (théorème ~\ref{ThoUYBDWQX}) et dont la norme dérive d'un produit scalaire. Nous sommes donc bien dans la définition~\ref{DefORuBdBN}.
\end{proof}

\begin{normaltext}
    Ces espaces seront utilisés pour de nombreuses applications. Nous en aurons besoin pour plusieurs combinaisons d'ensembles \( \Omega\) et de mesures \( \mu\).
    \begin{itemize}
        \item Pour \( \eR^d\)
        \item Pour \( S^1\)
        \item Pour \( \mathopen[ a , b \mathclose]\)
        \item Pour \( \mathopen[ 0 , 2\pi \mathclose[\)
            \item Pour \( \mathopen[ -T , T \mathclose[\)
    \end{itemize}
    Le premier est non compact et il est raisonnable de penser qu'il sera foncièrement différents des autres. À isomorphismes assez triviaux près, les espaces des fonctions sur les trois autres sont identiques. Nous nous attendons donc à ce qu'ils aient les mêmes propriétés. Notons que du point de vue de \( L^2\), étant donné qu'il y a un quotient par les parties de mesures nulles, prendre \( \mathopen] 0 , 2\pi \mathclose[\) ou \( \mathopen[ 0 , 2\pi \mathclose]\) ou n'importe quelle autre possibilité de ce genre revient au même.

    Afin de pouvoir utiliser ces espaces de façon optimale, et entre autres y définir les séries de Fourier, nous avons besoin, pour chacun d'entre eux de définir les éléments suivants :
    \begin{itemize}
        \item mesure
        \item produit de convolution
        \item le système trigonométrique (que nous allons montrer être une base hilbertienne)
        \item coefficients de Fourier
    \end{itemize}
    Ça fait pas mal de choses à définir. Il n'est pas besoin de définir un produit scalaire parce que le lemme \ref{LemIVWooZyWodb} nous en donne un générique.

    Les définitions qui viennent sont à prendre «tant que les formules ont un sens». Nous parlons donc de fonctions dans \( \Fun(\Omega,\eC)\), l'ensemble de toutes les fonctions sur \( \Omega\) à valeurs dans \( \eC\). Nous verrons plus tard les espaces de fonctions sur lesquels tout a un sens.
\end{normaltext}

%+++++++++++++++++++++++++++++++++++++++++++++++++++++++++++++++++++++++++++++++++++++++++++++++++++++++++++++++++++++++++++ 
\section{L'espace \( L^2(\eR^d)\)}
%+++++++++++++++++++++++++++++++++++++++++++++++++++++++++++++++++++++++++++++++++++++++++++++++++++++++++++++++++++++++++++

La mesure est celle de Lebesgue. Le produit de convolution est donné, pour \( f,g\in\Fun(\eR^d,\eC)\), par
\begin{equation}
    (f*g)(x)=\int_{\eR^d}f(y)g(x-y)dy
\end{equation}
Certaines de ses propriétés ont déjà été vues dans le théorème \ref{THOooMLNMooQfksn}.

En ce qui concerne le système trigonométrique, pour tout \( \xi\in \eR^d\) nous définirions bien
\begin{equation}
    e_{\xi}(x)= e^{i\xi\cdot x},
\end{equation}
genre pour faire que les transformations de Fourier sont des séries continues \ldots mais bon. Nous n'allons pas tenter le diable plus que ça, et nous ne définissons 
\begin{itemize}
    \item pas de système trigonométrique,
    \item pas de coefficients de Fourier non plus,
    \item pas de théorie des séries de Fourier sur \( \eR^d\).
\end{itemize}
Quand je disais que la non-compacité de \( \eR^d\) allait un peu changer les choses par rapport aux autres, je ne rigolais pas.

%+++++++++++++++++++++++++++++++++++++++++++++++++++++++++++++++++++++++++++++++++++++++++++++++++++++++++++++++++++++++++++ 
\section{L'espace \( L^2(S^1)\)}
%+++++++++++++++++++++++++++++++++++++++++++++++++++++++++++++++++++++++++++++++++++++++++++++++++++++++++++++++++++++++++++

L'espace \( S^1\) sera fait avec forces détails, parce qu'il va servir de base pour les espaces \( L^2(\mathopen[ 0 , 2\pi \mathclose[)\), \( L^2(\mathopen[ -T , T \mathclose[)\) ainsi que pour l'étude des fonctions périodiques sur \( \eR\).

En tant qu'ensemble,
\begin{equation}
    S^1=\{  e^{it} \}_{t\in \eR},
\end{equation}
sans garanties que cette paramétrisation soit une bijection.

Il y a essentiellement deux façons de définir une intégrale sur \( S^1\).
\begin{enumerate}
    \item Voir \( S^1\) comme une sous-variété de \( \eR^2\) et utiliser la définition \ref{PROPooOAHWooAfxvyv}. Cette façon a cependant deux inconvénients :
        \begin{itemize}
            \item Elle ne donne pas la tribu des mesurables sur \( S^1\), c'est-à-dire que cette méthode ne donne pas de façon évidente une théorie de la mesure sur \( S^1\).
            \item Il faut au moins deux cartes pour paramétrer le cercle. La fainéantise nous prévient que ça va être technique.
        \end{itemize}
    \item
        Rapporter la structure d'espace mesuré de \( \mathopen[ 0 , 2\pi \mathclose[\) vers \( S^1\), de force via le premier difféomorphisme qui nous passe par la tête, à savoir \( t\mapsto  e^{it}\).
\end{enumerate}
Nous allons choisir la seconde possibilité, en gardant en tête qu'elle fonctionne de façon très simple un peu par coup de chance, voir la remarque \ref{REMooOMYYooNFiKOs}\ref{ITEMooJTKCooYQknqo}.

%--------------------------------------------------------------------------------------------------------------------------- 
\subsection{Espace mesuré}
%---------------------------------------------------------------------------------------------------------------------------

\begin{normaltext}
    Nous considérons l'ensemble \( \mathopen[ 0 , 2\pi \mathclose[\) sur lequel nous mettons la tribu de Lebesgue et la mesure de Lebesgue, donc l'espace mesuré \( \big( \mathopen[ 0 , 2\pi \mathclose[, \Lebesgue\big)\). 

        Ici, les mesurables sont les intersections entre \( \mathopen[ 0 , 2\pi \mathclose[\) et les mesurable de Lebesgue de \( \eR\). Il s'agit de la tribu induite, définition \ref{DefDHTTooWNoKDP}.
        
        La proposition \ref{PROPooILOEooBiumKD} aidée par la bijection
        \begin{equation}
            \begin{aligned}
                \varphi\colon \mathopen[ 0 , 2\pi \mathclose[&\to S^1 \\
                    t&\mapsto  e^{it} 
            \end{aligned}
        \end{equation}
        nous permet de donner une structure d'espace mesuré sur \( S^1\) en posant
        \begin{equation}
            \tribA=\varphi(\Lebesgue)
        \end{equation}
        et
        \begin{equation}        \label{EQooOEVDooSYWrgT}
            \mu(B)=\frac{1}{ 2\pi }\lambda\big( \varphi^{-1}(B) \big)
        \end{equation}
        pour tout \( B\in \tribA\). Nous avons également fait un tout petit peu appel à la proposition \ref{PropooVXPMooGSkyBo} pour multiplier la mesure par \( \frac{1}{ 2\pi }\). Le point fort de la proposition \ref{PROPooILOEooBiumKD} est qu'elle nous donne une formule pour l'intégrale\footnote{Qui a un espace mesuré a une intégrale.} qui donne dans notre cas :
        \begin{equation}        \label{EQooCFPAooMcrDEQ}
            \int_{S^1}fd\mu=\frac{1}{ 2\pi }\int_{\mathopen[ 0 , 2\pi \mathclose[}(f\circ\varphi)d\lambda.            
        \end{equation}
    Vu que \( \mu(S^1)=1\), l'espace mesuré ainsi construit est fini\footnote{Définition \ref{DefBTsgznn}.} et a fortiori \( \sigma\)-fini.
\end{normaltext}

%--------------------------------------------------------------------------------------------------------------------------- 
\subsection{Autres tribus}
%---------------------------------------------------------------------------------------------------------------------------

Il y a d'autres façons de mettre une tribu sur \( S^1\).
\begin{itemize}
    \item En posant
        \begin{equation}
            \begin{aligned}
                \varphi\colon \eR&\to S^1 \\
                x&\mapsto  e^{ix}, 
            \end{aligned}
        \end{equation}
        nous pouvons considérer
        \begin{equation}
            \tribA_1=\{ \varphi(A)\tq A\in\Lebesgue(\eR) \}.
        \end{equation}
    \item
        Nous avons la tribu de Lebesgue sur \( \eC\). Nous pouvons induire sur \( S^1\) :
        \begin{equation}
            \tribA_2=\{ A\cap S^1\tq A\in \Lebesgue(\eC) \}.
        \end{equation}
\end{itemize}

\begin{probleme}
    Je suis quasiment certain que \( \tribA=\tribA_1=\tribA_2\). Et c'est en fait important pour pouvoir identifier les intégrales et les notions d'application mesurables.
\end{probleme}

%--------------------------------------------------------------------------------------------------------------------------- 
\subsection{Topologie}
%---------------------------------------------------------------------------------------------------------------------------

Nous considérons sur \( S^1\) la topologie induite de \( \eC\). Vu que \( S^1\) est fermé et borné dans \( \eC\), il en est une partie compacte. Par le lemme \ref{LEMooVYTRooKTIYdn}, l'espace \( S^1\) muni de sa topologie est un espace topologique compact.

Nous pouvons donc sans crainte affirmer que toute fonction continue \( f\colon S^1\to \eK\) est bornée et atteint ses bornes.

\begin{propositionDef}      \label{PROPooEQDBooDfOrTZ}
    Soit la formule
    \begin{equation}
        d( e^{ix},  e^{iy})=\inf_{k\in \eZ}| x-y+2k\pi |.
    \end{equation}
    \begin{enumerate}
        \item
            Elle est bien définie (ne dépend pas des choix de \( x\) et \( y\) donnant les mêmes points dans \( S^1\))
        \item
            L'infimum est en réalité un minimum : il est atteint par un certain \( k\in \eZ\) (qui, lui, dépend des choix).
        \item
            La formule définit une distance\footnote{Définition \ref{DefMVNVFsX}.} sur \( S^1\).
    \end{enumerate}
    Nous considérons sur \( S^1\) la topologie \( \tau_d\) découlant de cette distance.
\end{propositionDef}


\begin{proof}
    Point par point.
    \begin{enumerate}
        \item
            Soient \( x',y'\in \eR\) tels que \(  e^{ix'}= e^{ix}\) et \(  e^{iy'}= e^{iy}\). Alors \( x'=x+2l\pi\) et \( y'=y+2l'\pi\) pour certains entiers \( l,l'\in \eZ\) (corollaire \ref{CORooTFMAooHDRrqi}). Nous avons alors \( | x'-y'+2k\pi |=| x-y+2\pi(k+l-l') |\) et
            \begin{equation}
                \inf_{k\in \eZ}| x'-y'+2k\pi |=\inf_{k\in \eZ}| x-y+2k\pi |.
            \end{equation}
        \item
            Quels que soient \( x\) et \( y\) fixés, nous avons
            \begin{equation}
                \lim_{k\to \pm\infty} | x-y+2k\pi |=\infty.
            \end{equation}
            Donc l'infimum est forcément atteint par un \( k\in \eZ\).
        \item
            Pour la distance, il y a plusieurs points à prouver. 
            \begin{itemize}
                \item 
                    Pour tout \( z,z'\in S^1\) nous avons \( d(z,z')\geq 0\) parce que la distance est donnée par une valeur absolue.
                \item
                    Si \( d(z,z')=0\), alors il existe \( k\) tel que \( x=y+2k\pi\). Alors \(  e^{ix}= e^{i(y+2k\pi)}= e^{iy} e^{2ki\pi}= e^{iy}\). C'est-à-dire \( z=z'\).
                \item
                    Pour la symétrie, nous avons
                    \begin{equation}
                        | x-y+2k\pi |=| y-x-2k\pi |=| y-x+2k'\pi |
                    \end{equation}
                    en posant \( k'=-k\). L'infimum étant pris sur \( k\in \eZ\), nous avons al symétrique \( d( e^{ix},  e^{iy})=d( e^{iy}, e^{ix})\).
                \item
                    Pour attaquer l'inégalité triangulaire, nous considérons \( z_1= e^{ix_1}\), \( z_2= e^{ix_2}\) et \( z_3= e^{ix_3}\). Nous posons également \( k_1,k_2, k_3\in \eZ\) tels que \( d(z_1,z_3)=| x_1-x_3+2k_1\pi | \), \( d(z_1,z_2)=| x_1-x_2+2k_2\pi |\) et \( d(z_2,z_3)=| x_2-x_3+2k_3\pi |\). Nous avons alors
                    \begin{subequations}
                        \begin{align}
                            d(z_1,z_3)&=| x_1-x_3+2k_1\pi |=| x_1-x_2+x_2-x_3+2k_1\pi |\\
                            &=\inf_{k\in \eZ}| (x_1-x_2+2k_2\pi)+(x_2-x_3+2k_3\pi)+2k\pi |\\
                            &=\inf_{k\in \eZ}\Big( | x_1-x_2+2k_1\pi |+| x_2-x_3+2k_3\pi |+2k\pi \Big)\\
                            &=d(z_1,z_3)+d(z_2,z_3)
                        \end{align}
                    \end{subequations}
                    parce que le dernier infimum est réalisé par \( k=0\).
            \end{itemize}
        \end{enumerate}
\end{proof}

Le cercle est bien connu pour être symétrique et en particulier avoir une symétrie sous les rotations. Nous allons voir quelque résultats qui vont dans le sens de dire que la distance définie sur \( S^1\) respecte cette symétrie.

\begin{lemma}       \label{LEMooCQCAooAEctbe}
    Plusieurs points à propos de l'invariance de la topologie sous les rotations.
    \begin{enumerate}
        \item
            La distance est invariante sous les rotations, c'est-à-dire que si \( a,b\in S^1\) et si \( s\in \eR\), alors
            \begin{equation}
                d( e^{is}a, e^{is}b)=d(a,b).
            \end{equation}
        \item       \label{ITEMooCIPYooTyPQLj}
            Les boules sont préservées sous les rotations\footnote{Pas chaque boule séparément, mais l'ensemble des boules}, c'est-à-dire que
            \begin{equation}
                e^{is}B_d(a,r)=B_d( e^{is}a,r).
            \end{equation}
        \item
            La topologie est invariante sous les rotations : \(  e^{is}\tau_d=\tau_d\).
    \end{enumerate}
\end{lemma}

\begin{proof}
    Point par point.
    \begin{enumerate}
        \item
            Si \( a= e^{ix}\) et \( b= e^{iy}\), nous avons
            \begin{equation}
                d(a,b)=\inf_{k\in \eZ}| x-y+2k\pi |=\inf_{k\in \eZ}| (x-s)-(y-s)+2k\pi |=d( e^{is}a, e^{is}b).
            \end{equation}
            Et de un.
        \item
            Il faut une inclusion dans chaque sens.
            \begin{subproof}
                \item[\(  e^{is}B_d(a,r)\subset B_d( e^{is}a,r)\)]
                    Soit \( b\in  e^{is}B_d(a,r)\). Alors \( b= e^{is}b'\) pour un certain \( b'\in B_d(a,r)\). Nous avons alors, en utilisant le premier point,
                    \begin{equation}
                        d(b, e^{is})=d( e^{-is}b,a)=d(b',a)<r.
                    \end{equation}
                    Donc \( b\in B_d( e^{is}a,r)\).
                \item[\(  B_d( e^{is}a,r)\subset e^{is} B_d( a,r)\)]
                    Soit \( b\in B_d( e^{is}a,r)\). Nous devons prouver que \( b\in  e^{is}B_d(a,r)\), c'est-à-dire que \( b= e^{is}b'\) pour un certain \( b'\in B_d(a,r)\) ou encore que \(  e^{-is}b\in B_d(a,r)\). En utilisant encore le premier point,
                    \begin{equation}
                        d( e^{-is}b,a)=d(b, e^{is}a)<r.
                    \end{equation}
                    Donc oui, \(  e^{-is}b\in B(a,r)\).
            \end{subproof}
        \item
            Soit \( A\in\tau_d\). Si \( a\in  e^{is}A\), alors \( a= e^{is}a'\) pour un certain \( a'\in A\). Notre but est de prouver que \(  e^{is}A\) contient un voisinage de \( a\).

            Vu que \( a'\in A\), il existe \( r>0\) tel que \( B_d(a',r)\subset A\). Nous avons alors
            \begin{equation}
                e^{is}aB_d(a',r)\subset  e^{is}A,
            \end{equation}
            et comme \(  e^{is}B_d(a',r)=B_d( e^{is}a',r)=B_d(a,r)\) nous avons bien
            \begin{equation}
                B_d(a,r)\subset  e^{is}A.
            \end{equation}
    \end{enumerate}
\end{proof}

Nous allons voir maintenant quelque résultats à propos de \(B_d(1,r)\) qui a la bonne figure d'être un ouvert qui s'étale symétriquement en partant de \( 1\) (le point le plus à droite du cercle). Par rapport à la figure \ref{LabelFigJOQVoolPTsYPZK}, il s'agit ni plus ni moins que de voir qu'une boule de rayon \( r\) autour de \( 1\) est bien la partie indiquée (symétrique par rapport à \( 1\) et de longueur d'arc \( r\) des deux côtés). De plus, ce voisinage n'est autre que la partie du cercle située à droite de la ligne en pointillés.
\newcommand{\CaptionFigJOQVoolPTsYPZK}{Un voisinage de \( 1\) dans \( S^1\).}
\input{auto/pictures_tex/Fig_JOQVoolPTsYPZK.pstricks}

Ce lemme-ci montre que \( B_d(1,r)\) est une partie de \( S^1\) qui s'étale symétriquement autour de \( 1\).
\begin{lemma}       \label{LEMooMYNVooIWWsiV}
    Soit l'application
    \begin{equation}
        \begin{aligned}
            \varphi\colon \eR&\to S^1 \\
            x&\mapsto  e^{ix}.
        \end{aligned}
    \end{equation}
    Nous avons \( B_d(1,r)=\varphi\big( \mathopen] -r , r \mathclose[ \big)\).
\end{lemma}

\begin{proof}
    Soit \( b\in B_d(1,r)\) de la forme \( b= e^{iy}\) avec \( y\) choisi de telle sorte que \( d(1,b)=| y |\). Vu que \( d(1,b)<r\), nous avons \( | y |<r\) et donc \( b\in \varphi\big( \mathopen] -r , r \mathclose[ \big)\).

    Dans l'autre sens, si \( y\in\mathopen] -r , r \mathclose[\), alors
        \begin{equation}        \label{EQooRWASooVZnQCJ}
            d(1, e^{iy})=\inf_{k\in \eZ}| y+2k\pi |\leq | y |<r.
        \end{equation}
        Nous avons utilisé le fait que l'infimum sur \( k\in \eZ\) est plus petit ou égal à la valeur pour \( k=0\). Les inégalités \eqref{EQooRWASooVZnQCJ} montrent que \(  e^{iy}\in B_d(1,r)\).
\end{proof}

Le lemme suivant montre que que les boules autour de \( 1\) sont délimitées par la droite en pointillé de la figure \ref{LabelFigJOQVoolPTsYPZK}.
\begin{lemma}       \label{LEMooLINCooHJmJWx}
    Soit \( r\in \mathopen[ 0 , \pi \mathclose]\). Nous avons
    \begin{equation}
        B_d(1,r)=S^1\cap\{x+iy\tq x>\cos(r)\}.
    \end{equation}
\end{lemma}

\begin{proof}
    Si \( r=\pi\), alors \( B(1,r)=S^1\setminus\{ -1 \}\), alors que \( \cos(\pi)=-1\).

Si \( r<\pi\), alors nous partons de la formule \eqref{EQooRVPJooTMwNTU} qui dit que \(  e^{ir}=\cos(e)+i\sin(r)\). D'après le lemme \ref{LEMooMYNVooIWWsiV}, un élément de \( B_d(1,r)\) est de la forme \(  e^{iy}\) avec \( y\in \mathopen] -r , r \mathclose[\). Nous voudrions donc prouver que \( \cos(y)>\cos(r)\) dès que \( y\in\mathopen] -r , r \mathclose[\) et \( r<\pi\).

    %TODOooBZPCooAFVbcT
Sur \( \mathopen] -r , 0 \mathclose[\), la fonction \( \cos\) est croissante, donc si \( y<0\) alors 
    \begin{equation}
        \cos(y)>\cos(-r)=\cos(r).
    \end{equation}
De la même façon, sur \( \mathopen] 0,r \mathclose[\), la fonction \( \cos\) est décroissante, de telle sorte que si \( y>0\), alors \( \cos(y)>\cos(r)\).

    Nous avons prouvé que \( B_d(1,r)\subset S^1\cap  \{ x+iy\tq x>\cos(r) \}\). 

    Lançons nous dans la preuve de l'inclusion inverse.

    Soit \( x>\cos(r)\). Si \( x+iy\in S^1\), nous avons \( x+iy= e^{is}=\cos(s)+i\sin(s)\) pour un certain \( s\in\mathopen[ -\pi , \pi \mathclose[\). Notons que \( s=-\pi\) correspondrait au point \( -1\in S^1\), qui est exclu de notre étude parce que nous supposons \( r<\pi\). Donc \( s\in\mathopen] -\pi , \pi \mathclose[\).

    Nous avons donc \( \cos(r)<x=\cos(s)\). Et voila.
\end{proof}

\begin{proposition}
    La topologie \( \tau_d\) sur \( S^1\)\footnote{Définition \ref{PROPooEQDBooDfOrTZ}.} est la topologique induite depuis \( \eC\).
\end{proposition}

\begin{proof}
    Nous notons \( \tau_i\) la topologie induite (c'est-à-dire l'ensemble des ouverts) et \( \tau_d\) la topologie de la distance fraîchement définie. Nous allons également noter \( B_{\eC}(a,r)\) la boule dans \( \eC\) de centre \( a\) et de rayon \( r\), et \( B_d(z,r)\) celle dans \( S^1\), de centre \( z\in S^1\) et de rayon \( r\) pour notre distance \( d\).
    \begin{subproof}
        
    \item[\( \tau_i\subset\tau_d\)]
        Un élément général de \( \tau_i\) est de la forme \( \mO\cap S^1\) où \( \mO\) est un ouvert de \( \eC\). Soit \( a\in \mO\cap S^1\) et prouvons qu'il existe un ouvert de \( \tau_d\) contenant \( a\) et contenu dans \( \mO\cap S^1\); cela prouvera que \( \mO\cap S^1\) est ouvert de \( \tau_d\) par le théorème \ref{ThoPartieOUvpartouv}.

        Soient \( a= e^{ix}\) et \( r\) tel que \( B_{\eC}(a,r)\subset \mO\). Nous allons montrer que \( B_{d}(a,r)\subset B_{\eC}(a,r)\). Un élément général de \( B_d(a,r)\) est \( b= e^{iy}\) tel que
        \begin{equation}
            d(a,b)=\inf_{k\in \eZ}| x-y+2k\pi |\leq r.
        \end{equation}
        Quitte à redéfinir \( x\) ou \( y\) nous pouvons supposer que l'infimum est atteint en \( k=0\). En utilisant la proposition \ref{PROPooYMMKooSUBtoo} nous majorons :
        \begin{equation}
            | a-b |=|  e^{ix}- e^{iy} |\leq | x-y |=d(a,b)\leq r.
        \end{equation}
        Donc nous avons bien \( b\in B_{\eC}(a,r)\) dès que \( b\in B_d(a,r)\).

    \item[\( \tau_d\subset \tau_i\)]

        Ce sens est plus délicat parce que, si nous voulons suivre les mêmes pas que le premier sens, nous devrons nous appuyer sur la continuité de l'application \( \ln\colon \eC^*\to \eC\), laquelle n'est pas vraie en \( -1\) (voir par exemple le lemme \ref{LEMooMUOIooCnoWwq}).

        Soient \( A\in \tau_d\) et \( a\in A\). Nous devons prouver l'existence d'un ouvert \( \mO\) de \( \eC\) tel que \( S^1\cap\mO\) soit inclus à \( A\) et contienne \( a\). Nous allons prouver cela dans le cas \( a=1\) et ensuite propager le résultat en utilisant la symétrie de \( S^1\).

        \begin{subproof}
            \item[Si \( a=1\)]
                Vu que \( A\) est ouvert pour la topologie de la distance \( d\), et vu que \( 1\in A\), il existe \( r>0\) tel que \( B_d(1,r)\subset A\). Pour ce \( r\) le lemme \ref{LEMooLINCooHJmJWx} donne
                \begin{equation}
                    B_d(1,r)=S^1\cap\{ x+iy\tq x>\cos(r) \}.
                \end{equation}
                Nous montrons que \( \mO=B_{\eC}(1,\delta)\) avec \( \delta<1-\cos(r)\) fait l'affaire. Si \( x+iy\in B_{\eC}(1,\delta)\), alors \( x>1-\delta\) et donc
                \begin{equation}
                    1-\delta>1-(1-\cos(r))=\cos(r),
                \end{equation}
                ce qui prouve que la partie de \( \mO\) qui est dans \( S^1\) est bien dans \( B_d(1,r)\).
            \item[Si \( a\neq 1\)]

                Soient un ouvert quelconque \( A\in\tau_d\) ainsi que \( a= e^{ix}\in A\). Nous considérons \( r>0\) tel que \( B_d(a,r)\subset A\); nous avons \(  e^{-ix}B_d(d,r)\subset  e^{-ix}A\) et donc, en tenant compte du lemme \ref{LEMooCQCAooAEctbe}\ref{ITEMooCIPYooTyPQLj} :
                \begin{equation}
                    B_d(1,r)\subset  e^{-ix}A.
                \end{equation}
                Par le premier point, il existe un ouvert \( \mO\) de \( \eC\) tel que \( 1\in \mO\) et
                \begin{equation}
                    \mO\cap S^1\subset B_d(1,r)\subset  e^{-ix}A.
                \end{equation}
                Nous avons évidemment que \( a\in e^{ix}\mO\) et
                \begin{equation}
                    e^{ix}(\mO\cap S^1)\subset  e^{ix}B_d(1,r)\subset A.
                \end{equation}
                Donc \(  e^{ix}\mO\cap S^1\subset A\). Vu que \(  e^{ix}\mO\) est un ouvert de \( \eC\), l'ensemble \(  e^{ix}\mO\cap S^1\) est un ouvert de \( \tau_i\).
        \end{subproof}
\end{subproof}
\end{proof}

\begin{lemma}[\cite{MonCerveau}]        \label{LEMooTKFHooJaeMyc}
    Deux résultats de limites dans \( S^1\).
    \begin{enumerate}
        \item       \label{ITEMooEUDIooDuynRg}
            Pour tout \( a_0\in S^1\), nous avons
            \begin{equation}
                \lim_{s\to 1} d(a,as)=0.
            \end{equation}
        \item       \label{ITEMooXCBUooUxQldB}
            Si \( f\colon S^1\to \eC\) est continue en \( a\in S^1\), alors 
            \begin{equation}
                \lim_{s\to 1} f(as)=f(a).
            \end{equation}
    \end{enumerate}
\end{lemma}

\begin{proof}
    Point par point.
    \begin{enumerate}
        \item
            Soient \( a,s\in S^1\). Donnons une formule pour \( d(a,as)\). Si \( a= e^{ix}\) et \( s= e^{iy}\) nous avons \( as= e^{i(x+y)}\) et donc
            \begin{equation}
                d(a,as)=\inf_{k\in \eZ}| x-(x+y)+2k\pi |=\int_{k\in \eZ}| y+2k\pi |.
            \end{equation}
            Voila pour la formule. Maintenant la preuve de notre point.

            Soit \( \epsilon>0\). Si \( \delta<\epsilon\) et si \( s\in B(1,\delta\), alors il existe \( y\in \mathopen] -\delta , \delta \mathclose[\) tel que \( s= e^{iy}\) par le lemme \ref{LEMooMYNVooIWWsiV}. Pour un tel \( s\) nous avons
                \begin{equation}
                    d(a,sa)=\inf_{k\in \eZ}| y+2k\pi |\leq | y |<\delta<\epsilon.
                \end{equation}
                Nous nous avons trouvé \( \delta>0\) tel que \( s\in B(1,\delta)\) implique \( d(a,as)<\epsilon\). Cela est la limite que nous devions prouver.
            \item
                Soit \( \epsilon>0\). Soit \( r>0\) tel que si \( b\in B(a,r)\), alors \( | g(b)-g(a) |<\epsilon\); l'existence d'un tel \( r\) est la continuité de \( g\) en \( a\). Nous considérons \( \delta>0\) tel que \( s\in B(1,\delta)\) implique \( sa\in B(a,r)\); l'existence d'un tel \( \delta\) est le point \ref{ITEMooEUDIooDuynRg} de ce lemme.

                Avec tout cela nous avons \( | g(as)-g(a) |<\epsilon\) dès que \( s\in B(1,\delta)\). Cela est la limite \( \lim_{s\to 1} g(as)=g(a)\) que nous voulions prouver.
    \end{enumerate}
\end{proof}

\begin{lemma}
    Pour \( s\in S^1\), nous considérons l'application
    \begin{equation}
        \begin{aligned}
            \alpha_s\colon \Fun(S^1)&\to \Fun(S^1) \\
            \alpha_s(g)(u)&=g(u\bar s)-g(u).
        \end{aligned}
    \end{equation}
    Quelque propriétés avec \( 1\leq p<\infty\) :
    \begin{enumerate}
        \item
            Si \( f\in L^p(S^1)\), alors \( \alpha_s(f)\in L^p(S^1)\).
        \item
            Si \( f\) est continue dans \( L^p(S^1)\) nous avons la limite
            \begin{equation}
                \lim_{s\to 1} \alpha_s(f)=0
            \end{equation}
            dans \( L^p(S^1)\).
    \end{enumerate}
\end{lemma}

\begin{proof}
    D'abord un calcul de norme :
    \begin{equation}
        \| \alpha_s(f) \|_p^p=\int_{S^1}| f(u\bar s)-f(u) |^pdu\leq\int_{S^1}| f(u\bar s) |^pdu+\int_{S^1}| f(u) |^pdu=2\| f \|_p^p.
    \end{equation}
    Donc oui pour que \( \alpha_s(f)\in L^p(S^1)\).

    La fonction \( f\) étant supposée continue sur le compact \( S^1\), elle est majorée. Nous savons qu'en posant \( \| f \|_{\infty}\) nous avons \( | \alpha_s(f) |\leq 2M\). Donc la fonction constante
    \begin{equation}
        \begin{aligned}
            g\colon S^1&\to \eC \\
            u&\mapsto 2M 
        \end{aligned}
    \end{equation}
    est une fonction intégrable sur \( S^1\) qui majore \( | \alpha_s(f) |\) uniformément en \( s\). Soit une suite \( s_i\to 1\) dans \( S^1\), et posons \( f_i=\alpha_{s_i}(f)\). Alors nous avons 
    \begin{equation}
        \| f_i \|^p_p=\int_{S^1}| f_i(u) |^pdu
    \end{equation}
    et aussi \( | f_i |^p\leq (2M)^p\). Le théorème de la convergence dominée de Lebesgue \ref{ThoConvDomLebVdhsTf} nous permet de permuter limite et intégrale :
    \begin{equation}
        \lim_{i\to \infty} \| f_i \|_p^p=\int_{S^1}\lim_{i\to \infty} | f_i(u) |^pdu.
    \end{equation}
    Mais
    \begin{equation}
        \lim_{i\to \infty} f_i(u)=\lim_{i\to \infty} \alpha_{s_i}(f)(u)=\lim_{i\to \infty} \big( f(u\bar s_i)-f(u) \big)=0.
    \end{equation}
    La dernière limite est due au fait que \(  \lim_{s\to 1} g(us)=g(u) \) (lemme \ref{LEMooTKFHooJaeMyc}\ref{ITEMooXCBUooUxQldB}).
\end{proof}

%--------------------------------------------------------------------------------------------------------------------------- 
\subsection{Système trigonométrique}
%---------------------------------------------------------------------------------------------------------------------------

\begin{definition}
    La \defe{famille trigonométrique}{famille trigonométrique!sur \( S^1\)} sur \( S^1\) est l'ensemble de fonctions \( \{ e_n \}_{n\in \eZ}\) données par
    \begin{equation}
        \begin{aligned}
            e_n\colon S^1&\to \eC \\
            z&\mapsto z^n
        \end{aligned}
    \end{equation}
    avec \( n\in \eZ\). Un \defe{polynôme trigonométrique}{polynôme trigonométrique} est une application \( S^1\to \eC\) de la forme
    \begin{equation}
        \sum_{k=-n}^na_ke_k
    \end{equation}
    pour des nombres \( a_k\in \eC\), peut-être pas tous non-nuls (autrement dit, il n'est pas forcé d'avoir autant de termes négatifs que positifs).
\end{definition}
Le but de \( z\mapsto z^n\) dans cette définition est d'être lu \(  t\mapsto e^{in t}\) lorsqu'on considère les fonctions sur \( \mathopen[ 0 , 2\pi \mathclose[\).

\begin{proposition}     \label{PROPooOMGFooROFFFr}
    La famille trigonométrique est une famille orthonormale pour le produit scalaire \( L^2(S^1,\tribA,\mu)\).
\end{proposition}

\begin{proof}
    En utilisant la formule \eqref{EQooCFPAooMcrDEQ} nous avons
    \begin{subequations}
        \begin{align}
            \langle e_n, e_n\rangle =\int_{S^1}e_n\overline{ e_n }&=\frac{1}{ 2\pi }\int_{\mathopen[ 0 , 2\pi \mathclose[}e_n\big( \varphi(x) \big)\overline{ e_n\big( \varphi(x) \big) }\\
                &=\frac{1}{ 2\pi }\int_{\mathopen[ 0 , 2\pi \mathclose[} e^{inx} e^{-inx}dx=\frac{1}{ 2\pi }\int_{0}^{2\pi}1dx=1.
        \end{align}
    \end{subequations}

    Et nous avons également, pour \( m\neq n\) :
    \begin{equation}
        \langle e_n, e_m\rangle =\frac{1}{ 2\pi }\int_{\mathopen[ 0 , 2\pi \mathclose[} e^{i(n-m)x}dx=\frac{1}{ 2\pi }\left[ \frac{1}{ i(n-m) e^{i(n-m)x} } \right]_0^{2\pi}=0.
    \end{equation}
\end{proof}

\begin{remark}
    Vous aurez noté que le facteur \( \frac{1}{ 2\pi }\) qui permet d'avoir \( \langle e_n, e_n\rangle=1 \) ne provient ni de la définition du produit scalaire ni de celle de la famille trigonométrique, mais bien de la mesure, voir la définition \ref{EQooOEVDooSYWrgT}.
\end{remark}

\begin{remark}      \label{REMooUCANooVyXPxj}
    Notez aussi que nous avons bien \( \langle e_n, e_{-n}\rangle =0\). Il faut donc bien prendre tous les \( e_n\) avec \( n\in \eZ\) et non seulement \( n\in \eN\).
\end{remark}

\begin{proposition}     \label{PROPooTGBHooXGhdPR}
    Les polynômes trigonométriques forment une partie dense dans \( \big( C(S^1,\eC),\| . \|_{\infty} \big)\).
\end{proposition}

\begin{proof}
    Pour préciser les notations, \( C(S^1,\eC)\) est l'ensemble des fonctions continues de \( S^1\) vers \( \eC\), et l'espace topologique que nous considérons est cet ensemble sur lequel nous considérons la distance supremum.

    Nous utilisons le théorème de Stone-Weierstrass \ref{ThoWmAzSMF}.

    Le système contient une fonction constante non nulle, à savoir \( e_0\).

    Il sépare les points grâce à la fonction \( e_1\) qui n'est autre que la fonction identité \( z\mapsto z\). De plus l'ensemble des polynômes trigonométriques est stable par conjugaison parce que si
    \begin{equation}
        P=\sum_{k=-n}^na_ke_k,
    \end{equation}
    alors \( \bar P=\sum_{k=-n}^n\overline{ a_k e_ka}=\sum_{k=-n}^n\overline{ a_k }e_{-k}\) qui est encore un polynôme trigonométrique.
\end{proof}

\begin{definition}
    Si nous avons une fonction \( f\colon S^1\to \eC\), nous définissons ses \defe{coefficients de Fourier}{coefficients de Fourier} par
    \begin{equation}
        c_n(f)=\langle f, e_n\rangle 
    \end{equation}
    pourvu que l'intégrale existe.
\end{definition}

%--------------------------------------------------------------------------------------------------------------------------- 
\subsection{Convolution}
%---------------------------------------------------------------------------------------------------------------------------

La convolution sur \( \eR^n\) est donnée par la définition \ref{DEFooHHCMooHzfStu}. Nous voyons maintenant comment cela s'adapte à \( S^1\).
\begin{definition}      \label{DEFooSKWOooEdIHoH}
    Si \( f\) et \( g\) sont des fonctions sur \( S^1\) à valeurs dans \( \eC\), nous définissons la \defe{convolution}{convolution sur \( S^1\)} de \( f\) et \( g\) comme étant la fonction sur \( S^1\) définie par
    \begin{equation}        \label{EQooILQNooBKtSBj}
        (f*g)(z)=\int_{S^1}f(s)g(z\bar s)d\mu(s).
    \end{equation}
\end{definition}

Cette définition appelle plusieurs remarques.
\begin{itemize}
    \item
        Dès que \( z,s\in S^1\), nous avons \( z\bar s\in S^1\), de telle sorte qu'au moins l'intégrande ait un sens.
    \item Nous ne prétendons pas que l'intégrale \eqref{EQooILQNooBKtSBj} converge pour toutes les fonctions \( f\) et \( g\). Cela est une définition «pour tous les couples \( f,g\) pour lesquels l'intégrale fonctionne».
    \item
        Le lemme \ref{LEMooTYSSooItOiYE} nous dira que \( L^1(S^1)\) est stable par convolution : si \( f\) et \( g\) sont dans \( L^1\), alors \( f*g\) y est aussi.
    \item
        Dans la formule \eqref{EQooILQNooBKtSBj}, la variable \( s\) est vraiment une variable muette. Cette formule aurait également pu être écrite
    \begin{equation}        
        (f*g)(z)=\int_{S^1} \big[ s\mapsto f(s)g(z\bar s) \big]d\mu.
    \end{equation}
\end{itemize}

\begin{lemma}[\cite{ooPIYUooRQCQRz}]        \label{LEMooTYSSooItOiYE}
    Si \( f,g\in L^1(S^1)\), alors pour presque tout \( z\in S^1\), la fonction \( s\mapsto f(s)g(z\bar s)\) est dans \( L^1(S^1)\).
\end{lemma}

\begin{proof}
    Nous considérons la fonction
    \begin{equation}
        \begin{aligned}
            \psi\colon S^1\times S^1&\to \eC \\
            (z,s)&\mapsto f(s)g(z\bar s). 
        \end{aligned}
    \end{equation}
    \begin{subproof}
    \item[\( \psi\in L^1(S^1\times S^1)\)]
        Nous utilisons le corollaire \ref{CorTKZKwP}, et pour cela nous calculons les intégrales en chaîne\footnote{Dans les expressions suivantes, les symboles «\( ds\)» et «\( dz\)» n'ont pas d'autres valeurs que purement de notation pour indiquer le nom de la variable d'intégration.} :
        \begin{subequations}
            \begin{align}
                \int_{S^1}\left[ \int_{S^1}| f(s)g(z\bar s) |dz \right]]ds&=\int_{S^1}| f(s) |\underbrace{\left[ \int_{S^1}| g(z\bar s) |dz \right]}_{=A<\infty}ds\\
                &=A\int_{S^1}| f(s)ds |\\
                &<\infty.
            \end{align}
        \end{subequations}
        Le fait que \( A<\infty\) provient directement de l'hypothèse \( g\in L^1(S^1)\)\footnote{Avec un changement de variables \( z\mapsto z\bar s\) que je vous conseille d'être capable de justifier.}.
        
        Par le corollaire sus-cité nous avons bien \( \psi\in L^1(S^1\times S^1)\).
    \item[Et par Fubini]
        Le théorème de Fubini \ref{ThoFubinioYLtPI}\ref{ITEMooVFGWooZTePQS} nous renseigne que pour presque tout \( z\in S^1\), l'application
        \begin{equation}
            s\mapsto \psi(z,s)
        \end{equation}
        est dans \( L^1(S^1)\). Et la partie \ref{ThoFubinioYLtPI}\ref{ITEMooCYMKooUdizni} ajoute que l'application
        \begin{equation}        \label{EQooPLLBooJZsZzu}
            z\mapsto \int_{S^1}\psi(s,z)ds
        \end{equation}
        est également \( L^1(S^1)\).
    \item[Conclusion]
        L'application donnée en \eqref{EQooPLLBooJZsZzu} est précisément \( (f*g)\). Donc \( f*g\in L^1(S^1)\).
    \end{subproof}
\end{proof}

\begin{lemma}[\cite{MonCerveau}]
    Si \( f\in L^1(S^1)\) et si \( g\) est continue sur \( S^1\), alors \( f*g\) existe et est continue sur \( S^1\).
\end{lemma}

\begin{proof}
    Vu que \( S^1\) est compact, la continuité de \( g\) implique que \( g\) est bornée et donc dans \( L^1(S^1)\). Le lemme \ref{LEMooTYSSooItOiYE} dit alors que \( f*g\) est bien définie sur \( S^1\).

    Soit \( z_0\in S^1\). Nous montrons que \( f*g\) est continue en \( z_0\); pour cela nous considérons \( \epsilon>0\) et ensuite nous réfléchissons un peu. 
    
    Vu que \( g\) est continue sur \( S^1\) qui est compact, \( g\) y est uniformément continue par le théorème de Heine\ref{PROPooBWUFooYhMlDp}. Il existe donc un \( \delta>0\) tel que pour tout \( z_0\in S^1\), si \( z\in B(z_0,\delta)\), alors \( | g(z_0)-g(z) |<\epsilon\).

    Soit \( s\in S^1\). Si \( z\in B(z_0,\delta)\), alors \( \bar sz\in B(\bar sz_0,\delta)\) par le lemme \ref{LEMooCQCAooAEctbe}\ref{ITEMooCIPYooTyPQLj}. Dans ce cas nous avons aussi
    \begin{equation}
        | g(\bar s z_0)-g(\bar sz) |<\epsilon.
    \end{equation}
    Un peu de calcul maintenant. D'une part
    \begin{equation}
        (f*g)(z_0)-(f*g)(z)=\int_{S^1}f(s)\big( g(z_0\bar s)-g(z\bar s) \big)ds,
    \end{equation}
    et donc
    \begin{subequations}
        \begin{align}
            |(f*g)(z_0)-(f*g)(z)|&\leq\int_{S^1}|f(s)| \underbrace{\big|  g(z_0\bar s)-g(z\bar s) \big|}_{<\epsilon} ds\\
            &\leq \epsilon\int_{S^1}| f |\\
            &=A\epsilon
        \end{align}
    \end{subequations}
    pour une certaine constante \( A\) ne dépendant pas de \( z_0\).

    Nous avons prouvé que pour tout \( \epsilon>0\), il existe un \( \delta\) tel que \( z\in B(z_0,\alpha)\) implique 
    \begin{equation}
            |(f*g)(z_0)-(f*g)(z)|\leq A\epsilon,
    \end{equation}
    ce qui signifie que \( f*g\) est continue en \( z_0\).
\end{proof}

Notez que dans cette démonstration, l'uniforme continuité de \( g\) a été utilisée pour effectuer d'un seul coup la majoration pour tout \( s\) dans l'intégrale.

\begin{proposition}     \label{PROPooCSRNooDyClBY}
    Si \( f\in L^1(S^1)\), nous avons
    \begin{equation}
        f*e_n=c_n(f)e_n.
    \end{equation}
\end{proposition}

\begin{proof}
    Il s'agit d'un bon calcul. En considérant \( z= e^{i\theta}\) nous avons
    \begin{subequations}
        \begin{align}
            (f*e_n)(z)&=\int_{S^1}f(s)e_n(z\bar s)ds\\
            &=\frac{1}{ 2\pi }\int_{\mathopen[ 0 , 2\pi \mathclose[}f( e^{ix})e_n(  e^{i(\theta-x)}) dx\\
            &= e^{in\theta}\frac{1}{ 2\pi }\int_{\mathopen[ 0 , 2\pi \mathclose[}f( e^{ix}) e^{-inx}dx\\
            &=e_n( e^{i\theta})\frac{1}{ 2\pi }\int_{\mathopen[ 0 , 2\pi \mathclose[}f( e^{ix})\overline{ e_n( e^{ix}) }dx\\
                &=e_n(z)\int_{S^1}f(s)\overline{ e_n(s) }ds\\
                &=e_n(z)\langle f, e_n\rangle.
        \end{align}
    \end{subequations}
    Donc \( (f*e_n)(z)=\langle f, e_n\rangle e_n(z)\), c'est-à-dire que
    \begin{equation}
        f*e_n=c_n(f)e_n.
    \end{equation}
\end{proof}

\begin{lemma}       \label{LEMooDGHJooRAnwpy}
    Si \( P\) est un polynôme trigonométrique et si \( f\in L^1(S^1)\), alors \( f*P\) est également un polynôme trigonométrique.
\end{lemma}

\begin{proof}
    Soit \( P=\sum_{k=-n}^na_ke_k\). Par la linéarité du produit de convolution,
    \begin{equation}
        f*P=\sum_{k=-n}^na_kf*e_k=\sum_ka_kc_k(f)e_k
    \end{equation}
    où nous avons également utilisé la proposition \ref{PROPooCSRNooDyClBY}. Nous avons donc un polynôme trigonométrique dont les coefficients sont \( a_kc_k(f)\) au lieu de \( a_k\).
\end{proof}

%--------------------------------------------------------------------------------------------------------------------------- 
\subsection{Approximation de l'unité}
%---------------------------------------------------------------------------------------------------------------------------

\begin{lemma}[\cite{TUEWwUN}]       \label{LEMooUNFBooRCzwIn}
    Soient une fonction continue \( f\colon S^1\to \mathopen[ 0 , \infty \mathclose[\) et \( a\in S^1\) telle que \( f(z)<f(a)\) pour tout \( z\in S^1\setminus\{ a \}\). Alors la suite de fonctions \( f_n\colon S^1\to \eR\) donnée par
    \begin{equation}        \label{EQooTQYPooLZprJj}
        f_n(z)=\left( \int_{S^1}f^n \right)^{-1}f(z)^n
    \end{equation}
    est une approximation de l'unité\footnote{Définition \ref{DEFooEFGNooOREmBb}.} autour de \( a\).
\end{lemma}

\begin{proof}
    En plusieurs points, dont d'abord une série de vérifications pour voir que la formule a un sens.
    \begin{subproof}
    \item[Strictement positive]
    D'abord, vu que \( f\) prend ses valeurs dans \( \mathopen[ 0 , \infty \mathclose[\) et vu que \( f(z)<f(a)\), nous avons \( f(a)>0\) (strict). Peut-être que \( f\) s'annule à certains endroits de \( S^1\), mais pas \( a\).
    \item[\( f^n\) est intégrable sur \( S^1\)]
        La fonction \( f^n\) est dans les hypothèses de la proposition \ref{PROPooKFRSooANzglT} parce que \( S^1\) est compact, \( f^n\) y est continue et la mesure sur \( S^1\) est compatible avec la topologie (voir les hypothèses précises).
    \item[L'intégrale n'est pas nulle]
        Vu que \( f(a)>0\), il existe un ouvert \( A\) contenant \( a\) sur lequel \( f>0\). Nous avons alors
        \begin{equation}
            \int_Kf^n\geq \int_Af^n>0.
        \end{equation}
        Cela pour dire que l'inverse dans \eqref{EQooTQYPooLZprJj} ne pose pas de problèmes.
    \item[Norme]
        Vu que toutes les fonctions tant \( f\) que \( f_n\) sont positives, les valeurs absolues ne jouent aucun rôle et nous avons
        \begin{equation}
            \| f_n \|_1=\int_{S^1}f_nd\mu=\left( \int_{S^1}f^n \right)^{-1}\int_{S^1}| f(z) |^nd\mu=1.
        \end{equation}
        Ce calcul donne d'un seul coup les deux conditions
        \begin{itemize}
            \item \( \sup_k\| f_k \|=1\)
            \item \( \int_{S^1}f_n=1\) pour tout \( n\).
        \end{itemize}
    \end{subproof}
    Nous passons maintenant au vrai travail.
    Soit un voisinage \( V\) de \( a\) dans \( S^1\). Soit une suite croissante \( (t_k)\) qui converge vers \( f(a)\), c'est-à-dire \( 0<t_k<f(a)\). Nous posons
    \begin{equation}
        A_k=\{ x\in S^1\setminus V\tq f(x)\geq t_k \}.
    \end{equation}
    Cet ensemble est contenu dans \( S^1\) et est donc borné (pour la métrique de \( S^1\)).    
    \begin{subproof}
    \item[\( A_k\) est fermé]
        Attention : ici nous démontrons que \( A_k\) est fermé dans \( S^1\), et les complémentaires sont pris dans \( S^1\).

        Nous montrons que le complémentaire est ouvert en prenant \( y\in A^c\) et en montrant que \( y\) admet un voisinage contenu dans \( A^c\) (le fameux théorème \ref{ThoPartieOUvpartouv} que nous ne nous lasserons jamais de citer). Si \( y\in A^c\), il y a deux possibilités (non exclusives) : soit \( y\in V\) soit \( f(y)<t_k\). Si \( y\in V\), alors le voisinage \( V\) lui-même est encore dans \( A^c\). Si par contre \( f(y)<t_k\), alors par continuité, il existe un voisinage de \( y\) sur lequel \( f<t_k\).
    \item[\( A_k\) est compact]
        L'espace \( S^1\) est compact, par exemple grâce au lemme \ref{LEMooVYTRooKTIYdn}. La partie \( A_k\) est fermée dans le compact \( S^1\), donc elle est compacte par le lemme \ref{LemnAeACf}.
    \item[\( A_{k+1}\subset A_k\)]
        Si \( x\in A_{k+1}\), alors \( f(x)\geq t_{k+1}>t_k\). Donc \( f(x)>t_k\) et \( x\in A_k\).
    \item[Intersection vide]
        Si \( x\in\bigcap_{k\in \eN}A_k\), alors \( f(x)\geq t_k\) pour tout \( k\). En passant à la limite et en sachant que \( \lim_{k\to \infty} t_k=f(a)\), nous avons \( f(x)\geq f(a)\). Par hypothèse, cela n'est pas. Donc
        \begin{equation}
            \bigcap_{k\in \eN}A_k=\emptyset.
        \end{equation}
        Nous avons, dans un compact, des fermés emboîtés dont l'intersection est vide. Le corollaire \ref{CORooQABLooMPSUBf} nous dit qu'il existe un indice à partir duquel tous les \( A_k\) sont vides.
    \end{subproof}
    
    Soit \( \delta=t_k\) pour un \( k\) tel que \( A_k\) est vide. Nous avons
    \begin{equation}
        \{ x\in S^1\setminus V\tq f(x)\geq \delta \}=\emptyset,
    \end{equation}
    c'est-à-dire que sur \( S^1\setminus V\), nous avons \( f<\delta\) et donc
    \begin{equation}
        \int_{S^1\setminus V}f(s)^n<\int_{S^1\setminus V}\delta^n=\vol(S^1\setminus V)\delta^n
    \end{equation}
    où \( \vol(S^1\setminus V)=\int_{S^1\setminus V}1=\mu'(S^1\setminus V)\) est une constante réelle strictement positive.

    Nous avons aussi \( \delta<f(a)\) parce que \( \delta\) est un des \( t_k\) (et que cette suite croissante converge vers \( f(a)\) sans l'atteindre par hypothèse). Soit \( \delta_1\) tel que \( \delta<\delta_1<f(a)\)\quext{Dans \cite{TUEWwUN}, il prend \( \delta<\delta_1<1\) et je crois qu'il aurait dû écrire \( \varphi(0)\) au lieu de \( 1\).}.

    Nous posons
    \begin{equation}
        W=\{ x\in S^1\tq f(x)>\delta_1 \}.
    \end{equation}
    Cet ensemble n'est pas vide parce qu'il contient \( a\) et est ouvert parce que \( f\) est continue. Nous avons
    \begin{equation}
        \int_{S^1}f(s)^nds\geq \int_{W}f(s)^nds\geq \delta_1^n\vol(W).
    \end{equation}
    
    Nous avons donc déjà ces deux inégalités :
    \begin{equation}
        \int_{S^1\setminus V}f(s)^n<=\vol(S^1\setminus V)\delta^n
    \end{equation}
    et
    \begin{equation}
        \int_{S^1}f(s)^nds\geq \delta_1^n\vol(W).
    \end{equation}

    En ce qui concerne les fonctions \( f_n\) que nous voulions étudier,
    \begin{equation}
        f_n(z)=\left( \int_{S^1}f(s)^nds \right)^{-1}f(z)^n\leq \big( \vol(W)\delta_1^n \big)^{-1}f(z)^n,
    \end{equation}
    et donc
    \begin{equation}
        \int_{S^1\setminus V}f_n\leq\big( \vol(W)\delta_1^n \big)^{-1}\vol(S^1\setminus V)\sigma^n=\frac{ \vol(S^1\setminus V) }{ \vol(W) }\left( \frac{ \delta }{ \delta_1 } \right)^n.
    \end{equation}
    Étant donné que \( \delta<\delta_1\), nous avons \( (\delta/\delta_1)^n\to 0\). Donc aussi
    \begin{equation}
        \lim_{n\to \infty} \int_{S^1\setminus V}f_n(z)dz=0.
    \end{equation}
\end{proof}

Le théorème suivant est une version pour \( S^1\) du théorème \ref{ThoYQbqEez}. Le produit de convolution dans \( S^1\) est la définition \ref{DEFooSKWOooEdIHoH}.
\begin{theorem}[\cite{TUEWwUN,MonCerveau}]         \label{THOooIAOPooELSNxq}
    Soient \( (\varphi_k)\) une approximation de l'unité sur \( \Omega=S^1\) ainsi qu'une fonction \( g\colon S^1\to \eC\).
    \begin{enumerate}
        \item       \label{ITEMooNUDFooYLFIwR}
            Si \( g\) est mesurable et bornée sur \( \Omega\) et si \( g\) est continue en \( a_0\) alors
            \begin{equation}
                (\varphi_k*g)(a_0)\to g(a_0).
            \end{equation}
    \end{enumerate}
\end{theorem}

\begin{proof}
    Pour chaque \( k\in \eN\) nous posons 
    \begin{equation}
        d_k=\varphi_k*g-g.
    \end{equation}
    Le but de ce théorème est de montrer que \( d_k\to 0\) pour diverses notions de convergence.
    \begin{subproof}
        \item[Preuve du point \ref{ITEMooNUDFooYLFIwR}]
            Soit \( a_0\in S^1\). Par définition de l'approximation de l'unité, \( \int_{S^1}\varphi_k=1\) et donc on peut écrire \( g(a_0)=\int_{S^1}g(a_0)\varphi_k(s)ds\). En ce qui concerne \( d_k(a_0)\) nous avons alors
            \begin{subequations}
                \begin{align}
                    d_k(a_0)&=\int_{S^1}\varphi_k(s)g(a_0\bar s)ds-\int_{S^1}g(a_0)\varphi_k(s)ds\\
                    &=\int_{S^1}\varphi_k(s)\big( g(a_0\bar s)-g(a_0) \big).
                \end{align}
            \end{subequations}
            Nous pouvons passer à la norme (et non la valeur absolue parce que \( d_k\) prend ses valeurs dans \( \eC\)) :
            \begin{equation}
                | d_k(a_0) |\leq \int_{S^1}| \varphi_k(s) |\big| g(a_0\bar s)-g(a_0) \big|ds.
            \end{equation}
            La définition d'une approximation de l'unité nous permet de considérer \( M=\sup_k\| \varphi_k \|_1<\infty\). Le lemme \ref{LEMooTKFHooJaeMyc}\ref{ITEMooXCBUooUxQldB} nous permet, lui, de considérer \( \alpha>0\) tel que 
            \begin{equation}
                | g(a_0\bar s)-g(a_0) |<\epsilon
            \end{equation}
            dès que \( s\in B(1,\alpha)\)\footnote{Notez que \( s\in B(1,\alpha)\) si et seulement si \( \bar s\in B(1,\alpha)\). Il n'y a donc pas d'incohérence entre l'hypothèse sur \( s\) et notre condition sur \( g(a_0\bar s)\)}. Vu que la suite \( (\varphi_k)\) est une approximation de l'unité, nous avons
            \begin{equation}
                \int_{S^1\setminus B(1,\alpha)}| \varphi_k |=0.
            \end{equation}
            Soit \( k\) suffisamment grand pour avoir \( \int_{S^1\setminus B(1,\alpha)}| \varphi_k |<\epsilon\). Avec tout cela nous avons les majorations
            \begin{subequations}
                \begin{align}
                    | d_k(a_0) |&\leq \int_{S^1\setminus B(1,\alpha)}| \varphi_k(s) |\big| g(a_0\bar s)-g(a_0) \big|ds\\
                    &=\int_{S^1\setminus B(1,\alpha)}| \varphi_k(s) |\underbrace{\big| g(a_0\bar s)-g(a_0) \big|}_{\leq 2\| g \|_{\infty}}ds+\int_{B(1,\alpha)}| \varphi_k(s) |\underbrace{\big| g(a_0\bar s)-g(a_0) \big|}_{<\epsilon}ds\\
                    &\leq 2\| g \|_{\infty}\int_{S^1\setminus B(1,\alpha)}| \varphi_k(s) |ds+\epsilon\int_{S^1}| \varphi_k(s) |ds\\
                    &\leq \epsilon\big( 1+2\| g \|_{\infty} \big).
                \end{align}
            \end{subequations}
            Nous avons donc bien \( \lim_{k\to \infty} | d_k(a_0) |=0\) et donc la continuité de \( \varphi_k*g\) en \( a_0\).

    \end{subproof}
\end{proof}

Voici une version un peu forte sous l'hypothèse de continuité. Vu que \( S^1\) est compact, la continuité est en réalité une hypothèse assez forte : ça implique l'uniforme continuité et l'existence d'un maximum et d'un minimum.
\begin{proposition}
    Soient \( (\varphi_k)\) une approximation de l'unité sur \( \Omega=S^1\) ainsi qu'une fonction continue \( g\colon S^1\to \eC\).
    \begin{enumerate}
        \item       \label{ITEMooPHBJooOHDVoW}
        Si \( g\in L^p(\Omega)\) (\( 0\leq p<\infty\)) et si \( g\) est continue, alors\footnote{Vous noterez les \( p\in \mathopen] 0 , 1 \mathclose[\) en bonus par rapport au cas de \( \eR^n\).}
            \begin{equation}
                \varphi_k*g\stackrel{L^p}{\to}g.
            \end{equation}
        \item       \label{ITEMooLOSVooDtaugF}
            Si \( g\) est continue sur \( S^1\), alors
            \begin{equation}
                \varphi_k*g\stackrel{L^{\infty}}{\to}g
            \end{equation}
    \end{enumerate}
\end{proposition}

\begin{proof}
    
    En plusieurs points
    \begin{subproof}

        \item[\( \lim_{u\to 1} \| \tau_u(g)-g \|=0\)]
            Ceci est un petit point intermédiaire. Pour des besoins de notations, nous posons
            \begin{equation}
                \begin{aligned}
                    \tau_u(g)\colon S^1&\to \eC \\
                    s&\mapsto g(s\bar u)
                \end{aligned}
            \end{equation}
            pour \( u\in S^1\). 

            La fonction \( g\) est continue sur le compact \( S^1\), et y est donc uniformément continue\footnote{Théorème de Heine \ref{PROPooBWUFooYhMlDp}. C'est fondamentalement ce fait qui unifie les parties \ref{ITEMooPHBJooOHDVoW} et \ref{ITEMooLOSVooDtaugF} de cette preuve.}. Nous allons en déduire que \( \lim_{u\to 1} \| \tau_u(g)-g \|_{\infty}=0\).

            Soit \( \epsilon>0\). L'uniforme continuité de \( g\) signifie qu'il existe \( \delta>0\) tel que pour tout \( a\in S^1\) si \( s\in B(a,\delta)\), alors \( \big| g(s)-g(a) \big|<\epsilon\). Si \( u\in B(1,\delta)\) nous avons aussi \( \bar u\in B(1,\delta)\) et donc \( s\bar u\in B(s,\delta)\); ça c'est le lemme \ref{LEMooCQCAooAEctbe}\ref{ITEMooCIPYooTyPQLj}.

            Pour tout \( a\in S^1\) nous avons la chaîne
            \begin{equation}
                u\in B(1,\delta)\Rightarrow a\bar u\in B(a,\delta)\Rightarrow \big| g(a\bar u)-g(a) \big|<\epsilon.
            \end{equation}
            Cela étant valable pour tout \( a\), c'est encore valable en passant au supremum\footnote{Notez l'inégalité qui n'est plus stricte.} :
            \begin{equation}
                u\in B(1,\delta)\Rightarrow\sup_{a\in S^1}\big| g(a\bar u)-g(a) \big|\leq\epsilon
            \end{equation}
            et donc d'accord pour
            \begin{equation}
                \lim_{u\to 1} \| \tau_u(g)-g \|=0.
            \end{equation}

        \item[\( \| d_k \|_{\infty}\to 0\)]
            Nous prouvons la convergence uniforme sur \( S^1\) de \( d_k\) vers zéro. Ensuite nous verrons que la compacité de \( S^1\) permet d'en déduire les points \ref{ITEMooPHBJooOHDVoW} et \ref{ITEMooLOSVooDtaugF}.
            
            En utilisant la notation \( \tau_u\), nous pouvons écrire
            \begin{equation}
                d_k(s)=\int_{S^1}\varphi_k(u)g(s\bar u)du-g(s)=\int_{S^1}\varphi_k(u)\big( \underbrace{ g(s\bar u)}_{=\tau_u(g)(s)}-g(s) \big)du,
            \end{equation}
            et donc
            \begin{equation}
                | d_k(s) |\leq \int_{S^1}| \varphi_k(u) |\| \tau_u(g)-g \|_{\infty}du.
            \end{equation}
            Nous posons \( M=\sup_k\| \varphi_k \|_1\), et nous considérons \( \delta\) tel que \( \| \tau_u(g)-g \|_{\infty}<\epsilon\) pour tout \( u\in B(1,\delta)\). Ensuite nous subdivisons \( S^1\) en \( B(1,\delta)\) et \( B(1,\delta)^c\) :
            \begin{subequations}
                \begin{align}
                    \| d_k \|_{\infty}&\leq \int_{B(1,\delta)}| \varphi_k(u) |\underbrace{\| \tau_u(g)-g \|_{\infty}}_{\leq \epsilon}du+\int_{B(1,\delta)^c}| \varphi_k(u) |\| \tau_u(g)-g \|_{\infty}du\\
                    &\leq\epsilon M+2\| g \|_{\infty}\int_{B(1,\delta)^c}| \varphi_k(u) |du\\
                    &\leq \epsilon\big( M+2\| g \|_{\infty} \big)
                \end{align}
            \end{subequations}
            parce que pour chaque \( s\in S^1\) nous avons \( \tau_u(g)(s)-g(s)\) et donc \( \| \tau_u(g)-g \|_{\infty}\leq 2\| g \|_{\infty}\).

            Tout cela montre que \( d_k\stackrel{\| . \|_{\infty}}{\longrightarrow}0\).

        \item[Convergence \( L^p\), \( 0<p<\infty\)]

            Soit \( \epsilon>0\). Nous avons
            \begin{equation}
                \| d_k \|_p^p\leq \int_{S^1}\big| (\varphi_k*g)(s)-g(s) \big|^pds.
            \end{equation}
            Il existe un \( k\) à partir duquel \( \| \varphi_k*g-g \|_{\infty}<\epsilon\). Pour de tels \( k\) nous avons 
            \begin{equation}
                \| d_k \|_p^p<\epsilon^p.
            \end{equation}
            Ce passage est très possible dans le cas de \( S^1\) parce que \( \int_{S^1}1=1\). Dans le cas de \( \eR^d\), c'est pas du tout bon; c'est pour cela que nous avons un résultat un peu plus fort dans \( S^1\). La croissance de la fonction puissance (proposition \ref{PROPooRXLNooWkPGsO}) nous permet de conclure que \( \| d_k \|_p<\epsilon\).

            Nous avons donc la convergence \( L^p\) pour \( 0<p<\infty\).
        \item[Convergence \( L^{\infty}\)]
 
            Non, la convergence \( L^{\infty}\) n'est pas la convergence pour la norme \( \| . \|_{\infty}\). Voir la sous-section \ref{SUBSECooYFJTooBqrLXv}. Il n'en reste pas moins que si \( \epsilon>0\) et si \( k\) est assez grand pour que \( \| f_k-f \|_{\infty}<\epsilon\), nous aurons
            \begin{equation}
                N_{\infty}(f_k-f)\leq \| f_k-f \|<\epsilon.
            \end{equation}
    \end{subproof}
\end{proof}

%--------------------------------------------------------------------------------------------------------------------------- 
\subsection{Base hilbertienne (suite des polynômes trigononétriques)}
%---------------------------------------------------------------------------------------------------------------------------

Voici le plan pour la suite :
\begin{itemize}
    \item Construire un polynôme trigonométrique qui vérifie les hypothèse du lemme \ref{LEMooUNFBooRCzwIn}.
    \item En déduire une approximation de l'unité constituée de polynômes trigonométriques.
    \item Dire que si \( f\in L^2(S^1)\), alors \( f*\varphi_k\) est un polynôme trigonométrique dès que \( \varphi_k\) en est un.
    \item Invoquer le théorème \ref{THOooIAOPooELSNxq}\ref{ITEMooPHBJooOHDVoW} pour déduire que \( \varphi_k*f\) est une suite de polynômes trigonométriques dans \( L^2(S^1)\) qui converge \( \varphi_k*f\stackrel{L^2}{\longrightarrow}f\).
\end{itemize}

\begin{lemma}       \label{LEMooQQILooWlhntZ}
    La fonction \( P=e_1+e_{-1}\) est continue à valeurs réelles sur \( S^1\).
\end{lemma}

\begin{proof}
    Nous avons \( e_1(z)=z\) et \( e_{-1}(z)=z^{-1}\), c'est-à-dire que pour \( z= e^{ix}\) (\( x\in \eR\)), nous avons \( e_{-1}( e^{ix})= e^{-ix}\), de telle sorte que, en utilisant le lemme \ref{LEMooHOYZooKQTsXW} qui donne \(  e^{ix}\) en termes des fonctions trigonométriques usuelles :
    \begin{equation}
        (e_1+e_{-1})( e^{ix})= e^{ix}+ e^{-ix}=\cos(x)+i\sin(x)+\cos(x)-i\sin(x)=2\cos(x).
    \end{equation}
    Nous avons donc la continuité et les valeurs réelles.
\end{proof}

\begin{lemma}       \label{LEMooIDTVooYTpfEm}
    Il existe un polynôme trigonométrique à valeurs dans \( \mathopen[ 0 , \infty \mathclose[\) et tel que \( f(a)<f(1)\) pour tout \( a\neq 1\) dans \( S^1\).
\end{lemma}

\begin{proof}
    Le lemme \ref{LEMooQQILooWlhntZ} nous dit déjà que \( P=e_1+e_{-n}\) est continue à valeurs réelles. Or qui est continue sur un compact (ici \( S^1\)), atteint donc ses bornes. Il est donc facile de considérer\footnote{Par exemple, \( M\) est le maximum de \( | P |\).} \( M>0\) tel que \( Q=M+e_1+e_{-1}\) est à valeurs dans \( \mathopen[ 0 , \infty \mathclose[\).

    Une forme explicite de \( Q\) est que
    \begin{equation}
        Q( e^{ix})=M+2\cos(x).
    \end{equation}
    Le maximum de \( \cos(x)\) est obtenu en \( x=0\) et vaut \( 1\). Le maximum de \( Q\) est alors \( Q(1)=2+M\). Il n'est atteint qu'une seule fois sur \( S^1\) parce que pour avoir \( Q( e^{ix})=2+M\), il faut avoir \( 2\cos(x)=2\), c'est-à-dire \( x=2k\pi\). Mais \(  e^{i2k\pi}=1\).

    Donc \( Q(a)<M+2=Q(1)\) pour tout \( a\neq 1\) dans \( S^1\).
\end{proof}

\begin{proposition}
    Les polynômes trigonométriques \( \{ e_n \}_{n\in \eZ}\) forment une base hilbertienne de \( L^2(S^1)\).
\end{proposition}

\begin{proof}
    Le fait que les \( e_n\) soient orthonormée est la proposition \ref{PROPooOMGFooROFFFr}. Il reste à prouver que ce soit un système total. 
    
    Soit \( f\in L^2(S^1)\). Soit un polynôme \( Q\) vérifiant le lemme \ref{LEMooIDTVooYTpfEm}; nous posons
    \begin{equation}
        \varphi_k(z)=\left( \int_{S^1}Q^n \right)^{-1}Q(z)^n.
    \end{equation}
    Cela est une approximation de l'unité par la proposition \ref{LEMooUNFBooRCzwIn}. Les \( \varphi_k\) sont des polynômes trigonométriques parce que les \( Q^n\) le sont et que que \( \int_{S^1}Q^n\) est seulement un nombre.

    Le lemme \ref{LEMooDGHJooRAnwpy} nous dit alors que pour tout \( k\), la fonction
    \begin{equation}
        \varphi_k*f
    \end{equation}
    est un polynôme trigonométrique.
\end{proof}

Nous nous permettons de confirmer la remarque \ref{REMooUCANooVyXPxj} comme quoi il faut bien tous les \( e_n\) avec \( n\in \eZ\), parce que le polynôme trigonométrique \( Q\) est bien construit à partir de \( e_1+e_{-1}\).

%--------------------------------------------------------------------------------------------------------------------------- 
\subsection{Convolution, bis}
%---------------------------------------------------------------------------------------------------------------------------

\begin{lemma}       \label{LEMooLUBQooWLMFrN}
    Nous considérons l'application
    \begin{equation}
        \begin{aligned}
            \varphi\colon \eR&\to S^1 \\
            t&\mapsto e^{it}.
        \end{aligned}
    \end{equation}
    Soient \( t,u\in \eR\) tels que \( \varphi(t)=\varphi(u)\). Alors pour toutes fonctions pour lesquelles les intégrales convergent,
    \begin{equation}
        \int_0^{2\pi}(f\circ \varphi)(\theta)(g\circ\varphi)(t-\theta)\frac{ d\theta }{ 2\pi }=\int_0^{2\pi}(f\circ \varphi)(\theta)(g\circ\varphi)(u-\theta)\frac{ d\theta }{ 2\pi }.
    \end{equation}
\end{lemma}

\begin{proof}
    Si \( \varphi(u)=\varphi(t)\), alors \( u=t+2k\pi\) pour un certain \( k\in \eZ\). Cette condition implique que \( \varphi(t-\theta)=\varphi(u-\theta)\), et donc l'égalité 
    \begin{equation}
        \int_0^{2\pi}(f\circ \varphi)(\theta)(g\circ\varphi)(t-\theta)\frac{ d\theta }{ 2\pi }=\int_0^{2\pi}(f\circ \varphi)(\theta)(g\circ\varphi)(u-\theta)\frac{ d\theta }{ 2\pi }.
    \end{equation}
\end{proof}

\begin{definition}[Convolution sur \( S^1\)]
    Le lemme \ref{LEMooLUBQooWLMFrN} permet de définir
    \begin{equation}
        (f*g)\big( \varphi(t) \big)=\int_0^{2\pi}(f\circ \varphi)(\theta)(g\circ\varphi)(t-\theta)\frac{ d\theta }{ 2\pi }
    \end{equation}
    pour toutes les paires de fonctions \( f,g\in \Fun(S^1,\eC)\) pour lesquelles l'intégrale converge.
\end{definition}

%+++++++++++++++++++++++++++++++++++++++++++++++++++++++++++++++++++++++++++++++++++++++++++++++++++++++++++++++++++++++++++ 
\section{L'espace \( L^2\big( \mathopen[ a , b \mathclose] \big)\)}
%+++++++++++++++++++++++++++++++++++++++++++++++++++++++++++++++++++++++++++++++++++++++++++++++++++++++++++++++++++++++++++

L'espace \( L^2\big( \mathopen[ a , b \mathclose] \big)\) est l'espace générique sur lequel nous allons construire les espaces \( L^2\) sur \( \mathopen[ -T , T \mathclose]\) et \( \mathopen[ 0 , 2\pi \mathclose]\).

Si \( f\) et \( g\) sont dans \( L^2\big( \mathopen[ a , b \mathclose] \big)\), il n'est pas possible de définir \( f*g\) par la formule intégrale usuelle parce que \( f(x_0+t)\) n'existe pas pour tout \( x_0\) et \( t\) dans \( \mathopen[ a , b \mathclose]\). Donc soit nous utilisons un truc pas très net comme étendre les fonctions sur \( \mathopen[ a , b \mathclose]\) en fonctions périodiques sur \( \eR\), soit nous intégrons vraiment seulement sur \( \mathopen[ a , b \mathclose]\).

Nous n'allons suivre aucune de ces deux voies ou plutôt les deux en même temps. Nous allons seulement tout ramener de \( S^1\) que nous venons de travailler. 

\begin{definition}
    Sur \( \mathopen[ a , b \mathclose]\) nous considérons la mesure de Lebesgue et le produit
    \begin{equation}
        \langle f, g\rangle =\frac{1}{ b-a }\int_a^bf(x)\bar g(x)dx.
    \end{equation}
\end{definition}

Si vous voulez une raison inavouable pour justifier ce facteur \( \frac{1}{ b-a }\), remarquez que \( dx\) a pour unité le mètre. En mettant la facteur \( b-a\) (qui a aussi le mètre comme unité), le tout a les unités de \( fg\), comme il se doit pour le produit scalaire.

\begin{proposition}
    L'application
    \begin{equation}
        \begin{aligned}
            \phi\colon L^2\big( \mathopen[ a , b \mathclose] \big)&\to L^2(S^1) \\
            \phi(f)(z)&=f\big( (s^{-1}\circ \varphi^{-1})(z) \big)
        \end{aligned}
    \end{equation}
    où \( s\) est donnée par
    \begin{equation}
        \begin{aligned}
            s\colon \mathopen[ a , b \mathclose]&\to \mathopen[ 0 , 2\pi \mathclose] \\
            x&\mapsto 2\pi\frac{ x-a }{ b-a }
        \end{aligned}
    \end{equation}
    et \( \varphi\) est l'application usuelle
    \begin{equation}
        \begin{aligned}
            \varphi\colon \mathopen[ 0 , 2\pi \mathclose[&\to S^1 \\
                t&\mapsto  e^{it}
        \end{aligned}
    \end{equation}
    est une bijection isométrique.
\end{proposition}

\begin{proof}
    La preuve du fait que \( \phi\) est isométrique suffira pour prouver qu'elle prend bien ses valeurs dans \( L^2(S^1)\).
    \begin{subproof}
        \item[Isométrique]
            C'est un calcul :
            \begin{subequations}
                \begin{align}
                    \| \phi(f) \|^2&=\langle \phi(f), \phi(f)\rangle \\
                    &=\int_{S^1}| \phi(f) |^2\\
                    &=\frac{1}{ 2\pi }\int_{\mathopen[ 0 , 2\pi \mathclose[}| \phi(f)\big( \varphi(u) \big) |^2du\\
                        &=\frac{1}{ 2\pi }\int_0^{2\pi}| f\big( (s^{-1}\circ\varphi^{-1}\circ\varphi)(u) \big) |^2du\\
                        &=\frac{1}{ 2\pi }\int_0^{2\pi}| f\big( s^{-1}(u) \big) |^2du.
                \end{align}
            \end{subequations}
            Il est temps de faire le changement de variables\footnote{Nous le faisons de façon un peu informelle; soyez capable de bien justifier.} \( y=s^{-1}(u)\), c'est-à-dire
            \begin{equation}
                y=\frac{ b-a }{ 2\pi }u+a.
            \end{equation}
            En ce qui concerne la différentielle,
            \begin{equation}
                dy=\frac{ b-a }{ 2\pi }du
            \end{equation}
            et pour les bornes, si \( u=0\) alors \( y=a\) et si \( u=2\pi\), \( y=b\). Donc
            \begin{subequations}
                \begin{align}
                    \| \phi(f) \|^2&=\frac{1}{ 2\pi }\int_a^b| f(y) |^2\frac{ 2\pi }{ b-a }dy\\
                    &=\frac{1}{ b-a }\int_a^b| f |^2\\
                    &=\| f \|^2.
                \end{align}
            \end{subequations}
        \item[Injectif]
            Soit \( f\) telle que \( \phi(f)=0\). Alors pour tout \( z\in S^1\) nous avons
            \begin{equation}
                f\big( (s^{-1}\circ\varphi^{-1})(z) \big)=0.
            \end{equation}
            Vu que \( s^{-1}\circ\varphi^{-1}\colon S^1 \to \mathopen[ a , b \mathclose[\) est une bijection, pour tout \( u\in\mathopen[ a , b \mathclose[\) nous avons \( f(u)=0\). Donc \( f=0\) dans \( L^2\big( \mathopen[ a , b \mathclose] \big)\) parce que du point de vue de \( L^2\), que l'on prenne ou non les bornes, ce n'est pas important.
        \item[Surjectif]
            Si \( g\in L^2(S^1)\), alors en posant
            \begin{equation}
                f(u)=g\big( (\varphi\circ s)(u) \big)
            \end{equation}
            nous avons \( g=\phi(f)\).
    \end{subproof}
\end{proof}

\begin{definition}
    En ce qui concerne le produit de convolution, si \( f\) et \( g\) sont des fonctions sur \( \mathopen[ a , b \mathclose]\) nous définissons
    \begin{equation}
        f*g=\phi^{-1}\big( \phi(f)*\phi(g) \big)
    \end{equation}
    tant que les formules ont un sens.
\end{definition}

%+++++++++++++++++++++++++++++++++++++++++++++++++++++++++++++++++++++++++++++++++++++++++++++++++++++++++++++++++++++++++++ 
\section{Sur \( \mathopen[ 0 , 2\pi \mathclose[\)}
%+++++++++++++++++++++++++++++++++++++++++++++++++++++++++++++++++++++++++++++++++++++++++++++++++++++++++++++++++++++++++++

Le produit de convolution est un peut subtil parce que \( f(t-x)\) n'est pas défini a priori pour tout \( t,x\in \mathopen[ 0 , 2\pi \mathclose[\), vu que \( f\) n'est définie que sur \( \mathopen[ 0 , 2\pi \mathclose[\). Au moins trois solutions s'offrent à nous :
\begin{itemize}
    \item 
       considérer implicitement la fonction prolongée par périodicité.
   \item
       considérer les fonctions sur \( \eR/2\pi\), et définir un peut toutes les opérations modulo \( 2\pi\) (fastidieux)
   \item
       utiliser une bijection ayant les bonnes propriétés avec \( S^1\) sur lequel tout est déjà fait.
\end{itemize}
Nous sélectionnons la troisième voie. Pour cela nous considérons la fonction (attention, elle n'est pas tout à fait la même que celle plus haut)
\begin{equation}
    \begin{aligned}
        \varphi\colon \mathopen[ 0 , 2\pi \mathclose[&\to S^1 \\
            t&\mapsto  e^{it} 
    \end{aligned}
\end{equation}
qui est une bijection par la proposition \ref{PROPooZEFEooEKMOPT}. Pour le produit de convolution,
\begin{equation}
    (f * g)(x)=(f\circ \varphi^{-1})*(g\circ\varphi^{-1})\big( \varphi(x) \big)
\end{equation}
pour toutes les fonctions \( f,g\colon \mathopen[ 0 , 2\pi \mathclose[\to \eC\) pour lesquelles les intégrales en jeu ont un sens.

%+++++++++++++++++++++++++++++++++++++++++++++++++++++++++++++++++++++++++++++++++++++++++++++++++++++++++++++++++++++++++++ 
\section{Sur \( \mathopen[ -T , T \mathclose[\)}
%+++++++++++++++++++++++++++++++++++++++++++++++++++++++++++++++++++++++++++++++++++++++++++++++++++++++++++++++++++++++++++

Pour rappel, les éléments de \( L^2\) sont des classes de fonctions à valeurs dans \( \eC\).

\begin{proposition}     \label{PROPooHNJZooGfRCfU}
    Les fonctions    
    \begin{equation}
        \begin{aligned}
            e_n\colon \mathopen[ -T , T \mathclose]&\to \eC \\
            t&\mapsto \frac{1}{ \sqrt{ 2T } } e^{2\pi int/T}. 
        \end{aligned}
    \end{equation}
    forment une base hilbertienne de \( L^2\big( \mathopen[ -T , T \mathclose[ \big)\).
\end{proposition}

\begin{proof}
    Nous utilisons le théorème de Stone-Weierstrass \ref{ThoWmAzSMF}. Pour cela, nous considérons \( X=\mathopen[ -T , T \mathclose]\) qui est compact et Hausdorff ainsi que \( A\), l'ensemble des polynômes trigonométrique :
    \begin{equation}
        A=\{ \sum_{k=-n}^na_ke_k\tq a_k\in \eC \}_{n\in \eN}.
    \end{equation}
    Nous vérifions que ce \( A\) satisfait aux hypothèses de Stone-Weierstrass \ref{ThoWmAzSMF}.
    \begin{subproof}
        \item[\( A\) est une algèbre]
            Il s'agit seulement de vérifier que
            \begin{equation}
                e_ke_l=\frac{1}{ \sqrt{ 2T } }e_{k+l}.
            \end{equation}
        \item[\( A\) sépare les points]
            Soient \( x,y\in \mathopen[ -T , T \mathclose]\). Si \( e_k(x)=e_k(y)\)
    \end{subproof}
\end{proof}

%--------------------------------------------------------------------------------------------------------------------------- 
\subsection{Le cas dans \( \mathopen[ 0 , 2\pi \mathclose]\)}
%---------------------------------------------------------------------------------------------------------------------------

En pratique, nous n'allons pas souvent travailler avec des fonctions sur intervalle symétrique \( \mathopen[ -T , T \mathclose]\), mais le plus souvent nous serons sur \( \mathopen[ 0 , 2\pi \mathclose]\).

Nous notons ici une conséquence du théorème~\ref{ThoGVmqOro} dans le cas de l'espace \( L^2\). La proposition suivante est une petite partie du corollaire~\ref{CorQETwUdF}, qui sera d'ailleurs démontré de façon indépendante.

\begin{proposition}
    Si nous avons une suite de réels \( (a_k)\) telle que \( \sum_{k=0}^{\infty}| a_k |^2<\infty\) alors la suite
    \begin{equation}
        f_n(x)=\sum_{k=0}^na_k e^{ikx}
    \end{equation}
    converge dans \( L^2\big( \mathopen] 0 , 2\pi \mathclose[ \big)\).
\end{proposition}

\begin{proof}
    Quitte à séparer les parties réelles et imaginaires, nous pouvons faire abstraction du fait que nous parlons d'une série de fonctions à valeurs dans \( \eC\) au lieu de \( \eR\).

    Un simple calcul est :
    \begin{equation}    \label{EqHVdJxZT}
        \| f_n-f_m \|^2\leq\int_0^{2\pi}\sum_{k=n}^m| a_k |^2dx\leq 2\pi\sum_{k=n}^m| a_k |^2.
    \end{equation}
    Par hypothèse le membre de droite est \( | s_m-s_n |\) où \( s_k\) dénote la suite des sommes partielles de la série des \( | a_k |^2\). Cette dernière est de Cauchy (parce que convergente dans \( \eR\)) et donc la limite \( n\to\infty\) (en gardant \( m>n\)) est zéro. Donc la suite des \( f_n\) est de Cauchy dans \( L^2\) et donc converge dans \( L^2\).
\end{proof}

Adaptons tout cela pour l'espace \( L^2\big( \mathopen[ 0 , 2\pi \mathclose] \big)\). Nous posons
\begin{equation}        \label{EQooBFKDooMkCZOt}
    \langle f, g\rangle =\int_0^{2\pi}f(t)\overline{ g(t) }dt
\end{equation}
et
\begin{equation}        \label{EQooKMYOooLZCNap}
    e_n(t)=\frac{1}{ \sqrt{ 2\pi } } e^{int}.
\end{equation}


L'importance du système trigonométrique défini en \ref{DEFooGCZAooFecAHB} est d'être une base de \( L^2\big( \mathopen[ 0 , 2\pi \mathclose] \big)\), comme précisé dans le lemme suivant.
\begin{lemma}       \label{LEMooBJDQooLVPczR}
    Le système trigonométrique \( \{ e_n \}_{n\in \eZ}\) est une base hilbertienne de \( L^2\big( \mathopen[ 0 , 2\pi \mathclose] \big)\).
\end{lemma}

\begin{proof}
    Pour rappel, une base hilbertienne est la définition~\ref{DEFooADQXooFoIhTG}. Nous prouvons d'abord que le système est orthogonal. Nous avons
    \begin{equation}
        \langle e_n, e_m\rangle =\frac{1}{ 2\pi }\int_0^{2\pi} e^{i(n-m)t}dt.
    \end{equation}
    Si \( n=m\), alors cela est égal à \( 1\). Sinon c'est une intégrale simple :
    \begin{equation}
        \langle e_n, e_m\rangle =\left[ \frac{ i }{ n-m } e^{i(n-m)t} \right]_0^{2\pi}=0.
    \end{equation}
    Cela est pour l'orthogonalité.

    Pour que le système soit total, il faut que son espace vectoriel engendré soit dense. Cela est le théorème~\ref{ThoQGPSSJq}.
\end{proof}

Note : le théorème~\ref{ThoDPTwimI} donné aussi la densité, mais sera démontré plus tard, indépendamment. Voir aussi les thèmes~\ref{THEooPUIIooLDPUuq} et~\ref{THEMooNMYKooVVeGTU}.

Pour un élément donné \( f\in L^2\big( \mathopen[ 0 , 2\pi \mathclose] \big)\), nous définissons\nomenclature[Y]{\( S_nf\)}{somme partielle de série de Fourier}
\begin{equation}
    S_nf=\sum_{k=-n}^n\langle f, e_k\rangle e_k
\end{equation}
et nous avons le théorème suivant, qui récompense les efforts consentis à propos de la densité des polynômes trigonométriques dans \( L^2\).

\begin{theorem} \label{ThoYDKZLyv}
    Soit \( f\in L^2\big( \mathopen[ 0 , 2\pi \mathclose] \big)\). Nous avons égalité\footnote{Notons que la somme sur \( \eZ\) dans \eqref{EqXMMRpSN} est commutative; il n'est donc pas besoin d'être plus précis.}
    \begin{equation}    \label{EqXMMRpSN}
        f=\sum_{n\in \eZ}c_n(f)e_n
    \end{equation}
    dans \( L^2\).

    Nous avons aussi la convergence
\begin{equation}    \label{EqRBWKsYP}
    S_nf\stackrel{L^2}{\to} f.
\end{equation}
\end{theorem}

\begin{proof}
    Le système trigonométrique \( \{ e_n \}_{n\in \eZ}\) est total pour l'espace de Hilbert \( L^2\big( \mathopen[ 0 , 2\pi \mathclose] \big)\) (sans périodicité particulière). Donc le point~\ref{ItemQGwoIxi} du théorème~\ref{ThoyAjoqP} nous donne l'égalité demandée.

    La convergence \eqref{EqRBWKsYP} est une reformulation de l'égalité \eqref{EqXMMRpSN}.
\end{proof}

\begin{normaltext}
    Obtenir la convergence \( L^2\) ne demande pas d'hypothèses de périodicité : la convergence \eqref{EqRBWKsYP} est automatique du fait que le système trigonométrique soit total. Ce n'est cependant pas plus qu'une convergence \( L^2\) et elle ne demande pas \( f(0)=f(2\pi)\), même si pour chacun des \( e_k\) nous avons \( e_k(0)=e_k(2\pi)\).

    Si \( f(2\pi)\neq f(0)\), alors il existe tout de même une suite \( (f_n)\) convergente vers \( f\) au sens \( L^2\) telle que \( f_n(0)=f_n(2\pi)\). Cela ne contredit en rien le fait que \( e_k(0)=e_k(2\pi)\) parce que dans \( L^2\), la valeur d'un point seul n'a pas d'importance.

    Si nous voulons une vraie convergence ponctuelle voir uniforme \( (S_nf)(x)\to f(x)\), alors il faut ajouter des hypothèses sur la continuité de \( f\), sa périodicité ou le comportement des coefficients \( c_n\). Voir aussi le thème~\ref{THMooHWEBooTMInve}.
\end{normaltext}

\begin{example}     \label{EXooQDWUooLtuIOm}
    Si \( f\in L^2\big( \mathopen[ 0 , 2\pi \mathclose] \big)\) est (la classe de) une fonction à valeurs réelles, alors on peut la développer avec nettement moins de termes. D'abord nous savons que \( e_{-n}=\overline{ e_n }\), et donc
    \begin{equation}
        \langle f, e_n\rangle =\overline{ \langle f, e_{-n}\rangle  },
    \end{equation}
    ce qui donne
    \begin{equation}
        f=\sum_{n\in\eZ}\langle f, e_n\rangle e_n=\sum_{n\in \eN}\langle f, e_n\rangle e_n +\overline{ \langle f, e_n\rangle e_n }=\sum_{n\in \eN}\Re\big( \langle f, e_n\rangle e_n \big).
    \end{equation}
    Or
    \begin{equation}        \label{EQooMWJNooSjPCpR}
        \Re\big( \langle f, e_n\rangle e_n \big)=\frac{1}{ (2\pi)^{3/2} }\cos(nx)\int_0^{2\pi}f(t)\cos(nt)dt-\frac{1}{ (2\pi)^{3/2} }\sin(nx)\int_0^{2\pi}f(t)\sin(nt)dt.
    \end{equation}

    Considérons la fonction impaire \( \tilde f\in\L^2\big( [-2\pi,2\pi] \big)\) créée à partir de \( f\). Elle se développe de même et nous avons la même formule \eqref{EQooMWJNooSjPCpR} à part quelques coefficients et le fait que les intégrales sont entre \( -2\pi\) et \( 2\pi\). Vu que \( \tilde f\) est impaire, l'intégrale avec \( \cos(nt)\) s'annule et
    \begin{equation}
        \tilde f(x)=\sum_{n\in \eN}c_n\sin(nx)
    \end{equation}
    pour certains coefficients réels \( c_n\). Cette égalité est à considérer dans \( L^2\), c'est-à-dire presque partout et en particulier presque partout sur \( \mathopen[ 0 , 2\pi \mathclose]\).

    Donc les fonctions réelles sur \( \mathopen[ 0 , 2\pi \mathclose]\) peuvent être écrites sous la forme d'une série de seulement des sinus.

    Note : en choisissant \( \tilde f\) paire, nous aurions eu une série de cosinus.
\end{example}

%+++++++++++++++++++++++++++++++++++++++++++++++++++++++++++++++++++++++++++++++++++++++++++++++++++++++++++++++++++++++++++ 
\section{Théorème de la projection}
%+++++++++++++++++++++++++++++++++++++++++++++++++++++++++++++++++++++++++++++++++++++++++++++++++++++++++++++++++++++++++++

Avant d'entrer dans le vif du sujet, nous nous fendons d'une petite étude de fonction. Soit
\begin{equation}
    \begin{aligned}
        \phi\colon \mathopen[ 0 , 1 \mathclose]&\to \eR \\
        x&\mapsto \frac{ (1+x)^r }{ 1+x^r }. 
    \end{aligned}
\end{equation}
Un peu de calcul montre que
\begin{equation}
    \frac{ \phi'(x) }{ \phi(x) }=\frac{ r(1-x^{r-1}) }{ (1+x^r)(1+x) }.
\end{equation}

\begin{lemma}       \label{LEMooFKKEooDTypUd}
    Soient \( a,b>0\) et \( r>1\). Nous avons les inégalités
    \begin{equation}
        a^r+b^r\leq (a+b)^r\leq 2^{r-1}(a^r+b^r).
    \end{equation}
\end{lemma}

\begin{proof}
    Pour la première inégalité, nous posons \( f(x)=a^r+x^r\) et \( g(x)=(a+x)^r\). Nous avons \( f(0)=g(0)=a^r\), et
    \begin{subequations}
        \begin{align}
            f'(x)&=rx^{r-1}\\
            g'(x)&=r(a+x)^{r-1}.
        \end{align}
    \end{subequations}
    Vu que \( r>1\), la fonction \( t\mapsto t^{r-1}\) est croissante par la proposition \ref{PROPooRXLNooWkPGsO}.

    Nous passons à la seconde inégalité. Le lemme \ref{LEMooSXTXooZOmtKq} nous dit que la fonction \( f\colon x\mapsto x^r \) est convexe. Donc
    \begin{equation}
        f\left( \frac{ a }{2}+\frac{ b }{2} \right)\leq\frac{ 1 }{2}f(a)+\frac{ 1 }{2}f(b).
    \end{equation}
    De là nous déduisons
    \begin{equation}
        \frac{ (a+b)^r }{ 2^r }\leq \frac{ 1 }{2}(a^r+b^r),
    \end{equation}
    c'est-à-dire la seconde inégalité.
\end{proof}

Nous allons démontrer les inégalités de Hanner dans le théorème \ref{THOooZRRYooBTBQKW}. Vu que ce sera un peu longuet, nous faisons un lemme.
\begin{lemma}       \label{LEMooDHRCooQiSpyC}
    Soient \( z_1,z_2\in \eC\). Nous avons
    \begin{equation}        \label{EQooMUXVooSpGSyG}
        | z_1+z_2 |^p+| z_1-z_2 |^p\geq \big( | z_1 |+| z_2 | \big)^p+\big| | z_1 |-| z_2 | \big|^p.
    \end{equation}
\end{lemma}

\begin{proof}
    Soient \( z_1,z_2\in \eC\). Nous posons
    \begin{equation}        \label{EQooJKYZooFzbETG}
        d=| z_1+z_2 |^p+| z_1-z_2 |^p.
    \end{equation}
    Pour \( | z_1 |\) et \( | z_2 |\) fixés, nous nous demandons quel est le minimum possible de \( d\).

    Si \( | z_1 |=0\), alors le minimum est \( 2| z_2 |^p\) et si \( | z_2 |=0\) alors il est \( 2| z_1 |^p\). Pour les autres cas, nous posons $| z_1 |=a>0$ ainsi que \( b\in \eR\) et \( \theta\in \eR\) tels que\footnote{Proposition \ref{PROPooRFMKooURhAQJ}}
    \begin{equation}
        z_2=z_1a^{-1}b e^{i\theta}.
    \end{equation}
    Nous avons déjà que \( z_1+z_2=z_1(1+a^{-1}b e^{i\theta})\) et donc
    \begin{equation}
        | z_1+z_2 |=a| 1+a^{-1}b e^{i\theta} |=| a+b e^{i\theta} |
    \end{equation}
    parce que \( a>0\). De plus,
    \begin{equation}
        | a+b e^{i\theta} |^2= (a+b e^{i\theta})(a+b e^{-i\theta})=a^2+b^2+2ab\cos(\theta)
    \end{equation}
    parce que \(  e^{i\theta}+ e^{-i\theta}=\cos(\theta)\). Nous posons
    \begin{equation}
        d(\theta)=| a+b e^{i\theta} |^p+| a-b e^{i\theta} |^p.
    \end{equation}
    En développant,
    \begin{equation}
        d(\theta)=\big(a^2+b^2+2ab\cos(\theta)\big)^{p/2}+\big(a^2+b^2-2ab\cos(\theta)\big)^{p/2}.
    \end{equation}
    Trouvons le minimum de cette fonction de \( \theta\). D'abord sa dérivée :
    \begin{equation}
        d'(\theta)=pab\sin(\theta)\big[ \big( a^2+b^2-2ab\cos(\theta) \big)^{p/2-1}-\big( a^2+b^2+2ab\cos(\theta) \big)^{p/2-1}  \big]=pab\sin(\theta)s(\theta).
    \end{equation}
    Nous avons \( s(\theta)=0\) pour \( \theta=\pi/2\) et \( \theta=3\pi/2\). Il faut surtout remarquer que \( 1<p<2\), ce qui donne \( \frac{ p }{2}-1<0\). La fonction \( x\mapsto x^{p/2-1}\) est donc décroissante. Cela pour dire que
    \begin{equation}
        s(0)=\left( | a-b |^2 \right)^{p/2-1}-\left( | a+b |^2 \right)^{p/2-1}>0.
    \end{equation}
    De la même façon, \( s(\pi)=-s(0)<0\). Cela permet d'écrire un petit tableau de signe de \( d'\), et de conclure que \( d(\theta)\) a un minimum en \( 0\) et en \( \pi\). Calcul fait, nous avons
    \begin{equation}
        d(0)=d(\pi)=| a+b |^p+| a-b |^p.
    \end{equation}
    En reliant à \eqref{EQooJKYZooFzbETG} nous avons l'inégalité
    \begin{equation}        \label{EQooVHQOooJcheCR}
        | z_1+z_2 |^p+| z_1-z_2 |^p\geq (a+b)^p-| a-b |^p.
    \end{equation}
    Nous rappelons que \( a=| z_1 |\) et que \( z_2=z_1a^{-1}b e^{i\theta}\). Notons au passage que \( | z_2 |=b\), donc que ce que nous dit l'équation \eqref{EQooVHQOooJcheCR} est que
    \begin{equation}    
        | z_1+z_2 |^p+| z_1-z_2 |^p\geq \big( | z_1 |+| z_2 | \big)^p+\big| | z_1 |-| z_2 | \big|^p.
    \end{equation}
\end{proof}

Encore dans la catégorie des lemmes pour les inégalités de Hanner, nous avons celui-ci.
\begin{lemma}[\cite{MonCerveau,ooKGWWooAybolH}]     \label{LEMooTCNEooADpNai}
    La fonction
    \begin{equation}
        \begin{aligned}
        \eta\colon \mathopen] 0 , \infty \mathclose[&\to \eR \\
            a&\mapsto (a^{1/p}+1)^p+| a^{1/p}-1 |^p 
        \end{aligned}
    \end{equation}
    est strictement convexe.
\end{lemma}

\begin{proof}
    La fonction \( \eta\) est une fonction de classe \(  C^{\infty}\) sur \( \mathopen] 0 , \infty \mathclose[\setminus\{ 1 \}\). Quelle est sa régularité en \( a=1\) ? Le fait qu'elle y soit dérivable pas clair à cause de la valeur absolue. En tout cas, la fonction \( x\mapsto| x-1 |\) n'est pas dérivable en \( x=1\), mais peut-être que les exposants aident à lisser. Nous y reviendrons.

    Afin de  suivre les calculs nous introduisons quelque fonctions :
    \begin{subequations}
        \begin{align}
            so(x)&=1+x^{1/p}\\
            di(x)&=1-x^{1/p}\\
            dj(x)&=x^{1/p}-1
        \end{align}
    \end{subequations}
    Pour les dérivées, nous avons
    \begin{subequations}
        \begin{align}
            so'(x)&=\frac{1}{ p }x^{1/p-1}\\
            di'(x)&=-so'(x)\\
            dj'(x)&=so'(x).
        \end{align}
    \end{subequations}
    En divise les cas selon \( a<1\) ou \( a>1\).
    \begin{subproof}
    \item[Pour \( a<1\)]
        Nous avons
        \begin{equation}
            \eta(a)=so(a)^p+di(a)^p,
        \end{equation}
        et la première dérivée donne :
        \begin{equation}        \label{EQooCLXZooXClOwd}
            \eta'(a)=p\,so'(a)\big( so(a)^{p-1}-di(a)^{p-1} \big).
        \end{equation}
        Pour la seconde dérivée nous trouvons d'abord
        \begin{equation}
            \begin{aligned}[]
            \eta''(a)&=\left( \frac{ 1-p }{ p } \right)a^{\frac{ 1 }{ p }-2}\big( so(a)^{p-1}-di(a)^{p-1} \big)\\
            &\quad+\frac{ p-1 }{ p }a^{\frac{ 2 }{ p }-2}\big( so(a)^{p-2}+di(a)^{p-2} \big).
            \end{aligned}
        \end{equation}
        À partir de là, le truc est de substituer les expressions suivantes :
        \begin{subequations}
            \begin{align}
                so(a)^{p-1}&=so(a)^{p-2}so(a)=so(a)^{p-2}+so(a)^{p-2}a^{1/p}\\
                di(a)^{p-1}&=di(a)^{p-2}-x^{1/p}di(a)^{p-2}. 
            \end{align}
        \end{subequations}
        Plein de trucs se simplifient et nous obtenons
        \begin{equation}
            \eta''(a)=\frac{ p-1 }{ p }a^{\frac{1}{ p }-2}\big( di(a)^{p-1}-so(a)^{p-2} \big).
        \end{equation}
    \item[Pour \( a>1\)]
            Les calculs sont essentiellement les mêmes, en partant de
            \begin{equation}
                \eta(a)=so(a)^p+dj(a)^p.
            \end{equation}
            Les résultats sont :
            \begin{equation}    \label{EQooAJLHooGWjPlz}
                \eta'(a)=p\,so'(a)\big( so(a)^{p-1}+dj(a)^{p-1} \big),
            \end{equation}
            et
            \begin{equation}
                \eta''(a)=\frac{ p-1 }{ p }a^{\frac{1}{ p }-2}\big( dj(a)^{p-2}-so(a)^{p-2} \big).
            \end{equation}
    \end{subproof}
    Au final, nous avons pour tout \( a\neq 1\) :
    \begin{equation}
        \eta''(a)=\frac{ p-1 }{ p }a^{\frac{1}{ p }-2}\big( | 1-a^{1/p} |^{p-2}-(1+a^{1/p})^{p-2} \big).
    \end{equation}
    Ce qu'il se passe en \( a=1\) est encore une question ouvert que nous traitons maintenant.
    \begin{subproof}
        \item[Pour \( a=1\)]
            Les limites des expressions \eqref{EQooCLXZooXClOwd} et \eqref{EQooAJLHooGWjPlz} en \( a=1\) sont vite calculées et c'est \( 2^{p-1}\) dans les deux cas. Donc la dérivée admet une prolongation continue en \( a=1\). Nous allons prouver que la fonction \( \eta\) est en réalité dérivable en \( a=1\) et que la dérivée vaut \( 2^{p-1}\).

            Nous nous concentrons sur la partie difficile donnée par \( f(x)=| x^{1/p}-1 |^p\). Elle est donnée par
            \begin{equation}
                f(x)=\begin{cases}
                    di(x)^p    &   \text{si } x<1\\
                    dj(x)^p    &    \text{si } x>1\\
                    0    &    \text{si } x=1.
                \end{cases}
            \end{equation}
            Si \( f'(1)\) existe, alors elle est égale à la limite
            \begin{equation}
                f'(1)=\lim_{\epsilon\to 0}\frac{ f(1)-f(1-\epsilon) }{ \epsilon }.
            \end{equation}
            Les deux limites à calculer sont :
            \begin{equation}
                \lim_{\epsilon\to 0^+}\frac{ \big( (1+\epsilon)^{1/p}-1 \big)^p }{ \epsilon }
            \end{equation}
            et
            \begin{equation}
                \lim_{\epsilon\to 0^-}\frac{ \big( 1-(1+\epsilon)^{1/p} \big)^p }{ \epsilon }.
            \end{equation}
            La première se traite par la règle de l'Hospital\footnote{Proposition \ref{PROPooBZHTooHmyGsy}}, et le résultat est zéro. Pour la seconde, il faut juste transformer
            \begin{equation}
                \lim_{\epsilon\to 0^+}\frac{ \big( (1+\epsilon)^{1/p}-1 \big)^p }{ \epsilon }=\lim_{h\to 0^+} \frac{ \big( 1-(1-h)^{1/p} \big)^p }{ -h },
            \end{equation}
            qui se traite également par la règle de l'Hospital. Le résultat est également zéro.

            Donc \( \eta\) est dérivable en \( a=1\) et la dérivée vaut \(\eta'(1)= 2^{p-1}\).
    \end{subproof}
En récapitulant, nous avons \( \eta''>0\) sur $\mathopen] 0  , \infty \mathclose[\setminus\{ 0 \}$, donc \( \eta'\) est croissante sur cette partie (proposition \ref{PropGFkZMwD}). Vu que \( \eta'\) est continue sur \( \mathopen] 0 , \infty \mathclose[\), elle est même croissante (strictement) sur tout \( \mathopen] 0 , \infty \mathclose[\).

La proposition \ref{PropYKwTDPX} conclu que \( \eta\) est strictement convexe sur \( \mathopen] 0 , \infty \mathclose[\).
\end{proof}

Toujours dans la catégorie des lemmes pour les inégalités de Hanner, nous avons celui-ci.
\begin{lemma}[\cite{ooKGWWooAybolH}]
    Soit \( 1<p<2\). La fonction
    \begin{equation}
        \begin{aligned}
            \xi\colon \eR^+\times \eR^+&\to \eR \\
            (a,b)&\mapsto \big( a^{1/p}+b^{1/p} \big)^p+| a^{1/p}-b^{1/p} |^p
        \end{aligned}
    \end{equation}
    est convexe.
    
    Pour rappel, les conventions de données en \ref{REMooOCXLooKQrDoq} donnent \( \eR^+=\mathopen[ 0 , \infty \mathclose[\).
\end{lemma}

\begin{proof}
    La fonction \( \xi\) vérifie facilement les conditions suivante :
    \begin{itemize}
        \item \( \xi(a,b)=\xi(b,a)\),
        \item \( \xi(0,0)=0\),
        \item \( \xi(ta,tb)=t\xi(a,b)\) pour tout \( t\geq 0\).
    \end{itemize}
    Nous posons
    \begin{equation}
        \begin{aligned}
            \eta\colon \eR^+&\to \eR \\
            a&\mapsto \xi(a,1).
        \end{aligned}
    \end{equation}
    Le lemme \ref{LEMooTCNEooADpNai} dit que \( \eta\) est strictement convexe, et le lemme \ref{LEMooNUDOooVfVPkw} conclu que \( \xi\) est convexe.
\end{proof}

Et enfin, les inégalités de Hanner\quext{Dont la démonstration n'est pas terminée.}.
\begin{theorem}[Inégalités de Hanner\cite{ooKGWWooAybolH}]       \label{THOooZRRYooBTBQKW}
    Soit \( 1<p<2\). Nous avons :
    \begin{equation}
        \big( \| f \|_p+\| g \|_p \big)^p+\Big| \| f \|_p+\| g \|_p \Big|^p\leq \| f+g \|_p^p+\| f-g \|_p^p\leq 2\| f \|_p^p+2\| g \|_p^p.
    \end{equation}
    Il y a égalité si et seulement si \( f(t) \) et \( g(t)\) sont colinéaires pour presque tout \( t\).
\end{theorem}

\begin{proof}
    Nous supposons fait le cas de \( f,g\colon \Omega\to \eR^+\). Nous passons au cas de fonctions à valeurs dans \( \eC\).

    Considérons \( f,g\in L^p\). Nous posons
    \begin{subequations}
        \begin{align}
            f^*(t)=| f(t) |\\
            g^*(t)=| g(t) |,
        \end{align}
    \end{subequations}
    et nous écrivons l'inéquation du lemme \ref{LEMooDHRCooQiSpyC} pour les nombres \( f(t)\) et \( g(t)\) :
    \begin{equation}    \label{EQooICUMooPhECbN}
        \big| f(t)+g(t) \big|^p+\big| f(t)-g(t) \big|^p\geq \big(  f^*(t)+g^*(t) \big)+| f^*(t)-g^*(t) |,
    \end{equation}
    avec égalité si et seulement si il existe \( \lambda_1,\lambda_2\in \eR\) tels que \( \lambda_1f(t)+\lambda_2g(t)=0\).

    Nous intégrons sur \( \Omega\) (la variable d'intégration est \( t\in \Omega\)) :
    \begin{equation}        \label{EQooNFIAooQjjuOn}
        \| f+g \|^p+\| f-g \|^p\geq \| f^*+g^* \|^p+\| f^*-g^* \|^p.
    \end{equation}
    Vu que nous avons une inégalité pour tout \( t\), la seule manière d'avoir une égalité dans \eqref{EQooNFIAooQjjuOn} est d'avoir une égalité dans \eqref{EQooICUMooPhECbN} pour presque tout \( t\) dans \( \Omega\).
    
    Nous sommes maintenant en mesure de prouver la première inégalité. Nous avons :
    \begin{subequations}
        \begin{align}
            \big( \| f \|+\| g \| \big)^p+\big| \| f \|-\| g \| \big|^p
            &=\big( \| f^* \|+\| g^* \| \big)^p+\big| \| f^* \|-\| g^* \| \big|^p    \label{SUBEQooTYCRooPhXwLT}\\
            &\leq \| f^*+g^* \|^p+\| f^*-g^* \|^p       \label{SUBEQooXIHFooKJukkk}\\
            &\leq \| f+g \|^p+\| f-g \|^p   \label{SUBEQooGKIVooBNxNah}
        \end{align}
    \end{subequations}
    Justifications :
    \begin{itemize}
        \item Pour \ref{SUBEQooTYCRooPhXwLT} : \( \| f \|=\| f^* \|\) et \( \| g \|=\| g^* \|\).
        \item Pour \ref{SUBEQooXIHFooKJukkk} : la propriété est supposée prouvée pour les fonctions à valeurs dans \( \eR^+\).
        \item Pour \ref{SUBEQooGKIVooBNxNah} : la relation \eqref{EQooNFIAooQjjuOn}
    \end{itemize}

\end{proof}

\begin{theorem}[Théorème de la projection] \label{THOooRJFUooQivDKm}
    Nous considérons \( p>1\). Soit un sous-espace vectoriel fermé \( W\subset L^p(\Omega,\tribA,\mu)\) et \( u_0\in L^p\). Nous notons
    \begin{equation}
        d(u_0,W)=\inf_{w\in W}d(u_0,W).
    \end{equation}
    Alors il existe \( w_0\in W\) tel que \( \| u_0-w_0 \|=d(u_0,W)\).
\end{theorem}

\begin{proof}
    Nous allons séparer trois cas : \( p=2\), \( 1<p<2\) et \( p>2\). 
    \begin{subproof}
        \item[\( p=2\)]
            Pour \( p=2\), nous savons que \( L^2\) est un espace de Hilbert\footnote{Lemme \ref{LemIVWooZyWodb}.}, et nous avons déjà le théorème de la projection \ref{ThoProjOrthuzcYkz}.
        \item[\( p>2\)]
            Pour chaque \( x\in \Omega\) nous avons \( f(x), g(x)\in \eC\) et donc l'identité du parallélogramme\footnote{Proposition \ref{PropEQRooQXazLz} en remarquant que $(z_1,z_2)\mapsto z_1\bar z_2$ est un produit scalaire hermitien sur $\eC$.} :
            \begin{equation}        \label{EQooUBFEooDUjLnb}
                \big| f(x)-g(x) \big|^2+\big| f(x)+g(x) \big|=2| f(x) |+2| g(x) |^2.
            \end{equation}
            Vu que \( p>2\), la fonction \( s\colon x\mapsto  x^{p/2}\) est convexe (lemme \ref{LEMooSXTXooZOmtKq}). Calcul :
            \begin{subequations}
                \begin{align}
                    | f(x)-g(x) |^p+| f(x)+g(x) |^p&=\big( | f(x)-g(x) |^2 \big)^{p/2}+\big( | f(x)+g(x) |^2 \big)^{p/2}\\
                    &=s\big( | \ldots |^2 \big)+s\big( | \ldots |^2 \big)\\
                    &\leq \big( | f(x)-g(x) |^2+| f(x)+g(x) |^2 \big)^{p/2}     \label{SUBEQooRHAEooHkYNLH}\\
                    &=\big( 2| f(x) |^2+2| g(x) |^2 \big)^{p/2}                 \label{SUBEQooQFSLooJkoeqN}\\
                    &=2^{p/2}\big( | f(x) |^2+| g(x) |^2 \big)^{p/2}\\
                    &\leq  2^{p/2}2^{p/2-1}\big( | f(x) |^p+| g(x) |^p \big)     \label{SUBEQooQSUHooXKaWwO}\\
                    &=2^{p-1}\big( | f(x) |^p+| g(x) |^p \big)    
                \end{align}
            \end{subequations} 
            Justifications : 
            \begin{itemize}
                \item Pour \eqref{SUBEQooRHAEooHkYNLH} : la convexité de \( s\).
                \item Pour \eqref{SUBEQooQFSLooJkoeqN} : la relation \eqref{EQooUBFEooDUjLnb}.
                \item Pour \eqref{SUBEQooQSUHooXKaWwO} : la seconde inégalité du lemme \ref{SUBEQooQSUHooXKaWwO}.
            \end{itemize}
            Nous isolons \( | f(x)-g(x) |^p\) :
            \begin{subequations}
                \begin{align}
                    | f(x)-g(x) |^p&\leq 2^{p-1}\big( | f(x) |^p+| g(x) |^p \big)-| f(x)+g(x) |^p\\
                    &=2^p\left( \frac{ | f(x) |^p+| g(x) |^p }{2}-\left| \frac{ | f(x) |+| g(x) | }{2} \right|^p \right)
                \end{align}
            \end{subequations}
            Cette inégalité étant valable pour tout \( x\), nous pouvons intégrer sur \( \Omega\) et découper l'intégrale en petits morceaux :
            \begin{equation}        \label{EQooVNHSooPXjFNC}
                \| f-g \|^p_p\leq 2^p\left( \frac{ \| f \|_p^p+\| g \|_p^p }{2}- \| \frac{ f+g }{2} \|_p^p \right).
            \end{equation}
            Voila une bonne chose de prouvée. Nous pouvons maintenant passer au vif du sujet.

            Soit une suite \( w_j\) dans \( W\) telle que \( \| u_0-w_j \|\to d(u_0,W)\). Trois choses à savoir sur cette suite :
            \begin{enumerate}
                \item
                    Une telle suite existe parce que \( d(u_0,W)\) est défini comme un infimum.
                \item
                    Rien ne garanti qu'elle converge.
                \item
                    Même si elle convergeait, rien ne garantirait que la limite soit encore dans \( W\).
            \end{enumerate}
            Le troisième point est facile à régler : vu que \( W\) est fermé par hypothèse, une suite convergente contenue dans \( W\) a sa limite dans \( W\). Nous allons régler la convergence de \( w_j\) en prouvant qu'elle est de Cauchy.
            
            Remarquons que \( W\) est vectoriel, donc \( (w_j+w_k)/2\) est dans \( W\) pour tout \( j\) et \( k\); donc
            \begin{equation}
                \| \frac{ w_j+w_k }{2}-u_0 \|\geq d(u_0,W).
            \end{equation}
            En tenant compte de cela, nous écrivons l'inégalité \eqref{EQooVNHSooPXjFNC} avec \( f=w_j-u_0\) et \( g=w_k-u_0\) :
            \begin{equation}
                \| f-g \|_p^p=\| w_j-w_k \|_p^p\leq 2^p\left( \frac{ \| w_j-u_0 \|^p+\| w_k-u_0 \|^p }{2}-d(u_0,W) \right).
            \end{equation}
            Soit \( \epsilon>0\) et \( 0<\epsilon_1,\epsilon_2<\epsilon\) tels que \( \epsilon_1+\epsilon_2<\epsilon\). Il existe un \( N\) tel que si \( j,k>N\) alors \( \| w_j-u_0 \|^p\leq d(u_0,W)^p+\epsilon_1\) et \( \| w_k-u_0 \|^p\leq d(u_0,W)^p+\epsilon_2\). Pour de telles valeurs de \( j\) et \( k\), nous avons
            \begin{equation}
                \| w_j-w_k \|_p\leq 2\left( \frac{ \epsilon_1+\epsilon_2 }{2} \right)<2\epsilon^{1/p}.
            \end{equation}
            Donc la suite \( (w_j)\) est de Cauchy.

            L'espace \( L^p\) étant complet par le théorème \ref{ThoUYBDWQX}, nous en déduisons que \( (w_j)\) converge dans \( L^p\). Mais comme \( W\) est fermé, nous avons \( w_j\stackrel{L^p}{\longrightarrow}w\in W\).

            En termes de normes, nous avons
            \begin{equation}
                \| w-u_0 \|=\lim_j\| w_j-u_0 \|=d(W,u_0).
            \end{equation}
            Nous avons fini pour \( 1<p<2\).
    \end{subproof}
    <++>
\end{proof}
<++>

%+++++++++++++++++++++++++++++++++++++++++++++++++++++++++++++++++++++++++++++++++++++++++++++++++++++++++++++++++++++++++++
\section{Dualité et théorème de représentation de Riesz}
%+++++++++++++++++++++++++++++++++++++++++++++++++++++++++++++++++++++++++++++++++++++++++++++++++++++++++++++++++++++++++++

Dans la suite \( E'\) est le dual topologique, c'est-à-dire l'espace des formes linéaires et continues sur \( E\).

\begin{proposition}[\cite{PAXrsMn}, thème~\ref{THEMEooULGFooPscFJC}] \label{PropOAVooYZSodR}
    Soit \( 1<p<2\) et \( q\) tel que \( \frac{1}{ p }+\frac{1}{ q }=1\). L'application
    \begin{equation}
        \begin{aligned}
            \Phi\colon L^q\big( \mathopen[ 0 , 1 \mathclose] \big)&\to  L^p\big( \mathopen[ 0 , 1 \mathclose] \big)'  \\
            \Phi_g(f)&= \int_{\mathopen[ 0 , 1 \mathclose]}f\bar g.
        \end{aligned}
    \end{equation}
    est une isométrie linéaire surjective.
\end{proposition}

\begin{proof}
    Pour la simplicité des notations nous allons noter \( L^2\) pour \( L^2\big( \mathopen[ 0 , 1 \mathclose] \big)\), et pareillement pour \( L^p\).
    \begin{subproof}
        \item[\( \Phi_g\) est un élément de \( (L^p)'\)]

            Si \( f\in L^p\) et \( g\in L^q\) nous devons prouver que \( \Phi_q(f)\) est bien définie. Pour cela nous utilisons l'inégalité de Hölder\footnote{Proposition~\ref{ProptYqspT}.} qui dit que \( fg\in L^1\); par conséquent la fonction \( f\bar g\) est également dans \( L^1\) et nous avons
            \begin{equation}
                | \Phi_g(f) |\leq\int_{\mathopen[ 0 , 1 \mathclose]}| f\bar g |=\| fg \|_1\leq \| f \|_p\| g \|_q.
            \end{equation}
            En ce qui concerne la norme de l'application \( \Phi_g\) nous avons tout de suite
            \begin{equation}
                \| \Phi_g \|=\sup_{\| f\|_p=1}\big| \Phi_g(f) \big|\leq \| g \|_q.
            \end{equation}
            Cela signifie que l'application \( \Phi_g\) est bornée et donc continue par la proposition~\ref{PROPooQZYVooYJVlBd}. Nous avons donc bien \( \Phi_g\in (L^p)'\).

        \item[Isométrie]

            Afin de prouver que \( \| \Phi_g \|=\| g \|_q\) nous allons trouver une fonction \( f\in L^p\) telle que \( \frac{ | \Phi_g(f) | }{ \| f \|_p }=\| g \|_q\).  De cette façon nous aurons prouvé que \( | \Phi_g |\geq \| g \|_q\), ce qui conclurait que \( | \Phi_g |=\| g \|_q\).

            Nous posons \( f=g| g |^{q-2}\), de telle sorte que \( | f |=| g |^{q-1}\) et
            \begin{equation}
                \| f \|_p=\left( \int| g |^{p(q-1)} \right)^{1/p}=\left( \int | g |^q \right)^{1/p}=\| g \|_q^{q/p}
            \end{equation}
            où nous avons utilisé le fait que \( p(q-1)=q\). La fonction \( f\) est donc bien dans \( L^p\). D'autre part,
            \begin{equation}
                \Phi_g(f)=\int f\bar g=\int g| g |^{q-2}\bar g=\int | g |^q=\| g \|_q^q.
            \end{equation}
            Donc
            \begin{equation}
                \frac{ | \Phi_g(f) | }{ \| f \|_p }=\| g \|_q^{q-\frac{ q }{ p }}=\| g \|_q
            \end{equation}
            où nous avons encore utilisé le fait que \( q-\frac{ q }{ p }=\frac{ q(p-1) }{ p }=1\).

        \item[Surjectif]

            Soit \( \ell\in (L^p)'\); c'est une application \( \ell\colon L^p\to \eC\) dont nous pouvons prendre la restriction à \( L^2\) parce que la proposition~\ref{PropIRDooFSWORl} nous indique que \( L^2\subset L^p\). Nous nommons \( \phi\colon L^2\to \eC\) cette restriction.

            \begin{subproof}

                \item[\( \phi\in (L^2)'\)]

                    Nous devons montrer que \( \phi\) est continue pour la norme sur \( L^2\). Pour cela nous montrons que sa norme opérateur (subordonnée à la norme de \( L^2\) et non de \( L^p\)) est finie :
                    \begin{equation}
                        \sup_{f\in L^2}\frac{ | \phi(f) | }{ \| f \|_{2} }\leq \sup_{f\in L^2}\frac{ | \ell(f) | }{ \| f \|_p }<\infty.
                    \end{equation}
                    Nous avons utilisé l'inégalité de norme \( \| f \|_p\leq \| f \|_2\) de la proposition~\ref{PropIRDooFSWORl}\ref{ItemWSTooLcpOvXii}.

                \item[Utilisation du dual de \( L^2\)]

                    Étant donné que \( L^2\) est un espace de Hilbert (lemme~\ref{LemIVWooZyWodb}) et que \( \phi\in (L^2)'\), le théorème~\ref{ThoQgTovL} nous donne un élément \( g\in L^2\) tel que \( \phi(f)=\Phi_g(f)\) pour tout \( f\in L^2\).

                    Nous devons prouver que \( g\in L^q\) et que pour tout \( f\in L^p\) nous avons \( \ell(f)=\Phi_g(f)\).

                \item[\( g\in L^q\)]

                    Nous posons \( f_n=g| g |^{q-2}\mtu_{| g |<n}\). Nous avons d'une part
                    \begin{equation}    \label{EqEBUooOnlRHj}
                        \Phi_g(f_n)=\int_0^1f_n\bar g=\int_{| g |<n}| g |^q.
                    \end{equation}
                    Et d'autre part comme \( f_n\in L^2\) nous avons aussi \( \phi(f_n)=\Phi_g(f_n)\) et donc
                    \begin{subequations}
                        \begin{align}
                            0\leq \Phi(f_n)= \phi(f_n)&\leq \| \ell \|\| f_n \|_p\\
                            &=\| \ell \|\left( \int_{| g |<n}| g |^{(q-1)p} \right)^{1/p}\\
                            &=\| \ell \|\left( \int_{| g |<n}| g |^q \right)^{1/p}.
                        \end{align}
                    \end{subequations}
                    où nous avons à nouveau tenu compte du fait que \( p(q-1)=q\). En combinant avec \eqref{EqEBUooOnlRHj} nous trouvons
                    \begin{equation}
                        \int_{| g |<n}| g |^q\leq \| \ell \|\left( \int_{| g |<n}| g |^q \right)^{1/p},
                    \end{equation}
                    et donc
                    \begin{equation}
                        \left( \int_{| g |<n}| g |^{q} \right)^{1-\frac{1}{ p }}\leq \| \ell \|,
                    \end{equation}
                    c'est-à-dire
                    \begin{equation}
                        \Big( \int_{| g |<n}| g |^q \Big)^{1/q}\leq \| \ell \|.
                    \end{equation}

                    Si ce n'était pas encore fait nous nous fixons un représentant de la classe \( g\) (qui est dans \( L^2\)), et nous nommons également \( g\) ce représentant. Nous posons alors
                    \begin{equation}
                        g_n=| g |^q\mtu_{| g |<n}
                    \end{equation}
                    qui est une suite croissante de fonctions convergeant ponctuellement vers \( | g |^q\). Le théorème de Beppo-Levi~\ref{ThoRRDooFUvEAN} nous permet alors d'écrire
                    \begin{equation}
                        \lim_{n\to \infty} \int_{| q |<n}| g |^q=\int_{0}^1| g |^q.
                    \end{equation}
                    Mais comme pour chaque \( n\) nous avons \( \int_{| g |<n}| q |^q\leq \| \ell \|^q\), nous conservons l'inégalité à la limite et
                    \begin{equation}
                        \int_0^1| g |^q\leq \| \ell \|^q.
                    \end{equation}
                    Cela prouve que \( g\in L^p\).

                \item[\( \ell(f)=\Phi_g(f)\)]

                    Soit \( f\in L^p\). En vertu de la densité de \( L^2\) dans \( L^p\) prouvée dans le corollaire~\ref{CorFZWooYNbtPz} nous pouvons considérer une suite \( (f_n)\) dans \( L^2\) telle que \( f_n\stackrel{L^p}{\longrightarrow}f\). Pour tout \( n\) nous avons
                    \begin{equation}
                        \ell(f_n)=\Phi_g(f_n).
                    \end{equation}
                    Mais \( \ell\) et \( \Phi_g\) étant continues sur \( L^p\) nous pouvons prendre la limite et obtenir
                    \begin{equation}
                        \ell(f)=\Phi_g(f).
                    \end{equation}
            \end{subproof}
        \end{subproof}
\end{proof}

\begin{lemma}[\cite{MathAgreg}] \label{LemHNPEooHtMOGY}
    Soit \( (\Omega,\tribA,\mu)\) un espace mesuré fini. Soit \( g\in L^1(\Omega)\) et \( S\) fermé dans \( \eC\). Si pour tout \( E\in \tribA\) nous avons
    \begin{equation}
        \frac{1}{ \mu(E) }\int_Egd\mu\in S,
    \end{equation}
    alors \( g(x)\in S\) pour presque tout \( x\in \Omega\).
\end{lemma}

\begin{proof}
    Soit \( D=\overline{ B(a,r) }\) un disque fermé dans le complémentaire de \( S\) (ce dernier étant fermé, le complémentaire est ouvert). Posons \( E=g^{-1}(D)\). Prouvons que \( \mu(E)=0\) parce que cela prouverait que \( g(x)\in D\) pour seulement un ensemble de mesure nulle. Mais \( S^c\) pouvant être écrit comme une union dénombrable de disques fermés\footnote{Tout ouvert peut être écrit comme union dénombrable d'éléments d'une base de topologie par la proposition~\ref{PropMMKBjgY} et $\eC$ a une base dénombrable de topologie par la proposition~\ref{PropNBSooraAFr}.}, nous aurions \( g(x)\in S^c\) presque nulle part.

    Vu que \( \frac{1}{ \mu(E) }\int_Ea=a\) nous avons
    \begin{subequations}
        \begin{align}
            \big| \frac{1}{ \mu(E) }gd\mu-a \big|=\big| \frac{1}{ \mu(E) }\int_E(g-a) \big|\leq  \frac{1}{ \mu(E) }\int_E| g-a |\leq\frac{1}{ \mu(E) }\mu(E)r=r.
        \end{align}
    \end{subequations}
    Donc
    \begin{equation}
        \frac{1}{ \mu(E) }\int_Egd\mu\in D,
    \end{equation}
    ce qui est une contradiction avec le fait que \( D\subset S^c\).
\end{proof}

Dans toute la partie d'analyse fonctionnelle, sauf mention du contraire, nous considérons dans \( L^p\) des fonctions à valeurs complexes, et donc les éléments du dual sont des applications linéaires continues à valeurs dans \( \eC\).

\begin{theorem}[Théorème de représentation de Riesz, thème~\ref{THEMEooULGFooPscFJC}, \cite{MathAgreg,TLRRooOjxpTp,LRBWftc,OYRmzAa}]  \label{ThoLPQPooPWBXuv}
    Soit \( 1\leq p<\infty\) et un espace mesuré \( (\Omega,\tribA,\mu)\). Soit \( q\) tel que \( \frac{1}{ p }+\frac{1}{ q }=1\) avec la convention que \( q=\infty\) si \( p=1\). Alors l'application
    \begin{equation}
        \begin{aligned}
            \Phi\colon L^q&\to (L^p)' \\
            \Phi_g(f)&=\int_{\Omega}f\bar gd\mu
        \end{aligned}
    \end{equation}
    est une bijection isométrique dans les cas suivants :
    \begin{enumerate}
        \item       \label{ITEMooSQQBooWSFBmX}
            si \( 1<p<\infty\) et \( (\Omega,\tribA,\mu)\) est un espace mesuré quelconque,
        \item       \label{ITEMooCQGJooOWzjoV}
            si \( p=1\) et \( (\Omega,\tribA, \mu)\) est \( \sigma\)-fini.
    \end{enumerate}
\end{theorem}
\index{dual!de $L^p$}

\begin{proof}
    Par petits bouts.
    \begin{subproof}
        \item[\( \Phi\) est injective]
        Nous commençons par prouver que \( \Phi\) est injectif. Soient \( g,g'\in L^q\) tels que \( \Phi_g=\Phi_{g'}\). Alors pour tout \( f\in L^p\) nous avons
                \begin{equation}
                    \int_{\Omega}f(g-g')d\mu=0.
                \end{equation}
                Soient des parties \( A_i\) de mesures finies telles que \( \Omega=\bigcup_{i=1}^{\infty}A_i\). Étant donné que \( \mu(A_i)\) est fini, nous avons \( \mtu_{A_i}\in L^p(\Omega)\) et donc
                \begin{equation}
                    \int_{A_i}(g-g')d\mu=\int_{\Omega}\mtu_{A_i}(x)(g-g')(x)d\mu(x)=0.
                \end{equation}
                La proposition~\ref{PropRERZooYcEchc} nous dit alors que \( g-g'=0\) dans \( L^q(A_i)\). Pour chaque \( i\), la partie \( N_i=\{ x\in A_i\tq (g-g')(x)=0 \}\) est de mesure nulle.

                Vu que \( \Omega\) est l'union de tous les \( A_i\), la partie de \( \Omega\) sur laquelle \( g-g'\) est non nulle est l'union des \( N_i\) et donc de mesure nulle parce que une réunion dénombrable de parties de mesure nulle est de mesure nulle. Donc \( g-g'=0\) presque partout dans \( \Omega\), ce qui signifie \( g-g'=0\) dans \( L^q(\Omega)\).

            \item[La suite]

        La partie difficile est de montrer que \( \Phi\) est surjective.

        Soit \( \phi\in L^p(\Omega)'\). Si \( \phi=0\), c'est bien dans l'image de \( \Phi\); nous supposons donc que non. Nous allons commencer par prouver qu'il existe une (classe de) fonction \( g\in L^1(\Omega)\) telle que \( \Phi_g(f)=\phi(f)\) pour tout \(f\in L^{\infty}(\Omega,\mu)\); nous montrerons ensuite que \( g\in L^q\) et que le tout est une isométrie.

        \item[Une mesure complexe]

            Si \( E\in\tribA\) nous notons \( \nu(E)=\phi(\mtu_E)\). Nous prouvons maintenant que \( \nu\) est une mesure complexe\footnote{Définition~\ref{DefGKHLooYjocEt}.} sur \( (\Omega,\tribA)\). La seule condition pas facile est la condition de dénombrable additive. Il est déjà facile de voir que \( A\) et \( B\) sont disjoints, \( \nu(A\cup B)=\nu(A)+\nu(B)\). Soient ensuite des ensembles \( A_n\) deux à deux disjoints et posons \( E_k=\bigcup_{i\leq k}A_i\) pour avoir \( \bigcup_kA_k=\bigcup_kE_k\) avec l'avantage que les \( E_k\) soient emboîtés. Cela donne
            \begin{equation}
                \| \mtu_E-\mtu_{E_k} \|_p=\mu(E\setminus E_k)^{1/p},
            \end{equation}
            mais vu que \( 1\leq p<\infty\), avoir \( x_k\to 0\) implique d'avoir \( x_k^{1/p}\to 0\). Prouvons que \( \mu(E\setminus E_k)\to 0\). En vertu du lemme~\ref{LemPMprYuC} nous avons pour chaque \( k\) :
            \begin{equation}
                \mu(E\setminus E_k)=\mu(E)-\mu(E_k),
            \end{equation}
            et vu que \( E_k\to E\) est une suite croissante, le lemme~\ref{LemAZGByEs}\ref{ItemJWUooRXNPci}, sachant que \( \mu\) est une mesure « normale », donne
            \begin{equation}
                \lim_{n\to \infty} \mu(E_k)=\mu\big( \bigcup_kE_k \big).
            \end{equation}
            Donc effectivement \( \mu(E_k)\to \mu(E)\) et donc oui : \( \mu(E\setminus E_k)\to 0\). Jusqu'à présent nous avons
            \begin{equation}
                \lim_{k\to \infty} \| \mtu_E-\mtu_{E_k} \|_p=0,
            \end{equation}
            c'est-à-dire \( \mtu_{E_k}\stackrel{L^p}{\longrightarrow}\mtu_E\). La continuité de \( \phi\) sur \( L^p\) donne alors
            \begin{equation}
                \lim_{k\to \infty} \nu(E_k)=\lim_{k\to \infty} \phi(\mtu_{E_k})=\phi(\lim_{k\to \infty} \mtu_{E_k})=\phi(\mtu_E)=\nu(E).
            \end{equation}
            Par additivité finie de \( \nu\) nous avons
            \begin{equation}
                \nu(E_k)=\sum_{i\leq k}\nu(A_i)
            \end{equation}
            et en passant à la limite, \( \sum_{i=1}^{\infty}\nu(A_i)=\nu(\bigcup_{i}A_i)\). L'application \( \nu\) est donc une mesure complexe.

        \item[Mesure absolument continue]

            En prime, si \( \mu(E)=0\) alors \( \nu(E)=0\) parce que
            \begin{equation}
                \mu(E)=0\Rightarrow \| \mtu_E \|_p=0\Rightarrow \mtu_E=0\text{ (dans } L^p\text{)}\Rightarrow\phi(\mtu_E)=0
            \end{equation}

        \item[Utilisation de Radon-Nikodym]

            Nous sommes donc dans un cas où \( \nu\ll\mu\) et nous utilisons le théorème de Radon-Nikodym~\ref{ThoZZMGooKhRYaO} sous la forme de la remarque~\ref{RemSYRMooZPBhbQ} : il existe une fonction intégrable \( g\colon \Omega\to \eC\)\footnote{On peut écrire, pour utiliser de la notation compacte que $ g\in L^1(\Omega,\eC)$.} telle que pour tout \( A\in\tribA\),
            \begin{equation}
                \nu(A)=\int_A\bar gd\mu.
            \end{equation}
            C'est-à-dire que
            \begin{equation}
                \phi(\mtu_A)=\int_A\bar gd\mu=\int_{\Omega}\bar g\mtu_Ad\mu.
            \end{equation}
            Nous avons donc exprimé \( \phi\) comme une intégrale pour les fonctions caractéristiques d'ensembles.

        \item[Pour les fonctions étagées]

            Par linéarité si \( f\) est mesurable et étagée nous avons aussi
            \begin{equation}
                \phi(f)=\int f\bar gd\mu=\Phi_g(f).
            \end{equation}

        \item[Pour \( f\in L^{\infty}(\Omega)\)]

            Une fonction \( f\in L^{\infty}\) est une fonction presque partout bornée. Nous supposons que \( f\) est presque partout bornée par \( M\). Par ailleurs cette \( f\) est limite uniforme de fonctions étagées : \( \| f_k-f \|_{\infty}\to 0\) en posant \( f_k=f\mtu_{| f |\leq k}\). Pour chaque \( k \) nous avons l'égalité
            \begin{equation}    \label{EqPDCJooGNjuAO}
                \Phi_g(f_k)=\phi(f_k).
            \end{equation}
            Par ailleurs la fonction \( f_k\bar g\) est majorée par la fonction intégrable \( M\bar g\) et le théorème de la convergence dominée~\ref{ThoConvDomLebVdhsTf} nous donne
            \begin{equation}
                \lim_{k\to \infty} \Phi_g(f_i)=\lim_{k\to \infty} \int f_k\bar g=\int f\bar g=\Phi_g(f).
            \end{equation}
            Et la continuité de \( \phi\) sur \( L^p\) couplée à la convergence \( f_k\stackrel{L^p}{\longrightarrow}f\) donne \( \lim_{k\to \infty} \phi(f_k)=(f)\). Bref prendre la limite dans \eqref{EqPDCJooGNjuAO} donne
            \begin{equation}
                \Phi_g(f)=\phi(f)
            \end{equation}
            pour tout \( f\in L^{\infty}(\Omega)\).

        \item[La suite \ldots]

            Voici les prochaines étapes.
            \begin{itemize}
                \item Nous avons \( \int f\bar g=\phi(f)\) tant que \( f\in L^{\infty}\). Nous allons étendre cette formule à \( f\in L^p\) par densité. Cela terminera de prouver que notre application est une bijection.
                \item Ensuite nous allons prouver que \( \| \phi \|=\| \Phi_g \|\), c'est-à-dire que la bijection est une isométrie.
            \end{itemize}

        \item[De \( L^{\infty}\) à \( L^p\)]

            Soit \( f\in L^p\). Si nous avions une suite \( (f_n) \) dans \( L^{\infty}\) telle que \( f_n\stackrel{L^p}{\longrightarrow}f\) alors \( \lim \phi(f_n)=\phi(f)\) par continuité de \( \phi\). La difficulté est de trouver une telle suite de façon à pouvoir permuter l'intégrale et la limite :
            \begin{equation}    \label{EqLYYAooUQnbfV}
                \lim_{n\to \infty} \int_{\Omega}f_n\bar g=\int_{\Omega}\lim_{n\to \infty} f_n\bar g=\int_{\Omega}f\bar g=\Phi_g(f).
            \end{equation}
            Nous allons donc maintenant nous atteler à la tâche de trouver \( f_n\in L^{\infty}\) avec \( f_n\stackrel{L^p}{\longrightarrow}f\) et telle que \eqref{EqLYYAooUQnbfV} soit valide.

            Nous allons d'abord supposer que \( f\in L^p\) est positive à valeurs réelles. Nous avons alors par le théorème \ref{THOooXHIVooKUddLi} qu'il existe une suite croissante de fonction étagées (et donc \( L^{\infty}\)) telles que \( f_n\to f\) ponctuellement. De plus étant donné que \( | f_n |\leq | f |\), la proposition~\ref{PropBVHXycL} nous dit que \( f_n\stackrel{L^p}{\longrightarrow}f\). Pour chaque \( n\) nous avons
            \begin{equation}
                \int_{\Omega}f_n\bar g=\phi(f_n).
            \end{equation}
            Soit \( g^+\) la partie réelle positive de \( \bar g\). Alors nous avons la limite croissante ponctuelle \( f_ng^+\to fg^+\) et le théorème de la convergence monotone~\ref{ThoRRDooFUvEAN} nous permet d'écrire
            \begin{equation}
                \lim_{n\to \infty} \int f_ng^+=\int fg^+.
            \end{equation}
            Faisant cela pour les trois autres parties de \( \bar g\) nous avons prouvé que si \( f\in L^p\) est réelle et positive,
            \begin{equation}
                \int f\bar g=\phi(f),
            \end{equation}
            c'est-à-dire que \( \Phi_g(f)=\phi(f)\).

            Refaisant le tout pour les trois autres parties de \( f\) nous montrons que
            \begin{equation}
                \Phi_g(f)=\phi(f)
            \end{equation}
            pour tout \( f\in L^p(\Omega)\). Nous avons donc égalité de \( \phi\) et \( \Phi_g\) dans \(  (L^p)' \) et donc bijection entre \( (L^p)'\) et \( L^q\).

        \item[Isométrie : mise en place]

            Nous devons prouver que cette bijection est isométrique. Soit \( \phi\in (L^p)'\) et \( g\in L^q\) telle que \( \Phi_g=\phi\). Il faut prouver que
            \begin{equation}
                \| g \|_q=\| \phi \|_{(L^p)'}.
            \end{equation}

        \item[ \( \| \phi \|\leq \| g \|_q\) ]

            Nous savons que \( \phi(f)=\int f\bar g\), et nous allons écrire la définition de la norme dans \( (L^p)'\) :
            \begin{subequations}
                \begin{align}
                    \| \phi \|_{(L^p)'}&=\sup_{\| f \|_p=1}\big| \phi(f) \big|\\
                    &=\sup| \int f\bar g |\\
                    &\leq\sup\underbrace{\int| f\bar g |}_{=\| f\bar g \|_1}.
                \end{align}
            \end{subequations}
            Il s'agit maintenant d'utiliser l'inégalité de Hölder~\ref{ProptYqspT} :
            \begin{equation}
                \| \phi \|\leq \sup_{\| f \|_p=1}\| f \|_p\| \bar g \|_q=\| g \|_q.
            \end{equation}

            L'inégalité dans l'autre sens sera démontrée en séparant les cas \( p=1\) et \( 1<p<\infty\).

        \item[Si \( p=1\), une formule]
            Si \( E\) est un ensemble mesurable de mesure finie, alors
            \begin{equation}
                | \int_Egd\mu |=\big| \phi(\mtu_E) \big|.
            \end{equation}
            Mais le fait que \( \mu(E)<\infty\) donne que \( \mtu_E\in L^1(\Omega)\). Donc \( \mtu_E\in L^{\infty}\cap L^1\); nous pouvons alors écrire \( \phi(\mtu_E)=\int_{\Omega}\mtu_E\bar gd\mu\) et donc
            \begin{equation}    \label{EqUPCTooJvoKKI}
                | \int_{\Omega}\mtu_E\bar gd\mu |=|\int_Egd\mu |=\big| \phi(\mtu_E) \big|\leq \| \phi \|_{(L^1)'}\| \mtu_E \|_1=\| \phi \|\mu(E).
            \end{equation}
            Nous écrivons cela dans l'autre sens :
            \begin{equation}
                \| \phi \|\geq \frac{1}{ \mu(E) }| \int_{\Omega}\mtu_E\bar gd\mu |=| \frac{1}{ \mu(E) }\int_E\bar gd\mu |.
            \end{equation}
            Si nous prenons \( S=\{ t\in \eC\tq | t |\leq \| \phi \| \}\), c'est un fermé vérifiant que
            \begin{equation}        \label{EQooMRLGooYPEjUo}
                \frac{1}{ \mu(E) }\int_E\bar gd\mu\in S.
            \end{equation}

            Voila une petite formule qui va nous aider à utiliser le lemme \ref{LemHNPEooHtMOGY}. Nous ne pouvons cependant pas l'utiliser immédiatement parce que l'appartenance \eqref{EQooMRLGooYPEjUo} n'est vraie que pour les parties de mesure finie.

        \item[Si \( p=1\), conclusion\cite{MonCerveau}]

            Pour utiliser le lemme~\ref{LemHNPEooHtMOGY}, nous utilisons l'hypothèse que \( \Omega\) est \( \sigma\)-fini. Soient des mesurables \( A_i\) de mesure fine tels que \( \bigcup_{i\in \eN}A_i=\Omega\).

            Pour chaque \( i\) nous considérons la restriction \( g_i\colon A_i\to \eC\) de \( g\) à \( A_i\). Par le point précédent, elle vérifie
            \begin{equation}
                \frac{1}{ \mu(A_i) }\int_{A_i}\bar g_id\mu=\frac{1}{ \mu(A_i) }\int_{A_i}\bar gd\mu\in S.
            \end{equation}
            En appliquant le lemme \ref{LemHNPEooHtMOGY} à l'espace restreint \( (A_i,\tribA_i,\mu_i)\), nous concluons \( \bar g_i\in S\) presque partout, ce qui signifie que \( \| g_i \|_{\infty}\in S\). Nous en concluons que
            \begin{equation}
                \| g_i \|_{\infty}\leq \| \phi \|
            \end{equation}
            où, dans ce contexte, \( \| g_i \|_{\infty}\) signifie \( \sup_{x\in A_i}| g_i(x) |\).
            
            Nous avons alors
            \begin{equation}
                \| g \|_{\infty}=\sup_{x\in \Omega}| g(x) |=\sup_{i\in \eN}\| g_i \|_{\infty}\leq \| \phi \|.
            \end{equation}
            Une petite justification pour cela ? Prenons une suite \( x_k\) telle que \( | g(x_k) |\to \| g \|_{\infty}\). Vu que les \( A_i\) recouvrent \( \Omega\), existe un naturel \( i(k)\) tel que \( x_k\in A_{i(k)}\). Nous avons alors
            \begin{equation}
                | g(x_k) |\leq \| g_{i(k)} \|_{\infty}\leq \| \phi \|.
            \end{equation}
            Cela pour conclure que \( g\in L^{\infty}\).

            Notons que cet argument ne tient pas avec \( p> 1\) parce que l'équation \eqref{EqUPCTooJvoKKI} terminerait sur \( \| \phi \|\mu(E)^{1/p}\). Du coup l'ensemble \( S\) à prendre serait \( S=\{ t\in \eC\tq | t |\leq \| \phi \|\mu(E)^{1/p-1} \}\) et nous sommes en dehors des hypothèses du lemme parce qu'il n'y a pas d'ensemble \emph{indépendant} de \( E\) dans lequel l'intégrale \( \frac{1}{ \mu(E) }\int_{E}\bar gd\mu\) prend ses valeurs.

        \item[\( 1<p<\infty\)]

            La fonction
            \begin{equation}
                \alpha(x)=\begin{cases}
                    \frac{ g(x) }{ | g(x) | }    &   \text{si } g(x)\neq 0\\
                    1    &    \text{si } g(x)=0
                \end{cases}
            \end{equation}
            a la propriété de faire \( \alpha g=| g |\) en même temps que \( | \alpha(x) |=1\) pour tout \( x\). Nous définissons
            \begin{equation}
                E_n=\{ x\tq | g(x) |\leq n \}
            \end{equation}
            et
            \begin{equation}
                f_n=\mtu_{E_n}| g^{q-1} |\alpha.
            \end{equation}
            Ce qui est bien avec ces fonctions c'est que\footnote{C'est ici que nous utilisons le lien entre $p$ et $q$. En l'occurrence, de $1/p+1/q=1$ nous déduisons $q(p-1)=p$.}
            \begin{equation}
                | f_n |^p=| g^{p(q-1)} | \alpha |^p=| g |^q
            \end{equation}
            sur \( E_n\). Dans \( E_n\) nous avons \( | f_n |=| g^{q-1} |\leq n^{q-1}\) et dans \( E_n\) nous avons \( f_n=0\). Au final, \( f_n\in L^{\infty}\). Par ce que nous avons vu plus haut, nous avons alors
            \begin{equation}
                \phi(f_n)=\Phi_g(f_n).
            \end{equation}
            Par ailleurs,
            \begin{equation}
                f_n\bar g=\mtu_{E_n}| g^{q-1} |\frac{ g }{ | g | }\bar g,
            \end{equation}
            donc\quext{Dans \cite{MathAgreg}, cette équation arrive sans modules, ce qui me laisse entendre que \( \phi(f_n)\) est réel et positif pour pouvoir écrire que \( \phi(f_n)\leq \| \phi \|\| f_n \|_p\), mais je ne comprends pas pourquoi.}
            \begin{subequations}
                \begin{align}
                    \left|\int_{E_n}| g |^qd\mu\right|&=|\int_{\Omega}f_n\bar gd\mu|\\
                    &=|\phi(f_n)|\\
                    &\leq \| \phi \|\| f_n \|_p\\
                    &=\| \phi \|\left( \int_{E_n}| f_n |^p \right)^{1/p}\\
                    &=\| \phi \|\left( \int_{E_n}| g |^q \right)^{1/p}.
                \end{align}
            \end{subequations}
            Nous avons de ce fait une inégalité de la forme \( A\leq \| \phi \|A^{1/p}\) et donc aussi \( A^{1/p}\leq \| \phi \|^{1/p}A^{1/p^2}\), et donc \( A\leq \| \phi \|\| \phi \|^{1/p}A^{1/p^2}\). Continuant ainsi à injecter l'inégalité dans elle-même, pour tout \( k\in \eN\) nous avons :
            \begin{equation}
                \left| \int_{E_n}| g |^qd\mu \right| \leq\| \phi \|^{1+\frac{1}{ p }+\cdots+\frac{1}{ p^k }}\left( \int_{E_n}| g |^qd\mu \right)^{1/p^k}.
            \end{equation}
            Nous pouvons passer à la limite \( k\to \infty\). Sachant que \( p>1\) nous savons \( A^{1/k}\to 1\) et
            \begin{equation}
                1+\frac{1}{ p }+\cdots+\frac{1}{ p^k }\to\frac{ p }{ p-1 }=q.
            \end{equation}
            Nous avons alors
            \begin{equation}
                \int_{E_n}| g |^qd\mu\leq \| \phi \|^q.
            \end{equation}
            L'intégrale s'écrit tout aussi bien sous la forme \( \int_{\Omega}| g  |^q\mtu_{E_n}\). La fonction dans l'intégrale est une suite croissante de fonctions mesurables à valeurs dans \( \mathopen[ 0 , \infty \mathclose]\). Nous pouvons alors permuter l'intégrale et la limite \( n\to \infty\) en utilisant la convergence monotone (théorème~\ref{ThoRRDooFUvEAN}) qui donne alors \( \int_{\Omega}| g |^q\leq \| \phi \|^q\) ou encore
            \begin{equation}
                \| g \|_q\leq \| \phi \|.
            \end{equation}

            Ceci achève de prouver que l'application \( \phi\mapsto \Phi_g\) est une isométrie, et donc le théorème.
    \end{subproof}
\end{proof}

\begin{theorem}     \label{THOooXMVTooBAbyvr}
    Soit un espace mesuré \( (\Omega,\tribA,\mu)\).
    \begin{enumerate}
        \item   \label{ITEMooNCVEooTyNsoJ}
            Si \( 1<p<\infty\), alors \( L^p(\Omega,\tribA,\mu)\) est réflexif\footnote{Définition \ref{PROPooMAQSooCGFBBM}.}.
        \item   \label{ITEMooTQDJooFShTiA}
            Si \( (\Omega,\tribA,\mu)\) est \( \sigma\)-finie, alors
            \begin{enumerate}
                \item       \label{ITEMooHMMZooMQxWgB}
                    \( (L^1)'=L^{\infty}\)
                \item       \label{ITEMooBFFZooNxoHER}
                    \( L^1\subset (L^{\infty})' \).
            \end{enumerate}
    \end{enumerate}
\end{theorem}
\index{dual!de $L^p(\Omega)$}

\begin{proof}
    En plusieurs parties, en notant toujours \( p\) et \( q\) les exposants conjugués, c'est à dire \( \frac{1}{ p }+\frac{1}{ q }=1\).
    \begin{subproof}
        \item[Pour \ref{ITEMooNCVEooTyNsoJ}]
            Le théorème \ref{ThoLPQPooPWBXuv}\ref{ITEMooSQQBooWSFBmX} nous indique que
            \begin{equation}        \label{EQooQHBIooWrkiYC}
                (L^p)'=L^q
            \end{equation}
            au sens d'une bijection isométrique. Vu que \( 1<p<\infty\), nous avons aussi \( 1<q<\infty\) et donc \( (L^q)'=L^p\). En prenant le dual des deux côtés de \eqref{EQooQHBIooWrkiYC}, 
            \begin{equation}
                (L^p)''=(L^q)'=L^p,
            \end{equation}
            et nous avons prouvé que \( L^p\) est réflexif.
        \item[Pour \ref{ITEMooHMMZooMQxWgB}]
            Il s'agit du théorème \ref{ThoLPQPooPWBXuv}\ref{ITEMooCQGJooOWzjoV}.
        \item[Pour \ref{ITEMooBFFZooNxoHER}\cite{MonCerveau}]
            Il nous reste à couvrir le cas de \( (L^{\infty})'\). Pour \( g\in L^1\) nous prouvons que \( \Phi_g\in (L^{\infty})'\). 
            
            \begin{subproof}
                \item[\( \Phi_g(f)\) est bien définie]
                    Nous prouvons d'abord que si \( f\in L^{\infty}\), alors l'intégrale \( \int_{\Omega}f\bar g\) est bien définie. Par définition du supremum essentiel\footnote{Voir les définitions \ref{DEFooIQOOooLpJBqi} et \ref{DEFooXUKHooXYrlYq}.}, il existe \( M>0\) tel que \( | f(x) |<M\) pour tout \( x\) hors d'une partie \( A\) de mesure nulle. Nous avons alors
                    \begin{equation}
                        \int_{\Omega}|f\bar g|=\int_{\Omega\setminus A}| f\bar g |\leq M\int_{\Omega\setminus A}| f |= M\int_{\Omega}| f |<\infty.
                    \end{equation}
                \item[\( \Phi_g\) est continue]
                    Soit une suite \( f_k\stackrel{L^{\infty}}{\longrightarrow}f\) ainsi que \( g\in L^1\). Pour chaque \( k\), il existe une partie de mesure nulle \( A_k\) et un nombre \( M_k=\| f_k \|_{L^{\infty}}\) tel que \( | f_k(x) |<\| f_k \|_{L^{\infty}}\) pour tout \( x\) hors de \( A_k\). Nous avons alors
                    \begin{equation}
                        | \Phi_g(f_k) |\leq \int_{\Omega\setminus A_k}| f_k\bar g |d\mu\leq \| f_k \|_{L^{\infty}}\int_{\Omega\setminus A_k}| g |\leq \| f_k \|_{L^{\infty}}\| g \|_1.
                    \end{equation}
                    Vu que par hypothèse \( f_k\to 0\) dans \( L^{\infty}\), nous avons \( \| f_k \|_{L^{\infty}\to 0}\), et donc aussi
                    \begin{equation}
                        |\Phi_g(f_k)|\to 0.
                    \end{equation}
            \end{subproof}
    \end{subproof}
\end{proof}

\begin{proposition} \label{PropUKLZZZh}
    Soit \( f\in L^p(\Omega)\) telle que
    \begin{equation}
        \int_{\Omega}f\varphi=0
    \end{equation}
    pour tout \( \varphi\in C^{\infty}_c(\Omega)\). Alors \( f=0\) presque partout.
\end{proposition}

\begin{proof}
    Nous considérons la forme linéaire \( \Phi_f\in (L^q)'\) donnée par
    \begin{equation}
        \begin{aligned}
            \Phi_f\colon L^p&\to \eC \\
            u&\mapsto \int_{\Omega}fu
        \end{aligned}
    \end{equation}
    Par hypothèse cette forme est nulle sur la partie dense \(  C^{\infty}_c(\Omega)\). Si \( (\varphi_n)\) est une suite dans \(  C^{\infty}_c(\Omega)\) convergente vers \( u\) dans \( L^p\), nous avons pour tout \( n\) que
    \begin{equation}
        0=\Phi_f(\varphi_n)
    \end{equation}
    En passant à la limite, nous voyons que \( \Phi_f\) est la forme nulle. Elle est donc égale à \( \Phi_0\). La partie « unicité » du théorème de représentation de Riesz~\ref{ThoLPQPooPWBXuv} nous indique alors que \( f=0\) dans \( L^p\) et donc \( f=0\) presque partout.
\end{proof}

\begin{proposition} \label{PropLGoLtcS}
    Si \( f\in L^1_{loc}(I)\) est telle que
    \begin{equation}
        \int_If\varphi'=0
    \end{equation}
    pour tout \( \varphi\in  C^{\infty}_c(I)\), alors il existe une constante \( C\) telle que \( f=C\) presque partout.
\end{proposition}

\begin{proof}
    Soit \( \psi\in C^{\infty}_c(I)\) une fonction d'intégrale \( 1\) sur \( I\). Si \( w\in C^{\infty}_c(I)\) alors nous considérons la fonction
    \begin{equation}
        h=w-\psi\int_Iw,
    \end{equation}
    qui est dans \(  C^{\infty}_c(I)\) et dont l'intégrale sur \( I\) est nulle. Par la proposition~\ref{PropHFWNpRb}, la fonction \( h\) admet une primitive dans \(  C^{\infty}_c(I)\); et nous notons \( \varphi\) cette primitive. L'hypothèse appliquée à \( \varphi\) donne
    \begin{equation}
        0=\int_If\varphi'=\int_If\left( w-\psi\int_Iw \right)=\int_Ifw-\underbrace{\left( \int_If(x)\psi(x)dx \right)}_C\left( \int_Iw(y)dy \right)=\int_Iw(f-C).
    \end{equation}
    L'annulation de la dernière intégrale implique par la proposition~\ref{PropUKLZZZh} que \( f-C=0\) dans \( L^2\), c'est-à-dire \( f=C\) presque partout.
\end{proof}

Dans \cite{ooHGADooNGZnbt}, il est dit que « la preuve [du lemme suivant], un peu fastidieuse mais en rien ingénieuse, est laissée en exercice ». La preuve est donc de moi; elle est un tout petit peu ingénieuse mais en rien fastidieuse. J'espère ne pas m'être trompé et me demande bien ce que l'auteur avait en tête. Ma preuve s'appuie sur la proposition \ref{PROPooLIGIooPrHYlb} dont la preuve ne me paraît pas non plus «fastidieuse mais en rien ingénieuse».

\begin{lemma}[\cite{ooHGADooNGZnbt,MonCerveau}]        \label{LEMooLDQRooEGWDlm}
    Soient \( r>0\). Il existe \( \delta>0\) tel que pour tout \( s,t\in \eC\) vérifiant \( | s |\leq 1\), \( | t |\leq 1\) et \( | s-t |\geq r\) nous ayons
    \begin{equation}
        \left| \frac{ s+t }{ 2 } \right|^p\leq (1-\delta)\frac{ | s |^p+| t |^p }{2}.
    \end{equation}
\end{lemma}

\begin{proof}
    Soit \( r>0\). La partie de \( \eC^2\) donnée par
    \begin{equation}
        D=\{ (s,t)\in \eC^2\tq | s |\leq 1, | t |\leq 1,| s-t |\geq r \}
    \end{equation}
    est compacte. En effet elles est bornée (par la sphère de rayon \( \sqrt{ 2 }\)) et fermée comme intersection de fermée\footnote{Lemme \ref{LemQYUJwPC} suivit du théorème de Borel-Lebesgue \ref{ThoXTEooxFmdI}.}. Nous considérons la fonction \( \Delta\colon D\to \eR\) donnée par
    \begin{equation}
        \left| \frac{ s+t }{2} \right|^p=\Delta(s,t)\frac{ | s |^p+| t |^p }{2}.
    \end{equation}
    Si vous voulez une expression explicite,
    \begin{equation}
        \Delta(s,t)=\frac{ 2^{p-1}| s+t |^p }{ | s |^p+| t |^p }.
    \end{equation}
    Cela est bien défini et continu sur \( D\) parce que le complémentaire \( D^c\) (qui est ouvert) contient \( (0,0)\) et donc aussi un voisinage de \( (0,0)\).
\end{proof}
<++>

%+++++++++++++++++++++++++++++++++++++++++++++++++++++++++++++++++++++++++++++++++++++++++++++++++++++++++++++++++++++++++++
\section{Théorèmes de Hahn-Banach}
%+++++++++++++++++++++++++++++++++++++++++++++++++++++++++++++++++++++++++++++++++++++++++++++++++++++++++++++++++++++++++++

\begin{theorem}[Hahn-Banach\cite{brezis,TQSWRiz}]
    Soit \( E\), un espace vectoriel réel et une application \( p\colon E\to \eR\) satisfaisant
    \begin{enumerate}
        \item
            \( p(\lambda x)=\lambda p(x)\) pour tout \( x\in E\) et pour tout \( \lambda>0\),
        \item
            \( p(x+y)\leq p(x)+p(y)\) pour tout \( x,y\in E\).
    \end{enumerate}
    Soit de plus \( G\subset E\) un sous-espace vectoriel muni d'une application \( g\colon G\to \eR\) vérifiant \( g(x)\leq p(x)\) pour tout \( x\in G\). Alors il existe \( f\in\aL(E,\eR)\) telle que \( f(x)=g(x)\) pour tout \( x\in G\) et \( f(x)\leq p(x)\) pour tout \( x\in E\).
\end{theorem}
\index{théorème!Hahn-Banach}

\begin{proof}
    Si \( h\) une application linéaire définie sur un sous-espace de \( E\), nous notons \( D_h\) ledit sous-espace.

    \begin{subproof}
    \item[Un ensemble inductif]

        Nous considérons \( P\), l'ensemble des fonctions linéaires suivant
        \begin{equation}
            P=\Big\{  h\colon D_h\to \eR\tq
            \begin{cases}
                G\subset D_h\\
                h(x)=g(x)&\forall x\in G\\
                h(x)\leq p(x)&\forall x\in D_h
            \end{cases}
        \Big\}
        \end{equation}
        Cet ensemble est non vide parce que \( g\) est dedans. Nous le munissons de la relation d'ordre \( h_1\leq h_2\) si et seulement si \( D_{h_1}\subset D_{h_2}\) et \( h_2\) prolonge \( h_1\). Nous montrons à présent que \( P\) est un ensemble inductif. Soit un sous-ensemble totalement ordonné \( Q\subset P\); nous définissons une fonction \( h\) de la façon suivante. D'abord \( D_h=\sup_{l\in Q}D_l\) et ensuite
        \begin{equation}
            \begin{aligned}
                h\colon D_h&\to \eR \\
                x&\mapsto l(x)&\text{si } x\in D_l
            \end{aligned}
        \end{equation}
        Cela est bien définit parce que si \( x\in D_l\cap D_{l'}\) alors, vu que \( Q\) est totalement ordonné (i.e. \( l\leq l'\) ou \( l'\leq l\)), on a obligatoirement \( D_l\subset D_{l'}\) et \( l'\) qui prolonge \( l\) (ou le contraire). Donc \( h\) est un majorant de \( Q\) dans \( P\) parce que \( h\geq l\) pour tout \( l\in Q\). Cela montre que \( P\) est inductif (définition~\ref{DefGHDfyyz}). Le lemme de Zorn~\ref{LemUEGjJBc} nous dit alors que \( P\) possède un maximum \( f\) qui va être la réponse à notre théorème.

    \item[Le support de \( f\)]

        La fonction \( f\) est dans \( P\); donc \( f(x)\leq p(x)\) pour tout \( x\in D_h\) et \( f(x)=g(x)\) pour tout \( x\in G\). Pour terminer nous devons montrer que \( D_f=E\). Supposons donc que \( D_f\neq E\) et prenons \( x_0\notin D_f\). Nous allons contredire la maximalité de \( f\) en considérant la fonction \( h\) donnée par \( D_h=D_f+\eR x_0 \) et
        \begin{equation}
            h(x+tx_0)=f(x)+t\alpha
        \end{equation}
        où \( \alpha\) est une constante que nous allons fixer plus tard.

        Nous commençons par prouver que \( f\) est dans \( P\). Nous devons prouver que
        \begin{equation}    \label{EqOIXrlFe}
            h(x+tx_0)=f(x)+t\alpha\leq p(x+tx_0)
        \end{equation}
        Pour cela nous allons commencer par fixer \( \alpha\) pour avoir les relations suivantes :
        \begin{subequations}    \label{EqMDNkcQk}
            \begin{numcases}{}
                f(x)+\alpha\leq p(x+x_0)    \label{EqDYmRWEY}\\
                f(x)-\alpha\leq p(x-x_0)
            \end{numcases}
        \end{subequations}
        pour tout \( x\in D_f\). Ces relations sont équivalentes à demander \( \alpha \) tel que
        \begin{subequations}
            \begin{numcases}{}
                \alpha\leq p(x+x_0)-f(x)\\
                \alpha\geq f(x)-p(x-x_0)
            \end{numcases}
        \end{subequations}
        Nous nous demandons donc s'il existe un \( \alpha\) qui satisfasse
        \begin{equation}
            \sup_{y\in D_f}\big( f(y)-p(y-x_0) \big)\leq \alpha\leq \inf_{z\in D_f}\big( p(z+x_0)-f(z) \big).
        \end{equation}
        Ou encore nous devons prouver que pour tout \( y,z\in D_f\),
        \begin{equation}
            p(z+x_0)-f(x)\geq f(y)-p(y-x_0)\geq 0.
        \end{equation}
        Par les propriétés de \( p\) et de \( f\),
        \begin{equation}
        p(z+x_0)+p(y-x_0)-f(z)-f(y)\geq p(z+y)-f(z+y)\geq 0.
        \end{equation}
        La dernière inégalité est le fait que \( f\in P\). Un choix de \( \alpha\) donnant les inéquations \eqref{EqMDNkcQk} est donc possible.

        À partir des inéquations \eqref{EqMDNkcQk} nous obtenons la relation \eqref{EqOIXrlFe} de la façon suivante. Si \( t>0\) nous multiplions l'équation \eqref{EqDYmRWEY} par \( t\) :
        \begin{equation}
            tf(x)+t\alpha\leq tp(x+x_0).
        \end{equation}
        Et nous écrivons cette relation avec \( x/t\) au lieu de \( x \) en tenant compte de la linéarité de \( f\) :
        \begin{equation}
            f(x)+t\alpha\leq  tp\big( \frac{ x }{ t }+x_0 \big)=p(x+tx_0).
        \end{equation}
        Avec \( t<0\), c'est similaire, en faisant attention au sens des inégalités.

        Nous avons donc construit \( h\colon D_h\to \eR\) avec \( h\in P\), \( D_f\subset D_h\) et \( h(x)=f(x)\) pour tout \( x\in D_f\). Cela pour dire que \( h>f\), ce qui contredit la maximalité de \( f\). Le domaine de \( f\) est donc \( E\) tout entier.

        La fonction \( f\) est donc une fonction qui remplit les conditions.

    \end{subproof}
\end{proof}

\begin{definition}  \label{DefPJokvAa}
    Un espace topologique est \defe{localement convexe}{convexité!locale} si tout point possède un système fondamental de voisinages formé de convexes.
\end{definition}
%TODO : il faudrait parler de système fondamental de voisinages.

\begin{definition}[Hyperplan qui sépare]
    Soit \( E\) un espace vectoriel topologique ainsi que \( A\), \( B\) des sous-ensembles de \( E\). Nous disons que l'hyperplan d'équation \( f=\alpha\) \defe{sépare au sens large}{hyperplan!séparer!au sens large} les parties \( A\) et \( B\) si \( f(x)\leq \alpha\) pour tout \( x\in A\) et \( f(x)\geq \alpha\) pour tout \( x\in B\).

    La séparation est \defe{au sens strict}{hyperplan!sépare!au sens strict} s'il existe \( \epsilon>0\) tel que
    \begin{subequations}
        \begin{align}
            f(x)\leq \alpha-\epsilon&&\text{pour tout } x\in A\\
            f(x)\geq \alpha+\epsilon&&\text{pour tout } x\in B.
        \end{align}
    \end{subequations}
\end{definition}

\begin{theorem}[Hahn-Banach, première forme géométrique\cite{TQSWRiz}]  \label{ThoSAJjdZc}
    Soit \( E\) un espace vectoriel topologique et \( A\), \( B\) deux convexes non vides disjoints de \( E\). Si \( A\) est ouvert, il existe un hyperplan fermé qui sépare \( A\) et \( B\) au sens large.
\end{theorem}

\begin{theorem}[Hahn-Banach, seconde forme géométrique] \label{ThoACuKgtW}
    Soient un espace vectoriel topologique localement convexe\footnote{Définition~\ref{DefPJokvAa}.} ainsi que des convexes non vides disjoints \( A\) et \( B\) tels que \( A\) soit compact et \( B\) soit fermé. Alors il existe un hyperplan fermé qui sépare strictement \( A\) et \( B\).
\end{theorem}

\begin{proof}
    Vu que \( B\) est fermé, \( A\) est dans l'ouvert \( E\setminus B\). Donc si \( a\in A\), il existe un voisinage ouvert convexe de \( a\) inclus dans \( A\). Soit \( U_a\) un voisinage ouvert et convexe de \( 0\) tel que \( (a+U_a)\cap B=\emptyset\).

    Vu que la fonction \( (x,y)\mapsto x+y\) est continue, nous pouvons trouver un ouvert convexe \( V_a\) tel que \( V_a+V_a\subset U_a\). L'ensemble \( a+V_a\) est alors un voisinage ouvert de \( a\) et bien entendu \( \bigcup_a(a+V_a)\) recouvre \( A\) qui est compact. Nous en extrayons un sous-recouvrement fini, c'est-à-dire que nous considérons \( a_1,\ldots, a_n\in A\) tels que
    \begin{equation}
        A\subset \bigcup_{i=1}^n(a_i+V_{a_i}).
    \end{equation}
    Nous posons alors
    \begin{equation}
        V=\bigcap_{i=1}^nV_{a_i}.
    \end{equation}
    Cet ensemble est non vide parce et il contient un voisinage de zéro parce que c'est une intersection finie de voisinages de zéro. Soit \( x\in A+V\). Il existe \( i\) tel que
    \begin{equation}
        x\in a_i+U_{a_i}+V\subset a_i+V_{a_i}+V_{a_i}\subset a_i+U_{a_i}\subset E\setminus B.
    \end{equation}
    Donc \( (A+V)\cap B=\emptyset\). L'ensemble \( A+V\) est alors un ouvert convexe disjoint de \( B\). Par la première forme géométrique du théorème de Hahn-Banach~\ref{ThoSAJjdZc} nous avons un hyperplan qui sépare \( A+V\) de \( B\) au sens large : il existe \( f\in E'\setminus\{ 0 \}\) tel que \( f(a)+f(v)\leq f(b)\) pour tout \( a\in A\), \( v\in V\) et \( b\in B\).

    Il suffit donc de trouver un \( v\in V\) tel que \( f(v)\neq 0\) pour avoir la séparation au sens strict. Cela est facile : \( V\) étant un voisinage de zéro et \( f\) étant linéaire, si elle était nulle sur \( V\), elle serait nulle sur \( E\).
\end{proof}

%+++++++++++++++++++++++++++++++++++++++++++++++++++++++++++++++++++++++++++++++++++++++++++++++++++++++++++++++++++++++++++
\section{Théorème de Tietze}
%+++++++++++++++++++++++++++++++++++++++++++++++++++++++++++++++++++++++++++++++++++++++++++++++++++++++++++++++++++++++++++

\begin{definition}
Si \( E\) et \( F\) sont des espaces normés, une application \( f\colon E\to F\) est \defe{presque surjective}{presque!surjective} s'il existe \( \alpha\in\mathopen] 0 , 1 \mathclose[\) et \( C>0\) tels que pour tout \( y\in \overline{ B_F(0,1) }\), il existe \( x\in\overline{ B_E(0,C) }\) tel que \( \| y-f(x) \|\leq \alpha\).
\end{definition}

\begin{lemma}[\cite{KXjFWKA}]   \label{LemBQLooRXhJzK}
    Soient \( E\) et \( F\) des espaces de Banach et \( f\in\cL(E,F)\)\footnote{L'ensemble des applications linéaires continues}. Si \( f\) est presque surjective, alors
    \begin{enumerate}
        \item   \label{ItemTSOooYkxvBui}
            \( f\) est surjective
        \item\label{ItemTSOooYkxvBuii}
            pour tout \( y\in \overline{ B_F(0,1) }\), il existe \( x\in\overline{ B_E(0,\frac{ C }{ 1-\alpha }) }\) tel que \( y=f(x)\).
    \end{enumerate}
\end{lemma}
Le point~\ref{ItemTSOooYkxvBuii} est une précision du point~\ref{ItemTSOooYkxvBui} : il dit quelle est la taille de la boule de \( E\) nécessaire à obtenir la boule unité dans \( F\).

\begin{proof}
    Soit \( y\in \overline{ B_F(0,1) }\). Nous allons construire \( x\in B\big( 0,\frac{ C }{ 1-\alpha } \big)\) qui donne \( f(x)=y\). Ce \( x\) sera la limite d'une série que nous allons construire par récurrence. Pour \( n=1\) nous utilisons la presque surjectivité pour considérer \( x_1\in\overline{ B_E(0,C) } \) tel que \( \| y-f(x_1) \|\leq \alpha\). Ensuite nous considérons la récurrence
    \begin{equation}
        x_n\in \overline{ B_E(0,C) }
    \end{equation}
    tel que
    \begin{equation}
        \big\| y-\sum_{i=1}^n\alpha^{i-1}f(x_i) \big\|\leq \alpha^n
    \end{equation}
    Pour montrer que cela existe nous supposons que la série est déjà construire jusqu'à \( n>1\) :
    \begin{equation}
        \frac{1}{ \alpha^n }\Big( y-\sum_{i=1}^n\alpha^{i-1}f(x_i) \Big)\in \overline{ B_F(0,1) }
    \end{equation}
    À partir de là, par presque surjectivité il existe un \( x_{n+1}\in \overline{ B_E(0,C) }\) tel que
    \begin{equation}
        \big\| \frac{ y-\sum_{i=1}^n\alpha^{i-1}f(x_i) }{ \alpha^n }-f(x_{n+1}) \big\|\leq \alpha.
    \end{equation}
    En multipliant par \( \alpha^{n}\), le terme \( \alpha^nf(x_{n+1})\) s'intègre bien dans la somme :
    \begin{equation}
        \big\| y=\sum_{i=1}^{n+1}\alpha^{i-1}f(x_i) \big\|\leq \alpha^{n+1}.
    \end{equation}
    Nous nous intéressons à une éventuelle limite à la somme des \( \alpha^{n-1}x_n\). D'abord nous avons la majoration \( \| \alpha^{n-1}x_n \|\leq \alpha^{n-1}C\), et vu que par la définition de la presque surjectivité \( 0<\alpha<1\), la série
    \begin{equation}
        \sum_{n=1}^{\infty}\alpha^{n-1}x_n
    \end{equation}
    converge absolument\footnote{Définition~\ref{DefVFUIXwU}.} parce que la suite des normes est une suite géométrique de raison \( \alpha\). Vu que \( E\) est de Banach, la convergence absolue implique la convergence simple (la suite des sommes partielles est de Cauchy et Banach est complet). Nous posons
    \begin{equation}
        x=\sum_{n=1}^{\infty}\alpha^{n-1}x_n\in E,
    \end{equation}
    et en termes de normes, ça vérifie
    \begin{equation}
        \| x \|\leq\sum_{n=1}^{\infty}\alpha^{n-1}\| x_n \|\leq C\sum_{n=1}^{\infty}\alpha^{n-1}=\frac{ C }{ 1-\alpha }.
    \end{equation}
    Donc c'est bon pour avoir \( x\in B\big( 0,\frac{ C }{ 1-\alpha } \big)\). Nous devons encore vérifier que \( y=f(x)\). Pour cela nous remarquons que
    \begin{equation}
        \| y-f\Big( \sum_{n=1}^N\alpha^{n-1}x_n \Big) \|\leq \alpha^N.
    \end{equation}
    Nous pouvons prendre la limite \( N\to \infty\) et permuter \( f\) avec la limite (par continuité de \( f\)). Vu que \( 0<\alpha<1\) nous avons
    \begin{equation}
        \| y-f(x) \|=0.
    \end{equation}
\end{proof}

\begin{theorem}[Tietze\cite{KXjFWKA,ytMOpe}]   \label{ThoFFQooGvcLzJ}
    Soit un espace métrique \( (X,d)\) et un fermé \( Y\subset X\). Soit \( g_0\in C^0(Y,\eR)\). Alors \( g_0\) admet un prolongement continu sur \( X\).
\end{theorem}

\begin{proof}
    Soit l'opération de restriction
    \begin{equation}
        \begin{aligned}
            T\colon (C^0_b(X,\eR),\| . \|_{\infty})&\to (C^0_b(Y,\eR),\| . \|_{\infty}) \\
            f&\mapsto f|_Y.
        \end{aligned}
    \end{equation}
    L'application \( T\) est évidemment linéaire. Elle est de plus borné pour la norme opérateur usuelle donnée par la proposition~\ref{DefNFYUooBZCPTr} parce que \( \| T(f) \|\leq \| f \|<\infty\). L'application \( T\) est alors continue par la proposition~\ref{PROPooQZYVooYJVlBd}.

    \begin{subproof}
    \item[Presque surjection]

    Soit \( g\in C^0_b(Y,\eR)\) avec \( \| g \|_{\infty}\leq 1\). Nous posons
    \begin{subequations}
        \begin{align}
            Y^+=\{ x\in Y\tq \frac{1}{ 3 }\leq g(x)\leq 1 \}\\
            Y^-=\{ x\in Y\tq -1\leq g(x)\leq -\frac{1}{ 3 } \}.
        \end{align}
    \end{subequations}
    Nous considérons alors
    \begin{equation}
        \begin{aligned}
            f\colon X&\to \eR \\
            x&\mapsto \frac{1}{ 3 }\frac{ d(x,Y^-)-d(x,Y^+) }{ d(x,Y^-)+d(x,Y^+) }
        \end{aligned}
    \end{equation}
    Vu qu'en valeur absolue le dénominateur est plus grand que le numérateur nous avons \( \| f \|_{\infty}\leq \frac{1}{ 3 }\). Notons que
    \begin{itemize}
        \item Si \( x\in Y^+\) alors \( f(x)=\frac{1}{ 3 }\) et \( g(x)\in\mathopen[ \frac{1}{ 3 } , 1 \mathclose]\);
        \item Si \( x\in Y^-\) alors \( f(x)=-\frac{1}{ 3 }\) et \( g(x)\in\mathopen[-1,-\frac{1}{ 3 } \mathclose]\);
        \item Si \( x\) n'est ni dans \( Y^+\) ni dans \( Y^-\) alors nous avons\footnote{Nous rappelons que \( \| g \|=1\), donc \( g(x)\) est forcément ente \( -1\) et \( 1\).} \( g(x)\in\mathopen[ -\frac{1}{ 3 } , \frac{1}{ 3 } \mathclose]\) et donc \( \big| f(x)-g(x) \big|\leq \big| f(x) \big|+\big| g(x) \big|\leq \frac{ 2 }{ 3 }\).
    \end{itemize}
    Dans les deux cas nous avons \( \big| f(x)-g(x) \big|\in\mathopen[ 0 , \frac{ 2 }{ 3 } \mathclose]\) pour tout \( x\in X\). Cela prouve que
    \begin{equation}
        \| T(f)-g \|_{Y,\infty}\leq \frac{ 2 }{ 3 }.
    \end{equation}
    En résumé nous avons pris \( g\) dans la boule \( \overline{ B(0,1) }\) de \( \big( C^0_b(Y,\eR), \| . \|_{\infty} \big)\) et nous avons construit une fonction \( f\) dans la boule \( \overline{ B(0,\frac{1}{ 3 }) }\) de \( \big( C^0_b(X,\eR),\| . \|_{\infty} \big)\) telle que \( \| T(f)-g \|_{\infty}\leq \frac{ 2 }{ 3 }\). L'application \( T\) est donc une presque surjection avec \( \alpha=\frac{1}{ 3 }\) et \( C=\frac{ 2 }{ 3 }\).

\item[Prolongement dans les boules unité fermées]

    La proposition~\ref{PropSYMEZGU} nous assure que les espaces \( C^0_b(X,\eR)\) et \( C_b^0(Y,\eR)\) sont de Banach (complets), et le lemme~\ref{LemBQLooRXhJzK} nous dit alors que \( T\) est surjective et que pour tout \( g\in\overline{ B(0,1) }\), il existe
    \begin{equation}
        f\in\overline{ B\left( 0,\frac{ 1/3 }{ 1-\frac{ 2 }{ 3 } } \right) }=\overline{ B(0,1) }.
    \end{equation}
    telle que \( g=T(f)\).


\item[Prolongement pour les boules ouvertes]

    Jusqu'à présent nous avons montré qu'une fonction \( g\in\overline{ B(0,1) }\) admet une prolongement continu dans \( \overline{ B(0,1) }\). Nous allons montrer que si \( g\) est dans la boule ouverte \( B(0,1)\) de \( \big( C^0_b(Y,\eR),\| . \|_{\infty} \big)\) alors \( g\) admet un prolongement dans la boule ouverte \( B(0,1)\) de \( \big( C_b^0(X,\eR),\| . \|_{\infty} \big)\).

    Soit \( g\in B_{C^0_b(Y)}(0,1) \) et son prolongement \( h\in \overline{ B_{C_b^0(X)}(0,1) }\). Si \( \| h \|_{\infty}<1\) alors le résultat est vrai. Sinon nous considérons l'ensemble
    \begin{equation}
        Z=\{ x\in X\tq | h(x) |=1 \}.
    \end{equation}
    Nous avons \( Y\cap Z=\emptyset\) parce que nous avons \( h=g\) sur \( Y\) et nous avons choisi \( \| g \|_{\infty}<\infty\). Par ailleurs \( Y\) est fermé par hypothèse et \( Z\) est fermé parce que \( h\) est continue; par conséquent \( Y\cap Z\) est fermé, donc\footnote{Si vous avez l'intention de dire que \( \overline{ Y\cap Z }=\bar Y\cap\bar Z=Y\cap Z=\emptyset\), allez d'abord voir l'exemple~\ref{ExBFLooUNyvbw}. Ici c'est correct parce que \( Y\) et \( Z\) sont fermés.}
    \begin{equation}
        \bar Y\cap\bar Z=Y\cap Z=\emptyset.
    \end{equation}
    Nous posons
    \begin{equation}
        \begin{aligned}
            u\colon X&\to \eR^+ \\
            x&\mapsto \frac{ d(x,Z) }{ d(x,Y)+d(x,Z) }
        \end{aligned}
    \end{equation}
    Le dénominateur n'est pas nul parce qu'il faudrait \( d(x,Y)=d(x,Z)=0\), ce qui demanderait \( x\in \bar Y\cap\bar Z\), ce qui n'est pas possible. Nous posons \( f=uh\). Si \( x\in Y\) alors \( u(x)=1\), donc \( f\) est encore un prolongement de \( g\). De plus \( f\) est encore continue, et donc encore un bon candidat. Enfin si \( x\) est hors de \( Y\) alors \( d(x,Y)>0\) (strictement parce que \( Y\) est fermé) et donc \( 0<u(x)<1\), ce qui donne \( | f(x) |<| h(x) |\leq 1\). Donc \( \| f \|_{\infty}<1\).

    Nous avons donc trouvé qu'une fonction dans la boule ouverte \( B_{C^0_b(Y)}(0,1)\) se prolonge en une fonction dans la boule ouverte \( B_{C^0_b(X)}(0,1)\).

\item[Le cas non borné]

Soit enfin \( g_0\in C^0(Y,\eR)\). Nous allons nous ramener au cas de la boule unité ouverte en utilisant un homéomorphisme \( \phi\colon \eR\to \mathopen] -1 , 1 \mathclose[\). L'application \( g=\phi\circ g_0\) est dans la boule unité ouvert de \( C^0(Y,\eR)\) et donc admet un prolongement \( f\) dans la boule unité ouverte de \( C^0(X)\). L'application \( f_0=\phi^{-1}\circ f\) est un prolongement continu de \( g_0\).

    \end{subproof}
\end{proof}

Un homéomorphisme \( \phi\colon \eR\to \mathopen] -1 , 1 \mathclose[\) est donné par exemple par la fonction \( \phi(t)=\frac{ 2 }{ \pi }\arctan(t)\) dont le graphique est donné ci-dessous :
\begin{center}
    \input{auto/pictures_tex/Fig_FXVooJYAfif.pstricks}
\end{center}

%+++++++++++++++++++++++++++++++++++++++++++++++++++++++++++++++++++++++++++++++++++++++++++++++++++++++++++++++++++++++++++
\section{Espace de Schwartz}
%+++++++++++++++++++++++++++++++++++++++++++++++++++++++++++++++++++++++++++++++++++++++++++++++++++++++++++++++++++++++++++

Pour un multiindice \( \alpha=(\alpha_1,\ldots, \alpha_d)\in \eN^d\), nous notons
\begin{equation}
    \partial^{\alpha}\varphi=\partial_{x_1}^{\alpha_1}\ldots\partial_{x_d}^{\alpha_d}\varphi
\end{equation}
pour peu que la fonction \( \varphi\) soit \( | \alpha |=\alpha_1+\cdots +\alpha_d\) fois dérivable.

\begin{definition}  \label{DefHHyQooK}
    Soit \( \Omega\subset\eR^d\). L'\defe{espace de Schwartz}{espace!de Schwartz} \( \swS(\Omega)\) est le sous-ensemble de \(  C^{\infty}(\Omega)\) des fonctions dont toutes les dérivées décroissent plus vite que tout polynôme :
    \begin{equation}
        \swS(\Omega)=\big\{   \varphi\in C^{\infty}(\Omega)\tq\forall \alpha,\beta\in \eN^d, p_{\alpha,\beta}(\varphi)<\infty   \big\}
    \end{equation}
    où nous avons considéré
    \begin{equation}    \label{EqOWdChCu}
        p_{\alpha,\beta}(\varphi)=\sup_{x\in \Omega}| x^{\beta}(\partial^{\alpha}\varphi)(x) |=\| x^{\beta}\partial^{\alpha}\varphi \|_{\infty}.
    \end{equation}
\end{definition}

Pour simplifier les notations (surtout du côté de Fourier), nous allons parfois écrire \( M_i\varphi\)\nomenclature[Y]{\( M_i\varphi\)}{La fonction \( x\mapsto x_i\varphi(x)\)} pour la fonction \( x\mapsto x_i\varphi(x)\).

\begin{example}
    La fonction \(  e^{-x^2}\) est une fonction à décroissance rapide sur \( \eR\).
\end{example}

\begin{definition}
    Une fonction \( f\colon \eR^d\to \eC\) est dite à \defe{décroissance rapide}{fonction!à décroissance rapide} si elle décroît plus vite que n'importe quel polynôme. Plus précisément, si pour tout polynôme \( Q\), il existe un \( r>0\) tel que \(  | f(x) |<\frac{1}{ | Q(x) | } \) pour tout \( \| x \|\geq r\).
\end{definition}

\begin{proposition} \label{PropCSmzwGv}
    Une fonction Schwartz est à décroissance rapide.
\end{proposition}

\begin{proof}
    Nous commençons par considérer un polynôme \( P\) donné par
    \begin{equation}
        P(x)=\sum_kc_kx^{\beta_k}
    \end{equation}
    où les \( \beta_k\) sont des multiindices, les \( c_k\) sont des constantes et la somme est finie. Nous avons la majoration
    \begin{equation}
        \sup_{x\in \eR^d}| \varphi(x)P(x) |\leq\sum_k\sup_x\big| c_k\varphi(x) x^{\beta_k} \big|\leq\sum_k| c_k |p_{0,\beta_k}(\varphi)<\infty.
    \end{equation}
    Nous allons noter \( M_P\) la constante \( \sum_k| c_k |p_{0,\beta_k}(\varphi)\), de sorte que pour tout \( x\in \eR^d\) nous ayons \( | \varphi(x)P(x) |\leq M_P\) et donc
    \begin{equation}
        | \varphi(x) |\leq \frac{ M_P }{ | P(x) | }=\frac{1}{ | \frac{1}{ M_P }P(x) | }.
    \end{equation}
    Notons que cette inégalité est a fortiori correcte pour les \( x\) sur lesquels \( P\) s'annule.

    Soit maintenant un polynôme \( Q\). Nous considérons le polynôme \( P(x)=\| x \|Q(x)\). Étant de plus haut degré, pour toute constante \( C\) il existe un rayon \( r_C\) tel que \( | P(x) |\geq C| Q(x) |\) pour tout \( | x |\geq r_C\). En particulier pour \( | x |\geq r_{M_P}\) nous avons
    \begin{equation}
        | P(x) |\geq M_P| Q(x) |
    \end{equation}
    et donc, pour ces \( x\),
    \begin{equation}
        | \varphi(x) |\leq \frac{1}{ | \frac{1}{ M_P }P(x) | }\leq \frac{1}{ | Q(x) | }.
    \end{equation}
    La première inégalité est valable pour tout \( x\), et la seconde pour \( \| x \|\geq r_{M_P}\).
\end{proof}

\begin{corollary}[\cite{MonCerveau}]        \label{CORooZFPSooHCFUSH}
    Soit \( \varphi\) une fonction Schwartz sur \( \eR^m\times \eR^n\). Alors la fonction
    \begin{equation}
        y\mapsto \sup_{x\in \eR^n}| \varphi(x,y) |
    \end{equation}
    est intégrable.
\end{corollary}

\begin{proof}
    Soit un polynôme \( Q\) en la variable \( y\). Par la proposition~\ref{PropCSmzwGv}, il existe \( r>0\) tel que
    \begin{equation}
        | \varphi(x,t) |<\frac{1}{ Q(y) }
    \end{equation}
    pour tout \( \| (x,y) \|>r\). A fortiori l'inégalité tient pour tout \( | y |>r\). Donc
    \begin{equation}
        \int_{\eR^m}\sup_{x\in \eR^n}| \varphi(x,y) |dy=\int_{\| y \|\leq r}\sup_{x}| \varphi(x,y) |dy+\int_{ \| y \|>r  }\sup_{x}| \varphi(x,y) |dy.
    \end{equation}
    La première intégrale est bornée par \( \Vol\big( B(0,r) \big)\| f \|_{\infty}\) tandis que la seconde est bornée par l'intégrale de \( \frac{1}{ Q(y) }\). En prenant \( Q\) de degré suffisamment élevé en toutes les composantes de \( y\) nous avons intégrabilité.
\end{proof}

%---------------------------------------------------------------------------------------------------------------------------
\subsection{Topologie}
%---------------------------------------------------------------------------------------------------------------------------

\begin{lemmaDef}        \label{LEMDEFooZEFVooMMmiBr}
    Les \( p_{\alpha,\beta}\) donnés par l'équation \eqref{EqOWdChCu} ci-dessus sont des semi-normes\footnote{Définition~\ref{DefPNXlwmi}.}. La topologie considérée sur \( \swS(\eR^d)\) est celle des semi-normes \( p_{\alpha,\beta}\).
\end{lemmaDef}
%TODO : une preuve pour égayer la galerie.

\begin{normaltext}      \label{NORMooVQESooRwJShl}
Nous avons un enchaînement de résultats qui nous aident à prouver la continuité d'une application \( T\colon \swS(\eR^d)\to X\).
\begin{enumerate}
    \item
        La topologie de \( \swS(\eR^d)\) est donnée par une famille dénombrable de semi-normes. Donc la proposition~\ref{PROPooMJEQooHtIyeX} nous dit que \( \swS(\eR^d)\) est métrisable.
    \item
        La proposition~\ref{PROPooKNVUooMbLZoy} nous dit alors que si \( X\) est métrique, toute application séquentiellement continue \( T\colon \swS(\eR^d)\to X\) est continue.
    \item
        Donc si \( X\) est métrique, il suffit de prouver que pour \( f_n\stackrel{\swS(\eR^d)}{\longrightarrow}0\) nous avons \( T(f_n)\stackrel{X}{\longrightarrow} 0\) où \( f_n\colon \swS(\eR^d)\to X\). Dans les cas usuels, \( T\) sera une distribution et \( X=\eC\).
    \item
        En vertu de la proposition~\ref{PropQPzGKVk}, la convergence \( f_n\stackrel{\swS(\eR^d)}{\longrightarrow}0\) signifie que pour tout choix de multiindice \( \alpha\) et \( \beta\),  \( p_{\alpha,\beta}(f_n)\to 0\), c'est-à-dire
        \begin{equation}        \label{EQooPUJPooNbtNFh}
            \| x^{\beta}\partial^{\alpha}f_n \|_{\infty}\to 0.
        \end{equation}
    \item
        Et enfin, la technique pour montrer que \( T\colon \swS(\eR^d)\to \eC\) est continue est de montrer que sous l'hypothèse d'avoir \eqref{EQooPUJPooNbtNFh} pour tout choix de \( \alpha\) et \( \beta\), nous avons \( T(f_n)\to 0\) dans \( \eC\).
\end{enumerate}
\end{normaltext}

\begin{lemma}[\cite{OEVAuEz}]   \label{LemRJhCbkO}
    La topologie sur \( \swS(\eR^d)\) est donnée aussi par les semi-normes
    \begin{equation}
        q_{n,m}=\max_{| \alpha |\leq n}\sup_{x\in \eR^d}\big( 1+\| x \| \big)^m\big| \partial^{\alpha}\varphi(x) \big|.
    \end{equation}
    Autrement dit, une suite \( \varphi_n\stackrel{\swS(\eR^d)}{\to}0\) si et seulement si \( q_{n,m(\varphi)}\to 0\) pour tout \( n\) et \( m\).
\end{lemma}
Le fait que les \( q_{n,m}(\varphi)\) restent bornés est la proposition~\ref{PropCSmzwGv}. Cependant ce lemme est plus précis parce qu'en disant seulement que \( \varphi\) est majoré par des polynôme, nous ne disons pas que les polynômes correspondants aux \( \varphi_n\) tendent vers zéro si \( \varphi_n\stackrel{\swS}{\to}0\). Et d'ailleurs on ne sait pas très bien ce que signifierait \( P_n\to 0\) pour une suite de polynômes.

\begin{proposition}     \label{PropGNXBeME}
    Pour \( p\in\mathopen[ 1 , \infty \mathclose]\), l'espace \( \swS(\eR^d)\) s'injecte continument dans \( L^p(\eR^d)\).
\end{proposition}

\begin{proof}
    L'injection dont nous parlons est l'identité ou plus précisément l'identité suivie de la prise de classe. Il faut vérifier que cela est correct et continu, c'est-à-dire d'abord qu'une fonction à décroissance rapide est bien dans \( L^p\) et ensuite que si \( f_n\stackrel{\swS}{\to}0\), alors \( f_n\stackrel{L^p}{\to}0\).

    Commençons par \( p=\infty\). Alors \( \| f_n \|_{\infty}=p_{0,0}(f_n)\to 0\) parce que si \( f_n\stackrel{\swS}{\to}0\), alors en particulier \( p_{0,0}(f_n)\to 0\).

    Au tour de \( p<\infty\) maintenant. Nous savons qu'en dimension \( d\), la fonction
    \begin{equation}
        x\mapsto \frac{1}{ (1+\| x \|)^s }
    \end{equation}
    est intégrable dès que \( s>d\).
    %TODO : il faudrait une petite preuve de ça.
    Pour toute valeur de \( m\) nous avons
    \begin{equation}
        \| \varphi \|_p^p=\int_{\eR^d}| \varphi(x) |^pdx=\int_{\eR^d}\frac{ \big|    (1+\| x \|)^m\varphi(x)   \big|^p }{ \big( 1+\| x \| \big)^{mp} }\leq\int_{\eR^d}\frac{q_{0,m}(\varphi)^p}{ \big( 1+\| x \| \big)^{mp} }.
    \end{equation}
    En choisissant \( m\) de telle sorte que \( mp>d\), nous avons convergence de l'intégrale et donc \( \| \varphi \|_p<\infty\). Nous retenons que
    \begin{equation}    \label{EqVWfEFMk}
        \| \varphi \|_p^p\leq Cq_{0,m}(\varphi)^p
    \end{equation}
    pour une certaine constance \( C\) et un bon choix de \( m\).

    Ceci prouve que \( \swS(\eR^d)\subset L^p(\eR^d)\). Nous devons encore vérifier que l'inclusion est continue. Si \( \varphi_n\stackrel{\swS}{\to}0\), alors en particulier nous avons \( q_{0,m}(\varphi_n)\to 0\) par le lemme~\ref{LemRJhCbkO}. Par conséquent la majoration \eqref{EqVWfEFMk} nous dit que \( \| \varphi_n \|_p\to 0\) également.

\end{proof}
En résumé, si \( \varphi_n\stackrel{\swS(\eR^d)}{\to}\varphi\) alors \( \varphi_n\stackrel{L^p}{\to}\varphi\).

\begin{theorem}[\cite{MesIntProbb}]      \label{ThoRWEoqY}
    Soit \( \mu\) une mesure sur les boréliens de \( \eR^n\) finie sur les compacts. Alors \( \swD(\eR^n)\) est dense dans \( L^1(\eR^n,\Borelien(\eR^n),\mu)\).
\end{theorem}
\index{densité!de \( \swD(\eR^n)\) dans \( L^1(\eR^n)\)}

\begin{proposition}[\cite{ooIKXSooRlKVJR}]      \label{PROPooJNQZooIRbJei}
    La partie \( \swD(\eR^d)\) est dense dans \( \swS(\eR^d)\).
\end{proposition}

\begin{proof}
    Soit \( f\in \swS(\eR^d)\), et \( \phi\), une fonction de \( \swD(\eR^d)\) telle que \( \phi(x)=1\) pour \(| x |\leq 1 \) (l'existence de telles fonctions est discutée en~\ref{subsecOSYAooXXCVjv}). Soit aussi \( \phi_k(x)=\phi(x/k)\). Nous posons
    \begin{equation}
        f_k(x)=\phi_k(x)f(x),
    \end{equation}
    et nous allons prouver que pour tout multiindices \( \alpha\) et \( \gamma\),
    \begin{equation}
        p_{\alpha,\gamma}(f_k-f)=\| x^{\gamma}\partial^{\alpha}(f_k-f)  \|_{\infty}\to 0.
    \end{equation}
    Pour cela nous allons noter \(  \beta\leq \alpha  \) lorsque \( \beta\) est un multiindice contenu dans \( \alpha\). En utilisant la dérivée du produit nous avons
    \begin{subequations}
        \begin{align}
            (\partial^{\alpha}f_k)(x)&=\sum_{\beta\leq \alpha}(\partial^{\alpha-\beta}\phi_k)(x)\partial^{\beta}f(x)\\
            &=\sum_{\beta\leq \alpha}k^{-| \alpha-\beta |}(\partial^{\alpha-\beta}\phi)(x/k)(\partial^{\beta}f)(x)\\
            &=\sum_{\beta< \alpha}k^{-| \alpha-\beta |}(\partial^{\alpha-\beta}\phi)(x/k)(\partial^{\beta}f)(x) + \phi(x/k)(\partial^{\alpha}f)(x).
        \end{align}
    \end{subequations}
    Nous devons donc étudier et majorer
    \begin{equation}
        \begin{aligned}[]
        \sup_{x\in \eR^d}| x^{\gamma}\partial^{\alpha}(f_k-f) |&\leq \sup\big| x^{\gamma}  \sum_{\beta< \alpha}k^{-| \alpha-\beta |}(\partial^{\alpha-\beta}\phi)(x/k)(\partial^{\beta}f)(x)  \big|\\
        &\quad+\sup \big| x^{\gamma}\big( \phi(x/k)-1 \big)(\partial^{\alpha}f)(x) \big|\\
        \end{aligned}
    \end{equation}
    En ce qui concerne le second terme, soit \( \epsilon>0\), vu que \( f\) est Schwartz, il existe \( R\) tel que
    \begin{equation}
        | x^{\gamma}(\partial^{\alpha}f)(x) |<\epsilon
    \end{equation}
    dès que \( \| x \|>R\). En prenant \( k>R\),
    \begin{equation}
        | x^{\gamma}(\partial^{\alpha}f)(x) |\begin{cases}
            =0    &   \text{si } \| x \|<R\\
            \leq \epsilon    &    \text{si } \| x \|>R\text{.}
        \end{cases}
    \end{equation}
    En ce qui concerne le premier terme,
    \begin{subequations}
        \begin{align}
            \sup_{x\in \eR^d}\big| x^{\gamma}&\sum_{\beta<\alpha}k^{-|\alpha-\beta |}(\partial^{\alpha-\beta}\phi)(x/k)(\partial^{\beta}f)(x) \Big|\\
            &\leq \frac{1}{ k }\sup_{x}\big| \sum_{\beta<\alpha}(\partial^{\alpha-\beta}\phi)(x/k)(x^{\gamma}\partial^{\beta}f)(x) \big|\\
            &= \frac{1}{ k }\sup_{x}\big| \sum_{\beta<\alpha}(\partial^{\alpha-\beta}\phi)(x/k)  p_{\beta,\gamma}(f)   \big|
        \end{align}
    \end{subequations}
    La somme ne contient qu'un nombre fini de \( \beta\) différents, donc nous pouvons considérer un nombre \( K\) qui majore tous les \( p_{\beta,\gamma}(f)\) en même temps. La partie avec \( \phi\) peut être majorée par \( \| \partial^{\alpha-\beta}\phi \|_{\infty}\) (qui est fini) dont nous pouvons prendre le maximum sur \(\beta<\alpha\). Toute l'expression dans la somme est donc majorée par un nombre qui ne dépend ni de \( x\) ni de \( \beta\). Vu que la somme est finie, elle est majorée par ce nombre multiplié par le nombre de termes dans la somme et au final
    \begin{equation}
        \sup_{x\in \eR^d}\big| x^{\gamma}\sum_{\beta<\alpha}k^{-|\alpha-\beta |}(\partial^{\alpha-\beta}\phi)(x/k)(\partial^{\beta}f)(x) \Big|\leq \frac{ K' }{ k }.
    \end{equation}
    La limite \( k\to \infty\) ne fait alors plus de doutes.
\end{proof}

\begin{remark}
    Vu la topologie de \( \swS(\eR^d)\) (définition~\ref{LEMDEFooZEFVooMMmiBr}), la convergence \( f_k\stackrel{\swS(\eR^d)}{\longrightarrow}f\) peut être exprimée par le fait que pour tout \( k,l\),
    \begin{equation}
        t^kf_n^{(l)}\stackrel{unif}{\longrightarrow}t^kf^{(l)}.
    \end{equation}
    C'est-à-dire convergence uniforme de toutes les dérivées multipliées par n'importe quel polynôme.
\end{remark}

%---------------------------------------------------------------------------------------------------------------------------
\subsection{Produit de convolution}
%---------------------------------------------------------------------------------------------------------------------------

\begin{proposition}[Stabilité de Schwartz par convolution\footnote{Définition~\ref{DEFooHHCMooHzfStu}.} \cite{CXCQJIt}]     \label{PROPooUNFYooYdbSbJ}
    Si \( \varphi\in L^1(\eR^d)\) et \( \psi\in\swS(\eR^d)\), alors \( \varphi * \psi\in \swS(\eR^d)\).
\end{proposition}

\begin{proof}
    Nous devons prouver que
    \begin{equation}
        p_{\alpha,\beta}(\varphi*\psi)=\sup_{x\in \eR^d}| x^{\beta}(\partial^{\alpha}(\varphi*\psi))(x) |
    \end{equation}
    est borné pour tout multiindices \( \alpha\) et \( \beta\). En appliquant \( | \alpha |\) fois la proposition~\ref{PropHNbdMQe}, nous mettons toutes les dérivées sur \( \psi\) : \( \partial^{\alpha}(\varphi*\psi)=(\varphi*\partial^{\alpha}\psi)\). Cela étant fait, nous majorons
    \begin{subequations}
        \begin{align}
            \big| x^{\beta}(\varphi*\partial^{\alpha}\psi)(x) \big|&\leq| x^{\beta} |\int_{\eR^d} |\varphi(y)|\underbrace{\big| (\partial^{\alpha}\psi)(x-y)\big|}_{\leq\| \partial^{\alpha}\psi \|_{\infty}} dy \big|\\
            &\leq | x^{\beta} |  \| \partial^{\alpha}\psi \|_{\infty}\int_{\eR^d}| \varphi(y) |dy\\
            &\leq p_{\alpha,\beta}(\psi)\| \varphi \|_{_{L^1}}.
        \end{align}
    \end{subequations}
    Par conséquent, \( p_{\alpha,\beta}(\varphi*\psi)\leq \| \varphi \|_{L^1}p_{\alpha,\beta}(\psi)<\infty\).
\end{proof}

%+++++++++++++++++++++++++++++++++++++++++++++++++++++++++++++++++++++++++++++++++++++++++++++++++++++++++++++++++++++++++++
\section{Théorème de Montel}
%+++++++++++++++++++++++++++++++++++++++++++++++++++++++++++++++++++++++++++++++++++++++++++++++++++++++++++++++++++++++++++

\begin{theorem}[Montel\cite{KXjFWKA}]   \label{ThoXLyCzol}
    Soient \( \Omega\) un ouvert de \( \eC\) et \( \mF\) une famille de fonctions holomorphes sur \( \Omega\), uniformément bornée sur tout compact de \( \Omega\). Alors de toute suite dans \( \mF\) nous pouvons extraire une sous-suite convergeant uniformément sur tout compact de \( \Omega\).
\end{theorem}
\index{théorème!Montel}
\index{compacité!utilisation!théorème de Montel}
\index{suite!de fonctions!théorème de Montel}
\index{fonction!holomorphe!théorème de Montel}

\begin{proof}

    \begin{subproof}
    \item[Un ensemble équicontinu]

        Nous commençons par prendre une suite de compacts dans \( \Omega\) comme dans le lemme~\ref{LemGDeZlOo}, et une suite \( \delta_n\) de réels strictement positifs tels que
        \begin{equation}
            B(z,2\delta_n)\subset K_{n+1}
        \end{equation}
        pour tout \( z\in K_n\). Soient \( x,y\in K_n\) tels que \( | x-y |<\delta_n\); nous notons \( \partial B(x,2\delta_n)\) le cercle de rayon \( 2\delta_n\) autour de \( x\), parcouru dans le sens positif. La formule de Cauchy~\ref{EqPzUABM} nous donne
        \begin{equation}
                f(x)-f(y)=\frac{1}{ 2\pi i }\int_{\partial B}\left( \frac{ f(\xi) }{ \xi-x }-\frac{ f(\xi) }{ \xi-y } \right)d\xi
                =\frac{ x-y }{ 2\pi i }\int_{\partial B}\frac{ f(\xi) }{ (\xi-x)(\xi-y) }d\xi
        \end{equation}
        Nous majorons ça par
        \begin{equation}
            \big| f(x)-f(y) \big|\leq\frac{ | x-y | }{ 2\pi }\int_{\partial B}\frac{ | f(\xi) | }{ 2\delta_n^2 }d\xi\leq \frac{ | x-y | }{ \delta_n }M_n.
        \end{equation}
        Justifications :
        \begin{itemize}
            \item
                \( | \xi-x |=2\delta_n\) et \( | \xi-y |\geq \delta_n\) parce que \( \xi\) est au mieux sur le rayon passant par \( x\) et \( y\).
            \item
                \( | f(\xi) |\leq M_n\) où \( M_n\) est la borne uniforme de \( \mF\) sur le compact \( K_n\).
            \item
                Nous avons aussi fini par calculer l'intégrale dans laquelle il ne restait plus rien, ça a donné la circonférence du cercle de rayon \( 2\delta_n\).
        \end{itemize}
        Jusqu'à présent nous avons prouvé que l'ensemble
        \begin{equation}
            \mF_n=\{ f|_{K_n}\tq f\in\mF \}
        \end{equation}
        est équicontinu. Il est aussi équiborné par hypothèse.

    \item[Application du théorème d'Ascoli]

        L'ensemble \( \mF_n\) vérifie les hypothèses du théorème d'Ascoli~\ref{ThoKRbtpah}. Donc l'ensemble \( \mF_n\) est relativement compact dans \( C(K_n,\eC)\) pour la norme uniforme. Autrement dit l'ensemble \( \bar\mF\) est compact et si nous avons une suite de fonctions dans \( \mF_n\), il existe une sous-suite convergeant dans \( \bar\mF_n\), c'est-à-dire uniformément. Autrement dit il existe une fonction strictement croissante \( \varphi\colon \eN\to \eN\) telle que la suite \( k\mapsto f_{\varphi(k)}\) converge uniformément sur \( K_n\). La limite n'est cependant pas spécialement dans \( \mF_n\).

    \item[L'argument diagonal]

        La suite \( k\mapsto f_{\varphi_1\circ\ldots\varphi_k(k)}\) converge uniformément sur tous les \( K_n\). Si \( K\) est un compact de \( \Omega\), alors les petites propriétés sympas du lemme~\ref{LemGDeZlOo} nous disent que \( K\subset \Int(K_m)\) pour un certain \( m\). Ladite suite convergeant uniformément sur \( K_m\), elle converge uniformément sur \( K\) et nous avons montré la convergence uniforme sur tout compact de \( \Omega\).

    \end{subproof}
\end{proof}

\begin{corollary}[\cite{KXjFWKA}]
    Soient \( \Omega\) un ouvert connexe borné de \( \eC\) et \( a\in \Omega\). Soit \( f\) holomorphe sur \( \Omega\) telle que \( f(a)=a\) et \( | f'(a) |<1\).

    Alors de \( (f^n)\) on peut extraire une sous-suite convergeant uniformément sur tout compact de \( \Omega\) vers la fonction constante \( a\).
\end{corollary}
\index{prolongement!analytique!utilisation}

\begin{proof}
    Nous considérons un voisinage de \( a\) inclus dans \( \Omega\); sachant que \( | f(a) |<1\), nous trouvons un voisinage encore plus petit de \( a\) sur lequel \( | f'(z) |<1\).  Soit donc \( r\) tel que \( \overline{ B(a,r) }\subset \Omega\) et tel que \( | f'(z) |<1\) sur \( \overline{ B(a,r) }\). Étant donné que \( f'(z)\) est continue sur le compact \( \overline{ B(a,r) }\), nous en prenons le maximum \( \lambda\) (qui est strictement inférieur à \( 1\)) et nous avons au final
    \begin{equation}
        | f'(z) |\leq \lambda< 1
    \end{equation}
    pour tout \( z\in \overline{ B(a,r) }\). Le théorème des accroissements finis~\ref{val_medio_2} nous dit que
    \begin{equation}
        \big| f(z)-a \big|\leq \lambda| z-a |
    \end{equation}
    pour tout \( z\in\overline{ B(a,r) }\). C'est ici que nous utilisons l'hypothèse de convexité de \( \Omega\). Nous montrons alors par récurrence que
    \begin{equation}    \label{EqIQUzKpg}
        \big| f^n(z)-a \big|\leq \lambda^n| z-a |\leq \lambda^nr\leq r.
    \end{equation}
    L'ensemble \( A=\{ f^n\tq n\geq 1 \}\) est donc uniformément borné sur \( \overline{ B(a,r) }\) par \( a+r\). Autre manière de le dire : pour tout \( z\in\overline{ B(a,r) }\) nous avons
    \begin{equation}
        f^n(z)\in\overline{ B(a,r) }.
    \end{equation}
    La suite \( (f^n)\) est donc uniformément bornée sur tout compact de \( B(a,r)\). Le théorème de Montel~\ref{ThoXLyCzol} nous indique que l'on peut extraire une sous-suite convergente uniformément sur tout compact. Au vu de \eqref{EqIQUzKpg} cette convergence ne peut avoir lieu que vers une fonction \( g\) qui vaut la constante \( a\) sur \( B(a,r)\).

    D'autre par la fonction \( g\) est holomorphe en tant que limite uniforme de fonctions holomorphes, théorème~\ref{ThoArYtQO}. Or une fonction holomorphe constante sur un ouvert est constante sur tout son domaine d'holomorphie (principe d'extension analytique, théorème~\ref{ThoAVBCewB}).
\end{proof}


%+++++++++++++++++++++++++++++++++++++++++++++++++++++++++++++++++++++++++++++++++++++++++++++++++++++++++++++++++++++++++++
\section{Espaces de Bergman}
%+++++++++++++++++++++++++++++++++++++++++++++++++++++++++++++++++++++++++++++++++++++++++++++++++++++++++++++++++++++++++++

Source : \cite{ytMOpe}.

Soit \( \Omega\) un borné dans \( \eC\) et \( D\) le disque unité ouvert de \( \eC\).

\begin{definition}
    L'\defe{espace de Bergman}{espace!de Bergman}\index{Bergman (espace)} sur \( \Omega\), noté \( A^2(\Omega)\)\nomenclature[Y]{\( A^2(\Omega)\)}{espace de Bergman} est l'espace des fonctions holomorphes sur \( \Omega\) qui sont en même temps dans \( L^2(\Omega)\).
\end{definition}
Nous mettons sur \( A^2(\Omega)\) le produit scalaire usuel hérité de \( L^2\) :
\begin{equation}
    \langle f, g\rangle =\int_{\Omega}f(z)\overline{ g(z) }dz.
\end{equation}

\begin{lemma}   \label{LemIZxKfB}
    Soient un compact \( K\subset \Omega\) et une fonction \( f\in A^2(\Omega)\). Alors
    \begin{equation}
        \max_{z\in K}| f(z) |\leq \frac{1}{ \sqrt{\pi} }\frac{1}{ d(K,\partial \Omega) }\| f \|_2.
    \end{equation}
\end{lemma}

\begin{proof}
    Soient \( a\in \Omega\) et \( r>0\) tels que \( B(a,r)\subset\Omega\). Nous considérons aussi \( \rho\leq r\). La formule de Cauchy \eqref{EqPzUABM} nous donne
    \begin{equation}
        f(a)=\frac{1}{ 2\pi i }\int_{B(a,\rho)}\frac{ f(\xi) }{ \xi-a }f\xi=\frac{1}{ 2\pi }\int_0^{2\pi}f(a+\rho e^{i\theta})d\theta
    \end{equation}
    où nous avons utilisé le chemin \( \gamma(\theta)=a+\rho e^{i\theta}\), \( \gamma'(\theta)=i\rho e^{i\theta}\) et \( \rho=| \xi-a |\). Maintenant une astuce est d'écrire
    \begin{equation}
        \frac{ r^2 }{2}f(a)=\int_0^rf(a)\rho d\rho,
    \end{equation}
    et d'y substituer la valeur de \( f(a)\) que nous venons de calculer :
    \begin{subequations}
        \begin{align}
            \frac{ r^2 }{2}f(a)&=\int_0^r\frac{1}{ 2\pi }\int_0^{2\pi}f(a+\rho e^{i\theta})d\theta\rho d\rho\\
            &=\frac{1}{ 2\pi }\int_{B(a,r)}f(z)dz   &   \text{passage aux polaires}\\
            &=\frac{1}{ 2\pi }\langle 1, f\rangle_B   &   \text{produit scalaire sur } B(a,r)\\
            &\leq\frac{1}{ 2\pi }\sqrt{\langle 1, 1\rangle_B\langle f, f\rangle_B }
        \end{align}
    \end{subequations}
    Nous avons donc
    \begin{equation}
        r^2f(a)\leq \frac{1}{ \pi }\sqrt{\langle 1, 1\rangle_B\langle f, f\rangle_B},
    \end{equation}
    et donc
    \begin{equation}
        \pi r^2 f(a)\leq \sqrt{\pi r^2}\| f \|_2,
    \end{equation}
    parce que \( \langle f, f\rangle_B\leq \| f \|_2^2\). En effet le produit scalaire \( \| . \|_2\) est donné par une intégrale sur \( \Omega\) alors que \( B(a,r)\subset \Omega\) et que la fonction qu'on y intègre est positive (c'est \( | f(z) |^2\)). En simplifiant,
    \begin{equation}
        f(a)\leq \frac{1}{ \sqrt{\pi}r }\| f \|_2.
    \end{equation}
    Mais \( r\) a été choisi pour avoir \( B(a,r)\subset\Omega\), donc \( r\leq d(a,\partial \Omega)\) et
    \begin{equation}
        | f(a) |\leq \frac{1}{ d(a,\partial\Omega)\sqrt{\pi} }\| f \|_2.
    \end{equation}

    Maintenant si nous prenons \( a\in K\), nous avons encore la minoration \( d(a,\partial K)\leq d(a,\partial \Omega)\) et donc
    \begin{equation}
        | f(a) |\leq\frac{1}{ d(a,\partial K)\sqrt{\pi} }\| f \|_2.
    \end{equation}

\end{proof}

\begin{theorem}
    Soit \( \Omega\) un ouvert de \( \eC\).
    \begin{enumerate}
        \item
            L'espace \( A^2(\Omega)\) est un espace de Hilbert.
        \item
            Si \( D\) est la boule unité dans \( \eC\), une base hilbertienne de \( A^2(D)\) est donnée par les fonctions
            \begin{equation}
                e_n(z)=\sqrt{\frac{ n+1 }{ \pi }}z^n
            \end{equation}
            pour \( n\geq 0\).
    \end{enumerate}
\end{theorem}

\begin{proof}
    Nous commençons par montrer que \( A^2(\Omega)\) est complet. Pour cela nous considérons une suite de Cauchy \( (f_n)\) dans \( A^2(\Omega)\) et un compact \( K\subset \Omega\). Nous savons par le lemme~\ref{LemIZxKfB} que
    \begin{equation}
        \max_{z\in K}\big| f_n(z)-f_m(z) \big|\leq \frac{1}{ \sqrt{\pi}d(K,\partial\Omega) }\| f_n-f_m \|_2.
    \end{equation}
    Donc \( f_n\) converge uniformément sur \( K\). Par le théorème de Weierstrass~\ref{ThoArYtQO}, la fonction \( f\) est holomorphe. Il existe donc une fonction holomorphe \( f\) qui est limite uniforme sur tout compact de \( \Omega\) de la suite \( (f_n)\).

    Mais \( L^2(\Omega)\) étant complet, la suite \( (f_n)\) a une limite \( g\in L^2(\Omega)\). Ce que nous voudrions faire est prouver que \( f=g\). Notons que tel quel, ce n'est pas vrai parce que \( f\) est une vraie fonction alors que \( g\) est une classe. Ce que nous enseigne la proposition~\ref{PropWoywYG} est qu'il existe une sous-suite (qu'on note \( (g_n)\)) qui converge vers \( g\) presque partout. Dans cette dernière phrase, \( g_n\) et \( g\) sont de vraies fonctions, des représentants des classes dans \( L^2\).

    Nous déduisons que \( f=g\) presque partout (ici \( f\) et \( g\) sont les fonctions) parce que la sous-suite converge uniformément vers \( f\) en même temps que presque partout vers \( g\). Donc \( f=g\) dans \( L^2(\Omega)\) (ici \( f\) et \( g\) sont les classes). Donc \( f\in L^2(\Omega)\) et l'espace \( A^2(\Omega)\) est de Hilbert.

    Il nous faut encore prouver que \( (e_n)_{n\geq 0}\) est une base orthonormale. En ce qui concerne les produits scalaires,
    \begin{subequations}
        \begin{align}
            \langle e_m, e_n\rangle &=\sqrt{\frac{ (m+1)(n+1) }{ \pi }}\int_Dz^n\overline{ z^m }dz\\
            &=\sqrt{\frac{ (m+1)(n+1) }{ \pi^2 }}\int_0^1\rho\,d\rho\int_0^{2\pi}d\theta \rho^{m+n} e^{i\theta(n-m)}\\
            &=\sqrt{\frac{ (m+1)(n+1) }{ \pi^2 }}\frac{1}{ m+n+2 }\underbrace{\int_{0}^{2\pi} e^{i\theta(n-m)}d\theta}_{2\pi \delta_{mn}}\\
            &=\sqrt{\frac{ (n+1)^2 }{ \pi^2 }}\frac{1}{ 2n+2 }2\pi \delta_{nm}\\
            &=\delta_{nm}.
        \end{align}
    \end{subequations}
    Donc les fonctions données sont bien orthonormales. Nous devons montrer qu'elles sont denses dans \( A^2(D)\). Soit \( f\in A^2(D)\) et \( c_n(f)=\langle f, e_n\rangle \); nous allons montrer que
    \begin{equation}
        \| f \|_2^2=\sum_{n=0}^{\infty}| \langle f, e_n\rangle  |^2,
    \end{equation}
    parce que le point~\ref{ItemQGwoIx} du théorème~\ref{ThoyAjoqP} nous indique que ce sera suffisant pour avoir une base hilbertienne.

    Étant donné que \( f\) est holomorphe sur \( D\), le théorème~\ref{ThoUHztQe} nous développe \( f\) en série entière :
    \begin{equation}    \label{EqObkbPK}
        f(z)=\sum_{k=0}^{\infty}a_kz^k.
    \end{equation}
    En permutant la somme avec le produit scalaire,
    \begin{equation}
        c_n(f)=\int_Df(z)\bar e_n(z)=\sqrt{\frac{ n+1 }{ \pi }}\int_Df(z)\bar z^ndz.
    \end{equation}
    Afin de profiter de la convergence uniforme de la série \eqref{EqObkbPK} à l'intérieur de \( D\), nous allons exprimer l'intégrale sur \( D\) comme une intégrale sur \( | z |<r\) en faisant tendre \( r\) vers \( 1\) (par le bas). Pour ce faire nous considérons les fonctions
    \begin{equation}
        g_k(z)=\begin{cases}
            f(z)\bar z^n    &   \text{si } | z |<1-1/k\\
            0    &    \text{sinon.}
        \end{cases}
    \end{equation}
    Ces fonctions sont intégrables sur \( D\) et dominées par \( f(z)\bar z^n\) qui est intégrable sans dépendre de \( k\). Mais nous avons évidemment \( g_k(z)\to f(z)\bar z^n\). Le théorème de la convergence dominée permet alors de permuter l'intégrale et la limite \( k\to \infty\). Cela nous permet d'écrire
    \begin{equation}
        c_n(f)=\sqrt{\frac{ n+1 }{ \pi }}\lim_{r\to 1^-}\int_{| z |<r}\bar z^nf(z)dz=\sqrt{\frac{ n+1 }{ \pi }}\lim_{r\to 1^-}\int_{| z |<r}\sum_{k=0}^{\infty}a_kz^k\bar z^n.
    \end{equation}
    Par la convergence uniforme de la série entière \emph{à l'intérieur} du disque \( D\) nous pouvons permuter l'intégrale et la somme (proposition~\ref{PropfeFQWr}) :
    \begin{equation}
        c_n(f)=\sqrt{\frac{ n+1 }{ \pi }}\lim_{r\to 1^-}\sum_{k=0}^{\infty}a_k\int_{| z |<r}z^k\bar z^ndz.
    \end{equation}
    L'intégrale proprement dite est vite calculée et vaut
    \begin{equation}
        \int_{| z |<1}\bar z^nz^kdz=\frac{ \pi r^{2n+2} }{ n+1 }\delta_{kn}.
    \end{equation}
    Nous pouvons donc continuer le calcul de \( c_n(f)\) en effectuant la somme sur \( k\) qui se réduit à changer \( k\) en \( n\) puis en effectuant la limite :
    \begin{equation}
        c_n(f)=\sqrt{\frac{ n+1 }{ \pi }}\lim_{r\to 1^-}\sum_ka_k\frac{ \pi r^{2n+2} }{ n+1 }\delta_{kn}=\sqrt{\frac{ \pi }{ n+1 }}a_n.
    \end{equation}

    Nous effectuons le même genre de calculs pour évaluer \( \| f \|^2_2\) :
    \begin{subequations}
        \begin{align}
            \| f \|_2^2&=\int_D| f(z) |^2dz\\
            &=\lim_{r\to 1^-}\int_{| z |<r}f(z)\sum_{k=0}^{\infty}\bar a_k\bar z_kdz\\
            &=\lim_{r\to 1^-}\sum_{k=0}^{\infty}\bar a_k\int_{| z |<r}f(z)\bar z^kdz&\text{permuter } \sum\text{ et } \int\\
            &=\lim_{r\to 1^-}\sum_{k=0}^{\infty}\bar a_ka_k\frac{ \pi r^{2k+2} }{ k+1 }&\text{intégrale déjà faite}.
        \end{align}
    \end{subequations}
    Mais nous savons déjà que \( c_n(f)=\sqrt{\pi/(n+1)}\), donc ce qui est dans la somme est \( \pi\bar a_ka_k/(n+1)=| c_k(f) |^2\). Nous avons donc
    \begin{equation}
        \| f \|^2_2=\lim_{r\to 1^-}\sum_{k=0}^{\infty}| c_k(f) |^2 r^{2k+2}.
    \end{equation}
    La fonction (de \( r\)) constante \( | c_k(f) |^2\) domine \( | c_k(f)r^{2k+2} |\) tout en ayant une somme (sur \( k\)) qui converge; en effet la proposition~\ref{PropHKqVHj} nous indique que \( \sum_j| c_k(f) |^2\leq \| f \|_2^2\). Le théorème de la convergence dominée nous permet d'inverser la limite et la somme pour obtenir le résultat attendu :
    \begin{equation}
        \| f \|_2^2=\sum_{k=0}^{\infty}| c_k(f) |^2.
    \end{equation}
\end{proof}
