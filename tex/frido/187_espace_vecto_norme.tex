% This is part of Le Frido
% Copyright (c) 2008-2018
%   Laurent Claessens
% See the file fdl-1.3.txt for copying conditions.


%+++++++++++++++++++++++++++++++++++++++++++++++++++++++++++++++++++++++++++++++++++++++++++++++++++++++++++++++++++++++++++
\section{Séries}
%+++++++++++++++++++++++++++++++++++++++++++++++++++++++++++++++++++++++++++++++++++++++++++++++++++++++++++++++++++++++++++
\label{SECooYCQBooSZNXhd}

\begin{definition}\label{DefGFHAaOL}
    Soit \( (a_k)\) une suite dans un espace vectoriel normé \( (V,\| . \| )\). La suite des \defe{sommes partielles}{somme!partielle} associée est la suite \( (s_k)\) définie par
    \begin{equation}
        s_k=\sum_{i=0}^ka_i
    \end{equation}
    La \defe{série}{série!dans un espace vectoriel normé} associée est la limite des sommes partielles
    \begin{equation}
        \sum_{n=0}^{\infty}a_k=\lim_{k\to \infty} \sum_{k=0}^na_k
    \end{equation}
    si elle existe.

    Si une telle limite existe nous disons que \( \sum_{k=0}^{\infty}a_k\) \defe{converge}{série convergente} dans \( V\). Si la limite de la suite des sommes partielles n'existe pas nous disons que la série \defe{diverge}{série divergente}.
\end{definition}

\begin{remark}
    Si la limite de la suite des sommes partielles n'existe pas dans \( V\), alors elle peut parfois exister dans des extensions de \( V\). Par exemple une série de rationnels convergeant vers \( \sqrt{2}\) dans \( \eR\) ne converge pas dans \( \eQ\). Autre exemple : avec une bonne topologie sur \( \bar \eR\), une série peut ne pas converger dans \( \eR\) mais converger vers \( \pm\infty\) dans \( \bar \eR\).
\end{remark}

Dans le cas des espaces de fonctions, nous avons une norme importante : la norme uniforme définie par \( \| f \|_{\infty}=\sup\{ f(x) \}\) où le supremum est pris sur l'ensemble de définition de \( f\).
\begin{definition}[Convergence absolue] \label{DefVFUIXwU}
    Nous disons que la série \( \sum_{n=0}^{\infty}a_n\) dans l'espace vectoriel normé \( V\) \defe{converge absolument}{convergence absolue} si la série \( \sum_{n=0}^{\infty}\| a_n \|\) converge dans \( \eR\).
\end{definition}

La convergence absolue dans le cas d'un espace de fonctions muni de la norme uniforme s'appelle la convergence normale.

\begin{definition}[Convergence normale] \label{DefVBrJUxo}
    Une série de fonctions \( \sum_{n\in \eN}u_n \) converge \defe{normalement}{convergence normale} si la série de nombres \( \sum_n\| u_n \|_{\infty}\) converge. C'est à dire si la série converge absolument pour la norme \( \| f \|_{\infty}\).
\end{definition}

La convergence normale est à ne pas confondre avec la convergence uniforme.

\begin{definition}[Convergence uniforme]
    La somme \( \sum_nf_n\) \defe{converge uniformément}{convergence uniforme!série de fonctions} vers la fonction \( F\) si la suite des sommes partielles converge uniformément, c'est à dire si
    \begin{equation}        \label{EqLNCJooVCTiIw}
        \lim_{N\to \infty} \| \sum_{n=1}^Nf_n-F \|_{\infty}=0.
    \end{equation}
\end{definition}

\begin{proposition} \label{PropAKCusNM}
    Une série convergeant absolument dans un espace de Banach\footnote{Un espace vectoriel normé complet. Typiquement \( \eR\).} y converge au sens usuel.
\end{proposition}

\begin{proof}
    Soit \( (a_k)\) une suite dans un espace vectoriel normé complet dont la série converge absolument. Nous allons montrer que la suite des sommes partielles est de Cauchy. Cela suffira à montrer sa convergence par hypothèse de complétude.

    Nous avons
    \begin{equation}
        \| s_p-s_l \|=\| \sum_{k=l+1}^{p}a_k\|  \leq\sum_{k=l+1}^p\| a_k \|=\bar s_p-\bar s_l
    \end{equation}
    où \( \bar s_n=\sum_{k=0}^n \| a_k \|\) est la suite des sommes partielles de la série des normes (qui converge). Vu que la suite \( (\bar s_n)\) converge dans \( \eR\), elle y est de Cauchy par la proposition~\ref{PROPooTFVOooFoSHPg}. Donc il existe un \( N\) tel que \( p,l>N\) implique
    \begin{equation}
        \| s_p-s_l \|=\bar s_p-\bar s_l\leq \epsilon.
    \end{equation}
    Cela signifie que \( (s_n)\) est une suite de Cauchy et donc convergente.
\end{proof}

\begin{example}[Si l'espace n'est pas complet\cite{MonCerveau}]
    Dans un espace pas complet, il est possible de construire un série qui converge absolument sans converger au sens usuel.

    Nous allons trouver dans \( \eQ\) une série qui converge simplement vers \( \sqrt{ 2 }\) (et donc ne converge pas dans \( \eQ\)) mais absolument vers \( 4\).

    La base est que si \( A,B\in \eQ\) avec \( A<B\) il est possible de résoudre
    \begin{subequations}
        \begin{numcases}{}
            r_1+r_2=A\\
            | r_1 |+| r_2 |=B
        \end{numcases}
    \end{subequations}
    pour \( r_1,r_2\in \eQ\). Ce n'est pas très compliqué : la solution est \( r_1=(A+B)/2\) et \( r_2=(A-B)/2\).

    Nous considérons l'espace \( \eQ\) qui n'est pas complet dans \( \eR\). Soit une série \( (a_k)\) dans \( \eQ\) qui converge vers \( \sqrt{ 2 }\) (convergence dans \( \eR\)) nous nommons \( (s_k)\) la suite des ses sommes partielles. Soit aussi la suite \( (b_k)\) qui converge vers \( 4\) (zéro serait encore plus facile mais bon, juste pour faire un peu de généralité).

    Nous supposons que \( a_k<b_k\) pour tout \( k\) et que les deux suites sont constituées de rationnels positifs. Nous nommons \( (s_k)\) et \( (s'_k)\) les sommes partielles. En particulier \( s_n<s'_n\) et ce sont des suites croissantes.

    Nous savons comment trouver \( r_1,r_2\in \eQ\) tels que \( r_1+r_2=s_1\) et \( | r_1 |+| r_2 |=s'_1\). Par récurrence, si nous savons \( r_1,\ldots, r_k\) tels que
    \begin{subequations}
        \begin{numcases}{}
            r_1+\ldots +r_k=s_n\\
            |r_1|+\ldots +|r_k|=s'_n
        \end{numcases}
    \end{subequations}
    (avec, soit dit en passant \( k=2n\)), alors nous pouvons trouver des rationnels \( r_{k+1}\), \( r_{k+2}\) tels que
    \begin{subequations}
        \begin{numcases}{}
            r_1+\ldots +r_k+r_{k+1}+r_{k+2}=s_{n+1}\\
            |r_1|+\ldots +|r_k|+|r_{k+1}|+|r_{k+2}|=s'_{n+1},
        \end{numcases}
    \end{subequations}
    en effet il s'agit de résoudre
    \begin{subequations}
        \begin{numcases}{}
            r_{k+1}+r_{k+2}=s_{n+1}-r_1-\ldots-r_k=s_{n+1}-s_n>0\\
            | r_{k+1} |+| r_{k+2} |=s'_{n+1}-| r_1 | -\ldots -| r_k |=s'_{n+1}-s'_n>0.
        \end{numcases}
    \end{subequations}
    Cela se résout comme plus haut. Au final nous pouvons construire une suite \( (r_k)\) dans \( \eQ\) telle que
    \begin{equation}
        \sum_{k=0}^{2n}r_k=s_n
    \end{equation}
    et
    \begin{equation}
        \sum_{k=0}^{2n}| r_k |=s'_n.
    \end{equation}
\end{example}

\begin{remark}
    Nous savons que sur les espaces vectoriels de dimension finie toutes les normes sont équivalentes (théorème~\ref{DefEquivNorm}). La notion de convergence de série ne dépend alors pas du choix de la norme. Il n'en est pas de même sur les espaces de dimension infinie. Une série peut converger pour une norme mais pas pour une autre.
\end{remark}
Lorsque nous verrons la convergence de séries, nous verrons que la convergence normale est la convergence absolue pour la norme uniforme.

\begin{lemma}       \label{LemCAIPooPMNbXg}
    Si \( E\) et \( F\) sont des espaces de Banach\quext{Je crois qu'il ne faut pas que \( E\) soit complet.}, l'espace \( \aL(E,F)\) est également de Banach.
\end{lemma}

\begin{proof}
    Soit \( (u_n)\) une suite de Cauchy dans \( \aL(E,F)\); si \( x\in E\) il existe \( N\) tel que si \( l,m>N\) alors \( \| u_l-u_m \|<\epsilon\), c'est à dire que pour tout \( \| x \|=1\) on a \( \| u_l(x)-u_n(x) \|<\epsilon\). Cela signifie que \( u_n(x)\) est une suite de Cauchy dans l'espace complet \( F\). Cette suite converge et nous pouvons définir l'application \( u\colon E\to F\) par
    \begin{equation}
        u(x)=\lim_{n\to \infty} u_n(x).
    \end{equation}
    Il suffit maintenant de prouver que \( u\) est linéaire, ce qui est une conséquence directe de la linéarité de la limite :
    \begin{equation}
        u(\alpha x+\beta y)=\lim_{n\to \infty} \big( \alpha u_n(x)+\beta u_n(y) \big).
    \end{equation}
\end{proof}

\begin{proposition}[Thème~\ref{THEMEooPQKDooTAVKFH}]     \label{PropQAjqUNp}
    Soit \( E\) un espace de Banach (espace vectoriel normé complet). Si \( A\) est un endomorphisme de \( E\) satisfaisant  \( \| A \|<1\) pour la norme opérateur, alors \( (\mtu-A)\) est inversible et son inverse est donné par
    \begin{equation}
        (\mtu-A)^{-1}=\sum_{k=0}^{\infty}A^k.
    \end{equation}
\end{proposition}
\index{série!donnant \( (1-A)^{-1}\)}

\begin{proof}
    Étant donné que la norme opérateur est une norme algébrique (lemme~\ref{LEMooFITMooBBBWGI}), nous avons \( \| A^k \|\leq \| A \|^k\). Par conséquent la série \( \| A^k \|\) est majorée par la série géométrique qui converge. Par conséquent \( \sum_{k}A^k\) est une série absolument convergente et donc convergente par la proposition~\ref{PropAKCusNM} et le fait que \( \aL(E)\) est complet (proposition~\ref{LemCAIPooPMNbXg}).

    Montrons à présent que la somme est l'inverse de \( \mtu-A\) en utilisant le produit terme à terme autorisé par la proposition~\ref{PropQXqEPuG} :
    \begin{equation}
        \sum_{k=0}^nA^k(\mtu-A)=\sum_{k=0}^n(A^k-A^{k+1})=\mtu-A^{n+1}.
    \end{equation}
    Par conséquent
    \begin{equation}
        \| \mtu-\sum_{k=0}^nA^k(\mtu-A) \|=\| A^{n+1} \|\leq \| A \|^{n+1}\to 0.
    \end{equation}
\end{proof}

\begin{proposition}  \label{PROPooYDFUooTGnYQg}
    Si une série converge dans un espace complet, la norme de son terme général converge vers $0$.
\end{proposition}

\begin{proof}
    Soit une suite \( (a_n)\) dont la série converge vers \( s\). Soit \( \epsilon>0\). La suite des sommes partielles \( (s_n)\) est de Cauchy et converge vers \( s\) : \( s_n\to s\). En particulier il existe un \( N\) tel que si \( n>N\), nous avons \( \| s_n-s_{n-1} \|<\epsilon\). Pour de telles valeurs de \( n\) nous avons :
    \begin{equation}
        \| a_n \|=\| s_n-s_{n-1} \|\leq \epsilon.
    \end{equation}
    Cela prouve que \( a_n\to 0\).
\end{proof}

\begin{proposition}
    Si la série converge alors la somme est associative
\end{proposition}

\begin{proof}
    Associativité. Supposons que \( \sum_ka_k\) et \( \sum_kb_k\) convergent tous deux. Alors nous avons pour tout \( N\) :
    \begin{equation}
        \sum_{k=0}^N(a_k+b_k)=\sum_{k=0}^Na_k+\sum_{k=0}^Nb_k.
    \end{equation}
    Mais si deux limites existent alors la somme commute avec la limite. C'est le cas pour la limite \( N\to \infty\), donc
    \begin{equation}
        \lim_{N\to \infty} \sum_{k=1}^{\infty}(a_k+b_k)=\lim_{N\to \infty} \sum_{k=0}^{\infty}a_k+\lim_{N\to \infty} \sum_{k=0}^{\infty}b_k.
    \end{equation}
\end{proof}

%+++++++++++++++++++++++++++++++++++++++++++++++++++++++++++++++++++++++++++++++++++++++++++++++++++++++++++++++++++++++++++
\section{Série réelle}
%+++++++++++++++++++++++++++++++++++++++++++++++++++++++++++++++++++++++++++++++++++++++++++++++++++++++++++++++++++++++++++
\label{secseries}

La notion de série formalise le concept de somme infinie. L'absence de certaines propriétés de ces objets (problèmes de commutativité et même d'associativité) incite à la prudence et montre à quel point une définition précise est importante.


\subsection{Critères de convergence absolue}

Étant donné le terme général d'une série, il est souvent --dans les cas qui nous intéressent-- difficile de déterminer la somme de la série. L'exemple de la série géométrique est particulier, puisqu'on connaît une formule pour chaque somme partielle, mais pour l'exemple des séries de Riemann il n'y a aucune formule simple pour un $\alpha$ général. D'où l'intérêt d'avoir des critères de convergence ne nécessitant aucune connaissance de l'éventuelle limite de la série.

\begin{lemma}[Critère de comparaison]   \label{LemgHWyfG}
Soient $\sum_i a_i$ et $\sum_j
b_j$ deux séries à termes positifs vérifiant
\begin{equation*}
  0 \leq a_i \leq b_i
\end{equation*}
alors
\begin{enumerate}
\item si $\sum_i a_i$ diverge, alors $\sum_j b_j$ diverge,
\item si $\sum_j b_j$ converge, alors $\sum_i a_i$ converge
  (absolument).
  \end{enumerate}
\end{lemma}


\begin{proposition}[Critère d'équivalence\cite{TrenchRealAnalisys}]
 Soient $\sum_i a_i$ et $\sum_j b_j$ deux séries à termes positifs. Supposons l'existence de la limite (éventuellement infinie) suivante
\begin{equation}
  \limite i \infty \frac{a_i}{b_i} = \alpha
\end{equation}
avec \( \alpha\in \eR\cup\{ +\infty \}\). Alors
\begin{enumerate}
\item si $\alpha \neq 0$ et $\alpha\neq \infty$, alors
  \begin{equation}
    \sum_i a_i \text{~converge} \ssi \sum_j b_j\text{~converge,}
  \end{equation}
\item si $\alpha = 0$ et $\sum_j b_j$ converge, alors $\sum_i a_i$
  converge (absolument),
\item si $\alpha = +\infty$ et $\sum_j b_j$ diverge, alors $\sum_i
  a_i$ diverge.
\end{enumerate}
\end{proposition}

\begin{proof}
\begin{enumerate}
    \item
        Le fait que la suite $a_n/b_n$ converge vers $\alpha$ signifie que tant sa limite supérieure que sa limite inférieure convergent vers $\alpha$. En particulier la suite $\frac{ a_n }{ b_n }$ est bornée vers le haut et vers le bas. À partir d'un certain rang $N$, il existe $M$ tel que
        \begin{equation}
            \frac{ a_n }{ b_n }<M
        \end{equation}
        et il existe $m$ tel que
        \begin{equation}
            \frac{ a_n }{ b_n }>m.
        \end{equation}
        Nous avons donc $a_n<Mb_n$ et $a_n>mb_n$. La série de $(a_n)$ converge donc si et seulement si la série de $(b_n)$ converge.
    \item
        Si $\alpha=0$, cela signifie que pour tout $\epsilon$, il existe un rang tel que $\frac{ a_n }{ b_n }<\epsilon$, et donc tel que $a_n<\epsilon b_k$. La suite de $(a_i)$ converge donc dès que la suite de $(b_i)$ converge.
    \item
        Pour tout $M$, il existe un rang dans la suite à partir duquel on a $\frac{ a_i }{ b_i }>M$, et donc $a_k>Mb_k$. Si la série de $(b_k)$ diverge, la série de $(a_k)$ doit également diverger.
\end{enumerate}
\end{proof}


\begin{proposition}[Critère du quotient\cite{KeislerElemCalculus}]     \label{PropOXKUooQmAaJX}
    Soit $\sum_i a_i$ une série. Supposons l'existence de la limite (éventuellement infinie) suivante
    \begin{equation}
      \limite i \infty \abs{\frac{a_{i+1}}{a_i}} = L
    \end{equation}
    avec \( L\in \eR\cup\{ +\infty \}\).  Alors
    \begin{enumerate}
    \item si $L < 1$, la série converge absolument,
    \item si $L > 1$, la série diverge,
    \item si $L = 1$ le critère échoue : il existe des exemples de convergence et des exemples de divergence.
    \end{enumerate}
\end{proposition}
\index{critère du quotient}

\begin{proof}
\begin{enumerate}
    \item
        Soit $b$ tel que $L<b<1$. À partir d'un certain rang $K$, on a $\left| \frac{ a_{i+1} }{ a_i } \right| <b$. En particulier,
        \begin{equation}
            | a_{K+1} |<b| a_K |,
        \end{equation}
        et pour $a_{K+2}$ nous avons
        \begin{equation}
            | a_{K+2} |<b| a_{K+1} |<b^2| a_K |.
        \end{equation}
        Au final,
        \begin{equation}
            | a_{K+n} |<b^n| a_K |.
        \end{equation}
        Étant donné que la série $\sum_{n\geq K}b^n$ converge (parce que $b<1$), la queue de suite $\sum_{i\geq K}a_i$ converge, et par conséquent la suite au complet converge.
    \item
        Si $L>1$, on a
        \begin{equation}
            | a_K |<| a_{K+1} |<| a_{K+2} |<\ldots
        \end{equation}
        Il est donc impossible que la suite $(a_i)$ converge vers zéro. La série ne peut donc pas converger.
    \item
        Par exemple la suite harmonique $a_n=\frac{1}{ n }$ vérifie $L=1$, mais la série ne converge pas. Par contre, la suite $a_n=\frac{ 1 }{ n^2 }$ vérifie aussi le critère avec $L=1$ tandis que la série $\sum_n\frac{1}{ n^2 }$ converge.
\end{enumerate}
\end{proof}


\begin{proposition}[Critère de la racine\cite{TrenchRealAnalisys}]
    Soit $\sum_i a_i$ une série, et considérons
    \begin{equation*}
      \limsup_{i \rightarrow \infty} \sqrt[i]{\abs{a_i}} = L
    \end{equation*}
    avec \( L\in \eR\cup\{ +\infty \}\). Alors
    \begin{enumerate}
    \item si $L < 1$, la série converge absolument,
    \item si $L> 1$, la série diverge,
    \item si $L = 1$ le critère échoue.
    \end{enumerate}
\end{proposition}

\begin{proof}
    \begin{enumerate}
        \item
            Si $L<1$, il existe un $r\in \mathopen] 0 , 1 \mathclose[$ tel que $| a_n |^{1/n}<r$ pour les grands $n$. Dans ce cas, $| a_n |<r^{n}$, et la série converge absolument parce que la série $\sum_nr^n$ converge du fait que $r<1$.
        \item
            Si $L>1$, il existe un $r>1$ tel que $| a_n |^{1/n}>r>1$. Cela fait que $| a_n |$ prend des valeurs plus grandes que $n$ pour une infinité de termes. Le terme général $a_n$ ne peut donc pas être une suite convergente. Par conséquent la suite diverge au sens où elle ne converge pas.

    \end{enumerate}
\end{proof}

%---------------------------------------------------------------------------------------------------------------------------
\subsection{Critères de convergence simple}
%---------------------------------------------------------------------------------------------------------------------------

Les critères de comparaison, d'équivalence, du quotient et de la racine sont des critères de convergence absolue. Pour conclure à une convergence simple qui n'est pas une convergence absolue, le critère d'Abel sera notre outil principal.

\subsubsection{Critère d'Abel}

\begin{proposition}[Critère d'Abel]
    Soit la série $\sum_i c_iz_i$ avec
    \begin{enumerate}
        \item $(c_i)$ est une suite réelle décroissante qui tend vers zéro,
        \item $(z_i)$ est une suite dans $\eC$ dont la suite des sommes partielles est bornée dans $\eC$, c'est à dire qu'il existe un $M>0$ tel que pour tout $n$,
        \begin{equation}
            \left| \sum_{i=1}^nz_i \right| \leq M.
        \end{equation}
        Alors la série $\sum_ic_iz_i$ est convergente.
    \end{enumerate}
\end{proposition}
Remarquons que ce critère ne donne pas de convergence absolue.

%---------------------------------------------------------------------------------------------------------------------------
\subsection{Quelques séries usuelles}
%---------------------------------------------------------------------------------------------------------------------------
\label{SUBSECooDTYHooZjXXJf}

\begin{example}[Série harmonique]
    La \defe{série harmonique}{série!harmonique} est
    \begin{equation}
        \sum_{i=1}^\infty \frac1i
    \end{equation}
    et diverge (possède une limite $+\infty$).
\end{example}

\begin{example}[Série géométrique] \label{ExZMhWtJS}
    La \defe{série géométrique}{série!géométrique} de raison $q \in \eC$ est
    \begin{equation}    \label{EqZQTGooIWEFxL}
        \sum_{i=0}^\infty q^i.
    \end{equation}
    Étudions la somme partielle \( S_N=1+q+q^2+\cdots +q^{N}\). Nous avons évidemment $S_N-qS_N=1-q^{N+1}$ et donc
    \begin{equation}    \label{EqASYTiCK}
        S_N=\sum_{n=0}^Nq^n=\frac{ 1-q^{N+1} }{ 1-q }.
    \end{equation}
    La limite \( \lim_{N\to \infty} S_N\) existe si et seulement si \( | q |\leq 1\) et dans ce cas nous avons
    \begin{equation}    \label{EqRGkBhrX}
        \sum_{n=0}^{\infty}q^n=\frac{ 1 }{ 1-q }.
    \end{equation}
    La convergence est absolue.

    Si la somme commence en \( n=1\) au lieu de \( n=0\) alors
    \begin{equation}        \label{EqPZOWooMdSRvY}
        \sum_{n=1}^{\infty}q^n=\frac{1}{ 1-q }-1=\frac{ q }{ 1-q }.
    \end{equation}
\end{example}

Un cas particulier de la formule \eqref{EqASYTiCK} est le calcul de \( \sum_{j=1}^{N}q^{-j}\) bien utile lorsque l'on joue avec des nombres binaires (voir l'exemple~\ref{EXEMooRHENooGwumoA}). Nous avons
\begin{equation}        \label{EQooFMBAooEJkHWT}
    \sum_{j=1}^Nq^{-j}=\sum_{j=0}^Nq^{-j}-1=\frac{ 1-q^{-N} }{ q-1 }.
\end{equation}

\begin{example}[Série de Riemann]       \label{EXooCTYNooCjYQvJ}
    Pour $\alpha \in \eR$, la \defe{série de Riemann}{série!Riemann}
    \begin{equation}        \label{EqSerRiem}
        \sum_{i=1}^\infty \frac1{i^\alpha}
    \end{equation}
    converge (absolument, puisque réelle et positive) si et seulement si $\alpha > 1$, et diverge sinon.
\end{example}

\begin{example}[Série exponentielle] \label{ExIJMHooOEUKfj}
    La série exponentielle est la série (pour \( t\in \eR\))
    \begin{equation}
        \exp(t)=\sum_{k=0}^{\infty}\frac{ t^k }{ k! }.
    \end{equation}
    Nous montrons qu'elle converge pour tout \( t\in \eR\). Si \( a_k=t^k/k!\) alors \( \frac{ a_{k+1} }{ a_k }=\frac{ t }{ k }\) dont la limite \( k\to \infty\) est zéro (quel que soit \( t\)). En vertu du critère du quotient~\ref{PropOXKUooQmAaJX} la série exponentielle converge (absolument) pour tout \( t\in \eR\).

    Pour tout savoir de l'exponentielle et de ses variations, voir le thème~\ref{THEMEooKXSGooCsQNoY}.
\end{example}
\index{exponentielle!convergence}

\begin{example}[Série arithmético-géométrique\cite{QXuqdoo}]
    Une \defe{suite arithmético-géométrique}{suite!arithmético-géométrique} est une suite vérifiant pour tout \( n\) la relation
    \begin{equation}
        u_{n+1}=au_n+b
    \end{equation}
    avec \( a\) et \( b\) non nuls. Si elle possède une limite, cette dernière doit résoudre \( l=al+b\), et donc être donnée par
    \begin{equation}
        l=\frac{ b }{ 1-a }.
    \end{equation}

    Comportement amusant : la limite peut exister pour certains valeurs de \( a_0\) et pas pour d'autres. Mais elle ne dépend pas de \( a_0\) parmi ceux pour lesquelles la limite existe.

    Il n'est pas très compliqué de trouver le terme général de la suite en fonction de \( a\) et de \( b\). Il suffit de considérer la suite \( v_n=u_n-r\), et de remarquer que cette suite est géométrique :
    \begin{equation}
        v_{n+1}=av_n.
    \end{equation}
    Par conséquent \( v_n=a^nv_0\), ce qui donne pour la suite \( (u_n)\) la formule
    \begin{equation}
        u_n=a^n(u_0-r)+r.
    \end{equation}
\end{example}

%--------------------------------------------------------------------------------------------------------------------------- 
\subsection{Séries alternées}
%---------------------------------------------------------------------------------------------------------------------------

\begin{theorem}[Critère des séries alternées\cite{ooXFPIooCLUvzV}]      \label{THOooOHANooHYfkII} 
    Si \( a\) est une suite réelle décroissante à limite nulle, alors
    \begin{enumerate}
        \item
            La série \( \sum_n(-1)^na_n\) converge,
        \item
            si nous notons \( (S_n)\) la suite des sommes partielles, les sous-suites \( (S_{2n})\) et \( (S_{2n+1})\) sont adjacentes de limite \( \sum_{n=1}^{\infty}(-1)^na_n\).
        \item
            Si nous considérons le reste 
            \begin{equation}
                R_n=\sum_{k=n+1}^{\infty}(-1)^ka_k,
            \end{equation}
            nous avons
            \begin{subequations}
                \begin{align}
                    \signe(R_n)=(-1)^{n+1}\\
                    | R_n |\leq a_{n+1}.
                \end{align}
            \end{subequations}
    \end{enumerate}
\end{theorem}

\begin{proof}
    En termes de notations, nous allons écrire \( (S_n)\) la suite des sommes partielles de \( \sum_{k=0}^{\infty}(-1)^ka_k\). Nous notons \( (S_{2n})\) la suite des termes pairs de cette suite. C'est donc la suite \( n\mapsto S_{2n}\).
    Nous divisons en plusieurs morceaux.
    \begin{subproof}
        \item[\( S_{2n}\) est croissante]
            Nous avons simplement
            \begin{equation}
                S_{2n+2}-S_{2n}=a_{2n+2}-a_{2n+1}\leq 0.
            \end{equation}
        \item[\( (S_{2n+1})\) est décroissante]
            Même calcul.
        \item[Les suites \( (S_{2n})\) et \( S_{2n+1}\) sont adjacentes] Nous avons simplement
            \begin{equation}
                S_{2n+1}-S_{2n}=a_{2n+1}\to 0.
            \end{equation}
            Nous concluons par le théorème des suites adjacentes \ref{THOooZJWLooAtGMxD} que les sous-suites des termes pairs et impairs sont convergentes et convergent vers la même limite.
    \end{subproof}
    C'est le moment d'utiliser la proposition \ref{PROPooXOOCooGMqJNe} qui convaincra la lectrice que \( (S_n)\) converge vers la même limite, que nous notons \( S\). Le théorème des suites adjacentes nous dit encore que 
    \begin{equation}
        S_{2n+1}\leq S\leq S_{2n}
    \end{equation}
    et donc que \( R_{2n}=S-S_{2n}\leq 0\). Cela donne la majoration
    \begin{equation}
        | R_{2n} |=| S-S_n |=S_{2n}-S\leq S_{2n}-S_{2n+1}=a_{2n+1}.
    \end{equation}
    Nous faisons le même genre de majorations pour \( R_{2n+1}\).
\end{proof}

%---------------------------------------------------------------------------------------------------------------------------
\subsection{Moyenne de Cesaro}
%---------------------------------------------------------------------------------------------------------------------------

\begin{definition}
    Si \( (a_n)_{n\in \eN} \) est une suite dans \( \eR\) ou \( \eC\), alors sa \defe{moyenne de Cesaro}{moyenne!de Cesaro}\index{Cesaro!moyenne} est la limite (si elle existe) de la suite
    \begin{equation}
        c_n=\frac{1}{ n }\sum_{k=1}^na_k.
    \end{equation}
    En un mot, c'est la limite des moyennes partielles.
\end{definition}

\begin{lemma}       \label{LemyGjMqM}
    Si la suite \( (a_n)\) converge vers la limite \( \ell\) alors la suite admet une moyenne de Cesaro qui vaudra \( \ell\).
\end{lemma}

\begin{proof}
    Soit \( \epsilon>0\) et \( N\in \eN\) tel que \( | a_n-\ell |<\epsilon\) pour tout \( n>N\). En remarquant que
    \begin{equation}
        \frac{1}{ n }\sum_{k=1}^nk-\ell=\frac{1}{ n }\sum_{k=1}^n(a_k-\ell),
    \end{equation}
    nous avons
    \begin{subequations}
        \begin{align}
            | \frac{1}{ n }\sum_{k=1}^na_k-\ell |&\leq| \frac{1}{ n }\sum_{k=1}^N| a_k-\ell | |+\big| \frac{1}{ n }\sum_{k=N+1}^n\underbrace{| a_k-\ell |}_{\leq \epsilon} \big|\\
            &\leq \epsilon+\frac{ n-N-1 }{ n }\epsilon\\
            &\leq 2\epsilon.
        \end{align}
    \end{subequations}
    Dans ce calcul nous avons redéfinit \( N\) de telle sorte que le premier terme soit inférieur à \( \epsilon\).
\end{proof}

%---------------------------------------------------------------------------------------------------------------------------
\subsection{Écriture décimale d'un nombre}
%---------------------------------------------------------------------------------------------------------------------------

\begin{normaltext}      \label{NORMALooTZWYooPMgOIm}
    Soit \( b\geq 2\) un entier qui sera la base dans laquelle nous allons écrire les nombres. Nous considérons l'ensemble \( \eD_b\)\nomenclature[Y]{\( \eD_b\)}{l'ensemble de écritures décimales en base \( b\)} des suites dans \( \{ 0,1,\ldots, b-1 \}\) qui n'ont pas une queue de suite uniquement formée de \( b-1\). Autrement dit une suite \( (c_n)\) est dans \( \eD_b\) lorsque pour tout \( N\), il existe \( k>N\) tel que \( c_k\neq b-1\). Associé à cet ensemble nous considérons la fonction
    \begin{equation}    \label{EqXXXooOTsCK}
        \begin{aligned}
            \varphi_b\colon \eD_b&\to \mathopen[ 0 , 1 [ \\
                c&\mapsto \sum_{n=1}^{\infty}\frac{ c_n }{ b^n }.
        \end{aligned}
    \end{equation}
\end{normaltext}

\begin{lemma}
    La fonction \( \varphi_b\) est bien définie au sens où elle converge et prend ses valeurs dans \( \mathopen[ 0 , 1 [\).
\end{lemma}

\begin{proof}
    Tout se base sur la somme de la série géométrique \eqref{EqRGkBhrX} sous la forme
    \begin{equation}    \label{EqWZGooXJgwl}
        \sum_{k=0}^{\infty}\frac{1}{ b^k }=\frac{ b }{ b-1 }.
    \end{equation}
    La somme \eqref{EqXXXooOTsCK} est donc majorée par \( \sum_n\frac{ b-1 }{ b^n }\) qui converge.

    Pour prouver que l'image de \( \varphi_b\) est bien \( \mathopen[ 0 , 1 [\), nous savons qu'au moins un des \( c_n\) (en fait une infinité) est plus petit que \( b-1\), donc nous avons la majoration stricte\footnote{Notez que la somme \eqref{EqXXXooOTsCK} commence à un tandis que la série géométrique \eqref{EqWZGooXJgwl} commence à zéro.}
        \begin{equation}
            \varphi_b(c)<\sum_{n=1}^{\infty}\frac{ b-1 }{ b^n }=(b-1)\left( \sum_{n=1}^{\infty}\frac{1}{ b^n }-1 \right)=1
        \end{equation}
\end{proof}

Le fait d'introduire l'ensemble \( \eD\) au lieu de l'ensemble de toutes les suites est justifié par la proposition suivante. Elle explique pourquoi un nombre possède au maximum deux écritures décimales distinctes et que ces deux sont obligatoirement de la forme, par exemple en base \( 10\) :
\begin{equation}
    0.34599999999\ldots=0.34600000\ldots
\end{equation}
mais qu'un nombre commençant par \( 0.347\) ne peut pas être égal. C'est pour cela que dans la définition de \( \eD_b\) nous avons exclu les suites qui terminent par tout des \( b-1\).
\begin{proposition} \label{PropSAOoofRlQR}
    Soit la fonction
    \begin{equation}
        \begin{aligned}
            \varphi\colon \{ 0,\ldots, b-1 \}^{\eN}&\to \mathopen[ 0 , 1 [ \\
                x&\mapsto \sum_{n=1}^{\infty}\frac{ x_n }{ b^n }.
        \end{aligned}
    \end{equation}
    Si \( \varphi(x)=\varphi(y)\) et si \( n_0\) est le plus petit entier tel que \( x_{n_0}\neq y_{n_0}\) alors soit
    \begin{equation}
        x_{n_0}-y_{n_0}=1
    \end{equation}
    et \( x_n=0\), \( y_n=b-1\) pour tout \( n>n_0\), soit le contraire : \( y_{n_0}-x_{n_0}=1\) avec \( y_n=0\) et \( x_n=b-1\) pour tout \( n>n_0\).
\end{proposition}

\begin{proof}
    Nous nous basons sur la formule (facilement dérivable depuis \eqref{EqWZGooXJgwl}) suivante :
    \begin{equation}
        \sum_{k=n_0+1}^{\infty}\frac{1}{ b^k }=\frac{1}{ b^{n_0+1} }\frac{ b }{ b-1 }.
    \end{equation}
    Nous avons
    \begin{equation}
        0=\varphi(x)-\varphi(y)=\frac{ x_{n_0}-y_{n_0} }{ b^{n_0} }+\sum_{n=n_0+1}^{\infty}\frac{ x_n-y_n }{ b^n }\geq \frac{ x_{n_0}-y_{n_0} }{ b^{n_0} }-\sum_{n=n_0+1}^{\infty}\frac{ b-1 }{ b^n }=\frac{ x_{n_0}-y_{n_0}-1 }{ b^{n_0} }.
    \end{equation}
    Le dernier terme étant manifestement positif\footnote{C'est ici qu'intervient la subdivision entre le cas \( x_{n_0}-y_{n_0}=1\) ou le contraire. En effet si «ce dernier terme était manifestement \emph{négatif}», il aurait fallu majorer avec de \( 1-b\) au lieu de \( 1-b\).}, il est nul et nous avons \( x_{n_0}-y_{n_0}=1\).

    Nous avons donc maintenant
    \begin{equation}    \label{EqHWQoottPnb}
        0=\varphi(x)-\varphi(y)=\frac{1}{ b^{n_0} }+\sum_{n=n_0+1}^{\infty}\frac{ x_n-y_n }{ b^n }.
    \end{equation}
    Nous majorons la dernière somme de la façon suivante, en supposant que \( | x_n-y_n |\neq b-1\) pour un certain \( n>n_0\) :
    \begin{equation}
        \left| \sum_{n=n_0+1}^{\infty}\frac{ x_n-y_n }{ b^n } \right| \leq\sum_{n=n_0+1}^{\infty}\frac{ | x_n-y_n | }{ b^n }<\sum_{n=n_0+1}^{\infty}\frac{ b-1 }{ b^n }=\frac{1}{ b^{n_0} }.
    \end{equation}
    Étant donné cette inégalité stricte, l'équation \eqref{EqHWQoottPnb} ne peut pas être correcte (valoir zéro). Nous avons donc \( | x_n-b_n |=b-1\) pour tout \( n>n_0\). Donc pour chaque \( n>n_0\) nous avons soit \( x_n=0\) et \( y_n=b-1\), soit \( a_n=b-1\) et \( b_n=0\). Pour conclure il faut encore prouver que le choix doit être le même pour tout \( n\).

    Nous nous mettons dans le cas \( x_{n_0}-y_{n_0}=1\); dans ce cas nous avons bien l'égalité \eqref{EqHWQoottPnb} sans petites nuances de signes. Nous écrivons
    \begin{equation}
        \sum_{n=n_0+1}^{\infty}\frac{ x_n-y_n }{ b^n }=(b-1)\sum_{n=n_0+1}^{\infty}\frac{ (-1)^{s_n} }{ b^n }
    \end{equation}
    où \( s_n\) est pair ou impair suivant que \( x_n=0\), \( y_n=b-1\) ou le contraire. Si un des \( (-1)^{s_n}\) est pas \( -1\) alors nous avons l'inégalité stricte
    \begin{equation}
        (b-1)\sum_{n=n_0+1}^{\infty}\frac{ (-1)^{s_n} }{ b^n }>(b-1)\sum_{n=n_0+1}^{\infty}\frac{-1}{ b^n }=-\frac{1}{ b^{n_0} }.
    \end{equation}
    Dans ce cas il est impossible d'avoir \( \varphi(x)-\varphi(y)=0\). Nous en concluons que \( (-1)^{s_n}\) est toujours \( -1\), c'est à dire \( x_n-y_n=1-b\), ce qui laisse comme seule possibilité \( x_n=0\) et \( y_n=b-1\).
\end{proof}

\begin{theorem} \label{ThoRXBootpUpd}
    L'application \( \varphi_b\colon \eD_b\to \mathopen[ 0 , 1 [\) est bijective.
\end{theorem}

\begin{proof}
    En ce qui concerne l'injection, nous savons de la proposition~\ref{PropSAOoofRlQR} que si \( \varphi_b(x)=\varphi_b(y)\) pour \( x,y\in\{ 0,\ldots, b-1 \}^{\eN}\), alors soit \( x\) soit \( y\) a une queue de suite composée uniquement de \( b-1\), ce qui est exclu dans \( \eD_b\). Nous en déduisons que \( \varphi_b\) est bien injective en prenant \( \eD_b\) comme ensemble départ.

    La partie lourde est la surjectivité. Nous prenons \( x\in \mathopen[ 0 , 1 [\) et nous allons construire par récurrence une suite \( a\in \eD_b\) telle que \( \varphi_b(a)=x\). Si il existe \( a_1\in\{ 0,\ldots, b-1 \}\) tel que \( x=a_1/b\) alors nous prenons la suite \( (a_1,0,\ldots, )\) et nous avons évidemment \( \varphi(a)=x\). Sinon il existe \( a_1\in\{ 0,\ldots, b-1 \}\) tel que
        \begin{equation}
            \frac{ a_1 }{ b }<x<\frac{ a_1+1 }{ b }
        \end{equation}
        parce que les autres possibilités pour \( x\) sont dans l'ensemble \( \mathopen[ 0 , 1 \mathclose[\setminus\{ \frac{ k }{ b } \}_{k=0,\ldots, b-1}\) que nous subdivisons en
        \begin{equation}
        \mathopen] 0 , \frac{1}{ b } \mathclose[\cup\mathopen] \frac{1}{ b } , \frac{ 2 }{ b } \mathclose[\cup\ldots\cup\mathopen] \frac{ b-1 }{ b } , 1 \mathclose[.
        \end{equation}
        Pour la récurrence nous supposons avoir trouvé \( a_1,\ldots, a_n\) tels que
        \begin{equation}
            \sum_{k=1}^n\frac{ a_k }{ b^k }< x<\sum_{k=1}^{n-1}\frac{ a_k }{ b^k }+\frac{ a_n+1 }{ b^n }.
        \end{equation}
    Encore une fois s'il existe \( a_{n+1}\in\{ 0,\ldots, b-1 \}\) tel que \( \sum_{k=1}^{n+1}\frac{ a_k }{ b^k }=x\) alors nous prenons ce \( a_{n+1}\) et nous complétons la suite avec des zéros pour avoir \( \varphi(a)=x\). Sinon
%nous subdivisions l'intervalle \( \mathopen]  \frac{ a_n }{ b^n }, \frac{ a_n }{ b^n }+\frac{ a_n+1 }{ b^n } \mathclose[\) (auquel nous retranchons les \( b\) nombres déjà traités) en
 %       \begin{equation}
 %       \mathopen] \frac{ a_n }{ b^n } , \frac{ a_n }{ b^n }+\frac{1}{ b^{n+1} } \mathclose[ \cup \mathopen] \frac{ a_n }{ b^n }+\frac{1}{ b^{n+1} } , \frac{ a_n }{ b^n }+\frac{2}{ b^{n+1} } \mathclose[\cup\ldots\cup\mathopen] \frac{ a_n }{ b^n }+\frac{ b-1 }{ b^{n+1} } , \frac{ a_n }{ b^n }+\frac{ 1 }{ b^n } \mathclose[.
 %       \end{equation}
        , pour simplifier les notations nous notons \( x'=x-\sum_{k=1}^{n}\frac{ a_k }{ b^k }\) et nous avons
        \begin{equation}
            0<x'<\frac{ a_n+1 }{ b^n }.
        \end{equation}
        Le nombre \( x'\) est forcément dans un des intervalles
        \begin{equation}
                \mathopen] \frac{ s }{ b^{n+1} } , \frac{ s+1 }{ b^{n+1} } \mathclose[
        \end{equation}
        avec \( s\in\{ 0,\ldots, b-1 \}\). Nous prenons le \( s\) correspondant à \( x'\) comme \( a_{n+1}\). Dans ce cas nous avons
        \begin{equation}
            \sum_{k=1}^{n+1}\frac{ a_k }{ b^k }< x<\sum_{k=1}^{n+1}\frac{ a_k }{ b^k }+\frac{1}{ b^{n+1} }.
        \end{equation}
        Note : les deux inégalités sont strictes. La première parce que s'il y avait égalité, nous nous serions déjà arrêté en complétant avec des zéros. La seconde parce que
        \begin{equation}
            \sum_{k=n+2}^{\infty}\frac{ a_k }{ b^k }\leq \sum_{k=n+2}^{\infty}\frac{ b-1 }{ b^k }=\frac{1}{ b^{n+1} }
        \end{equation}
        où l'égalité n'est possible que si \( a_k=b-1\) pour tout \( k\geq n+2\). Dans ce cas nous aurions eu
        \begin{equation}
            x=\sum_{k=1}^{n}\frac{ a_k }{ b^k }+\frac{ a_{n+1}+1 }{ b^{n+1} }
        \end{equation}
        et nous aurions choisi le nombre \( a_{n+1}\) autrement et complété la suite par des zéros à partir de là. Notons que cela prouve au passage que la suite que nous sommes en train de construire est bien dans \( \eD_b\) parce qu'elle ne contiendra pas de queue de suite composée de \( b-1\).

        Ceci termine la construction par récurrence de la suite \( a\in \eD_b\). Par construction nous avons pour tout \( N\geq 1\),
        \begin{equation}
            \sum_{k=1}^N\frac{ a_k }{ b^k }\leq x\leq \sum_{k=1}^N\frac{ a_k }{ b^k }+\frac{1}{ b^{N+1} },
        \end{equation}
        autrement dit : \( \varphi_b(a_1,\ldots, a_N)\in B(x,\frac{1}{ b^{N+1} })\). Nous avons donc bien convergence
        \begin{equation}
            \lim_{N\to \infty} \varphi_b(a_1,\ldots, a_N)=x
        \end{equation}
        et l'application \( \varphi_b\) est surjective.
\end{proof}

L'application \( \varphi_b^{-1}\colon \mathopen[ 0 , 1 [\to \eD_b\) est la \defe{décomposition décimale}{décimale!décomposition} en base \( b\) des nombres de \( \mathopen[ 0 , 1 [\).

Tout cela nous permet de montrer entre autres que \( \eR\) n'est pas dénombrable. Vu qu'il y a une bijection entre \( \mathopen[ 0 , 1 [\) et \( \eD_b\), il suffit de prouver que \( \eD_b\) est non dénombrable. De plus il suffit de démontrer que \( \eD_b\) est non dénombrable pour un entier \( b\geq 2\) donné.

\begin{proposition}[\cite{KZIoofzFLV}]  \label{PropNNHooYTVFw}
    Il n'existe pas de surjection \( \eN\to \eD_b\). Autrement dit \( \eD_b\) est non dénombrable.
\end{proposition}

\begin{proof}
    Nous prenons \( b\neq 2\) pour des raisons qui seront claires plus tard. Soit \( f\colon \eN\to \eD_b\). Pour \( i\in \eN\) nous notons
    \begin{equation}
        f(n)=(c_i^{(n)})_{i\geq 1},
    \end{equation}
    et nous définissons la suite
    \begin{equation}
        c_k=\begin{cases}
            0    &   \text{si } c_k^{(k)}\neq 0\\
            1    &    \text{si } c_k^{(k)}=0.
        \end{cases}
    \end{equation}
    Cela est une suite dans \( \eD_b\) parce que \( b\neq 2\) et que la suite ne contient que des \( 0\) et des \( 1\). Mais nous n'avons \( f(n)=c\) pour aucun \( n\in \eN\) parce que nous avons \( c_n\neq f(n)_n\).

    Si \( b=2\) alors nous savons que \( \eD_2\sim\mathopen[ 0 , 1 [\sim \eD_3\). Donc \( \eD_2\sim \eD_3\) et \( \eD_2\) ne peut pas plus être mis en bijection avec \( \eN\) que \( \eD_3\).
\end{proof}
\begin{remark}
    La preuve ne fonctionne pas en base \( b=2\) parce que rien n'empêche d'avoir une queue de \( 1\). Il y a alors toutefois moyen de se débrouiller en construisant la suite \( c\) de façon plus subtile. Si \( b=2\) et \( n\in \eN\) alors \( f(n)\) est une suite de \( 0\) et \( 1\) contenant une infinité de \( 0\) (parce qu'il n'y a pas de queue de suite ne contenant que des \( 1\)). Nous construisons alors \( c\) de la façon suivante : d'abord nous recopions \( f(0)\) jusqu'à son \emph{deuxième} zéro que nous changeons en \( 1\); nommons \( n_0\) le rang de ce deuxième zéro. Ensuite nous recopions les éléments de \( f(1) \) à partir du rang \( n_0+1\) jusqu'au second zéro que nous changeons en \( 1\), etc.

    Le fait de prendre le deuxième zéro nous garanti que la suite \( c\) n'aura pas de queue de suite ne contenant que des \( 1\).

    Notons que cette construction s'adapte à tout \( b\); il suffit de prendre le second terme qui n'est pas \( b-1\) et le remplacer par \( b-1\).
\end{remark}

\begin{corollary}
    L'ensemble \( \mathopen[ 0 , 1 [\) n'est pas dénombrable.
\end{corollary}

\begin{proof}
    L'ensemble \( \mathopen[ 0 , 1 [\) est en bijection avec \( \eD_b\) que nous venons de prouver n'être pas dénombrable.
\end{proof}

%+++++++++++++++++++++++++++++++++++++++++++++++++++++++++++++++++++++++++++++++++++++++++++++++++++++++++++++++++++++++++++
\section{Applications multilinéaires}
%+++++++++++++++++++++++++++++++++++++++++++++++++++++++++++++++++++++++++++++++++++++++++++++++++++++++++++++++++++++++++++

\begin{definition}[Application multilinéaire]       \label{DefFRHooKnPCT}
    Une application $T: \eR^{m_1}\times \ldots \times\eR^{m_k}\to\eR^p $ est dite \defe{\( k\)-linéaire}{application!multilinéaire} si pour tout $X=(x_1, \ldots,x_k)$ dans $ \eR^{m_1}\times \ldots \times\eR^{m_k}$ les applications $x_i\mapsto T(x_1, \ldots, x_i,\ldots,x_k)$ sont linéaires pour tout $i$ dans $\{1,\ldots,k\}$, c'est à dire
	\begin{equation}
		\begin{aligned}[]
			T(\cdot,x_2, \ldots, x_i,\ldots,x_k)&\in \mathcal{L}(\eR^{m_1}, \eR^p),\\
			T(x_1,\cdot, \ldots, x_i,\ldots,x_k)&\in \mathcal{L}(\eR^{m_2}, \eR^p),\\
						& \vdots\\
			T(x_1, \ldots, x_i,\ldots,x_{k-1},\cdot)&\in \mathcal{L}(\eR^{m_k}, \eR^p).\\
		\end{aligned}
	\end{equation}
	En particulier lorsque $k=2$, nous parlons d'applications \defe{bilinéaires}{bilinéaire}. Vous pouvez deviner ce que sont les applications \emph{tri}linéaire ou \emph{quadri}linéaire.
\end{definition}

L'ensemble des applications $k$-linéaires de $ \eR^{m_1}\times \ldots \times\eR^{m_k}$ dans $\eR^p$ est noté $\mathcal{L}(\eR^{m_1}\times \ldots \times\eR^{m_k}, \eR^p)$ ou $\mathcal{L}(\eR^{m_1}, \ldots,\eR^{m_k}; \eR^p)$.

\begin{example}
  Soit $A$ une matrice avec $m$ lignes et $n$ colonnes. L'application bilinéaire de $\eR^m\times \eR^n$ dans $\eR$ associée à $A$ est définie par
\[
T_A(x,y)= x^TAy=\sum_{i,j}a_{i,j}x_i y_j, \qquad \forall x\in \eR^m, \, y \in \eR^n.
\]
\end{example}

Nous énonçons la proposition suivante dans le cas d'espaces vectoriels normés\footnote{Sans hypothèses sur la dimension.} parce que nous allons l'utiliser dans ce cas, mais le cas particulier \( E_i=\eR^{m_i}\) et \( F=\eR^p\) est important.
\begin{proposition} \label{PropUADlSMg}
    Soient des espaces vectoriels normés \( E_i\) et \( F\). Une application \( n\)-linéaire
    \begin{equation}
        T\colon E_1\times\ldots\times E_n\to F
    \end{equation}
    est est continue si et seulement s'il existe un réel $L\geq 0$ tel que
  \begin{equation}\label{limitatezza}
     \|T(x_1, \ldots,x_n)\|_F\leq L \|x_1\|_{F_1}\cdots\|x_n\|_{F_n}, \qquad \forall x_i\in E_i.
  \end{equation}
\end{proposition}

\begin{proof}
    Pour simplifier l'exposition nous nous limitons au cas $n=2$ et nous notons $T(x,y)=x*y$

    Supposons que l'inégalité \eqref{limitatezza} soit satisfaite.
    \begin{equation}\label{LimImplCont}
      \begin{aligned}
        \|x*y-x_0*y_0\|&=\|(x-x_0)*y-x_0*(y-y_0)\|\\
    &\leq \|(x-x_0)*y\|+\|x_0*(y-y_0)\|\\
    &\leq L\|x-x_0\|\|y\| + L\|x_0\|\|y-y_0\|.
      \end{aligned}
    \end{equation}
    Si $x\to x_0$ et $y\to y_0$  on voit que $T$ est continue en passant à la limite aux deux côtes de l'inégalité \eqref{LimImplCont}.

    Soit $T$ continue en $(0,0)$. Évidemment\footnote{Dans la formule suivante, les trois zéros sont les zéros de trois espaces différents.} $0*0=0$, donc il existe $\delta>0$ tel que si $x\in B_{E_1}(0,\delta)$ et $y\in B_{E_2}(0,\delta)$ alors $\|x*y\|\leq 1$. En particulier si \( (x,y)\in B_{E_1\times E_2}(0,\delta)\) nous sommes dans ce cas. Soient maintenant  $x\in E_1\setminus\{ 0 \}$  et $y\in E_2\setminus\{ 0\}$
    \begin{equation}
        x*y=\left(\frac{\|x\|}{\delta}\frac{\delta x}{\|x\|}\right)*\left(\frac{\|y\|}{\delta}\frac{\delta y}{\|y\|}\right)
    =\frac{\|x\|\|y\|}{\delta^2} \left(\frac{\delta x}{\|x\|}\right)*\left(\frac{\delta y}{\|y\|}\right).
     \end{equation}
    On remarque que $\delta x/\|x\|_m$ est dans la boule de rayon $\delta$ centrée en $0_m$ et que $\delta y/\|y\|_n$ est dans la boule de rayon $\delta$ centrée en $0_n$. On conclut
    \[
     x*y\leq \frac{\|x\|_m\|y\|_n}{\delta^2}.
    \]
    Il faut prendre $L=1/\delta^2$.
\end{proof}

La norme de \( T\) est alors définie comme la plus petite constante \( L\) qui fait fonctionner la proposition~\ref{PropUADlSMg}.
\begin{definition}  \label{DefKPBYeyG}
	La norme sur l'espace $\aL(E_1\times \cdots\times E_n, F)$ des applications $k$-linéaires et continues est
	\begin{equation}
        \|T\|_{E_1\times \ldots\times E_n}=\sup\{ \|T(u_1, \ldots,u_k)\|_{F}\,\vert\,\|u_i\|_{E_i}\leq 1, i=1,\ldots, k \}.
	\end{equation}
\end{definition}
Nous avons donc automatiquement
\begin{equation}    \label{EqYLnbRbC}
    \| T(u,v) \|\leq \| T \|\| u \|\| v \|.
\end{equation}
Et nous notons que cette norme est uniquement définie pour les applications linéaires continues. Ce n'est pas très grave parce qu'alors nous définissons \( \| T \|=\infty\) si \( T\) n'est pas continue. Cela pour retrouver le principe selon lequel on est continue si et seulement si on est borné.

\begin{proposition}\label{isom_isom}
  On définit les fonctions
  \begin{equation}
    \begin{array}{rccc}
      \omega_g: & \mathcal{L}(\eR^{m}\times\eR^{n}, \eR^p)&\to &\mathcal{L}(\eR^{m}, \mathcal{L}(\eR^{n}, \eR^p)),\\
      \omega_d: & \mathcal{L}(\eR^{m}\times\eR^{n}, \eR^p)&\to &\mathcal{L}(\eR^{n}, \mathcal{L}(\eR^{m}, \eR^p)),
    \end{array}
  \end{equation}
par
\[
\omega_g(T)(x)=T(x,\cdot), \qquad \forall x\in\eR^m,
\]
et
\[
\omega_d(T)(y)=T(\cdot, y), \qquad \forall y\in\eR^n.
\]
Les fonctions $\omega_g$ et $\omega_d$ sont des isomorphismes qui préservent les normes.
\end{proposition}

\begin{lemma}[Leibnitz pour les formes bilinéaires\cite{SNPdukn}]\label{LemFRdNDCd}
    Si \( B\colon E\times F\to G\) est bilinéaire et continue, elle est \(  C^{\infty}\) et
    \begin{equation}    \label{EqXYJgDBt}
        dB_{(x,y)}(u,v)=B(x,v)+B(u,y).
    \end{equation}
\end{lemma}

\begin{proof}
    D'abord le membre de droite de \eqref{EqXYJgDBt} est une application linéaire et continue, donc c'est un bon candidat à être différentielle. Nous allons prouver que ça l'est, ce qui prouvera la différentiabilité de \( B\). Avec ce candidat, le numérateur de la définition \eqref{EqIQuRGmO} s'écrit dans notre cas
    \begin{equation}
        B\big( (x,y)+(u,v) \big)-B(x,y)-B(x,v)-B(u,y)=B(u,v).
    \end{equation}
    Il reste à voir que
    \begin{equation}
        \lim_{ (u,v)\to (0,0) } \frac{ B(u,v) }{ \| (u,v) \| }=0
    \end{equation}
    Par l'équation \eqref{EqYLnbRbC} nous avons
    \begin{equation}
        \frac{ \| B(u,v) \| }{ \| (u,v) \| }\leq \frac{ \| B \|\| u \|\| v \| }{ \| u \| }=\| B \|\| v \|
    \end{equation}
    parce que \( \| (u,v) \|\geq \| u \|\). À partir de là il est maintenant clair que
    \begin{equation}
        \lim_{(u,v)\to (0,0)}\frac{ \| B(u,v) \| }{ \| (u,v) \| }=0,
    \end{equation}
    ce qu'il fallait.
\end{proof}

\begin{proposition}[Règle de Leibnitz\cite{SNPdukn}]
    Soient \( E,F_1,F_2\) des espaces vectoriels normés, \( U\) ouvert dans \( E\) et des applications de classe \( C^r\) (\( r\geq 1\))
    \begin{subequations}
        \begin{align}
            f_1\colon U\to F_1\\
            f_2\colon U\to F_2\\
        \end{align}
    \end{subequations}
    et \( B\in\cL(F_1\times F_2,G)\). Alors l'application
    \begin{equation}
        \begin{aligned}
            \varphi\colon U&\to G \\
            x&\mapsto B\big( f_1(x),f_2(x) \big)
        \end{aligned}
    \end{equation}
    est de classe \( C^r\) et
    \begin{equation}    \label{EqMNGBXWc}
        d\varphi_x(u)=\varphi\big( (df_1)_x(u),f_2(x) \big)+\varphi\big( f_1(x),(df_2)_x(u) \big).
    \end{equation}
\end{proposition}
\index{Leibnitz!applications entre espaces vectoriels normés}

\begin{proof}
    Par hypothèse \( B\) est continue (c'est la définition de l'espace \( \cL\)), et donc \(  C^{\infty}\) par le lemme~\ref{LemFRdNDCd}. Par ailleurs la fonction \( f_1\times f_2\) est de classe \( C^r\) parce que \( f_1\) et \( f_2\) le sont et parce que la proposition~\ref{PropOYtgIua} le dit. L'application composée \( B\circ(f_1\times f_2)\) est donc également de classe \( C^r\) par le théorème~\ref{ThoAGXGuEt}.

    Il ne nous reste donc qu'à prouver la formule~\ref{EqMNGBXWc}. En utilisant la différentielle du produit cartésien\footnote{Proposition~\ref{PropOYtgIua}.} nous avons
    \begin{equation}
        f\big( B\circ(f_1\times f_2) \big)_x(h)=dB_{(f_1\times f_2)(x)}\big( (df_1)_x(h),(df_2)_x(h) \big).
    \end{equation}
    Nous développons cela en utilisant le lemme~\ref{LemFRdNDCd} :
    \begin{subequations}
        \begin{align}
        d\big( B\circ(f_1\times f_2) \big)_x(h)&=dB_{\big( f_1(x),f_2(x) \big)}\big( (df_1)_x(h),(df_2)_x(h) \big)\\
        &=B\big( f_1(x),(df_2)_x(h) \big)+B\big( (df_1)_x(h),f_2(x) \big),
        \end{align}
    \end{subequations}
    comme souhaité.
\end{proof}

%---------------------------------------------------------------------------------------------------------------------------
\subsection{Différentielle partielle}
%---------------------------------------------------------------------------------------------------------------------------

\begin{definition}[Différentielle partielle]    \label{VJM_CtSKT}
    Soient \( E\), \( F\) et \( G\) des espaces vectoriels normés et une fonction \( f\colon E\times F\to G\). Nous définissons sa \defe{différentielle partielle}{différentielle!partielle} sur l'espace \( E\) par
    \begin{equation}
        \begin{aligned}
            d_1f_{(x_0,y_0)}\colon E&\to G \\
            u&\mapsto \Dsdd{ f(x_0+tu,y_0 }{t}{0} .
        \end{aligned}
    \end{equation}
    La différentielle \( d_2\) se définit de la même façon.
\end{definition}

\begin{proposition}[\cite{SNPdukn}] \label{PropLDN_nHWDF}
    Soient \( E_1\), \( E_2\) et \( F\) des espaces vectoriels normés, soit un ouvert \( U\subset E_1\times E_2\) et une fonction \( f\colon U\to F\).
    \begin{enumerate}
        \item   \label{ItemRDD_oPmXVi}
            Si \( f\) est différentiable alors les différentielles partielles existent et
            \begin{subequations}
                \begin{align}
                    d_1f_{(x_0,y_0)}(u)=df_{(x_0,y_0)}(u,0)\\
                    d_2f_{(x_0,y_0)}(v)=df_{(x_0,y_0)}(0,v)
                \end{align}
            \end{subequations}
            où \( u\in E_1\) et \( v\in E_2\).
        \item
            Si \( f\) est différentiable alors
            \begin{equation}
                df_{(x_0,y_0)}(u,v)=d_1f_{(x_,y_0)}(u)+d_2f_{(x_0,y_0)}(v).
            \end{equation}
    \end{enumerate}
\end{proposition}

\begin{proof}
    Nous posons \( \alpha=(x_0,y_0)\in U\) et
    \begin{equation}
        \begin{aligned}
            j_{\alpha}^{(1)}\colon E_1&\to E_1\times E_2 \\
            x&\mapsto (x,y_0).
        \end{aligned}
    \end{equation}
    C'est une fonction de classe \(  C^{\infty}\) et
    \begin{equation}
        (dj_{\alpha}^{(1)})_{x_0}(u)=\Dsdd{ j_{\alpha}^{(1)}(x_0+tu) }{t}{0}=\Dsdd{ (x_0+tu,y_0) }{t}{0}=(u,0).
    \end{equation}
    D'autre part
    \begin{subequations}
        \begin{align}
            (d_1f)_{\alpha}(u)&=\Dsdd{ f(x_0+tu,y_0) }{t}{0}\\
            &=\Dsdd{ (f\circ j_{\alpha}^{(1)})(x_0+tu) }{t}{0}\\
            &=\big( d(f\circ j_{\alpha}^{(1)}) \big)_{x_0}(u).
        \end{align}
    \end{subequations}
    À ce moment nous utilisons la règle des différentielles composées~\ref{ThoAGXGuEt} pour dire que
    \begin{equation}
        (d_1f)_{\alpha}(u)=df_{j_{\alpha}^{(1)}(x_0)}\circ (dj_{\alpha}^{(1)})_{x_0}(u)=df_{\alpha}(u,0).
    \end{equation}
    Voila qui prouve déjà le point~\ref{ItemRDD_oPmXVi}.

    Pour la suite nous considérons les fonctions
    \begin{equation}
        \begin{aligned}[]
            P_1(x,y)&=x,&&&J_1(u)&=(u,0),\\
            P_2(x,y)&=y,&&&J_2(v)&=(0,v)
        \end{aligned}
    \end{equation}
    et nous avons l'égalité évidente
    \begin{equation}
        J_1\circ P_1+J_2\circ P_2=\mtu
    \end{equation}
    sur \( E_1\times E_2\). En appliquant \( df_{\alpha}\) à cette dernière égalité, en appliquant à \( (u,v)\) et en utilisant la linéarité de \( df_{\alpha}\) nous trouvons
    \begin{subequations}
        \begin{align}
            df_{\alpha}(u,v)&=df_{\alpha}\big( (J_1\circ P_1)(u,v) \big)+df_{\alpha}\big( (J_2\circ P_2)(u,v) \big)\\
            &=df_{\alpha}(u,0)+df_{\alpha}(0,v)\\
            &=(d_1f)_{\alpha}(u)+(d_2f)_{\alpha}(v)
        \end{align}
    \end{subequations}
    où nous avons utilisé le point~\ref{ItemRDD_oPmXVi} pour la dernière égalité.
\end{proof}

%+++++++++++++++++++++++++++++++++++++++++++++++++++++++++++++++++++++++++++++++++++++++++++++++++++++++++++++++++++++++++++
\section{Exponentielle de matrice}
%+++++++++++++++++++++++++++++++++++++++++++++++++++++++++++++++++++++++++++++++++++++++++++++++++++++++++++++++++++++++++++
\label{secAOnIwQM}

\begin{proposition}     \label{PropPEDSooAvSXmY}
    Soit \( V\) un espace vectoriel de dimension finie et \( A\in\End(V)\). La série
    \begin{equation}
        \exp(A)=\mtu+A+\frac{ A^2 }{ 2 }+\frac{ A^3 }{ 3 }+\ldots =\sum_{k=1}^{\infty}\frac{ A^k }{ k! }.
    \end{equation}
    converge normalement dans \( \big( \End(V),\| . \|_{op} \big)\).  L'\defe{exponentielle}{exponentielle!de matrice} de la matrice \( A\) est cette matrice.
\end{proposition}

\begin{proof}
    Vu que la norme opérateur est une norme d'algèbre par le lemme~\ref{LEMooFITMooBBBWGI}, nous avons pour tout \( k\) la majoration \( \| A^k \|\leq \| A \|^k\). Nous avons donc
    \begin{equation}
        \sum_{k=0}^{\infty}\frac{ \| A^k \| }{ k! }\leq \sum_k\frac{ \| A \|^k }{ k! }.
    \end{equation}
    La dernière somme converge en vertu de la convergence de la série exponentielle donnée en exemple~\ref{ExIJMHooOEUKfj}.
\end{proof}

Étant donné que c'est une limite, il y a une question de convergence et donc de topologie. C'est pour cela que nous ne pouvions pas introduire l'exponentielle de matrice avant d'avoir introduit la norme des matrices. La convergence de la série pour toute matrice sera prouvée au passage dans la proposition~\ref{PropFMqsIE}.


La fonction exponentielle \(  x\mapsto e^{x}\) n'est pas un polynôme en \( x\), mais nous avons le résultat marrant suivant.
\begin{proposition} \label{PropFMqsIE}
    Si \( u\) est un endomorphisme, alors \( \exp(u)\) est un polynôme en \( u\)\footnote{Nan, mais j'te jure : \( \exp\) n'est pas un polynôme, mais $\exp(u)$ est un polynôme de \( u\).}.
\end{proposition}

\begin{proof}
    Nous considérons l'application
    \begin{equation}
        \begin{aligned}
            \varphi_u\colon \eK[X]&\to \End(E) \\
            P&\mapsto P(u)
        \end{aligned}
    \end{equation}
    Étant donné que l'image de \( \varphi_u\) est un fermé dans \( \End(E)\), il suffit de montrer que la série
    \begin{equation}
        \sum_{k=0}^{\infty}\frac{ \varphi_u(X)^k }{ k! }
    \end{equation}
    converge dans \( \End(E)\) pour qu'elle converge dans \( \Image(\varphi_u)\). Pour ce faire nous nous rappelons de la norme opérateur\footnote{Définition~\ref{DefNFYUooBZCPTr}.} et de la propriété fondamentale \( \| A^k \|\leq \| A \|^k\). En notant \( A=\varphi_u(X)\),
    \begin{equation}
        \left\| \sum_{k=n}^m\frac{ A^k }{ k! } \right\|\leq \sum_{k=n}^m\frac{ \| A^k \| }{ k! }\leq \sum_{k=n}^m\frac{ \| A \|^k }{ k! },
    \end{equation}
    ce qui est une morceau du développement de \(  e^{\| A \|}\). La limite \( n\to\infty\) est donc zéro par la convergence de l'exponentielle réelle. La suite des sommes partielles de  $e^{A}$ est donc de Cauchy. La série converge donc parce que nous sommes dans un espace vectoriel réel de dimension finie (\( \End(E)\)).
\end{proof}
% TODO : et tant qu'on y est, justifier la convergence de la série de l'exponentielle réelle.

\begin{remark}
    Pourquoi \( \exp(u)\) est-il un polynôme d'endomorphisme alors que \( \exp\) n'est pas un polynôme ? Lorsque nous disons que la fonction \( x\mapsto \exp(x)\) n'est pas un polynôme, nous sommes en train de localiser la fonction \( \exp\) à l'intérieur de l'espace de toutes les fonctions \( \eR\to \eR\), c'est à dire à l'intérieur d'un espace de dimension infinie. Au contraire lorsqu'on parle de \( \exp(u)\) et qu'on le compare aux endomorphismes \( P(u)\), nous sommes en train de repérer \( \exp(u)\) à l'intérieur de l'espace des matrices qui est de dimension finie. Il n'est donc pas étonnant que l'on parvienne moins à faire la distinction.

    Si par contre nous considérons \( \exp\) en tant qu'application \( \exp\colon \End(E)\to \End(E)\), ce n'est pas un polynôme.

    Si \( u\) et \( v\) sont des endomorphismes, nous aurons des polynômes \( P\) et \( Q\) tels que \( e^u=P(u)\) et \( e^v=Q(v)\); mais nous n'aurons en général évidemment pas \( P=Q\). En cela, \( \exp\) n'est pas un polynôme.
\end{remark}

%+++++++++++++++++++++++++++++++++++++++++++++++++++++++++++++++++++++++++++++++++++++++++++++++++++++++++++++++++++++++++++
\section{Espace dual}
%+++++++++++++++++++++++++++++++++++++++++++++++++++++++++++++++++++++++++++++++++++++++++++++++++++++++++++++++++++++++++++
\label{SECooKOJNooQVawFY}

\begin{definition}
    Soit un espace vectoriel normé \( (V,\| . \|)\) sur le corps \( \eC\) ou \( \eR\) (que nous nommons \( \eK\)). Son \defe{dual topologique}{dual topologique}, noté \( V'\) est l'ensemble des applications linéaires continues \( V\to \eK\).
\end{definition}

%---------------------------------------------------------------------------------------------------------------------------
\subsection{Topologies}
%---------------------------------------------------------------------------------------------------------------------------

Il est possible de mettre sur \( V'\) (au moins) deux topologies distinctes. La première est la topologie de la norme opérateur; rien de nouveau pour elle. La seconde est la topologie \( *\)-faible dont nous avons déjà un peu parlé dans la définition~\ref{DefHUelCDD}.

En termes de notations, nous allons noter les semi-normes de la topologie faible par
\begin{equation}
    p_x(\varphi)=| \varphi(x) |
\end{equation}
pour \( x\in V\) et \( \varphi\in V'\). À droite, les barres dénotent soit la valeur absolue (si \( \eK=\eR\)), soit le module (si \( \eK=\eC\)).

\begin{lemma}       \label{LEMooFMAUooQBIeTh}
    Soit \( \varphi\in V'\) et \( x\in V\). Alors
    \begin{equation}
        p_x(\varphi)\leq\frac{ \| \varphi \| }{ \| x \| }.
    \end{equation}
    Si \( \varphi_0\in V'\), si \( r>0\) et si \( x\in V\) nous avons aussi :
    \begin{equation}
        B(\varphi_0,r)\subset B_x(\varphi_0,\frac{ r }{ \| x \| }).
    \end{equation}
\end{lemma}

\begin{proof}
    En posant \( x'=x/\| x \|\) nous avons
    \begin{equation}
        p_x(\varphi)=| \varphi(x) |=\frac{1}{ \| x \| }| \varphi(x') |\leq \frac{1}{ \| x \| }\| \varphi \|.
    \end{equation}

    En ce qui concerne la seconde affirmation, si \( \varphi\in B(\varphi_0,r)\) alors en notant \( x'=x/\| x \|\) nous avons :
    \begin{equation}
        p_x(\varphi_0-\varphi)=| \varphi_0(x)-\varphi(x) |=\frac{1}{ \| x \| }| \varphi_0(x')-\varphi(x') |\leq\frac{1}{ \| x \| }\|\varphi_0-\varphi  \|\leq \frac{ r }{ \| x \| }.
    \end{equation}
    Donc \( \varphi\in B_x\big( \varphi_0,\frac{ r }{ \| x \| } \big)\).
\end{proof}

\begin{proposition}
    En ce qui concerne la convergence d'une suite \( (\varphi_k)\) dans \( V'\) mais si elle vérifie
    \begin{equation}
        \varphi_k\stackrel{\| . \|}{\longrightarrow}\varphi
    \end{equation}
    alors
    \begin{equation}
        \varphi_k\stackrel{*}{\longrightarrow}\varphi.
    \end{equation}
\end{proposition}

\begin{proof}
    Soit une suite \( (\varphi_k)\) dans \( V'\), convergente vers \( \varphi\) pour la topologie de la norme.  Soit \( x\in V\), et \( x'=x/\| x \|\). Nous avons
    \begin{equation}
        p_x(\varphi_k-\varphi)=\frac{1}{ \| x \| }| \varphi_k(x')-\varphi(x) |\leq\frac{1}{ \| x \| }\| \varphi_k-\varphi \|\to 0.
    \end{equation}
\end{proof}

\begin{lemma}       \label{LEMooEAVEooAFveHn}
    La translation dans \( V'\) est une opération continue pour la topologie de la norme opérateur et pour celle de la topologie \( *\).
\end{lemma}

\begin{proof}
    Soit une suite \( \varphi_k\) tendant vers \( 0\); nous devons prouver que \( \tau_{\sigma}(\varphi_k)\to \tau_{\sigma}(0)=\sigma\). Et ce, pour chacune des deux topologies.

    \begin{subproof}
        \item[Norme opérateur]

            L'hypothèse \( \varphi_k\stackrel{\| . \|}{\longrightarrow} 0\) signifie que \( \| \varphi_k \|\to 0\), c'est à dire que
            \begin{equation}
                \sup_{\| v \|=1}| \varphi_k(v) |\to 0.
            \end{equation}
            Nous avons alors
            \begin{equation}
                \| \tau_{\sigma}(\varphi_k)-\sigma \|=\sup_{\| v \|=1}| \tau_{\sigma}(\varphi_k)v-\sigma(v) |=\sup_{\| v \|=1}| \varphi_k(v) |\to 0.
            \end{equation}
            Donc d'accord pour \( \tau_{\sigma}(\varphi)\to \sigma\).

        \item[Topologie $*$]

            Nous supposons maintenant que \( \varphi_k\stackrel{*}{\longrightarrow}0\). Pour tout \( v\in V\) nous avons
            \begin{equation}
                p_v\big( \tau_{\sigma}(\varphi_k)-\sigma \big)=\big| \tau_{\sigma}(\varphi_k)v-\sigma(v) \big|=| \varphi_k(v) |=p_v(\varphi_k).
            \end{equation}
            Mais par hypothèse, \( p_v(\varphi_k)\to 0\).
    \end{subproof}
\end{proof}

Pour la suite, nous allons préfixer par \( N\) les concepts liés à la topologie de \( V'\) associée à la norme opérateur et par \( *\), les concepts de la topologie \( *\).

\begin{proposition}     \label{PROPooFGXAooFRWweD}
    Soit un espace vectoriel normé \( V\). Un \( *\)-ouvert et toujours un \( N\)-ouvert.
\end{proposition}

\begin{proof}
    Soit un \( *\)-ouvert \( \mO\) de \( V'\). Il existe donc \( x\in V\) et \( r>0\) tels que \( B_x(\varphi,r)\subset \mO\). Nous avons alors, en utilisant le lemme~\ref{LEMooFMAUooQBIeTh},
    \begin{equation}
        B(\varphi,r\| x \|)\subset B_x(\varphi,r)\subset \mO.
    \end{equation}
    Donc \( \mO\) est un \( N\)-ouvert.
\end{proof}

\begin{corollary}
    Soit un espace topologique \( X\). Si \( f\colon (V',*)\to X\) est continue, alors \( f\colon (V',\| . \|)\to X\) est continue.
\end{corollary}

\begin{proof}
    Soit un ouvert \( \mO\) de \( X\). Vu que \( f\) est \( *\)-continue, la partie \( f^{-1}(\mO)\) est un \( *\)-ouvert de \( V'\). Il est onc un \( N\)-ouvert de \( V'\) par la proposition~\ref{PROPooFGXAooFRWweD}.
\end{proof}

%---------------------------------------------------------------------------------------------------------------------------
\subsection{Réflexivité}
%---------------------------------------------------------------------------------------------------------------------------

Pour la suite nous notons \( V''\) le dual de \( (V',\| . \|)\). Certes en tant qu'ensembles, \( (V',*)\) et \( (V',\| . \|) \) sont identiques, mais comme ils n'ont pas la même topologie, les duaux ne sont pas les mêmes.

Bref, \( V''\) est l'ensemble des applications linéaires continues \( (V',\| . \|)\to \eC\). Et lorsque nous disons \( \eC\) ici, ça peut aussi bien être \( \eR\) selon le contexte.

De plus nous considérons que \( V''\) la norme opérateur qui dérive de la norme de \( V'\), laquelle dérive de la norme vectorielle sur \( V\).

\begin{propositionDef}      \label{PROPooMAQSooCGFBBM}
    Soit un espace vectoriel normé $V$ sur $\eR$ ou $\eC$. Nous considérons l'application
    \begin{equation}
        \begin{aligned}
            J\colon V&\to V'' \\
            J(x)\varphi&= \varphi(x).
        \end{aligned}
    \end{equation}
    \begin{enumerate}
        \item       \label{ITEMooNVVSooNFXgnE}
            L'application \( J\) est bien définie : \( J(x)\) est continue.
        \item       \label{ITEMooKURHooZZWpbu}
            L'application \( J\) est continue.
        \item       \label{ITEMooTFYVooKhMOjp}
             Elle est injective.
    \end{enumerate}

    Lorsque \( J\) est bijective, l'espace \( V\) est dit \defe{réflexif}{réflexif}.
\end{propositionDef}

\begin{proof}
    Point par point.
    \begin{subproof}
        \item[\ref{ITEMooNVVSooNFXgnE}]
            Nous commençons par montrer que \( J(x)\colon (V',\| . \|)\to \eC\) est continue pour chaque \( x\in V\). Soit une suite \( \varphi_k\stackrel{\| . \|}{\longrightarrow}0\). Nous avons :
            \begin{equation}
                J(x)\varphi_k=\varphi_k(x)\leq \| \varphi_k \|\| x \|\to 0
            \end{equation}
            où vous aurez noté l'utilisation du lemme~\ref{LEMooIBLEooLJczmu}.  Cela prouve que \( J(x)\) est continue et donc que \( J\) est bien à valeurs dans \( V''\).
        \item[\ref{ITEMooKURHooZZWpbu}]

            Soit une suite \( x_k\stackrel{V}{\longrightarrow}0\), et étudions \( \| J(x_k) \|\) pour la norme dans \( V''\). Nous posons \( x'_k=x_k/\| x_k \|\) et nous calculons (encore une fois, nous écrivons «\( \eC\)», mais ça pourrait être \( \eR\))
            \begin{equation}
                \| J(x_k) \|=\sup_{\| \varphi \|=1}| J(x_k)\varphi |_{\eC}=\sup_{\| \varphi \|=1}| \varphi(x_k) |=\| x_k \|\sup_{\| \varphi \|=1}| \varphi(x'_k) |\leq \| x_k \|\to 0.
            \end{equation}
            La dernière inégalité pourrait être sans doute une égalité\quext{Écrivez moi si vous en êtes certain.}, mais nous n'en avons pas besoin ici.
    \end{subproof}
\end{proof}

%+++++++++++++++++++++++++++++++++++++++++++++++++++++++++++++++++++++++++++++++++++++++++++++++++++++++++++++++++++++++++++
\section{Mini introduction aux nombres \texorpdfstring{p}{$p$}-adiques}
%+++++++++++++++++++++++++++++++++++++++++++++++++++++++++++++++++++++++++++++++++++++++++++++++++++++++++++++++++++++++++++

\subsection{La flèche d'Achille}\label{s:un}

C'est un grand classique que je donne ici juste comme introduction pour montrer que des séries infinies peuvent donner des nombres finis de manière tout à fait intuitive.

Achille tire une flèche vers un arbre situé à $\unit{10}{\meter}$ de lui. Disons que la flèche avance à une vitesse constante de $\unit{1}{\meter\per\second}$. Il est clair que la flèche mettra $\unit{10}{\second}$ pour toucher l'arbre. En $\unit{5}{\second}$, elle aura parcouru la moitié de son chemin. On le note :
\[
\text{temps}=5s+\ldots
\]
Reste \( \unit{5}{\meter}\) à faire. En $\unit{2.5}{\second}$, elle aura fait la moitié de ce chemin chemin, soit $2.5m=\frac{10}{4}m$. On le note :
\[
\text{temps}=\frac{10}{2}s+\frac{10}{4}s+
\]
Reste $2.5m$ à faire. La moitié de ce trajet, soit $\frac{10}{8}m$, est parcouru en $\frac{10}{8}s$; on le note encore, mais c'est la dernière fois !

\[
\text{temps}=\frac{10}{2}s+\frac{10}{4}s+\frac{10}{8}s+
\]
En continuant ainsi à regarder la flèche qui parcours des demi-trajets puis des demi de demi-trajets et encore des demi de demi de demi-trajets, et en sachant que le temps total est $10s$, on trouve :
\[
10\left( \frac{1}{2}+\frac{1}{4}+\frac{1}{8}+\frac{1}{16}+\ldots  \right)=10.
\]
On doit donc croire que la somme jusqu'à l'infini des inverses des puissances de deux vaut $1$ :
\[
   \sum_{n=1}^{\infty}\frac{1}{2^n}=1.
\]
Cela peut être démontré à la loyale.

\subsection{La tortue et Achille}

Maintenant qu'on est convaincu que des sommes infinies peuvent représenter des nombres tout à fait normaux, passons à un truc plus marrant.

Achille, qui marche peinard à $\unit{10}{\meter\per\hour}$, part avec $1m$ d'avance sur une tortue qui avance à $\unit{1}{\meter\per\hour}$. Le temps que la tortue arrive au point de départ d'Achille, Achille aura parcouru $10m$, et le temps que la tortue mettra pour arriver à ce point, eh bien, Achille ne sera déjà plus là : il sera à $100m$. Si la tortue tient bon pendant un temps infini, et si l'on est confiant en le genre de raisonnements faits à la section~\ref{s:un}, elle rattrapera Achille dans
\[
1m+10m+100m+1000m+\ldots
\]
Autant dire que ça ne risque pas d'arriver. Et pourtant, mettons en équations :
\begin{subequations}
    \begin{numcases}{}
        x_{\text{Achile}}(t)=1+10t\\
        x_{\text{tortue}}(t)=t.
    \end{numcases}
\end{subequations}
La tortue rejoint Achille au temps \( t\) tel que \( x_{\text{Achille}(t)}=x_{\text{tortue}}(t)\). Un mini calcul donne $t=-1/9$. Physiquement, c'est une situation logique. Peut-on en déduire une égalité mathématique du style de
\[
1+10+100+1000+\ldots=-\frac{1}{9}\; ???
\]
Là où les choses deviennent jolies, c'est quand on cherche à voir ce que peut bien être la valeur d'un hypothétique $x=1+10+100+1000+\ldots$. En effet, logiquement on devrait avoir
\begin{equation*}
\begin{split}
\frac{x}{10}&=\frac{1}{10}+1+10+100+\ldots\\
            &=\frac{1}{10}+x.
\end{split}
\end{equation*}
Reste à résoudre l'équation du premier degré : $\frac{x}{10}=x+\frac{1}{10}$. Ai-je besoin de donner la solution ?

%---------------------------------------------------------------------------------------------------------------------------
\subsection{Dans les nombres \texorpdfstring{p}{$ p$}-adiques, c'est vrai}
%---------------------------------------------------------------------------------------------------------------------------

Nous nous proposons d'apprendre sur les nombres \( p\)-adiques juste ce qu'il faut pour montrer que l'égalité
\begin{equation}
    \sum_{k=0}^{\infty}10^k=-\frac{1}{ 9 }
\end{equation}
est vraie dans les nombres \( 5\)-adiques. Tout ce qu'il faut est sur \wikipedia{fr}{Nombre_p-adique}{wikipedia}.

Soit \( a\in \eN\) et \( p\), un nombre premier. La \defe{valuation}{valuation!$p$-adique} \( p\)-adique de \( a\) est l'exposant de \( p\) dans la décomposition de \( a\) en nombres premiers. On la note \( v_p(a)\). Pour un rationnel on définit
\begin{equation}
    v_p\left( \frac{ a }{ b } \right)=v_p(a)-v_p(b)
\end{equation}
La \defe{valeur absolue}{valeur absolue!$p$-adique} \( p\)-adique de \( r\in \eQ\) est
\begin{equation}
    | r |_p=p^{-v_p(r)}.
\end{equation}
Nous posons \( | 0 |_p=0\). De là nous considérons la distance
\begin{equation}
    d_p(x,y)=| x-y |_p.
\end{equation}

\begin{lemma}
    L'espace \( (\eQ,d_p)\) est un espace métrique\footnote{Définition~\ref{DefMVNVFsX}}.
\end{lemma}
\index{topologie!\( p\)-adique}

Nous considérons maintenant \( p=5\). Étant donné que \( a=5\cdot 2\) nous avons \( v_5(10)=1\) et
\begin{equation}
    v_5\left( \frac{1}{ 9 } \right)=v_5(1)-v_5(9)=0.
\end{equation}
Nous avons
\begin{equation}
    \sum_{k=0}^N10^k+\frac{1}{ 9 }=\frac{ 10^{N+1} }{ 9 }
\end{equation}
mais
\begin{equation}
    v_p\left( \frac{ 10^{N+1} }{ 9 } \right)=v_5(10^{N+1})-v_5(9)=N+1.
\end{equation}
Par conséquent
\begin{equation}
    d_5\big( \sum_{k=0}^N10^k,-\frac{1}{ 9 } \big)=| \frac{ 10^{N+1} }{ 9 } |_p=p^{-(N+1)}.
\end{equation}
En passant à la limite,
\begin{equation}
    \lim_{N\to \infty} d_5\big( \sum_{k=0}^N10^k,-\frac{1}{ 9 } \big)=0,
\end{equation}
ce qui signifie que\footnote{Voir la définition~\ref{DefGFHAaOL} de la convergence d'une série dans un espace métrique.}
\begin{equation}
    \sum_{k=0}^{\infty}10^k=-\frac{1}{ 9 }.
\end{equation}
