% This is part of Le Frido
% Copyright (c) 2008-2017
%   Laurent Claessens
% See the file fdl-1.3.txt for copying conditions.


%+++++++++++++++++++++++++++++++++++++++++++++++++++++++++++++++++++++++++++++++++++++++++++++++++++++++++++++++++++++++++++
\section{Suites}
%+++++++++++++++++++++++++++++++++++++++++++++++++++++++++++++++++++++++++++++++++++++++++++++++++++++++++++++++++++++++++++
\label{SECooLLUGooOwZRyI}

%--------------------------------------------------------------------------------------------------------------------------- 
\subsection{Limites, convergence}
%---------------------------------------------------------------------------------------------------------------------------

Nous disons qu'une suite réelle $(x_n)$ converge\footnote{Voir la définition \ref{PropLimiteSuiteNum} pour plus de détail.} vers $\ell$ lorsque pour tout $\varepsilon$, il existe un $M$ tel que
\begin{equation}
	n>N\Rightarrow | x_n-\ell |\leq\varepsilon.
\end{equation}
Le concept fondamental de cette définition est la notion de valeur absolue qui permet de donner la «distance» entre deux réels. Dans un espace vectoriel normé quelconque, cette notion est généralisée par la distance associée à la norme (définition \ref{DefEVNetDistance}). Nous pouvons donc facilement définir le concept de convergence d'une suite dans un espace vectoriel normé.

\begin{definition}		\label{DefCvSuiteEGVN}
	Soit une suite $(x_n)$ dans un espace vectoriel normé $V$. Nous disons qu'elle est
    \defe{convergente}{convergence!dans un espace vectoriel normé} s'il existe un élément $\ell\in V$ tel que
	\begin{equation}
		\forall \varepsilon>0,\,\exists N\in\eN\tq n\geq N\Rightarrow \| x_n-l \|<\varepsilon.
	\end{equation}
	Dans ce cas, $\ell$ est appelé la \defe{limite}{limite!suite} de la suite $(x_n)$.
\end{definition}


\begin{lemma}		\label{LemLimAbarA}
	Soit $(x_n)$ une suite convergente contenue dans un ensemble $A\subset V$. Alors la limite $x_n$ appartient à $\bar A$.
\end{lemma}

\begin{proof}
	Supposons que nous ayons une partie $A$ de $V$, et une suite $(x_n)$ dont la limite $\ell$ se trouve hors de $\bar A$. Dans ce cas, il existe un $r>0$ tel que\footnote{Une autre manière de dire la même chose : si $\ell\notin\bar A$, alors $d(\ell,A)>0$.} $B(\ell,r)\cap A=\emptyset$. Si tous les éléments $x_n$ de la suite sont dans $A$, il n'y en a donc aucun tel que $d(x_n,\ell)=\| x_n-\ell \|<r$. Cela contredit la notion de convergence $x_n\to \ell$.
\end{proof}

Nous avons déjà mentionné dans l'exemple \ref{ParlerEncoredeF} que zéro était un point adhérent à l'ensemble $F=\{ (-1)^n/n\tq n\in\eN_0 \}$. Nous savons maintenant que $0$ étant la limite de la suite, il est automatiquement adhérent à l'ensemble des éléments de la suite.

\begin{corollary}		\label{CorAdhEstLim}
	Soit $a$ un point de l'adhérence d'une partie $A$ de $V$. Alors il existe une suite d'éléments dans $A$ qui converge vers $a$.
\end{corollary}

\begin{proof}
	Si $a\in A$, alors nous pouvons prendre la suite constante $x_n=a$. Si $a$ n'est pas dans $A$, alors $a$ est dans $\partial A$, et pour tout $n$, il existe un point de $A$ dans la boule $B(a,\frac{1}{ n })$. Si nous nommons $x_n$ ce point, la suite ainsi construite est une suite contenue dans $A$ et qui converge vers $a$ (ce dernier point est laissé à la sagacité du lecteur ou de la lectrice).
\end{proof}

En termes savants, ce corollaire signifie que la fermeture $\bar A$ est composé de $A$ plus de toutes les limites de toutes les suites contenues dans $A$.

%--------------------------------------------------------------------------------------------------------------------------- 
\subsection{Critère de Cauchy}
%---------------------------------------------------------------------------------------------------------------------------


\begin{lemma}
    Une suite de Cauchy dans un espace vectoriel normé admettant une sous-suite convergente est elle-même convergente vers la même limite.
\end{lemma}

\begin{proof}
    Soit \( (a_n)\) une suite de Cauchy dans un espace vectoriel normé \( E\) et \( \ell\) la limite d'une sous-suite de \( (a_n)\). Soit \( \epsilon>0\) et \( N\in \eN\) tel que \( \| a_m-a_p \|<\epsilon\) dès que \( m,p\geq N\). Nous allons montrer que si \( k>N\) alors \( \| a_k-\ell \|<2\epsilon\). Pour cela nous considérons un \( n>N\) tel que \( \| a_n-\ell \|\leq \epsilon\) et nous calculons
    \begin{equation}
        \| a_k-\ell \|\leq \| a_k-a_n \|+\| a_n-\ell \|\leq 2\epsilon.
    \end{equation}
\end{proof}

\begin{definition}
    Nous disons que deux suites \( (u_n)\) et \( (v_n)\) sont \defe{équivalentes}{equivalence@équivalence!de suites} s'il existe une fonction \( \alpha\colon \eN\to \eR\) telle que
    \begin{enumerate}
        \item
            pour tout \( n\) à partir d'un certain rang, \( u_n=v_n\alpha(n)\)
        \item
            \( \alpha(n)\to 1\).
    \end{enumerate}
\end{definition}

\begin{lemma}
    Si les suites \( (u_n)\) et \( (v_n)\) sont équivalentes et si \( (v_n)\) admet une limite \( l\) différente de \( 1\), alors les suites \( (\ln u_n)\) et \( (\ln v_n)\) sont équivalentes.
\end{lemma}

\begin{proof}
    En effet si \( u_n=v_n\alpha(n)\) alors
    \begin{equation}
        \ln(u_n)=\ln(v_n)+\ln\big( \alpha(n) \big)=\ln(v_n)\left( 1+\frac{ \ln\big( \alpha(n) \big) }{ \ln(v_n) } \right),
    \end{equation}
    et comme \( \alpha(n)\to 1\), la parenthèse tend vers \( 1\).
\end{proof}

\begin{lemma}[Formule de Stirling\cite{MEHuVnb}]        \label{LemCEoBqrP}
    Nous avons l'équivalence de suites
    \begin{equation}
        n!\sim \left( \frac{ n }{ e } \right)^n\sqrt{2\pi n}.
    \end{equation}
\end{lemma}
\index{formule!Stirling}

Dans le cas des espaces de dimension finie, le fait d'être complet est automatique, comme le montre la proposition suivante.
\begin{proposition}     \label{PROPooGJDTooXOoYfw}
    Soit \( \big( E,\| . \| \big)\) un espace vectoriel normé de dimension finie sur un corps \( \eK\) qui est complet\footnote{La définition est \ref{DefKCGBooLRNdJf}, mais si vous n'avez pas envie de vous embarquer trop loin, dites juste «toutes les suites de Cauchy convergent». Typiquement c'est \( \eR\) ou \( \eC\).}. Alors \( E\) est complet\footnote{Définition \ref{DEFooHBAVooKmqerL}.}.
\end{proposition}
Pour rappel, la complétude de l'espace métrique \( \eR\) est la proposition \ref{PROPooTFVOooFoSHPg}.

\begin{proof}
    Nous considérons une suite de Cauchy \( (f_n)\) dans \( E\) et si \( \{ e_{\alpha} \} \) est une base orthonormée de \( E\) nous définissons les coefficients \( f_n=\sum_{\alpha}a_{n\alpha}e_{\alpha} \). La somme sur \( \alpha\) est finie par hypothèse sur la dimension de \( E\).

    Nous avons
    \begin{equation}
        \| f_n-f_m \|=\| \sum_{\alpha}(a_{n\alpha}-a_{m\alpha})e_{\alpha} \|=\sum_{\alpha}| a_{n\alpha}-a_{m\alpha} |^2.
    \end{equation}
    Pour tout \( \epsilon\), il existe \( N\) tel que si \( m,n>N\) alors \( | a_{n\alpha}-a_{m\alpha} |<\sqrt{ \epsilon }\). Autrement dit, pour chaque \( \alpha\), la suite \( (a_{n\alpha})_{\alpha\in \eN}\) est de Cauchy dans \( \eK\) et converge donc dans \( \eK\). Soit \( a_{\alpha}\) la limite et définissons \( f=\sum_{\alpha}a_{\alpha}e_{\alpha}\). Nous avons alors
    \begin{equation}
        \| f_n-f \|=\| \sum_{\alpha}(a_{n\alpha}-a_{\alpha})e_{\alpha} \|,
    \end{equation}
    dont la limite \( n\to \infty\) est bien zéro. Donc la suite \( (f_n)\) converge vers \( f\in E\). L'espace \( E\) est alors complet.
\end{proof}

%--------------------------------------------------------------------------------------------------------------------------- 
\subsection{Approximation}
%---------------------------------------------------------------------------------------------------------------------------

Le lemme suivant est surtout intéressant en dimension infinie.
\begin{lemma}
    Soit un espace vectoriel normé \( V\) et un sous-espace vectoriel dense \( A\). Soit \( v\in V\); il existe une suite \( (v_n)\) dans \( A\) telle que \( v_n\stackrel{V}{\longrightarrow}v\) et \( \| v_n \|\leq \| v \|\) pour tout \( n\).
\end{lemma}

\begin{proof}
    Vu que \( A\) est dense, il existe une suite \( a_n\) dans \( A\) telle que \( a_n\to v\). Ensuite il suffit de poser
    \begin{equation}
        v_n=\frac{ n }{ n+1 }\frac{ \| v \| }{ \| a_n \| }a_n.
    \end{equation}
    Par construction nous avons toujours
    \begin{equation}
        \| v_n \|=\frac{ n }{ n+1 }\| v \|\leq \| v \|.
    \end{equation}
    Et de plus, la norme étant continue\footnote{Où dans le calcul suivant nous utilisons la continuité de la norme ? Posez-vous la question.},
    \begin{equation}
        \lim_{n\to \infty} v_n=\lim_{n\to \infty} \frac{ n }{ n+1 }\lim_{n\to \infty} \frac{ \| v \| }{ \| v_n \| }\lim_{n\to \infty} v_n=v.
    \end{equation}

    Le fait que \( v_n\) soit dans \( A\) est dû au fait que \( A\) soit vectoriel.
\end{proof}

\begin{proposition}     \label{PROPooVEMGooYKhMFy}
    Soit un espace vectoriel normé \( V\) et un sous-espace vectoriel dense \( A\). Soit \( v\in V\); pour tout \( a\in \eR\) nous avons
    \begin{equation}
        \sup\{ | v\cdot a |\tq a\in A\text{ et }\| a \|\leq \lambda \}=\lambda\| v \|.
    \end{equation}
\end{proposition}

\begin{proof}
    D'abord pour tout \( a\in A\) vérifiant \( \| a \|\leq \lambda\) l'inégalité de Cauchy-Schwarz \ref{ThoAYfEHG} donne
    \begin{equation}
        | v\cdot a |\leq \| v \|\| a \|\leq \lambda\| v \|.
    \end{equation}
    Donc le supremum dont on parle est majoré par \( \lambda\| v \|\).

    Il nous faut l'inégalité dans l'autre sens. Par densité nous pouvons choisir une suite \( v_n\in A\) tel que \( v_n\to v\). Ensuite nous posons
    \begin{equation}
        a_n=\frac{ \lambda }{ \| v_n \| }v_n.
    \end{equation}
    Nous avons \( \| a_n \|=\lambda\) pour tout \( n\) et
    \begin{equation}
        | v\cdot a_n |=\frac{ \lambda }{ \| v_n \| }| v\cdot v_n |,
    \end{equation}
    et en passant à la limite,
    \begin{equation}
        \lim_{n\to \infty} | v\cdot a_n |=\frac{ \lambda }{ \| v \| }\| v\cdot v \|=\lambda\| v \|.
    \end{equation}
    Donc l'ensemble sur lequel nous prenons le supremum contient une suite convergente vers \( \lambda\| v \|\). Le supremum est donc au moins aussi grand que cela.
\end{proof}

%+++++++++++++++++++++++++++++++++++++++++++++++++++++++++++++++++++++++++++++++++++++++++++++++++++++++++++++++++++++++++++ 
\section{Séries}
%+++++++++++++++++++++++++++++++++++++++++++++++++++++++++++++++++++++++++++++++++++++++++++++++++++++++++++++++++++++++++++
\label{SECooYCQBooSZNXhd}

\begin{definition}\label{DefGFHAaOL}
    Soit \( (a_k)\) une suite dans un espace vectoriel normé \( (V,\| . \| )\). La suite des \defe{sommes partielles}{somme!partielle} associée est la suite \( (s_k)\) définie par 
    \begin{equation}
        s_k=\sum_{i=0}^ka_i
    \end{equation}
    La \defe{série}{série!dans un espace vectoriel normé} associée est la limite des sommes partielles
    \begin{equation}
        \sum_{n=0}^{\infty}a_k=\lim_{k\to \infty} \sum_{k=0}^na_k
    \end{equation}
    si elle existe.

    Si une telle limite existe nous disons que \( \sum_{k=0}^{\infty}a_k\) \defe{converge}{convergence!série} dans \( V\). Si la limite de la suite des sommes partielles n'existe pas nous disons que la série \defe{diverge}{série!divergence}.
\end{definition}

\begin{remark}
    Si la limite de la suite des sommes partielles n'existe pas dans \( V\), alors elle peut parfois exister dans des extensions de \( V\). Par exemple une série de rationnels convergeant vers \( \sqrt{2}\) dans \( \eR\) ne converge pas dans \( \eQ\). Autre exemple : avec une bonne topologie sur \( \bar \eR\), une série peut ne pas converger dans \( \eR\) mais converger vers \( \pm\infty\) dans \( \bar \eR\).
\end{remark}

Dans le cas des espaces de fonctions, nous avons une norme importante : la norme uniforme définie par \( \| f \|_{\infty}=\sup\{ f(x) \}\) où le supremum est pris sur l'ensemble de définition de \( f\).
\begin{definition}[Convergence absolue] \label{DefVFUIXwU}
    Nous disons que la série \( \sum_{n=0}^{\infty}a_n\) dans l'espace vectoriel normé \( V\) \defe{converge absolument}{convergence!absolue} si la série \( \sum_{n=0}^{\infty}\| a_n \|\) converge dans \( \eR\).
\end{definition}

La convergence absolue dans le cas d'un espace de fonctions muni de la norme uniforme s'appelle la convergence normale.

\begin{definition}[Convergence normale] \label{DefVBrJUxo}
    Une série de fonctions \( \sum_{n\in \eN}u_n \) converge \defe{normalement}{convergence!normale} si la série de nombres \( \sum_n\| u_n \|_{\infty}\) converge. C'est à dire si la série converge absolument pour la norme \( \| f \|_{\infty}\).
\end{definition}

La convergence normale est à ne pas confondre avec la convergence uniforme. 

\begin{definition}[Convergence uniforme]
    La somme \( \sum_nf_n\) \defe{converge uniformément}{convergence!uniforme!série de fonctions} vers la fonction \( F\) si la suite des sommes partielles converge uniformément, c'est à dire si 
    \begin{equation}        \label{EqLNCJooVCTiIw}
        \lim_{N\to \infty} \| \sum_{n=1}^Nf_n-F \|_{\infty}=0.
    \end{equation}
\end{definition}

\begin{proposition} \label{PropAKCusNM}
    Une série convergeant absolument dans un espace de Banach\footnote{Un espace vectoriel normé complet. Typiquement \( \eR\).} y converge au sens usuel.
\end{proposition}

\begin{proof}
    Soit \( (a_k)\) une suite dans un espace vectoriel normé complet dont la série converge absolument. Nous allons montrer que la suite des sommes partielles est de Cauchy. Cela suffira à montrer sa convergence par hypothèse de complétude.

    Nous avons
    \begin{equation}
        \| s_p-s_l \|=\| \sum_{k=l+1}^{p}a_k\|  \leq\sum_{k=l+1}^p\| a_k \|=\bar s_p-\bar s_l
    \end{equation}
    où \( \bar s_n=\sum_{k=0}^n \| a_k \|\) est la suite des sommes partielles de la série des normes (qui converge). Vu que la suite \( (\bar s_n)\) converge dans \( \eR\), elle y es de Cauchy par la proposition \ref{PROPooTFVOooFoSHPg}. Donc il existe un \( N\) tel que \( p,l>N\) implique
    \begin{equation}
        \| s_p-s_l \|=\bar s_p-\bar s_l\leq \epsilon.
    \end{equation}
    Cela signifie que \( (s_n)\) est une suite de Cauchy et donc convergente.
\end{proof}

\begin{example}[Si l'espace n'est pas complet\cite{MonCerveau}]
    Dans un espace pas complet, il est possible de construire un série qui converge absolument sans converger au sens usuel.

    Nous allons trouver dans \( \eQ\) une série qui converge simplement vers \( \sqrt{ 2 }\) (et donc ne converge pas dans \( \eQ\)) mais absolument vers \( 4\).

    La base est que si \( A,B\in \eQ\) avec \( A<B\) il est possible de résoudre
    \begin{subequations}
        \begin{numcases}{}
            r_1+r_2=A\\
            | r_1 |+| r_2 |=B
        \end{numcases}
    \end{subequations}
    pour \( r_1,r_2\in \eQ\). Ce n'est pas très compliqué : la solution est \( r_1=(A+B)/2\) et \( r_2=(A-B)/2\).

    Nous considérons l'espace \( \eQ\) qui n'est pas complet dans \( \eR\). Soit une série \( (a_k)\) dans \( \eQ\) qui converge vers \( \sqrt{ 2 }\) (convergence dans \( \eR\)) nous nommons \( (s_k)\) la suite des ses sommes partielles. Soit aussi la suite \( (b_k\) qui converge vers \( 4\) (zéro serait encore plus facile mais bon, juste pour faire un peu de généralité).

    Nous supposons que \( a_k<b_k\) pour tout \( k\) et que les deux suites sont constituées de rationnels positifs. Nous nommons \( (s_k)\) et \( (s'_k)\) les sommes partielles. En particulier \( s_n<s'_n\) et ce sont des suites croissantes.

    Nous savons comment trouver \( r_1,r_2\in \eQ\) tels que \( r_1+r_2=s_1\) et \( | r_1 |+| r_2 |=s'_1\). Par récurrence, si nous savons \( r_1,\ldots, r_k\) tels que
    \begin{subequations}
        \begin{numcases}{}
            r_1+\ldots +r_k=s_n\\
            |r_1|+\ldots +|r_k|=s'_n
        \end{numcases}
    \end{subequations}
    (avec, soit dit en passant \( k=2n\)), alors nous pouvons trouver des rationnels \( r_{k+1}\), \( r_{k+2}\) tels que
    \begin{subequations}
        \begin{numcases}{}
            r_1+\ldots +r_k+r_{k+1}+r_{k+2}=s_{n+1}\\
            |r_1|+\ldots +|r_k|+|r_{k+1}|+|r_{k+2}|=s'_{n+1},
        \end{numcases}
    \end{subequations}
    en effet il s'agit de résoudre
    \begin{subequations}
        \begin{numcases}{}
            r_{k+1}+r_{k+2}=s_{n+1}-r_1-\ldots-r_k=s_{n+1}-s_n>0\\
            | r_{k+1} |+| r_{k+2} |=s'_{n+1}-| r_1 | -\ldots -| r_k |=s'_{n+1}-s'_n>0.
        \end{numcases}
    \end{subequations}
    Cela se résout comme plus haut. Au final nous pouvons construire une suite \( (r_k)\) dans \( \eQ\) telle que
    \begin{equation}
        \sum_{k=0}^{2n}r_k=s_n
    \end{equation}
    et
    \begin{equation}
        \sum_{k=0}^{2n}| r_k |=s'_n.
    \end{equation}
\end{example}

\begin{remark}
    Nous savons que sur les espaces vectoriels de dimension finie toutes les normes sont équivalentes (théorème \ref{DefEquivNorm}). La notion de convergence de série ne dépend alors pas du choix de la norme. Il n'en est pas de même sur les espaces de dimension infinie. Une série peut converger pour une norme mais pas pour une autre.
\end{remark}
Lorsque nous verrons la convergence de séries, nous verrons que la convergence normale est la convergence absolue pour la norme uniforme.

\begin{lemma}       \label{LemCAIPooPMNbXg}
    Si \( E\) et \( F\) sont des espaces de Banach\quext{Je crois qu'il ne faut pas que \( E\) soit complet.}, l'espace \( \aL(E,F)\) est également de Banach.
\end{lemma}

\begin{proof}
    Soit \( (u_n)\) une suite de Cauchy dans \( \aL(E,F)\); si \( x\in E\) il existe \( N\) tel que si \( l,m>N\) alors \( \| a_l-a_m \|<\epsilon\), c'est à dire que pour tout \( \| x \|=1\) on a \( \| u_l(x)-u_n(x) \|<\epsilon\). Cela signifie que \( u_n(x)\) est une suite de Cauchy dans l'espace complet \( F\). Cette suite converge et nous pouvons définir l'application \( u\colon E\to F\) par
    \begin{equation}
        u(x)=\lim_{n\to \infty} u_n(x).
    \end{equation}
    Il suffit maintenant de prouver que \( u\) est linéaire, ce qui est une conséquence directe de la linéarité de la limite :
    \begin{equation}
        u(\alpha x+\beta y)=\lim_{n\to \infty} \big( \alpha u_n(x)+\beta u_n(y) \big).
    \end{equation}
\end{proof}

\begin{proposition}[Thème \ref{THEMEooPQKDooTAVKFH}]     \label{PropQAjqUNp}
    Soit \( E\) un espace de Banach (espace vectoriel normé complet). Si \( A\) est un endomorphisme de \( E\) satisfaisant  \( \| A \|<1\) pour la norme opérateur, alors \( (\mtu-A)\) est inversible et son inverse est donné par
    \begin{equation}
        (\mtu-A)^{-1}=\sum_{k=0}^{\infty}A^k.
    \end{equation}
\end{proposition}
\index{série!donnant \( (1-A)^{-1}\)}

\begin{proof}
    Étant donné que la norme opérateur est une norme algébrique (lemme \ref{LEMooFITMooBBBWGI}), nous avons \( \| A^k \|\leq \| A \|^k\). Par conséquent la série \( \| A^k \|\) est majorée par la série géométrique qui converge. Par conséquent \( \sum_{k}A^k\) est une série absolument convergence et donc convergente par la proposition \ref{PropAKCusNM} et le fait que \( \aL(E)\) est complet (proposition \ref{LemCAIPooPMNbXg}).
    
    Montrons à présent que la somme est l'inverse de \( \mtu-A\) en utilisant le produit terme à terme autorisé par la proposition \ref{PropQXqEPuG} :
    \begin{equation}
        \sum_{k=0}^nA^k(\mtu-A)=\sum_{k=0}^n(A^k-A^{k+1})=\mtu-A^{n+1}.
    \end{equation}
    Par conséquent 
    \begin{equation}
        \| \mtu-\sum_{k=0}^nA^k(\mtu-A) \|=\| A^{n+1} \|\leq \| A \|^{n+1}\to 0.
    \end{equation}
\end{proof}

\begin{proposition}  \label{PROPooYDFUooTGnYQg}   %   \label{propnseries_propdebase}
    Si une série converge dans un espace complet, la norme de son terme général converge vers $0$.
\end{proposition}

\begin{proof}
    Soit \( \epsilon>0\). Si une série converge dans un espace complet, sa suite des sommes partielles est de Cauchy. En particulier il existe un \( N\) tel que si \( n>N\), nous avons \( \| s_n-s_{n-1} \|<\epsilon\). Or évidemment \( s_ -s_{n-1}=a_n\). Donc pour \( n>N\) nous avons \( \| a_n \|<\epsilon\).
\end{proof}

\begin{proposition}
    Si la série converge alors la somme est associative
\end{proposition}

\begin{proof}
    Associativité. Supposons que \( \sum_ka_k\) et \( \sum_kb_k\) convergent tous deux. Alors nous avons pour tout \( N\) :
    \begin{equation} 
        \sum_{k=0}^N(a_k+b_k)=\sum_{k=0}^Na_k+\sum_{k=0}^Nb_k.
    \end{equation}
    Mais si deux limites existent alors la somme commute avec la limite. C'est le cas pour la limite \( N\to \infty\), donc
    \begin{equation}
        \lim_{N\to \infty} \sum_{k=1}^{\infty}(a_k+b_k)=\lim_{N\to \infty} \sum_{k=0}^{\infty}a_k+\lim_{N\to \infty} \sum_{k=0}^{\infty}b_k.
    \end{equation}
\end{proof}

%+++++++++++++++++++++++++++++++++++++++++++++++++++++++++++++++++++++++++++++++++++++++++++++++++++++++++++++++++++++++++++
\section{Série réelle}
%+++++++++++++++++++++++++++++++++++++++++++++++++++++++++++++++++++++++++++++++++++++++++++++++++++++++++++++++++++++++++++
\label{secseries}

La notion de série formalise le concept de somme infinie. L'absence de certaines propriétés de ces objets (problèmes de commutativité et même d'associativité) incitent à la prudence et montrent à quel point une définition précise est importante. 


\subsection{Critères de convergence absolue}

Étant donné le terme général d'une série, il est souvent --dans les cas qui nous intéressent-- difficile de déterminer la somme de la série. L'exemple de la série géométrique est particulier, puisqu'on connait une formule pour chaque somme partielle, mais pour l'exemple des séries de Riemann il n'y a aucune formule simple pour un $\alpha$ général. D'où l'intérêt d'avoir des critères de convergence ne nécessitant aucune connaissance de l'éventuelle limite de la série.

\begin{lemma}[Critère de comparaison]   \label{LemgHWyfG}
Soient $\sum_i a_i$ et $\sum_j
b_j$ deux séries à termes positifs vérifiant
\begin{equation*}
  0 \leq a_i \leq b_i
\end{equation*}
alors
\begin{enumerate}
\item si $\sum_i a_i$ diverge, alors $\sum_j b_j$ diverge,
\item si $\sum_j b_j$ converge, alors $\sum_i a_i$ converge
  (absolument).
  \end{enumerate}
\end{lemma}


\begin{proposition}[Critère d'équivalence\cite{TrenchRealAnalisys}]
 Soient $\sum_i a_i$ et $\sum_j b_j$ deux séries à termes positifs. Supposons l'existence de la limite (éventuellement infinie) suivante
\begin{equation}
  \limite i \infty \frac{a_i}{b_i} = \alpha 
\end{equation}
avec \( \alpha\in \eR\cup\{ +\infty \}\). Alors
\begin{enumerate}
\item si $\alpha \neq 0$ et $\alpha\neq \infty$, alors
  \begin{equation}
    \sum_i a_i \text{~converge} \ssi \sum_j b_j\text{~converge,}
  \end{equation}
\item si $\alpha = 0$ et $\sum_j b_j$ converge, alors $\sum_i a_i$
  converge (absolument),
\item si $\alpha = +\infty$ et $\sum_j b_j$ diverge, alors $\sum_i
  a_i$ diverge.
\end{enumerate}
\end{proposition}

\begin{proof}
\begin{enumerate}
    \item
        Le fait que la suite $a_n/b_n$ converge vers $\alpha$ signifie que tant sa limite supérieure que sa limite inférieure convergent vers $\alpha$. En particulier la suite $\frac{ a_n }{ b_n }$ est bornée vers le haut et vers le bas. À partir d'un certain rang $N$, il existe $M$ tel que 
        \begin{equation}
            \frac{ a_n }{ b_n }<M
        \end{equation}
        et il existe $m$ tel que
        \begin{equation}
            \frac{ a_n }{ b_n }>m.
        \end{equation}
        Nous avons donc $a_n<Mb_n$ et $a_n>mb_n$. La série de $(a_n)$ converge donc si et seulement si la série de $(b_n)$ converge.
    \item
        Si $\alpha=0$, cela signifie que pour tout $\epsilon$, il existe un rang tel que $\frac{ a_n }{ b_n }<\epsilon$, et donc tel que $a_n<\epsilon b_k$. La suite de $(a_i)$ converge donc dès que la suite de $(b_i)$ converge.
    \item
        Pour tout $M$, il existe un rang dans la suite à partir duquel on a $\frac{ a_i }{ b_i }>M$, et donc $a_k>Mb_k$. Si la série de $(b_k)$ diverge, la série de $(a_k)$ doit également diverger.
\end{enumerate}
\end{proof}


\begin{proposition}[Critère du quotient\cite{KeislerElemCalculus}]     \label{PropOXKUooQmAaJX}
    Soit $\sum_i a_i$ une série. Supposons l'existence de la limite (éventuellement infinie) suivante
    \begin{equation}
      \limite i \infty \abs{\frac{a_{i+1}}{a_i}} = L
    \end{equation}
    avec \( L\in \eR\cup\{ +\infty \}\).  Alors
    \begin{enumerate}
    \item si $L < 1$, la série converge absolument,
    \item si $L > 1$, la série diverge,
    \item si $L = 1$ le critère échoue : il existe des exemple de convergence et des exemples de divergence.
    \end{enumerate}
\end{proposition}

\begin{proof}
\begin{enumerate}
    \item
        Soit $b$ tel que $L<b<1$. À partir d'un certain rang $K$, on a $\left| \frac{ a_{i+1} }{ a_i } \right| <b$. En particulier,
        \begin{equation}
            | a_{K+1} |<b| a_K |,
        \end{equation}
        et pour $a_{K+2}$ nous avons
        \begin{equation}
            | a_{K+2} |<b| a_{K+1} |<b^2| a_K |.
        \end{equation}
        Au final,
        \begin{equation}
            | a_{K+n} |<b^n| a_K |.
        \end{equation}
        Étant donné que la série $\sum_{n\geq K}b^n$ converge (parce que $b<1$), la queue de suite $\sum_{i\geq K}a_i$ converge, et par conséquent la suite au complet converge.
    \item
        Si $L>1$, on a
        \begin{equation}
            | a_K |<| a_{K+1} |<| a_{K+2} |<\ldots
        \end{equation}
        Il est donc impossible que la suite $(a_i)$ converge vers zéro. La série ne peut donc pas converger.
    \item
        Par exemple la suite harmonique $a_n=\frac{1}{ n }$ vérifie $L=1$, mais la série ne converge pas. Par contre, la suite $a_n=\frac{ 1 }{ n^2 }$ vérifie aussi le critère avec $L=1$ tandis que la série $\sum_n\frac{1}{ n^2 }$ converge.
\end{enumerate}
\end{proof}


\begin{proposition}[Critère de la racine\cite{TrenchRealAnalisys}]
    Soit $\sum_i a_i$ une série, et considérons
    \begin{equation*}
      \limsup_{i \rightarrow \infty} \sqrt[i]{\abs{a_i}} = L 
    \end{equation*}
    avec \( L\in \eR\cup\{ +\infty \}\). Alors
    \begin{enumerate}
    \item si $L < 1$, la série converge absolument,
    \item si $L> 1$, la série diverge,
    \item si $L = 1$ le critère échoue.
    \end{enumerate}
\end{proposition}

\begin{proof}
    \begin{enumerate}
        \item
            Si $L<1$, il existe un $r\in \mathopen] 0 , 1 \mathclose[$ tel que $| a_n |^{1/n}<r$ pour les grands $n$. Dans ce cas, $| a_n |<r^{n}$, et la série converge absolument parce que la série $\sum_nr^n$ converge du fait que $r<1$.
        \item
            Si $L>1$, il existe un $r>1$ tel que $| a_n |^{1/n}>r>1$. Cela fait que $| a_n |$ prend des valeurs plus grandes que $n$ pour une infinité de termes. Le terme général $a_n$ ne peut donc pas être une suite convergente. Par conséquent la suite diverge au sens où elle ne converge pas.

    \end{enumerate}
\end{proof}

%---------------------------------------------------------------------------------------------------------------------------
\subsection{Critères de convergence simple}
%---------------------------------------------------------------------------------------------------------------------------

Les critères de comparaison, d'équivalence, du quotient et de la racine sont des critères de convergence absolue. Pour conclure à une convergence simple qui n'est pas une convergence absolue, le critère d'Abel sera notre outil principal.  

\subsubsection{Critère d'Abel}

\begin{proposition}[Critère d'Abel]
    Soit la série $\sum_i c_iz_i$ avec
    \begin{enumerate}
        \item $(c_i)$ est une suite réelle décroissante qui tend vers zéro,
        \item $(z_i)$ est une suite dans $\eC$ dont la suite des sommes partielles est bornée dans $\eC$, c'est à dire qu'il existe un $M>0$ tel que pour tout $n$,
        \begin{equation}
            \left| \sum_{i=1}^nz_i \right| \leq M.
        \end{equation}
        Alors la série $\sum_ic_iz_i$ est convergente.
    \end{enumerate}
\end{proposition}
Remarquons que ce critère ne donne pas de convergence absolue.

\begin{corollary}[Critère des séries alternées]\index{critère!série alternée}       \label{CoreMjIfw}
    Si \( (a_n)\) est une suite décroissante à limite nulle, alors la série
  \begin{equation}
    \sum_{n=0}^\infty {(-1)}^n a_n
  \end{equation}
  converge simplement.
\end{corollary}

%--------------------------------------------------------------------------------------------------------------------------- 
\subsection{Quelques séries usuelles}
%---------------------------------------------------------------------------------------------------------------------------
\label{SUBSECooDTYHooZjXXJf}

\begin{example}[Série harmonique]
    La \defe{série harmonique}{série!harmonique} est
    \begin{equation}
        \sum_{i=1}^\infty \frac1i
    \end{equation}
    et diverge (possède une limite $+\infty$).
\end{example}

\begin{example}[Série géométrique] \label{ExZMhWtJS}
    La \defe{série géométrique}{série!géométrique} de raison $q \in \eC$ est
    \begin{equation}    \label{EqZQTGooIWEFxL}
        \sum_{i=0}^\infty q^i.
    \end{equation}
    Étudions la somme partielle \( S_N=1+q+q^2+\cdots +q^{N}\). Nous avons évidemment $S_N-qS_N=1-q^{N+1}$ et donc
    \begin{equation}    \label{EqASYTiCK}
        S_N=\sum_{n=0}^Nq^n=\frac{ 1-q^{N+1} }{ 1-q }.
    \end{equation}
    La limite \( \lim_{N\to \infty} S_N\) existe si et seulement si \( | q |\leq 1\) et dans ce cas nous avons
    \begin{equation}    \label{EqRGkBhrX}
        \sum_{n=0}^{\infty}q^n=\frac{ 1 }{ 1-q }.
    \end{equation}
    La convergence est absolue.

    Si la somme commence en \( n=1\) au lieu de \( n=0\) alors
    \begin{equation}        \label{EqPZOWooMdSRvY}
        \sum_{n=1}^{\infty}q^n=\frac{1}{ 1-q }-1=\frac{ q }{ 1-q }.
    \end{equation}
\end{example}

Un cas particulier de la formule \eqref{EqASYTiCK} est le calcul de \( \sum_{j=1}^{N}q^{-j}\) bien utile lorsque l'on joue avec des nombres binaires (voir l'exemple \ref{EXEMooRHENooGwumoA}). Nous avons
\begin{equation}        \label{EQooFMBAooEJkHWT}
    \sum_{j=1}^Nq^{-j}=\sum_{j=0}^Nq^{-j}-1=\frac{ 1-q^{-N} }{ q-1 }.
\end{equation}

\begin{example}[Série de Riemann]       \label{EXooCTYNooCjYQvJ}
    Pour $\alpha \in \eR$, la \defe{série de Riemann}{série!Riemann}
    \begin{equation}        \label{EqSerRiem}
        \sum_{i=1}^\infty \frac1{i^\alpha}
    \end{equation}
    converge (absolument, puisque réelle et positive) si et seulement si $\alpha > 1$, et diverge sinon.
\end{example}

\begin{example}[Série exponentielle] \label{ExIJMHooOEUKfj}
    La série exponentielle est la série (pour \( t\in \eR\))
    \begin{equation}
        \exp(t)=\sum_{k=0}^{\infty}\frac{ t^k }{ k! }.
    \end{equation}
    Nous montrons qu'elle converge pour tout \( t\in \eR\). Si \( a_k=t^k/k!\) alors \( \frac{ a_{k+1} }{ a_k }=\frac{ t }{ k }\) dont la limite \( k\to \infty\) est zéro (quel que soit \( t\)). En vertu du critère du quotient \ref{PropOXKUooQmAaJX} la série exponentielle converge (absolument) pour tout \( t\in \eR\).

    Toutes les propriétés de la fonction de \( t\) ainsi définie sont dans le théorème \ref{ThoRWOZooYJOGgR}.
\end{example}
\index{exponentielle!convergence}

\begin{example}[Série arithméticogémétrique\cite{QXuqdoo}]
    La \defe{suite arithméticogémétrique}{suite!arithméticogéométrique} est une suite de la forme \( u_{n+1}=au_n+b\) avec \( a\) et \( b\) non nuls. Si elle possède une limite, cette dernière doit résoudre \( l=al+n\), et donc être égale à 
    \begin{equation}
        r=\frac{ b }{ 1-a }.
    \end{equation}
    Il n'est pas très compliqué de trouver le terme général de la suite en fonction de \( a\) et de \( b\). Il suffit de considérer la suite \( v_n=u_n-r\), et de remarquer que cette suite est géométrique :
    \begin{equation}
        v_{n+1}=av_n.
    \end{equation}
    Par conséquent \( v_n=a^nv_0\), ce qui donne pour la suite \( (u_n)\) la formule
    \begin{equation}
        u_n=a^n(u_0-r)+r.
    \end{equation}
\end{example}

%---------------------------------------------------------------------------------------------------------------------------
\subsection{Moyenne de Cesaro}
%---------------------------------------------------------------------------------------------------------------------------

Si \( (a_n)_{n\in \eN} \) est une suite dans \( \eR\) ou \( \eC\), alors sa \defe{moyenne de Cesaro}{moyenne!de Cesaro}\index{Cesaro!moyenn} est la limite (si elle existe) de la suite
\begin{equation}
    c_n=\frac{1}{ n }\sum_{k=1}^na_k.
\end{equation}
En un mot, c'est la limite des moyennes partielles.

\begin{lemma}       \label{LemyGjMqM}
    Si la suite \( (a_n)\) converge vers la limite \( \ell\) alors la suite admet une moyenne de Cesaro qui vaudra \( \ell\).
\end{lemma}

\begin{proof}
    Soit \( \epsilon>0\) et \( N\in \eN\) tel que \( | a_n-\ell |<\epsilon\) pour tout \( n>N\). En remarquant que
    \begin{equation}
        \frac{1}{ n }\sum_{k=1}^nk-\ell=\frac{1}{ n }\sum_{k=1}^n(a_k-\ell),
    \end{equation}
    nous avons
    \begin{subequations}
        \begin{align}
            | \frac{1}{ n }\sum_{k=1}^na_k-\ell |&\leq| \frac{1}{ n }\sum_{k=1}^N| a_k-\ell | |+\big| \frac{1}{ n }\sum_{k=N+1}^n\underbrace{| a_k-\ell |}_{\leq \epsilon} \big|\\
            &\leq \epsilon+\frac{ n-N-1 }{ n }\epsilon\\
            &\leq 2\epsilon.
        \end{align}
    \end{subequations}
    Dans ce calcul nous avons redéfinit \( N\) de telle sorte que le premier terme soit inférieur à \( \epsilon\).
\end{proof}

%--------------------------------------------------------------------------------------------------------------------------- 
\subsection{Écriture décimale d'un nombre}
%---------------------------------------------------------------------------------------------------------------------------

Soit \( b\geq 2\) un entier qui sera la base dans laquelle nous allons écrire les nombres. Nous considérons l'ensemble \( \eD_b\)\nomenclature[Y]{\( \eD_b\)}{l'ensemble de écritures décimales en base \( b\)} des suites dans \( \{ 0,1,\ldots, b-1 \}\) qui n'ont pas une queue de suite uniquement formée de \( b-1\). Autrement dit une suite \( (c_n)\) est dans \( \eD_b\) lorsque pour tout \( N\), il existe \( k>N\) tel que \( c_k\neq b-1\). Associé à cet ensemble nous considérons la fonction
\begin{equation}    \label{EqXXXooOTsCK}
    \begin{aligned}
        \varphi_b\colon \eD_b&\to \mathopen[ 0 , 1 [ \\
            c&\mapsto \sum_{n=1}^{\infty}\frac{ c_n }{ b^n }. 
    \end{aligned}
\end{equation}

\begin{lemma}
    La fonction \( \varphi_b\) est bien définie au sens où elle converge et prend ses valeurs dans \( \mathopen[ 0 , 1 [\).
\end{lemma}
    
\begin{proof}
    Tout se base sur la somme de la série géométrique \eqref{EqRGkBhrX} sous la forme
    \begin{equation}    \label{EqWZGooXJgwl}
        \sum_{k=0}^{\infty}\frac{1}{ b^k }=\frac{ b }{ b-1 }.
    \end{equation}
    La somme \eqref{EqXXXooOTsCK} est donc majorée par \( \sum_n\frac{ b-1 }{ b^n }\) qui converge.

    Pour prouver que l'image de \( \varphi_b\) est bien \( \mathopen[ 0 , 1 [\), nous savons qu'au moins un des \( c_n\) (en fait une infinité) est plus petit que \( b-1\), donc nous avons la majoration stricte\footnote{Notez que la somme \eqref{EqXXXooOTsCK} commence à un tandis que la série géométrique \eqref{EqWZGooXJgwl} commence à zéro.}
        \begin{equation}
            \varphi_b(c)<\sum_{n=1}^{\infty}\frac{ b-1 }{ b^n }=(b-1)\left( \sum_{n=1}^{\infty}\frac{1}{ b^n }-1 \right)=1
        \end{equation}
\end{proof}

Le fait d'introduire l'ensemble \( \eD\) au lieu de l'ensemble de toutes les suites est justifié par la proposition suivante. Elle explique pourquoi un nombre possède au maximum deux écritures décimales distinctes et que ces deux sont obligatoirement de la forme, par exemple en base \( 10\) :
\begin{equation}
    0.34599999999\ldots=0.34600000\ldots
\end{equation}
mais qu'un nombre commençant par \( 0.347\) ne peut pas être égal. C'est pour cela que dans la définition de \( \eD_b\) nous avons exclu les suites qui terminent par tout des \( b-1\).
\begin{proposition} \label{PropSAOoofRlQR}
    Soit la fonction
    \begin{equation}
        \begin{aligned}
            \varphi\colon \{ 0,\ldots, b-1 \}^{\eN}&\to \mathopen[ 0 , 1 [ \\
                x&\mapsto \sum_{n=1}^{\infty}\frac{ x_n }{ b^n }. 
        \end{aligned}
    \end{equation}
    Si \( \varphi(x)=\varphi(y)\) et si \( n_0\) est le plus petit entier tel que \( x_{n_0}\neq y_{n_0}\) alors soit
    \begin{equation}
        x_{n_0}-y_{n_0}=1
    \end{equation}
    et \( x_n=0\), \( y_n=b-1\) pour tout \( n>n_0\), soit le contraire : \( y_{n_0}-x_{n_0}=1\) avec \( y_n=0\) et \( x_n=b-1\) pour tout \( n>n_0\).
\end{proposition}

\begin{proof}
    Nous nous basons sur la formule (facilement dérivable depuis \eqref{EqWZGooXJgwl}) suivante :
    \begin{equation}
        \sum_{k=n_0+1}^{\infty}\frac{1}{ b^k }=\frac{1}{ b^{n_0+1} }\frac{ b }{ b-1 }.
    \end{equation}
    Nous avons 
    \begin{equation}
        0=\varphi(x)-\varphi(y)=\frac{ x_{n_0}-y_{n_0} }{ b^{n_0} }+\sum_{n=n_0+1}^{\infty}\frac{ x_n-y_n }{ b^n }\geq \frac{ x_{n_0}-y_{n_0} }{ b^{n_0} }-\sum_{n=n_0+1}^{\infty}\frac{ b-1 }{ b^n }=\frac{ x_{n_0}-y_{n_0}-1 }{ b^{n_0} }.
    \end{equation}
    Le dernier terme étant manifestement positif\footnote{C'est ici qu'intervient la subdivision entre le cas \( x_{n_0}-y_{n_0}=1\) ou le contraire. En effet si «ce dernier terme était manifestement \emph{négatif}», il aurait fallu majorer avec de \( 1-b\) au lieu de \( 1-b\).}, il est nul et nous avons \( x_{n_0}-y_{n_0}=1\).

    Nous avons donc maintenant
    \begin{equation}    \label{EqHWQoottPnb}
        0=\varphi(x)-\varphi(y)=\frac{1}{ b^{n_0} }+\sum_{n=n_0+1}^{\infty}\frac{ x_n-y_n }{ b^n }.
    \end{equation}
    Nous majorons la dernière somme de la façon suivante, en supposant que \( | x_n-y_n |\neq b-1\) pour un certain \( n>n_0\) :
    \begin{equation}
        \left| \sum_{n=n_0+1}^{\infty}\frac{ x_n-y_n }{ b^n } \right| \leq\sum_{n=n_0+1}^{\infty}\frac{ | x_n-y_n | }{ b^n }<\sum_{n=n_0+1}^{\infty}\frac{ b-1 }{ b^n }=\frac{1}{ b^{n_0} }.
    \end{equation}
    Étant donné cette inégalité stricte, l'équation \eqref{EqHWQoottPnb} ne peut pas être correcte (valoir zéro). Nous avons donc \( | x_n-b_n |=b-1\) pour tout \( n>n_0\). Donc pour chaque \( n>n_0\) nous avons soit \( x_n=0\) et \( y_n=b-1\), soit \( a_n=b-1\) et \( b_n=0\). Pour conclure il faut encore prouver que le choix doit être le même pour tout \( n\).

    Nous nous mettons dans le cas \( x_{n_0}-y_{n_0}=1\); dans ce cas nous avons bien l'égalité \eqref{EqHWQoottPnb} sans petites nuances de signes. Nous écrivons
    \begin{equation}
        \sum_{n=n_0+1}^{\infty}\frac{ x_n-y_n }{ b^n }=(b-1)\sum_{n=n_0+1}^{\infty}\frac{ (-1)^{s_n} }{ b^n }
    \end{equation}
    où \( s_n\) est pair ou impair suivant que \( x_n=0\), \( y_n=b-1\) ou le contraire. Si un des \( (-1)^{s_n}\) est pas \( -1\) alors nous avons l'inégalité stricte
    \begin{equation}
        (b-1)\sum_{n=n_0+1}^{\infty}\frac{ (-1)^{s_n} }{ b^n }>(b-1)\sum_{n=n_0+1}^{\infty}\frac{-1}{ b^n }=-\frac{1}{ b^{n_0} }.
    \end{equation}
    Dans ce cas il est impossible d'avoir \( \varphi(x)-\varphi(y)=0\). Nous en concluons que \( (-1)^{s_n}\) est toujours \( -1\), c'est à dire \( x_n-y_n=1-b\), ce qui laisse comme seule possibilité \( x_n=0\) et \( y_n=b-1\).
\end{proof}

\begin{theorem} \label{ThoRXBootpUpd}
    L'application \( \varphi_b\colon \eD_b\to \mathopen[ 0 , 1 [\) est bijective.
\end{theorem}

\begin{proof}
    En ce qui concerne l'injection, nous savons de la proposition \ref{PropSAOoofRlQR} que si \( \varphi_b(x)=\varphi_b(y)\) pour \( x,y\in\{ 0,\ldots, b-1 \}^{\eN}\), alors soit \( x\) soit \( y\) a une queue de suite composée uniquement de \( b-1\), ce qui est exclu dans \( \eD_b\). Nous en déduisons que \( \varphi_b\) est bien injective en prenant \( \eD_b\) comme ensemble départ.

    La partie lourde est la surjectivité. Nous prenons \( x\in \mathopen[ 0 , 1 [\) et nous allons construire par récurrence une suite \( a\in \eD_b\) telle que \( \varphi_b(a)=x\). Si il existe \( a_1\in\{ 0,\ldots, b-1 \}\) tel que \( x=a_1/b\) alors nous prenons la suite \( (a_1,0,\ldots, )\) et nous avons évidemment \( \varphi(a)=x\). Sinon il existe \( a_1\in\{ 0,\ldots, b-1 \}\) tel que
        \begin{equation}
            \frac{ a_1 }{ b }<x<\frac{ a_1+1 }{ b }
        \end{equation}
        parce que les autres possibilités pour \( x\) sont dans l'ensemble \( \mathopen[ 0 , 1 \mathclose[\setminus\{ \frac{ k }{ b } \}_{k=0,\ldots, b-1}\) que nous subdivisons en
        \begin{equation}
        \mathopen] 0 , \frac{1}{ b } \mathclose[\cup\mathopen] \frac{1}{ b } , \frac{ 2 }{ b } \mathclose[\cup\ldots\cup\mathopen] \frac{ b-1 }{ b } , 1 \mathclose[.
        \end{equation}
        Pour la récurrence nous supposons avoir trouvé \( a_1,\ldots, a_n\) tels que
        \begin{equation}
            \sum_{k=1}^n\frac{ a_k }{ b^k }< x<\sum_{k=1}^{n-1}\frac{ a_k }{ b^k }+\frac{ a_n+1 }{ b^n }.
        \end{equation}
    Encore une fois s'il existe \( a_{n+1}\in\{ 0,\ldots, b-1 \}\) tel que \( \sum_{k=1}^{n+1}\frac{ a_k }{ b^k }=x\) alors nous prenons ce \( a_{n+1}\) et nous complétons la suite avec des zéros pour avoir \( \varphi(a)=x\). Sinon 
%nous subdivisions l'intervalle \( \mathopen]  \frac{ a_n }{ b^n }, \frac{ a_n }{ b^n }+\frac{ a_n+1 }{ b^n } \mathclose[\) (auquel nous retranchons les \( b\) nombres déjà traités) en
 %       \begin{equation}
 %       \mathopen] \frac{ a_n }{ b^n } , \frac{ a_n }{ b^n }+\frac{1}{ b^{n+1} } \mathclose[ \cup \mathopen] \frac{ a_n }{ b^n }+\frac{1}{ b^{n+1} } , \frac{ a_n }{ b^n }+\frac{2}{ b^{n+1} } \mathclose[\cup\ldots\cup\mathopen] \frac{ a_n }{ b^n }+\frac{ b-1 }{ b^{n+1} } , \frac{ a_n }{ b^n }+\frac{ 1 }{ b^n } \mathclose[.
 %       \end{equation}
        , pour simplifier les notations nous notons \( x'=x-\sum_{k=1}^{n}\frac{ a_k }{ b^k }\) et nous avons
        \begin{equation}
            0<x'<\frac{ a_n+1 }{ b^n }.
        \end{equation}
        Le nombre \( x'\) est forcément dans un des intervalles
        \begin{equation}
                \mathopen] \frac{ s }{ b^{n+1} } , \frac{ s+1 }{ b^{n+1} } \mathclose[
        \end{equation}
        avec \( s\in\{ 0,\ldots, b-1 \}\). Nous prenons le \( s\) correspondant à \( x'\) comme \( a_{n+1}\). Dans ce cas nous avons
        \begin{equation}
            \sum_{k=1}^{n+1}\frac{ a_k }{ b^k }< x<\sum_{k=1}^{n+1}\frac{ a_k }{ b^k }+\frac{1}{ b^{n+1} }.
        \end{equation}
        Note : les deux inégalités sont strictes. La première parce que s'il y avait égalité, nous nous serions déjà arrêté en complétant avec des zéros. La seconde parce que 
        \begin{equation}
            \sum_{k=n+2}^{\infty}\frac{ a_k }{ b^k }\leq \sum_{k=n+2}^{\infty}\frac{ b-1 }{ b^k }=\frac{1}{ b^{n+1} }
        \end{equation}
        où l'égalité n'est possible que si \( a_k=b-1\) pour tout \( k\geq n+2\). Dans ce cas nous aurions eu
        \begin{equation}
            x=\sum_{k=1}^{n}\frac{ a_k }{ b^k }+\frac{ a_{n+1}+1 }{ b^{n+1} }
        \end{equation}
        et nous aurions choisit le nombre \( a_{n+1}\) autrement et complété la suite par des zéros à partir de là. Notons que cela prouve au passage que la suite que nous sommes en train de construire est bien dans \( \eD_b\) parce qu'elle ne contiendra pas de queue de suite composée de \( b-1\).

        Ceci termine la construction par récurrence de la suite \( a\in \eD_b\). Par construction nous avons pour tout \( N\geq 1\),
        \begin{equation}
            \sum_{k=1}^N\frac{ a_k }{ b^k }\leq x\leq \sum_{k=1}^N\frac{ a_k }{ b^k }+\frac{1}{ b^{N+1} }, 
        \end{equation}
        autrement dit : \( \varphi_b(a_1,\ldots, a_N)\in B(x,\frac{1}{ b^{N+1} })\). Nous avons donc bien convergence
        \begin{equation}
            \lim_{N\to \infty} \varphi_b(a_1,\ldots, a_N)=x
        \end{equation}
        et l'application \( \varphi_b\) est surjective.
\end{proof}

L'application \( \varphi_b^{-1}\colon \mathopen[ 0 , 1 [\to \eD_b\) est la \defe{décomposition décimale}{décimale!décomposition} en base \( b\) des nombres de \( \mathopen[ 0 , 1 [\).

Tout cela nous permet de montrer entre autres que \( \eR\) n'est pas dénombrable. Vu qu'il y a une bijection entre \( \mathopen[ 0 , 1 [\) et \( \eD_b\), il suffit de prouver que \( \eD_b\) est non dénombrable. De plus il suffit de démontrer que \( \eD_b\) est non dénombrable pour un entier \( b\geq 2\) donné.

\begin{proposition}  \label{PropNNHooYTVFw} 
    Il n'existe pas de surjection \( \eN\to \eD_b\). Autrement dit \( \eD_b\) est non dénombrable.
\end{proposition}

\begin{proof}
    Nous prenons \( b\neq 2\) pour des raisons qui seront claires plus tard. Soit \( f\colon \eN\to \eD_b\). Pour \( i\in \eN\) nous notons 
    \begin{equation}
        f(n)=(c_i^{(n)})_{i\geq 1},
    \end{equation}
    et nous définissons la suite
    \begin{equation}
        c_k=\begin{cases}
            0    &   \text{si } c_k^{(k)}\neq 0\\
            1    &    \text{si } c_k^{(k)}=0.
        \end{cases}
    \end{equation}
    Cela est une suite dans \( \eD_b\) parce que \( b\neq 2\) et que la suite ne contient que des \( 0\) et des \( 1\). Mais nous n'avons \( f(n)=c\) pour aucun \( n\in \eN\) parce que nous avons \( c_n\neq f(n)_n\).

    Si \( b=2\) alors nous savons que \( \eD_2\sim\mathopen[ 0 , 1 [\sim \eD_3\). Donc \( \eD_2\sim \eD_3\) et \( \eD_2\) ne peut pas plus être mis en bijection avec \( \eN\) que \( \eD_3\).
\end{proof}
\begin{remark}
    La preuve ne fonctionne pas en base \( b=2\) parce que rien n'empêche d'avoir une queue de \( 1\). Il y a alors toutefois moyen de se débrouiller en construisant la suite \( c\) de façon plus subtile. Si \( b=2\) et \( n\in \eN\) alors \( f(n)\) est une suite de \( 0\) et \( 1\) contenant une infinité de \( 0\) (parce qu'il n'y a pas de queue de suite ne contenant que des \( 1\)). Nous construisons alors \( c\) de la façon suivante : d'abord nous recopions \( f(0)\) jusqu'à son \emph{deuxième} zéro que nous changeons en \( 1\); nommons \( n_0\) le rang de ce deuxième zéro. Ensuite nous recopions les éléments de \( f(1) \) à partir du rang \( n_0+1\) jusqu'au second zéro que nous changeons en \( 1\), etc.

    Le fait de prendre le deuxième zéro nous garanti que la suite \( c\) n'aura pas de queue de suite ne contenant que des \( 1\).

    Notons que cette construction s'adapte à tout \( b\); il suffit de prendre le second terme qui n'est pas \( b-1\) et le remplacer par \( b-1\).
\end{remark}

\begin{corollary}
    L'ensemble \( \mathopen[ 0 , 1 [\) n'est pas dénombrable.
\end{corollary}

\begin{proof}
    L'ensemble \( \mathopen[ 0 , 1 [\) est en bijection avec \( \eD_b\) que nous venons de prouver n'être pas dénombrable.
\end{proof}

%+++++++++++++++++++++++++++++++++++++++++++++++++++++++++++++++++++++++++++++++++++++++++++++++++++++++++++++++++++++++++++
\section{Produit fini d'espaces vectoriels normés}
%+++++++++++++++++++++++++++++++++++++++++++++++++++++++++++++++++++++++++++++++++++++++++++++++++++++++++++++++++++++++++++
\label{sec_prod}

Dans cette sections nous parlons de produits finis d'espaces. Cela ne signifie pas que chacun des espaces soient séparément de dimension finie.

%---------------------------------------------------------------------------------------------------------------------------
\subsection{Norme}
%---------------------------------------------------------------------------------------------------------------------------

La définition de la norme sur un produit d'espaces vectoriels normés découle immédiatement de la définition de la distance \ref{DefZTHxrHA} :
\begin{definition}  \label{DefFAJgTCE}
    Soient $V$ et $W$ deux espaces vectoriels normés. On appelle \defe{espace produit}{produit!d'espaces vectoriels normés} de $V$ et $W$ le produit cartésien $V\times W$ 
    \begin{equation}
    V\times W=\{(v,w)\,|\, v\in V,\, w\in W\},
    \end{equation}
    muni de la norme $\|\cdot \|_{V\times W}$
    \begin{equation}	\label{EqNormeVxWmax}
        \|(v,w) \|_{V\times W}=\max\{\|v\|_{V},\|w\|_W\}.
    \end{equation}
\end{definition}
Il est presque immédiat de vérifier que le produit cartésien $V\times W$ est un espace vectoriel pour les opération de somme et multiplication par les scalaires définies composante par composante. C'est à dire,  si $(v_1,w_1)$, $(v_2,w_2)$ sont dans $V\times W$ et $a$, $b$ sont des scalaires, alors  
\begin{equation}
 a (v_1,w_1)+ b(v_2,w_2)=(av_1,aw_1)+ (bv_2,bw_2)=(av_1+bv_2,aw_1+bw_2).
\end{equation}

\begin{lemma}
	L'opération $\|\cdot \|_{V\times W}\colon V\times W\to \eR$ est une norme.
\end{lemma}

\begin{proof}
	On doit vérifier les trois conditions de la définition \ref{DefNorme}.
	\begin{itemize}
		\item Soit $(v,w)$ dans $V\times W$ tel que $\|(v,w)\|_{V\times W}=\max\{\|v\|_{V},\|w\|_W\}=0$. Alors $\|v\|_V=0$ et $\|w\|_W=0$, donc $v=0_V$ et $w=0_W$. Cela implique $(v,w)=(0_v,0_w)=0_{V\times W}$. 
		\item Pour tout $a$ dans $\eR$ et $(v,w)$ dans $V\times W$,  la norme $\|a (v,w)\|_{V\times W}$ est donnée par  $\max\{\|av\|_{V},\|aw\|_W\}$. On peut factoriser $\|av\|_{V}=|a|\|v\|_{V}$ et $\|aw\|_W=|a|\|w\|_W$ et donc $\|a (v,w)\|_{V\times W}=|a|\max\{\|v\|_{V},\|w\|_W\}=|a|\|(v,w)\|_{V\times W}$.
		\item Soient $(v_1,w_1)$ et $(v_2,w_2)$ dans $V\times W$. 
		\begin{equation}
			\begin{aligned}
				\|(v_1,w_1)+(v_2,w_2)\|_{V\times W}&=\max\{\|v_1+v_2\|_{V},\|w_1+w_2\|_W\}\\
				&\leq \max\{\|v_1\|_V+\|v_2\|_{V},\|w_1\|_W+\|w_2\|_W\}\\
				&\leq\max\{\|v_1\|_V,\|w_1\|_W\}+ \max\{\|v_2\|_{V},\|w_2\|_W\}\\
				&=\|(v_1,w_1)\|_{V\times W}+\|(v_2,w_2)\|_{V\times W}.
			\end{aligned}
		\end{equation}
	\end{itemize} 
\end{proof}

Toutes ces définitions se généralisent à un produit fini d'espaces vectoriels normés. Si les espaces \( V_i\) sont des espaces vectoriels normés, nous pouvons mettre sur le produit une topologie et une norme :
\begin{itemize}
    \item La topologie produit donnée en \ref{DefIINHooAAjTdY}
    \item La norme maximum \( \| v_1,\ldots, v_n \|_{max}=\max\{ \| v_1 \|,\ldots, \| v_n \| \}\). Dans le membre de droites, toutes les normes sont différentes.
\end{itemize}
Une question qui vient est la compatibilité entre ces deux constructions. Est-ce que la topologie associée à la norme maximum est le topologie produit ? Oui.

\begin{lemma}       \label{LEMooWVVCooIGgAdJ}
    La topologie de la norme maximum est la topologie produit\footnote{Définition \ref{DefIINHooAAjTdY}.}.
\end{lemma}

En particulier, pour la topologie de la norme maximum, la convergence d'une suite implique la convergence «composante par composante» par la proposition \ref{PROPooNRRIooCPesgO}.

\begin{proposition}[\cite{ooCUHNooNYIeGt}]      \label{PROPooQFTSooPFfbCc}
    Soient des espaces vectoriels normés \( V\) et \( W\) ainsi qu'une forme sesquilinéaire \( \phi\colon V\times W\to \eC\). Il y a équivalence des faits suivants.
    \begin{enumerate}
        \item
            \( \phi\) est continue.
        \item
            \( \phi\) est continue en \( (0,0)\)
        \item
            \( \phi\) est bornée
        \item
            Il existe \( C\geq 0\) telle que \( | \phi(x,y) |\leq C\| x \|\| y \|  \) pour tout \( (x,y)\in V\times W\).
    \end{enumerate}
    De plus la norme de \( \phi\) est alors donnée par
    \begin{equation}
        \| \phi \|=\min\{  C\geq 0\tq | \phi(x,y) |\leq C\| x \|\| y \|\forall (x,y)\in V\times W  \}.
    \end{equation}
\end{proposition}

On remarque tout de suite que la norme $\|.\|_\infty$ sur $\eR^2$ est la norme de l'espace produit $\eR\times\eR$. En outre cette définition nous permet de trouver plusieurs nouvelles normes dans les espaces $\eR^p$. Par exemple, si nous écrivons $\eR^4$ comme $\eR^2\times \eR^2$ on peut munir $\eR^4$ de la norme produit
\[
\|(x_1,x_2,x_3,x_4)\|_{\infty, 2}=\max\{\|(x_1,x_2)\|_\infty, \|(x_3,x_4)\|_2\}. 
\]    
Les applications de projection de l'espace produit $V\times W$ vers les espaces <<facteurs>>, $V$ $W$ sont notées $\pr_V$ et $\pr_W$ et sont définies par
\begin{equation}
	\begin{aligned}
		\pr_V\colon V\times W&\to V \\
		(v,w)&\mapsto v 
	\end{aligned}
\end{equation}
et
\begin{equation}
	\begin{aligned}
		\pr_W\colon V\times W &\to W \\
		(v,w)&\mapsto w. 
	\end{aligned}
\end{equation}
Les inégalités suivantes sont évidentes
\begin{equation}
	\begin{aligned}[]
		\|\pr_V(v,w)\|_V&\leq \|(v,w)\|_{V\times W} \\
		\|\pr_W(v,w)\|_W&\leq \|(v,w)\|_{V\times W}.
	\end{aligned}
\end{equation}
La topologie de l'espace produit est induite par les topologies des espaces <<facteurs>>. La construction est faite en deux passages : d'abord nous disons que une partie $A\times B$ de $V\times W$ est ouverte si $A$ et $B$ sont des parties ouvertes de $V$ et de $W$ respectivement.  Ensuite nous définissons que une partie quelconque de $V\times W$ est ouverte si elle est une intersection finie ou une réunion de parties ouvertes de $V\times W$ de la forme $A\times B$. 

Ce choix de topologie donne deux propriétés utiles de l'espace produit 
\begin{enumerate}
	\item
		Les projections sont des \defe{applications ouvertes}{application!ouverte}. Cela veut dire que l'image par $\pr_V$ (respectivement $\pr_W$) de toute partie ouverte de $V\times W$ est une partie ouverte de $V$ (respectivement $W$). 
	\item 
		Pour toute partir $A$ de $V$ et $B$ de $W$, nous avons $\Int (A\times B)=\Int A\times \Int B$.\label{PgovlABeqbAbB}
\end{enumerate}
Une propriété moins facile a prouver est que pour toute partie $A$ de $V$ et $B$ de $W$ nous avons  $\overline{A\times B}=\bar{A}\times \bar{B}$. Voir le lemme \ref{LemCvVxWcvVW}.
% position 26329
%et l'exercice \ref{exoGeomAnal-0009}.
  
Ce que nous avons dit jusqu'ici est valable pour tout produit d'un nombre fini d'espaces vectoriels normés. En particulier, pour tout $m>0$  l'espace  $\eR^m$ peut être considéré comme le produit de $m$ copies de $\eR$. 

\begin{example}
	Si $V$ et $W$ sont deux espaces vectoriels, nous pouvons considérer le produit $E=V\times W$. Les projections $\pr_V$ et $\pr_W$\nomenclature{$\pr_V$}{projection de $V\times W$ sur $V$}, définies dans la section \ref{sec_prod}, sont des applications linéaires. 

	En effet, la projection $\pr_V\colon V\times W\to V$ est donnée par $\pr_V(v,w)=v$. Alors,
	\begin{equation}
		\begin{aligned}[]
			\pr_V\big( (v,w)+(v',w') \big)&=\pr_V\big( (v+v'),(w+w') \big)\\
			&=v+v'\\
			&=\pr_V(v,w)+\pr_V(v',w'),
		\end{aligned}
	\end{equation}
	et
	\begin{equation}
		\pr_V\big( \lambda(v,w) \big)=\pr_V\big( (\lambda v,\lambda w) \big)=\lambda v=\lambda\pr_V(v,w).
	\end{equation}
	Nous laissons en exercice le soin d'adapter ces calculs pour montrer que $\pr_W$ est également une projection.
\end{example}

\begin{proposition} \label{PropDXR_KbaLC}
    Si \( \mO\) est un voisinage de \( (a,b)\) dans \( V\times W\) alors \( \mO\) contient un ouvert de la forme \( B(a,r)\times B(b,r)\).
\end{proposition}

\begin{proof}
    Vu que \( \mO\) est un voisinage, il contient un ouvert et donc une boule
    \begin{equation}
        B\big( (a,b),r \big)=\{ (v,w)\in V\times W\tq \max\{ \| v-a \|,\| w-b \| \}< r \}.
    \end{equation}
    Évidemment l'ensemble \( B(a,r)\times B(b,r)\) est dedans.
\end{proof}

%---------------------------------------------------------------------------------------------------------------------------
\subsection{Suites}
%---------------------------------------------------------------------------------------------------------------------------

Nous allons maintenant parler de suites dans $V\times W$. Nous noterons $(v_n,w_n)$ la suite dans $V\times W$ dont l'élément numéro $n$ est le couple $(v_n,w_n)$ avec $v_n\in V$ et $w_n\in W$. La notions de convergence de suite découle de la définition de la norme via la définition usuelle \ref{DefCvSuiteEGVN}. Il se fait que dans le cas des produits d'espaces, la convergence d'une suite est équivalente à la convergence des composantes. Plus précisément, nous avons le lemme suivant.
\begin{lemma}		\label{LemCvVxWcvVW}
	La suite $(v_n,w_n)$ converge vers $(v,w)$ dans $V\times W$ si et seulement les suites $(v_n)$ et $(w_n)$ convergent séparément vers $v$ et $w$ respectivement dans $V$ et $W$. 
\end{lemma}

\begin{proof}
	Pour le sens direct, nous devons étudier le comportement de la norme de $(v_n,w_n)-(v,w)$ lorsque $n$ devient grand. En vertu de la définition de la norme dans $V\times W$ nous avons
	\begin{equation}
		\Big\| (v_n,w_n)-(v,w) \Big\|_{V\times W}=\max\big\{ \| v_n-v \|_V,\| w_n-w \|_W \big\}.
	\end{equation}
	Soit $\varepsilon>0$. Par définition de la convergence de la suite $(v_n,w_n)$, il existe un $N\in\eN$ tel que $n>N$ implique
	\begin{equation}
		\max\big\{ \| v_n-v \|_V,\| w_n-w \|_W \big\}<\varepsilon,
	\end{equation}
	et donc en particulier les deux inéquations
	\begin{subequations}
		\begin{align}
			\| v_n-v \|&<\varepsilon\\
			\| w_n-w \|&<\varepsilon.
		\end{align}
	\end{subequations}
	De la première, il ressort que $(v_n)\to v$, et de la seconde que $(w_n)\to w$.

	Pour le sens inverse, nous avons pour tout $\varepsilon$ un $N_1$ tel que $\| v_n-v \|_V\leq\varepsilon$ pour tout $n>N_1$ et un $N_2$ tel que $\| w_n-w \|_W\leq\varepsilon$ pour tout $n>N_2$. Si nous posons $N=\max\{ N_1,N_2 \}$ nous avons les deux inégalités simultanément, et donc
	\begin{equation}
		\max\big\{ \| v_n-v \|_V,\| w_n-w \|_W \big\}<\varepsilon,
	\end{equation}
	ce qui signifie que la suite $(v_n,w_n)$ converge vers $(v,w)$ dans $V\times W$.
\end{proof}

\begin{proposition}[\cite{MonCerveau}]          \label{PROPooKDGOooDjWQct}
    Soit un espace \( E\) muni d'un produit scalaire à valeurs dans \( \eK\) (si \( \eK=\eC\) nous supposons le produit hermitien, mais ce n'est pas très important ici). Alors l'application
    \begin{equation}
        \begin{aligned}
            a\colon E\times E&\to \eK \\
            (x,y)&\mapsto \langle x, y\rangle  
        \end{aligned}
    \end{equation}
    est continue.
\end{proposition}

\begin{proof}
    Nous ne disons pas que l'espace \( V\times V\) est muni d'un produit scalaire. Mais en tout cas c'est un espace métrique, et \( \eK\) l'est aussi. Donc \( a\) est une application entre deux espaces métriques et elle sera continue si et seulement si elle est séquentiellement continue (proposition \ref{PropFnContParSuite}\ref{ItemWJHIooMdugfu}).

    Soit donc une suite convergente dans \( E\times E\), c'est à dire \( (x_k,y_k)\stackrel{E\times E}{\longrightarrow}(x,y)\). Nous devons démontrer que \( \langle x_k, y_k\rangle \stackrel{\eR}{\longrightarrow}\langle x, y\rangle \). Les majorations usuelles donnent
    \begin{subequations}
        \begin{align}
            \big| \langle x_k, y_k\rangle -\langle x, y\rangle  \big|&\leq \big| \langle x_k, y_k\rangle -\langle x, y_k\rangle  \big|+\big| \langle x, y_k\rangle -\langle x, y\rangle  \big|\\
            &=\big| \langle x_k-x, y_k\rangle  \big|+\big| \langle x, y_k-y\rangle  \big|.
        \end{align}
    \end{subequations}
    Nous savons du lemme \ref{LemCvVxWcvVW} que les suites \( (x_k)\) et \( (y_k)\) sont séparément convergentes : \( x_k\stackrel{E}{\longrightarrow}x\) et \( y_k\stackrel{E}{\longrightarrow}y\). En utilisant l'inégalité de Cauchy-Schwarz \ref{EQooZDSHooWPcryG} nous trouvons
    \begin{equation}
        \big| \langle x_k-x, y_k\rangle  \big|\leq \| x_k-x \|\| y_k \|.
    \end{equation}
    Nous avons \( \| x_k-x \|\to 0\) et \( \| y_k \|\to \| y \|\), et par la règle du produit de limites dans \( \eR\) nous avons que \( \big| \langle x_k-x, y_k\rangle  \big|\to 0\).
\end{proof}

\begin{remark}		\label{RemTopoProdPasRm}
	Il faut remarquer que la norme \eqref{EqNormeVxWmax} est une norme \emph{par défaut}. C'est la norme qu'on met quand on ne sait pas quoi mettre. Or il y a au moins un cas d'espace produit dans lequel on sait très bien quelle norme prendre : les espaces $\eR^m$. La norme qu'on met sur $\eR^2$ est
	\begin{equation}
		\| (x,y) \|=\sqrt{x^2+y^2},
	\end{equation}
	et non la norme «par défaut» de $\eR^2=\eR\times\eR$ qui serait
	\begin{equation}
		\| (x,y) \|=\max\{ | x |,| y | \}.
	\end{equation}
	Les théorèmes que nous avons donc démontré à propos de $V\times W$ ne sont donc pas immédiatement applicables au cas de $\eR^2$.

	Cette remarque est valables pour tous les espaces $\eR^m$. À moins de mention contraire explicite, nous ne considérons jamais la norme par défaut \eqref{EqNormeVxWmax} sur un espace $\eR^m$.
\end{remark}

Étant donné la remarque \ref{RemTopoProdPasRm}, nous ne savons pas comment calculer par exemple la fermeture du produit d'intervalle $\mathopen] 0,1 ,  \mathclose[\times\mathopen[ 4 , 5 [$. Il se fait que, dans $\eR^m$, les fermetures de produits sont quand même les produits de fermetures.

\begin{proposition}		\label{PropovlAxBbarAbraB}
	Soit $A\subset\eR^m$ et $B\subset\eR^m$. Alors dans $\eR^{m+n}$ nous avons $\overline{ A\times B }=\bar A\times \bar B$.
\end{proposition}

La démonstration risque d'être longue; nous ne la faisons pas ici.

%--------------------------------------------------------------------------------------------------------------------------- 
\subsection{Continuité du produit de matrices}
%---------------------------------------------------------------------------------------------------------------------------
\label{SUBSECooOAWAooFcyUfI}

Nous avons introduit des normes sur \( \eM(n,\eK)\), entre autres la norme opérateur de la définition \ref{DefNFYUooBZCPTr}. Qui dit norme dit topologie. Il advient alors la question évidente : est-ce que des opérations aussi élémentaires que le produit de matrices sont continues pour ces topologies ?
 
Une façon simple de répondre à cela est d'introduire sur \( \eM(n,\eK)\) une nouvelle norme très simple : celle de \( \eK^n\). C'est la topologie par composante. Pour cette topologie, il est simple de voir que le produit matriciel est continu parce que les éléments de \( AB\) sont des polynômes en les éléments de \( A\) \( B\). Ensuite il suffit d'invoquer l'équivalence de toutes les normes (théorème \ref{ThoNormesEquiv}).

Voyons comment montrer cela de façon plus directe (bien que le raisonnement précédent soit une démonstration qui devrait déjà avoir convaincu les plus sceptiques). La preuve suivante va donc s'amuser à bien préciser les topologies et caractérisations utilisées.

\begin{lemma}
    Si \( \| . \|\) est une norme algébrique sur \( \eM(n,\eK)\) (\( \eK\) est \( \eR\) ou \( \eC\)) alors l'application
    \begin{equation}
        \begin{aligned}
            p\colon \eM(n,\eK)\times \eM(n,\eK)&\to \eM(n,\eK) \\
            (A,B)&\mapsto AB 
        \end{aligned}
    \end{equation}
    est continue.
\end{lemma}

\begin{proof}
    L'espace \( \eM(n,\eK)\times \eM(n,\eK)\) est métrique (définition \ref{DefFAJgTCE}), donc la caractérisation séquentielle de la continuité (proposition \ref{PropXIAQSXr}) s'applique. Nous considérons donc une suite \( (A_k,B_k)\) dans \( \eM(n,\eK)\times \eM(n,\eK)\) convergente vers \( AB\).

    Nous savons que la topologie sur \( \eM(n,\eK)\times \eM(n,\eK)\) est la topologie produit (lemme \ref{LEMooWVVCooIGgAdJ}) et que celle-ci donne la convergence composante par composante dès que nous avons convergence d'une suite; c'est la proposition \ref{PROPooNRRIooCPesgO}. Nous avons donc \( A_k\stackrel{\eM(n,\eK)}{\longrightarrow}A\) et \( B_k\stackrel{\eM(n,\eK)}{\longrightarrow}B\).

    Voila pour le contexte. Maintenant, la preuve de la continuité. Nous effectuons les majorations suivantes :
    \begin{subequations}
        \begin{align}
            \| p(A_k,B_K)-AB \|&\leq \| p(A_k,B_k)-p(A_k,B) \|+\| p(A_k,B)-AB \|\\
            &=\| A_Kb_k-A_kB \|+\| A_kB-AB \|\\
            &=\| A_k(B_k-B) \|+\| (A_k-A)B \|\\
            &\leq \underbrace{\| A_k \|}_{\to \| A \|}\underbrace{\| B_k-B \|}_{\to 0}+\underbrace{\| A_k-A \|}_{\to 0}\| B \|.
        \end{align}
    \end{subequations}
\end{proof}

%+++++++++++++++++++++++++++++++++++++++++++++++++++++++++++++++++++++++++++++++++++++++++++++++++++++++++++++++++++++++++++ 
\section{Applications multilinéaires}
%+++++++++++++++++++++++++++++++++++++++++++++++++++++++++++++++++++++++++++++++++++++++++++++++++++++++++++++++++++++++++++

\begin{definition}[Application multilinéaire]       \label{DefFRHooKnPCT}
    Une application $T: \eR^{m_1}\times \ldots \times\eR^{m_k}\to\eR^p $ est dite \defe{\( k\)-linéaire}{application!multilinéaire} si pour tout $X=(x_1, \ldots,x_k)$ dans $ \eR^{m_1}\times \ldots \times\eR^{m_k}$ les applications $x_i\mapsto T(x_1, \ldots, x_i,\ldots,x_k)$ sont linéaires pour tout $i$ dans $\{1,\ldots,k\}$, c'est à dire
	\begin{equation}
		\begin{aligned}[]
			T(\cdot,x_2, \ldots, x_i,\ldots,x_k)&\in \mathcal{L}(\eR^{m_1}, \eR^p),\\
			T(x_1,\cdot, \ldots, x_i,\ldots,x_k)&\in \mathcal{L}(\eR^{m_2}, \eR^p),\\
						& \vdots\\
			T(x_1, \ldots, x_i,\ldots,x_{k-1},\cdot)&\in \mathcal{L}(\eR^{m_k}, \eR^p).\\
		\end{aligned}
	\end{equation}
	En particulier lorsque $k=2$, nous parlons d'applications \defe{bilinéaires}{bilinéaire}. Vous pouvez deviner ce que sont les applications \emph{tri}linéaire ou \emph{quadri}linéaire.
\end{definition}

L'ensemble des applications $k$-linéaires de $ \eR^{m_1}\times \ldots \times\eR^{m_k}$ dans $\eR^p$ est noté $\mathcal{L}(\eR^{m_1}\times \ldots \times\eR^{m_k}, \eR^p)$ ou $\mathcal{L}(\eR^{m_1}, \ldots,\eR^{m_k}; \eR^p)$.

\begin{example}
  Soit $A$ une matrice avec $m$ lignes et $n$ colonnes. L'application bilinéaire de $\eR^m\times \eR^n$ dans $\eR$ associée à $A$ est définie par
\[
T_A(x,y)= x^TAy=\sum_{i,j}a_{i,j}x_i y_j, \qquad \forall x\in \eR^m, \, y \in \eR^n.
\]
\end{example}

Nous énonçons la proposition suivante dans le cas d'espaces vectoriels normés\footnote{Sans hypothèses sur la dimension.} parce que nous allons l'utiliser dans ce cas, mais le cas particulier \( E_i=\eR^{m_i}\) et \( F=\eR^p\) est important.
\begin{proposition} \label{PropUADlSMg}
    Soient des espaces vectoriels normés \( E_i\) et \( F\). Une application \( n\)-linéaire
    \begin{equation}
        T\colon E_1\times\ldots\times E_n\to F
    \end{equation}
    est est continue si et seulement s'il existe un réel $L\geq 0$ tel que
  \begin{equation}\label{limitatezza}
     \|T(x_1, \ldots,x_n)\|_F\leq L \|x_1\|_{F_1}\cdots\|x_n\|_{F_n}, \qquad \forall x_i\in E_i.
  \end{equation}
\end{proposition}

\begin{proof}
    Pour simplifier l'exposition nous nous limitons au cas $n=2$ et nous notons $T(x,y)=x*y$

    Supposons que l'inégalité \eqref{limitatezza} soit satisfaite. 
    \begin{equation}\label{LimImplCont}
      \begin{aligned}
        \|x*y-x_0*y_0\|&=\|(x-x_0)*y-x_0*(y-y_0)\|\\
    &\leq \|(x-x_0)*y\|+\|x_0*(y-y_0)\|\\
    &\leq L\|x-x_0\|\|y\| + L\|x_0\|\|y-y_0\|.
      \end{aligned}
    \end{equation}
    Si $x\to x_0$ et $y\to y_0$  on voit que $T$ est continue en passant à la limite aux deux côtes de l'inégalité \eqref{LimImplCont}.

    Soit $T$ continue en $(0,0)$. Évidemment\footnote{Dans la formule suivante, les trois zéros sont les zéros de trois espaces différents.} $0*0=0$, donc il existe $\delta>0$ tel que si $x\in B_{E_1}(0,\delta)$ et $y\in B_{E_2}(0,\delta)$ alors $\|x*y\|\leq 1$. En particulier si \( (x,y)\in B_{E_1\times E_2}(0,\delta)\) nous sommes dans ce cas. Soient maintenant  $x\in E_1\setminus\{ 0 \}$  et $y\in E_2\setminus\{ 0\}$
    \begin{equation}
        x*y=\left(\frac{\|x\|}{\delta}\frac{\delta x}{\|x\|}\right)*\left(\frac{\|y\|}{\delta}\frac{\delta y}{\|y\|}\right)
    =\frac{\|x\|\|y\|}{\delta^2} \left(\frac{\delta x}{\|x\|}\right)*\left(\frac{\delta y}{\|y\|}\right).
     \end{equation}
    On remarque que $\delta x/\|x\|_m$ est dans la boule de rayon $\delta$ centrée en $0_m$ et que $\delta y/\|y\|_n$ est dans la boule de rayon $\delta$ centrée en $0_n$. On conclut 
    \[
     x*y\leq \frac{\|x\|_m\|y\|_n}{\delta^2}.
    \]
    Il faut prendre $L=1/\delta^2$.
\end{proof}

La norme de \( T\) est alors définie comme la plus petite constante \( L\) qui fait fonctionner la proposition \ref{PropUADlSMg}.
\begin{definition}  \label{DefKPBYeyG}
	La norme sur l'espace $\aL(E_1\times \cdots\times E_n, F)$ des applications $k$-linéaires et continues est 
	\begin{equation}
        \|T\|_{E_1\times \ldots\times E_n}=\sup\{ \|T(u_1, \ldots,u_k)\|_{F}\,\vert\,\|u_i\|_{E_i}\leq 1, i=1,\ldots, k \}.
	\end{equation}
\end{definition}
Nous avons donc automatiquement
\begin{equation}    \label{EqYLnbRbC}
    \| T(u,v) \|\leq \| T \|\| u \|\| v \|.
\end{equation}
Et nous notons que cette norme est uniquement définie pour les applications linéaires continues. Ce n'est pas très grave parce qu'alors nous définissons \( \| T \|=\infty\) si \( T\) n'est pas continue. Cela pour retrouver le principe selon lequel on est continue si et seulement si on est borné.

\begin{proposition}\label{isom_isom}
  On définit les fonctions
  \begin{equation}
    \begin{array}{rccc}
      \omega_g: & \mathcal{L}(\eR^{m}\times\eR^{n}, \eR^p)&\to &\mathcal{L}(\eR^{m}, \mathcal{L}(\eR^{n}, \eR^p)),\\
      \omega_d: & \mathcal{L}(\eR^{m}\times\eR^{n}, \eR^p)&\to &\mathcal{L}(\eR^{n}, \mathcal{L}(\eR^{m}, \eR^p)),
    \end{array}
  \end{equation}
par 
\[
\omega_g(T)(x)=T(x,\cdot), \qquad \forall x\in\eR^m,
\]
et
\[
\omega_d(T)(y)=T(\cdot, y), \qquad \forall y\in\eR^n.
\]
Les fonctions $\omega_g$ et $\omega_d$ sont des isomorphismes qui préservent les normes.    
\end{proposition}


%+++++++++++++++++++++++++++++++++++++++++++++++++++++++++++++++++++++++++++++++++++++++++++++++++++++++++++++++++++++++++++ 
\section{Exponentielle de matrice}
%+++++++++++++++++++++++++++++++++++++++++++++++++++++++++++++++++++++++++++++++++++++++++++++++++++++++++++++++++++++++++++
%\label{subsecAOnIwQM}
\label{secAOnIwQM}

\begin{enumerate}
    \item
        En ce qui concerne la continuité, nous aurons évidemment besoin de théorie à propos de l'inversion de limites et de sommes. Nous en parlerons donc en \ref{subsecXNcaQfZ}.
    \item 
        Les séries entières de matrices seront traitées plus en détail autour de la proposition \ref{PropFIPooSSmJDQ}.
\end{enumerate}

\begin{proposition}     \label{PropPEDSooAvSXmY}
    Soit \( V\) un espace vectoriel de dimension finie et \( A\in\End(V)\). La série
    \begin{equation}
        \exp(A)=\mtu+A+\frac{ A^2 }{ 2 }+\frac{ A^3 }{ 3 }+\ldots =\sum_{k=1}^{\infty}\frac{ A^k }{ k! }.
    \end{equation}
    converge normalement dans \( \big( \End(V),\| . \|_{op} \big)\).  L'\defe{exponentielle}{exponentielle!de matrice} de la matrice \( A\) est cette matrice.
\end{proposition}

\begin{proof}
    Vu que la norme opérateur est une norme d'algèbre par le lemme \ref{LEMooFITMooBBBWGI}, nous avons pour tout \( k\) la majoration \( \| A^k \|\leq \| A \|^k\). Nous avons donc
    \begin{equation}
        \sum_{k=0}^{\infty}\frac{ \| A^k \| }{ k! }\leq \sum_k\frac{ \| A \|^k }{ k! }.
    \end{equation}
    La dernière somme converge en vertu de la convergence de la série exponentielle donnée en exemple \ref{ExIJMHooOEUKfj}.
\end{proof}

Étant donné que c'est une limite, il y a une question de convergence et donc de topologie. C'est pour cela que nous ne pouvions pas introduire l'exponentielle de matrice avant d'avoir introduit la norme des matrices. La convergence de la série pour toute matrice sera prouvée au passage dans la proposition \ref{PropFMqsIE}.


La fonction exponentielle \(  x\mapsto e^{x}\) n'est pas un polynôme en \( x\), mais nous avons les résultat marrant suivant.
\begin{proposition} \label{PropFMqsIE}
    Si \( u\) est un endomorphisme, alors \( \exp(u)\) est un polynôme en \( u\)\footnote{Nan, mais j'te jure : \( \exp\) n'est pas un polynôme, mais $\exp(u)$ est un polynôme de \( u\).}.
\end{proposition}

\begin{proof}
    Nous considérons l'application
    \begin{equation}
        \begin{aligned}
            \varphi_u\colon \eK[X]&\to \End(E) \\
            P&\mapsto P(u)
        \end{aligned}
    \end{equation}
    Étant donné que l'image de \( \varphi_u\) est un fermé dans \( \End(E)\), il suffit de montrer que la série
    \begin{equation}
        \sum_{k=0}^{\infty}\frac{ \varphi_u(X)^k }{ k! }
    \end{equation}
    converge dans \( \End(E)\) pour qu'elle converge dans \( \Image(\varphi_u)\). Pour ce faire nous nous rappelons de la norme opérateur\footnote{Définition \ref{DefNFYUooBZCPTr}.} et de la propriété fondamentale \( \| A^k \|\leq \| A \|^k\). En notant \( A=\varphi_u(X)\),
    \begin{equation}
        \left\| \sum_{k=n}^m\frac{ A^k }{ k! } \right\|\leq \sum_{k=n}^m\frac{ \| A^k \| }{ k! }\leq \sum_{k=n}^m\frac{ \| A \|^k }{ k! },
    \end{equation}
    ce qui est une morceau du développement de \(  e^{\| A \|}\). La limite \( n\to\infty\) est donc zéro par la convergence de l'exponentielle réelle. La suite des sommes partielles de  $e^{A}$ est donc de Cauchy. La série converge donc parce que nous sommes dans un espace vectoriel réel de dimension finie (\( \End(E)\)).
\end{proof}
% TODO : et tant qu'on y est, justifier la convergence de la série de l'exponentielle réelle.

\begin{remark}
    Pourquoi \( \exp(u)\) est-il un polynôme d'endomorphisme alors que \( \exp\) n'est pas un polynôme ? Lorsque nous disons que la fonction \( x\mapsto \exp(x)\) n'est pas un polynôme, nous sommes en train de localiser la fonction \( \exp\) à l'intérieur de l'espace de toutes les fonctions \( \eR\to \eR\), c'est à dire à l'intérieur d'un espace de dimension infinie. Au contraire lorsqu'on parle de \( \exp(u)\) et qu'on le compare aux endomorphismes \( P(u)\), nous sommes en train de repérer \( \exp(u)\) à l'intérieur de l'espace des matrices qui est de dimension finie. Il n'est donc pas étonnant que l'on parvienne moins à faire la distinction.

    Si par contre nous considérons \( \exp\) en tant qu'application \( \exp\colon \End(E)\to \End(E)\), ce n'est pas un polynôme.

    Si \( u\) et \( v\) sont des endomorphismes, nous aurons des polynômes \( P\) et \( Q\) tels que \( e^u=P(u)\) et \( e^v=Q(v)\); mais nous n'aurons en général évidemment pas \( P=Q\). En cela, \( \exp\) n'est pas un polynôme.
\end{remark}

%+++++++++++++++++++++++++++++++++++++++++++++++++++++++++++++++++++++++++++++++++++++++++++++++++++++++++++++++++++++++++++ 
\section{Espace dual}
%+++++++++++++++++++++++++++++++++++++++++++++++++++++++++++++++++++++++++++++++++++++++++++++++++++++++++++++++++++++++++++
\label{SECooKOJNooQVawFY}

\begin{definition}
    Soit un espace vectoriel normé \( (V,\| . \|)\) sur le corps \( \eC\) ou \( \eR\) (que nous nommons \( \eK\)). Son \defe{dual topologique}{dual topologique}, noté \( V'\) est l'ensemble des applications linéaires continues \( V\to \eK\).
\end{definition}


Il est possible de mettre sur \( V'\) (au moins) deux topologies distinctes. La première est la topologie de la norme opérateur; rien de nouveau pour elle. La seconde est la topologie \( *\)-faible dont nous avons déjà un peu parlé dans la définition \ref{DefHUelCDD}.

\begin{propositionDef}
Soit un espace vectoriel normé $V$ sur $\eR$ ou $\eC$. Nous considrons l'application 
\end{propositionDef}<++>

En termes de notations, nous allons noter les semi-normes de la topologie faible par
\begin{equation}
    p_x(\varphi)=| \varphi(x) |
\end{equation}
pour \( x\in V\) et \( \varphi\in V'\). À droite, les barres dénotent soit la valeur absolue (si \( \eK=\eR\)), soit le module (si \( \eK=\eC\)).

\begin{lemma}
    Soit \( \varphi\in V'\) et \( x\in V\). Alors
    \begin{equation}
        p_x(\varphi)\leq\frac{ \| \varphi \| }{ \| x \| }.
    \end{equation}
\end{lemma}

\begin{proof}
    En posant \( x'=x/\| x \|\) nous avons
    \begin{equation}
        p_x(\varphi)=| \varphi(x) |=\frac{1}{ \| x \| }| \varphi(x') |\leq \frac{1}{ \| x \| }\| \varphi \|.
    \end{equation}
\end{proof}

\begin{proposition}
    En ce qui concerne la convergence d'une suite \( (\varphi_k)\) dans \( V'\), elle dépend des topologies, mais si elle vérifie
    \begin{equation}
        \varphi_k\stackrel{\| . \|}{\longrightarrow}\varphi
    \end{equation}
    alors
    \begin{equation}
        \varphi_k\stackrel{*}{\longrightarrow}\varphi.
    \end{equation}
\end{proposition}

\begin{proof}
    Soit une suite \( (\varphi_k)\) dans \( V'\), convergente vers \( \varphi\) pour la topologie de la norme.  Soit \( x\in V\), et \( x'=x/\| x \|\). Nous avons 
    \begin{equation}
        p_x(\varphi_k-\varphi)=\frac{1}{ \| x \| }| \varphi_k(x')-\varphi(x) |\leq\frac{1}{ \| x \| }\| \varphi_k-\varphi \|\to 0.
    \end{equation}
\end{proof}

%+++++++++++++++++++++++++++++++++++++++++++++++++++++++++++++++++++++++++++++++++++++++++++++++++++++++++++++++++++++++++++
\section{Mini introduction aux nombres \texorpdfstring{p}{$p$}-adiques}
%+++++++++++++++++++++++++++++++++++++++++++++++++++++++++++++++++++++++++++++++++++++++++++++++++++++++++++++++++++++++++++

\subsection{La flèche d'Achille}\label{s:un}

C'est un grand classique que je donne ici juste comme introduction pour montrer que des série infinies peuvent donner des nombres finis de manière tout à fait intuitive.

Achille tire une flèche vers un arbre situé à $\unit{10}{\meter}$ de lui. Disons que la flèche avance à une vitesse constante de $\unit{1}{\meter\per\second}$. Il est clair que la flèche mettra $\unit{10}{\second}$ pour toucher l'arbre. En $\unit{5}{\second}$, elle aura parcouru la moitié de son chemin. On le note :
\[
\text{temps}=5s+\ldots
\]
Reste \( \unit{5}{\meter}\) à faire. En $\unit{2.5}{\second}$, elle aura fait la moitié de ce chemin chemin, soit $2.5m=\frac{10}{4}m$. On le note :
\[
\text{temps}=\frac{10}{2}s+\frac{10}{4}s+
\]
Reste $2.5m$ à faire. La moitié de ce trajet, soit $\frac{10}{8}m$, est parcouru en $\frac{10}{8}s$; on le note encore, mais c'est la dernière fois !

\[
\text{temps}=\frac{10}{2}s+\frac{10}{4}s+\frac{10}{8}s+
\]
En continuant ainsi à regarder la flèche qui parcours des demi-trajets puis des demi de demi-trajets et encore des demi de demi de demi-trajets, et en sachant que le temps total est $10s$, on trouve :
\[
10\left( \frac{1}{2}+\frac{1}{4}+\frac{1}{8}+\frac{1}{16}+\ldots  \right)=10.
\]
On doit donc croire que la somme jusqu'à l'infini des inverse des puissances de deux vaut $1$ :
\[
   \sum_{n=1}^{\infty}\frac{1}{2^n}=1.
\]
Cela peut être démontré à la loyale.

\subsection{La tortue et Achille}

Maintenant qu'on est convaincu que des sommes infinies peuvent représenter des nombres tout à fait normaux, passons à un truc plus marrant.

Achille, qui marche peinard à $\unit{10}{\meter\per\hour}$, part avec $1m$ d'avance sur une tortue qui avance à $\unit{1}{\meter\per\hour}$. Le temps que la tortue arrive au point de départ d'Achille, Achille aura parcouru $10m$, et le temps que la tortue mettra pour arriver à ce point, eh bien, Achille ne sera déjà plus là : il sera à $100m$. Si la tortue tient bon pendant un temps infini, et si l'on est confiant en le genre de raisonnements faits à la section \ref{s:un}, elle rattrapera Achille dans 
\[
1m+10m+100m+1000m+\ldots
\]
Autant dire que ça ne risque pas d'arriver. Et pourtant, mettons en équations : 
\begin{subequations}
    \begin{numcases}{}
        x_{\text{Achile}}(t)=1+10t\\
        x_{\text{tortue}}(t)=t.
    \end{numcases}
\end{subequations}
La tortue rejoints Achille au temps \( t\) tel que \( x_{\text{Achille}(t)}=x_{\text{tortue}}(t)\). Un mini calcul donne $t=-1/9$. Physiquement, c'est une situation logique. Peut-on en déduire une égalité mathématique du style de 
\[
1+10+100+1000+\ldots=-\frac{1}{9}\; ???
\]
Là où les choses deviennent jolies, c'est quand on cherche à voir ce que peut bien être la valeur d'un hypothétique $x=1+10+100+1000+\ldots$. En effet, logiquement on devrait avoir
\begin{equation*}
\begin{split}
\frac{x}{10}&=\frac{1}{10}+1+10+100+\ldots\\
            &=\frac{1}{10}+x.
\end{split}
\end{equation*}
Reste à résoudre l'équation du premier degré : $\frac{x}{10}=x+\frac{1}{10}$. Ai-je besoin de donner la solution ?

%---------------------------------------------------------------------------------------------------------------------------
\subsection{Dans les nombres \texorpdfstring{p}{$ p$}-adiques, c'est vrai}
%---------------------------------------------------------------------------------------------------------------------------

Nous nous proposons d'apprendre sur les nombres \( p\)-adiques juste ce qu'il faut pour montrer que l'égalité
\begin{equation}
    \sum_{k=0}^{\infty}10^k=-\frac{1}{ 9 }
\end{equation}
est vraie dans les nombres \( 5\)-adiques. Tout ce qu'il faut est sur \wikipedia{fr}{Nombre_p-adique}{wikipedia}.

Soit \( a\in \eN\) et \( p\), un nombre premier. La \defe{valuation}{valuation!$p$-adique} \( p\)-adique de \( a\) est l'exposant de \( p\) dans la décomposition de \( a\) en nombres premiers. On la note \( v_p(a)\). Pour un rationnel on définit
\begin{equation}
    v_p\left( \frac{ a }{ b } \right)=v_p(a)-v_p(b)
\end{equation}
La \defe{valeur absolue}{valeur absolue!$p$-adique} \( p\)-adique de \( r\in \eQ\) est 
\begin{equation}
    | r |_p=p^{-v_p(r)}.
\end{equation}
Nous posons \( | 0 |_p=0\). De là nous considérons la distance
\begin{equation}
    d_p(x,y)=| x-y |_p.
\end{equation}

\begin{lemma}
    L'espace \( (\eQ,d_p)\) est un espace métrique\footnote{Définition \ref{DefMVNVFsX}}.
\end{lemma}
\index{topologie!\( p\)-adique}

Nous considérons maintenant \( p=5\). Étant donné que \( a=5\cdot 2\) nous avons \( v_5(10)=1\) et
\begin{equation}
    v_5\left( \frac{1}{ 9 } \right)=v_5(1)-v_5(9)=0.
\end{equation}
Nous avons
\begin{equation}
    \sum_{k=0}^N10^k+\frac{1}{ 9 }=\frac{ 10^{N+1} }{ 9 }
\end{equation}
mais
\begin{equation}
    v_p\left( \frac{ 10^{N+1} }{ 9 } \right)=v_5(10^{N+1})-v_5(9)=N+1.
\end{equation}
Par conséquent
\begin{equation}
    d_5\big( \sum_{k=0}^N10^k,-\frac{1}{ 9 } \big)=| \frac{ 10^{N+1} }{ 9 } |_p=p^{-(N+1)}.
\end{equation}
En passant à la limite,
\begin{equation}
    \lim_{N\to \infty} d_5\big( \sum_{k=0}^N10^k,-\frac{1}{ 9 } \big)=0,
\end{equation}
ce qui signifie que\footnote{Voir la définition \ref{DefGFHAaOL} de la convergence d'une série dans un espace métrique.}
\begin{equation}
    \sum_{k=0}^{\infty}10^k=-\frac{1}{ 9 }.
\end{equation}
