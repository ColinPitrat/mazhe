% This is part of Mes notes de mathématique
% Copyright (c) 2011-2015,2017
%   Laurent Claessens
% See the file fdl-1.3.txt for copying conditions.

%+++++++++++++++++++++++++++++++++++++++++++++++++++++++++++++++++++++++++++++++++++++++++++++++++++++++++++++++++++++++++++ 
\section{Théorème de Von Neumann}
%+++++++++++++++++++++++++++++++++++++++++++++++++++++++++++++++++++++++++++++++++++++++++++++++++++++++++++++++++++++++++++

\begin{lemma}[\cite{KXjFWKA}]
    Soit \( G\), un sous groupe fermé de \( \GL(n,\eR)\) et 
    \begin{equation}
        \mL_G=\{ m\in \eM(n,\eR)\tq  e^{tm}\in G\,\forall t\in\eR \}.
    \end{equation}
    Alors \( \mL_G\) est un sous-espace vectoriel de \( \eM(n,\eR)\).
\end{lemma}

\begin{proof}
    Si \( m\in\mL_G\), alors \( \lambda m\in\mL_G\) par construction. Le point délicat à prouver est le fait que si \( a,b\in \mL_G\), alors \( a+b\in\mL_G\). Soit \( a\in \eM(n,\eR)\); nous savons qu'il existe une fonction \( \alpha_a\colon \eR\to \eM\) telle que
    \begin{equation}
        e^{ta}=\mtu+ta+\alpha_a(t)
    \end{equation}
    et 
    \begin{equation}
        \lim_{t\to 0} \frac{ \alpha(t) }{ t }=0.
    \end{equation}
    Si \( a\) et \( b\) sont dans \( \mL_G\), alors \(  e^{ta} e^{tb}\in G\), mais il n'est pas vrai en général que cela soit égal à \(  e^{t(a+b)}\). Pour tout \( k\in \eN\) nous avons
    \begin{equation}
        e^{a/k} e^{b/k}=\left( \mtu+\frac{ a }{ k }+\alpha_a(\frac{1}{ k }) \right)\left( \mtu+\frac{ b }{ k }+\alpha_b(\frac{1}{ k }) \right)=\mtu+\frac{ a+b }{2}+\beta\left( \frac{1}{ k } \right)
    \end{equation}
   où \( \beta\colon \eR\to \eM\) est encore une fonction vérifiant \( \beta(t)/t\to 0\). Si \( k\) est assez grand, nous avons
   \begin{equation}
       \left\| \frac{ a+b }{ k }+\beta(\frac{1}{ k })  \right\|<1,
   \end{equation}
   et nous pouvons profiter du lemme \ref{LemQZIQxaB} pour écrire alors
   \begin{equation}
       \left(  e^{a/k} e^{b/k} \right)^k= e^{k\ln\big(\mtu+\frac{ a+b }{ k }+\beta(\frac{1}{ k })\big)}.
   \end{equation}
   Ce qui se trouve dans l'exponentielle est
   \begin{equation}
       k\left[ \frac{ a+b }{ k }+\alpha( \frac{1}{ k })+\sigma\left( \frac{ a+b }{ k }+\alpha(\frac{1}{ k }) \right) \right].
   \end{equation}
   Les diverses propriétés vues montrent que le tout tend vers \( a+b\) lorsque \( k\to \infty\). Par conséquent
   \begin{equation}
       \lim_{k\to \infty} \left(  e^{a/k} e^{b/k} \right)^k= e^{a+b}.
   \end{equation}
   Ce que nous avons prouvé est que pour tout \( t\), \(  e^{t(a+b)}\) est une limite d'éléments dans \( G\) et est donc dans \( G\) parce que ce dernier est fermé.
\end{proof}

Vu que \( \mL_G\) est un sous-espace vectoriel de \( \eM(n,\eR)\), nous pouvons considérer un supplémentaire \( M\).

\begin{lemma}   \label{LemHOsbREC}
    Il n'existe pas se suites \( (m_k)\) dans \( M\setminus\{ 0 \}\) convergeant vers zéro et telle que \(  e^{m_k}\in G\) pour tout \( k\).
\end{lemma}

\begin{proof}
    Supposons que nous ayons \( m_k\to 0\) dans \( M\setminus\{ 0 \}\) avec \(  e^{m_k}\in G\). Nous considérons les éléments \( \epsilon_k=\frac{ m_k }{ \| m_k \| }\) qui sont sur la sphère unité de \(\GL(n,\eR)\). Quitte à prendre une sous-suite, nous pouvons supposer que cette suite converge, et vu que \( M\) est fermé, ce sera vers \( \epsilon\in M\) avec \( \| \epsilon \|=1\). Pour tout \( t\in \eR\) nous avons
    \begin{equation}
        e^{t\epsilon}=\lim_{k\to \infty}  e^{t\epsilon_k}.
    \end{equation}
    En vertu de la décomposition d'un réel en partie entière et décimale, pour tout \( k\) nous avons \( \lambda_k\in \eZ\) et \( | \mu_k |\leq \frac{ 1 }{2}\) tel que \( t/\| m_k \|=\lambda_k+\mu_k\). Avec ça,
    \begin{equation}
        e^{t\epsilon}=\lim_{k\to \infty}\exp\Big( \frac{ t }{ m_k }m_k \Big)=\lim_{k\to \infty}  e^{\lambda_km_k} e^{\mu_km_k}.
    \end{equation}
    Pour tout \( k\) nous avons \(  e^{\lambda_km_k}\in G\). De plus \( | \mu_k |\) étant borné et \( m_k\) tendant vers zéro nous avons \(  e^{\mu_km_k}\to 1\). Au final
    \begin{equation}
        e^{t\epsilon}=\lim_{k\to \infty}  e^{t\epsilon_k}\in G
    \end{equation}
    Cela signifie que \( \epsilon\in\mL_G\), ce qui est impossible parce que nous avions déjà dit que \( \epsilon\in M\setminus\{ 0 \}\).
\end{proof}

\begin{lemma}   \label{LemGGTtxdF}
    L'application
    \begin{equation}
        \begin{aligned}
            f\colon \mL_G\times M&\to \GL(n,\eR) \\
            l,m&\mapsto  e^{l} e^{m} 
        \end{aligned}
    \end{equation}
    est un difféomorphisme local entre un voisinage de \( (0,0)\) dans \( \eM(n,\eR)\) et un voisinage de \( \mtu\) dans \( \exp\big( \eM(n,\eR) \big)\).
\end{lemma}
Notons que nous ne disons rien de \(  e^{\eM(n,\eR)}\). Nous n'allons pas nous embarquer à discuter si ce serait tout \( \GL(n,\eR)\)\footnote{Vu les dimensions y'a tout de même peu de chance.} ou bien si ça contiendrait ne fut-ce que \( G\).

\begin{proof}
    Le fait que \( f\) prenne ses valeurs dans \( \GL(n,\eR)\) est simplement dû au fait que les exponentielles sont toujours inversibles. Nous considérons ensuite la différentielle : si \( u\in \mL_G\) et \( v\in M\) nous avons
    \begin{equation}
        df_{(0,0)}(u,v)=\Dsdd{ f\big( t(u,v) \big) }{t}{0}=\Dsdd{  e^{tu} e^{tv} }{t}{0}=u+v.
    \end{equation}
    L'application \( df_0\) est donc une bijection entre \( \mL_G\times M\) et \( \eM(n,\eR)\). Le théorème d'inversion locale \ref{ThoXWpzqCn} nous assure alors que \( f\) est une bijection entre un voisinage de \( (0,0)\) dans \( \mL_G\times M\) et son image. Mais vu que \( df_0\) est une bijection avec \( \eM(n,\eR)\), l'image en question contient un ouvert autour de \( \mtu\) dans \( \exp\big( \eM(n,\eR) \big)\).
\end{proof}

\begin{theorem}[Von Neumann\cite{KXjFWKA,ISpsBzT,Lie_groups}]       \label{ThoOBriEoe}
    Tout sous-groupe fermé de \( \GL(n,\eR)\) est une sous-variété de \( \GL(n,\eR)\).
\end{theorem}
\index{théorème!Von Neumann}
\index{exponentielle!de matrice!utilisation}

\begin{proof}
    Soit \( G\) un tel groupe; nous devons prouver que c'est localement difféomorphe à un ouvert de \( \eR^n\). Et si on est pervers, on ne va pas faire localement difféomorphe à un ouvert de \( \eR^n\), mais à un ouvert d'un espace vectoriel de dimension finie. Nous allons être pervers.

    Étant donné que pour tout \( g\in G\), l'application 
    \begin{equation}
        \begin{aligned}
            L_g\colon G&\to G \\
            h&\mapsto gh 
        \end{aligned}
    \end{equation}
    est de classe \(  C^{\infty}\) et d'inverse \(  C^{\infty}\), il suffit de prouver le résultat pour un voisinage de \( \mtu\).

    Supposons d'abord que \( \mL_G=\{ 0 \}\). Alors \( 0\) est un point isolé de \( \ln(G)\); en effet si ce n'était pas le cas nous aurions un élément \( m_k\) de \( \ln(G)\) dans chaque boule \( B(0,r_k)\). Nous aurions alors \( m_k=\ln(a_k)\) avec \( a_k\in G\) et donc
    \begin{equation}
        e^{m_k}=a_k\in G.
    \end{equation}
    De plus \( m_k\) appartient forcément à \( M\) parce que \( \mL_G\) est réduit à zéro. Cela nous donnerait une suite \( m_k\to 0\) dans \( M\) dont l'exponentielle reste dans \( G\). Or cela est interdit par le lemme \ref{LemHOsbREC}. Donc \( 0\) est un point isolé de \( \ln(G)\). L'application \(\ln\) étant continue\footnote{Par le lemme \ref{LemQZIQxaB}.}, nous en déduisons que \( \mtu\) est isolé dans \( G\). Par le difféomorphisme \( L_g\), tous les points de \( G\) sont isolés; ce groupe est donc discret et par voie de conséquence une variété.

    Nous supposons maintenant que \( \mL_G\neq\{ 0 \}\). Nous savons par la proposition \ref{PropXFfOiOb} que 
    \begin{equation}
        \exp\colon \eM(n,\eR)\to \eM(n,\eR)
    \end{equation}
    est une application \(  C^{\infty}\) vérifiant \( d\exp_0=\id\). Nous pouvons donc utiliser le théorème d'inversion locale \ref{ThoXWpzqCn} qui nous offre donc l'existence d'un voisinage \( U\) de \( 0\) dans \( \eM(n,\eR)\) tel que \( W=\exp(U)\) soit un ouvert de \( \GL(n,\eR)\) et que \( \exp\colon U\to W\) soit un difféomorphisme de classe \(  C^{\infty}\).

    Montrons que quitte à restreindre \( U\) (et donc \( W\) qui reste par définition l'image de \( U\) par \( \exp\)), nous pouvons avoir \( \exp\big( U\cap\mL_G \big)=W\cap G\). D'abord \( \exp(\mL_G)\subset G\) par construction. Nous avons donc \( \exp\big( U\cap\mL_G \big)\subset W\cap G\). Pour trouver une restriction de \( U\) pour laquelle nous avons l'égalité, nous supposons que pour tout ouvert \( \mO\) dans \( U\), 
    \begin{equation}
        \exp\colon \mO\cap\mL_G\to \exp(\mO)\cap G
    \end{equation}
    ne soit pas surjective. Cela donnerait un élément de \( \mO\cap\complement\mL_G\) dont l'image par \( \exp\) n'est pas dans \( G\). Nous construisons ainsi une suite en considérant une boule \( B(0,\frac{1}{ k })\) inclue à \( U\) et \( x_k\in B(0,\frac{1}{ k })\cap\complement\mL_G\) vérifiant \(  e^{x_k}\in G\). Vu le choix des boules nous avons évidemment \( x_k\to 0\).

    L'élément \(  e^{x_k}\) est dans \(  e^{\eM(n,\eR)}\) et le difféomorphisme du lemme \ref{LemGGTtxdF}\quext{Il me semble que l'utilisation de ce lemme manque à l'avant-dernière ligne de la preuve chez \cite{KXjFWKA}.} nous donne \( (l_k,m_k)\in \mL_G\times M\) tel que \(  e^{l_k} e^{m_k}= e^{x_k}\). À ce point nous considérons \( k\) suffisamment grand pour que \(  e^{x_k}\) soit dans la partie de l'image de \( f\) sur lequel nous avons le difféomorphisme. Plus prosaïquement, nous posons
    \begin{equation}
        (l_k,m_k)=f^{-1}( e^{x_k})
    \end{equation}
    et nous profitons de la continuité pour permuter la limite avec \( f^{-1}\) :
    \begin{equation}
        \lim_{k\to \infty} (l_k,m_k)=f^{-1}\big( \lim_{k\to \infty}  e^{x_k} \big)=f^{-1}(\mtu)=(0,0).
    \end{equation}
    En particulier \( m_k\to 0\) alors que \(  e^{m_k}= e^{x_k} e^{-l_k}\in G\). La suite \( m_k\) viole le lemme \ref{LemHOsbREC}. Nous pouvons donc restreindre \( U\) de telle façon à avoir
    \begin{equation}
        \exp\big( U\cap\mL_G \big)=W\cap G.
    \end{equation}
    Nous avons donc un ouvert de \( \mL_G\) (l'ouvert \( U\cap\mL_G\)) qui est difféomorphe avec l'ouvert \( W\cap G\) de \( G\). Donc \( G\) est une variété et accepte \( \mL_G\) comme carte locale.

\end{proof}

\begin{remark}
    En termes savants, nous avons surtout montré que si \( G\) est un groupe de Lie d'algèbre de Lie \( \lG\), alors l'exponentielle donne un difféomorphisme local entre \( \lG\) et \( G\).
\end{remark}

%+++++++++++++++++++++++++++++++++++++++++++++++++++++++++++++++++++++++++++++++++++++++++++++++++++++++++++++++++++++++++++ 
\section{Densité des polynômes}
%+++++++++++++++++++++++++++++++++++++++++++++++++++++++++++++++++++++++++++++++++++++++++++++++++++++++++++++++++++++++++++

%---------------------------------------------------------------------------------------------------------------------------
\subsection{Théorème de Stone-Weierstrass}
%---------------------------------------------------------------------------------------------------------------------------

Voir le thème \ref{THEooPUIIooLDPUuq}.

Note : le lemme \ref{LemYdYLXb} est utilisé dans la démonstration du théorème \ref{ThoWmAzSMF}; c'est pour cela que nous l'avons isolé.

\begin{lemma}       \label{LemYdYLXb}
    Il existe une suite de polynômes sur \( \mathopen[ 0 , 1 \mathclose]\) convergent uniformément vers la fonction racine carré.
\end{lemma}

\begin{proof}
    Nous donnons cette suite par récurrence :
    \begin{subequations}
        \begin{align}
            P_0(t)&=0\\
            P_{n+1}(t)&=P_n(t)+\frac{ 1 }{2}\big( t-P_n(t)^2 \big).
        \end{align}
    \end{subequations}
    Nous commençons par montrer que pour tout \( t\in \mathopen[ 0 , 1 \mathclose]\), \( P_n(t)\in\mathopen[ 0 , \sqrt{t} \mathclose]\). Pour \( P_0\), c'est évident. Ensuite nous avons
    \begin{subequations}
        \begin{align}
            P_{n+1}(t)-\sqrt{t}&=P_n(t)-\sqrt{t}+\frac{ 1 }{2}(t-P_n(t)^2)\\
            &=\big( P_n(t)-\sqrt{t} \big)\left( 1-\frac{ 1 }{2}\frac{ t-P_n(t)^2 }{ P_n(t)-\sqrt{t} } \right)\\
            &=\big( P_n(t)-\sqrt{t} \big)\left( 1-\frac{ \sqrt{t}+P_n(t) }{2} \right)\\
            &\leq 0
        \end{align}
    \end{subequations}
    parce que \( \sqrt{t} \leq 1\) et \( P_n(t)\leq 1\) par hypothèse de récurrence.

    Nous savons au passage que \( P_n(t)\) est une suite réelle croissante parce que \( t-P_n(t)^2\geq t-(\sqrt{t})^2=0\). La suite \( P_n(t)\) est donc croissante et majorée par \( \sqrt{t}\); elle converge donc. Les candidats limites sont déterminés par l'équation
    \begin{equation}
        \ell=\ell+\frac{ 1 }{2}(t-\ell^2),
    \end{equation}
    dont les solutions sont \( \ell=\pm\sqrt{t}\). La suite étant positive, nous avons une convergence ponctuelle de \( P_n\) vers la racine carré. Cette suite étant une suite croissante de fonctions continues sur un compact, convergeant ponctuellement vers une fonction continue, la convergence est uniforme par le théorème de Dini \ref{ThoUFPLEZh}.
\end{proof}

\begin{lemma}           \label{LemUuxcqY}
    Soit \( K\), un compact de \( \eR\) et \( f_n\) une suite de fonctions sur \( K\) convergeant uniformément vers \( f\). Soit \( g\colon X\to K\) une fonction depuis un espace topologique \( K\). Alors \( f_n\circ g\) converge uniformément vers \( f\circ g\).
\end{lemma}

\begin{proof}
    En effet, pour tout \( x\in X\) nous avons
    \begin{equation}
        \| (f_n\circ g)-(f\circ g) \|_{\infty}=\sup_{x\in X} \| f_n\big( g(x) \big)-f\big( g(x) \big) \|\leq \| f_n-f \|_{\infty}.
    \end{equation}
    Par conséquent, si \( \epsilon\>0\) est donné, il suffit de choisir \( n\) de telle sorte à avoir \( \| f_n-f \|_{\infty}<\epsilon\) et nous avons \( \| (f_n\circ g)-(f\circ g) \|_{\infty}\leq \epsilon\).
\end{proof}

\begin{definition}
    Nous disons qu'une algèbre \( A\) de fonctions sur un espace \( X\) \defe{sépare les points}{sépare!les points} de \( X\) si pour tout \( x_1\neq x_2\) il existe \( g\in A\) telle que \( g(x_1)\neq g(x_2)\).
\end{definition}

Nous pouvons maintenant énoncer et démontrer une forme nettement plus générale du théorème de Stone-Weierstrass.
\begin{theorem}[Stone-Weierstrass\cite{MGecheleSW}] \label{ThoWmAzSMF}
    Soit \( X\), un espace compact et Hausdorff et \( A\) une sous algèbre de \( C(X,\eR)\) contenant une fonction constante non nulle. Alors \( A\) est dense dans \( \Big( C(X,\eR),\| . \|_{\infty}\Big)\) si et seulement si \( A\) sépare les points de \(X\).

    Nous pouvons remplacer \( \eR\) par \( \eC\) si de plus l'algèbre \( A\) est auto-adjointe : \( g\in A\) implique \( \bar g\in A\).
\end{theorem}
\index{théorème!Stone-Weierstrass}

\begin{proof}
    Nous allons écrire la démonstration en plusieurs étapes (dont la première est le lemme \ref{LemYdYLXb}).

    \begin{description}
        \item[Première étape] Pour tout \( x\neq y\in X\) et pour tout \( \alpha,\beta\in \eR\), il existe une fonction \( f\in A\) telle que \( f(x)=\alpha\) et \( f(y)=\beta\). 

            En effet, vu que \( A\) sépare les points nous pouvons considérer une fonction \( g\in A\) telle que \( g(x)\neq g(y)\) et ensuite poser
            \begin{equation}
                f(z)=\alpha+\frac{ \alpha-\beta }{ g(y)-g(x) }\big( g(z)-g(x) \big).
            \end{equation}
            Les constantes faisant partie de \( A\), cette fonction \( f\) est encore dans \( A\).

        \item[Seconde étape] Pour tout \( n\)-uples de fonctions \( f_1,\ldots, f_n\) dans \( \bar A\), les fonctions \( \min(f_1,\ldots, f_n)\) et \( \max(f_1,\ldots, f_n)\) sont dans \( \bar A\).

            Nous le démontrons pour \( n=2\); le reste allant évidemment par récurrence. Soient \( f,g\in \bar A\). Étant donné que
            \begin{subequations}
                \begin{align}
                    \max(f,g)&=\frac{ f+g }{2}+\frac{ | f-g | }{2}\\
                    \min(f,g)&=\frac{ f+g }{2}-\frac{ | f-g | }{2},
                \end{align}
            \end{subequations}
            if suffit de montrer que si \( f\in\bar A\) alors \( | f |\in \bar A\). Si \( f\) est nulle, c'est évident; supposons que \( f\neq 0\) et posons \( M=\| f \|_{\infty}\neq 0\). Pour tout \( x\in X\) nous avons
            \begin{equation}
                \frac{ f(x)^2 }{ M^2 }\in \mathopen[ 0 , 1 \mathclose].
            \end{equation}
            Nous considérons alors la suite
            \begin{equation}
                h_n=P_n\circ\frac{ f^2 }{ M^2 }
            \end{equation}
            où \( P_n\) est une suite de polynômes convergent uniformément vers la racine carré (voir lemme \ref{LemYdYLXb}). Le lemme \ref{LemUuxcqY} nous assure que \( h_n\) converge uniformément vers \( \frac{ | f | }{ M }\) dans \( C(X,\eR)\). Étant donné que \( \bar A\) est également une algèbre, \( h_n\) est dans \( \bar A\) pour tout \( n\) et la limite s'y trouve également (pour rappel, la fermeture \( \bar A\) est celle de la topologie de la convergence uniforme).

        \item[Troisième étape] Soit \( \epsilon>0\), \( f\in C(X,\eR)\) et \( x\in X\). Il existe une fonction \( g_x\in \bar A\) telle que 
            \begin{subequations}
                \begin{numcases}{}
                    g_x(x)=f(x)\\
                    g_x(y)\leq f(y)+\epsilon
                \end{numcases}
            \end{subequations}
            pour tout \( y\in X\).

            Soit \( z\in X\setminus\{ x \}\) et une fonction \( h_z\) telle que \( h_z(x)=f(x)\) et \( h_z(z)=f(z)\). Une telle fonction existe par une des étapes précédentes. Étant donné que \( f\) et \( h_z\) sont continues, il existe un voisinage ouvert \( V_z\) de \( z\) sur lequel
            \begin{equation}
                h_z(y)\leq f(y)+\epsilon
            \end{equation}
            pour tout \( y\in V_z\). Nous pouvons sélectionner un nombre fini de points \( z_1,\ldots, z_n\) tels que les ouverts \( V_{z_1},\ldots, V_{z_n}\) recouvrent \( X\) (parce que \( X\) est compact, de tout recouvrement par des ouverts, nous extrayons un sous recouvrement fini.). Nous posons 
            \begin{equation}
                g_x=\min(h_{z_1},\ldots, h_{z_n})\in \bar A.
            \end{equation}
            Si \( y\in X\), nous sélectionnons le \( i\) tel que \( h_{z_i}(y)\leq f(y)+\epsilon\) et nous avons
            \begin{equation}
                g_x(y)\leq h_{z_i}(y)\leq f(y)+\epsilon.
            \end{equation}
            
        \item[Étape \wikipedia{fr}{Final_Doom}{finale}] Soit \( \epsilon>0\) et \( f\in C(X,\eR)\). Pour chaque \( x\in X\) nous considérons une fonction \( g_x\in \bar A\) telle que
            \begin{subequations}
                \begin{numcases}{}
                    g_x(x)=f(x)\\
                    g_x(y)\leq f(y)+\epsilon
                \end{numcases}
            \end{subequations}
            pour tout \( y\in X\). Les fonctions \( f\) et \( g_x\) sont continues, donc il existe un voisinage ouvert \( W_x\) de \( x\) sur lequel
            \begin{equation}
                g_x(y)\geq f(y)-\epsilon.
            \end{equation}
            De ces \( W_x\) nous extrayons un sous recouvrement fini de \( X\) : \( W_{x_1},\ldots, W_{x_m}\) et nous posons
            \begin{equation}
                \varphi=\max(g_{x_1},\ldots, g_{x_n})\in \bar A.
            \end{equation}
            Si \( y\in X\), il existe un \( i\) tel que 
            \begin{equation}
                \varphi(y)\geq g_{x_i}(y)\geq f(y)-\epsilon.
            \end{equation}
            La première inégalité est le fait que \( \varphi\) est le maximum des \( g_{x_k}\), et la seconde est le choix de \( i\). Donc pour tout \( y\in X\) nous avons
            \begin{equation}        \label{EqJMxHaF}
                f(y)-\epsilon\leq \varphi(y)\leq f(y)+\epsilon.
            \end{equation}
            La première inégalité est ce que l'on vient de faire. La seconde est le fait que pour tout \( i\) nous ayons \( g_{x_i}(y)\leq f(y)+\epsilon\); le fait que \( \varphi\) soit le maximum sur les \( i\) ne change pas l'inégalité.

            Le fait que les inégalités \eqref{EqJMxHaF} soient vraies pour tout \( y\in X\) signifie que \( \| \varphi-f \|_{\infty}\leq \epsilon\), et donc que \( f\in \bar{\bar A}=\bar A\).
    \end{description}

    Tout cela prouve que \( C(X,\eR)\subset \bar A\). L'inclusion inverse est le fait que \( C(X,\eR)\) est fermé pour la norme \( \| . \|_{\infty}\), étant donné qu'une limite uniforme de fonctions continues est continue.

\end{proof}

Le théorème suivant est un des énoncés les plus classiques de Stone-Weierstrass. Il découle évidement du théorème général \ref{ThoWmAzSMF} (encore qu'il faut alors bien comprendre qu'il faut traiter la fonction \( x\mapsto \sqrt{x}\) séparément). Il en existe cependant une preuve indépendante.
%TODO : trouver cette preuve indépendante.
\begin{theorem}     \label{ThoGddfas}   \index{théorème!Stone-Weierstrass}
    Soit \( f\), une fonction continue de l'intervalle compact \( \mathopen[ a , b \mathclose]\) à valeurs dans \( \eR\). Alors pour tout \( \epsilon>0\), il existe un polynôme \( P\) tel que \( \| P-f \|_{\infty}<\epsilon\).

    Autrement dit, les polynômes sont denses dans \( C\mathopen[ a , b \mathclose]\) pour la norme uniforme.
\end{theorem}

\begin{corollary}   \label{CorRSczQD}
    Si \( X\subset \eR\) est compact et de mesure finie\footnote{Dans \( \eR\) cette hypothèse est évidemment superflue par rapport à l'hypothèse de compacité; mais ça suggère des généralisations \ldots}, alors l'ensemble des polynômes est denses dans \( \big( C(X,\eR),\| . \|_2 \big)\).
\end{corollary}

\begin{proof}
    Si \( f\) est une fonction dans \( C(X,\eR)\) et si \( \epsilon\geq 0\) est donné alors nous pouvons considérer un polynôme \( P\) tel que \( \| f-P \|_{\infty}\leq \epsilon\). Dans ce cas nous avons
    \begin{equation}
        \| f-P \|_2^2=\int_X| f(x)-P(x) |^2dx\leq \int_X\epsilon^2dx=\epsilon^2\mu(X)
    \end{equation}
    où \( \mu(X)\) est la mesure de \( X\) (finie par hypothèse).
\end{proof}

%--------------------------------------------------------------------------------------------------------------------------- 
\subsection{Primitive de fonction continue}
%---------------------------------------------------------------------------------------------------------------------------

\begin{proposition}[\cite{MQKDooSuEGxk}]    \label{PropQACVooBnHtRJ}
    Soit un intervalle compact \( K\) de \( \eR\) et une suite \( (f_n)\) de fonctions continues sur \( K\) telles que \( f_n\stackrel{unif}{\longrightarrow}f\). Si chacune des fonctions \( f_n\) a une primitive sur \( K\) alors \( f\) également.
\end{proposition}

\begin{proof}
    Soit \( x_0\in K\) et les primitives \( F_n\) choisies\footnote{Les fonctions \( F_n\) étant dérivables sont continues.} pour avoir \( F_n'f_n\) et \( F_n(x_0)=0\). Nous allons voir que \( (F_n)\) est une suite de Cauchy dans \( \big( K,\| . \|_{\infty} \big)\). Soient \( n,m\in \eN\) et \( x\in K\). Nous avons
    \begin{subequations}
        \begin{align}
            \| F_n-F_m \|_{\infty}&\leq \| F_n(x)-F_m(x) \|\\
            &=\| (F_n-F_m)(x) \|\\
            &\leq \| F'_n-F'_m \|_{[x,x_0]}\| x-x_0 \|
        \end{align}
    \end{subequations}
    où nous avons utilisé le théorème des accroissements finis \ref{ThoNAKKght}. Vu que \( x\in K\) et que \( K\) est borné, \( \| x-x_0 \|\) est majoré par \( \diam(K)\) et
    \begin{subequations}
        \begin{align}
            \| F_n-F_m \|_K\leq \| f_n-f_m \|_K\diam(K).
        \end{align}
    \end{subequations}
    Vu que \( (f_n) \) est de Cauchy, si \( n\) et \( m\) sont assez grands, cela tend vers zéro. La suite \( (F_n)\) converge donc vers une certaine fonction \( F\).

    Le théorème \ref{ThoSerUnifDerr} nous permet de permuter la limite et la dérivée pour conclure que \( F'=f\) et donc que \( f\) a une primitive sur \( K\).
\end{proof}

\begin{proposition}[\cite{MQKDooSuEGxk}]        \label{PropKKGAooDQYGKg}
    Soit un intervalle ouvert \( I\) de \( \eR\) et une fonction \( f\colon I\to \eR\) qui admet une primitive sur tout compact de \( I\). Alors \( f\) a une primitive sur \( I\).
\end{proposition}
\index{primitive!de fonction continue}

\begin{proof}
    Nous considérons une suite exhaustive\footnote{Voir le lemme \ref{LemGDeZlOo}.} de compacts \( K_n\) pour \( I\) et \( x_0\in K_0\). Nous considérons aussi \( F_n\) la primitive de \( f\) sur \( K_n\) telle que \( F_n(x_0)=0\) (possible parce que \( x_0\in K_n\) pour tout \( n\)). Les fonctions \( F_n\) sont des restrictions les une des autres, et nous pouvons définir
    \begin{equation}
        \begin{aligned}
            F\colon I&\to \eR \\
            x&\mapsto F_n(x)\text{ si } x\in K_n. 
        \end{aligned}
    \end{equation}
    Nous avons évidemment \( F(x_0)=0\) et nous allons prouver que \( F\) est une primitive de \( f\) sur \( I\). Soit \( x\in I\) vu que \( I\) est ouvert, nous pouvons choisir \( n_0\) tel que \( x\in\Int(K_{n_0})\). Les fonctions \( F\) et \( F_{n_0}\) sont égales sur \( K_n\) et donc sur un ouvert autour de \( x\). Par conséquent \( F\) est dérivable en \( x\) et \( F'(x)=F'_{n_0}(x)=f(x)\).
\end{proof}

\begin{theorem}    \label{ThoEXXyooCLwgQg}
    Soit \( I\) un intervalle ouvert de \( \eR\). Une fonction continue sur \( I\) admet une primitive\footnote{Définition \ref{DefXVMVooWhsfuI}.} sur \( I\).
\end{theorem}

\begin{proof}
    Sur chaque compact de \( I\), la fonction \( f\) est limite uniforme de polynômes\footnote{Si tu veux te passer de Stone-Weierstrass, tu peux prouver que toute fonction continue sur un compact est limite uniforme de fonctions affines par morceaux, par exemple. Voir \cite{MQKDooSuEGxk}.} (théorème de Stone-Weierstrass \ref{ThoGddfas}). Donc \( f\) est primitivable sur tout compact de \( I\) (proposition \ref{PropQACVooBnHtRJ}) et donc sur \( I\) par la proposition \ref{PropKKGAooDQYGKg}.
\end{proof}

\begin{proposition} \label{PropHFWNpRb}
    Soit \( I \) un intervalle borné ouvert de \( \eR\). Une fonction \( h\in C^{\infty}_c(I)\) admet une primitive dans \(  C^{\infty}_c(I)\) si et seulement si \( \int_Ih=0\).
\end{proposition}

\begin{proof}
    Si une primitive \( H\) de \( h\) est à support compact, alors
    \begin{equation}
        \int_Ih=H(b)-H(a)=0-0=0.
    \end{equation}
    Pas de problèmes dans ce sens.

    Supposons maintenant que \( \int_Ih=0\). Le fait que \( h\) admette une primitive dans \(  C^{\infty}(I)\) est évident : toute fonction continue admet une primitive\footnote{Théorème \ref{ThoEXXyooCLwgQg}.}. Soit \( H\) une telle primitive et \( \tilde H=H-H(b)\). Alors \( \tilde H(b)=0\) et 
    \begin{equation}
        \tilde H(a)=H(a)-H(b)=-\int_Ih=0.
    \end{equation}
    Nous rappelons que le support d'une fonction est \emph{la fermeture} de l'ensemble des points de non-annulation.

    Supposons que le support de \( h\) soit inclus dans \( \mathopen[ m , M \mathclose]\subset\mathopen] a , b \mathclose[\). En prenant des nombres \( m'\) et \( M'\) tels que \( a<m'<m\) et \( M<M'<b\) (nous insistons sur le caractère strict de ces inégalités), la fonction \( h\) est nulle sur \( \mathopen[ a , m' \mathclose]\) et sur \( \mathopen[ M' , b \mathclose]\); la fonction \( \tilde H\) doit donc y être constante. Mais nous avons déjà vu que \( \tilde H(a)=\tilde H(b)=0\). Donc l'ensemble des points sur lesquels \( \tilde H\) n'est pas nul est inclus à \( \mathopen] m' , M' \mathclose[\) et donc est strictement (des deux côtés) inclus à \( I\).
\end{proof}

%---------------------------------------------------------------------------------------------------------------------------
\subsection{Théorème taubérien de Hardi-Littlewood}
%---------------------------------------------------------------------------------------------------------------------------

Un théorème \defe{taubérien}{taubérien}\index{théorème!taubérien} est un théorème qui compare les modes de convergence d'une série.

\begin{lemma}
    Si \( f\) et \( g\) sont des fonctions continues, alors \( s(x)=\max\{ f(x),g(x) \}\) est également une fonction continue.
\end{lemma}

\begin{proof}
    Soit \( x_0\) et prouvons que \( s\) est continue en \( x_0\). Si \( f(x_0)\neq g(x_0)\) (supposons \( f(x_0)>g(x_0)\) pour fixer les idées), alors nous avons un voisinage de \( x_0\) sur lequel \( f>g\) et alors \( s=f\) sur ce voisinage et la continuité provient de celle de \( f\).

    Si au contraire \( f(x_0)=g(x_0)=s(x_0)\) alors si \( (a_n)\) est une suite tendant vers \( x_0\), nous prenons \( N\) tel que \( \big| f(a_n)-f(x_0) \big|\leq \epsilon\) pour tout \( n>N\) et \( M\) tel que \( \big| g(a_n)-g(x_0) \big|\leq \epsilon\) pour tout \( n> M\). Alors pour tout \( n>\max\{ N,M \}\) nous avons
    \begin{equation}
        \big| s(a_n)-s(x_0) \big|\leq \epsilon,
    \end{equation}
    d'où la continuité de \( s\) en \( x_0\).
\end{proof}

La proposition suivante dit que si une fonction connaît un saut, alors on peut le lisser par une fonction continue.
\begin{proposition} \label{PropTIeYVw}
    Soit \( f\) continue sur \( \mathopen[ a , x_0 [\) et sur \( \mathopen[ x_0 , b \mathclose]\) avec \( f(x_0^-)<f(x_0)\). En particulier nous supposons que \( f(x^-)\) existe et est finie. Alors pour tout \( \epsilon>0\), il existe une fonction continue \( s\) telle que sur \( \mathopen[ a , b \mathclose]\) on ait \( s\leq f\) et
    \begin{equation}
        \int_a^bs(x)-f(x)\,dx\leq \epsilon.
    \end{equation}
\end{proposition}

\begin{proof}
    Nous notons \( A\) la taille du saut :
    \begin{equation}
        A=f(x_0)-f(x_0^-).
    \end{equation}
    Quitte à changer \( a\) et \( b\), nous pouvons supposer que
    \begin{equation}
        f(x)<f(x_0)+\frac{ A }{ 3 }
    \end{equation}
    pour \( x\in \mathopen[ a , x_0 [\) et 
    \begin{equation}
        f>f(x_0)+\frac{ 2A }{ 3 }
    \end{equation}
    pour \( x\in \mathopen[ x_0 , b \mathclose]\). C'est le théorème des valeurs intermédiaires qui nous permet de faire ce choix.

    Soit \( m(x)\) la droite qui joint le point \( \big( x_0-\epsilon, f(x_0-\epsilon) \big)\) au point \( \big( x_0,f(x_0^+) \big)\). Nous posons
    \begin{equation}
        s(x)=\begin{cases}
            f(x)    &   \text{si } x<x_0-\epsilon\\
            \max\{ m(x),f(x) \}    &   \text{si } x_0-\epsilon\leq x\leq x_0\\
            f(x)    &    \text{si }x>x_0.
        \end{cases}
    \end{equation}
    En vertu des différents choix effectués, c'est une fonction continue. En effet
    \begin{equation}
        s(x_0-\epsilon)=\max\{ f(x_0-\epsilon),f(x_0,\epsilon) \}=f(x_0-\epsilon)
    \end{equation}
    et 
    \begin{equation}
        s(x_0)=\max\{ m(x_0),f(x_0^+) \}=f(x_0^+)
    \end{equation}
    parce que \( m(x_0)=f(x_0^+)\). En ce qui concerne l'intégrale, si nous posons
    \begin{equation}
        M=\sup_{x,y\in \mathopen[ a , b \mathclose]}| f(x)-f(y) |,
    \end{equation}
    nous avons
    \begin{equation}
        \int_a^bs-f=\int_{x_0-\epsilon}^{x_0}s-f\leq \epsilon M.
    \end{equation}
\end{proof}

\begin{lemma}\label{LemauxrKN}
    Pour tout polynôme \( P\), nous avons la formule
    \begin{equation}
        \lim_{x\to 1^-} (1-x)\sum_{n=0}^{\infty}x^nP(x^n)=\int_0^1P(x)dx.
    \end{equation}
\end{lemma}

\begin{proof}
    D'abord pour \( P=1\), la formule se réduit à la série harmonique connue. Ensuite nous prouvons la formule pour le polynôme \( P=X^k\) et la linéarité fera le reste pour les autres polynômes. Nous avons
    \begin{equation}
        (1-x)\sum_nx^nx^{kn}=(1-x)\sum_n(x^{1+k})^n=\frac{ 1-x }{ 1-x^{1+k} }=\frac{1}{ 1+x+\cdots+x^k }.
    \end{equation}
    Donc
    \begin{equation}
        \lim_{x\to 1^-} (1-x)\sum_nx^nP(x^n)=\frac{1}{ 1+k }.
    \end{equation}
    Par ailleurs, c'est vite vu que
    \begin{equation}
        \int_0^1 x^kdx=\frac{1}{ k+1 }.
    \end{equation}
\end{proof}

\begin{theorem}[Hardy-Littlewood\cite{ytMOpe}]\index{théorème!Hardy-Littlewood}\index{Hardy-Littlewood (théorème)}      \label{ThoPdDxgP}
    Soit \( (a_n)\) une suite réelle telle que
    \begin{enumerate}
        \item
            \( \frac{ a_n }{ n }\) tends vers une constante,
        \item
            \( F(x)=\sum_{n=0}^{\infty}a_nx^n\) a un rayon de convergence \( \geq 1\),
        \item
            \( \lim_{x\to 1^-} F(x)=l\).
    \end{enumerate}
    Alors \( \sum_{n=0}^{\infty}a_n=l\).
\end{theorem}
\index{convergence!suite numérique}
\index{série!nombres}
\index{série!fonctions}
\index{limite!inversion}
\index{approximation!par polynômes}

\begin{proof}
    Quitte à prendre la suite \( b_0=a_0-l\) et \( b_n=a_n\), on peut supposer \( l=0\).

    Soit \( \Gamma\) l'ensemble des fonctions
    \begin{equation}
         \gamma\colon \mathopen[ 0 , 1 \mathclose]\to \eR 
    \end{equation}
    telles que 
    \begin{enumerate}
        \item
            $\sum_{n=0}^{\infty}a_n\gamma(x^n)$ converge pour \( 0\leq x<1\),
        \item
            \( \lim_{x\to 1^-} \sum_{n\geq 0}a_n\gamma(^n)=0\).
    \end{enumerate}
    Ce \( \Gamma\) est un espace vectoriel.
    \begin{subproof}
    \item[Les polynômes sont dans \( \Gamma\)]
        Soit \( \gamma(t)=t^s\). Pour \( 0\leq x<1\) nous avons
        \begin{equation}
            \sum_{n=0}^{\infty}a_n\gamma(x^n)=\sum_{n=0}^{\infty}a_nx^{ns}<\sum_{n=0}^{\infty}a_nx^n.
        \end{equation}
        Donc la condition de convergence est vérifiée. En ce qui concerne la limite,
        \begin{equation}
            \lim_{x\to 1^-} \sum_{n=0}^{\infty}a_nx^{ns}=\lim_{x\to 1^-} F(x^s)=0
        \end{equation}
        parce que par hypothèse, \( \lim_{x\to 1^-} F(x)=0\).

    \item[Définition de la fonction qui va donner la réponse]
        Nous considérons la fonction \( g=\mtu_{\mathopen[ \frac{ 1 }{2} , 1 \mathclose]}\), c'est à dire
        \begin{equation}
            g(t)=\begin{cases}
                0    &   \text{si } 0\leq t<1/2\\
                1    &    \text{si } 1/2\leq t\leq 1.
            \end{cases}
        \end{equation}
        Nous montrons que si \( g\in \gamma\), alors le théorème est terminé. Si \( 0\leq x\leq 1\), on a \( 0\leq x^n<1/2\) dès que
        \begin{equation}
            n>-\frac{ \ln(2) }{ \ln(x) }
        \end{equation}
        avec une note comme quoi \( \ln(x)<0\), donc la fraction est positive. Nous désignons par \( N_x\) la partie entière de ce \( n\) adapté à \( x\). L'idée est que la fonction  \( g(x^n)\) est la fonction indicatrice de \(0 \leq n\leq N_x\), et donc
        \begin{equation}
            \sum_{n\geq 0}a_ng(x^n)=\sum_{n=0}^{N_x}a_n.
        \end{equation}
        Mais si \( x\to 1^-\), alors \( N_x\to \infty\), donc
        \begin{equation}
            \lim_{N\to \infty} \sum_{n=0}^Na_n=\lim_{x\to 1^-} \sum_{n=0}^{N_x}a_n=\lim_{x\to 1^-} \sum_{n\in \eN}a_ng(x^n),
        \end{equation}
        et cela fait zéro si \( g\in \Gamma\).
        
    \item[Approximation de \( g\) par des polynômes]

        Nous considérons la fonction
        \begin{equation}
            h(t)=\frac{ g(t)-t }{ t(1-1) }=\begin{cases}
                \frac{1}{ t-1 }    &   \text{si } t\in \mathopen[ 0 , 1/2 [\\
                \frac{1}{ t }    &    \text{si } t\in \mathopen[ 1/2 , 1 \mathclose].
            \end{cases}
        \end{equation}
        La seconde égalité est au sens du prolongement par continuité. La fonction \( h\) est une fonction non continue qui fait un saut de \( -2\) à \( 2\) en \( x=1/2\). En vertu de la proposition \ref{PropTIeYVw} (un peu adaptée), nous pouvons considérer deux fonctions continues \( s_1\) et \( s_2\) telles que
        \begin{equation}
            s_1\leq h\leq s_2
        \end{equation}
        et
        \begin{equation}
            \int_{0}^1s_2-s_1\leq \epsilon.
        \end{equation}
        Notons que l'inégalité \( s_1\leq s_2\) doit être stricte sur au moins un petit intervalle autour de \( x=1/2\). Soient \( P_1\) et \( P_2\), deux polynômes tels que \( \| P_1-s_1 \|_{\infty}\leq \epsilon\) et \( \| P_2-s_2 \|_{\infty}\leq \epsilon\) (ici la norme supremum est prise sur \( \mathopen[ 0 , 1 \mathclose]\)). C'est le théorème de Stone-Weierstrass (\ref{ThoGddfas}) qui nous permet de le faire.

        Nous posons aussi\footnote{À ce niveau, je crois qu'il y a une faute de frappe dans \cite{ytMOpe}.}
        \begin{subequations}
            \begin{align}
                Q_1=P_1+\epsilon\\
                Q_2=P_2-\epsilon.
            \end{align}
        \end{subequations}
        Nous avons
        \begin{equation}
            \int_0^1Q_1-Q_2\leq\int_0^1 Q_1-P_1+P_1-P_2+P_2-Q_2.
        \end{equation}
        Pour majorer cela, d'abord \( Q_1-P_1=P_2-Q2=\epsilon\), ensuite,
        \begin{equation}
            P_1-P_2=P_1-s_1+s_1-s_2+s_2-P_2
        \end{equation}
        dans lequel nous avons \( P_1-s_1\leq \epsilon\), \( s_2-P_2\leq \epsilon\) et \( \int_0^1s_1-s_2\leq\epsilon\). Au final, nous posons \( q=Q_2-Q_1\) et nous avons
        \begin{equation}
            \int_0^1q\leq 5\epsilon.
        \end{equation}
        Enfin nous posons aussi
        \begin{equation}
            R_i(x)=x+x(1-x)Q_i.
        \end{equation}
        Ces polynômes vérifient \( R_i(0)=0\), \( R_i(1)=1\) et
        \begin{equation}
            R_1\leq g\leq R_2
        \end{equation}
        parce que
        \begin{equation}
            Q_1\leq P_1\leq h\leq  P_2\leq Q_2
        \end{equation}
        et
        \begin{equation}
            t+t(1-t)Q_1\leq \underbrace{t+t(1-t)h(t)}_{g(t)}\leq t+t(1-t)Q_2.
        \end{equation}
        
    \item[Preuve que \( g\) est dans \( \Gamma\)]

        D'abord si \( 0\leq x<1\), \( x^N<\frac{ 1 }{2}\) pour un certain \( N\), et alors \( g(x^N)=0\). Du coup la série
        \begin{equation}
            \sum_{n=0}^{\infty}a_ng(x^n)=\sum_{n=0}^{N}a_n
        \end{equation}
        est une somme finie qui converge donc.

        D'autre part nous prenons \( M\) tel que \( | a_n |<\frac{ M }{ n }\) pour tout \( n\). Nous majorons \( \sum_{n \in \eN}a_ng(x^n)\) en utilisant \( R_1\). Mais vu que \( R_1\) est un polynôme, nous pouvons dire que \( | \sum_{n=0}^{\infty}a_nR_1(x^n) |\leq \epsilon\) en prenant \( x\in\mathopen[ \lambda , 1 [\) et \( \lambda\) assez grand. Nous avons :
        \begin{subequations}
            \begin{align}
                \left| \sum_{n=0}^{\infty}a_ng(x^n) \right| &\leq\left| \sum_{n=0}^{\infty}a_ng(x^n)-\sum_{n=0}^{\infty}a_nR_1(x^n) \right| +\underbrace{\left| \sum_{n=0}^{\infty}a_nR_1(x^n) \right|}_{\leq \epsilon} \\
                &\leq \epsilon+\sum_{n=0}^{\infty}| a_n |(g-R_1)(x^n)\\
                &\leq \epsilon+\sum_{n=0}^{\infty}| a_n |(R_2-R_1)(x^n)\\
                &\leq \epsilon+M\sum_{n=0}^{\infty}\frac{ x^n(1-x^n) }{ n }(Q_2-Q_1)(x^n)   &R_2-R_1=x(1-x)(Q_2-Q_1)\\
                &=\epsilon+M\sum_{n=0}^{\infty}\frac{ x^n(1-x^n) }{ n }q(x^n)\\
                &\leq \epsilon+M(1-x)\sum_nx^nq(x^n)   \label{subeqtZXDvu} 
            \end{align}
        \end{subequations}
        où la ligne \eqref{subeqtZXDvu} provient d'une majoration sauvage de \( 1/n\) par \( 1\) et de \( 1-x^n\) par \( 1-x\). Par le lemme \ref{LemauxrKN}, nous avons alors
        \begin{equation}
            \lim_{x\to 1^-} | \sum_na_ng(x^n) |\leq \epsilon+M\int_0^1q\leq 6\epsilon.
        \end{equation}
    \end{subproof}
\end{proof}

%---------------------------------------------------------------------------------------------------------------------------
\subsection{Théorème de Müntz}
%---------------------------------------------------------------------------------------------------------------------------

\begin{theorem}[Théorème de Müntz\cite{jqZSyG,oYGash,ooRIPFooALoEWM}]  \label{ThoAEYDdHp}
    Soit \( C_0\big( \mathopen[ 0 , 1 \mathclose] \big)\), l'espace des fonctions continues sur \( \mathopen[ 0 , 1 \mathclose]\) muni de la norme \( \| . \|_{\infty}\) ou \( \| . \|_2\) et une suite \( (\alpha_n)\) strictement croissante de nombres positifs. Nous notons \( \phi_{\lambda}\) la fonction \( x\mapsto x^{\lambda}\).

    Alors 
    \begin{equation}
        \overline{  \Span\{1, \phi_{\alpha_n} \} }   
    \end{equation}
    est dense dans \( C_0\big( \mathopen[ 0 , 1 \mathclose] \big)\)  si et seulement si 
    \begin{equation}
        \sum_{n=2}^{\infty}\frac{1}{ \alpha_n }=+\infty.
    \end{equation}
\end{theorem}

Nous prouvons le théorème pour la norme \( \| . \|_2\).
\begin{proof}
    Soit \( m\in \eR^+\); nous notons \( \Delta_N(m)\) la distance entre \( \phi_m\) et \( \Span\{ \phi_{\alpha_1},\ldots, \phi_{\alpha_N} \}\). Cette distance peut être évaluée avec le déterminant de Gram\index{déterminant!Gram} (proposition \ref{PropMsZhIK})
    \begin{equation}
        \Delta_N(m)^2=\frac{ G(\phi_m,\phi_{\alpha_1},\ldots, \phi_{\alpha_N}) }{ G(\phi_{\alpha_1},\ldots, \phi_{\alpha_N}) }.
    \end{equation}
    Pour calculer cela nous avons besoin des produits scalaires\footnote{C'est ici qu'on se particularise à la norme \( \| . \|_2\).}
    \begin{equation}
        \langle \phi_a, \phi_b\rangle =\int_0^1 x^{a+b}dx=\frac{1}{ a+b+1 }.
    \end{equation}
    Pour avoir des notation plus compactes, nous notons \( \alpha_0=m\). Donc nous avons à calculer le déterminant
    \begin{equation}
        G(\phi_m,\phi_{\alpha_1},\ldots, \phi_{\alpha_N})=\det\begin{pmatrix}
            \frac{1}{ \alpha_i+\alpha_j+1 }
         \end{pmatrix}
    \end{equation}
    où \( i,j=0,\ldots, N\). Nous reconnaissons un déterminant de Cauchy (proposition \ref{ProptoDYKA})\index{déterminant!Cauchy} en posant, dans \( \frac{1}{ \alpha_i+\alpha_j+1 }\), \( a_i=\alpha_i\) et \( b_j=\alpha_j+1\). Étant donné que \( b_j-b_i=a_j-a_i\), nous avons
    \begin{equation}
        G(\phi_m,\phi_{\alpha_1},\ldots, \phi_{\alpha_N})=\frac{ \prod_{0\leq i<j\leq N}  (\alpha_j-\alpha_i)^2 }{ \prod_{i=0}^N\prod_{j=0}^N (\alpha_i+\alpha_j+1).}
    \end{equation}
    Nous séparons maintenant les termes où \( i\) ou \( j\) sont nuls. En ce qui concerne le dénominateur, il faut prendre tous les couples \( (i,j)\) avec \( i\) et \( j\) éventuellement égaux à zéro. Nous décomposant cela en trois paquets. Le premier est \( (0,0)\); le second est \( (0,i)\) (chaque couple arrive en fait deux fois parce qu'il y a aussi \( (i,0)\)); et le troisième sont les \( i,j\) tous deux différents de zéro :
    \begin{equation}
        (2m+1)\prod_{ij}(\alpha_i+\alpha_j+1)\prod_i(\alpha_i+m+1)^2.
    \end{equation}
    Notons que dans le produit central, le carré est contenu dans le fait qu'on écrit \( \prod_{ij}\) et non \( \prod_{i<j}\). Nous avons donc
    \begin{equation}
        G(\phi_m,\phi_{\alpha_1},\ldots, \phi_{\alpha_N})=\frac{ \prod_{i<j}(\alpha_i-\alpha_j)^2\prod_i(\alpha_i-m)^2 }{ (2m+1)\prod_{ij}(\alpha_i+\alpha_j+1)\prod_i(\alpha_i+m+1)^2 }.
    \end{equation}
    
    Le calcul de \( G(\phi_{\alpha_1},\ldots, \phi_{\alpha_N})\) est plus simple\footnote{Je crois qu'il y a une faute de frappe dans le dénominateur de \cite{jqZSyG}.} :
    \begin{equation}
        G(\phi_{\alpha_1},\ldots, \phi_{\alpha_N})=\frac{ \prod_{i<j}(\alpha_i-\alpha_j)^2 }{ \prod_{ij}(\alpha_i+\alpha_j+1) }.    
    \end{equation}
    En divisant l'un par l'autre il ne reste que les facteurs comprenant \( m\) et en prenant la racine carré,
    \begin{equation}    \label{EqANiuNB}
        \Delta_N(m)=\frac{1}{ \sqrt{2m+1} }\prod_{i=1}^N\left| \frac{ \alpha_i-m }{ \alpha_i+m+1 } \right| .
    \end{equation}
    
    Nous passons maintenant à la preuve proprement dite. Supposons que \( V=\Span\{ \phi_{\alpha_i},i\in \eN \}\) est dense. Si \( m\) est un des \( \alpha_i\), il peut évidemment être approché par les \( \phi_{\alpha_i}\). Mais vue la densité de \( V\), un \( \phi_m\) avec \( m\neq \alpha_i\) (pour tout \( i\)) alors \( \phi_m\) peut également être arbitrairement approché par les \( \phi_{\alpha_i}\), c'est à dire que
    \begin{equation}
        \lim_{N\to \infty} \Delta_N(m)=0.
    \end{equation}
    Nous posons 
    \begin{equation}
        u_n=\ln\left( \frac{ \alpha_n-m }{ \alpha_n+m+1 } \right)
    \end{equation}
    et nous prouvons que la série \( \sum_nu_n\) diverge. En effet nous nous souvenons de la formule \( \ln(ab)=\ln(a)+\ln(b)\), de telle sorte que la \( N\)\ieme somme partielle de \( \sum_nu_n\) est
    \begin{equation}
        \ln\left( \frac{ \alpha_1-m }{ \alpha_1+m+1 }\cdot\ldots\cdot \frac{ \alpha_N-m }{ \alpha_N+m+1 } \right)=\ln\left( \sqrt{2m+1}\Delta_N(m) \right),
    \end{equation}
    qui tends vers \( -\infty\) lorsque \( N\to \infty\).

    Si la suite \( (\alpha_n)\) est majorée et plus généralement si nous n'avons pas \( \alpha_n\to \infty\), alors évidemment la série \( \sum_n\frac{1}{ \alpha_n }\) diverge. Nous supposons donc que \( \lim_{n\to \infty} \alpha_n=\infty\). Nous avons aussi\quext{Je crois qu'il y a une faute de signe dans la dernière expression de \cite{oYGash}.}
    \begin{equation}
        u_n=\ln\left( \frac{ \alpha_n-m }{ \alpha_n+m+1 } \right)=\ln\left( 1-\frac{ 2m+1 }{ \alpha_n+m+1 } \right)\sim-\frac{ 2m+1 }{ \alpha_n }.
    \end{equation}
    Une justification est donné à l'équation \eqref{EqGICpOX}. Ce que nous avons surtout est
    \begin{equation}
        \sum_n u_n\sim -(2m+1)\sum_n\frac{1}{ \alpha_n }.
    \end{equation}
    Étant donné que la série de gauche diverge, celle de droite diverge\footnote{Nous utilisons le fait que si \( u_n=\sum v_n\) en tant que suites et si \( \sum_nu_n\) diverge, alors \( \sum_nv_n\) diverge.}.

    Nous faisons maintenant le sens opposé : nous supposons que la série \( \sum_n1/\alpha_n\) diverge et nous nous posons
    \begin{equation}
        V=\Span\{ \phi_{\alpha_n}\tq n\in \eN \}.
    \end{equation}
    Il suffit de prouver que \( \phi_m\in \bar V\) pour tout \( m\) parce qu'un corollaire du théorème de Stone-Weierstrass \ref{CorRSczQD} montre que \( \Span\{ \phi_k\tq k\in \eN \}\) est dense dans \( C\) pour la norme \( \| . \|_2\). 
    
    Si \( \alpha_n\to \infty\), nous avons :
    \begin{equation}
        u_n\sim\frac{ 2m+1 }{ \alpha_n }\to 0
    \end{equation}
    et alors \( \Delta_N(m)\to 0\). Dans ce cas nous avons immédiatement \( \phi_m\in \bar V\).

    Si par contre \( \alpha_n\) ne tend pas vers l'infini, nous repartons de l'expression \eqref{EqANiuNB}, nous posons \( 0<\alpha=\sup_i\alpha_i\) et nous calculons :
    \begin{subequations}
        \begin{align}
            \sqrt{2m+1}\Delta_N(m)&=\prod_{i=1}^N\frac{ | \alpha_i-m | }{ \alpha_i+m+1 }\\
            &\leq \prod_{i=1}^N\frac{ \alpha_i+m }{ \alpha_i+m+1 }\\
            &=\prod_{i=1}^N\left( 1-\frac{ 1 }{ \alpha_i+m+1 } \right)\\
            &\leq \prod_{i=1}^N\left( 1-\frac{1}{ \alpha+m+1 } \right)\\
            &=\left( 1-\frac{1}{ \alpha+m+1 } \right)^N.
        \end{align}
    \end{subequations}
    Cette dernière expression tend vers \( 0\) lorsque \( N\to \infty\).
\end{proof}

\begin{remark}      \label{REMooGPYYooCQJwFa}
    Certaines sources\footnote{Dont le rapport du jury 2014} citent le théorème de Müntz comme ceci (avec un implicite que \( \alpha_i\neq 0\)):
    \begin{equation}        \label{EQooPCSZooUDSzwQ}
        \overline{ \Span\{1, \phi_{\alpha_i} \} }=C\big( \mathopen[ 0 , 1 \mathclose] \big) \Leftrightarrow \sum_{i\geq 1}\frac{1}{ \alpha_i }=+\infty.
    \end{equation}
    Que penser de la présence explicite du \( 1\) (c'est à dire de \( \phi_0\)) ou non dans l'ensemble ?

    Première chose : la présence éventuelle de \( \phi_0\) est la raison pour laquelle nous faisons commencer la somme à \( i=2\) et non \( i=1\). Dans le même ordre d'idée, si $\Span\{ \phi_{\alpha_i} \}$  est dense, alors en prenant n'importe quelle queue de suite, ça reste dense.

    Prouvons donc l'énoncé \eqref{EQooPCSZooUDSzwQ}. Si \( \Span\{ 1,\phi_{\alpha_i} \}\) est dense, alors en posant \( \beta_1=0\), \( \beta_i=\alpha_{i-1}\) notre théorème prouve que \( \sum_{\beta=2}^{\infty}\frac{1}{ \beta_i }=+\infty\), cela est exactement que \( \sum_{i=1}^{\infty}\frac{1}{ \alpha_i }=+\infty\). Dans l'autre sens, si \( \sum_{i\geq 1}\frac{1}{ \alpha_i }=+\infty\), alors nous avons aussi \( \sum_{i\geq 2}\frac{1}{ \alpha_i }=+\infty\) et notre théorème dit que \( \Span \{ \phi_{\alpha_i} \}\) est dense. A fortiori, \( \Span\{ 1,\phi_{\alpha_i} \}\) est dense.
\end{remark}

\begin{example}
    Nous savons depuis le théorème \ref{ThonfVruT} que la somme des inverses des nombres premiers diverge.
\end{example}
