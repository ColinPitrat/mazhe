% This is part of Le Frido
% Copyright (c) 2017
%   Laurent Claessens
% See the file fdl-1.3.txt for copying conditions.

%+++++++++++++++++++++++++++++++++++++++++++++++++++++++++++++++++++++++++++++++++++++++++++++++++++++++++++++++++++++++++++ 
\section{Méthode des caractéristiques pour l'ordre \( 1\)}
%+++++++++++++++++++++++++++++++++++++++++++++++++++++++++++++++++++++++++++++++++++++++++++++++++++++++++++++++++++++++++++
\label{SECooHKSLooOCYNDz}

Nous\cite{ooEIHMooRXOzwa,ooAUICooVUjyqo} voulons étudier l'équation d'ordre \( 1\)
\begin{equation}        \label{EQooGJKZooCCEpRj}
    a(x,y)\frac{ \partial u }{ \partial x }(x,y)+b(x,y)\frac{ \partial u }{ \partial y }(x,y)+c(x,y)u(x,y)=f(x,y)
\end{equation}
Le champ de vecteurs associé à cette équation est
\begin{equation}
    v=\begin{pmatrix}
        a    \\ 
        b    
    \end{pmatrix},
\end{equation}
et l'équation peut être écrite sous la forme
\begin{equation}
    (v\cdot\nabla)+cu=f.
\end{equation}

\begin{definition}
    Le \defe{flot}{flot} de ce champ de vecteurs sont les courbes paramétriques \( \gamma(t)=\big( x(t), y(t) \big)\) vérifiant \( \gamma't(t)=v\big( \gamma(t) \big)\).
\end{definition}
Les équation du flot pour l'équation \eqref{EQooGJKZooCCEpRj} sont
\begin{subequations}        \label{EQooURBTooVdFzZR}
            \begin{numcases}{}
                x'(t)=a\big( x(y),y(t) \big)\\
                y'(t)=b\big( x(y),y(t) \big).
            \end{numcases}
        \end{subequations}
Ce sont des équations différentielles ordinaires. Un système de deux équations couplées du premier ordre.

Quel est l'intérêt du flot ? Nous allons voir que sur la ligne \( t\mapsto\gamma(t)\), la fonction \( u\) est constante. Or des solutions \( \gamma\) au système \eqref{EQooURBTooVdFzZR}, il y en aura plusieurs : une pour chaque valeurs des constantes d'intégration. Pour peu que ces lignes recouvrent tout le plan, nous pourrons résoudre l'équation de départ ligne par ligne.

Nous posons
\begin{subequations}
    \begin{align}
        \tilde u(t)&=u\big( x(t),y(t) \big)\\
        \tilde c(t)&=c\big( x(t),y(t) \big)\\
        \tilde f(t)&=f\big( x(t),y(t) \big).
    \end{align}
\end{subequations}
La fonction \( \tilde u\) est une fonction \( \eR\to \eR\) normale qui se dérive normalement, en suivant la règle de dérivation des fonctions composées :
\begin{subequations}
    \begin{align}
    \tilde u'(t)&=\frac{ \partial u }{ \partial x }\big( x(t),y(t) \big)x'(t)+\frac{ \partial u }{ \partial y }\big( x(t),y(t) \big)y'(t)\\
    &=a\frac{ \partial u }{ \partial x }+b\frac{ \partial u }{ \partial y }\\
    &=f\big( x(t),y(t) \big)-c\big( x(t),y(t) \big)u\big( x(t),y(t) \big)\\
    &=\tilde f(t)-\tilde c(t)\tilde u(t).
    \end{align}
\end{subequations}
Nous avons pour \( \tilde u\) l'équation différentielle ordinaire
\begin{equation}
    \tilde u'+\tilde c\tilde u=\tilde f
\end{equation}
qui est résolue par la proposition \ref{PROPooZCXQooPQpkdQ}.

%--------------------------------------------------------------------------------------------------------------------------- 
\subsection{Un exemple complet un peu minimal}
%---------------------------------------------------------------------------------------------------------------------------

Nous considérons l'équation différentielle\cite{ooEIHMooRXOzwa}
\begin{equation}
    \frac{ \partial u }{ \partial x }-\frac{ \partial u }{ \partial y }-(x-y)u=0.
\end{equation}
Et nous allons la résoudre.

Les équations du flot, sont simples parce que les coefficient sont des constantes : \( x'(t)=1\), \( y'(t)=-1\). Donc
\begin{subequations}
    \begin{align}
        x(t)&=t+C_1\\
        y(t)&=-t+C_2.
    \end{align}
\end{subequations}
A priori nous avons une caractéristique pour chaque choix de \( (C_1,C_2)\) et nous espérons que le tout recouvre le plan \( \eR^2\). En fait seule une des deux constantes doit être laissée libre, l'autre consiste seulement en décaler le paramètre \( t\). Nous posons donc \( C_1=0\) et nous considérons les courbes caractéristiques
\begin{equation}
    \gamma_C(t)=\begin{pmatrix}
        t    \\ 
        -t+C    
    \end{pmatrix}.
\end{equation}
Ces courbes recouvrent bien tout le plan. Pour savoir les valeurs de \( u\) sur la courbe \( \gamma_C\), nous devons résoudre l'équation différentielle ordinaire 
\begin{equation}
    \tilde u_C'+\tilde c\tilde u_C=\tilde f,
\end{equation}
en sachant que \( \tilde c(t)=c\big( x(t),y(t) \big)=-\big( x(t)-y(t) \big)=2t-C\). Cela se fait en suivant la méthode décrite dans l'exemple \ref{EXooVVLGooPWaHUI} et résumée dans la proposition \ref{PROPooZCXQooPQpkdQ}.

En termes de notations, \( \tilde u_C(t)=u\big( \gamma_C(t) \big)\). Récrivons l'équation :
\begin{equation}
    \tilde u'(t)-(2t-C)\tilde u(t)=0.
\end{equation}
La méthode pour la résoudre est de mettre les \( \tilde u\) d'un côté et les \( t\) de l'autre :
\begin{equation}
    \frac{ \tilde u' }{ u }=2t-C.
\end{equation}
En intégrant par rapport à \( t\) des deux côtés,
\begin{equation}
    \ln(\tilde u)=t^2-Ct+K_C,
\end{equation}
c'est à dire (avec redéfinition de \( K_C\))
\begin{equation}     
    \tilde u(t)=K_C e^{t^2-Ct}
\end{equation}
ou encore
\begin{equation}   \label{EQooSSTJooEQfRnP}
    u\big( \gamma_C(t) \big)=K_C e^{t^2-Ct}
\end{equation}
où \( C\) est le paramètre que nous déterminons en sachant sur quelle caractéristique se trouve le point \( (x,y)\) où nous voulons calculer \( u(x,y)\) et \( K\) est une constante (strictement positive parce que si vous avez suivi le mouvement, c'est une exponentielle) qui doit être déterminée par les conditions initiales. Dès que \( K\) est fixé pour un des points de la courbe \( \gamma_C\), alors il est fixé pour tous les points.

Ce que nous avons obtenu est qu'il existe un \( K_C\) tel que pour tout \( t\) nous avons
\begin{equation}
    u\big( \gamma_C(t) \big)=K_C e^{t^2-Ct}.
\end{equation}

Soit donc un point \( (x_0,y_0)\in \eR^2\). Nous devons d'abord déterminer où ce point se trouve par rapport aux caractéristiques, c'est à dire quelle est la valeur de \( C\) pour laquelle \( (x_0,y_0)\) est sur la courbe \( \gamma_C\), et ensuite déterminer pour quelle valeur de \( t\) nous aurons \( \gamma_C(t)=(x_0,y_0)\). À résoudre :
\begin{equation}
    \gamma_C(t_0)=\begin{pmatrix}
        t_0    \\ 
        -t_0+C    
    \end{pmatrix}=\begin{pmatrix}
        x_0    \\ 
        y_0    
    \end{pmatrix}.
\end{equation}
Donc \( t_0=x_0\) et \( C=x_0+y_0\). En reprenant \eqref{EQooSSTJooEQfRnP} nous avons
\begin{equation}
    u\big( \gamma_C(t_0) \big)=K e^{x_0^2-(x_0+y_0)x_0}=K e^{-x_0y_0}.
\end{equation}

Pour peu que des conditions soient donnée sur chaque caractéristique, nous pouvons déterminer \( K\). Attention : ce \( K\) est une constante d'intégration de l'équation différentielle ordinaire pour \( \tilde u\). Donc elle n'est valable que sur chaque caractéristique séparément. Cela n'est donc pas du tout une constante sur \( \eR^2\).

Nous pouvons maintenant écrire la solution générale de l'équation de départ. L'équation cartésienne de la courbe \( \gamma_C\) est
\begin{equation}
    x+y=C.
\end{equation}
Donc \( K\) est une fonction de \( x+y\), pas de \( x\) et \( y\) séparément. Cela est important à comprendre. A priori nous avons
\begin{equation}
    u(x,y)=K(x,y) e^{-xy}
\end{equation}
où \( K(x,y)\) est constante sur la courbe \( \gamma_C\) contenant \( (x,y)\). Nous avons 
\begin{itemize}
    \item si \( x_1+y_1=x_2+y_2\),
    \item alors il existe \( C\) tel que \( (x_1,y_1)\) et \( (x_2,y_2)\) sont sur \( \gamma_C\),
    \item alors \( K(x_1,y_2)=K(x_2,y_2)\).
\end{itemize}
Donc il existe une fonction \( \eR\to \eR\) telle que \( K(x,y)=f(x+y)\). 

Au final, la solution générale de l'équation est
\begin{equation}
    u(x,y)=f(x+y) e^{-xy}
\end{equation}
où \( f\) est une fonction à déterminer par les conditions initiales qui peuvent être données. Typiquement nous espérons que les conditions imposent une et une seule valeur de \( u\) sur chacune des courbes \( \gamma_C\).

%--------------------------------------------------------------------------------------------------------------------------- 
\subsection{Un théorème d'existence et d'unicité}
%---------------------------------------------------------------------------------------------------------------------------

\begin{proposition}[Équation de transport à coefficients variables\cite{ooAUICooVUjyqo,ooPMPXooEpbDkm}]
    Soit une fonction \( c\colon \eR^2\to \eR\) continue et uniformément Lipschitziennes en sa première variable et \( g\in C^1(\eR)\). Alors l'équation aux dérivées partielles de premier ordre
    \begin{subequations}        \label{SUBEQSooKWBFooKkpihH}
        \begin{numcases}{}
        \frac{ \partial u }{ \partial x }(x,y)+c(x,y)\frac{ \partial u }{ \partial y }(x,y)=0\\
        u(0,y)=g(y)
        \end{numcases}
    \end{subequations}
    admet une unique solution de classe \( C^1\).

    Cette solution est construite de la façon suivante\footnote{Le fait que la construction ait un sens fait partie des choses à prouver}. D'abord nous considérons la solution \( Y\) au problème
    \begin{subequations}
        \begin{numcases}{}
            \frac{ \partial Y }{ \partial s }(s;x,y)=c\big( s,Y(s;x,y) \big)\\
            Y(x;x,y)=y,
        \end{numcases}
    \end{subequations}
    et ensuite le problème \eqref{SUBEQSooKWBFooKkpihH} a pour unique solution
    \begin{equation}
        u(x,y)=g\big( Y(0;x,y) \big).
    \end{equation}
\end{proposition}

\begin{proof}
    
\end{proof}
<++>

%+++++++++++++++++++++++++++++++++++++++++++++++++++++++++++++++++++++++++++++++++++++++++++++++++++++++++++++++++++++++++++ 
\section{Méthode des caractéristique pour l'ordre \( 2\)}
%+++++++++++++++++++++++++++++++++++++++++++++++++++++++++++++++++++++++++++++++++++++++++++++++++++++++++++++++++++++++++++

%--------------------------------------------------------------------------------------------------------------------------- 
\subsection{Principe général}
%---------------------------------------------------------------------------------------------------------------------------

Soit l'opérateur différentiel agissant sur \( C^2(\eR^2)\) :
\begin{equation}
    D=a(x,y)\frac{ \partial^2 }{ \partial x^2 }+b(x,y)\frac{ \partial^2 }{ \partial x\partial y }+c(x,y)\frac{ \partial^2 }{ \partial y^2 }.
\end{equation}
Nous voulons résoudre des équations du type \( Du=0\) pour \( u\colon \eR^2\to \eR\).

Pour commencer\cite{ooEIHMooRXOzwa}, et c'est le point crucial, nous voyons \( D\) comme un polynôme en \( \partial_x\) et \( \partial_y\) et nous le factorisons : si
\begin{equation}
    aX^2+bXY+cY=(\alpha X+\beta Y)(\gamma X+\delta Y)
\end{equation}
alors nous avons
\begin{equation}
    D=\left( \alpha\frac{ \partial  }{ \partial x }+\beta\frac{ \partial  }{ \partial y } \right)\left( \gamma\frac{ \partial  }{ \partial x }+\delta\frac{ \partial y }{ \partial  } \right)+\text{ termes d'ordre inférieurs}.
\end{equation}
Les «termes d'ordre inférieurs» sont deux comme \( \alpha(x,y)\frac{ \partial \delta }{ \partial x }\frac{ \partial  }{ \partial y }\).


L'astuce est de poser 
\begin{equation}
    v=(\gamma\partial_x+\delta\partial_y)u,
\end{equation}
et de résoudre le système
\begin{subequations}        \label{SUBESQooGCMNooEDWQHd}
    \begin{numcases}{}
        (\alpha\partial_x+\beta\partial_y)v=0\\
        (\gamma\partial_x+\delta\partial_y)u=v.
    \end{numcases}
\end{subequations}
Cela sont deux équations différentielles du premier ordre pour lesquelles nous avons déjà des techniques décrites en la section \ref{SECooHKSLooOCYNDz}.

Afin que les fonctions \( \alpha\), \( \beta\), \( \gamma\) et \( \delta\) soient réelles, il faut que \( b^2-4ac\geq 0\). Sachant que \( a=\alpha\gamma\), \( b=\alpha\delta+\beta\gamma\) et \( c=\beta\delta\) cette condition sur \( a\), \( b\) et \( c\) donne
\begin{equation}
    (\alpha\delta+\beta\gamma)^2-4\alpha\gamma\beta\delta\geq 0.
\end{equation}
Cela revient à
\begin{equation}
    (\alpha\delta-\beta\gamma)^2\geq 0.
\end{equation}
Nous supposons à présent que l'inégalité soit stricte (cas hyperbolique). Nous avons en particulier que
\begin{equation}
    \alpha\delta-\beta\gamma\neq 0.
\end{equation}
Cette condition implique que les équations
\begin{equation}
    \begin{aligned}[]
        \frac{ dx }{ dt }=\alpha(x,y)&&\frac{ dy }{ dt }=\beta(x,y)
    \end{aligned}
\end{equation}
sont indépendantes des équations
\begin{equation}
    \begin{aligned}[]
        \frac{ dx }{ dt }=\gamma(x,y)&&\frac{ dy }{ dt }=\delta(x,y)
    \end{aligned}
\end{equation}
Ce sont les équations caractéristiques des équations \eqref{SUBESQooGCMNooEDWQHd}.

%--------------------------------------------------------------------------------------------------------------------------- 
\subsection{Exemple : l'équation d'onde}
%---------------------------------------------------------------------------------------------------------------------------

Nous considérons l'équation aux dérivées partielles
\begin{equation}
    \frac{ \partial^2u }{ \partial t^2 }-c^2\frac{ \partial^2 u }{ \partial x^2 }=0
\end{equation}
où \( c\) est une constante réelle. Nous en cherchons des solutions de classe \( C^2\).

L'opérateur différentiel est donné par le polynôme \( P(T,X)=T^2-c^2X^2\) qui se factorise en
\begin{equation}
    P=(T+cX)(T-cX),
\end{equation}
c'est à dire que nous pouvons récrire l'équation des ondes sous la forme
\begin{equation}
    (\partial_t+c\partial_x)(\partial_t-c\partial_x)u=0.    
\end{equation}
Nous posons donc \( v=(\partial_t-c\partial_x)u\) et nous avons le système\cite{ooUQOJooSPNjlt}
\begin{subequations}
    \begin{numcases}{}
        (\partial_t+c\partial_x)v=0   \label{SUBEQooLQAPooVoJccp}\\
        (\partial_t-c\partial_x)u=v     \label{SUBEQooPWXMooDjThlJ}.
    \end{numcases}
\end{subequations}
La méthode des caractéristique est efficace pour résoudre la première, et pour trouver la solution générale de l'homogène associée à la seconde.

Nous nous lançons dans la résolution de \eqref{SUBEQooLQAPooVoJccp}. Le flot est \( v=\begin{pmatrix}
    1    \\ 
    c    
\end{pmatrix}\), et nous cherchons ses courbes intégrales sous la forme \( \varphi(t)=\big( t,x(t) \big)\). Immédiatement, \( x'(t)=c\), ce qui donne
\begin{equation}
    \gamma_C(t)=\begin{pmatrix}
        t    \\ 
        ct+C    
    \end{pmatrix}.
\end{equation}
Cela donne une caractéristique pour chaque valeur de \( C\). En posant \( \tilde v_C(t)=v(t,ct+C)\) nous avons
\begin{equation}
    \tilde v'_C(t)=\frac{ \partial v }{ \partial t }\big( \gamma_C(t) \big)+c\frac{ \partial v }{ \partial t }\big( \gamma_C(t) \big)=0.
\end{equation}
Donc \( \tilde v_C\) est une fonction constante. Donc \( u\) est constant sur la courbe \( \gamma_C\) dont l'équation cartésienne est \( x-ct=C\). Cela implique que
\begin{equation}
    v(t,x)=f(x-ct)
\end{equation}
où \( f\) est une fonction de classe \( C^1\). En effet si \( (t_1,x_1)\) et \( (t_2,x_2)\) vérifient \( x_1-ct_1=x_2-ct_2\) alors \( v(t_1,x_2)=v(t_2,x_2)\). Le fait que \( f\) soit \( C^1\) est une demande que \( u\) soit au final dans \( C^2\).

Nous devons maintenant résoudre l'équation \eqref{SUBEQooPWXMooDjThlJ}
\begin{equation}
    (\partial_t-c\partial_x)u=v.
\end{equation}

Nous allons agir conformément à la stratégie expliquée par le lemme \ref{LEMooEWUPooXNJMcc}. Nous devons résoudre \( Du=v\) avec
\begin{equation}
    \begin{aligned}
        D\colon C^2(\eR)&\to D(C^2(\eR)) \\
        u&\mapsto (\partial_t-c\partial_x)u. 
    \end{aligned}
\end{equation}
Par la même méthode des caractéristiques que celle déjà menée plus haut nous trouvons \( \ker(D)\) comme solution générale de \( (\partial_t-c\partial_x)u_G=0\). C'est à dire
\begin{equation}
    u_G=g(x+ct)
\end{equation}
où \( g\) est une fonction quelconque de classe \( C^2\).

Il nous faut maintenant une solution particulière de 
\begin{equation}
    (\partial_t-c\partial_x)u_P(t,x)=f(x-ct).
\end{equation}
Si \( F\) est une primitive de \( f\) alors 
\begin{equation}
    u_P(t,x)=-\frac{ 1 }{ 2c }F(x-ct)
\end{equation}
fonctionne. Vu que \( f\) est quelconque dans \( C^1(\eR)\), la fonction \( F\) est un élément quelconque de \( C^2(\eR)\). Au final, la solution générale de l'équation des ondes est
\begin{equation}
    u(t,x)=g_1(x+ct)+g_2(x-ct)
\end{equation}
où \( g_1\) et \( g_2\) sont des éléments de \( C^2(\eR)\).

