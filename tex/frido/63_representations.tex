% This is part of Mes notes de mathématique
% Copyright (c) 2011-2013,2015,2017
%   Laurent Claessens
% See the file fdl-1.3.txt for copying conditions.

%+++++++++++++++++++++++++++++++++++++++++++++++++++++++++++++++++++++++++++++++++++++++++++++++++++++++++++++++++++++++++++
\section{Représentations et caractères}
%+++++++++++++++++++++++++++++++++++++++++++++++++++++++++++++++++++++++++++++++++++++++++++++++++++++++++++++++++++++++++++

\begin{definition}
    Une représentation est \defe{fidèle}{représentation!fidèle} si elle est injective en tant que application \( G\to \GL(V)\). Ce ne sont pas chacun des \( \rho(g)\) qui doivent être injectifs. La dimension de \( V\) est le \defe{degré}{degré!d'une représentation} de la représentation \( (V,\rho)\).
\end{definition}

\begin{definition}
    Si \( G\) est un groupe, l'ensemble des homomorphismes \( \Hom(G,\eC^*)\) est un groupe pour la multiplication. Un élément de \( \Hom(G,\eC^*)\) est un \defe{caractère abélien}{caractère!abélien}. Le nom «abélien» vient du fait que le caractère prenne ses valeurs dans \( \eC^*\). Nous notons \( \hat G=\Hom(G,\eC^*)\)\nomenclature[R]{\( \hat G\)}{groupe des caractères de \( G\)}.
\end{definition}

\begin{theorem}
    Soit \( G\) un groupe abélien fini. Alors \( G\) est isomorphe à \( \hat G\).
\end{theorem}
L'isomorphisme n'est pas canonique.

\begin{proof}
    Étant donné la structure des groupes abéliens finis donnée par le théorème \ref{ThoRJWVJd}, nous commençons par nous concentrer sur \( G=\eZ/n\eZ\). Nous allons montrer que
    \begin{equation}
        \Hom(\eZ/n\eZ)\simeq \gU_n=\{ \xi\in \eC\tq \xi^n=1 \}.
    \end{equation}
    Pour cela nous avons l'isomorphisme
    \begin{equation}
        \begin{aligned}
            \psi\colon \Hom(\eZ,\eC^*)&\to \eC^* \\
            f&\mapsto f(1). 
        \end{aligned}
    \end{equation}
    Notons que si \( f\in \Hom(\eZ,\eC^*)\), alors \( f(k)=f(1)^k\), donc \( \psi\) est bien un isomorphisme. Cela nous amène à définir
    \begin{equation}
        \begin{aligned}
            \varphi\colon \Hom\Big( (\eZ/n\eZ,+),(\eC^*n\cdot) \Big)&\to \gU_n \\
            g&\mapsto f(1). 
        \end{aligned}
    \end{equation}
    Remarquons que pour tout \( f\in\Hom(\eZ/n\eZ,\eC^*)\) on a bien \( f(1)^n=1\). En effet si \( [k]\in \eZ/n\eZ\), alors \( f\big( [k] \big)=f(1)^k\) et en particulier
    \begin{equation}
        f(1)^n=f([n])=f(0)=1.
    \end{equation}
    Donc \( f(1)\in \gU_n\). Le \( \varphi\) est injective parce que si \( f(1)=g(1)\) alors \( f=g\) du fait que \( f(k)=f(1)^k=g(1)^k=g(k)\).

    Nous en sommes à avoir prouvé que \( \hat{\eZ/n\eZ}\simeq \gU_n\). Il faudrait encore montrer que \( \gU_n\simeq \eZ/n\eZ\). Pour cela nous nous rappelons du lemme \ref{LemWHQGooXyeJiw} nous ayant raconté que le groupe \( \gU_n\) des racines de l'unité était cyclique et d'ordre \( n\). Il est donc bien isomorphe à \( \eZ/n\eZ\).

    Passons au cas où
    \begin{equation}
        G\simeq \eZ/d_1\eZ\times \eZ/d_2\eZ\times \ldots\times \eZ/n_k\eZ.
    \end{equation}
    Dans ce cas nous montrons que 
    \begin{equation}
        \begin{aligned}
            \alpha\colon \bigtimes_{i=1}^k\Hom(\eZ/d_i\eZ,\eC^*)&\to \Hom(G,\eC^*) \\
            \alpha(\chi_1,\ldots, \chi_k)(g_1,\ldots, g_k)&= \chi_1(g_1)\ldots\chi_k(g_k). 
        \end{aligned}
    \end{equation}
    Ce \( \alpha\) est injectif parce qu'en appliquant l'égalité
    \begin{equation}
        \alpha(\chi_1,\ldots, \chi_k)=\alpha(\chi'_1,\ldots, \chi'_k)
    \end{equation}
    à l'élément \( g=(9,\ldots, 1,\ldots, 0)\) alors nous trouvons \( \chi_i(1)=\chi_i'(1)\) parce que \( \chi_j(0)=1\). Du coup \( \chi_i=\chi'_i\).

    L'application \( \alpha\) est en plus surjective. En effet si \( \chi\in\Hom(G,\eC^*)\), alors nous définissons
    \begin{equation}
        \chi_i(g_i)=\chi(0,\ldots, g_i,\ldots, 0),
    \end{equation}
    et nous avons alors \( \alpha(\chi_1,\ldots, \chi_k)=\chi\).

    Nous devons encore montrer que \( \alpha\) est un homomorphisme. Si \( \chi,\chi'\in\bigtimes_{i=1}^k\Hom(\eF_{d_i},\eC^*)\), alors
    \begin{subequations}
        \begin{align}
            \alpha(\chi\chi')(g_1,\ldots, g_k)&=(\chi_1\chi'_1)(g_1)\ldots (\chi_k\chi_k')(g_k)\\
            &=\chi_1(g_1)\ldots \chi_k(g_k)\chi'_1(g_1)\ldots \chi_k'(g_k)\\
            &=\alpha(\chi)(g_1,\ldots, g_k)\alpha(\chi')(g_1,\ldots, g_k)\\
            &=\big( \alpha(\chi)\alpha(\chi') \big)(g_1,\ldots, g_k).
        \end{align}
    \end{subequations}
    Donc \( \alpha(\chi\chi')=\alpha(\chi)\alpha(\chi')\).
\end{proof}

\begin{theorem}
    Soit \( G\) un groupe abélien fini. Les groupes \( G\) et \( \hat{\hat G}\) sont isomorphes et un isomorphisme canonique est donné par \( \alpha\colon g\mapsto f_g\) donné par
    \begin{equation}
        f_g(\chi)=\chi(g).
    \end{equation}
\end{theorem}

\begin{proof}

    D'abord \( f_g\) est bien un caractère de \( \hat G\) parce que
    \begin{equation}
        f_g(\chi\chi')=(\chi\chi')(g)=\chi(g)\chi'(g)=f_g(\chi)f_g(\chi').
    \end{equation}
    Le fait que \( \alpha\) soit un homomorphisme de groupes est direct :
    \begin{equation}
        f_{gg'}(\chi)=\chi(gg')=\chi(g)\chi(g')=f_g(\chi)f_{g'}(\chi)=(f_gf_{g'})(\chi).
    \end{equation}

    D'autre part nous savon que \( G\) et \( \hat{\hat G}\) ont le même cardinal. Il suffit donc de prouver l'injectivité de \( \alpha\) pour être sûr de la bijectivité. Pour cela nous devons prouver que si \( g\neq e\) alors \( f_g\neq f_e\). Nous savons que pour tout caractère \( \chi\in \hat G\), \( f_e(\chi)=\chi(e)=1\). Donc pour tout \( g\in G\setminus\{ e \}\), nous devons trouver \( \chi\in \hat G\) tel que \( \chi(g)\neq 1\).

    En vertu de ce que nous connaissons sur la structure des groupes abéliens finis (théorème \ref{ThoRJWVJd}), nous commençons \( G=\eZ/n\eZ\) et considérons le caractère donné par \( \chi([1])= e^{2i\pi/n}\). Ce \( \chi\) est un isomorphisme entre \( G\) et \( \gU(n)\); nous n'avons \( \chi([k])=0\) que si \( [k]=[n]=[0]\). Pour rappel dans \( \eZ/n\eZ\), le neutre est \( e=0\) et non \( e=1\).

    Passons au cas général :
    \begin{equation}
        G\simeq \eZ/n_1\eZ\times \ldots\times \eZ/n_k\eZ
    \end{equation}
    Si \( g=(g_1,\ldots, g_k)\) est non nul dans \( G\), alors il existe \( i\) tel que \( g_i\neq 0\) et on prend
    \begin{equation}
        \chi(g_1,\ldots, g_k)=\chi_i(g_i)
    \end{equation}
    où \( \chi_i\) est le caractère \( \chi_i([1])= e^{2\pi i/n_i}\). Ce \( \chi\) est alors un caractère non trivial de \( G\).
    
\end{proof}

%---------------------------------------------------------------------------------------------------------------------------
\subsection{Crochet de dualité et transformée de Fourier}
%---------------------------------------------------------------------------------------------------------------------------

Si \( G\) est un groupe abélien, nous définissons le crochet de dualité entre \( G\) et \( \hat G\) par
\begin{equation}
    \begin{aligned}
        \langle ., .\rangle \colon G\times \hat G&\to \eC^* \\
        \langle g, \chi\rangle &=\chi(g). 
    \end{aligned}
\end{equation}
Notons que l'image de ce crochet n'est pas \( \eC^*\) entier, mais seulement le groupe unitaire \( \gU(n)\) où \( n\) est l'exposant\footnote{Définition \ref{DefvtSAyb}.} de \( G\).


Si \( f,g\) sont des applications de \( G\) dans \( \eC\), alors on leur associe le produit scalaire
\begin{equation}
    \langle f, g\rangle =\frac{1}{ | G | }\sum_{s\in G}f(s)\overline{ g(s) }.
\end{equation}

\begin{lemma}
    Les caractères de \( G\) forment une base orthonormée de \( \eC^G\) pour ce produit scalaire.    
\end{lemma}

\begin{proof}
    Étant donné que les \( \chi(s)\) sont des nombres complexe de module \( 1\), nous avons \( \chi(s)\overline{ \chi(s) }=1\) et par conséquent \( \langle \chi, \chi\rangle =1\).

    Si par contre \( \chi\neq\chi'\), alors il existe \( s_\in G\) tel que \( \chi(s_0)\neq \chi'(s_0)\). Dans ce cas en effectuant un changement de variable \( s\to s_0s\) dans la sommation,
    \begin{subequations}
        \begin{align}
            \langle \chi, \chi'\rangle &=\frac{1}{ | G | }\sum_{s\in G}\chi(s)\overline{ \chi'(s) }\\
            &=\frac{1}{ | G | }\sum_{s\in G}\chi(s_0s)\overline{ \chi'(s_0s) }\\
            &=\frac{1}{ | G | }\chi(s_0)\chi'(s_0)\sum_{s\in G}\chi(s)\overline{ \chi'(s) }.
        \end{align}
    \end{subequations}
    Donc nous avons trouvé
    \begin{equation}
        \langle \chi, \chi'\rangle \big( 1-\chi(s_0)\overline{ \chi'(s_0) } \big)=0.
    \end{equation}
    Mais vu que \( \chi(s_0)\neq \chi(s'_0)\), la parenthèse est non nulle (pour rappel \( \chi(s_0)\) est un complexe de module \( 1\)) et par conséquent \( \langle \chi, \chi'\rangle =0\).

    Nous déduisons immédiatement que les caractères forment une famille libre parce que si \( \sum_i\chi_i=0\) (la somme est sur tous les caractères), alors en prenant le produit scalaire avec \( \chi_k\),
    \begin{equation}
        \sum_ia_i\langle \chi_k, \chi_i\rangle =0,
    \end{equation}
    et donc \( a_k=0\).

    Les caractères forment donc un système libre orthonormé. De plus l'espace engendré à la bonne dimension parce que le cardinal de l'ensemble des caractères est la dimension (complexe) de l'espace des fonctions de \( G\) dans \( \eC\) parce que, en utilisant l'isomorphisme entre \( G\) et \( \hat G\),
    \begin{equation}
        \Card\hat G=\Card(G)=\dim_{\eC}\eC^G.
    \end{equation}
    La première 
\end{proof}

Du fait que les caractères forment une base orthonormée, nous pouvons écrire, pour toute application \( f\colon G\to \eC\),
\begin{equation}    \label{EqnnsXWC}
    f=\sum_{\chi\in\hat G}\langle \chi, f\rangle \chi.
\end{equation}
À une fonction \( f\colon G\to \eC\) nous associons la \defe{transformée de Fourier}{transformée!de Fourier!groupe abélien fini}\index{Fourier!transformée!groupe abélien fini}
\begin{equation}
    \begin{aligned}
        \hat f\colon \hat G&\to \eC \\
        \chi&\mapsto \langle \chi, f\rangle . 
    \end{aligned}
\end{equation}
Nous avons donc aussi une espèce de formule d'inversion
\begin{equation}
    f=\sum_{\chi\in\hat G}\hat f(\chi)\chi
\end{equation}
qui n'est qu'une réécriture de \ref{EqnnsXWC}.

%---------------------------------------------------------------------------------------------------------------------------
\subsection{Groupes non abéliens}
%---------------------------------------------------------------------------------------------------------------------------

Nous avons vu que le groupe des caractères \( \hat G\) contenait toute l'information sur un groupe abélien. Malheureusement, pour les groupes non abéliens, ça ne va pas suffire, et nous allons introduire la notion de représentations, dont les caractères seront un cas particulier de dimension un.

\begin{proposition}
    Soit \( G\) un groupe (pas spécialement abélien). Nous avons
    \begin{equation}
        \hat G\simeq\Hom\big( G/D(G),\eC^* \big).
    \end{equation}
\end{proposition}

\begin{proof}
    Ce qui fait fonctionner la preuve est le fait que si \( f\colon G\to \eC^*\) est un homomorphisme, alors \( f\) s'annule sur \( D(G)\). L'isomorphisme est
    \begin{equation}
        \begin{aligned}
            \psi\colon \hat G&\to \Hom\big( G/D(G),\eC^* \big) \\
            \psi(f)[g]&=f(g). 
        \end{aligned}
    \end{equation}
    Cette application est bien définie parce que si \( f\) est un homomorphisme,
    \begin{equation}
        f(gklk^{-1}l^{-1})=f(g).
    \end{equation}
    D'autre part \( \psi\) est un homomorphisme de groupe parce que
    \begin{equation}
        \psi(f_1f_2)[g]=(f_1f_2)(g)=f_1(g)f_2(g)=\psi(f_1)[g]\psi(f_2)[g]=\big( \psi(f_1)\psi(f_2) \big)[g].
    \end{equation}
    Pour l'injectivité de \( \psi\), soit \( f_1\) et \( f_2\) telles que \( \psi(f_1)=\psi(f_2)\). Alors pour tout \( g\in G\) nous avons
    \begin{equation}
        \psi(f_1)[g]=\psi(f_2)[g]
    \end{equation}
    et donc \( f_1(g)=f_2(g)\).

    Enfin \( \psi\) est surjective. En effet, soit \( \bar f\in\Hom\big( G/D(G),\eC^* \big)\). Alors nous obtenons \( \psi(f)=\bar f\) en posant
    \begin{equation}
        f(g)=\bar f[g].
    \end{equation}
    Il faut juste vérifier que le \( f\) ainsi défini est dans \( \hat G\), c'est à dire que \( f(g_1g_2)=f(g_1)f(g_2)\).
\end{proof}

Cette proposition nous montre que
\begin{equation}
    \hat G=\widehat{G/D(G)},
\end{equation}
alors que \( G/D(G)\) est abélien; il n'est donc pas tellement possible que \( \hat G\) contienne beaucoup d'informations intéressantes sur \( G\).

%---------------------------------------------------------------------------------------------------------------------------
\subsection{Représentations linéaires des groupes finis}
%---------------------------------------------------------------------------------------------------------------------------

Si \( \dim V=1\), alors \( \GL(V)=\eC^*\) et les représentation sont les caractères abéliens.

\begin{example} \label{ExKUAyUD}
    Considérons le triangle équilatéral \( A,B,C\), par exemple donné par les points
    \begin{subequations}
        \begin{numcases}{}
            A=1\\
            B=(-\frac{ 1 }{2},\frac{ \sqrt{3} }{2})\\
            C=(-\frac{ 1 }{2},-\frac{ \sqrt{3} }{2})\\
        \end{numcases}
    \end{subequations}
    Dans la base (pas orthonormée) \( \{ A,B \}\) de \( \eR^2\), ces trois points sont donnés par
    \begin{equation}
        \begin{aligned}[]
            A&=\begin{pmatrix}
                1    \\ 
                0    
            \end{pmatrix}&B&=\begin{pmatrix}
                0    \\ 
                1    
            \end{pmatrix}&C&=\begin{pmatrix}
                -1    \\ 
                -1    
            \end{pmatrix}.
        \end{aligned}
    \end{equation}
    Le groupe symétrique\index{groupe!symétrique!action sur un triangle} \( S_3\) agit sur le triangle par permutation des sommets. Vues dans la base \( \{ A,B \}\), les transpositions correspondent aux matrices
    \begin{subequations}
        \begin{align}
            (A,B)&\to\begin{pmatrix}
                0    &   1    \\ 
                1    &   0    
            \end{pmatrix}\\
            (A,C)&\to \begin{pmatrix}
                -1    &   0    \\ 
                -1    &   1    
            \end{pmatrix}\\
            (B,C)&\to\begin{pmatrix}
                1    &   -1    \\ 
                0    &   -1    
            \end{pmatrix}.
        \end{align}
    \end{subequations}
    La permutation \( (A,B,C)\) s'écrit comme \( (A,B,C)=(A,C)(A,B)\) et on lui associe la matrice
    \begin{equation}
        (A,B,C)\to\begin{pmatrix}
            0    &   -1    \\ 
            1    &   -1    
        \end{pmatrix}.
    \end{equation}
    C'est bien le produit des matrices de \( (A,C)\) et de \( (A,B)\). De la même façon nous avons
    \begin{equation}
        (BAC)<++>
    \end{equation}
    % TODO : il y a manifestement quelque chose à terminer ici.
    <++>
\end{example}

Si \( (V,\rho)\) et \( (V',\rho')\) sont deux représentations du groupe \( G\), alors nous définissons la \defe{somme directe}{somme directe (de représentations)} par \( \big( V\oplus V',\rho\oplus\rho' \big)\) donné par
\begin{equation}
    (\rho\oplus\rho')(g)=\begin{pmatrix}
        \rho(g)    &   0    \\ 
        0    &   \rho'(g)    
    \end{pmatrix}\in \GL(V\oplus V').
\end{equation}

Nous noterons souvent \( 2V\) pour la représentations \( (V,\rho)\oplus (V,\rho)\) et plus généralement l'écriture
\begin{equation}
    V=\bigoplus_i k_iW_i
\end{equation}
signifiera la représentation somme de \( k_i\) termes de la représentation \( W_i\). Ici encore un abus est commis entre la représentation \( (\rho_i,W_i)\) et l'espace \( W_i\).

%---------------------------------------------------------------------------------------------------------------------------
\subsection{Module}
%---------------------------------------------------------------------------------------------------------------------------

Nous considérons la \( \eC\)-algèbre \( G[\eC]\)\nomenclature[G]{\( \eC[G]\)}{combinaisons d'éléments de \( G\) à coefficients dans \( \eC\)} des combinaisons (formelles) d'éléments de \( G\) à coefficients dans \( G\), c'est à dire l'ensemble
\begin{equation}
    \eC[G]=\{ \sum_{s\in G}a_ss \}
\end{equation}
avec le produit hérité de la bilinéarité :
\begin{equation}
    \sum_{s\in G}\sum_{t\in G}a_sb_tst=\sum_s\sum_t a_sb_{s^{-1}t}t,
\end{equation}
et la somme
\begin{equation}
    (\sum_sa_ss)+\sum_tb_tt=\sum_{s\in G}(a_s+b_s)s.
\end{equation}
Le tout est une \( \eC\)-algèbre agissant sur \( V\) par
\begin{equation}
    \left( \sum_sa_ss \right)v=\sum_{s\in G}a_s\rho(s)v\in V
\end{equation}

Les sous-modules indécomposables seront les représentations irréductibles.

\begin{definition}
    La représentation \( (V,\rho)\) du groupe \( G\) est \defe{irréductible}{irréductible!représentation}\index{représentation!irréductible} si les seuls sous-espaces invariants de \( V\) sous \( \rho(G)\) sont $V$ et \( \{ 0 \}\).
\end{definition}

\begin{example}
    La représentation de \( S_3\) sur \( \eR^2\) donnée par les permutations des sommets d'un triangle équilatéral donnée dans l'exemple \ref{ExKUAyUD} est irréductible.
\end{example}

La question qui vient est de savoir si une représentation possédant des sous-espaces invariants peut être écrite comme la somme de représentations irréductibles.

\begin{proposition} \label{PropHeyoAN}  \index{représentation!irréductible}
    Soit \( (V,\rho)\) une représentation linéaire de dimension finie d'un groupe fini\footnote{La démonstration marche aussi pour les groupes compacts, mais il faudrait des intégrales.}. Si \( W_1\) est un sous-espace stable\footnote{c'est à dire si \( \rho\) n'est pas irréductible.}, alors il existe un sous-espace \( W_2\) également stable et tel que \( V=W_1\oplus W_2\).

    Toute représentation linéaire est décomposable en représentations irréductibles.
\end{proposition}

\begin{proof}
    Soit \( P\colon V\to V\) un projecteur sur \( W_1\), c'est à dire que \( P^2=P\) et \( P(V)=W_1\). Pour construire un tel projecteur, on peut par exemple prendre un supplémentaire de \( W_1\) dans \( V\) puis utiliser la décomposition\footnote{Ou encore prendre une base de \( W_1\), l'étendre en une base de \( V\) et définir \( P\) comme l'annulation des coefficients des vecteurs «complétant» la base.}. Nous considérons l'opérateur
    \begin{equation}
        P_G=\frac{1}{ | G | }\sum_{g\in G}\rho(g)\circ P\circ \rho(g)^{-1}.
    \end{equation}
    Prouvons que ce \( P_G\) est encore un projecteur. D'abord pour tout \( g\in G\) nous avons
    \begin{equation}
        \rho(g)P_G\rho(g)^{-1}=\frac{1}{ | G | }\sum_{s\in G}\rho(gs)P\rho(gs)^{-1}=P_G.
    \end{equation}
    La dernière égalité est un changement de variables dans la somme\footnote{Et c'est ça qui demande un peu de technique pour écrire la preuve dans le cas d'un groupe compact : il faut une mesure de Haar.}. Cela signifie que \( P_G\rho=\rho P_G\). Nous avons même \( P_GP=P\) parce que si \( v\in W_1\), alors 
    \begin{subequations}
        \begin{align}
            P_G(v)&=\frac{1}{ | G | }\sum_{s\in G}\rho(s)P\underbrace{\rho(s)^{-1} v}_{\in W_1}\\
            &=\frac{1}{ | G | }\sum_s\rho(s)\rho(s)^{-1} v\\
            &=v.
        \end{align}
    \end{subequations}
    Avec cela nous pouvons conclure que \( P_G^2=P_G\) parce que
    \begin{subequations}
        \begin{align}
            P_G\circ P_G&=\frac{1}{ | G | }\sum_g P_G\rho(g)P\rho(g)^{-1}\\
            &=\frac{1}{ | G | }\sum_g \rho(g)P_GP\rho(g)^{-1}\\
            &=\frac{1}{ | G | }\sum_g \rho(g)P\rho(g)^{-1}\\
            &=P_G.
        \end{align}
    \end{subequations}
    Donc \( P_G\) est un projecteur, est stable sous les conjugaisons par \( \rho(g)\) et commute avec \( \rho(g)\). Nous décomposant \( \id\) de façon évidente en
    \begin{equation}
        \id=P_G+(\id-P_G).
    \end{equation}
    Étant donné que l'opérateur \( P_G\) commute avec tous les \( \rho(g)\), les noyaux de \( P_G\) et \( \id-P_G\) sont des sous-espaces invariants. Vu que \( P_G\) est un projecteur, nous avons \( q(P_G)=0\) avec \( q(X)=X^2-X\). Pour appliquer le lemme des noyaux (théorème \ref{ThoDecompNoyayzzMWod}), nous remarquons que \( q(X)=X(X-1)\) et donc 
    \begin{equation}
        V=\ker P_G\oplus\ker(P_G-\mtu).
    \end{equation}
    Si nous posons \( W_2=\ker P_G\), il reste à voir que \( \ker(P_G\mtu)=W_1\). D'abord \( W_1\subset\ker(P_G-\id)\) parce que si \( w\in W_1\), ce dernier étant stable,
    \begin{subequations}
        \begin{align}
            P_Gw&=\frac{1}{ | G | }\sum_{g\in G}\rho(g)P\underbrace{\rho(g)^{-1}w}_{\in W_1}\\
            &=\frac{1}{ | G | }\sum_{g\in G}w\\
            &=w.
        \end{align}
    \end{subequations}
    Pour prouver l'inclusion inverse, nous savons que \( P_G\) et \( P\) sont des projecteurs tels que \( P_GP=P\), ce qui signifie que l'image de \( P_G\) est inclue à celle de \( P\), c'est à dire à \( W_1\). Mais \( \Image(P_G)=\ker(\mtu-P_G)\), donc
    \begin{equation}
        \ker(\mtu-P_G)=\Image(P_G)\subset\Image(P)=W_1.
    \end{equation}

    La représentation \( \rho\) se décompose donc en deux sous-représentations \( (\rho,W_1)\) et \( \rho,W_2\). Si l'une des deux n'est pas irréductible, le processus peut recommencer. Vu que la dimension de \( V\) est finie, toute représentation se décompose en une somme finie de représentation irréductibles.
\end{proof}

%---------------------------------------------------------------------------------------------------------------------------
\subsection{Structure hermitienne}
%---------------------------------------------------------------------------------------------------------------------------

Soit \( (\rho,V)\) une représentation de \( G\) sur un espace vectoriel complexe \( V\). Nous voulons munir \( V\) d'un produit scalaire hermitien (définition \ref{DefMZQxmQ}) tel que les opérateurs \( \rho(g)\) soient tous des isométries. C'est à dire que nous voudrions définir \( \langle u, v\rangle_G\) de telle sorte à avoir
\begin{equation}
    \langle \rho(g)u, \rho(g)v\rangle_G =\langle u, v\rangle_G
\end{equation}
pour tout \( g\in G\). Nous commençons par considérer un produit hermitien \( \langle ., .\rangle \) quelconque et puis nous définissons
\begin{equation}
    \langle u, v\rangle_G=\frac{1}{ | G | }\sum_{g\in G}\langle \rho(g)u, \rho(g)v\rangle.
\end{equation}
Nous devons vérifier que c'est un produit. La seule des conditions dont la vérification n'est pas immédiate est celle de positivité. Pour tout \( g\in G\) et tout \( v\in V\), nous avons \( \langle \rho(g)v, \rho(g)v\rangle \) est positif et nul si et seulement si \( \rho(g)v=0\). Étant donné que \( \rho(e)v=v\), parmi les termes de la somme
\begin{equation}
    \langle u, u\rangle_G=\frac{1}{ | G | }\sum_{g\in G}\langle \rho(g)v, \rho(g)v\rangle,
\end{equation}
au moins un est strictement positif (pourvu que \( v\neq 0\)); les autres sont positifs ou nuls. Par conséquent \( \langle v, v\rangle_G=0\) si et seulement si \( v=0\).

Donc les groupes finis peuvent être vus comme des parties de groupes d'isométrie. De la même façon, en utilisant une mesure de Haar pour faire la moyenne, nous pouvons plonger les groupes compacts dans des groupes unitaires.

%---------------------------------------------------------------------------------------------------------------------------
\subsection{Caractères}
%---------------------------------------------------------------------------------------------------------------------------

\begin{definition}
    Soit \( (V,\rho)\) une représentation linéaire du groupe \( G\). Le \defe{caractère}{caractère} de \( \rho\) est la fonction
    \begin{equation}
        \begin{aligned}
            \chi_{\rho}\colon G&\to \eC \\
            s&\mapsto \tr\big( \rho(s) \big).
        \end{aligned}
    \end{equation}
\end{definition}
Par invariance cyclique de la trace, nous avons
\begin{equation}
    \chi_{\rho}(sts^{-1})=\chi_{\rho}(t),
\end{equation}
ce qui fait que le caractère est une fonction constante sur les classes de conjugaison.

\tiny
D'après sa fiche wikipédia, le marquis de Sade, passionné de théâtre, faisait des représentations qui avaient du caractère.
\normalsize


Un \defe{caractère irréductible}{caractère!irréductible} est un caractère d'une représentation irréductible.

\begin{definition}
    Une application \( f\colon G\to \eC\) est \defe{centrale}{centrale (application)} si elle est constante sur les classes de conjugaison.
\end{definition}
Les traces sont des applications centrales.

L'ensemble des fonctions centrales sur un groupe fini (ou tout au moins ayant un nombre fini de classes de conjugaison) est un \( \eC\)-espace vectoriel de dimension égale au nombre de classes, et nous pouvons mettre le produit scalaire
\begin{equation}    \label{EqJrEpVI}
    \langle f, g\rangle =\frac{1}{ | G | }\sum_{s\in G}f(s)\overline{ g(s) }.
\end{equation}
C'est une forme hermitienne sur l'espace des fonctions centrales.

%+++++++++++++++++++++++++++++++++++++++++++++++++++++++++++++++++++++++++++++++++++++++++++++++++++++++++++++++++++++++++++
\section{Équivalence de représentations et caractères}
%+++++++++++++++++++++++++++++++++++++++++++++++++++++++++++++++++++++++++++++++++++++++++++++++++++++++++++++++++++++++++++

Cette section prend des éléments des articles \wikipedia{fr}{Lemme_de_Schur}{lemme de Schur}, \wikipedia{fr}{Caractère_d'une_représentation_d'un_groupe_fini}{caractère d'une représentation}, \wikipedia{fr}{Fonction_centrale_d'un_groupe_fini}{fonction centrale} et \wikipedia{fr}{Trace_(algèbre)}{trace} de wikipédia.

Nous disons que les deux représentations \( (V,\rho)\) et \( (V',\rho')\) sont \defe{équivalentes}{equivalence@équivalence!de représentations} s'il existe une bijection linéaire \( f\colon V\to V'\) telle que
\begin{equation}
    f\circ \rho=\rho'\circ f.
\end{equation}
Nous disons alors que \( f\) \defe{entrelace}{entrelacement} \( \rho\) et \( \rho'\).

\begin{theorem}[Théorème de Schur]\index{théorème!Schur}\index{Schur (théorème)}    \label{ThoyftobH}
    Si \( (V,\rho)\) et \( (V',\rho')\) sont des représentations irréductibles non équivalentes alors la seule application linéaire \( f\colon V\to V'\) entrelaçant \( \rho\) et \( \rho'\) est la fonction nulle.

    En d'autres termes, soit les représentations sont équivalentes (et il y a un isomorphisme), soit il n'y a même pas un homomorphisme.
\end{theorem}

\begin{proof}
    Soit \( f\in\aL(V,V')\) telle que \( f\circ \rho=\rho'\circ f\). Alors \( \ker f\) est un sous-espace stable sous \( \rho(G)\), et \( \Image(f)\) est un sous-espace de \( V'\) stable par \( \rho'(G)\). Par irréductibilité, nous avons que \( \ker(f)=\{ 0 \}\) ou \( V\). Même chose pour \( \Image(f)\). Il y a deux possibilités.
    \begin{enumerate}
        \item
            Si \( \ker(f)=\{ 0 \}\), alors \( \Image(f)\neq \{ 0 \}\) et alors \( \Image(f)=V'\). Du coup \( f\) est injective et surjective, c'est à dire est un isomorphisme.
        \item
            Si \( \ker(f)=V\), alors \( f=0\).
    \end{enumerate}
\end{proof}

\begin{corollary}[Schur pour les représentations sur \( \eC\)]
    Soit \( (V,\rho)\) une représentation irréductible, alors l'ensemble
    \begin{equation}
        \End_{G}(V,\rho)=\{ f\in\End(V)\tq \rho\circ f=f\circ \rho \}
    \end{equation}
    est l'ensemble des homothéties.
\end{corollary}

\begin{proof}
    Soit \( f\in\End_{G}(V,\rho)\). Vu que l'espace est sur \( \eC\), l'endomorphisme \( f\) a une valeur propre \( \lambda\). L'opérateur \( g=f-\lambda\mtu\) est aussi un opérateur d'entrelacement de \( \rho\) alors que \( \ker(g)\neq \{ 0 \}\) par définition de valeur propre. Du coup \( \ker(g)=V\), ce qui signifie que \( f\) est l'isométrie de rapport \( \lambda\) : \( f=\lambda\id\).
\end{proof}

\begin{lemma}   \label{LempUSOlo}
    Si \( (\rho,V)\) et \( (\rho',V')\) sont des représentations équivalentes de caractères \( \chi\) et \( \chi'\), alors \( \chi=\chi'\).
\end{lemma}

\begin{proof}
    Si \( A\colon V\to V'\) est un isomorphisme d'espace vectoriel entrelaçant \( \rho\) et \( \rho'\), c'est à dire si pour tout \( g\), \( \rho'(g)A=A\rho(g)\), alors \( \rho'(g)=A\rho(g)A^{-1}\) et
    \begin{equation}
        \chi'(g)=\tr\big( \rho'(g) \big)=\tr\big( A\rho(g)A^{-1} \big)=\tr\big( \rho(g) \big)
    \end{equation}
    parce que la trace est un invariant de similitude (lemme \ref{LemhbZTay}).
\end{proof}

\begin{lemma}   \label{LemJqIZns}
    Si \( \chi\) est le caractère de la représentation complexe \( (V,\rho)\) du groupe fini \( G\), alors pour tout \( g\in G\) nous avons \( \chi(g^{-1})=\overline{ \chi(g) }\).
\end{lemma}

\begin{proof}
    Par le corollaire \ref{CorpZItFX} au théorème de Lagrange, nous avons \( g^{| G |}=e\) et donc en tant qu'opérateur, \( \rho(g)^{| G |}=\mtu\). Les valeurs propres de \( \rho(g)\) sont donc des racines de l'unité. Si nous notons \( \lambda_i\) ces valeurs propres, alors \( \chi(g)=\sum_i\lambda_i\), et en considérant la matrice dans sa base de diagonalisation (lemme de Schur complexe, \ref{LemSchurComplHAftTq}), nous voyons que
    \begin{equation}
        \chi(g^{-1})=\tr\big( \rho(g)^{-1} \big)=\sum_i\frac{1}{ \lambda_i }.
    \end{equation}
    Mais \( \lambda_i\) étant une racine de l'unité nous avons \( \frac{1}{ \lambda_i }=\bar\lambda_i\), ce qui fait que
    \begin{equation}
        \chi(g^{-1})=\sum_i\bar\lambda_i=\overline{ \chi(g) }.
    \end{equation}
\end{proof}

\begin{proposition} \label{PropJzbfWi}
    Soient deux représentations irréductibles complexes \( (V,\rho)\) et \( (V',\rho')\) du même groupe fini \( G\), et \( \chi\) et \( \chi'\) leurs caractères respectifs. Nous avons
    \begin{enumerate}
        \item
            \( \langle \chi, \chi'\rangle =0\) si \( \rho\) et \( \rho'\) ne sont pas équivalentes.
        \item
            \( \langle \chi, \chi'\rangle =1\) si les représentations sont équivalentes.
    \end{enumerate}
\end{proposition}

\begin{proof}
    Nous considérons les bases \( \{ e_1,\ldots, e_n \} \) de \( V\) et \( \{ f_1,\ldots, f_m \}\) de \( V'\). Puis nous considérons la matrice $F(k,l)=E_{kl}\in\eM_{m,n}(\eC)$ où pour rappel, \( E_{kl}\) est la matrice de composantes \( (E_{kl})_{ij}=\delta_{ki}\delta_{lj}\). Nous posons
    \begin{equation}
        F_G(k,l)=\frac{1}{ | G | }\sum_{g\in G}\rho(g)\circ F(k,l)\circ\rho'(g)^{-1}.
    \end{equation}
    En nous permettant de ne pas réécrire les indices \( k\) et \(l \) de \( F\) et \( F_G\), nous montrons que \( F_G\) entrelace \( \rho\) et \( \rho'\) :
    \begin{subequations}
        \begin{align}
            F_G\circ\rho'(t)&=\frac{1}{ | G | }\sum_{s\in G}\rho(s)\circ F\circ \rho'(s^{-1})\circ \rho(t)\\
            &=\frac{1}{ | G | }\sum_s\rho(s)F\rho'(s^{-1} t)\\
            &=\frac{1}{ | G | }\sum_{k}\rho(tk)F\rho'(k^{-1})\\
            &=\frac{1}{ | G | }\rho(t) \sum_k\rho(k)F\rho'(k^{-1})\\
            &=\rho(t)\circ F_G.
        \end{align}
    \end{subequations}
    Dans ce calcul nous avons effectué le changement de variables \( k=(s^{-1} t)^{-1}\) qui donne \( s=tk\).

    Par ailleurs nous avons
    \begin{subequations}
        \begin{align}
            \Big( \rho(g)F(k,l)\rho'(g^{-1}) \Big)_{ij}&=\sum_{r=1}^n\sum_{s=1}^m\rho(g)_{ir}F(k,l)_{rs}\rho'(g^{-1})_{sj}\\
            &=\sum_{rs}\rho(g)_{ir}\delta_{kr}\delta_{ls}\rho'(g^{-1})_{sj}\\
            &=\rho(g)_{ik}\rho'(g^{-1})_{lj},
        \end{align}
    \end{subequations}
    et par conséquent
    \begin{equation}    \label{Eqgvpzfz}
        F_G(k,l)_{ij}=\frac{1}{ | G | }\sum_{g\in G}\rho(g)_{ik}\rho'(g^{-1})_{lj}.
    \end{equation}
    Si \( \chi\) et \( \chi'\) sont les caractères de \( \rho\) et \( \rho'\), alors nous avons le produit \eqref{EqJrEpVI} qui donne
    \begin{subequations}
        \begin{align}
            \langle \chi, \chi'\rangle &=\frac{1}{ | G | }\sum_{g\in G}\chi(g)\overline{ \chi'(g) }\\
            &=\frac{1}{ | G | }\sum_g\chi(g)\chi'(g^{-1})    &\text{lemme \ref{LemJqIZns}}\\
            &=\frac{1}{ | G | }\sum_g\sum_{i=1}^n\sum_{j=1}^m\rho(g)_{ii}\rho'(g^{-1})_{jj}     \label{sEqKYywTM}\\
            &=\sum_{ij}F_G(i,j)_{ij}    &\text{par \eqref{Eqgvpzfz}}.
        \end{align}
    \end{subequations}
    Si les représentations \( \rho\) et \( \rho'\) ne sont pas équivalentes, le fait que \( F_G\) en soit un opérateur d'entrelacement implique par le théorème de Schur \ref{ThoyftobH} que \( F_G=0\) et donc \( \langle \chi, \chi'\rangle =0\).

    Si au contraire les représentation sont équivalentes, alors le lemme \ref{LempUSOlo} nous dit que \( \chi=\chi'\) et nous reprenons la définition :
    \begin{equation}
        \langle \chi, \chi\rangle =\frac{1}{ | G | }\sum_g\chi(g)\overline{ \chi(g) }=\frac{1}{ | G | }\sum_{g\in G}1=1
    \end{equation}
    parce que les nombres \( \chi(g)\) sont des racines de l'unité.
\end{proof}


%--------------------------------------------------------------------------------------------------------------------------- 
\subsection{Représentation régulière}
%---------------------------------------------------------------------------------------------------------------------------

Nous notons \( \lambda\) la \defe{représentation régulière gauche}{représentation!régulière gauche}, agissant sur le \( \eK\)-espace vectoriel des fonctions \( G\to \eK\) par
\begin{equation}
    \Big( \lambda(g)f \Big)(g)=f(g^{-1}h).
\end{equation}
D'autre part nous considérons les fonctions \( \delta_g\colon G\to \eK\) (ici \( \eK\) est \( \eR\) ou \( \eC\) ou pire) définie par
\begin{equation}
    \delta_g(h)=\begin{cases}
        1    &   \text{si } g=h\\
        0    &    \text{sinon.}
    \end{cases}
\end{equation}
La représentation régulière agit sur les fonctions \( \delta_s\) de la façon suivante :
\begin{equation}
    \lambda(g)\delta_s=\delta_{gs}
\end{equation}
parce que \( \big( \lambda(g)\delta_s \big)(h)=\delta_s(g^{-1}h)=\delta_{gs}(h)\).

\begin{lemma}
    Le caractère de la représentation régulière gauche est donné par 
    \begin{equation}        \label{EqUuoVNa}
        \chi_{\lambda}=| G |\delta_e.
    \end{equation}
\end{lemma}

\begin{proof}
    Appliquer l'équation \eqref{EqUuoVNa} fonctionne parce que \( \chi_{\lambda}(e)\) est la dimension de l'espace des fonctions sur \( G\), c'est à dire \( | G |\). Si par contre \( g\neq e\), alors \( \lambda(g)\) est une matrice de permutation (dans la base des \( \delta_h\)) et a donc tous ses éléments diagonaux nuls.
\end{proof}

Si \( \rho\) est une représentation et si \( f\) est une fonction sur le groupe, alors nous considérons l'opérateur
\begin{equation}
    \rho_f=\sum_{g\in G}f(g)\rho(g).
\end{equation}

\begin{proposition}[\cite{NhCRhg}]  \label{PropEAXkAY}
    Si \( (\rho,V)\) est une représentation irréductible et si \( f\) est une fonction centrale sur \( G\), alors l'opérateur \( \rho_f\) est une homothétie de \( V\) de rapport
    \begin{equation}
        \frac{1}{ \dim V }\sum_{g\in G}f(g)\chi(g)
    \end{equation}
    où \( \chi\) est le caractère de \( \rho\).
\end{proposition}

\begin{proof}
    Nous commençons par voir que \( \rho_f\) entrelace \( \rho\). En effet,
    \begin{subequations}
        \begin{align}
            \rho(t)^{-1}\circ\rho_f\circ\rho(t)&=\sum_gf(g)\rho(t^{-1}gt)\\
            &=\sum_hf(tht^{-1})\rho(g)      & h=t^{-1}gt\\
            &=\sum_hf(h)\rho(h)\\
            &=\rho_f
        \end{align}
    \end{subequations}
    où en écrivant \( f(tht^{-1})=f(h)\), nous avons utilisé le fait que \( f\) était centrale. Étant donné que \( \rho_f\) entrelace une représentation irréductible, le lemme de Schur (\ref{ThoyftobH}) nous indique que \( \rho_f\) est une homothétie. Soit \( k\) le facteur d'homothétie. Alors d'une part \( \tr(\rho_f)=nk\). D'autre part,
    \begin{subequations}
        \begin{align}
            \tr(\rho_f)&=\tr\big( \sum_gf(g)\rho(g) \big)\\
            &=\sum_gf(g)\tr\big( \rho(g) \big)\\
            &=\sum_gf(g)\chi(g).
        \end{align}
    \end{subequations}
    Du coup effectivement 
    \begin{equation}
        k=\frac{1}{ n }\sum_{g\in G}f(g)\chi(g).
    \end{equation}
\end{proof}

%--------------------------------------------------------------------------------------------------------------------------- 
\subsection{Caractères et représentations : suite et fin}
%---------------------------------------------------------------------------------------------------------------------------

\begin{lemma}
    Un groupe fini n'a (à équivalence près) qu'un nombre fini de représentations irréductibles.
\end{lemma}

\begin{proof}
    Les caractères irréductibles forment un système orthonormé (proposition \ref{PropJzbfWi}) et donc libre parmi les fonctions centrales. Donc il y a au plus autant de caractères irréductibles que la dimension de l'espace des fonctions centrales; et ce dernier est de dimension finie donnée par le nombre de classes de conjugaison de \( G\).
\end{proof}

Nous savons que les caractères de deux représentations irréductibles sont égaux. Étant donné qu'il n'existe qu'un nombre fini de représentations irréductibles, il existe un nombre fini de caractères irréductibles. Nous pouvons donc fixer les notations suivantes. Les caractères irréductibles seront notés \( \{ \varphi_i \}_{i=1,\ldots, h}\) et nous noterons \( (\sigma_i,W_i)\) une représentation ayant le caractère \( \varphi_i\).

\begin{theorem}[\cite{NhCRhg}]
    Soit \( (\rho,V)\) une représentation de \( G\) de caractère \( \chi\). Alors sa décomposition en représentations irréductibles est donnée par
    \begin{equation}
        (V,\rho)=\bigoplus_{i=1}^hk_i(W_i,\sigma_i)
    \end{equation}
    avec \( k_i=\langle \chi, \varphi_i\rangle \). En particulier, à permutation près des facteurs, la décomposition d'une représentation en représentations irréductibles est unique.
\end{theorem}

\begin{proof}
    La décomposition de \( \chi\) en caractères irréductibles est donnée par \( \chi=\sum_ik_i\varphi_i\); en prenant le produit de cette égalité avec \( \varphi_j\) et en tenant compte de l'othonormalité des caractères irréductibles,
    \begin{equation}
        \langle \chi, \varphi_j\rangle =\sum_ik_i\langle \varphi_i, \varphi_j\rangle =k_j.
    \end{equation}
\end{proof}

Le théorème suivant est ce qui nous permet de dire que l'étude des caractères et l'étude des représentations, c'est la même chose.

\begin{theorem} \label{ThoWGkfADd}
    Soit \( G\) un groupe fini\footnote{Nous sommes depuis longtemps dans l'étude des représentations des groupes finis.}.
    \begin{enumerate}
        \item   \label{ItemZReOWoHi}
            Deux représentations sont équivalentes si et seulement si elles ont même caractères.
        \item   \label{ItemZReOWoHii}
            Si \( \chi\) est le caractère d'une représentation, alors 
            \begin{enumerate}
                \item
                    \( \langle \chi, \chi\rangle \in \eN\) 
                \item
                    \( \langle \chi, \chi\rangle =1\) si et seulement si la représentation est irréductible.
            \end{enumerate}
    \end{enumerate}
\end{theorem}

\begin{proof}
    Nous démontrons chaque point séparément.
    \begin{enumerate}
        \item
            
    Le fait que deux représentations équivalentes aient même caractère est le lemme \ref{LempUSOlo}. Nous montrons l'autre sens. Si \( (\rho,V)\) et \( (\rho',V')\) sont deux représentations irréductibles de décompositions
    \begin{subequations}
        \begin{align}
            V&=\bigoplus_ik_iW_i\\
            V'&=\bigoplus_ik_i'W_i,
        \end{align}
    \end{subequations}
    alors si \( \chi=\chi'\), nous avons \( k_i=k'_i\) et les représentations sont identiques.

\item

    Soit \( (\rho,V)\) une représentation ayant \( \chi\) comme caractère. En posant \( k_i=\langle \chi, \varphi_i\rangle \) nous avons la décomposition en représentations irréductibles
    \begin{equation}
        V=\bigoplus_ik_iW_i,
    \end{equation}
    et aussi
    \begin{equation}
        \langle \chi, \chi\rangle =\langle \sum_ik_i\varphi_i, \sum_jk_j\varphi_j\rangle =\sum_ik_i^2\in \eN.
    \end{equation}
    Ce nombre est de plus égal à \( 1\) si et seulement si tous les termes de la somme sont nuls sauf un qui vaudrait \( 1\). Ce cas donne une représentation irréductible.

    \end{enumerate}
    
\end{proof}

\begin{proposition} \label{PropYLnxIjk}
    Si \( (\lambda,R)\) est la représentation régulière gauche de décomposition en représentations irréductibles
    \begin{equation}
        R=\bigoplus_ik_iW_i,
    \end{equation}
    alors
    \begin{enumerate}
        \item
            \( k_i=\dim W_i\),
        \item       \label{ITEMooLXIJooDxkGJh}
            \( \sum_i(\dim W_i)^2=|G|\),
        \item   \label{ItemEXAjTIh}
            pour tout \( g\in G\), \( \sum_i(\dim W_i)\varphi_i(g)=0\)\footnote{Cette propriété est appelée «orthogonalité des colonnes» pour une raison qui apparaîtra au moment de compléter le tableau \eqref{EqOKtZYFQ}.}.
        \item
            Si \( \{ (n_i,\varphi_i) \}\) est la liste des couples dimension,caractère des représentations irréductibles non équivalentes, alors pour tout \( s\in G\setminus\{ e \}\) nous avons \( \sum_{i=1}^pn_i\varphi_i(s)=0\) où la somme porte sur les représentations irréductibles non équivalentes.
    \end{enumerate}
\end{proposition}

\begin{proof}
    Nous notons \( r\) le caractère de la représentation régulière gauche. Nous avons
    \begin{equation}
        k_i=\langle r, \varphi_i\rangle =\frac{1}{ | G | }\sum_{s\in G}r(s)\overline{ \varphi_i(s) }=\overline{ \varphi_i(e) }.
    \end{equation}
    Mais \( \varphi_i(e)=\dim W_i\in \eR\), donc nous avons bien \( k_i=\dim W_i\). Le caractère de la représentation régulière peut alors s'exprimer de deux façons :
    \begin{equation}
        | G |\delta_e=\sum_i(\dim W_i)\varphi_i.
    \end{equation}
    En évaluant cette égalité en \( e\) nous trouvons directement
    \begin{equation}
        | G |=\sum_i(\dim W_i)^2,
    \end{equation}
    et en l'évaluant en \( s\neq e\), nous trouvons
    \begin{equation}
        0=\sum_i(\dim W_i)\varphi_i(s).
    \end{equation}
\end{proof} 

Le théorème suivant est valable pour les groupes finis (comme toute cette section).
\begin{theorem}[\cite{NhCRhg}]  \label{Thogocemg}
    Les caractères irréductibles \( \chi_1,\ldots, \chi_h\) forment une base orthonormé des fonctions centrales sur \( G\).
\end{theorem}

\begin{proof}
    Nous savons déjà qu'ils forment un système orthonormé. Considérons le sous-espace \( H=\Span\{ \varphi_i \}_{i=1,\ldots, h}\) de l'espace des fonctions centrales sur \( G\). En vertu de la proposition \ref{PropXrTDIi}, il nous suffit de prouver que \( H^{\perp}=0\). Soit donc \( f\), une fonction centrale appartenant à \( H^{\perp}\). Pour tout \( i\), nous avons \( \langle f, \varphi_i\rangle =0\) et donc aussi \( \langle \bar f, \bar\varphi_i\rangle =0\).

    Considérant une représentation irréductible \( (\sigma,W)\) de caractère \( \varphi\), nous savons par la proposition \ref{PropEAXkAY} que l'opérateur
    \begin{equation}
        \sigma_{\bar f}=\sum_g\bar f(g)\varphi(g)
    \end{equation}
    est une homothétie de rapport \( \langle \bar f, \bar\varphi\rangle/\dim W=0\). Étant donné que toute les représentations sont des sommes directes de représentations irréductibles, en réalité l'opérateur \( \rho_{\bar f}\) est nul pour toute représentation \( \rho\). En particulier pour la représentation régulière,
    \begin{equation}
        0=\lambda_{\bar f}(\delta_t)=\sum_{g\in G}\bar f(g)\lambda(g)(\delta_t)=\sum_g\bar f(g)\delta_{ft}.
    \end{equation}
    En écrivant cette égalité avec \( t=e\) et puis en appliquant à \( k\in G\) nous trouvons
    \begin{equation}
        0=\sum_g\bar f(g)\delta_g(k)=\bar f(k).
    \end{equation}
    Donc \( \bar f=0\) et \( f\) est nulle.
\end{proof}

\begin{corollary}   \label{CorbdcVNC}
    Le nombre de représentations irréductibles non équivalentes d'un groupe fini est égal à son nombre de classes de conjugaison.
\end{corollary}

\begin{proof}
    Le nombre de classes de conjugaison est la dimension de l'espace des fonctions centrales qui elle-même est égale au nombre de caractères irréductibles par le théorème \ref{Thogocemg}. Enfin deux caractères irréductibles sont égaux si et seulement si les représentations sous-jacentes sont équivalentes.
\end{proof}

\begin{corollary}       \label{CORooWAGXooByrelO}
    Toutes les représentations irréductibles d'un groupe abélien sont de dimension \( 1\).
\end{corollary}

\begin{proof}
    Le corollaire \ref{CorbdcVNC} nous dit qu'il y a autant de représentations unitaires qu'il n'y a de représenations irréductibles (non équivalentes). Mais les classes de conjugaisons sont des singletons (lemme \ref{LEMooQYBJooYwMwGM}). Nous avons donc exactement \( | G |\) représentations irréductibles lorsque \( G\) est abélien.

    Mais d'autre part la proposition \ref{PropYLnxIjk}\ref{ITEMooLXIJooDxkGJh} donne \( \sum_i(\dim W_i)^2=| G |\) lorsque la somme parcours les représentations irréductibles. Il y a \( | G |\) termes à la somme, donc tous les termes doivent être \( 1\).
\end{proof}

%+++++++++++++++++++++++++++++++++++++++++++++++++++++++++++++++++++++++++++++++++++++++++++++++++++++++++++++++++++++++++++ 
\section{Représentation produit tensoriel}
%+++++++++++++++++++++++++++++++++++++++++++++++++++++++++++++++++++++++++++++++++++++++++++++++++++++++++++++++++++++++++++

Soient \( \rho\) et \( \phi\), deux représentations d'un groupe \( G\) sur des espaces vectoriels \( V\) et \( W\). La représentation \defe{produit tensoriel}{produit!tensoriel!de représentations}\index{représentation!produit tensoriel} est la représentation
\begin{equation}
    \begin{aligned}
        \rho\otimes\phi\colon G&\to \GL(V\otimes W) \\
        (\rho\otimes\phi)(g)(v\otimes w)&=\rho(g)v\otimes \phi(g)w. 
    \end{aligned}
\end{equation}
Pour trouver son caractère, nous considérons une base \( \{ e_i \}\) de \( V\) et une base \( \{ e_{\alpha} \}\) de \( W\), et la base \( \{ e_i\otimes e_{\alpha} \}\) de \( V\otimes W\). Donc
\begin{equation}
    (\rho\otimes \phi)(g)(e_i\otimes e_{\alpha})=\rho(g)e_i\otimes \phi(g)e_{\alpha}.
\end{equation}
Nous devons savoir quelle est la composante «\( e_i\otimes e_{\alpha}\)» de cette dernière expression, et c'est évidemment
\begin{equation}
    \rho(g)_{ii}\rho_{\alpha\alpha}, 
\end{equation}
ce qui nous amène à dire que
\begin{equation}
    \tr(\rho\otimes \phi)(g)=\sum_i\sum_{\alpha}\rho(g)_{ii}\phi(g)_{\alpha\alpha}=\tr\big( \rho(g) \big)\tr\big( \phi(g) \big),
\end{equation}
c'est à dire au final que
\begin{equation}    \label{EqOTmvfjf}
    \chi_{\rho\otimes \phi}=\chi_{\rho}\chi_{\phi}.
\end{equation}

%+++++++++++++++++++++++++++++++++++++++++++++++++++++++++++++++++++++++++++++++++++++++++++++++++++++++++++++++++++++++++++
\section{Exemple sur le groupe symétrique}
%+++++++++++++++++++++++++++++++++++++++++++++++++++++++++++++++++++++++++++++++++++++++++++++++++++++++++++++++++++++++++++

Soit \( G=S_3\), un des premiers groupes finis non abéliens. On en a une représentation de dimension deux en tant que permutation des sommets d'un triangle équilatéral, donnée dans l'exemple \ref{ExKUAyUD}; nous notons \( \rho\) cette représentation.

Nous y avons aussi la représentation de signature donnée par
\begin{equation}
    \begin{aligned}
        \epsilon\colon S_3&\to \GL(\eC) \\
        \sigma&\mapsto \epsilon(\sigma)\id. 
    \end{aligned}
\end{equation}
Et enfin il y a la représentation triviale. Ce sont les trois représentations irréductibles; pour rappel il y a autant de représentations irréductibles que de classes de conjugaison (corollaire \ref{CorbdcVNC}).

\begin{center}
    \begin{tabular}[]{ccccc}
        Classe de conjugaison   &   taille  &   \( \chi_1\) &   \( \chi_{\epsilon}\)    &   $\chi_{\rho}$\\
         $\id$   &   $1$    &   $1$    &   $1$    &   $2$    \\
         \( (A,B)\)   &   $3$    &   $1$    &   $-1$    &   $0$    \\
         \( (A,B,C)\)   &   \( 2\)    &   \( 1\)    &   \( 1\)    &   $-1$    \\
    \end{tabular}
\end{center}

Nous calculons par exemple le produit scalaire
\begin{subequations}
    \begin{align}
        \langle \chi_1, \chi_{\epsilon}\rangle &=\frac{1}{ 6 }\big( 1\cdot\chi_1(\id)\overline{ \chi_{\epsilon}(\id) }+3\cdot \chi_1(A,B)\overline{ \chi_{\epsilon}(A,B) }+2\cdot \chi_1(A,B,C)\overline{ \chi_{\epsilon}(A,B,C) } \big)\\
        &=0.
    \end{align}
\end{subequations}
D'autre part nous avons aussi
\begin{equation}
    \langle \chi_{\rho}, \chi_{\rho}\rangle =\frac{1}{ 6 }(1\cdot2\cdot 2+3\cdot 0+2\cdot 1)=1.
\end{equation}

%+++++++++++++++++++++++++++++++++++++++++++++++++++++++++++++++++++++++++++++++++++++++++++++++++++++++++++++++++++++++++++ 
\section{Table des caractères du groupe symétrique \texorpdfstring{$ S_4$}{S4}}
%+++++++++++++++++++++++++++++++++++++++++++++++++++++++++++++++++++++++++++++++++++++++++++++++++++++++++++++++++++++++++++
\label{SecUMIgTmO}
\index{groupe!de permutation!caractères de \( S_4\)}
\index{caractère!de \( S_4\)}
\index{représentation!de groupe fini!caractères de \( S_4\)}

Pour la table des caractères de \( S_4\), voir \cite{KXjFWKA}.

%--------------------------------------------------------------------------------------------------------------------------- 
\subsection{Calculs à partir de rien ou presque}
%---------------------------------------------------------------------------------------------------------------------------

Nous savons que les classes de conjugaison dans \( S_4\) sont caractérisées par la structure des décompositions en cycles (proposition \ref{PropEAHWXwe}). Elles sont données dans l'exemple \ref{ExVYZPzub}.

Nous avons donc \( 5\) classes de conjugaison, et il nous faut donc \( 5\) représentations irréductibles non équivalentes (corollaire \ref{CorbdcVNC}) dont nous allons chercher les caractères. 

La première est la représentation triviale de dimension \( 1\); nous notons \( \chi_1\) son caractère et nous avons la ligne
\begin{equation}
    \begin{array}[]{c|c||c|c|c|c|c|}
        &\text{dimension}&\id&(12)&(123)&(1234)&(12)(34)\\
          \hline
          \chi_1&1&1&1&1&1&1\\ 
    \end{array}
\end{equation}
Ensuite nous avons la signature qui est un morphisme non trivial \( \epsilon\colon S_n\to \{ -1,1 \}\). Nous avons alors la ligne
\begin{equation}    \label{EqGNRavtl}
    \begin{array}[]{c|c||c|c|c|c|c|}
        &\text{dimension}&\id&(12)&(123)&(1234)&(12)(34)\\
          \hline
          \chi_{\epsilon}&1&1&-1&1&-1&1\\ 
    \end{array}
\end{equation}
Une troisième représentation pas trop compliquée à trouver est celle 
\begin{equation}
    \begin{aligned}
        \rho_p\colon S_4&\to \GL(4,\eC) \\
        \rho_p(\sigma)e_i&=e_{\sigma(i)}. 
    \end{aligned}
\end{equation}
Cela n'est pas une représentation irréductible parce que \( \eC^4\) se décompose en deux sous-espaces stables :
\begin{subequations}
    \begin{align}
        D&=\Span(1,1,1,1)\\
        H&=\{ x\in\eC^4\tq x_1+x_2+x_3+x_4=0 \}.
    \end{align}
\end{subequations}
La représentation induite sur \( D\) est la représentation triviale. Puis sur \( H\), elle induit une autre représentations que nous allons noter \( \rho_s\). 

Nous allons à présent déduire le caractère de la représentation \( \rho_s\) et prouver qu'elle est irréductible. Il est cependant possible de sauter cette étape en échange d'un certain travail sur les isométries du tétraèdre. Voir la proposition \ref{PROPooVNLKooOjQzCj} et ensuite 

Nous avons la décomposition \( \rho_p=\rho_1\oplus \rho_s\) et donc
\begin{equation}
    \chi_p=\chi_1+\chi_s.
\end{equation}
Nous savons déjà \( \chi_1\). Le caractère \( \chi_p\) n'est pas très compliqué parce que \( \chi_p(\sigma)\) est une matrice de permutation des vecteurs de base. Donc la matrice \( \rho_p(\sigma)\) a un \( 1 \) sur la diagonale pour les \( i\) tels que \( \sigma(i)=i\). Nous avons donc
\begin{subequations}
    \begin{align}
    \chi_p(\id)&=4&\chi_p(12)&=2\\
    \chi_p\big( (12)(34) \big)&=0&\chi_p(123)&=1\\
    \chi_p(1234)&=0.
    \end{align}
\end{subequations}
Le caractère \( \chi_s\) peut être calculé par simple soustraction :
\begin{equation}   \label{EqILZsKfo}
    \begin{array}[]{c|c||c|c|c|c|c|}
        &\text{dimension}&\id&(12)&(123)&(1234)&(12)(34)\\
          \hline
          \chi_s&3&3&1&0&-1&-1\\ 
    \end{array}
\end{equation}
Avant d'ajouter cette ligne au tableau des représentations irréductibles nous devons savoir si \( \rho_s\) en est une. Pour cela, tant que nous avons son caractère nous pouvons utiliser le critère du théorème \ref{ThoWGkfADd} :
\begin{equation}
    \langle \chi_s, \chi_s\rangle =\frac{1}{ | S_4 | }\sum_{\sigma\in S_4}\chi_s(\sigma)^2.
\end{equation}
Nous avons tout de suite \( | S_4 |=4\cdot 3\cdot 2=24\) et puis
\begin{equation}
    24\langle \chi_s, \chi_s\rangle =3^2+6\cdot 1^2+8\cdot 0^2+6\cdot(-1)^2+3\cdot (-1)^2=24,
\end{equation}
donc oui, le caractère est irréductible parce que \( \langle \chi_s, \chi_s\rangle =1\). Et nous pouvons donc ajouter la ligne \eqref{EqILZsKfo} à notre tableau. Par ailleurs, nous notons qu'elle est de dimension \( 3\).

Pour le reste nous savons qu'il y a autant de représentations irréductibles que de classes de conjugaison, de telle sorte qu'il ne manque que deux représentations irréductibles. De plus la proposition \ref{PropYLnxIjk} nous dit que si \( n_i\) est la dimension de la \( i\)\ieme\ représentation irréductible, alors
\begin{equation}
    | S_4 |=\sum_in_i^2.
\end{equation}
Dans notre situation, si nous nommons \( n_1\) et \( n_2\) les dimensions des deux représentations qui nous manquent, nous avons \( 24=n_1^2+n_2^2+(1^2+1^2+3^2)\), c'est à dire \( n_1^2+n_2^2=13\). Il n'y a pas des tonnes de sommes de deux carrés qui font \( 13\). Il y a \( n_1=2\) et \( n_2=3\), et c'est tout.

Nous recherchons donc encore une représentation de dimension \( 2\) et une de dimension \( 3\). Pour cela nous allons un peu regarder les produits tensoriels qui s'offrent à nous. Pour faire une dimension \( 3\), il faut faire le produit d'une de dimension \( 1\) par une de dimension \( 3\). Là encore le choix est très limité et nous demande d'essayer
\begin{equation}
    \rho_W=\rho_s\otimes \rho_{\epsilon}
\end{equation}
qui agit sur l'espace \( V_2\otimes V_{\epsilon}\) par
\begin{equation}
    \rho_W(g)(v\otimes x)=\rho_s(g)v\otimes \rho_{\epsilon}(g)x.
\end{equation}
Pour savoir son caractère nous utilisons la petite formule toute simple \eqref{EqOTmvfjf} : nous multiplions case par case les tableaux \eqref{EqILZsKfo} et \eqref{EqGNRavtl} :
\begin{equation}
    \begin{array}[]{c|c||c|c|c|c|c|}
        &\text{dimension}&\id&(12)&(123)&(1234)&(12)(34)\\
          \hline
          \chi_{W}&3&3&-1&0&1&-1\\ 
    \end{array}
\end{equation}
Avant de réellement ajouter cette ligne au tableau, nous devons nous assurer qu'elle est bien irréductible. Nous utilisons le même critère : \( \langle \chi_W, \chi_W\rangle =1\), donc c'est bon.

Pour trouver le dernier caractère, que nous nommerons \( \chi_u\), il ne faut pas beaucoup d'imagination. Il suffit d'utiliser les relations d'orthogonalité du théorème \ref{Thogocemg}, en sachant que la dimension est \( 2\) et qu'alors \( \chi_W(\id)=2\), c'est pas trop compliqué :
\begin{equation}    \label{EqOKtZYFQ}
    \begin{array}[]{c|c||c|c|c|c|c|}
        &\text{dimension}&\id&(12)&(123)&(1234)&(12)(34)\\
          \hline
          \chi_1&1&1&1&1&1&1\\ 
          \hline
          \chi_{\epsilon}&1&1&-1&1&-1&1\\ 
          \hline
          \chi_s&3&3&1&0&-1&-1\\ 
          \hline
          \chi_W&3&3&-1&0&1&-1\\ 
          \hline
          \chi_u&2&2&b&c&d&e\\ 
    \end{array}
\end{equation}
Les relations d'orthogonalité des colonnes de la propriété \ref{PropYLnxIjk} nous permettent de calculer les coefficients manquants. En pratique, il suffit de prendre le produit scalaire de chaque ligne avec la première et d'égaler avec zéro. Nous trouvons \( b=0\), \( c=1\), \( d=0\), et \( e=2\). Le tableau final est :
\begin{equation}
    \begin{array}[]{c|c||c|c|c|c|c|}
        &\text{dimension}&\id&(12)&(123)&(1234)&(12)(34)\\
          \hline
          \chi_1&1&1&1&1&1&1\\ 
          \hline
          \chi_{\epsilon}&1&1&-1&1&-1&1\\ 
          \hline
          \chi_s&3&3&1&0&-1&-1\\ 
          \hline
          \chi_W&3&3&-1&0&1&-1\\ 
          \hline
          \chi_u&2&2&0&-1&0&2\\ 
    \end{array}
\end{equation}
Notons que nous sommes parvenus à remplir la dernière ligne sans rien savoir de la représentation qui va avec. Nous allons cependant donner une interprétation géométrique et fixer cette représentation comme agissant sur le triangle équilatéral en \ref{NORMooQQCYooILyOxc}.

%TODO : il est dit que le caractère d'une représentation la caractérise. Il faudrait voir ça et faire ici.

%--------------------------------------------------------------------------------------------------------------------------- 
\subsection{Représentation de \( S_4\) via les isométries du tétraèdre}
%---------------------------------------------------------------------------------------------------------------------------
\label{SUBSECooLEUAooGGjGIZ}

Une des représentations trouvées (la représentation \( \rho_s\)) peut être vue comme le groupe \( \Iso(T)\) des isométries affine du tétraèdre. Nous avons vu en la proposition \ref{PROPooVNLKooOjQzCj} qu'il existe un isomorphisme de groupe \( S_4\simeq \Iso(T)\) lorsque \( T\) est un tétraèdre régulier de \( \eR^3\).

Si le barycentre de \( T\) est situé à l'origine de \( \eR^3\), alors les éléments de \( \Iso(T)\) sont des applications linéaires parce que
\begin{itemize}
    \item les affinités laissent invariantes les barycentres (proposition \ref{PROPooGSPZooRnVgiU}),
    \item les affinités qui laissent l'origine invariante sont linéaires (corollaire \ref{CORooATCNooUwEPNI}).
\end{itemize}
Nous allons à présent calculer la trace de cette représentation, en utilisant le fait que nous la connaissions explicitement. Nous savons que les caractères sont constants sur les classes de conjugaison; nous allons donc écrire une matrice par classe de conjugaison (qui sont données dans l'exemple \ref{EXooQAXRooBsPURs}). 

Pour tout cela nous allons considérer un tétraèdre dont le centre (isobary) est en \( (0,0,0)\) et une base de \( \eR^3\) formée de trois sommets \( e_1\), \( e_2\) et \( e_3\). Vu que l'isobarycentre des quatre sommes est en \( (0,0,0)\), le quatrième somme est forcément le point de coordonnées \( e_4(-1,-1,-1)\), de telle sorte que \( e_1+e_2+e_3+e_4=0\).

\begin{description}
    \item[Les transpositions]

        Quelle isométrie de $\eR^3$ permute deux sommets du tétraèdre sans bouger les autres ? Pour permuter les sommets \( e_1\) et \( e_2\) en laissant \( e_3\) et \( e_4\), c'est le symétrie par rapport au plan médiateur de \( [e_1,e_2]\). Ce plan passe par les sommets \( e_3\) et \( e_4\), parce que le tétraèdre étant régulier, les points \( e_3\) \( e_4\) sont équidistants de \( e_1\) et \( e_2\). Le lemme \ref{LEMooVBVUooOTFFXT} dit qu'alors ces points dont partie du plan médiateur.

        Dans notre base, la matrice de la transposition précédemment nommée \( (12)\) est
        \begin{equation}
            \begin{pmatrix}
                0    &   1    &   0    \\
                1    &   0    &   0    \\
                0    &   0    &   1
            \end{pmatrix},
        \end{equation}
        dont la trace est \( 1\). Donc \( \chi_s(12)=1\).

    \item[Les bitranspositions]

        La bitransposition \( (12)(34)\) est le produit des transpositions selon les plans médiateur de \( [e_1,e_2]\) et \( [e_3,e_4]\). Ces deux plans sont perpendiculaires, et l'intersection est la droite qui passe par les milieux. Cette droite est perpendiculaire aux deux segments en même temps. La matrice est :
        \begin{equation}
            \begin{pmatrix}
                0    &    1   &   -1    \\
                1    &   0    &   -1    \\
                0    &   0    &   -1
            \end{pmatrix}
        \end{equation}
        parce que \( e_1\mapsto e_2\), \( e_2\mapsto e_1\) et \( e_3\mapsto e_4\). Pour rappel, la matrice est formée des images des vecteurs de base. Cela donne
        \begin{equation}
            \chi_s\big( (12)(34) \big)=-1.
        \end{equation}

    \item[Les \( 3\)-cycles]
        
        La symétrie qui permute cycliquement les point \( e_1\), \( e_2\) et \( e_3\) est la rotation d'angle \( 2\pi/3\) dans le plan formé par les extrémités de ces trois vecteurs. Heureusement, la trace est invariante par changement de base; donc nous pouvons calculer la trace d'une rotation d'angle \( 2\pi/3\) dans n'importe quelle base. Par exemple :
        \begin{equation}
            \chi_s\big( (12)(34) \big)=\tr\begin{pmatrix}
                1    &   0    &   0    \\
                0    &   \cos(2\pi/3)    &   \sin(2\pi/3)    \\
                0    &   -\sin(2\pi/3)    &   \cos(2\pi/3)
            \end{pmatrix}=1+2\cos(2\pi/3)=0.
        \end{equation}
        
        Notons que, sans cette interprétation géométrique, nous y arrivons aussi facilement : dans notre base le \( 3\)-cycle est \( e_1\mapsto e_2\mapsto e_3\mapsto e_1\), donc la matrice est :
        \begin{equation}
            \begin{pmatrix}
                0    &   0    &   1    \\
                1    &   0    &   0    \\
                0    &   1    &   0
            \end{pmatrix},
        \end{equation}
        dont la trace est manifestement nulle : \( \chi_s\big( (123) \big)=0\).

    \item[Le \( 4\)-cycle]

        Il fait \( e_1\mapsto e_2\mapsto e_3\mapsto e_4\mapsto e_1\), dont la matrice est
        \begin{equation}        \label{EQooONDUooYlduup}
            \begin{pmatrix}
                0    &   0    &   -1    \\
                1    &   0    &   -1    \\
                0    &   1    &   -1
            \end{pmatrix},
        \end{equation}
        et la trace est \( \chi_s\big( (1,2,3,4) \big)=-1\).
\end{description}
Nous avons retrouvé les caractères de la représentation \( \rho_s\), et nous pouvons vérifier qu'elle est irréductible.

%--------------------------------------------------------------------------------------------------------------------------- 
\subsection{À propos de la représentation \( \rho_u\)}
%---------------------------------------------------------------------------------------------------------------------------

Nous nous penchons à présent sur la représentations \( \rho_u\) dont nous ne savons rien à part qu'elle est de dimension \( 2\) et son caractère. 

\begin{lemma}
    Nous avons \( \rho_u(s)=\id\) pour tout \( s\in V_4\).
\end{lemma}

\begin{proof}
    Tous les éléments de \( V_4\) sont conjugués (à part l'identité, mais pour elle le résultat est clair), donc il suffit de prouver le résultat pour un élément quelconque.

    L'endomorphisme \( \rho_u\big( (12)(34) \big)\) est un endomorphisme d'ordre \( 2\) sur \( \eR^2\) dont la trace est \( 2\). Imposons donc
    \begin{equation}
        \begin{pmatrix}
            a    &   b    \\ 
            c    &   d    
        \end{pmatrix}=\begin{pmatrix}
            a    &   b    \\ 
            c    &   d    
        \end{pmatrix}=\begin{pmatrix}
            1    &   0    \\ 
            0    &   1    
        \end{pmatrix}
    \end{equation}
    sous la contrainte \( a+d=2\). La résolution est assez rapide et donne \( b=c=0\), \( a=d=1\). 

    Vous voulez une démonstration plus technologique ? Remarquez que l'opérateur \( A=\rho_u\big( (12)(34) \big)\) vérifie \( A^2=1\), donc le polynôme \( X^2-1\) est un polynôme annulateur de \( A\). Il est peut-être minimal ou peut être pas, mais en tout cas le polynôme minimal divise celui-là et donc est soit \( X-1\) soit \( X+1\) soit \( X^2-1\). Dans les trois cas il est scindé à racines simples, et l'endomorphisme \( A\) est diagonalisable par le théorème \ref{ThoDigLEQEXR}\ref{ItemThoDigLEQEXRii}.

    Mais comme \( A^2=1\), les valeurs propres (ce qui est sur la diagonale) de \( A\) ne peuvent être que \( \pm1\). La trace étant \( 2\), les éléments diagonaux ne peuvent être que \( 1\). Et \( A=\id\).
\end{proof}


Le groupe \( V_4\) définit en \ref{NORMooQAZTooBQLqDn} est normal dans \( S_4\), donc le quotient \( S_4/V_4\) est un groupe par le lemme \ref{LEMooFNVRooRCkjLc}.

\begin{lemma}
    L'application
    \begin{equation}
        \begin{aligned}
            \tilde \rho_u\colon S_4/V_4&\to \GL(2,\eR) \\
            [g]&\mapsto \rho_u(g) 
        \end{aligned}
    \end{equation}
    est bien définie et donne une représentation irréductible de \( S_4/V_4\).
\end{lemma}

\begin{proof}
    Montrons que c'est bien définit. Si \( s\in V_4\) nous devons prouver que \( \rho_u(gs)=\rho_u(g)\). Vu que \( \rho_u\) est un homomorphisme (c'est une représentation), et que \( \rho_u(s)=\id\) nous avons directement
    \begin{equation}
        \rho_u(gs)=\rho_u(g)\rho_u(s)=\rho_u(g).
    \end{equation}
    
    Nous devons prouver que la représentation \( \tilde \rho_u\) est irréductible. Si un sous espace non trivial \( \Span(x)\) était stabilisé par \( \tilde \rho_u\), il serait également stabilisé par \( \rho_u\). Mais comme \( \rho_u\) est irréductible, elle ne stabilise personne.
\end{proof}

\begin{lemma}
    Le groupe \( S_4/V_4\) est un groupe non-abélien, isomorphe à \( S_3\).
\end{lemma}

\begin{proof}
    Le groupe \( S_4/V_4\) a une représentation irréductible de dimension \( 2\), et n'est donc pas abélien par le corollaire \ref{CORooWAGXooByrelO}. 
    
    Il contient \( | S_4 |/| V_4 |=24/4=6\) éléments (théorème de Lagrange \ref{ThoLagrange}). Or \( 6=3\times 2\), donc le groupe \( S_4/V_4\) est dans le cas non-abélien du théorème \ref{ThoLnTMBy}\ref{ITEMooFQXIooFLAiUD}. Cette partie parle d'unicité du groupe non-abélien d'ordre \( 6\). Or \( S_3\) est un groupe non-abélien d'ordre \( 6\), donc \( S_4/V_4\) est isomorphe à \( S_3\).
\end{proof}

Attention : il n'est pas correct de dire que \( S_4/V_4\) est un sous-groupe de \( S_4\) juste parce que c'est un quotient de \( S_4\); ce n'est en général pas vrai (exemple \ref{EXooFNIKooHxePSs}).

\begin{normaltext}      \label{NORMooQQCYooILyOxc}
    Nous sommes maintenant aptes à identifier la représentation \( \rho_u\). D'abord nous nous rappelons de la représentation \( \rho_s\colon S_4\to \Iso(T)\) de \( S_4\) sur le tétraèdre. Ensuite si \( A\) est un sommet dudit tétraèdre et que \( S_3\subset S_4\) est la partie qui fixe \( A\) alors nous avons une représentation
    \begin{equation}
        \rho_s\colon S_3\to \Iso(T)
    \end{equation}
    qui agit en réalité sur le triangle équilatéral \( T'\) opposé au sommet \( A\).

    Nous avons finalement la chaine d'homomorphismes de groupes
    \begin{equation}
        S_4\stackrel{\pr}{\longrightarrow} S_4/V_4\stackrel{\simeq}{\longrightarrow} S_3\stackrel{\rho_s}{\longrightarrow}\Iso(T')
    \end{equation}
    Cela est donc une représentation \( S_4\to\Iso(T')\). Elle est de dimension \( 2\) et est irréductible (elle contient les rotations d'angle \( 2\pi/3\) qui ne fixent aucune direction). Elle est donc la représentation \( \rho_u\) qui est la seule irréductible de dimension \( 2\).

    Nous avons donc montré que la représentation \( \rho_u\) dont nous ne savions rien est la représentation de \( S_4\) sur un triangle équilatéral obtenue à partir de celle de \( S_4\) sur le tétraèdre, en fixant un point.
\end{normaltext}<++>

%+++++++++++++++++++++++++++++++++++++++++++++++++++++++++++++++++++++++++++++++++++++++++++++++++++++++++++++++++++++++++++ 
\section{Table de caractères du groupe diédral}
%+++++++++++++++++++++++++++++++++++++++++++++++++++++++++++++++++++++++++++++++++++++++++++++++++++++++++++++++++++++++++++
\label{SecWMzheKf}
Cette section vient de \cite{KXjFWKA}; nous avons comme but d'établir la table des caractères des représentations complexes du groupe diédral \( D_n\).
\index{groupe!de permutation}
\index{groupe!diédral!générateurs (utilisation)}
\index{représentation!groupe diédral}
\index{caractère!groupe diédral}

%--------------------------------------------------------------------------------------------------------------------------- 
\subsection{Représentations de dimension un}
%---------------------------------------------------------------------------------------------------------------------------

Nous nous occupons des représentations de \( D_n\) sur \( \eC\). Les applications linéaires \( \eC\to \eC\) sont seulement les multiplications par des nombres complexes. Nous cherchons donc \( \psi\colon D_n\to \eC^*\).

Nous savons que \( D_n\) est généré\footnote{Voir proposition \ref{PropLDIPoZ} et tout ce qui suit.} par \( s\) et \( r\). Vu que \( s^2=1\), nous avons
\begin{equation}
    \psi(s)^2=\psi(s^2)=\psi(1)=1,
\end{equation}
donc \( \psi(s)\in\{ -1,1 \}\). Nous savons aussi que \( srsr=1\), donc
\begin{equation}
    \psi(s)^2\psi(r)^2=1,
\end{equation}
ce qui donne \( \psi(r)\in\{ -1,1 \}\).

Nous avons donc quatre représentations de dimension un données par
\begin{equation*}
    \begin{array}[]{|c||c|c|}
        \hline
        &\psi(r)=1&\psi(r)=-1\\
        \hline\hline
        \psi(s)=1&\rho^{++}&\rho^{+-}\\
        \hline
        \psi(s)=-1&\rho^{-+}&\rho^{--}\\
        \hline
    \end{array}
\end{equation*}
Attention au fait que nous devons aussi avoir la relation \( \psi(r)^n=\psi(r^n)=1\). Donc \( \psi(r)\) doit être une racine \( n\)\ieme\ de l'unité. Nous allons donc devoir avoir un compte différent selon la parité de \( n\). Nous en reparlerons à la fin, au moment de faire les comptes. En ce qui concerne les caractères correspondants,
\begin{equation*}
    \begin{array}[]{|c||c|c|}
        \hline
        &r^k&sr^k\\
        \hline\hline
        \chi^{++}&1&1\\
        \hline
        \chi^{+-}&(-1)^k&(-1)^k\\
        \hline
        \chi^{-+}&1&-1\\
        \hline
        \chi^{--}&(-1)^k&(-1)^{k+1}\\
        \hline
    \end{array}
\end{equation*}
Étant donné qu'ils sont tous différents, ce sont des représentations deux à deux non équivalentes, lemme \ref{LempUSOlo}.

%--------------------------------------------------------------------------------------------------------------------------- 
\subsection{Représentations de dimension deux}
%---------------------------------------------------------------------------------------------------------------------------

Nous cherchons maintenant les représentations \( \rho\colon D_n\to \End(\eC^2)\). Ici nous supposons connue la liste des éléments de \( D_n\) donnée par le corollaire \ref{CorWYITsWW}. Soit \( \omega= e^{2i\pi/n}\) et \( h\in \eZ\); nous considérons la représentation \( \rho^{(h)}\) de \( D_n\) définie par
\begin{subequations}
    \begin{align}
        \rho^{(h)}(r^k)&=\begin{pmatrix}
            \omega^{hk}    &   0    \\ 
            0    &   \omega^{-hk}    
        \end{pmatrix}\\
        \rho^{(h)}(st^k)&=\begin{pmatrix}
            0    &   \omega^{-hk}    \\ 
            \omega^{hk}    &   0    
        \end{pmatrix}.
    \end{align}
\end{subequations}
Cela donne bien \( \rho^{(h)}\) sur tous les éléments de \( D_n\) par la proposition \ref{PropLDIPoZ}. Nous pouvons restreindre le domaine de \( h\) en remarquant d'abord que \( \rho^{(h)}=\rho^{(h+n)}\), et ensuite que les représentations \( \rho^{(h)}\) et \( \rho^{(-h)}\) sont équivalentes. Un opérateur d'entrelacement est donné par \( T=\begin{pmatrix}
    0    &   1    \\ 
    1    &   0    
\end{pmatrix}\), et il est facile de vérifier que \( T\rho^{(h)}(x)=\rho^{-h}(x)T\) avec \( x=r^k\) puis avec \( x=sr^k\). 

Donc \( \rho^{(h)}\simeq\rho^{(-h)}\simeq\rho^{(n-h)}\) et nous pouvons restreindre notre étude à \( 0\leq h\leq \frac{ n }{2}\).

Nous allons séparer les cas \( n=0\), \( h=n/2\) et les autres. En effet si nous notons par commodité \( a=\omega^h\), alors un vecteur \( (x,y)\) est vecteur propre de \( \rho^{(h)}(s)\) et de \( \rho^{(h)}(r)\) si et seulement s'il vérifie les systèmes d'équations
\begin{subequations}        \label{SubEqsGXZoxLq}
    \begin{numcases}{}
        ax=\lambda x\\
        \frac{1}{ a }y=\lambda y
    \end{numcases}
\end{subequations}
et
\begin{subequations}    \label{SubEqsFYZmzhT}
    \begin{numcases}{}
        \frac{1}{ a }y=\mu x\\
        ax=\mu y
    \end{numcases}
\end{subequations}
avec \( \lambda\) et \( \mu\) des nombres non nuls. Une représentation sera réductible si et seulement si ces deux systèmes acceptent une solution non nulle commune. Il est vite vu que si \( x\neq 0\) et \( y\neq 0\), alors \( a^2=1\), ce qui signifie \( h=0\) ou \( h=n/2\). Sinon, il n'y a pas de solutions, et la représentation associée est irréductible.

\begin{enumerate}
    \item
        \( h=0\). Nous avons
        \begin{equation}
            \begin{aligned}[]
                \rho^{(0)}(r^k)&=\begin{pmatrix}
                    1    &   0    \\ 
                    0    &   1    
                \end{pmatrix}& \rho^{(0)}(sr^k)=\begin{pmatrix}
                    0    &   1    \\ 
                    1    &   0    
                \end{pmatrix},
            \end{aligned}
        \end{equation}
        donc le caractère de cette représentation est \( \chi^{(0)}(r^k)=2\) et \( \chi^{(0)}(sr^k)=0\). Donc nous avons
        \begin{equation}
            \chi^{(0)}=\chi^{++}+\chi^{-+}.
        \end{equation}
        Il y a maintenant (au moins) quatre façons de voir que la représentation \( \rho^{(0)}\) est réductible.
        \begin{description}

            \item[Première méthode]
                Trouver un opérateur d'entrelacement. Pour cela nous calculons les matrices :
        \begin{subequations}
            \begin{align}
                S(r)&=(\rho^{++}\oplus \rho^{-+})(r^k)=\begin{pmatrix}
                    \rho^{++}(r^k)    &   0    \\ 
                    0  &   \rho^{-+}(r^k)    
                \end{pmatrix}=\begin{pmatrix}
                    1    &   0    \\ 
                    0    &   1    
                \end{pmatrix}\\
                S(sr^k)&=(\rho^{++}\oplus \rho^{-+})(sr^k)=\begin{pmatrix}
                    \rho^{++}(sr^k)    &   0    \\ 
                    0  &   \rho^{-+}(sr^k)    
                \end{pmatrix}=\begin{pmatrix}
                    1    &   0    \\ 
                    0    &   -1    
                \end{pmatrix}\\
            \end{align}
        \end{subequations}
        Nous cherchons une matrice \( T\) telle que \( TS(r^k)=\rho^{(0)}(r^k)T\) et \( TS(sr^k)=\rho^{(0)}(sr^k)T\). Étant donné que \( S(r^k)=\mtu=\rho^{(0)}(r^k)\), la première contrainte n'en est pas une. Nous pouvons vérifier qu'avec \( T=\begin{pmatrix}
            1    &   1    \\ 
            1    &   -1    
        \end{pmatrix}\), nous avons bien
        \begin{equation}
            T\begin{pmatrix}
                1    &   0    \\ 
                0    &   -1    
            \end{pmatrix}=\begin{pmatrix}
                0    &   1    \\ 
                1    &   0    
            \end{pmatrix}.
        \end{equation}
        Donc ce \( T\) entrelace \( \rho^{++}\oplus \rho^{-+}\) avec \( \rho^{(0)}\) qui sont donc deux représentations équivalentes. Donc \( \rho^{(0)}\) est réductible et ça ne nous intéresse pas de la lister.
            \item[Seconde méthode] 
                Invoquer le théorème \ref{ThoWGkfADd}\ref{ItemZReOWoHi} pour dire que, les caractères étant égaux, les représentations sont équivalentes.

    \item[Troisième méthode]
        Utiliser le théorème \ref{ThoWGkfADd}\ref{ItemZReOWoHii} et calculer \( \langle \chi^{(0)}, \chi^{(0)}\rangle \) :
        \begin{subequations}
            \begin{align}
                \langle \chi^{(0)}, \chi^{(0)}\rangle &=\frac{1}{ | D_n | }\sum_{g\in D_n}| \chi^{(0)}(g) |^2\\
                &=\frac{1}{ 2n }\big(4+0+4(n-1)\big)\\
                &=2.
            \end{align}
        \end{subequations}
        Ici le \( 4\) est pour le \( 1\), le zéro est pour les termes \( sr^k\) et \( 4(n-1)\) est pour les \( n-1\) termes \( r^k\). Vu que le résultat n'est pas \( 1\), la représentation \( \rho^{(0)}\) n'est pas irréductible.
        
    \item[Quatrième méthode] 
        Regarder les solutions des systèmes \eqref{SubEqsGXZoxLq} et \eqref{SubEqsFYZmzhT} dont nous avons parlé plus haut.

    \end{description}

    La première méthode a l'avantage d'être simple et ne demander aucune théorie particulière à part les définitions. La seconde méthode est la plus rapide, mais demande un théorème très puissant. La troisième utilise également un théorème assez avancé, mais a l'avantage sur les deux autres méthodes de ne pas avoir besoin de savoir a priori un candidat décomposition de \( \rho^{0)}\); cette méthode est applicable même sans faire la remarque que \( \chi^{(0)}=\chi^{++}+\chi^{-+}\).

    Quoi qu'il en soit, nous ne listons pas \( \chi^{(0)}\) dans notre \href{http://fr.wikipedia.org/wiki/Aide:Unicode}{table de caractères}.

    \item
        \( h=n/2\). Vu que \( \omega^{n/2}= e^{i\pi}=-1\), nous avons
        \begin{equation}
            \begin{aligned}[]
                \rho^{(n/2)}(r^k)&=\begin{pmatrix}
                    (-1)^k    &   0    \\ 
                    0    &   (-1)^k    
                \end{pmatrix}&
                \rho^{(n/2)}(sr^k)&=\begin{pmatrix}
                    0   &   (-1)^k    \\ 
                    (-1)^k    &  0    
                \end{pmatrix}&
            \end{aligned},
        \end{equation}
        et donc
        \begin{subequations}
            \begin{align}
                \chi^{(n/2)}(r^k)&=2(-1)^k\\
                \chi^{(n/2)}(sr^k)&=0.
            \end{align}
        \end{subequations}
        Il est vite vu que \( \chi^{(n/2)}=\chi^{+-}+\chi^{-+}\). Ergo la représentation \( \rho^{(n/2)}\) n'est pas irréductible.

    \item
        \( 0<h<\frac{ n }{2}\). Dans ce cas nous avons \( \omega^h\neq \omega^{-h}\), et en regardant les systèmes d'équations donnés plus haut, nous voyons que \( \rho^{(h)}(s)\) et \( \rho^{(h)}(r)\) n'ont pas de vecteurs propres communs. Donc ces représentations sont irréductibles. 

        Nous devons cependant encore vérifier si elles sont deux à deux non équivalentes. Supposons que pour \( h\neq h'\) nous ayons une matrice \( T\in \GL(2,\eC)\) telle que \( T\rho^{(h)}(r)T^{-1}=\rho^{(h')}(r)\). Cela impliquerait en particulier que les matrices \( \rho^{(h)}(r)\) et \( \rho^{(h')}(r)\) aient même valeurs propres. Nous aurions donc \( \{ \omega^h,\omega^{-h} \}=\{ \omega^{h'},\omega^{-h'} \}\). Mais cela est impossible avec \( 0<h<h'<\frac{ n }{2}\). Donc toutes ces représentations sont distinctes.

\end{enumerate}

Le caractère de la représentation \( \rho^{(h)}\) est \( \chi^{(h)}(r^k)=\omega^{hk}+\omega^{-hk}=2\cos\left( \frac{ 2\pi hk }{ n } \right)\).

Nous ajoutons donc la ligne suivante à notre liste :
\begin{equation*}
    \begin{array}[]{|c||c|c|}
        \hline
        &r^k&sr^k\\
        \hline\hline
        \chi^{(h)}&2\cos\left( \frac{ 2\pi hk }{ n } \right)&0\\
        \hline
    \end{array}
\end{equation*}

%--------------------------------------------------------------------------------------------------------------------------- 
\subsection{Le compte pour \texorpdfstring{$ n$}{n} pair}
%---------------------------------------------------------------------------------------------------------------------------

Nous avons \( 4\) représentations de dimension \( 1\) puis \( \frac{ n }{2}-1\) représentations de dimension \( 2\). En tout nous avons 
\begin{equation}
 \frac{ n }{2}+3
\end{equation}
représentations irréductibles modulo équivalence. Cela fait le compte en vertu des classes de conjugaisons listées en \ref{SubsubsecROVmHuM}. Pour rappel, le nombre de représentations non équivalentes est égal au nombre de classes de conjugaison par le corollaire \ref{CorbdcVNC}. Notons que c'est cela qui justifie le fait que nous ne devons pas chercher d'autres représentations. Nous sommes sûrs de les avoir toutes trouvées.

%--------------------------------------------------------------------------------------------------------------------------- 
\subsection{Le compte pour \texorpdfstring{$ n$}{n} impair}
%---------------------------------------------------------------------------------------------------------------------------

Nous avions fait mention plus haut du fait que si \( \psi\) est une représentation de dimension \( 1\), le nombre \( \psi(r)\) devait être une racine \( n\)\ieme\ de l'unité. Donc en dimension \( 1\) nous avons seulement les représentations \( \rho^{++}\) et \( \rho^{-+}\). Pour celles de dimension \( 2\), nous en avons \( \frac{ n-1 }{2}\). En tout nous avons donc
\begin{equation}
    \frac{ n+3 }{2}
\end{equation}
représentations irréductibles modulo équivalence. Cela fait le compte en vertu des classes de conjugaisons listées en \ref{GJIzDEP}.
