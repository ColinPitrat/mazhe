% This is part of Mes notes de mathématique
% Copyright (c) 2009-2017
%   Laurent Claessens
% See the file fdl-1.3.txt for copying conditions.

%--------------------------------------------------------------------------------------------------------------------------- 
\subsection{La méthode de Rothstein-Trager}
%---------------------------------------------------------------------------------------------------------------------------
\label{subSecBCRYooRVjFpS}

Mes sources pour parler d'intégration de fractions rationnelles : \cite{CPheFRq}.

\begin{theorem}[Rothstein-Trager] \label{ThoXJFatfu}
    Soient \( P,Q\in \eQ[X]\) premiers entre eux avec \( \pgcd(P,Q)=1\) et \( \deg(P)<\deg(Q)\). Nous supposons que \( Q\) est unitaire et sans facteurs carrés. Supposons que nous puissions écrire, dans un extension \( \eK\) de \( \eQ\) la primitive de \( P/Q\) de la façon suivante :
    \begin{equation}        \label{EqCHVaDay}
        \int\frac{ P }{ Q }=\sum_{i=1}^n c_i\ln(P_i)
    \end{equation}
    où les \( c_i\) sont des constantes non nulles et deux à deux distinctes et où les \( P_i\) sont des polynômes unitaires non constants sans facteurs carrés et premiers deux à deux entre eux dans \( \eK[X]\).

    Alors les \( c_i\) sont les racines distinctes du polynôme
    \begin{equation}
        R(Y)=\res_X(P-YQ',Q)\in \eK[Y]
    \end{equation}
    et
    \begin{equation}
        P_i=\pgcd(P-c_iQ',Q).
    \end{equation}
\end{theorem}
\index{théorème!Rothstein-Trager}
\index{fraction!rationnelle!intégration}
\index{intégration!fraction rationnelle}
\index{déterminant!résultant}
\index{résultant!utilisation}

\begin{proof}
    Nous posons 
    \begin{equation}
        U_i=\prod_{j\neq i}P_j.
    \end{equation}
    \begin{subproof}
    \item[Question de division]
    Ensuite nous dérivons formellement l'équation \eqref{EqCHVaDay} et nous multiplions les deux côtés du résultat par \( \prod_{j=1}^nP_j\) :
    \begin{equation}        \label{EqGSJKyDw}
        P\prod_{j=1}^nP_j=Q\sum_{i=1}^nc_i\frac{ P'i }{ P_i }\prod_{j=1}^nP_j=Q\sum_{i=1}^nc_iP'_iU_i.
    \end{equation}
    Une première chose que nous en tirons est que \( Q\) divise le produit \( P\prod_{j=1}^nP_j\); mais \( P\) et \( Q\) étant premiers entre eux, 
    \begin{equation}
        Q\divides \prod_{j=1}^nP_j
    \end{equation}
    par le théorème de Gauss \ref{ThoLLgIsig}.

    Une seconde chose que nous tirons de \eqref{EqGSJKyDw} est que \( P_j\) divise \( Q\sum_{i=1}^nc_iP'_iU_i\). De cette somme, à cause du \( U_i\) qui est divisé par \( P_j\) pour tout \( i\) sauf \( i=j\), le polynôme \( P_j\) divise tous les termes sauf peut-être un. Donc il les divise tous et en particulier
    \begin{equation}
        P_j\divides Qc_jP'_JU_j
    \end{equation}
    En nous souvenant que les \( P_k\) sont premiers entre eux, \( P_j\) ne divise pas \( U_j\). De plus \( P_j\) étant sans facteurs carrés, les polynômes \( P_j\) et \( P'_j\) sont premiers entre eux. Il ne reste que \( Q\). Nous en déduisons que
    \begin{equation}
        P_j\divides Q
    \end{equation}
    pour tout \( 1\leq j\leq n\). Et vu que les \( P_i\) sont premiers entre eux, le fait que chacun divise \( Q\) implique que leur produit divise \( Q\), c'est à dire
    \begin{equation}
        \prod_{j=1}^nP_j\divides Q.
    \end{equation}
    Or nous avions déjà prouvé la division contraire. Du fait que les deux polynômes sont unitaires nous en déduisons qu'ils sont en réalité égaux :
    \begin{equation}        \label{EqJImORVe}
        Q=\prod_{j=1}^nP_j.
    \end{equation}
    Nous pouvons simplifier les deux membres de \eqref{EqGSJKyDw} par cela :
    \begin{equation}        \label{EqJMtGhGR}
        P=\sum_{i=1}^nc_iP'_iU_i.
    \end{equation}
    
\item[Encore un peu de division]

    En dérivant \eqref{EqJImORVe} nous trouvons
    \begin{equation}
        Q'=\sum_{j=1}^nP'_jU_j,
    \end{equation}
    et en écrivant \( P\) sous sa forme \eqref{EqJMtGhGR},
    \begin{equation}    \label{EqLZoYqxP}
        P-c_iQ'=\sum_{j=1}^nc_jP'_jU_j-\sum_{j=1}^nc_iP'_jU_j=\sum_{j=1}^n(c_j-c_i)P'_jU_j.
    \end{equation}
    Le terme \( i=j\) de la somme est nul; en ce qui concerne les autres termes, ils sont divisés par \( P_i\) parce que \( P_i\divides U_j\). Donc \( P_i\) divise tous les termes de la somme et nous avons
    \begin{equation}
        P_i\divides P-c_iQ'.
    \end{equation}
    
\item[Un pgcd pour continuer]

    Nous montrons à présent que \( P_i=\pgcd(P-c_iQ',Q)\). Pour cela nous utilisons la multiplicativité du PGCD lorsque les facteurs sont premiers entre eux :
    \begin{equation}
        \pgcd(P-c_iQ',Q)=\pgcd(P-c_iQ',\prod_{j=1}^nP_j)=\prod_{j=1}^n\pgcd(P-c_iQ',P_j).
    \end{equation}
    Nous remplaçons \( P-c_iQ'\) par son expression \eqref{EqLZoYqxP} et nous écrivons un des facteurs du produit :
    \begin{equation}
        \pgcd(P-c_iQ',P_j)=\pgcd(\sum_{k=1}^n(c_k-c_i)P'_kU_k,P_j)
    \end{equation}
    Le polynôme \( P_j\) divise tous les \( U_k\) sauf celui avec \( k=j\). Donc le lemme \ref{LemUELTuwK} nous permet de dire
    \begin{equation}
        \pgcd(P-c_iQ',P_j)=\pgcd\big( (c_j-c_i)P'_jU_j,P_j \big)=\begin{cases}
            1    &   \text{si } i\neq j\\
            P_j    &    \text{si } i=j\text{.}
        \end{cases}
    \end{equation}
    La seconde ligne provient du fait que nous ayons déjà montré que \( P_j\divides P-c_jQ'\). En fin de compte,
    \begin{equation}
        \pgcd(P-c_iQ',Q)=P_i.
    \end{equation}
    
\item[Une histoire de résultant]

    Les nombres \( c_i\) sont tels que les polynômes \( P-c_iQ'\) et \( Q\) ne sont pas premiers entre eux. Vu que les \( P_i\) sont non nuls, la proposition \ref{PropAPxzcUl} nous dit que le résultant
    \begin{equation}
        \res_X(P-c_iQ',Q)=0.
    \end{equation}
    Donc les \( c_i\) sont des racines du polynôme (en \( Y\))
    \begin{equation}    \label{EqOOimwJj}
        R(Y)=\res_X(P-YQ',Q).
    \end{equation}
    Nous n'avons pas prouvé qu'ils étaient \emph{toutes} les racines\footnote{De plus, nous n'avons pas de garanties que ces racines soient dans \( \eQ\), et en fait il y a des cas dans lesquels les \( c_i\) n'y sont pas.}.

\item[Toutes les racines]

    Nous allons maintenant montrer que les \( c_i\) étaient toutes les racines imaginables du polynôme \eqref{EqOOimwJj} dans toutes les extensions de \( \eQ\). Soit donc \( c\) une racine de \( R\) dans une extension \( \hat \eK\) de \( \eK\) qui ne soit pas parmi les \( c_i\) de la formule \eqref{EqCHVaDay}. Étant donné que \( c\) est racine du résultat, les polynômes \( P-cQ'\) et \( Q\) ont un PGCD non trivial, c'est à dire non constant. Donc
    \begin{equation}
        \pgcd(P-cQ',Q)=s\in \hat\eK[X]
    \end{equation}
    est un polynôme non constant. Si \( T\) un facteur irréductible de \( S\), alors \( T\) divise \( P-cQ'\) et \( Q\), mais \( Q=\prod_{i=1}^nP_i\) avec les \( P_i\) premiers entre eux. Donc \( T\) ne peut diviser que l'un (et exactement un) d'entre eux\footnote{On ne peut pas diviser deux trucs qui sont premiers entre eux; c'est une question de cohérence, madame !}. Soit \( P_{i_0}\) celui qui est divisé par \( T\). La relation \eqref{EqLZoYqxP} dans ce contexte donne :
    \begin{equation}
        P-cQ'=\sum_{j=1}^n(c_j-c)P'_jU_j
    \end{equation}
    Le polynôme \( T\) divise tous les \( U_j\) avec \( j\neq i_0\), mais comme en plus il divise \( P-cQ'\), il divise aussi le dernier terme de la somme :
    \begin{equation}
        T\divides (c_{i_0}-c)P'_{i_0}U_{i_0}.
    \end{equation}
    Le polynôme \( T\) ne divisant pas \( U_{i_0}\) et \(  (c_{i_0}-c)  \) étant non nul, nous concluons que \( T\) divise \( P'_{i_0}\). Mais cela n'est pas possible parce que nous avons supposé que \( P_{i_0}\) était sans facteur carré, ce qui voulait entre autres dire que \( P_{i_0}\) et \( P'_{i_0}\) n'ont pas de facteurs communs.

    \end{subproof}
\end{proof}

Ce théorème suggère la méthode suivante pour trouver la primitive de la fraction rationnelle \( P/Q\) (si elle vérifie les hypothèses)
\begin{enumerate}
    \item
        Écrire le résultant \( R(y)=\res_X(P-yQ',Q)\) et en trouver les racines \( \{ c_i \}_{i=1,\ldots, n}\).
    \item
        Calculer les \( P_p=\pgcd(P-c_iQ',Q)\).
    \item
        Écrire la réponse :
        \begin{equation}
            \int\frac{ P }{ Q }=\sum_{i=1}^nc_i\ln(P_i).
        \end{equation}
\end{enumerate}

Notons que le polynôme \( R(Y)\) est de degré \( \deg(Q)\) (pour le voir, faire un peu de comptage de lignes et colonnes dans la matrice de Sylvester), donc il n'est a priori pas pire à factoriser que \( Q\) lui-même\footnote{C'est de la factorisation de \( Q\) qu'on a besoin pour utiliser la méthode de décomposition en fractions simples.}. Mais il se peut que nous ayons de la chance et que \( R\) soit plus facile que \( Q\).

À part qu'on a peut-être plus de chance avec \( R\) qu'avec \( Q\), l'avantage de la méthode est qu'elle permet d'éviter de passer par des extensions de \( \eQ\) non nécessaires\quext{J'imagine que pour un ordinateur, c'est plus facile d'éviter les extensions.}.

\begin{example}
    Prenons la fraction rationnelle \( \frac{ x }{ x^2-3 }\). L'intégration via les fraction simples est :
\begin{equation}
    \int\frac{ x }{ x^2-3 }dx=\frac{ 1 }{2}\int\frac{ 1 }{ x-\sqrt{3} }+\frac{ 1 }{2}\int\frac{1}{ x+\sqrt{3} }=\frac{ 1 }{2}\ln(x-\sqrt{3})+\frac{ 1 }{2}\ln(x+\sqrt{3})=\frac{ 1 }{2}\ln(x^2-3).
\end{equation}
Nous voyons que dans la réponse, il n'y a pas de racines. Passer par l'extension \( \eQ[\sqrt{3}]\) est par conséquent peut-être un effort inutile. Voyons comment les choses se mettent avec la méthode Rothstein-Trager.

D'abord
\begin{subequations}
    \begin{align}
        R(Y)&=\res_X(X-2YX,X^2-3)\\
        &=\res_X\big( (1-2Y)X,X^2-3 \big)\\
        &=\det\begin{pmatrix}
            1-2Y    &   0    &   0    \\
            0    &   1-2Y    &   0    \\
            1    &   0    &   -3
        \end{pmatrix}\\
        &=(1-2Y)\big( -3(1-2Y) \big)\\
        &=-3(1-2Y)^2,
    \end{align}
\end{subequations}
dont les solutions sont faciles : il n'y a que la racine double \( y=\frac{ 1 }{2}\). La somme \eqref{EqCHVaDay} sera donc réduite à un seul terme avec \( c_1=\frac{ 1 }{2}\). Nous calculons \( P_1\) :
\begin{equation}
    P_1=\pgcd(X-\frac{ 1 }{2}2X,X^2-3)=\pgcd(0,X^2-3)=X^2-3,
\end{equation}
et par conséquent
\begin{equation}
    \int\frac{ X }{ X^2-3 }=\frac{ 1 }{2}\ln(X^2-3).
\end{equation}
À aucun moment nous ne sommes sortis de \( \eQ\).
\end{example}

Comme vu sur cet exemple, l'intérêt du théorème de Rothstein-Trager est de permettre, lorsqu'on a de la chance, d'en profiter, et non de nous en rendre compte à la fin en remarquant bêtement que la réponse pouvait s'écrire dans \( \eQ[X]\).

\begin{remark}
    Afin d'utiliser cette méthode, il faut s'assurer que \( Q\) soit sans facteurs carrés. Si nous devons intégrer un \( \frac{ P }{ Q }\) quelconque, nous devons commencer par écrire
    \begin{equation}
        Q=Q_1Q_2^2Q_3^3\ldots Q_r^r,
    \end{equation}
    et ensuite il y a moyen de ramener l'intégrale de \( P/Q\) à des intégrales de \( \frac{ P }{ Q_1\ldots Q_r }\). Cela ne demande pas de factoriser complètement \( Q\), mais seulement de trouver ses facteurs irréductibles \( Q_i\) dans \( \eQ[X]\).

    Dans l'exemple donné plus haut, \( Q=X^2-3\) a des facteurs irréductibles autres que \( Q\) lui-même dans \( \eR[X]\), mais nous n'en avons pas besoin.
\end{remark}

Voici une exemple où nous évitons de passer par les complexes.

\begin{example}     \label{EXooIPEQooGKDjea}
    À calculer : \( \int\frac{1}{ x^3+x }\). La décomposition en fractions simples donne :
    \begin{equation}
        \frac{1}{ x^3+x }= \frac{1}{ x }-\frac{ 1/2 }{ x-i }-\frac{ 1/2 }{ x+i }.
    \end{equation}
    Déjà cette décomposition passe par l'extension \( \eQ[i]\), et le calcul de la primitive de \( \frac{1}{ x+i }\) demande le logarithme complexe qui ne sera vu que dans la proposition \ref{PROPooNIJVooKueuYJ} au prix d'un peu de sang.
    Nous verrons dans l'exemple \ref{EXooAKEDooZgjocX} comment ça se passe en passant par les complexes.

    En ce qui concerne la méthode de Rothstein-Trager, nous commençons par calculer le résultant (qui est tout de même un peu de calcul) :
    \begin{subequations}
        \begin{align}
        P(y)&=\res_X\big( -3yX^2-y+1,X^3-X \big)\\
        &=\det\begin{pmatrix}
            -3y    &   0    &   1-y    &   0    &   0\\  
            0    &   -3y    &   0    &   1-y    &   0\\  
            0    &   0    &   -3y    &   0    &   1-y\\  
            1    &   0    &   1    &   0    &   0\\  
            0    &   1    &   0    &   1    &   0    
        \end{pmatrix}\\
        &=-(y-1)^2(2y+1)^2
        \end{align}
    \end{subequations}
    
    \begin{verbatim}
----------------------------------------------------------------------
| Sage Version 5.7, Release Date: 2013-02-19                         |
| Type "notebook()" for the browser-based notebook interface.        |
| Type "help()" for help.                                            |
----------------------------------------------------------------------
sage: y=var('y')
sage: R=matrix(5,5,[-3*y,0,1-y,0,0,0,-3*y,0,1-y,0,0,0,-3*y,0,1-y,1,0,1,0,0,0,1,0,1,0])
sage: R.determinant().factor()                                                        
-(y - 1)*(2*y + 1)^2
    \end{verbatim}
    Les solutions sont \( c_1=1\) et \( c_2=-\frac{ 1 }{2}\). Nous pouvons alors calculer les \( P_i\) :
    \begin{equation}
        P_1=\pgcd(-3X^2,X^3+X)=X
    \end{equation}
    et
    \begin{equation}
        P_2=\pgcd(\frac{ 3 }{2}X^2+\frac{ 3 }{2},X^3+X)=X^2+1,
    \end{equation}
    et finalement
    \begin{equation}
        \int\frac{ 1 }{ X^3+X }=\ln(X)-\frac{ 1 }{2}\ln(X^2+1).
    \end{equation}
\end{example}

Notons qu'il n'y a pas de miracles : lorsque la réponse contient des racines, nous ne pouvons pas couper à passer par des extensions et factoriser un peu à la dure.

\begin{example} \label{ExYQODuyU}
    Nous voulons calculer
    \begin{equation}
        \int\frac{ 1 }{ X^2+1 }.
    \end{equation}
    Nous posons donc \( P=1\) et \( Q=X^2+1\). Le résultant à calculer est
    \begin{subequations}
        \begin{align}
            P(y)=\res_X(-2yX+1,X^2+1)=\det\begin{pmatrix}
                -2y    &   1    &   0    \\
                0    &   -2y    &   1    \\
                1    &   0    &   1
            \end{pmatrix}=4y^2+1.
        \end{align}
    \end{subequations}
    Les racines de cela sont complexes et il n'y a donc pas d'échappatoires : \( c_1=\frac{ i }{2}\), \( c_2=-\frac{ i }{2}\). Ensuite, étant donné que \( X^2+1=(X+i)(X-i)=i(-iX+1)(X-1)\) nous avons
    \begin{equation}
        P_1=\pgcd(-iX+1,X^2+1)=X+i.
    \end{equation}
    Notons que de façon naturelle, nous aurions écrit \( P_1=1-iX\), mais par convention nous considérons le PGCD unitaire. Cela ne change rien à la réponse parce que changer \( P_i\) en \( kP_i\) ne fait que rajouter une constante \( \ln(k)\) à la primitive trouvée.

    De la même façon,
    \begin{equation}
        P_2=\pgcd(1+iX,X^2+1)=X-i.
    \end{equation}
    Au final nous écrivons
    \begin{equation}
        \int\frac{1}{ X^2+1 }=\frac{ i }{2}\ln(X+i)-\frac{ i }{2}\ln(X-i).
    \end{equation}
\end{example}

\begin{remark}
    Tout cela est si nous voulons absolument écrire la primitive avec des logarithmes de polynômes. Pour celui de l'exemple \ref{ExYQODuyU}, nous avons trouvé
    \begin{equation}    \label{EqBCjCCbs}
        \int\frac{1}{ X^2+1 }=\frac{ i }{2}\ln(X+i)-\frac{ i }{2}\ln(X-i).
    \end{equation}
    Mais
    \begin{verbatim}
    sage: f(x)=1/(x**2+1)                                                                                                                      
    sage: f.integrate(x)
    x |--> arctan(x)
    \end{verbatim}
    Si nous acceptons de passer aux fonctions trigonométriques (inverses), la primitive prend un tour très différent et bien réel. Ces deux visions de l'univers sont bien entendu\footnote{Si on croit que la mathématique est cohérente.} compatibles. En effet, afin de tomber juste, nous allons prendre la primitive
    \begin{equation}
        f(x)=\frac{ i }{2}\ln(ix-1)-\frac{ i }{2}\ln(ix+1)
    \end{equation}
    au lieu de \eqref{EqBCjCCbs}. Il s'agit seulement de multiplier l'intérieur des logarithmes, ce qui ne donne qu'une constante de différence. Ensuite nous passons à la forme trigonométrique des nombres complexes : \( ix-1=\sqrt{x^2+1} e^{i\arctan(-x)}\) et \( ix+1=\sqrt{x^2+1} e^{i\arctan(x)}\). Avec un peu de calcul,
    \begin{equation}
        f(x)=-\frac{ 1 }{2}\Big( \arctan(-x)-\arctan(x) \Big)=\arctan(x).
    \end{equation}
\end{remark}

%TODO : lorsque j'aurai fait la construction du logarithme et ses propriétés, il faudra en faire référence ici. 

%+++++++++++++++++++++++++++++++++++++++++++++++++++++++++++++++++++++++++++++++++++++++++++++++++++++++++++++++++++++++++++
\section{Rappel sur les intégrales usuelles}
%+++++++++++++++++++++++++++++++++++++++++++++++++++++++++++++++++++++++++++++++++++++++++++++++++++++++++++++++++++++++++++

%TODO : l'utilisation des macros \og et \fg ne se justifie plus : les enlever.

Soit une fonction
\begin{equation}
    \begin{aligned}
        f\colon \mathopen[ a , b \mathclose]\subset\eR&\to \eR^+ \\
        x&\mapsto f(x) .
    \end{aligned}
\end{equation}
L'intégrale de $f$ sur le segment $\mathopen[ a , b \mathclose]$, notée $\int_a^bf(x)dx$ est le nombre égal à l'aire de la surface située entre le graphe de $f$ et l'axe des $x$, comme indiqué à la figure \ref{LabelFigKKLooMbjxdI}. % From file KKLooMbjxdI
\newcommand{\CaptionFigKKLooMbjxdI}{L'intégrale de $f$ entre $a$ et $b$ représente la surface sous la fonction.}
\input{auto/pictures_tex/Fig_KKLooMbjxdI.pstricks}

\begin{definition}
    Si $f$ est une fonction de une variable à valeurs réelles, une \defe{primitive}{primitive} de $f$ est une fonction $F$ telle que $F'=f$.
\end{definition}

Toute fonction continue admet une primitive.

\begin{theorem}[Théorème fondamental du caclul intégral]
    Si $f$ est une fonction positive et continue, et si $F$ est une primitive de $f$, alors
    \begin{equation}
        \int_a^bf(x)dx=F(b)-F(a).
    \end{equation}
\end{theorem}

\begin{remark}
    Si $f$ est une fonction continue par morceaux, l'intégrale de $f$ se calcule comme la somme des intégrales de ses morceaux. Plus précisément si nous avons $a=x_0<x_1<\ldots<x_n=b$ et si $f$ est continue sur $\mathopen] x_i , x_{i+1} \mathclose[$ pour tout $i$, alors nous posons
    \begin{equation}
        \int_a^bf(x)dx=\int_{x_0}^{x_1}f(x)dx+\int_{x_1}^{x_2}f(x)dx+\cdots+\int_{x_{n-1}}^{n_n}f(x)dx.
    \end{equation}
    Sur chacun des morceaux, l'intégrale se calcule normalement en passant par une primitive.
\end{remark}

%+++++++++++++++++++++++++++++++++++++++++++++++++++++++++++++++++++++++++++++++++++++++++++++++++++++++++++++++++++++++++++
\section{Intégrales le long de chemins}
%+++++++++++++++++++++++++++++++++++++++++++++++++++++++++++++++++++++++++++++++++++++++++++++++++++++++++++++++++++++++++++

%---------------------------------------------------------------------------------------------------------------------------
\subsection{Circulation d'un champ de vecteur}
%---------------------------------------------------------------------------------------------------------------------------

\begin{definition}
    Soit $F\colon \eR^3\to \eR^3$ un champ de vecteurs et un chemin $\sigma\colon \mathopen[ a , b \mathclose]\to \eR^3$. On appelle \defe{circulation}{circulation} de $F$ le long du chemin $\sigma$ le scalaire
    \begin{equation}        \label{EqDeffvkZwh}
        \int_a^b F\big( \sigma(t) \big)\cdot \sigma'(t)dt.
    \end{equation}
    Il existe de nombreuses notations pour cela; entre autres :
    \begin{equation}
        \int_{\sigma}F=\int_{\sigma} F\cdot ds.
    \end{equation}
\end{definition}

En physique, la circulation de la force le long d'un chemin est la travail de la force.

\begin{example}
    À la surface de la Terre, le champ de gravitation est donné par
    \begin{equation}
        G(x,y,z)=-mg\begin{pmatrix}
            0    \\ 
            0    \\ 
            1    
        \end{pmatrix}.
    \end{equation}
    Si nous considérons un mobile qui monte à vitesse constante jusqu'à la hauteur $h$, c'est à dire le chemin
    \begin{equation}
        \sigma(t)=\begin{pmatrix}
            0    \\ 
            0    \\ 
            t    
        \end{pmatrix}
    \end{equation}
    avec $t\in\mathopen[ 0 , h \mathclose]$. Le travail de la gravitation est alors donné par
    \begin{equation}
        W=\int_0^hG\big( \sigma(t) \big)\cdot\begin{pmatrix}
            0    \\ 
            0    \\ 
            1    
        \end{pmatrix}=
        -mg\int_0^h\begin{pmatrix}
            0    \\ 
            0    \\ 
            1    
        \end{pmatrix}\cdot\begin{pmatrix}
            0    \\ 
            0    \\ 
            1    
        \end{pmatrix}=-mgh.
    \end{equation}
    Cela est bien le résultat usuel de l'énergie potentielle. Nous allons voir bientôt que nous nommons la fonction $mgh$ énergie \emph{potentielle} précisément parce que la force dérive de ce potentiel.
\end{example}

\begin{example}
    Soit le chemin
    \begin{equation}
        \begin{aligned}
            \sigma\colon \mathopen[ 0 , 2\pi \mathclose]&\to \eR^3 \\
            t&\mapsto \begin{pmatrix}
                \sin(t)    \\ 
                \cos(t)    \\ 
                t    
            \end{pmatrix}.
        \end{aligned}
    \end{equation}
    et le champ de vecteurs
    \begin{equation}
        F\begin{pmatrix}
            x    \\ 
            y    \\ 
            z    
        \end{pmatrix}=\begin{pmatrix}
            x    \\ 
            y    \\ 
            z    
        \end{pmatrix}.
    \end{equation}
    La circulation de ce champ de vecteur le long de l'hélice $\sigma$ est
    \begin{equation}
        \begin{aligned}[]
            \int_{\sigma}F\cdot ds&=\int_0^{2\pi}(F\circ \sigma)(t)\cdot \sigma'(t)dt\\
            &=\int_0^{2\pi}\begin{pmatrix}
                \sin(t)    \\ 
                \cos(t)    \\ 
                t    
            \end{pmatrix}\cdot
            \begin{pmatrix}
                \cos(t)    \\ 
                \sin(t)    \\ 
                1    
            \end{pmatrix}dt\\
            &=\int_0^{2\pi}tdt\\
            &=\left[ \frac{ t^2 }{2} \right]_0^{2\pi}\\
            &=2\pi^2.
        \end{aligned}
    \end{equation}
    
\end{example}

\begin{proposition}
    La circulation d'un champ de vecteurs le long d'un chemin ne dépend pas de la paramétrisation. En d'autres termes, si $\sigma_1$ et $\sigma_2$ sont deux chemins équivalents, alors
    \begin{equation}
        \int_{\sigma_1}F=\int_{\sigma_2}F.
    \end{equation}
\end{proposition}

\begin{proof}
    Soient deux chemins $\sigma_1\colon \mathopen[ a , b \mathclose]\to \eR^3$ et $\sigma_2\colon \mathopen[ c , d \mathclose]\to \eR^3$ équivalents, c'est à dire tels que
    \begin{equation}
        \sigma_1(t)=\sigma_2\big( \varphi(t) \big)
    \end{equation}
    où $\varphi\colon \mathopen[ a , b \mathclose]\to \mathopen[ c , d \mathclose]$ strictement croissante. En utilisant le fait que $\sigma_1(t)=\varphi'(t)\sigma_2'\big( \varphi(t) \big)$, nous avons
    \begin{equation}
        \begin{aligned}[]
            \int_{\sigma_1}F\cdot ds&=\int_a^bF\big( \sigma_1(t) \big)\cdot\sigma_1'(t)dt\\
            &=\int_a^bF\Big( \sigma_2\big( \varphi(t) \big) \Big)\cdot\sigma_2'\big( \varphi(t) \big)\varphi'(t)dt\\
            &=\int_{\varphi(a)}^{\varphi(b)}F\big( \sigma_2(s) \big)\cdot\sigma_2(s)ds\\
            &=\int_c^dF\big( \sigma_2(s) \big)\cdot \sigma_2'(s)ds\\
            &=\int_{\sigma_2}F\cdot ds.
        \end{aligned}
    \end{equation}
    où nous avons effectué le changement de variables $s=\varphi(t)$, $ds=\varphi'(t)dt$.
\end{proof}

\begin{remark}
    Si $\sigma_2$ est le chemin opposé de $\sigma$, alors
    \begin{equation}
        \int_{\sigma_2}F=-\int_{\sigma_1}F.
    \end{equation}
\end{remark}

%+++++++++++++++++++++++++++++++++++++++++++++++++++++++++++++++++++++++++++++++++++++++++++++++++++++++++++++++++++++++++++
\section{Circulation d'un champ conservatif}
%+++++++++++++++++++++++++++++++++++++++++++++++++++++++++++++++++++++++++++++++++++++++++++++++++++++++++++++++++++++++++++

Si nous avons une fonction scalaire $V\colon \eR^3\to \eR$, nous pouvons construire un champ de vecteur en prenant le gradient :
\begin{equation}
    F(x)=\nabla V(x).
\end{equation}
On dit que le champ de vecteur $F$ \defe{dérive}{champ dérivant d'un potentiel} de $V$, et on dit que $V$ est le \defe{potentiel}{potentiel} de $F$. Nous posons la définition suivante :
\begin{definition}
    Un champ de vecteurs $F\colon \eR^3\to \eR^3$ est un champ \defe{conservatif}{champ!conservatif} s'il existe une fonction $V\colon \eR^3\to \eR$ telle que
    \begin{equation}
        F(x)=\nabla V(x).
    \end{equation}
    Nous disons aussi parfois que le champ $V$ \emph{dérive d'un potentiel} ou bien qu'il s'agit d'un \emph{champ de gradient}.
\end{definition}

Les champs de vecteurs conservatifs sont particulièrement importants parce que presque toutes les forces connues en physiques dérivent d'un potentiel. Nous verrons que la terminologie «conservatif» provient du fait que les forces de ce type conservent l'énergie associée.


\begin{proposition}
    Considérons une fonction $V\colon \eR^3\to \eR$ (que nous appellerons \emph{potentiel}) et le champ de vecteur qui en dérive :
    \begin{equation}
        F=\nabla V.
    \end{equation}
    Alors 
    \begin{equation}
        \int_{\sigma}F\cdot ds=V\big( \sigma(b) \big)-V\big( \sigma(a) \big).
    \end{equation}
    Autrement dit, le travail nécessaires pour déplacer un objet d'un point à un autre dans un champ de force conservatif vaut la différence de potentiel entre le point de départ et le point d'arrivée.
\end{proposition}

\begin{proof}
    Par définition,
    \begin{equation}        \label{Eqintparddeftrav}
        \int_{\sigma} F\cdot ds=\int_a^b F\big( \sigma(t) \big)\cdot \sigma'(t)dt.
    \end{equation}
    Nous pouvons transformer l'intégrante de la façon suivante :
    \begin{equation}
        \begin{aligned}[]
            F\big( \sigma(t) \big)\cdot\sigma'(t)&=\nabla V\big( \sigma(t) \big)\cdot\sigma'(t)\\
            &=\frac{ \partial V }{ \partial x }\big( \sigma(t) \big)\sigma_x'(t) +\frac{ \partial V }{ \partial y }\big( \sigma(t) \big)\sigma_y'(t) +\frac{ \partial V }{ \partial z }\big( \sigma(t) \big)\sigma_z'(t)\\
            &=\frac{ d }{ dt }\Big[ V\big( \sigma(t) \big) \Big]
        \end{aligned}
    \end{equation}
    où nous avons posé
    \begin{equation}
        \sigma(t)=\begin{pmatrix}
            \sigma_x(t)    \\ 
            \sigma_y(t)    \\ 
            \sigma_z(t)    
        \end{pmatrix}
    \end{equation}
    et utilisé à l'envers la formule de dérivation de fonction composée pour
    \begin{equation}
             \frac{ d }{ dt }\Big[ V\big( \sigma(t) \big) \Big]=\Big( (V\circ\sigma)(t) \Big)'.
    \end{equation}
    En remettant ces expressions dans l'intégrale \eqref{Eqintparddeftrav},
    \begin{equation}
        \int_{\sigma}F\cdot ds=\int_a^b\frac{ d }{ dt }\Big[ V\big( \sigma(t) \big) \Big]dt=V\big( \sigma(b) \big)-V\big( \sigma(a) \big).
    \end{equation}
\end{proof}

\begin{example}
    Nous savons que le champ de gravitation dérive d'un potentiel. À la surface de la Terre, le potentiel de gravitation vu par une masse $m$ est donné par la fonction $V(x,y,z)=mgz$. Si nous voulons soulever cette masse d'une hauteur $h$, cela demandera toujours une énergie $mgh$, quel que soit le chemin suivit : en ligne droite vertical, en diagonal, en hélice, \ldots
\end{example}

\begin{example}
    À plus grande échelle, le champ de gravitation est encore un champ qui dérive d'un potentiel. En coordonnées sphériques,
    \begin{equation}
        V(\rho,\theta,\varphi)=k\frac{ m }{ \rho }
    \end{equation}
    Lorsqu'un satellite a une orbite de rayon $R$ autour la Terre, il reste sur la sphère $\rho=R$. Donc il reste sur une surface sur laquelle $V$ est constante. Il n'y a donc pas de travail de la force de gravitation ! C'est pour cela qu'un satellite peut tourner pendant des siècles sans apport énergétique.
\end{example}

\begin{example}
    Soit le champ de vecteurs
    \begin{equation}
        F\begin{pmatrix}
            x    \\ 
            y    
        \end{pmatrix}=\begin{pmatrix}
            y    \\ 
            x    
        \end{pmatrix}
    \end{equation}
    et le chemin
    \begin{equation}
        \sigma(t)=\begin{pmatrix}
            t^4/4    \\ 
            \sin^3(t\frac{ \pi }{2}).    
        \end{pmatrix}
    \end{equation}
    Nous voulons calculer la circulation de $F$ le long du chemin $\sigma$ entre $t=0$ et $t=1$.

    La première chose à voir est que $F=\nabla V$ avec $V(x,y)=xy$. Donc la circulation sera donnée par
    \begin{equation}
        \int_{\sigma}F\cdot ds=V\big( \sigma(1) \big)-V\big( \sigma(0) \big)=V\big( \frac{1}{ 4 },1 \big)-V(0,0)=\frac{1}{ 4 }-0=\frac{1}{ 4 }.
    \end{equation}
    Nous n'avons pas réellement calculé l'intégrale.
\end{example}

%+++++++++++++++++++++++++++++++++++++++++++++++++++++++++++++++++++++++++++++++++++++++++++++++++++++++++++++++++++++++++++
\section{Intégration de fonction à deux variables}
%+++++++++++++++++++++++++++++++++++++++++++++++++++++++++++++++++++++++++++++++++++++++++++++++++++++++++++++++++++++++++++

%---------------------------------------------------------------------------------------------------------------------------
\subsection{Intégration sur un domaine rectangulaire}
%---------------------------------------------------------------------------------------------------------------------------
\label{PgRapIntMultFubiniRect}

Soit une fonction positive
\begin{equation}
    \begin{aligned}
        f\colon \mathopen[ a , b \mathclose]\times\mathopen[ c , d \mathclose]&\to \eR^+ \\
        (x,y)&\mapsto f(x,y). 
    \end{aligned}
\end{equation}

L'intégrale de $f$ sur le rectangle $\mathopen[ a , b \mathclose]\times\mathopen[ c , d \mathclose]$ est le volume sous le graphe de la fonction. C'est à dire le volume de l'ensemble
\begin{equation}
    \{ (x,y,z)\tq (x,y)\in\mathopen[ a , b \mathclose]\times\mathopen[ c , d \mathclose], z\leq f(x,y) \}.
\end{equation}

\begin{theorem}[Théorème de Fubini]
    Soit une fonction $f\colon \eR^2\to \eR$ une fonction continue par morceaux sur $\mR=\mathopen[ a , b \mathclose]\times\mathopen[ c , d \mathclose]$. Alors
    \begin{equation}
        \int_{\mR}f(x,y)dxdy=\int_a^b\left[ \int_c^df(x,y)dy \right]dx=\int_c^d\left[ \int_a^bf(x,y)dx \right]dy.
    \end{equation}
\end{theorem}
\index{théorème!Fubini!version compacte dans \( \eR^2\)}

En pratique, nous utilisons le théorème de Fubini pour calculer les intégrales sur des rectangles.

\begin{example}
    Nous voudrions intégrer la fonction $f(x,y)-4+x^2+y^2$ sur le rectangle de la figure \ref{LabelFigVNBGooSqMsGU}. % From file VNBGooSqMsGU
\newcommand{\CaptionFigVNBGooSqMsGU}{Intégration sur un rectangle}
\input{auto/pictures_tex/Fig_VNBGooSqMsGU.pstricks}

    L'ensemble sur lequel nous intégrons est donné par le produit cartésien d'intervalles $E=[0,1]\times[0,2]$. Le théorème de Fubini montre que nous pouvons intégrer séparément sur l'intervalle horizontal et vertical :
    \begin{equation}
    	\int_{E=[0,1]\times[0,2]}f=\int_{[0,1]}\left( \int_{[0,2]}(4-x^2-y^2)dy \right)dx.
    \end{equation}
    Ces intégrales sont maintenant des intégrales usuelles qui s'effectuent en calculant des primitives :
    \begin{equation}
        \begin{aligned}[]
            \int_0^1\int_0^2(4-x^2-y^2)dy\,dx&=\int_0^1\left[ 4y-x^2y-\frac{ y^3 }{ 3 } \right]_0^2dx\\
            &=\int_0^1(8-2x^2-\frac{ 8 }{ 3 })dx\\
            &=\left[ \frac{ 16x }{ 3 }-\frac{ 2x^3 }{ 3 } \right]_0^1\\
            &=\frac{ 14 }{ 3 }.
        \end{aligned}
    \end{equation}
    Avec Sage, on peut faire comme ceci :

    \begin{verbatim}
----------------------------------------------------------------------
| Sage Version 4.6.1, Release Date: 2011-01-11                       |
| Type notebook() for the GUI, and license() for information.        |
----------------------------------------------------------------------
sage: f(x,y)=4-x**2-y**2                  
sage: f.integrate(y,0,2).integrate(x,0,1)
(x, y) |--> 14/3

    \end{verbatim}

\end{example}

%---------------------------------------------------------------------------------------------------------------------------
\subsection{Intégration sur un domaine non rectangulaire}
%---------------------------------------------------------------------------------------------------------------------------
\label{PgRapIntMultFubiniTri}

Nous voulons maintenant intégrer la fonction $f(x,y)=x^2+y^2$ sur le triangle de la figure \ref{LabelFigCURGooXvruWV}. % From file CURGooXvruWV
\newcommand{\CaptionFigCURGooXvruWV}{Intégration sur un triangle}
\input{auto/pictures_tex/Fig_CURGooXvruWV.pstricks}

Étant donné que $y$ varie de $0$ à $2$ et que \emph{pour chaque $y$}, la variable $x$ varie de $0$ à $y$, nous écrivons l'intégrale sur le triangle sous la forme :
\begin{equation}
	\int_{\text{triangle}}(x^2+y^2)dx dy=\int_0^2\left( \int_0^y(x^2+y^2)dx \right)dy.
\end{equation}

Il existe principalement deux types de domaines non rectangulaires : les «horizontaux» et les «verticaux», voir figure \ref{LabelFigHCJPooHsaTgI}. % From file HCJPooHsaTgI
\newcommand{\CaptionFigHCJPooHsaTgI}{Deux types de surfaces. Nous avons tracé un rectangle qui contient chacune des deux surfaces. L'intégrale sur un domaine sera l'intégrale sur le rectangle de la fonction qui vaut zéro en dehors du domaine.}
\input{auto/pictures_tex/Fig_HCJPooHsaTgI.pstricks}

Les surfaces horizontales sont de la forme 
\begin{equation}
    D=\{ (x,y)\tq x\in\mathopen[ a , b \mathclose],\varphi_1(x)\leq y\leq \varphi_2(x) \}
\end{equation}
où $\varphi_1$ et $\varphi_2$ sont les deux fonctions qui bornent le domaine. Le domaine $D$ est la région comprise entre les graphes de $\varphi_1$ et $\varphi_2$. Pour un tel domaine nous avons
\begin{equation}
    \iint_Df(x,y)dxdy=\int_a^bdx\int_{\varphi_1(x)}^{\varphi_2(x)}f(x,y)dy.
\end{equation}

Les surfaces verticales sont de la forme 
\begin{equation}
    D=\{ (x,y)\tq y\in\mathopen[ c , d \mathclose],\psi_1(y)\leq x\leq \psi_2(y) \}
\end{equation}
où $\varphi_1$ et $\varphi_2$ sont les deux fonctions qui bornent le domaine. Le domaine $D$ est la région comprise entre les graphes de $\varphi_1$ et $\varphi_2$. Dans ces cas nous avons
\begin{equation}
    \iint_Df=\int_c^d dy\int_{\psi_1(y)}^{\psi_2(y)} f(x,y)dx.
\end{equation}

\begin{proposition}
    L'aire du domaine $D$ vaut l'intégrable de la fonction $f(x,y)=1$ sur $D$ :
    \begin{equation}
        Aire(D)=\iint_Ddxdy.
    \end{equation}
\end{proposition}

\begin{proof}
    Supposons que le domaine soit du type «horizontal». En utilisant le théorème de Fubini avec $f(x,y)=1$ nous avons
    \begin{equation}
        \iint_Ddxdy=\int_a^b\left[ \int_{\varphi_1(x)}^{\varphi_2(x)}dy \right]dx=\int_a^b\big[ \varphi_2(x)-\varphi_1(x) \big].
    \end{equation}
    Cela représente l'aire sous $\varphi_2$ moins l'aire sous $\varphi_1$, et par conséquent l'aire contenue entre les deux.
\end{proof}

\begin{example}
    Cherchons la surface du disque de centre $(0,0)$ et de rayon $1$ dessinée à la figure \ref{LabelFigCMMAooQegASg}. % From file CMMAooQegASg
\newcommand{\CaptionFigCMMAooQegASg}{En bleu, la fonction $\sqrt{r^2-x^2}$ et en rouge, la fonction $-\sqrt{r^2-x^2}$.}
\input{auto/pictures_tex/Fig_CMMAooQegASg.pstricks}

    Le domaine est donné par $\varphi_1(x)\leq y\leq \varphi_2(x)$ et $x\in\mathopen[ -r ,r \mathclose]$ où $\varphi_1(x)=-\sqrt{r^2-x^2}$ et $\varphi_2(x)=\sqrt{r^2-x^2}$. L'aire est donc donnée par
    \begin{equation}
        A=\int_{-r}^r\big[ \varphi_2(x)-\varphi_1(x) \big]dx=2\int_{-r}^r\sqrt{r^2-x^2}dx=4\int_0^r\sqrt{r^2-x^2}.
    \end{equation}
    Nous effectuons le premier changement de variables $x=ru$, donc $dx=rdu$. En ce qui concerne les bornes, si $x=0$, alors $u=0$ et si $x=r$, alors $u=1$. L'intégrale à calculer devient
    \begin{equation}
        A=4\int_0^1\sqrt{r^2-r^2u^2}rdu=4r^2\int_0^1\sqrt{1-u^2}du.
    \end{equation}
    Cette dernière intégrale se calcule en posant
    \begin{equation}
        \begin{aligned}[]
            u&=\sin(t)&du&=\cos(t)dt\\
            u&=0&t&=0\\
            u&=1&t&=\pi/2.
        \end{aligned}
    \end{equation}
    Nous avons
    \begin{equation}
        A=4r^2\int_0^{\pi/2}\sqrt{1-\sin^2(t)}\cos(t)dt=4r^2\int_0^{\pi/2}\cos^2(t)dt.
    \end{equation}
    En utilisant la formule $2\cos^2(x)=1+\cos(2x)$, nous avons
    \begin{equation}
        A=4r^2\int_0^{\pi/2}\frac{ 1+\cos(2t) }{ 2 }dt=\pi r^2.
    \end{equation}
\end{example}

%---------------------------------------------------------------------------------------------------------------------------
\subsection{Changement de variables}
%---------------------------------------------------------------------------------------------------------------------------

Comme dans les intégrales simples, il y a souvent moyen de trouver un changement de variables qui simplifie les expressions.  Le domaine $E=\{ (x,y)\in\eR^2\tq x^2+y^2<1 \}$ par exemple s'écrit plus facilement $E=\{ (r,\theta)\tq r<1 \}$ en coordonnées polaires. Le passage aux coordonnées polaire permet de transformer une intégration sur un domaine rond à une intégration sur le domaine rectangulaire $\mathopen]0,2\pi\mathclose[\times\mathopen]0,1\mathclose[$. La question est évidement de savoir si nous pouvons écrire
\begin{equation}
	\int_Ef=\int_{0}^{2\pi}\int_0^1f(r\cos\theta,r\sin\theta)drd\theta.
\end{equation}
Hélas ce n'est pas le cas. Il faut tenir compte du fait que le changement de base dilate ou contracte certaines surfaces.

Soit $\varphi\colon D_1\subset\eR^2\to D_2\subset \eR^2$ une fonction bijective de classe $C^1$ dont l'inverse est également de classe $C^1$. On désigne par $x$ et $y$ ses composantes, c'est à dire que
\begin{equation}
    \varphi(u,v)=\begin{pmatrix}
        x(u,v)    \\ 
        y(u,v)    
    \end{pmatrix}
\end{equation}
avec $(u,v)\in D_1$.

\begin{theorem}     \label{ThoChamDeVarIntDDf}
    Soit une fonction continue $f\colon D_2\to \eR$. Alors
    \begin{equation}
        \iint_{\varphi(D_1)}f(x,y)dxdy=\iint_{D_1}f\big( x(u,v),y(u,v) \big)| J_{\varphi}(u,v) |dudv
    \end{equation}
    où $J_{\varphi}$ est le Jacobien de $\varphi$.
\end{theorem}
Pour rappel,
\begin{equation}
    J_{\varphi}(u,v)=\det\begin{pmatrix}
        \frac{ \partial x }{ \partial u }    &   \frac{ \partial x }{ \partial v }    \\ 
        \frac{ \partial y }{ \partial u }    &   \frac{ \partial u }{ \partial v }    
    \end{pmatrix}.
\end{equation}
Ne pas oublier de prendre la valeur absolue lorsqu'on utilise le Jacobien dans un changement de variables.

%///////////////////////////////////////////////////////////////////////////////////////////////////////////////////////////
\subsubsection{Le cas des coordonnées polaires}
%///////////////////////////////////////////////////////////////////////////////////////////////////////////////////////////

La fonction qui donne les coordonnées polaires est
\begin{equation}
    \begin{aligned}
        \varphi\colon \eR^+\times\mathopen] 0 , 2\pi \mathclose[&\to \eR^2 \\
        (r,\theta)&\mapsto\begin{pmatrix}
            r\cos(\theta)    \\ 
            r\sin(\theta)    
        \end{pmatrix}.
    \end{aligned}
\end{equation}
Son Jacobien vaut
\begin{equation}
    J_{\varphi}(r,\theta)=\det\begin{pmatrix}
        \frac{ \partial x(r,\theta) }{ \partial r }    &   \frac{ \partial x(r,\theta) }{ \partial \theta }    \\ 
        \frac{ \partial y(r,\theta) }{ \partial r }    &   \frac{ \partial y(r,\theta) }{ \partial \theta }    
    \end{pmatrix}=
    \begin{vmatrix}
        \cos(\theta)    &   -r\sin(\theta)    \\ 
        \sin(\theta)    &   r\cos(\theta)    
    \end{vmatrix}=r.
\end{equation}

\begin{example}
    Calculons la surface du disque $D$ de rayon $R$. Nous devons calculer
    \begin{equation}
        \iint_Ddxdy.
    \end{equation}
    Pour passer au polaires, nous savons que le disque est décrit par 
    \begin{equation}
        D=\{ (r,\theta)\tq 0\leq r\leq R,0\leq\theta\leq 2\pi \}.
    \end{equation}
    Nous avons donc
    \begin{equation}
        \iint_Ddxdy=\iint_{D}r\,drd\theta=\int_0^{2\pi}\int_0^Rr\,drd\theta=2\pi\int_0^Rr\,dr=\pi R^2.
    \end{equation}
\end{example}

\begin{example}     \label{ExpmfDtAtV}
    Montrons comment intégrer la fonction $f(x,y)=\sqrt{1-x^2-y^2}$ sur le domaine délimité par la droite $y=x$ et le cercle $x^2+y^2=y$, représenté sur la figure \ref{LabelFigHFAYooOrfMAA}. Pour trouver le centre et le rayon du cercle $x^2+y^2=y$, nous commençons par écrire $x^2+y^2-y=0$, et ensuite nous reformons le carré : $y^2-y=(y-\frac{ 1 }{2})^2-\frac{1}{ 4 }$.

\newcommand{\CaptionFigHFAYooOrfMAA}{Passage en polaire pour intégrer sur un morceau de cercle.}
\input{auto/pictures_tex/Fig_HFAYooOrfMAA.pstricks}

    Le passage en polaire transforme les équations du bord du domaine en
    \begin{equation}
        \begin{aligned}[]
            \cos(\theta)&=\sin(\theta)\\
            r^2&=r\sin(\theta).
        \end{aligned}
    \end{equation}
    L'angle $\theta$ parcours donc $\mathopen] 0 , \pi/4 \mathclose[$, et le rayon, pour chacun de ces $\theta$ parcours $\mathopen] 0 , \sin(\theta) \mathclose[$. La fonction à intégrer se note maintenant $f(r,\theta)=\sqrt{1-r^2}$. Donc l'intégrale à calculer est
    \begin{equation}		\label{PgOMRapIntMultFubiniBoutCercle}
        \int_{0}^{\pi/4}\left( \int_0^{\sin(\theta)}\sqrt{1-r^2}r\,rd \right).
    \end{equation}
    Remarquez la présence d'un $r$ supplémentaire pour le jacobien.

    Notez que les coordonnées du point $P$ sont $(1,1)$.
\end{example}

En pratique, lors du passage en coordonnées polaires, le «$dxdy$» devient «$r\,drd\theta$».

%///////////////////////////////////////////////////////////////////////////////////////////////////////////////////////////
\subsubsection{Les coordonnées cylindriques}
%///////////////////////////////////////////////////////////////////////////////////////////////////////////////////////////

En ce qui concerne les coordonnées cylindriques, le Jacobien est donné par
\begin{equation}
    J(r,\theta,z)=\begin{vmatrix}
        \frac{ \partial x }{ \partial r }    &   \frac{ \partial x }{ \partial \theta }    &   \frac{ \partial x }{ \partial z }    \\
        \frac{ \partial y }{ \partial r }    &   \frac{ \partial y }{ \partial \theta }    &   \frac{ \partial y }{ \partial z }    \\
        \frac{ \partial z }{ \partial r }    &   \frac{ \partial z }{ \partial \theta }    &   \frac{ \partial z }{ \partial z }    
    \end{vmatrix}=
    \begin{vmatrix}
        \cos\theta    &   -r\sin\theta    &   0    \\
        \sin\theta    &   r\cos\theta    &   0    \\
        0    &   0    &   1
    \end{vmatrix}=r.
\end{equation}
Nous avons donc $dx\,dy\,dz=r\,dr\,d\theta\,dz$.

%///////////////////////////////////////////////////////////////////////////////////////////////////////////////////////////
\subsubsection{Coordonnées sphériques}
%///////////////////////////////////////////////////////////////////////////////////////////////////////////////////////////

Le calcul est un peu plus long :
\begin{equation}
    \begin{aligned}[]
        J(\rho,\theta,\varphi)&=\begin{vmatrix}
            \frac{ \partial x }{ \partial \rho }    &   \frac{ \partial x }{ \partial \theta }    &   \frac{ \partial x }{ \partial \varphi }    \\
            \frac{ \partial y }{ \partial \rho }    &   \frac{ \partial y }{ \partial \theta }    &   \frac{ \partial y }{ \partial \varphi }    \\
            \frac{ \partial z }{ \partial \rho }    &   \frac{ \partial z }{ \partial \theta }    &   \frac{ \partial z }{ \partial \varphi }    
        \end{vmatrix}\\ 
        &=
        \begin{vmatrix}
            \sin\theta\cos\varphi    &   \rho\cos\theta\cos\varphi    &   -\rho\sin\theta\sin\varphi    \\
            \sin\theta\sin\varphi    &   \rho\cos\theta\sin\varphi    &   -\rho\sin\theta\cos\varphi    \\
            \cos\theta               &   -\rho\sin\theta              &   0
        \end{vmatrix}\\
        &=\rho^2\sin\theta.
    \end{aligned}
\end{equation}
Donc 
\begin{equation}
    dx\,dy\,dz=\rho^2\sin(\theta)\,d\rho\,d\theta\,d\varphi.
\end{equation}

%///////////////////////////////////////////////////////////////////////////////////////////////////////////////////////////
\subsubsection{Un autre système utile}
%///////////////////////////////////////////////////////////////////////////////////////////////////////////////////////////

Un changement de variables que l'on voit assez souvent est
\begin{subequations}
    \begin{numcases}{}
        u=x+y\\
        v=x-y.
    \end{numcases}
\end{subequations}
Afin de calculer son jacobien, il faut d'abord exprimer $x$ et $y$ en fonctions de $u$ et $v$ :
\begin{subequations}
    \begin{numcases}{}
        x=(u+v)/2\\
        y=(u-v)/2.
    \end{numcases}
\end{subequations}
La matrice jacobienne est
\begin{equation}
    \begin{pmatrix}
        \frac{ \partial x }{ \partial u }    &   \frac{ \partial x }{ \partial v }    \\ 
        \frac{ \partial y }{ \partial u }    &   \frac{ \partial y }{ \partial v }    
    \end{pmatrix}=
    \begin{pmatrix}
        \frac{ 1 }{2}    &   \frac{ 1 }{2}    \\ 
        \frac{ 1 }{2}    &   -\frac{ 1 }{2}    
    \end{pmatrix}.
\end{equation}
Le déterminant vaut $-\frac{1}{ 2 }$. Nous avons donc
\begin{equation}
    dxdy=\frac{ 1 }{2}dudv.
\end{equation}
Nous insistons sur le fait que c'est $\frac{ 1 }{2}$ et non $-\frac{ 1 }{2}$ qui intervient parce que que la formule du changement de variable demande d'introduire la \emph{valeur absolue} du jacobien.

\begin{example}
    Calculer l'intégrale de la fonction $f(x,y)=x^2-y^2$ sur le domaine représenté sur la figure \ref{LabelFigVWFLooPSrOqz}. % From file VWFLooPSrOqz
\newcommand{\CaptionFigVWFLooPSrOqz}{Un domaine qui s'écrit étonnament bien avec un bon changement de coordonnées.}
\input{auto/pictures_tex/Fig_VWFLooPSrOqz.pstricks}
    
    Les droites qui délimitent le domaine d'intégration sont
    \begin{equation}
        \begin{aligned}[]
            y&=-x+2\\
            y&=x-2\\
            y&=x\\
            y&=-x
        \end{aligned}
    \end{equation}
    Le domaine est donc donné par les équations
    \begin{subequations}
        \begin{numcases}{}
            y+x<2\\
            y-x>-2\\
            y-x<0 \\
            y+x>0.
        \end{numcases}
    \end{subequations}
    En utilisant le changement de variables $u=x+y$, $v=x-y$ nous trouvons le domaine $0<u<2$, $0<v<2$. En ce qui concerne la fonction, $f(x,y)=(x+y)(x-y)$ et par conséquent
    \begin{equation}
        f(u,v)=uv.
    \end{equation}
    L'intégrale à calculer est simplement
    \begin{equation}
        \int_0^2\int_0^2 uv\,dudv=\int_0^2 u\,du\left[ \frac{ v^2 }{ 2 } \right]_0^2=2\int_0^2u\,du=4.
    \end{equation}
    
\end{example}




%+++++++++++++++++++++++++++++++++++++++++++++++++++++++++++++++++++++++++++++++++++++++++++++++++++++++++++++++++++++++++++
\section{Les intégrales triples}
%+++++++++++++++++++++++++++++++++++++++++++++++++++++++++++++++++++++++++++++++++++++++++++++++++++++++++++++++++++++++++++

Les intégrales triples fonctionnent exactement de la même manière que les intégrales doubles. Il s'agit de déterminer sur quelle domaine les variables varient et d'intégrer successivement par rapport à $x$, $y$ et $z$. Il est autorisé de permuter l'ordre d'intégration\footnote{En toute rigueur, cela n'est pas vrai, mais nous ne considérons seulement des cas où cela est autorisé.} à condition d'adapter les domaines d'intégration. 

\begin{example}
    Soit le domaine parallélépipédique rectangle 
    \begin{equation}
        R=\mathopen[ 0 , 1 \mathclose]\times \mathopen[ 1 , 2 \mathclose]\times\mathopen[ 0 , 4 \mathclose].
    \end{equation}
    Pour intégrer la fonction $f(x,y,z)=x^2y\sin(z)$ sur $R$, nous faisons
    \begin{equation}
        \begin{aligned}[]
            I&=\iiint_Rx^2y\sin(z)\,dxdydz\\
            &=\int_0^1dx\int_1^2dy\int_0^4x^2y\sin(z)dz\\
            &=\int_0^1dx\int_1^2 x^2y(1-\cos(4))dy\\
            &=\int_0^1\frac{ 3 }{2}(1-\cos(4))x^2dx\\
            &=\frac{ 1 }{2}\big( 1-\cos(4) \big).
        \end{aligned}
    \end{equation}
    
    \begin{verbatim}
----------------------------------------------------------------------
| Sage Version 4.6.1, Release Date: 2011-01-11                       |
| Type notebook() for the GUI, and license() for information.        |
----------------------------------------------------------------------
sage: f(x,y,z)=x**2*y*sin(z)                                                                                                                                                            
sage: f.integrate(x,0,1).integrate(y,1,2).integrate(z,0,4)                                                                                                                               
(x, y, z) |--> -1/2*cos(4) + 1/2
    \end{verbatim}
\end{example}


\begin{example}
    Soit $D$ la région délimitée par le plan $x=0$, $y=0$, $z=2$ et la surface d'équation
    \begin{equation}
        z=x^2+y^2.
    \end{equation}
    Cherchons à calculer $\iiint_Dx\,dx\,dy\,dz$. Ici, un dessin indique que le volume considéré est $z\geq x^2+y^2$. Il y a plusieurs façon de décrire cet ensemble. Une est celle-ci :
    \begin{equation}
        \begin{aligned}[]
            z&\colon 0\to 2\\
            x&\colon 0\to \sqrt{z}\\
            y&\colon 0\to \sqrt{z-x^2}.
        \end{aligned}
    \end{equation}
    Cela revient à dire que $z$ peut prendre toutes les valeurs de $0$ à $2$, puis que pour chaque $z$, la variable $x$ peut aller de $0$ à $\sqrt{z}$, mais que pour chaque $z$ et $x$ fixés, la variable $y$ ne peut pas dépasser $\sqrt{z-x^2}$. En suivant cette méthode, l'intégrale à calculer est
    \begin{equation}
        \int_0^2dz\int_0^{\sqrt{z}}dx\int_0^{\sqrt{z-x^2}}f(x,y,z)dy.
    \end{equation}
    \begin{verbatim}
----------------------------------------------------------------------
| Sage Version 4.6.1, Release Date: 2011-01-11                       |
| Type notebook() for the GUI, and license() for information.        |
----------------------------------------------------------------------
sage: f(x,y,z)=x
sage: assume(z>0)
sage: assume(z-x**2>0)
sage: f.integrate(y,0,sqrt(z-x**2)).integrate(x,0,sqrt(z)).integrate(z,0,2)
(x, y, z) |--> 8/15*sqrt(2)
    \end{verbatim}
    Notez qu'il a fallu aider Sage en lui indiquant que $z>0$ et $z-x^2>0$.

    Une autre paramétrisation serait
    \begin{equation}
        \begin{aligned}[]
            x&\colon 0\to \sqrt{2}\\
            y&\colon 0\to \sqrt{2-x^2}\\
            z&\colon x^2+y^2\to 2.
        \end{aligned}
    \end{equation}
    \begin{verbatim}
----------------------------------------------------------------------
| Sage Version 4.6.1, Release Date: 2011-01-11                       |
| Type notebook() for the GUI, and license() for information.        |
----------------------------------------------------------------------
sage: f(x,y,z)=x
sage: assume(2-x**2>0)
sage: f.integrate(y,0,sqrt(z-x**2)).integrate(x,0,sqrt(z)).integrate(z,0,2)
(x, y, z) |--> 8/15*sqrt(2)
    \end{verbatim}

    Écrivons le détail de cette dernière intégrale :
    \begin{equation}
        \begin{aligned}[]
            I&=\int_0^{\sqrt{2}}dx\int_0^{\sqrt{2-x^2}}dy\int_{x^2+y^2}^2xdz\\
            &=\int_0^{\sqrt{2}}dx\int_0^{\sqrt{2-x^2}}x(2-x^2-y^2)dy\\
            &=\int_0^{\sqrt{2}}dx\,x\left[ (2-x^2)y-\frac{ y^3 }{ 3 } \right]_0^{\sqrt{2-x^2}}\\
            &=\int_0^{\sqrt{2}}\frac{ 2 }{ 3 }x(2-x^2)^{3/2}dx.
        \end{aligned}
    \end{equation}
    Ici nous effectuons le changement de variable $u=x^2$, $du=2xdx$. Ne pas oublier de changer les bornes de l'intégrale :
    \begin{equation}
        I=\frac{1}{ 3 }\int_0^2(2-u)^{3/2}du.
    \end{equation}
    Le changement de variable $t=2-u$, $dt=-du$ fait venir (attention aux bornes !!)
    \begin{equation}
        I=-\frac{1}{ 3 }\int_2^0t^{3/2}dt=\frac{1}{ 3 }\left[ \frac{ t^{5/2} }{ 5/2 } \right]_0^2=\frac{ 8 }{ 15 }\sqrt{2}.
    \end{equation}
       
\end{example}

%---------------------------------------------------------------------------------------------------------------------------
\subsection{Volume}
%---------------------------------------------------------------------------------------------------------------------------

Parmi le nombreuses interprétations géométriques de l'intégrale triple, notons celle-ci :
\begin{proposition}
    Soit $D\subset \eR^3$. Le volume de $D$ est donné par 
    \begin{equation}
        Vol(D)=\iiint_D dxdydz.
    \end{equation}
    C'est à dire l'intégrale de la fonction $f(x,y,z)=1$ sur $D$.
\end{proposition}
Suivant les points de vue, cette proposition peut être considérée comme une \emph{définition}] du volume.

\begin{example}     \label{ExemVolSphCart}
    Calculons le volume de la sphère de rayon $R$. Le domaine de variation des variables $x$, $y$ et $z$ pour la sphère est
    \begin{equation}
        \begin{aligned}[]
            x&\colon -R\to R\\
            y&\colon -\sqrt{R^2-x^2}\to \sqrt{R^2-x^2}\\
            z&\colon -\sqrt{R^2-x^2-y^2}\to \sqrt{R^2-x^2-y^2}.
        \end{aligned}
    \end{equation}
    Par conséquent nous devons calculer l'intégrale
    \begin{equation}
        V=\int_{-R}^Rdx\int_{-\sqrt{R^2-x^2}}^{\sqrt{R^2-x^2}}dy\int_{-\sqrt{R^2-x^2-y^2}}^{\sqrt{R^2-x^2-y^2}}dz.
    \end{equation}
    La première intégrale est simple :
    \begin{equation}
        V=2\int_{-R}^Rdx\int_{-\sqrt{R^2-x^2}}^{\sqrt{R^2-x^2}}\sqrt{R^2-x^2-y^2}dy.
    \end{equation}
    Afin de simplifier la notation, nous posons $a=R^2-x^2$. Ceci n'est pas un changement de variables : juste une notation provisoire le temps d'effectuer l'intégration sur $y$. Étudions donc
    \begin{equation}
        I=\int_{-\sqrt{a}}^{\sqrt{a}}\sqrt{a-y^2}dy,
    \end{equation}
    ce qui est la surface du demi-disque de rayon $\sqrt{a}$. Nous avons donc
    \begin{equation}
        I=\frac{ \pi a }{ 2 }=\frac{ \pi }{ 2 }(R^2-x^2),
    \end{equation}
    et
    \begin{equation}
        V=2\int_{-R}^R\frac{ \pi }{ 2 }(R^2-x^2)dx=\pi\left[ R^2x-\frac{ x^3 }{ 3 } \right]_{-R}^R=\frac{ 4 }{ 3 }\pi R^3.
    \end{equation}    
\end{example}

\begin{example}
    Nous pouvons calculer le volume de la sphère en utilisant les coordonnées sphériques. Les bornes des variables pour la sphère de rayon $R$ sont
    \begin{equation}
        \begin{aligned}[]
            \rho&\colon 0\to R\\
            \theta&\colon 0\to \pi\\
            \varphi&\colon 0\to 2\pi.
        \end{aligned}
    \end{equation}
    En n'oubliant pas le jacobien $\rho^2\sin(\theta)$, l'intégrale à calculer est
    \begin{equation}
        V=\int_0^Rd\rho\int_0^{2\pi}d\varphi\int_0^{\pi}\rho^2\sin(\theta)d\theta
    \end{equation}
    L'intégrale sur $\varphi$ fait juste une multiplication par $2\pi$. Celle sur $\rho$ vaut
    \begin{equation}
        \int_0^R\rho^2d\rho=\frac{ R^3 }{ 3 }.
    \end{equation}
    L'intégrale sur $\theta$ donne
    \begin{equation}
        \int_0^{\pi}\sin(\theta)d\theta=[-\cos(\theta)]_{0}^{\pi}=2.
    \end{equation}
    Le tout fait par conséquent
    \begin{equation}
        V=\frac{ 4 }{ 3 }\pi R^3.
    \end{equation}
    Sans contestes, le passage aux coordonnées sphériques a considérablement simplifié le calcul par rapport à celui de l'exemple \ref{ExemVolSphCart}.
\end{example}


%+++++++++++++++++++++++++++++++++++++++++++++++++++++++++++++++++++++++++++++++++++++++++++++++++++++++++++++++++++++++++++
\section{Un petit peu plus formel}
%+++++++++++++++++++++++++++++++++++++++++++++++++++++++++++++++++++++++++++++++++++++++++++++++++++++++++++++++++++++++++++

%---------------------------------------------------------------------------------------------------------------------------
\subsection{Intégration sur un domaine non rectangulaire}
%---------------------------------------------------------------------------------------------------------------------------

\newcommand{\CaptionFigPONXooXYjEot}{Intégrer sur des domaines plus complexes.}
\input{auto/pictures_tex/Fig_PONXooXYjEot.pstricks}

La méthode de Fubini ne fonctionne plus sur un domaine non rectangulaire tel que celui de la figure \ref{LabelFigPONXooXYjEot}. Nous allons donc utiliser une astuce. Considérons le domaine \begin{equation}
	E=\{ (x,y)\in\eR^2\tq a<x<b\text{ et } \alpha(x)<y<\beta(x) \}
\end{equation}
représenté sur la figure \ref{LabelFigPONXooXYjEot}. Nous considérons la fonction
\begin{equation}
	\tilde f(x,y)=\begin{cases}
	f(x,y)	&	\text{si }(x,y)\in E\\
	0	&	 \text{sinon.}
\end{cases}
\end{equation}
Ensuite intégrons $\tilde f$ sur un rectangle qui englobe la surface à intégrer à l'aide de Fubini. Étant donné que $\tilde f=f$ sur la surface et que $\tilde f$ est nulle en dehors, nous avons
\begin{equation}
	\int_Ef=\int_E\tilde f=\int_{\text{rectangle}}\tilde f=\int_a^b\left( \int_{\alpha(x)}^{\beta(x)}f(x,y)dy \right)dx.
\end{equation}

Dans le cas de l'intégrale de $f(x,y)=x^2+y^2$ sur le triangle de la figure \ref{LabelFigCURGooXvruWV}, nous avons
\begin{equation}
	\int_{\text{triangle}}(x^2+y^2)dx dy=\int_0^2\left( \int_0^y(x^2+y^2)dx \right)dy.
\end{equation}

\begin{remark}
    Le nombre $\iint_{D}f(x,y)dxdy$ ne dépend pas du choix du rectangle englobant $D$.
\end{remark}

En pratique, nous calculons l'intégrale en utilisant une extension du théorème de Fubini :
\begin{theorem}
    Soit $f\colon D\subset\eR^2\to \eR$ une fonction continue où $D$ est un domaine de type vertical ou horizontal.
    \begin{enumerate}
        \item
            Si $D$ est vertical, alors
            \begin{equation}
                \iint_Df=\int_a^b\left[ \int_{\varphi_1(x)}^{\varphi_2(x)}f(x,y)dy \right]dx.
            \end{equation}
        \item
            Si $D$ est horizontal, alors
            \begin{equation}
                \iint_Df=\int_c^d\left[ \int_{\psi_1(y)}^{\psi_2(y)}f(x,y)dx \right]dy.
            \end{equation}
    \end{enumerate}
    
\end{theorem}

%---------------------------------------------------------------------------------------------------------------------------
\subsection{Changement de variables}
%---------------------------------------------------------------------------------------------------------------------------


Le théorème du changement de variable est le suivant.
\begin{theorem}
Soit $g\colon A\to B$ un difféomorphisme. Soient $F\subset B$ un ensemble mesurable et borné et $f\colon F\to \eR$ une fonction bornée et intégrable. Supposons que $g^{-1}(F)$ soit borné et que $Jg$ soit borné sur $g^{-1}(F)$. Alors
\begin{equation}
    \int_Ff(x)dy=\int_{g^{-1}(F)}f\big( g(x) \big)| Jg(x) |dx
\end{equation}
\end{theorem}
Pour rappel, $Jg$ est le déterminant de la matrice \href{http://fr.wikipedia.org/wiki/Matrice_jacobienne}{jacobienne} (aucun lien de \href{http://fr.wikipedia.org/wiki/Jacob}{parenté}) donnée par
\begin{equation}
	Jg=\det\begin{pmatrix}
	\partial_xg_1	&	\partial_yg_1	\\ 
	\partial_xg_2	&	\partial_tg_2	
\end{pmatrix}.
\end{equation}
Un \defe{difféomorphisme}{difféomorphisme} est une application $g\colon A\to B$ telle que $g$ et $g^{-1}\colon B\to A$ soient de classe $C^1$.

%///////////////////////////////////////////////////////////////////////////////////////////////////////////////////////////
					\subsubsection{Coordonnées polaires}
%///////////////////////////////////////////////////////////////////////////////////////////////////////////////////////////

Les coordonnées polaires sont données par le difféomorphisme
\begin{equation}
	\begin{aligned}
		g\colon \mathopen]0,\infty\mathclose[\times\mathopen]0,2\pi\mathclose[ &\to\eR^2\setminus D\\
		(r,\theta)&\mapsto \big( r\cos(\theta),r\sin(\theta) \big)
	\end{aligned}
\end{equation}
où $D$ est la demi droite $y=0$, $x\geq 0$. Le fait que les coordonnées polaires ne soient pas un difféomorphisme sur tout $\eR^2$ n'est pas un problème pour l'intégration parce que le manque de difféomorphisme est de mesure nulle dans $\eR^2$. Le jacobien est donné par
\begin{equation}
	Jg=\det\begin{pmatrix}
	\partial_rx	&	\partial_{\theta}x	\\ 
	\partial_ry	&	\partial_{\theta}y
\end{pmatrix}=\det\begin{pmatrix}
	\cos(\theta)	&	-r\sin(\theta)	\\ 
	\sin(\theta)	&	r\cos(\theta)	
\end{pmatrix}=r.
\end{equation}

%///////////////////////////////////////////////////////////////////////////////////////////////////////////////////////////
					\subsubsection{Coordonnées sphériques}
%///////////////////////////////////////////////////////////////////////////////////////////////////////////////////////////
\label{SubSubCoordSpJxhMwm}

Les coordonnées sphériques sont données par
\begin{equation}		\label{OMEqChmVarSpherique}
	\left\{
\begin{array}{lllll}
x=r\cos\theta\sin\varphi	&			&r\in\mathopen] 0 , \infty \mathclose[\\
y=r\sin\theta\sin\varphi	&	\text{avec}	&\theta\in\mathopen] 0 , 2\pi \mathclose[\\
z=r\cos\varphi			&			&\phi\in\mathopen] 0 , \pi \mathclose[.
\end{array}
\right.
\end{equation}
Le jacobien associé est $Jg(r,\theta,\varphi)=-r^2\sin\varphi$. Rappelons que ce qui rentre dans l'intégrale est la valeur absolue du jacobien.

Si nous voulons calculer le volume de la sphère de rayon $R$, nous écrivons donc
\begin{equation}
	\int_0^Rdr\int_{0}^{2\pi}d\theta\int_0^{\pi}r^2 \sin(\phi)d\phi=4\pi R=\frac{ 4 }{ 3 }\pi R^3.
\end{equation}
Ici, la valeur absolue n'est pas importante parce que lorsque $\phi\in\mathopen] 0,\pi ,  \mathclose[$, le sinus de $\phi$ est positif.

Des petits malins pourraient remarquer que le changement de variable \eqref{OMEqChmVarSpherique} est encore une paramétrisation de $\eR^3$ si on intervertit le domaine des angles : 
\begin{equation}
	\begin{aligned}[]
		\theta&\colon 0 \to \pi\\
		\phi	&\colon 0\to 2\pi,
	\end{aligned}
\end{equation}
alors nous paramétrons encore parfaitement bien la sphère, mais hélas
\begin{equation}		\label{EqOMVolumeIncorrectSphere}
	\int_0^Rdr\int_{0}^{\pi}d\theta\int_0^{2\pi}r^2 \sin(\phi)d\phi=0.
\end{equation}
Pourquoi ces «nouvelles» coordonnées sphériques sont-elles mauvaises ? Il y a que quand l'angle $\phi$ parcours $\mathopen] 0 , 2\pi \mathclose[$, son sinus n'est plus toujours positif, donc la \emph{valeur absolue} du jacobien n'est plus $r^2\sin(\phi)$, mais $r^2\sin(\phi)$ pour les $\phi$ entre $0$ et $\pi$, puis $-r^2\sin(\phi)$ pour $\phi$ entre $\pi$ et $2\pi$. Donc l'intégrale \eqref{EqOMVolumeIncorrectSphere} n'est pas correcte. Il faut la remplacer par
\begin{equation}
	\int_0^Rdr\int_{0}^{\pi}d\theta\int_0^{\pi}r^2 \sin(\phi)d\phi- \int_0^Rdr\int_{0}^{\pi}d\theta\int_{\pi}^{2\pi}r^2 \sin(\phi)d\phi = \frac{ 4 }{ 3 }\pi R^3
\end{equation}

%+++++++++++++++++++++++++++++++++++++++++++++++++++++++++++++++++++++++++++++++++++++++++++++++++++++++++++++++++++++++++++
\section{Aire et primitive}
%+++++++++++++++++++++++++++++++++++++++++++++++++++++++++++++++++++++++++++++++++++++++++++++++++++++++++++++++++++++++++++

Soit $f\colon \eR\to \eR$, une fonction continue, et $x\in\eR$. Pour chaque $x\in\eR$, nous pouvons considérer le nombre $F(x)$ défini par
\begin{equation}
	F(x)=\int_a^x f(t)dt.
\end{equation}

%The result is on figure \ref{LabelFigVSZRooRWgUGu}. % From file VSZRooRWgUGu
\newcommand{\CaptionFigVSZRooRWgUGu}{Surface sous une courbe}
\input{auto/pictures_tex/Fig_VSZRooRWgUGu.pstricks}

La fonction $F$ ainsi définie a deux importantes propriétés :
\begin{enumerate}

\item
C'est une primitive de $f$,
\item
Elle donne la surface en dessous de $f$ entre les points $a$ et $x$, voir la figure \ref{LabelFigVSZRooRWgUGu}.

\end{enumerate}

Notons que tant que $f$ est positive, la surface est croissante.

La manière de calculer la surface comprise entre deux fonctions est dessinée à la figure \ref{LabelFigQOBAooZZZOrl}. % From file QOBAooZZZOrl
\newcommand{\CaptionFigQOBAooZZZOrl}{Le calcul de la surface comprise entre deux fonctions.}
\input{auto/pictures_tex/Fig_QOBAooZZZOrl.pstricks}
%See also the subfigure \ref{LabelFigQOBAooZZZOrlssLabelSubFigQOBAooZZZOrl0}
%See also the subfigure \ref{LabelFigQOBAooZZZOrlssLabelSubFigQOBAooZZZOrl1}
%See also the subfigure \ref{LabelFigQOBAooZZZOrlssLabelSubFigQOBAooZZZOrl2}

La surface entre les deux fonctions $y_1(x)$ et $y_2(x)$ se calcule comme suit.
\begin{enumerate}

\item
On calcule les intersections entre $y1$ et $y_2$. Notons $a$ et $b$ les ordonnées obtenues.
\item
La surface demandée est la différence entre la surface sous la fonction $y_1$ (la plus grande) et la surface sous la fonction $y_2$ (la plus petite), donc
\begin{equation}
	S=\int_{a}^by_1-\int_a^by_1.
\end{equation}

\end{enumerate}

%---------------------------------------------------------------------------------------------------------------------------
					\subsection{Longueur d'arc de courbe}
%---------------------------------------------------------------------------------------------------------------------------

La longueur de l'arc de courbe de la fonction $y=f(x)$ entre les abscisses $x_0$ et $x_1$ est donné par la formule
\begin{equation}		\label{EqLongArcCourbe}
	l(x_0,x_1)=\int_{x_0}^{x_1}\sqrt{1+y'(t)^2}dt.
\end{equation}

Lorsque la courbe est donnée sous forme paramétrique
\begin{subequations}
\begin{numcases}{}
	x=x(t)\\
	y=y(t),
\end{numcases}
\end{subequations}
alors la formule devient
\begin{equation}		\label{EqLongArcParam}
	l(t_1,t_2)=\int_{t_1}^{t_2}\sqrt{\dot x(t)^2+\dot y(t)^2}dt,
\end{equation}
où $\dot x(t)=x'(t)$.

%---------------------------------------------------------------------------------------------------------------------------
					\subsection{Aire de révolution}
%---------------------------------------------------------------------------------------------------------------------------

Pour savoir l'aire engendrée par la ligne $y=f(x)$ entre $a$ et $b$ autour de l'axe $Ox$, on utilise la formule
\begin{equation}
	S=2\pi\int_a^b\sqrt{1+f'(x)^2}f(x)dx.
\end{equation}
