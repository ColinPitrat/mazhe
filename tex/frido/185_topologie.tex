% This is part of Mes notes de mathématique
% Copyright (c) 2008-2018
%   Laurent Claessens, Carlotta Donadello
% See the file fdl-1.3.txt for copying conditions.

%+++++++++++++++++++++++++++++++++++++++++++++++++++++++++++++++++++++++++++++++++++++++++++++++++++++++++++++++++++++++++++
\section{Limite et continuité}
%+++++++++++++++++++++++++++++++++++++++++++++++++++++++++++++++++++++++++++++++++++++++++++++++++++++++++++++++++++++++++++

\begin{definition}
    Un \defe{homéomorphisme}{homéomorphisme} est une application bijective continue entre deux espaces topologiques dont la réciproque est continue. Deux espaces topologiques homéomorphes sont dits \defe{isomorphes}{isomorphisme!d'espaces topologiques}.
\end{definition}

\begin{definition}[Limite d'une fonction, thème~\ref{THEMEooGVCCooHBrNNd}]\label{DefYNVoWBx}
    Soient \( X\) et \( Y\) des espaces topologiques, \( A\subset X\) et \( a\in\bar A\). Soit encore une fonction \( f\colon A\to Y\). L'élément \( y\in Y\) est une \defe{limite}{limite!d'une fonction} de \( f\) en \( a\) si pour tout voisinage \( V\) de \( y\), il existe un voisinage \( W\) de \( a\) dans \( X\) tel que
    \begin{equation}
        f\big( W\cap A\setminus\{a\} \big)\subset V.
    \end{equation}

    Si un tel élément est unique, alors nous disons que cet élément est la \defe{limite}{limite d'une fonction} de \( f\) et nous notons
    \begin{equation}
        \lim_{x\to a} f(x)=y.
    \end{equation}
\end{definition}

\begin{remark}
    Nous ne saurions trop insister sur le fait que la valeur de \( f\) en \( a\) n'intervient pas dans la définition de la limite de \( f\) en \( a\). Il n'est même pas pas nécessaire que \( f\) soit définie en \( a\) pour que l'on puisse parler de limite de \( f\) en \( a\). Par exemple nous avons
    \begin{equation}
        \lim_{x\to 1} \frac{ x^2-1 }{ x-1 }=2,
    \end{equation}
    alors que la fonction n'est pas définie en \( x=1\).

    Plus généralement, un peu par principe, toutes les fois que la notion de limite apporte une information, le point où l'on prend la limite est spécial. Sinon on ne calculerait pas la limite, mais on regarderait directement la valeur de la fonction. Cela est typiquement le cas lorsque nous verrons les dérivées. En effet, regardons (en faisans du semblant d'anticiper) la définition  \eqref{DEFooOYFZooFWmcAB}. Dans la formule
    \begin{equation}
        f'(a)=\lim_{x\to a} \frac{ f(x)-f(a) }{ x-a },
    \end{equation}
    la fonction sur laquelle nous prenons la limite n'est \emph{jamais} définie en \( x=a\).

    Cela est intimement liée à ce que je raconte dans~\ref{SUBSECooVHKCooYRFgrb}.
\end{remark}


\begin{example} \label{EXooSHKAooZQEVLB}
    Oui, il y a moyen de converger vers plusieurs points distincts si l'espace n'est pas super cool. Nous pouvons par exemple\cite{EJVQuas} considérer la droite réelle munie de sa topologie usuelle et y ajouter un point $0'$ (qui clone le réel $0$) dont les voisinages sont les voisinages de $0$ dans lesquels nous remplaçons $0$ par $0'$. Dans cet espace, la suite $(1/n)$ converge à la fois vers $0$ et $0'$.
\end{example}

\begin{proposition}[Unicité de la limite pour un espace séparé]\label{PropFObayrf}
    Soient \( X\) un espace topologique, \( A\) une partie de \( X\) et \( Y\) un espace topologique séparé\footnote{Définition~\ref{DefYFmfjjm}.}. Nous considérons une fonction \( f\colon A\to Y\). Si \( a\in\bar A\), alors \( f\) admet au plus une limite en \( a\).
\end{proposition}
\index{limite!unicité}

\begin{proof}
    Soient \( y\) et \( y'\) des limites de \( f\) en \( a\), ainsi que des voisinages \( V\) et \( V'\) de \( y\) et \( y'\). Nous prenons également les voisinages \( W\) et \( W'\) correspondants :
    \begin{subequations}
        \begin{numcases}{}
            f(W\cap A)\subset V\\
            f(W'\cap A)\subset V'.
        \end{numcases}
    \end{subequations}
    Quitte à prendre des sous-ensembles nous pouvons supposer que \( W\) et \( W'\) sont ouverts. L'ensemble \( W\cap W'\) est un ouvert contenant \( a\) et intersecte donc \( A\). L'ensemble \( (W\cap W')\cap A\) est non vide et
    \begin{subequations}
        \begin{numcases}{}
            f(W\cap W'\cap A)\subset f(W\cap A)\subset V\\
            f(W\cap W'\cap A)\subset f(W'\cap A)\subset V'.
        \end{numcases}
    \end{subequations}
    Donc les ensembles \( V\) et \( V'\) ont une intersection. Au final nous avons prouvé que deux voisinages de \( y\) et \( y'\) ont forcément une intersection; étant donné que \( Y\) est séparé, nous devons avoir \( y=y'\).
\end{proof}

La définition suivante est \emph{la} définition de la continuité dans tous les cas.
\begin{definition}[Fonction continue]\label{DefOLNtrxB}
    Une fonction \( f\colon X\to Y\) est \defe{continue au point}{continue!fonction!en un point} \( a\in X\) si \( f(a)\) est une limite\footnote{Définition~\ref{DefYNVoWBx}.} de \( f\) en \( a\).

    Une fonction \( f\colon X\to Y\) est \defe{continue}{continue!fonction entre espaces topologiques} si pour tout ouvert \( \mO\) de \( Y\), l'ensemble \( f^{-1}(\mO)\) est ouvert dans \( X\).
\end{definition}
La proposition~\ref{PropQZRNpMn} donnera des détails sur ce qu'il se passe lorsque l'espace est métrique.

\begin{theorem} \label{ThoESCaraB}
    Une fonction \( f\colon X\to Y\) est une fonction continue si et seulement si elle est continue en chacun des points de \( X\).
\end{theorem}

\begin{proof}
    En deux parties.
    \begin{subproof}
    \item[Sens direct]
        Nous supposons que \( f\) est une fonction continue. Soient \( a\in X\) et \( V\) un voisinage de \( f(a)\). Nous considérons \( \mO\), un voisinage ouvert de \( f(a)\) contenu dans \( V\); l'ensemble \( f^{-1}(\mO)\) est alors un ouvert contenant \( a\), et l'image de \( f^{-1}(\mO)\) par \( f\) est bien entendu contenue dans \( V\).

    \item[Sens inverse]

        Soit \( \mO\) un ouvert de \( Y\). Pour prouver que \( f^{-1}(\mO)\) est un ouvert de\( X\), nous allons considérer un élément \( a\in f^{-1}(\mO)\) et montrer qu'il existe un voisinage ouvert de \( a\) contenu dans \( f^{-1}(\mO)\); le théorème~\ref{ThoPartieOUvpartouv} nous assurera alors que \( f^{-1}(\mO)\) est ouvert.

        L'ensemble \( \mO\) est un voisinage ouvert de \( f(a)\) parce que \( a\) a été choisi dans \( f^{-1}(\mO)\). Donc la continuité de \( f\) en \( a\)\footnote{Définition~\ref{DefOLNtrxB}.} nous assure qu'il existe un voisinage \( W\) de \( a\) tel que \( f(W)\subset\mO\). En prenant un ouvert contenant \( a\) à l'intérieur de \( W\) nous avons un voisinage ouvert de \( a\) contenu dans \( f^{-1}(\mO)\).
    \end{subproof}
\end{proof}

\begin{remark}
    À cause de l'éventuelle non unicité de la limite, deux fonctions continues et égales sur un sous-ensemble dense ne sont pas spécialement égales. Ce sera vrai sur les espaces métriques et plus généralement pour les espaces séparés. Voir l'exemple~\ref{EXooSHKAooZQEVLB} et la proposition~\ref{PropFObayrf}.
\end{remark}

%+++++++++++++++++++++++++++++++++++++++++++++++++++++++++++++++++++++++++++++++++++++++++++++++++++++++++++++++++++++++++++
\section{Compacité}
%+++++++++++++++++++++++++++++++++++++++++++++++++++++++++++++++++++++++++++++++++++++++++++++++++++++++++++++++++++++++++++

La compacité est le thème~\ref{THEMEooQQBHooLcqoKB}.

%---------------------------------------------------------------------------------------------------------------------------
\subsection{Quelques propriétés}
%---------------------------------------------------------------------------------------------------------------------------

\begin{definition}
    Une famille \( \mA\) de parties de \( X\) a la \defe{propriété d'intersection finie non vide}{propriété d'intersection non vide} si tout sous-ensemble fini de \( \mA\) a une intersection non vide.
\end{definition}

\begin{proposition}\label{PropXKUMiCj}
    Soient \( X\) un espace topologique et \( K\subset X\). Les propriétés suivantes sont équivalentes :
    \begin{enumerate}
        \item\label{ItemXYmGHFai}
            \( K\) est compact.
        \item\label{ItemXYmGHFaii}
            Si \( \{ F_i \}\) est une famille de fermés telle que \( K\bigcap_{i\in I}F_i=\emptyset\), alors il existe un sous-ensemble fini \( A\) de \( I\) tel que \( K\bigcap_{i\in A}F_i=\emptyset\).
        \item\label{ItemXYmGHFaiii}
            Si \( \{ F_i \}_{i\in I}\) est une famille de fermés telle que \( K\bigcap_{i\in A}F_i\neq\emptyset\) pour tout choix de \( A\) fini dans \( I\), alors l'intersection complète est non vide : \( K\bigcap_{i\in I}F_i\neq\emptyset\).
        \item\label{ItemXYmGHFaiv}
            Toute famille ayant la propriété d'intersection finie non vide a une intersection non vide.
    \end{enumerate}
\end{proposition}

\begin{proof}
    Les propriétés~\ref{ItemXYmGHFaiii} et~\ref{ItemXYmGHFaii} sont équivalentes par contraposition. De plus le point~\ref{ItemXYmGHFaiv} est une simple reformulation en français de la propriété~\ref{ItemXYmGHFaiii}.

    Prouvons~\ref{ItemXYmGHFai} \( \Rightarrow\)~\ref{ItemXYmGHFaii}. Soit \( \{ F_i \}_{i\in I}\) une famille de fermés tels que \( K\bigcap_{i\in I}F_i=\emptyset\). Les complémentaires \( \mO_i\) de \( F_i\) dans \( X\) recouvrent \( K\) et donc on peut en extraire un sous-recouvrement fini :
    \begin{equation}
        K\subset\bigcup_{i\in A}\mO_i
    \end{equation}
    pour un certain sous-ensemble fini \( A\) de \( I\). Pour ce même choix \( A\), nous avons alors aussi
    \begin{equation}
        K\bigcap_{i\in A}F_i=\emptyset.
    \end{equation}

    L'implication~\ref{ItemXYmGHFaii} \( \Rightarrow\)~\ref{ItemXYmGHFai} est la même histoire.
\end{proof}

\begin{proposition}[\cite{OCBrmKo}]\label{PropUCUknHx}
    Tout compact d'un espace topologique séparé est fermé.
\end{proposition}
\index{compact!implique fermé}

\begin{proof}
    Soient \( X\) un espace séparé et \( K\) compact dans \( X\). Nous considérons \( x\in\complement K\) et, par hypothèse de séparation, pour chaque \( x\in K\) nous considérons un voisinage ouvert \( V_x\) de \( x\) et un voisinage ouvert \( W_x\) de \( y\) tels que \( V_x\cap W_x=\emptyset\). Bien entendu les \( V_x\) forment un recouvrement ouvert de \( K\) dont nous pouvons extraire un sous-recouvrement fini : soit \( S\) fini dans \( K\) tel que
    \begin{equation}
        K\subset\bigcup_{x\in S}V_x.
    \end{equation}
    L'ensemble \( W=\bigcap_{x\in S}W_x\) est une intersection finie d'ouverts autour de \( y\) et est donc un ouvert autour de \( y\). De plus \( W\) n'intersecte pas \( K\) parce que si \( x\in K\) alors \( x\in V_s\) pour un certain \( s\in S\) par conséquent \( x\) n'est pas dans \( W_s\) et donc pas non plus dans l'intersection.

    L'ouvert \( W\) prouve que \( y\) est dans l'intérieur du complémentaire de \( K\), ce qui nous permet de conclure que le complémentaire de \( K\) est ouvert (théorème~\ref{ThoPartieOUvpartouv}), aka que \( K\) est fermé.
\end{proof}

\begin{lemma}[\cite{SNSposN}]   \label{LemnAeACf}
    Une partie fermée d'une compact est elle-même compacte.
\end{lemma}
\index{fermé!dans un compact}

\begin{proof}
%    Nous allons utiliser la caractérisation de la compacité en termes de suites donnée par le théorème de Bolzano-Weierstrass~\ref{ThoBWFTXAZNH}. Soit \( K\) un compact et \( F\) un fermé dans \( K\). Nous considérons une suite \( (x_n)\) dans \( F\); par la compacité de \( K\) nous pouvons considérer une sous-suite \( (y_n)\) qui converge dans \( K\) (proposition~\ref{ThoBWFTXAZNH}). Étant donné que \( (y_n)\) est une suite convergente contenue dans \( F\) et étant donné que \( F\) est fermé, la limite est dans \( F\), ce qui prouve que \( (x_n)\) possède une sous-suite convergente dans $F$ et par conséquent que \( F\) est compact.

    Soient \( F\) fermé dans un compact \( K\) et \( \{ \mO_i \}_{i\in I}\) un recouvrement de \( F\) par des ouverts. Vu que \( F\) est fermé, \( F^c\) est ouvert et \( \{ \mO_i \}_{i\in I}\cup\{ K\setminus F \}\) est un recouvrement de \( K\) par des ouverts. Si nous en extrayons un sous-recouvrement fini, c'est un recouvrement de \( F\), et en supprimant éventuellement l'ouvert \( K\setminus F\), ça reste un sous-recouvrement fini de \( F\) tout en étant extrait de \( \{ \mO_i \}_{i\in I}\).
\end{proof}

\begin{proposition}     \label{PropGBZUooRKaOxy}
    Si \( V\) est une partie de l'espace topologique \( X\) muni de la topologie induite \( \tau_V\) de celle de \( X\), et si \( K\) est un compact de \( (V,\tau_V)\) alors \( K\) est un compact de \( (X,\tau_X)\).
\end{proposition}

\begin{proof}
    Soient \(   (\mO_\alpha)_{\alpha\in A}  \) des ouverts de \( X\) recouvrant \( K\). Alors les ensembles \( V\cap \mO_{\alpha}\) recouvrent également \( K\), mais sont des ouverts de \( V\). Donc il en existe un sous-recouvrement fini. Soient donc \( (V\cap\mO_i)_{i\in I}\) recouvrant \( K\) avec \( I\) un sous-ensemble fini de \( A\). Les ensembles \( (\mO_i)_{i\in I}\) recouvrent encore \( K\) et sont des ouverts de \( X\).
\end{proof}

\begin{theorem}     \label{ThoImCompCotComp}
L'image d'un compact par une fonction continue est un compact
\end{theorem}
Dans le cadre des espaces vectoriels normés, ce théorème est démontré en la proposition~\ref{PropContinueCompactBorne}.

\begin{proof}
    Soit $K\subset X$, un ensemble compact, et regardons $f(K)$; en particulier, nous considérons $\Omega$, un recouvrement de $f(K)$ par des ouverts. Nous avons que
    \begin{equation}
        f(K)\subseteq\bigcup_{\mO\in\Omega}\mO.
    \end{equation}
    Par construction, nous avons aussi
    \begin{equation}
        K\subseteq\bigcup_{\mO\in\Omega}f^{-1}(\mO),
    \end{equation}
    en effet, si $x\in K$, alors $f(x)$ est dans un des ouverts de $\Omega$, disons $f(x)\in \mO_0$, et évidemment, $x\in f^{-1}(\mO)$.  Les $f^{-1}(\mO)$ recouvrent le compact $K$, et donc on peut en choisir un sous-recouvrement fini, c'est à dire un choix de $\{ f^{-1}(\mO_1),\ldots,f^{-1}(\mO_n) \}$ tels que
    \begin{equation}
        K\subseteq \bigcup_{i=1}^nf^{-1}(\mO_i).
    \end{equation}
    Dans ce cas, nous avons que
    \begin{equation}
        f(K)\subseteq\bigcup_{i=1}^n\mO_i,
    \end{equation}
    ce qui prouve la compacité de $f(K)$.
\end{proof}

%+++++++++++++++++++++++++++++++++++++++++++++++++++++++++++++++++++++++++++++++++++++++++++++++++++++++++++++++++++++++++++
\section{Topologie induite}
%+++++++++++++++++++++++++++++++++++++++++++++++++++++++++++++++++++++++++++++++++++++++++++++++++++++++++++++++++++++++++++

\begin{definition}  \label{DefVLrgWDB}
Soit \( X\) un espace topologique et \( A\subset X\). L'ensemble \( A\) devient un espace topologique en lui-même par la \defe{topologie induite}{topologie!induite} de \( X\). Un ouvert de \( A\) est un ensemble de la forme \( A\cap\mO\) où \( \mO\) est un ouvert de \( X\).
\end{definition}

\begin{lemma}       \label{LemkUYkQt}
    Si \( B\subset A\) alors la fermeture de \( B\) pour la topologie de \( A\) (induite de \( X\)) que nous noterons \( \tilde B\) est
    \begin{equation}
        \tilde B=\bar B\cap A
    \end{equation}
    où \( \bar B\) est la fermeture de \( B\) pour la topologie de \( X\).
\end{lemma}

\begin{proof}
    Si \( a\in \bar B\cap A\), un ouvert de \( A\) autour de \( a\) est un ensemble de la forme \( \mO\cap A\) où \( \mO\) est un ouvert de \( X\). Vu que \( a\in\bar B\), l'ensemble \( \mO\) intersecte \( B\) et donc \( (\mO\cap A)\cap B\neq \emptyset\). Donc \( a\) est bien dans l'adhérence de \( B\) au sens de la topologie de \( A\).

    Pour l'inclusion inverse, soit \( a\in \tilde  B\), et montrons que \( a\in \bar B\cap A\). Par définition \( a\in A\), parce que \( \tilde B\) est une fermeture dans l'espace topologique \( A\). Il faut donc seulement montrer que \( a\in\bar B\). Soit donc \( \mO\) un ouvert de \( X\) contenant \( a\); par hypothèse \( \mO\cap A\) intersecte \( B\) (parce que tout ouvert de \( A\) contenant \( a\) intersecte \( B\)). Donc \( \mO\) intersecte \( B\). Cela signifie que tout ouvert (de \( X\)) contenant \( a\) intersecte \( B\), ou encore que \( a\in \bar B\).
\end{proof}

\begin{example} \label{ExloeyoR}
    Si \( A\) est un ouvert de \( X\), on pourrait croire que la topologie induite n'a rien de spécial. Il est vrai que \( B\) sera ouvert dans \( A\) si et seulement s'il est ouvert dans \( X\), mais des choses se passent quand même. Prenons \( X=\eR\) et \( A=\mathopen] 0 , 1 \mathclose[\). Si \( B=\mathopen] \frac{ 1 }{2} , 1 \mathclose[ \), alors la fermeture de \( B\) dans \( A\) sera \( \tilde B=\mathopen[ \frac{ 1 }{2} , 1 [\) et non \( \mathopen[ \frac{ 1 }{2} , 1 \mathclose]\) comme on l'aurait dans \( \eR\).
\end{example}

Prendre la topologie induite de \( \eR\) vers un fermé de \( \eR\) donne des boules un peu spéciales comme le montre l'exemple suivant.

\begin{example}  \label{ExKYZwYxn}
    Quid de la boule ouverte \( B(1,\epsilon)\) dans le compact \( \mathopen[ 0 , 1 \mathclose]\) ? Par définition c'est
    \begin{equation}
        B(1,\epsilon)=\{ x\in\mathopen[ 0 , 1 \mathclose]\tq | x-1 |<\epsilon \}=\mathopen] 1-\epsilon , 1 \mathclose].
    \end{equation}
    Oui, cela est \emph{ouvert} dans \( \mathopen[ 0 , 1 \mathclose]\). C'est d'ailleurs un des ouverts de la topologie induite de \( \eR\) sur \( \mathopen[ 0 , 1 \mathclose]\).

    Donc pour la topologie de \( \mathopen[ 0 , 1 \mathclose]\), toutes les boules ouvertes \( B(x,\delta)\) avec \( x\in\mathopen[ 0 , 1 \mathclose]\) sont incluses à \( \mathopen[ 0 , 1 \mathclose]\).
\end{example}

Au niveau de la notion de continuité, il n'y a pas trop de changements en passant de \( \eR\) à \( \eQ\) muni de la topologie induite.
\begin{example}     \label{EXooHWIIooYYbfGE}
    Que signifie d'être continue pour une fonction \( f\colon \eQ\to \eR\) ? D'après le théorème~\ref{ThoESCaraB}, il s'agit d'être continue en chaque point de \( \eQ\). Il s'agit donc, par la définition~\ref{DefOLNtrxB} que pour tout \( q\in \eQ\), le nombre \( f(q)\) soit une limite de \( f\) pour \( x\to q\).

    L'espace d'arrivée étant \( \eR\), un voisinage de \( f(q)\) est pris comme une boule de taille \( \epsilon\). La continuité de \( f\) exige qu'il y ait un voisinage \( W\) de \( q\) dans \( \eQ\) tel que pour tout \( q'\in W\) (différent que \( q\)), \( | f(q)-f(q') |<\epsilon\).

    Qu'est-ce qu'un ouvert dans \( \eQ\) ? D'après la définition~\ref{DefVLrgWDB} de la topologie induite, ce sont les ensembles \( \eQ\cap\mO\) avec \( \mO\) ouvert dans \( \eR\). Tout cela pour dire que pour tout \( \epsilon>0\), il doit exister \( \delta>0\) tel que pour tout \( q'\in \eQ\) tel que \( 0<| q-q' |<\delta\), nous ayons \( | f(q)-f(q') |\).

    Bref, c'est exactement le mécanisme usuel de la continuité sur \( \eR\), sauf qu'il faut seulement considérer les rationnels.
\end{example}


\begin{lemma}   \label{LemPESaiVw}
    Soit \( A\subset X\) muni de la topologie induite de \( X\) et \( (x_n)\) une suite dans \( A\). Si \( x_n\stackrel{A}{\longrightarrow}x\), alors \( x_n\stackrel{X}{\longrightarrow}x\).
\end{lemma}

\begin{proof}
    Soit \( \mO\) un ouvert autour de \( x\) dans \( X\). Alors \( A\cap\mO\) est un ouvert autour de \( x\) dans \( A\) et il existe \( N\in \eN\) tel que si \( n\geq N\), alors \( x_n\in A\cap\mO\subset\mO\).
\end{proof}

\begin{lemma}[\cite{MonCerveau}]        \label{LemBWSUooCCGvax}
    Soit \( (X,\tau_X)\) un espace topologique et \( S\subset X\), un fermé de \( X\) sur lequel nous considérons la topologie induite \( \tau_S\). Si \( F\) est un fermé de \( (S,\tau_S)\) alors \( F\) est fermé de \( (X,\tau_X)\).
\end{lemma}

\begin{proof}
    Nous savons que le complémentaire de \( F\) dans \( S\) est un ouvert de \( (S,\tau_S)\) : il existe un ouvert \( \Omega\in \tau_X\) tel que \( S\setminus F=S\cap \Omega\). Si maintenant nous considérons le complémentaire de \( S\) dans \( X\) nous avons
    \begin{equation}
        F^c=(S\setminus F)\cup (X\setminus S)=(S\cap \Omega)\cup S^c=(S\cap \Omega)\cup(S^c\cap \Omega)\cup S^c=\Omega\cup S^c.
    \end{equation}
    Vu que \( \Omega\) et \( S^c\) sont des ouverts de \( X\), l'union est un ouvert. Donc \( F^c\in \tau_X\) et \( F\) est un fermé de \( X\).
\end{proof}

%+++++++++++++++++++++++++++++++++++++++++++++++++++++++++++++++++++++++++++++++++++++++++++++++++++++++++++++++++++++++++++
\section{Connexité}
%+++++++++++++++++++++++++++++++++++++++++++++++++++++++++++++++++++++++++++++++++++++++++++++++++++++++++++++++++++++++++++

Dès qu'un ensemble est muni d'une métrique, nous pouvons définir les boules ouvertes, les voisinages et les sous-ensembles ouverts. Dès que l'on a identifié les sous-ensemble ouverts de $E$, nous disons que $E$ devient un \defe{espace topologique}{espace!topologique}. Nous allons maintenant un pas plus loin.

\begin{definition}  \label{DefIRKNooJJlmiD}
     Lorsque $E$ est un espace topologique, nous disons qu'un sous-ensemble $A$ est \defe{non connexe}{connexité!définition} quand on peut trouver des ouverts $O_1$ et $O_2$ disjoints tels que
    \begin{equation}    \label{EqDefnnCon}
        A=(A\cap O_1)\cup (A\cap O_2),
    \end{equation}
    et tels que $A\cap O_1\neq\emptyset$, et $A\cap O_2\neq\emptyset$. Si un sous-ensemble n'est pas non-connexe, alors on dit qu'il est connexe.
\end{definition}
Une autre façon d'exprimer la condition \eqref{EqDefnnCon} est de dire que $A$ n'est pas connexe quand il est contenu dans la réunion de deux ouverts disjoints qui intersectent tous les deux $A$.

\begin{proposition} \label{PropHSjJcIr}
    Soit \( X\) un espace topologique. Les conditions suivantes sont équivalentes.
    \begin{enumerate}
        \item
            L'espace \( X\) est connexe.
        \item
            Si \( X=A\sqcup B\) avec \( A\) et \( B\) fermés dans \( X\), alors \( A=\emptyset\) ou \( B=\emptyset\).
        \item       \label{ITEMooNIPZooIDPmEf}
            Si \( A\subset X\) avec \( A\) ouvert et fermé en même temps, alors \( A=\emptyset\) ou \( A=X\).
        \item
            Toute application continue \( X\to \eZ\) est constante.
    \end{enumerate}
\end{proposition}
%TODO : une preuve.

\begin{proposition}\label{PropGWMVzqb}
    L'image d'un ensemble connexe par une fonction continue est connexe.
\end{proposition}

\begin{proof}
    Soit \( f\colon X\to Y\) une application continue entre deux espaces topologiques, et \( E\) une partie connexe de \( X\). Nous devons montrer que \( f(E)\) est connexe dans \( Y\).

    Par l'absurde nous considérons \( A\) et \( B\), deux ouverts de \( Y\) disjoints recouvrant \( f(E)\). Étant donné que \( f\) est continue, les ensembles \( f^{-1}(A)\) et \( f^{-1}(B)\) sont ouverts dans \( X\). De plus ces deux ensembles recouvrent \( E\).

    Si \( x\) est un élément de \( f^{-1}(A)\cap f^{-1}(B)\), alors \( f(x)\in A\cap B\), ce qui est impossible parce que nous avons supposé que \( A\) et \( B\) étaient disjoints. Par conséquent \( f^{-1}(A)\) et \( f^{-1}(B)\) sont deux ouverts disjoints recouvrant \( E\). Contradiction avec la connexité de \( E\). Nous concluons que \( f(E)\) est connexe.
\end{proof}
Une application de ce théorème sera le théorème de valeurs intermédiaires~\ref{ThoValInter}.


\begin{example}
    Les espaces topologiques \( \eR\) et \( \eR^2\) ne sont pas homéomorphes.
\end{example}

\begin{proof}
    Soit \( f\colon \eR\to \eR^2\) un homéomorphisme. Nous posons \( E=f\big( \eR\setminus\{ 0 \} \big)\) et \( z_0=f(0)\). Vu que \( f\) est bijective nous avons
    \begin{equation}
        E=\eR^2\setminus\{ z_0 \},
    \end{equation}
    qui est connexe.

    Vu que \( E\) est connexe et que \( f^{-1}\) est continue, la proposition~\ref{PropGWMVzqb} nous dit que \( f^{-1}(E)\) est connexe. Mais par définition, \( f^{-1}(E)=\eR\setminus\{ 0 \}\) qui n'est pas connexe.
\end{proof}


\begin{proposition}
    Si \( A\subset X\) est connexe et si \( A\subset B\subset \bar A\), alors \( B\) est connexe.
\end{proposition}
%TODO : une preuve.

\begin{proposition} \label{PropIWIDzzH}
    Stabilité de la connexité par union.
    \begin{enumerate}
        \item
    Une union quelconque de connexes ayant une intersection non vide est connexe.
\item
    Si \( A_1,\ldots, A_n\) sont des connexes de \( X\) avec \( A_i\cap A_{i+1}\neq \emptyset\), alors l'union \( \bigcup_{i=1}^nA_i\) est connexe.
    \end{enumerate}
\end{proposition}

\begin{proof}
    \begin{enumerate}
        \item
    Soient \( \{ C_i \}_{i\in I}\) un ensemble de connexes et un point \( p\) dans l'intersection : \( p\in\bigcap_{i\in I}C_i\). Supposons que l'union ne soit pas connexe. Alors nous considérons \( A\) et \( B\), deux ouverts disjoints recouvrant tous les \( C_i\) et ayant chacun une intersection non vide avec l'union.

    Supposons pour fixer les idées que \( p\in A\) et prenons \( x\in B\cap\bigcup_{i\in I}C_i\). Il existe un \( j\in I\) tel que \( x\in C_j\). Avec tout cela nous avons
    \begin{enumerate}
        \item
            \( C_j\subset A\cup B\),
        \item
            \( C_j\cap A\neq \emptyset\) parce que \( p\) est dans l'intersection,
        \item
            \( C_j\cap B\neq\emptyset\) parce que \( x\) est dans cette intersection.
    \end{enumerate}
    Cela contredit le fait que \( C_j\) soit connexe.

\item

    Pour la seconde partie nous procédons de proche en proche. D'abord \( A_1\cup A_2\) est connexe par la première partie, ensuite \( (A_1\cup A_2)\cup A_3\) est connexe parce que les connexes \( A_1\cap A_2\) et \( A_3\) ont un point d'intersection par hypothèse, et ainsi de suite.
    \end{enumerate}
\end{proof}
%+++++++++++++++++++++++++++++++++++++++++++++++++++++++++++++++++++++++++++++++++++++++++++++++++++++++++++++++++++++++++++
\section{Un peu de topologie réelle}
%+++++++++++++++++++++++++++++++++++++++++++++++++++++++++++++++++++++++++++++++++++++++++++++++++++++++++++++++++++++++++++

Afin de pouvoir étudier la topologie des espaces métriques, il faut savoir quelques propriétés des réels parce que nous allons étudier la fonction distance qui est une fonction continue à valeurs dans les réels.

%---------------------------------------------------------------------------------------------------------------------------
\subsection{Suites numériques}
%---------------------------------------------------------------------------------------------------------------------------

La propriété suivante est le plus souvent donnée comme une définition lorsque seule la topologie sur \( \eR\) est considérée.
\begin{proposition}[Limite d'une suite numérique]	\label{PropLimiteSuiteNum}
	La suite $(x_n)$ est convergente si et seulement s'il existe un réel $\ell$ tel que
	\begin{equation}		\label{EqDefLimSuite}
		\forall \epsilon>0,\,\exists N\in\eN\tq\forall n\geq N,\,| x_n-\ell |<\epsilon.
	\end{equation}
	Dans ce cas, le nombre $\ell$ est la limite de la suite $(x_n)$. Nous dirons aussi souvent que la suite \defe{converge}{} vers le nombre $\ell$.
\end{proposition}
\index{convergence!suite numérique}
\index{limite!suite numérique}

\begin{proof}
    La limite d'une suite dans un espace topologique est la définition~\ref{DefXSnbhZX}.

    Si \( x_n\to \ell\) et si \( \epsilon>0\) il existe \( N_{\epsilon}\) tel que pour tout \( n\geq N\) nous avons \( x_n\in B(\ell,\epsilon)\) (parce que cette boule est un ouvert contenant \( \ell\)). Vu la définition d'une boule (définition~\ref{ThoORdLYUu}) et de la norme sur \( \eR\) (par~\ref{ooLCMFooQjMaxV}), cette condition est bien \( | x_n-\ell |<\epsilon\).

    Dans l'autre sens, soit \( \mO\) un ouvert contenant \( \ell\). Par définition de la topologie, il existe \( \epsilon>0\) tel que \( B(\ell,\epsilon)\subset \mO\). La condition \eqref{EqDefLimSuite} nous assure qu'il existe \( N_{\epsilon} \) tel que pour tout \( n\geq N_{\epsilon}\) nous ayons
    \begin{equation}
        x_n\in B(\ell,\epsilon)\subset\mO,
    \end{equation}
    ce qui assure que la suite \( (x_n)\) converge vers \( \ell\) pour la topologie métrique de \( \eR\).
\end{proof}

Une façon équivalente d'exprimer le critère \eqref{EqDefLimSuite} est de dire que pour tout $\epsilon$ positif, il existe un rang $N\in\eR$ tel que l'intervalle $\mathopen[ \ell-\epsilon , \ell+\epsilon \mathclose]$ contient tous les termes $x_n$ au-delà de $N$.

Il est à noter que le rang $N$ dont il est question dans la définition de suite convergente dépend de~$\epsilon$.

\begin{example}
	Quelques suites usuelles.
	\begin{enumerate}
		\item
			La suite $x_n=\frac{1}{ n }$ converge vers $0$.
		\item
			La suite $x_n=(-1)^n$ ne converge pas.
	\end{enumerate}
\end{example}

Une suite est dite \defe{contenue}{} dans un ensemble $A$ si $x_n\in A$ pour tout $n$. Une suite est \defe{bornée supérieurement}{bornée!suite} s'il existe un $M$ tel que $x_n\leq M$ pour tout $n$. De la même manière, la suite est bornée inférieurement s'il existe un $m$ tel que $x_n\geq m$ pour tout $n$.

Le lemme suivant est souvent utilisé pour prouver qu'une suite est convergente.
\begin{lemma}		\label{LemSuiteCrBorncv}
	Une suite croissante et bornée supérieurement converge. Une suite décroissante bornée inférieurement est convergente.
\end{lemma}

\begin{theorem}
	Toute suite réelle contenue dans un compact admet une sous-suite convergente.
\end{theorem}

\begin{proof}
	Soit $(x_n)$ une suite contenue dans la partie bornée $A\subset\eR$. Nous disons qu'un élément $x_k$ de la suite est \emph{maximal} s'il est plus grand ou égal que tous les suivants : $x_k\geq x_{k'}$ dès que $k'\geq k$.

	Si la suite a un nombre infini d'éléments maximaux, alors la suite des éléments maximaux est décroissante. Si nous n'avons qu'un nombre fini de points maximaux, alors la suite de départ est croissante à partir du dernier point maximal.

	Dans les deux cas nous avons trouvé une sous-suite des $x_n$ qui est monotone (décroissante ou croissante selon le cas), et donc convergente parce que contenue dans un borné (lemme~\ref{LemSuiteCrBorncv}).
\end{proof}

% Inutile de replacer cette proposition plus loin : on en a besoin pour démontrer Weierstrass. Quitte à maintenir, il faut réénoncer pour un espace vectoriel normé et prouver.
\begin{proposition}		\label{PropCvRpComposante}
	Une suite $(x_n)$ dans $\eR^m$ est convergente dans $\eR^m$ si et seulement si les suites de chaque composante sont convergentes dans $\eR$. Dans ce cas nous avons
	 \begin{equation}
		 \lim x_n=\Big( \lim(x_n)_1,\lim (x_n)_2,\ldots,\lim (x_n)_m \Big)
	 \end{equation}
	 où $(x_n)_k$ dénote la $k$-ième composante de $(x_n)$.
\end{proposition}

\begin{example}
	La suite $x_n=\big( \frac{1}{ n },1-\frac{1}{ n } \big)$ converge vers $(0,1)$ dans $\eR^2$. En effet, en utilisant la proposition~\ref{PropCvRpComposante}, nous devons calculer séparément les limites
	\begin{equation}
		\begin{aligned}[]
			\lim\frac{1}{ n }&=0\\
			\lim\big( 1-\frac{1}{ n } \big)&=1.
		\end{aligned}
	\end{equation}
\end{example}

\begin{example}
	Étant donné que la suite $(-1)^n$ n'est pas convergente, la suite $x_n=\big( (-1)^n,\frac{1}{ n } \big)$ n'est pas convergente dans $\eR^2$.
\end{example}

%---------------------------------------------------------------------------------------------------------------------------
\subsection{Maximum, supremum et compagnie}
%---------------------------------------------------------------------------------------------------------------------------

Ce n'est un secret pour personne que $\eR$ est un \href{http://fr.wikipedia.org/wiki/Relation_d'ordre}{ensemble totalement ordonné}\footnote{Définition~\ref{DefooYALBooHSXZqB}.} : il y a des éléments plus grands que d'autres, et mieux : à chaque fois que je prends deux éléments différents dans $\eR$, il y en a un des deux qui est plus grand que l'autre. Il n'y a pas d'\emph{ex æquo} dans $\eR$.

  Si je regarde l'intervalle $I=[0,1]$, je peux même dire que $10$ est plus grand que tous les éléments de $I$. Nous disons que $10$ est un \emph{majorant} de $[0,1]$. La définition est la suivante.
\begin{definition}
Lorsque $A$ est un sous-ensemble de $\eR$, on dit que $s$ est un \defe{majorant}{majorant} de $A$ si $s$ est plus grand que tous les éléments de $A$. En d'autres termes, si
\[
  \forall x\in A,\,s\geq x.
\]
\end{definition}
Je me permet d'insister sur le fait que l'inégalité n'est pas stricte. Ainsi, $1$ est un majorant de $[0,1]$. Dès qu'un ensemble a un majorant, il en a plein. Si $s$ majore l'ensemble $A$, alors évident $s+1$, $s+4$, $s+\pi^2$ majorent également $A$.

\begin{example}
Une petite galerie d'exemples de majorants.
\begin{itemize}
\item L'intervalle fermé $[4,8]$ admet entre autres $8$ et $130$ comme majorants,
\item l'intervalle ouvert $]4,8[$ admet également $8$ et $130$ comme majorants,
\item $7$ n'est pas un majorant de $[1,5]\cup]8,32]$,
\item $10/10$ majore les notes qu'on peut obtenir à un devoir.
\item l'intervalle $[4,\infty[$ n'a pas de majorants.
\end{itemize}
\end{example}

\begin{definition}
Le \defe{supremum}{supremum} d'un ensemble est le plus petit majorant. En d'autres terme, $s$ est un supremum de $A$ si tout nombre plus petit que $s$ ne majore pas $A$, ou encore,
\[
  \forall x<s,\exists y\in A\text{ tel que } y>x.
\]
Nous disons que $M$ est un \defe{maximum}{maximum} de $A$ si $M$ est un supremum \emph{et} $M\in A$.
\end{definition}
Quand $s$ est un supremum de $A$, ça veut dire que le moindre pas vers la gauche que l'on fait à partir de $s$ (c'est à dire le moindre $\epsilon$), et on tombe dans $A$, ou tout au moins, il existe des éléments de $A$ qui sont plus grand que $s-\epsilon$.

\subsubsection{\ldots et quelques exemples}
%//////////////////////

En matières de notations, le maximum de l'ensemble $A$ est noté $\max A$, le supremum est noté $\sup A$. Le minimum et l'infimum sont notés $\min A$ et $\inf A$.

\begin{example}
Exemples de différence entre majorant, supremum et maximum.
\begin{itemize}
\item Le nombre $10$ est un supremum, majorant et maximum de l'intervalle fermé $[0,10]$,
\item Le nombre $10$ est un majorant et un supremum, mais pas un maximum de l'intervalle ouvert $]0,10[$,
\item Le nombre $136$ est un majorant, mais ni un maximum ni un supremum de l'intervalle $[0,10]$.
\end{itemize}
\end{example}

En utilisant les notations concises, ces différents cas s'écrivent ainsi :
\begin{align*}
10&=\max[0,10]=\sup[0,10]	& 10&=\sup[0,10[
\end{align*}


\begin{example}
Si on dit que un pont s'effondre à partir d'une charge de $10$ tonnes, alors $10$ tonnes est un \emph{supremum} des charges que le pont peut supporter : si on met $9,999999$ tonnes dessus, il tient encore le coup, mais si on ajoute un gramme, alors il s'effondre (on sort de l'ensemble des charges acceptables).
\end{example}

\begin{example}
Si on dit qu'un pont résiste jusqu'à $10$ tonnes, alors $10$ tonnes est un \emph{maximum} de la charge acceptable. Sur ce pont-ci, on peut ajouter le dernier gramme. Mais à partir de là, le moindre truc qu'on ajoute, il s'effondre.
\end{example}

Lorsque vous lisez que la charge maximale d'un camion est de \(2.5 \) tonnes, est-ce que cela veut dire que vous pouvez y mettre \( 2.5\) tonnes, mais qui si un oiseau se pose dessus, le camion s'effondre ? Ou bien est-ce que cela signifie qu'à \( 2.5\) tonnes le camion s'écroule, mais que toute charge inférieure est valable ?

C'est à cette rude question que nous allons nous attaquer maintenant.

\begin{definition}
Soit une partie $A$ de $\eR$. Nous disons qu'un nombre $M$ est un \defe{majorant}{majorant} de $A$ si $M$ est plus grand ou égal que tous les éléments de $A$, c'est à dire si
\begin{equation}
	\forall a\in A,\, M\geq a.
\end{equation}
Un \defe{minorant}{minorant} de $A$ est un nombre $m$ tel que
\begin{equation}
	\forall a\in A,\, m\leq a.
\end{equation}
\end{definition}


\begin{propositionDef}[\wikipedia{en}{http://en.wikipedia.org/wiki/Least_upper_bound_principle}{Least uppert bound principle}.]		\label{DefSupeA}
Soit $A$ une partie majorée de $\eR$. Il existe un unique élément \( M\in \eR\) tel que
\begin{enumerate}
	\item
		$M\geq x$ pour tout $x\in A$,
	\item
		pour tout $\varepsilon$, le nombre $M-\varepsilon$ n'est pas un majorant de $a$, c'est à dire qu'il existe un élément $x\in A$ tel que $x>M-\varepsilon$.
\end{enumerate}

Cet élément est nommé \defe{supremum}{supremum} de $A$ et est noté \( \sup(A)\). De la même façon, \defe{l'infimum}{infimum} de $A$, noté $\inf A$, est le plus grand de ses minorants.
\end{propositionDef}
Par convention, si la partie n'est pas bornée vers le haut, nous dirons que son supremum n'existe pas, ou bien qu'il est égal à $+\infty$, suivant les contextes. Pour votre culture générale, sachez toutefois que $\infty\notin\eR$.

\begin{proof}
    Nous faisons la preuve pour l'infimum.

    \begin{subproof}
    \item[Unicité]

    En ce qui concerne l'unicité, soient \( m_1\) et \( m_2\), deux infimums de \( A\). Supposons \( m_1>m_2\). Alors il existe \( \epsilon>0\) tel que \( m_2<m_2+\epsilon<m_1\) (c'est le lemme~\ref{LemooHLHTooTyCZYL}). Cela prouve que \( m_2+\epsilon\) est un minorant de \( A\) et donc que \( m_2\) n'est pas un infimum.

\item[Existence]

	Soit $A$, une partie de $\eR$. Nous allons trouver son infimum en suivant une méthode de dichotomie. Pour cela nous allons construire trois suites en même temps de la façon suivante. D'abord nous choisissons un point $x_0$ de $A$ et un point $x_1$ qui minore $A$ (qui existe par hypothèse) :
	\begin{equation}
		\begin{aligned}[]
			x_0&\text{ est un élément de }A,\\
			x_1&\text{ est un minorant de }A,\\
			a_0&=x_0\\
			b_0&=x_1\\
			b_1&=x_1.
		\end{aligned}
	\end{equation}
	Ensuite, nous faisons la récurrence suivante :
	\begin{equation}
		\begin{aligned}[]
			x_{n+1}&=\frac{ a_n+b_n }{2},\\
			a_{n+1}&=\begin{cases}
                a_{n}	&	\text{si }x_{n+1} \text{ minore } A \\
				x_{n+1}	&	 \text{sinon},
			\end{cases}\\
			b_{n+1}&=\begin{cases}
                x_{n+1}	&	\text{si }x_{n+1} \text{ minore } A\\
				b_n	&	 \text{sinon}.
			\end{cases}
		\end{aligned}
	\end{equation}
    Nous allons montrer que \( (a_n)\) et \( (b_n)\) sont des suites convergentes de même limite et que cette limite est l'infimum de \( A\).

	Soit $n\in\eN$; il y a deux possibilités. Soit $a_n=a_{n-1}$ et $b_n=x_n$, soit $a_n=x_n$ et $b_n=b_{n-1}$. Supposons que nous soyons dans le premier cas (le second se traite de façon similaire). Alors nous avons
	\begin{equation}
		\begin{aligned}[]
			| a_n-b_n |&=| a_{n-1}-x_n |\\
			&=\left| a_{n-1}-\frac{ a_{n-1}+b_{n-1} }{2} \right| \\
			&=\frac{ 1 }{2}| a_{n-1}-b_{n-1} |,
		\end{aligned}
	\end{equation}
	ce qui prouve que $| a_n-b_n |\to 0$. Nous montrons maintenant que la suite \( (a_n)\) est de Cauchy. En effet nous avons
    \begin{equation}
        | a_n-a_{n-1} |=\begin{cases}
          0\\
          \left| \frac{ a_n -b_n}{ 2} \right|
      \end{cases}\leq \frac{1}{ 2n }.
    \end{equation}
    Il en est de même pour la suite \( (b_n)\). Ce sont deux suites de Cauchy (donc convergentes par la proposition~\ref{PROPooTFVOooFoSHPg}) qui convergent vers la même limite. Soit \( \ell\) cette limite.

	Le nombre $\ell$ minore $A$. En effet si $a\in A$ est plus petit que $\ell$, les éléments $b_n$ tels que $| b_n-\ell |<| a-\ell |$ ne peuvent pas minorer $A$. D'autre part, pour tout $\epsilon$, le nombre $\ell+\epsilon$ ne peut pas minorer $A$. En effet, $\ell$ est la limite de la suite décroissante $(a_n)$, donc il existe $a_n$ entre $\ell$ et $\ell+\epsilon$. Mais $a_n$ ne minore pas $A$, donc $\ell+\epsilon$ ne minore pas non plus $A$.

	Nous avons prouvé que toute partie minorée de $\eR$ possède un infimum.
    \end{subproof}

    La preuve que toute partie majorée possède un supremum se fait de la même façon.

\end{proof}


\begin{definition}
	Si le supremum d'un ensemble appartient à l'ensemble, nous l'appelons \defe{maximum}{maximum}. De la même façon si l'infimum d'un ensemble appartient à l'ensemble, nous disons que c'est le \defe{minimum}{minimum}.
\end{definition}

\begin{example}
	Pour les intervalles, ces notions sont simples : les bornes de l'intervalle sont les supremum et infimum, et ce sont des minima et maxima si l'intervalle est fermé.
	\begin{enumerate}
		\item
			$A=\mathopen[ 1 , 2 \mathclose]$. Tous les nombres plus petits ou égaux à $1$ sont minorants, $1$ est infimum et minimum. Le nombre $2$ est un majorant, le maximum et le supremum.
		\item
			$B=\mathopen] 3 , \pi \mathclose[$. Le nombre $\pi$ est le supremum et est un majorant, mais n'est pas le maximum (parce que $\pi\notin B$). L'ensemble $B$ n'a pas de maximum. Bien entendu, $-1000$ est un minorant.
	\end{enumerate}
    Dans les deux cas, le nombre $53$ est un majorant.
\end{example}

Il existe évidemment de nombreux exemples plus vicieux.

\begin{example}
	Prenons $E=\{ \frac{1}{ n }\tq n\in\eN_0 \}$, dont les premiers points sont indiqués sur la figure~\ref{LabelFigSuiteUnSurn}. Cet ensemble est constitué des nombres $1$, $\frac{ 1 }{2}$, $\frac{1}{ 3 }$, \ldots Le plus grand d'entre eux est $1$ parce que tous les nombres de la forme $\frac{1}{ n }$ avec $n\geq 1$ sont plus petits ou égaux à $1$. Le nombre $1$ est donc maximum de $E$.

	L'ensemble $E$ n'a par contre pas de minimum parce que tout élément de $E$ s'écrit $\frac{1}{ n }$ pour un certain $n$ et est plus grand que $\frac{1}{ n+1 }$ qui est également dans $E$.

	Prouvons que zéro est l'infimum de $E$. D'abord, tous les éléments de $E$ sont strictement positifs, donc zéro est certainement un minorant de $E$. Ensuite, nous savons que pour tout $\varepsilon>0$, il existe un $n$ tel que $\frac{1}{ n }$ est plus petit que $\varepsilon$. L'ensemble $E$ possède donc un élément plus petit que $0+\varepsilon$, et zéro est bien l'infimum.
\end{example}

\newcommand{\CaptionFigSuiteUnSurn}{Les premiers points du type $x_n=1/n$.}
\input{auto/pictures_tex/Fig_SuiteUnSurn.pstricks}

L'exemple suivant est une source classique d'erreurs en ce qui concerne l'infimum. Il sera à relire après avoir vu la définition de limite (définition~\ref{PropLimiteSuiteNum}).

\begin{example}
	Les premiers points de l'ensemble $F=\{ \frac{ (-1)^n }{ n }\tq n\in\eN_0 \}$ sont représentés à la figure~\ref{LabelFigSuiteInverseAlterne}. Bien que (comme nous le verrons plus tard) la limite de la suite $x_n=(-1)^n/n$ soit zéro, il n'est pas correct de dire que zéro est l'infimum de l'ensemble $F$. Le dessin, au contraire, montre bien que $-1$ est le minium (aucun point est plus bas que $-1$), tandis que le maximum est $1/2$.

	Nous reviendrons avec cet exemple dans la suite. Pour l'instant, ayez bien en tête que zéro n'est rien de spécial pour l'ensemble $F$ en ce qui concerne les notions de maximum, minimum et compagnie.
\end{example}
\newcommand{\CaptionFigSuiteInverseAlterne}{Les quelques premiers points du type $(-1)^n/n$.}
\input{auto/pictures_tex/Fig_SuiteInverseAlterne.pstricks}

%---------------------------------------------------------------------------------------------------------------------------
\subsection{Ouverts, voisinage, topologie}
%---------------------------------------------------------------------------------------------------------------------------

Lorsque $x\in E$, nous disons qu'un \defe{voisinage}{voisinage} de $x$ est n'importe quel sous-ensemble de $E$ qui contient une boule ouverte centrée en $x$. Nous disons qu'un ensemble est \defe{ouvert}{ouvert} s'il contient un voisinage de chacun de ses points. Par convention, nous disons que l'ensemble vide est ouvert.

\begin{definition}
L'ensemble des boules ouvertes d'un espace métrique forment la \defe{topologie}{topologie!métrique} de l'espace.
\end{definition}

Nous allons dire qu'une partie $A$ d'un espace métrique est \defe{bornée}{bornée} s'il existe une boule\footnote{À titre d'exercice, je te laisse te convaincre que l'on peut dire boule \emph{ouverte} ou \emph{fermée} au choix sans changer la définition.} qui contient $A$.

\begin{lemma}  \label{LemSupOuvPas}
Le supremum d'un ensemble ouvert n'est pas dans l'ensemble (et n'est donc pas un maximum).
\end{lemma}

\begin{proof}
Soit $\mO$, un ensemble ouvert et $s$, son supremum. Si $s$ était dans $\mO$, on aurait un voisinage $B=B(s,r)$ de $s$ contenu dans $\mO$. Le point $s+r/2$ est alors à la fois dans $\mO$ et plus grand que $s$, ce qui contredit le fait que $s$ soit un supremum de $\mO$.
\end{proof}

Par le même genre de raisonnements, on montre que l'union et l'intersection de deux ouverts sont encore des ouverts.

\begin{remark}
L'intersection d'une \emph{infinité} d'ouverts n'est pas spécialement un ouvert comme le montre l'exemple suivant :
\[
  \mO_i=]1,2+\frac{ 1 }{ i }[.
\]
Tous les ensembles $\mO_i$ contiennent le point $2$ qui est donc dans l'intersection. Mais quel que soit le $\epsilon>0$ que l'on choisisse, le point $2+\epsilon$ n'est pas dans $\mO_{(1/\epsilon)+1}$. Donc aucun point au-delà de $2$ n'est dans l'intersection, ce qui prouve que $2$ ne possède pas de voisinages contenus dans $\bigcap_{i=1}^{\infty}\mO_i$.
\end{remark}

\begin{proposition}
Prouver que, quels que soient les ensembles $A$ et $B$ dans $\eR$, nous avons
\[
  \sup(A\cap B)\leq\sup A\leq\sup(A\cup B).
\]
\end{proposition}

%---------------------------------------------------------------------------------------------------------------------------
\subsection{Intervalles}
%---------------------------------------------------------------------------------------------------------------------------

%///////////////////////////////////////////////////////////////////////////////////////////////////////////////////////////
\subsubsection{Connexité}
%///////////////////////////////////////////////////////////////////////////////////////////////////////////////////////////

Nous allons déterminer tous les sous-ensembles connexes\footnote{Définition~\ref{DefIRKNooJJlmiD}.} de $\eR$. Pour cela nous relisons d'abord la notion d'intervalle donnée en~\ref{DefEYAooMYYTz}. La partie \( I\subset \eR\) est un intervalle si pour tout \( a,b\in I\), tout nombre entre \( a\) et \( b\) est également dans \( I\). Cette définition englobe tous les exemples connus d'intervalles ouverts, fermés avec ou sans infini : $[a,b]$, $[a,b[$, $]-\infty,a]$, \ldots L'ensemble \( \eR\) lui-même est un intervalle.

Si \( I\) est un intervalle, les nombres \( \inf(I)\) et \( \sup(I)\)\footnote{Qui existent par la proposition~\ref{DefSupeA}, quitte à poser \( \pm\infty\) comme infimum et supremum lorsque \( I\) n'est pas borné.} sont les \defe{extrémités}{extrémité!d'un intervalle} de \( I\).

\begin{definition}      \label{DefLISOooDHLQrl}
	Étant donnés deux points $a$ et $b$ dans $\eR^p$ on appelle \defe{segment}{segment!dans $\eR^p$} d'extrémités $a$ et $b$, et on note $[a,b]$, l'image de $[0,1]$ par l'application $s: [0,1]\to \eR^p$, $s(t)= (1-t)a+tb$.  On pose $]a,b[=s\left(]0,1[\right)$, et  $]a,b]=s\left(]0,1]\right)$.
\end{definition}
Il faut observer que le segment $[a,b]$ est une courbe orientée : certes en tant que ensembles, $[a,b]=[b,a]$, mais si nous regardons la fonction de $t$ correspondante à $[b,a]$, nous voyons qu'elle va dans le sens inverse de celle qui correspond à $[a,b]$. Nous approfondirons ces questions lorsque nous parlerons d'arcs paramétrés autour de la section~\ref{SecArcGeometrique}.

Le segment $[b,a]$ est l'image de l'application $r\colon [0,1]\to \eR^p$ donnée par $r(t)=(1-t)b+ta$.

Une des nombreuses propositions qui vont servir à prouver le théorème des \href{http://fr.wikipedia.org/wiki/Théorème_des_valeurs_intermédiaires}{valeurs intermédiaires}~\ref{ThoValInter} est la suivante.
\begin{proposition} \label{PropInterssiConn}
    Une partie de $\eR$ est connexe si et seulement si c'est un intervalle.
\end{proposition}
\index{connexité!et intervalles}

\begin{proof}
    La preuve est en deux parties. D'abord nous démontrons que si un sous-ensemble de $\eR$ est connexe, alors c'est un intervalle; et ensuite nous démontrons que tout intervalle est connexe.

    Afin de prouver qu'un ensemble connexe est toujours un intervalle, nous allons prouver que si un ensemble n'est pas un intervalle, alors il n'est pas connexe. Prenons $A$, une partie de $\eR$ qui n'est pas un intervalle. Il existe donc $a$, $b\in A$ et un $x_0$ entre $a$ et $b$ qui n'est pas dans $A$. Comme le but est de prouver que $A$ n'est pas connexe, il faut couper $A$ en deux ouverts disjoints. L'élément $x_0$ qui n'est pas dans $A$ est le bon candidat pour effectuer cette coupure. Prenons $M$, un majorant de $A$ et $m$, un minorant de $A$, et définissons
    \begin{align*}
        \mO_1&=]m,x_0[\\
        \mO_2&=]x_0,M[.
    \end{align*}
    Si $A$ n'a pas de minorant, nous remplaçons la définition de $\mO_1$ par $]-\infty,x_0[$, et si $A$ n'a pas de majorant, nous remplaçons la définition de $\mO_2$ par $]x_0,\infty[$. Dans tous les cas, ce sont deux ensembles ouverts dont l'union recouvre tout $A$. En effet, $\mO_1\cup \mO_2$ contient tous les nombres entre un minorant de $A$ et un majorant sauf $x_0$, mais on sait que $x_0$ n'est pas dans $A$. Cela prouve que $A$ n'est pas connexe.

    Jusqu'à présent nous avons prouvé que si un ensemble n'est pas un intervalle, alors il ne peut pas être connexe. Pour remettre les choses à l'endroit, prenons un ensemble connexe, et demandons-nous s'il peut être autre chose qu'un intervalle ? La réponse est \emph{non} parce que s'il était autre chose, il ne serait pas connexe.

    Prouvons à présent que tout intervalle est connexe. Pour cela, nous refaisons le coup de \href{http://fr.wikipedia.org/wiki/Contraposée}{la contraposée}. Nous allons donc prendre une partie $A$ de $\eR$, supposer qu'elle n'est pas connexe et puis prouver qu'elle n'est alors pas un intervalle. Nous avons deux ouverts disjoints $\mO_1$ et $\mO_2$ tels que $A\subset \mO_1\cup \mO_2$. Prenons $a\in A_1$ et $b\in A_2$. Pour fixer les idées, on suppose que $a<b$. Maintenant, le jeu est de montrer qu'il existe une point $x_0$ entre $a$ et $b$ qui ne soit pas dans $A$ (cela montrerait que $A$ n'est pas un intervalle). Nous allons prouver que c'est le cas du point
    \[
      x_0=\sup\{ x\in\mO_1\tq x<b \}.
    \]
    Étant donné que l'ensemble $\mA=\{ x\in\mO_1\tq x<b \}$ est ouvert\footnote{C'est l'intersection entre l'ouvert $\mO_1$ et l'ouvert $\{x\tq x<b \}$.}, le point $x_0$ n'est pas dans l'ensemble par le lemme~\ref{LemSupOuvPas}. Nous avons donc
    \begin{itemize}
        \item soit $x_0$ n'est pas dans $\mO_1$,
        \item soit $x_0\leq b$,
        \item soit les deux en même temps.
    \end{itemize}
    Nous allons montrer qu'un tel $x_0$ ne peut pas être dans $A$. D'abord, remarquons que $\sup\mA\leq\sup\mO$ parce que $\mA$ est une intersection de $\mO$ avec quelque chose. Ensuite, il n'est pas possible que $x_0$ soit dans $\mO_2$ parce que tout élément de $\mO_2$ possède un voisinage contenu dans $\mO_2$. Un point de $\mO_2$ est donc toujours strictement plus grand que le supremum de $\mO_1$.

    Maintenant, remarque que si $x_0\leq b$, alors $x_0=b$, sinon $b$ serait un majorant de $\mA$ plus petit que $x_0$, ce qui n'est pas possible vu que $x_0$ est le supremum de $\mA$ et donc le plus petit majorant. Oui mais si $x_0=b$, c'est que $x_0\in\mO_2$, ce qu'on vient de montrer être impossible. Nous voilà déjà débarrassé des deuxièmes et troisièmes possibilités.

    Si la première possibilité est vraie, alors $x_0$ n'est pas dans $A$ parce qu'on a aussi prouvé que $x_0\notin\mO_2$. Or n'être ni dans $\mO_1$ ni dans $\mO_2$ implique de ne pas être dans $A$. Ce point $x_0=\sup\mA$ est donc hors de $A$.

    Oui, mais comme $a\in\mA$, on a obligatoirement que $x_0\geq a$. Mais par construction, on a aussi que $x_0\leq b$ (ici, l'inégalité est même stricte, mais ce n'est pas important). Donc
    \[
      a\leq x_0\leq b
    \]
    avec $a$, $b\in A$, et $x_0\notin A$. Cela finit de prouver que $A$ n'est pas un intervalle.
\end{proof}

\begin{example}
    L'application
    \begin{equation}
        \begin{aligned}
            f\colon \mathopen] 0 , 2\pi \mathclose[&\to S^1 \\
                x&\mapsto \begin{pmatrix}
                    \cos(x)    \\
                    \sin(x)
                \end{pmatrix}
        \end{aligned}
    \end{equation}
    est un homéomorphisme. Vu que \( \mathopen] 0 , 2\pi \mathclose[\) est connexe (proposition~\ref{PropInterssiConn}) la proposition~\ref{PropGWMVzqb} implique que le cercle privé d'un point est connexe.
\end{example}

%///////////////////////////////////////////////////////////////////////////////////////////////////////////////////////////
\subsubsection{Compacité}
%///////////////////////////////////////////////////////////////////////////////////////////////////////////////////////////

\begin{lemma}[\cite{JUwQXOF}]\label{LemOACGWxV}
    Si \( a<b\in \eR\) alors le segment \( \mathopen[ a , b \mathclose]\) est compact\footnote{Définition~\ref{DefJJVsEqs}}.
\end{lemma}
\index{compact!intervalle \( \mathopen[ a , b \mathclose]\)}

\begin{proof}
    Soit \( \{ \mO_i \}_{i\in I}\) un recouvrement de \( \mathopen[ a , b \mathclose]\) par des ouverts. Nous posons
    \begin{equation}
        M=\{ x\in\mathopen[ a , b \mathclose]\tq \mathopen[ a , x \mathclose] \text{ admet un sous-recouvrement fini extrait de } \{ \mO_i \}_{i\in I} \}.
    \end{equation}
    Notre but est de prouver que \( b\in M\).
    \begin{subproof}

    \item[\( a\) est dans \( M\)]

        Le point \( a\) est naturellement dans un des \( \mO_i\). L'intervalle \( \mathopen[ a , a \mathclose]\) est donc recouvert par un seul des \( \mO_i\).

    \item[\( M\) est un intervalle]

        Soient \( m\in M\) et \( m'\in\mathopen[ a , m [\). Le sous-recouvrement fini qui recouvre \( \mathopen[ a , m \mathclose]\) recouvre a fortiori \( \mathopen[ a , m' \mathclose]\).

    \item[Les trois possibilités restantes]
        À ce niveau de la preuve, il reste trois possibilités pour \( M\) soit il est de la forme \( \mathopen[ a , c \mathclose]\) ou \( \mathopen[ a , c [\) avec \( c<b\), soit il est de la forme \( \mathopen[ a , b \mathclose]\). Nous allons maintenant éliminer les deux premiers cas.

    \item[Ce que \( M\) n'est pas]

        D'abord \( M\) n'est pas de la forme \( \mathopen[ a , c [\) avec \( c<b\). En effet nous commençons par considérer \( \mO_{i_0}\) un ouvert du recouvrement contenant \( c\) et \( m<c\) dans \( \mO_{i_0}\). Si nous joignons \( \mO_{i_0}\) à un recouvrement fini de \( \mathopen[ a , m \mathclose]\) alors nous avons un recouvrement fini de \( \mathopen[ a , c \mathclose]\), et donc \( c\in M\).

        Ensuite \( M\) n'est pas de la forme \( \mathopen[ a , c \mathclose]\) avec \( c<b\). En effet si on a un recouvrement fini de \( \mathopen[ a , c \mathclose]\) par des ouverts, alors un de ces ouverts contient \( c\) et donc contient des éléments de \( \mathopen[ a , b \mathclose]\) plus grands que \( c\).
    \end{subproof}
    Nous déduisons que \( M=\mathopen[ a , b \mathclose]\) et qu'il est possible d'extraire un sous-recouvrement fini recouvrant \( \mathopen[ a , b \mathclose]\).
\end{proof}

\begin{lemma}[\cite{MonCerveau}]\label{LemCKBooXkwkte}
    Si \( K_1\) et \( K_2\) sont des compacts dans \( \eR\) alors \( K_1\times K_2\) est compact dans \( \eR^2\).
\end{lemma}

\begin{proof}
    Soit \( \{ \mO_i \}_{i\in I}\) un recouvrement de \( K_1\times K_2\) par des ouverts; grâce au lemme~\ref{LemOWVooZKndbI} nous pouvons supposer que ce sont des carrés. Pour chaque \( x\in K_1\), l'ensemble \( \{ x \}\times K_2\) est compact et donc recouvert par un nombre fini des \( \mO_i\). Soit \( R_x\) un ensemble fini des \( \mO_i\) recouvrant \( \{ x \}\times K_2\).

    Vu que \( R_x\) est une collection finie de carrés nous pouvons considérer \( m_x\), le minimum des rayons. L'ensemble \( K_1\) est recouvert par les boules \( B(x,m_x)\) et il existe donc une collection finie de \( \{ x_i \}_{i\in A}\) tels que \( B(x_i,m_{x_i})\) recouvre \( K_1\).

    Alors \( \{ R_{x_i} \}_{i\in A}\) recouvre \( K_1\times K_2\) parce que \( R_{x_i}\) recouvre l'ensemble \( B(x_i,m_{x_i})\times \{ K_2 \}\).
\end{proof}

\begin{theorem}[Théorème de Borel-Lebesgue] \label{ThoXTEooxFmdI}
    Une partie d'un espace vectoriel normé réel de dimension finie est compacte si et seulement si elle est fermée et bornée.
\end{theorem}
\index{théorème!Borel-Lebesgue}
\index{compact!fermé et borné}

\begin{proof}
    Sens direct.
    \begin{subproof}
    \item[Compact implique borné]
        En effet si \( K\) est non borné dans \( E\) alors \( K\) contient une suite \( (x_n)\) avec \( \| x_n \|>n\). Les boules \( B_i(x_i,\frac{ 1 }{3})\) sont disjointes. On pose \( \mO_0=\complement\bigcup_i\overline{ B(x_i,\frac{1}{ 5 }) }\), qui est ouvert comme complément d'un fermé. Pour \( i\geq 1\) nous posons \( \mO_i=B(x_i,\frac{1}{ 4 })\). Nous avons
        \begin{equation}
            K\subset\bigcup_{i\in \eN}\mO_i
        \end{equation}
        mais vu que \( x_i\) est uniquement dans \( \mO_i\), nous ne pouvons pas extraire de sous-recouvrement fini.
    \item[Compact implique fermé]
        Cela est la proposition~\ref{PropUCUknHx}.
    \end{subproof}
    Sens réciproque.
    \begin{subproof}
    \item[Un intervalle fermé et borné est compact dans \( \eR\)]
        C'est le lemme~\ref{LemOACGWxV}.
    \item[Un produit de segments est compact]
        Le produit de deux compacts de \( \eR\) est un compact dans \( \eR^2\) par le lemme~\ref{LemCKBooXkwkte}.
    \item[Un fermé et borné est compact]
        Soit \( K\) fermé et borné. Vu que \( K\) est borné, il est contenu dans un produit de segments. L'ensemble \( K\) est donc compact parce que fermé dans un compact, lemme~\ref{LemnAeACf}.
    \end{subproof}
\end{proof}

\begin{example}[Un compact non fermé]
    En général, un compact n'est pas toujours fermé. Si nous prenons par exemple un ensemble \( X\) de plus de deux points muni de la topologie grossière \( \{ \emptyset,X \}\). Toutes les parties de cet espace sont compactes, mais les seuls fermés sont \( \{ \emptyset,X \}\). Toutes les autres parties sont alors compactes et non fermées.
\end{example}

\begin{example}[Compacité de la boule unité]
    La boule unité fermée \( \overline{ B(0,1) }\) d'un espace vectoriel normé de dimension finie est compacte parce que fermée et bornée. En dimension infinie, cela n'est plus le cas. Certes la boule unité est encore fermée et bornée, mais elle n'est plus compacte. En effet nous allons donner un recouvrement par des ouverts duquel il ne sera pas possible d'extraire un sous-recouvrement fini.

    Autour de chacune des extrémités des vecteurs de base, nous considérons la boule \( A_i=B(e_i,\frac{1}{ 3 })\). Ensuite aussi l'ouvert
    \begin{equation}
        B(0,1)\setminus\bigcup_i\overline{ B(e_i,\frac{1}{ 4 })}.
    \end{equation}
    Le tout recouvre \( B(0,1)\) mais toutes les premières boules sont nécessaires.
\end{example}
\index{compact!boule unité}

Le théorème de Bolzano-Weierstrass nous permettra de prouver plus simplement la non compacité en dimension infinie. Voir l'exemple~\ref{ExEFYooTILPDk}.

%---------------------------------------------------------------------------------------------------------------------------
\subsection{Image d'un compact}
%---------------------------------------------------------------------------------------------------------------------------
Un résultat important dans la théorie des fonctions sur les espaces vectoriels normés est qu'une fonction continue sur un compact est bornée et atteint ses bornes. Ce résultat sera (dans d'autres cours) énormément utilisé pour trouver des maxima et minima de fonctions. Le théorème exact est le suivant.

\begin{theorem}\label{ThoMKKooAbHaro}
    Une fonction à valeurs réelles continue sur un compact est bornée et atteint ses bornes.


	C'est à dire qu'il existe $x_0\in K$ tel que $f(x_0)=\inf\{ f(x)\tq x\in K \}$ ainsi que $x_1$ tel que $f(x_1)=\sup\{ f(x)\tq x\in K \}$.
\end{theorem}
\index{compact!et fonction continue}

\begin{proof}
    Soient un espace topologique compact \( K\) et une fonction continue \( f\colon K\to \eR\). Alors le théorème~\ref{ThoImCompCotComp} indique que \( f(K)\) est compact. Par conséquent \( f(K)\) est un fermé borné de \( \eR\) par le théorème de Borel-Lebesgue~\ref{ThoXTEooxFmdI}. Vu que \( f(K)\) est borné, la fonction \( f\) est bornée.

    De plus \( f(K)\) étant fermé, son infimum est un minimum et son supremum est un maximum : il existe \( x\in K\) tel que \( f(x)=\sup f(K)\) et il existe \( y\in K\) tel que \( f(y)=\inf f(K)\).
\end{proof}

\begin{theorem}[Théorème de Bolzano-Weierstrass]		\label{ThoBolzanoWeierstrassRn}
	Toute suite contenue dans un compact de \( \eR^m\) admet une sous-suite convergente.
\end{theorem}
\index{compact!Bolzano-Weierstrass dans \( \eR^n\)}
\index{théorème!Bolzano-Weierstrass dans \( \eR^n\)}
Ce théorème est un cas particulier du théorème~\ref{ThoBWFTXAZNH} qui donne pour tout espace métrique, l'équivalence entre la compacité et la compacité séquentielle.

\begin{proof}
    Nous rappelons qu'une partie compacte de \( \eR^n\) est fermée et bornée par le théorème de Borel-Lebesgue~\ref{ThoXTEooxFmdI}.

    Soit $(x_n)$ une suite contenue dans une partie bornée de $\eR^m$. Considérons $(a_n)$, la suite réelle des premières composantes des éléments de $(x_n)$ : pour chaque $n\in\eN$, le nombre $a_n$ est la première composante de $x_n$. Étant donné que la suite $(x_n)$ est bornée, il existe un $M$ tel que $\| x_n \|<M$. La croissance de la fonction racine carrée donne
	\begin{equation}
        | a_n |\leq\| x_n \|\leq M.
	\end{equation}
	La suite $(a_n)$ est donc une suite réelle bornée et donc contient une sous-suite convergente. Soit $a_{I_1}$ une sous-suite convergente de $(a_n)$. Nous considérons maintenant $x_{I_1}$, c'est à dire la suite de départ dont on a enlevé tous les éléments qu'il faut pour qu'elle converge en ce qui concerne la première composante.

	Si nous considérons la suite $b_{I_1}$ des \emph{secondes} composantes de $x_{I_1}$, nous en extrayons, de la même façon que précédemment, une sous-suite convergente, c'est à dire que nous avons un $I_2\subset I_1$ tel que $b_{I_2}$ est convergent. Notons que $a_{I_2}$ est une sous-suite de la (sous) suite convergente $x_{I_1}$, et donc $a_{I_2}$ est encore convergente.

	En continuant ainsi, nous construisons une sous-sous-sous-suite $x_{I_3}$ telle que la suite des \emph{troisièmes} composantes est convergente. Lorsque nous avons effectué cette procédure $m$ fois, la suite $x_{I_m}$ est une suite dont toutes les composantes convergent, et donc est une suite convergente par la proposition~\ref{PropCvRpComposante}.

	Le tableau suivant donne un petit schéma de la façon dont nous procédons. Les $\bullet$ sont les éléments de la suite que nous gardons, et les $\times$ sont ceux que nous «jetons».
	\begin{equation}
		\begin{array}{lccccccccccc}
			x_{\eN}	&	\bullet&\bullet&\bullet&\bullet&\bullet&\bullet&\bullet&\bullet&\bullet&\bullet&\ldots\\
			x_{I_1}	&	\times&\bullet&\bullet&\times&\bullet&\times&\times&\bullet&\bullet&\bullet&\ldots\\
			x_{I_2}	&	\times&\bullet&\times&\times&\bullet&\times&\times&\bullet&\bullet&\times&\ldots\\
			\vdots\\
			x_{I_m}	&	\times&\times&\times&\times&\bullet&\times&\times&\times&\bullet&\times&\ldots
		\end{array}
	\end{equation}
	La première ligne, $x_{\eN}$, est la suite de départ.
\end{proof}

\begin{corollary}   \label{CorFHbMqGGyi}
    Si un suite est croissante et bornée alors elle est convergente.
\end{corollary}

\begin{proof}
    Nous nommons \( (x_n)\) la suite et nous prenons un majorant \( M\). Toute la suite est alors contenue dans le compact \( \mathopen[ x_0 , M \mathclose]\), ce qui donne une sous-suite \( (x_{\alpha(n)})\) convergente par le théorème de Bolzano-Weierstrass~\ref{ThoBolzanoWeierstrassRn}. Si \( \ell\) est la limite de cette sous-suite alors nous avons \( \ell\geq x_n\) pour tout \( n\).

    Pour tout \( \epsilon>0\) il existe \( K\) tel que si \( n>K\) alors \( | \ell-x_{\alpha(n)} |<\epsilon\). Vu que \( \ell\) majore la suite nous avons même
    \begin{equation}
        x_{\alpha(n)}+\epsilon>\ell.
    \end{equation}
    Vu que la suite est croissante pour tout \( m>\alpha(K)\) nous avons \( x_m+\epsilon>l\), ce qui signifie \( | x_m-\ell |<\epsilon\).
\end{proof}
Nous aurons une version pour les fonctions croissantes et bornées en la proposition~\ref{PropMTmBYeU}.

La proposition suivante dit que la notion d'ensemble non dénombrable ne prend pas réellement de force entre \( \eR\) et \( \eR^n\) : il n'y a pas moyen de caser \( \eR\) dans \( \eR^n\) de façon à ce qu'il y tienne à son aise.
\begin{proposition}
    Une partie non dénombrable de \( \eR^n\) possède un point d'accumulation.
\end{proposition}

\begin{proof}
    Soit une partie \( A\subset \eR^n\) sans point d'accumulation. Nous allons prouver que \( A\) est dénombrable.

    Soient les compacts \( K_n=\overline{ B(0,n) }\). La partie \( A\cap K_n\) est finie; sinon elle aurait une partie en bijection avec \( \eN\) (proposition~\ref{PROPooUIPAooCUEFme}) et donc une suite. Or une suite dans un compact possède un point d'accumulation par le théorème~\ref{ThoBolzanoWeierstrassRn}.

    Donc tous les \( A\cap K_n\) sont finis. Vu que \( A=\bigcup_nA\cap K_n\), l'ensemble \( A\) est une réunion dénombrable d'ensembles finis. Il est donc dénombrable.
\end{proof}

\begin{proposition}		\label{PropContinueCompactBorne}
	Soient $V$ et $W$ deux espaces vectoriels normés. Soient $K$ une partie compacte de $V$ et $f\colon K\to W$ une fonction continue. Alors l'image $f(K)$ est compacte dans $W$.
\end{proposition}
Ce résultat est démontré dans un cadre plus général par le théorème~\ref{ThoImCompCotComp}.

\begin{proof}
	Nous allons prouver que $f(K)$ est fermée et bornée.
    \begin{subproof}
		\item[$f(K)$ est fermé] Nous allons prouver que si $(y_n)$ est une suite convergente contenue dans $f(K)$, alors la limite est également contenue dans $f(K)$. Dans ce cas, nous aurons que l'adhérence de $f(K)$ est contenue dans $f(K)$ et donc que $f(K)$ est fermé. Pour chaque $n\in\eN$, le vecteur $y_n$ appartient à $f(K)$ et donc il existe un $x_n\in K$ tel que $f(x_n)=y_n$. La suite $(x_n)$ ainsi construite est une suite dans le fermé $K$ et possède donc une sous-suite convergente (proposition~\ref{ThoBolzanoWeierstrassRn}). Notons $(x'_n)$ cette sous-suite convergente, et $a$ sa limite : $\lim(x'_n)=a\in K$. Le fait que la limite soit dans $K$ provient du fait que $K$ est fermé.

			Nous pouvons considérer la suite $f(x'_n)$ dans $W$. Cela est une sous-suite de la suite $(y_n)$, et nous avons $\lim f(x'_n)=a$ parce que $f$ est continue. Par conséquent nous avons
			\begin{equation}
				f(a)=\lim f(x'_n)=\lim y_n.
			\end{equation}
			Cela prouve que la limite de $(y_n)$ est dans $f(K)$ et par conséquent que $f(K)$ est fermé.

		\item[$f(K)$ est borné]
			Si $f(K)$ n'est pas borné, nous pouvons trouver une suite $(x_n)$ dans $K$ telle que
			\begin{equation}		\label{EqfxnWgeqn}
				\| f(x_n) \|_W>n
			\end{equation}
			Mais par ailleurs, l'ensemble $K$ étant compact (et donc fermé), nous avons une sous-suite $(x'_n)$ qui converge dans $K$. Disons $\lim(x'_n)=a\in K$.

			Par la continuité de $f$ nous avons alors $f(a)=\lim f(x'_n)$, et donc
			\begin{equation}
				| f(a) |=\lim | f(x'_n) |.
			\end{equation}
			La suite $f(x'_n)$ est alors une suite bornée, ce qui n'est pas possible au vu de la condition \eqref{EqfxnWgeqn} imposée à la suite de départ $(x_n)$.
    \end{subproof}
\end{proof}

\begin{corollary}	\label{CorFnContinueCompactBorne}
	Si $f\colon K\to \eR$ est une application continue où $K$ est une partie compacte d'un espace vectoriel normé, alors \( f(K)\) est borné.
\end{corollary}

\begin{proof}
	En effet, la proposition~\ref{PropContinueCompactBorne} montre que $f(K)$ est compact et donc borné.
\end{proof}

% TODO: regarder ceci à propos des compacts.
% En particulier, si on recouvre $A$ par l'ensemble des boules
% $B(x,1)$ où $x$ parcourt $A$ (de sorte que tout point de $A$ est
% dans « sa » boule, et donc la réunion des boules recouvre bien
% $A$), on doit pouvoir en tirer un recouvrement fini, c'est-à-dire
% des boules $B(x_1,1), B(x_2,1), \ldots, B(x_k,1)$ (avec $k$ un
% naturel) dont la réunion contient $A$.

% Il me semble que c'est le coup qu'il ne faut vérifier le sous-recouvrement que pour des recouvrements composés d'ouverts issus d'une base donnée de la topologie.

%---------------------------------------------------------------------------------------------------------------------------
\subsection{Connexité par arcs}
%---------------------------------------------------------------------------------------------------------------------------

\begin{definition}
  Le sous-ensemble $A \subset \eR^n$ est \defe{connexe par arcs}{connexe!par arc} si pour tout $x, y \in A$, il existe un chemin\footnote{Attention : ici quand on dit \emph{chemin}, on demande que l'application soit continue. Dans de nombreux cours de géométrie différentielle, on demande $ C^{\infty}$. Il faut s'adapter au contexte.} contenu dans $A$ les reliant, c'est-à-dire une application continue
  \begin{equation*}
    \gamma : [0,1] \to \eR^n \tq \gamma(0) = x~\text{et}~\gamma(1) = y
  \end{equation*}
  avec $\gamma(t) \in A$ pour tout $t\in [0,1]$.
\end{definition}

La connexité d'un ensemble n'implique pas sa connexité par arc. Il suffit pour cela de prendre un ensemble constitué de deux connexes reliés par un chemin de longueur infinie (le graphe d'une fonction de type \( \sin(1/x)\) par exemple).

%TODO : placer ici l'exemple de ce graphe à qui on ajoute le segment vertical de (0,0) à (0,1).

%---------------------------------------------------------------------------------------------------------------------------
\subsection{Topologie de la droite réelle complétée}
%---------------------------------------------------------------------------------------------------------------------------
\label{SUBSECooKRRUooSlZSmM}

Nous introduisons l'ensemble \( \bar\eR=\eR\cup\{ \pm\infty \}\). À présent les symboles \( +\infty\) et \( -\infty\) n'ont aucune signification particulière; il s'agit seulement de deux éléments que nous ajoutons à \( \eR\) pour former un ensemble que nous notons \( \bar \eR\).

Pas plus tard qu'immédiatement nous leur donnons une signification en définissant une topologie sur \( \bar\eR\). Les ouverts sur \( \bar \eR\) sont
\begin{enumerate}
    \item
        tous les ouverts de \( \eR\),
    \item
        les intervalles de la forme \( \mathopen] -\infty , a \mathclose[\) pour tous les \( a\in \eR\),
    \item
        les intervalles de la forme \( \mathopen] a , +\infty \mathclose[\) pour tous les \( a\in \eR\),
    \item
        la topologie engendrée par toutes ces parties de \( \bar \eR\).
\end{enumerate}

Par construction, les boules de \( \eR\) et les intervalles \( \mathopen] -\infty , a \mathclose[\) et \( \mathopen] a , +\infty \mathclose[\) forment une base de topologie pour \( \bar \eR\).

Si \( f\) est une fonction \( f\colon \eR\to \eR\), que signifie \( \lim_{x\to \infty} f(x)\) ? Il s'agit de considérer la fonction élargie
\begin{equation}
    \begin{aligned}
        \tilde f\colon \bar \eR&\to \bar\eR \\
        x&\mapsto \begin{cases}
            f(x)    &   \text{si } x\in \eR\\
            0    &    \text{si } x=\pm\infty.
        \end{cases}
    \end{aligned}
\end{equation}
Ensuite, c'est la définition topologie usuelle de la limite. Notons que les limites en \( a\) ne dépendent pas de la valeur effective de \( f\) en \( a\), donc le prolongement par \( 0\) est sans conséquences. Nous pouvions tout aussi bien prolonger par \( 4\).

Le même raisonnement tient pour donner un sens à \( \lim_{x\to a} f(x)=\pm \infty\).

%---------------------------------------------------------------------------------------------------------------------------
\subsection{Limite pointée ou épointée ?}
%---------------------------------------------------------------------------------------------------------------------------
\label{SUBSECooVHKCooYRFgrb}

Nous sommes des intégristes sur ce point. Nous ne savons pas ce qu'est la limite pointée et nous ne voulons pas le savoir; tant et si bien que nous disons «limite» là où d'autres auraient le scrupule de préciser «épointée». Le fait est que nous voulons :
\begin{enumerate}
    \item
        la définition de la continuité sur un intervalle est que l'image inverse d'un ouvert est un ouvert
    \item
        une fonction est continue en \( a\) si et seulement si sa limite en \( a\) existe et vaut \( f(a)\)
    \item
        une fonction est continue sur un intervalle si et seulement si elle est continue en chacun de ses points.
\end{enumerate}
Si par incroyable vous lisiez ces lignes dans l'idée de préparer un concours d'enseignement en France, soyez toutefois au courant de l'existence d'un truc qui s'appelle «limite pointée», et que, pire, dans certains pays de la francophonie, certains utilisent le mot «limite» sans précisions pour désigner le machin qui au mieux a le droit de se nommer «limite pointée».

Je dis rarement du mal de Wikipédia, et je prends presque toujours Wikipédia comme référence en cas de doute, mais si vous voulez mon avis sur ce point, vous pouvez lire \cite{ooEJTBooFehGbZ}.

